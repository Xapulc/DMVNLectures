\documentclass[a4paper]{article}
\usepackage[utf]{dmvn}
\usepackage{amscd}
\renewenvironment{dfn}[1]{\par\vskip\theoremskip\textbf{Определение~{#1}.}}
{\par\vskip\theoremskip}
\newenvironment{thh}{\par\vskip\theoremskip\textbf{Теорема.}\normalfont \itshape}
{\par\vskip\theoremskip}
\newenvironment{imp*}{\par\vskip\theoremskip\textbf{Следствие.}\normalfont \itshape}
{\par\vskip\theoremskip}
\newenvironment{stm*}{\par\vskip\theoremskip\textbf{Утверждение.}\normalfont \itshape}
{\par\vskip\theoremskip}
\newenvironment{lem}[1]{\par\vskip\theoremskip\textbf{Лемма~{#1}.}\normalfont \itshape}
{\par\vskip\theoremskip}
\newenvironment{stmm}[1]{\par\vskip\theoremskip\textbf{Утверждение~{#1}.}\normalfont \itshape}
{\par\vskip\theoremskip}
\newenvironment{lem*}{\par\vskip\theoremskip\textbf{Лемма.}\normalfont \itshape}
{\par\vskip\theoremskip}
\newenvironment{thn}[1]{\par\vskip\theoremskip\textbf{Теорема~{#1}.}\normalfont \itshape}
{\par\vskip\theoremskip}

\DeclareMathOperator{\Ext}{Ext}
\newcommand{\emd}[1]{\emph{#1}}
\newcommand{\rc}{\rceil}
\newcommand{\vc}{\mathcal{v}}
%\setcounter{subsubsection}{0}

\begin{document}
\dmvntitle{Курс лекций по}{математическому анализу}{Лектор
Валериан Иванович Гаврилов} {II курс, 4 семестр, поток
механиков}{Москва, 2007 г.}

\pagebreak


\dmvntrail
\medskip
\medskip
\medskip
\tableofcontents \pagebreak


\hrule\begin{center}\LARGE \bf Часть 6.\hfill Интегрирование в
$\R^m$, $m\ge1$.\end{center}\hrule
\section{Двойной и кратные интегралы}

В сравнении с предыдущими семестрами, чтение курса математического
анализа в четвёртом семестре имеет следующие особенности. Прежде
всего, значительная часть излагаемого материала носит
\emph{геометрический} характер и исчерпывающие \emph{аналитические}
доказательства часто становятся весьма громоздкими. Далее, уменьшено
число недельных лекционных часов (на одну четверть). И наконец,
параллельно читается полный годовой курс дифференциальной геометрии.

Указанные обстоятельства заставляют меня при изложении многих
результатов опускать в их доказательствах значительную часть деталей
(в основном геометрического характера) в расчёте на здравый смысл
слушателей и на ранее приобретённые ими (я надеюсь!)
профессиональные знания, которые позволяют восстановить эти детали
самостоятельно. Кроме того, я широко пользуюсь плодами курса
дифференциальной геометрии.


\subsection{Мера Жордана на плоскости и в пространстве}

\subsubsection{Терминология}

Рассмотрим произвольное множество $A$ в пространстве $\R^m$,
$m\ge1$. Символом $[A]$ обозначим множество всех точек прикосновения
множества $A$; то есть, $[A]$ --- замыкание множества $A$ и
замкнутое множество в $\R^m$, $A\subset[A]$. Символом $]A[$
обозначим совокупность всех внутренних точек множества $A$, так что
$]A[\subset A$ (возможно, $]A[=\es$). Тогда множество $[A]\backslash
]A[$ называют границей множества $A$, обозначают гр. $A$ (или
$\partial A$), а его точки называют граничными точками множества
$A$. Если обозначить $\R^m \backslash A = \rceil A$ и заметить, что
$\rceil ]A[=[\rceil A]$, то \eqa{1}{гр.\,А = [A]\backslash ]A[ = [A]
\cap \rceil ]A[ = [A] \cap [\rceil A].}

Из последней формулы непосредственно следует, что граница всякого
множества (как пересечение замкнутых множеств) замкнута, и что
\eqa{2}{гр.\, A = гр. \,\rceil А.}

\begin{stmm}
1 Граница объединения, пересечения и разности двух множеств
содержится в объединении их границ.
\end{stmm}

\begin{proof}
Заметим, что $[A\cup B]=[A]\cup[B]$. В силу (1), \ml{гр.\,(A\cup B)
= [A\cup B] \cap [\; \rceil (A\cup B)] = ([A] \cup [B]) \cap [\; \rc
A \cap \rc B ] \subset ([A] \cup [B]) \cap ([\; \rc A] \cap [ \; \rc
B]) = ([A] \cap [ \; \rc A] \cap [ \; \rc B]) \cup \\ \cup ([B] \cap
[ \; \rc B] \cap [ \; \rc A]) \subset ([A] \cap [ \; \rc A]) \cup
([B] \cap [ \; \rc B])=гр.\,А \cup гр.\, B.}

Далее, в силу доказанного и (2), \eqa{3}{гр.\,(A\cap B) = гр.\, \rc
(A\cap B) = гр.\,(\rc A \cup \rc B) \subset гр.\, \rc A \cup гр.\,
\rc B = гр.\, A \cup гр.\, B.} Наконец, согласно (2) и (3),
\equ{гр.\, (A\backslash B) = гр.\, (A\cap \rc B) \subset гр.\,А \cup
гр. \, \rc B = гр.\,А \cup гр.\, B.}
\end{proof}

\begin{stmm}
2 $[A]=A\cup гр.\,A$ для любого множества $A$ в $\R^m$.
\end{stmm}

\begin{proof}
Так как $A\subset[A]$ и $гр.\, A\subset[A]$, то $A\cup гр.\,A\subset
[A]$. С другой стороны, так как $[A]\backslash A = [A] \cap \rc A
\subset [A] \cap [ \; \rc A] = гр. \, A$, то $[A] = A\cup
([A]\backslash A) \subset A \cup гр.\, A.$
\end{proof}

\subsubsection{Квадрируемость плоской фигуры}
\textbf{Многоугольной фигурой} $P$ на плоскости называют объединение
конечного числа многоугольников, лежащих в этой плоскости. Из
школьного курса известно понятие площади многоугольника. Поэтому
можно говорить о площади $\mu(P)$ многоугольной фигуры $P$ и число
$\mu(P)\ge0$ обладает свойствами:
\begin{points}{-2}
\item \emph{Аддитивность}. Если $P_1$ и $P_2$ --- многоугольные фигуры без
общих внутренних точек, то $\mu(P_1\cup P_2)=\mu(P_1) + \mu(P_2)$.

\item \emph{Инвариантность}. Если многоугольные фигуры $P_1$ и $P_2$
конгруэнтны (то есть, существует биекция множеств $P_1$ и $P_2$,
сохраняющая расстояние между точками плоскости --- ортогональное
преобразование плоскости), то $\mu(P_1)=\mu(P_2)$.

\item \emph{Монотонность}. Если $P_1\subseteq P_2$, то $\mu(P_1)\le
\mu(P_2)$.
\end{points}

\begin{proof}
По условию, $P_2 = P_1 \cup (P_2 \backslash P_1)$ и $P_2\backslash
P_1$ --- многоугольная фигура, не имеющая общих внутренних точек с
$P_1$. Согласно $1^{\circ}$, $\mu(P_2)=\mu(P_1) + \mu(P_2\backslash
P_1) \ge \mu(P_1)$, так как $\mu(P_2\backslash P_1)\ge0$.
\end{proof}

\begin{note}
Для любой многоугольной фигуры $P$ справедливо
$\mu([P])=\mu(]P[)=\mu(P)$.
\end{note}

Произвольное ограниченное множество $F$ на плоскости называют
\textbf{плоской фигурой}. Рассмотрим всевозможные многоугольные
фигуры $P$, содержащиеся в $F$, и многоугольные фигуры $Q$,
содержащие $F$. Числовое множество $\hc{\mu(P) \; | \; P\subset F}$
ограничено сверху любым $\mu(Q)$, $Q\supset F$, а числовое множество
$\hc{\mu(Q) \; | \; Q\supset F}$ ограничено снизу (например, нулём).
Следовательно, существуют \eqa{4}{\sup \hc{\mu(P) \; | \; P\subset
F} = \mu_*(F)=\mu_* \mbox{ и } \inf\hc{\mu(Q) \; | \; Q\supset F} =
\mu^*(F) = \mu^*.} Отметим, что если не существуют $P$, которые
содержатся в $F$, то полагаем $\mu_*=0$.

Число $\mu_*$ называют \textbf{нижней площадью} фигуры $F$, а
$\mu^*$ --- \textbf{верхней площадью} фигуры $F$, и $\mu_*(F)\le
\mu^*(F)$.

\begin{dfn}
1 Плоскую фигуру $F$ называют квадрируемой (или имеющей площадь),
если $\mu^*=\mu_*$, и общее значение $\mu=\mu(F) = \mu_* = \mu^*$
называют площадью фигуры $F$. Другое обозначение: $\mu(F) = пл.\,
F$.

Если $F=P$ --- многоугольная фигура, то $\mu(F) = \mu(P) = \mu_*(P)
= \mu^*(P)$ совпадает с площадью многоугольной фигуры $P$.
\end{dfn}

\begin{theorem}
Плоская фигура $F$ квадрируема $\Lra$ для любого числа $\ep>0$ можно
указать такие многоугольные фигуры $P$ и $Q$, что $P\subset F\subset
Q$ и $\mu(Q) - \mu(P) < \ep$.
\end{theorem}

\begin{proof}
\textbf{Необходимость}. Пусть фигура $F$ квадрируема, то есть
$\mu_*=\mu^*$. По характеристическому свойству точных верхних и
нижних граней числовых множеств, для произвольного $\ep>0$ найдутся
многоугольные фигуры $P$ и $Q$, что $P\subset F$ и $Q\supset F$ и
\equ{\mu_* - \frac{\ep}2 < \mu(P) \le \mu_* \mbox{ и } \mu^* \le
\mu(Q) < \mu^*+\frac{\ep}2.} Отсюда, с учётом $\mu_*=\mu^*$
заключаем, что $\mu(Q) - \mu(P) < \ep$.

\textbf{Достаточность}. Для любого $\ep>0$ существуют многоугольные
фигуры $P\subset F$ и $Q\supset F$, что $\mu(Q) - \mu(P) < \ep$. Так
как, на основании (4), $\mu(P) \le \mu_*\le \mu^* \le \mu(Q)$, то
$0\le \mu^* - \mu_* \le \mu(Q) - \mu(P) < \ep$ и $\mu^* = \mu_*$ в
силу произвольности $\ep>0$.
\end{proof}

\begin{theorem}
Для квадрируемости плоской фигуры $F$ необходимо и достаточно, чтобы
для любого числа $\ep>0$ существовали такая содержащая $F$
квадрируемая фигура $G$ и такая содержащаяся в $F$ квадрируемая
фигура $E$, что $\mu(G) - \mu(E) < \ep$.
\end{theorem}

\begin{proof}
\textbf{Необходимость} установлена в теореме 1, как $G=Q$ и $E=P$
--- квадрируемые фигуры.

\textbf{Достаточность}. Фиксируем произвольное $\ep>0$ и, по условию
теоремы, находим такие квадрируемые фигуры $E$ и $G$, что $E\subset
F\subset G$, и \eqa{5}{\mu(G) - \mu(E) < \frac{\ep}2.} Так как $E$ и
$G$ квадрируемы, то как и в доказательстве теоремы 1, находим
многоугольные фигуры $Q\supset G$ и $P\subset E$, для которых
$\mu(Q) - \mu(G) < \frac{\ep}4$ и $\mu(E)-\mu(P)<\frac{\ep}4$. На
основании последних неравенств и неравенства (5) заключаем, что
$\mu(Q)-\mu(P)<\ep$, и так как $P\subset E\subset F\subset G\subset
Q$, то, по теореме 1, $F$ --- квадрируемая фигура.
\end{proof}

\subsubsection{Фигуры нулевой площади}

Плоская фигура $F$ квадрируема и $пл.\, F = \mu(F)=0$ тогда и только
тогда, когда $\mu^*(F)=0$ (ибо тогда и $\mu_*(F)=0$, так что
$\mu_*(F)=\mu^*(F)$). Другими словами, фигура $F$ имеет нулевую
площадь тогда и только тогда, когда она содержится в многоугольной
фигуре сколь угодно малой площади.

Понятно, что всякая часть фигуры площади 0 квадрируема и имеет
площадь 0, и что объединение двух (а следовательно, и любого
конечного числа) фигур нулевой площади есть фигура нулевой площади.
Последнее утверждение является также следствием такого понятного
свойства, которое приведём без доказательства.

\begin{stmm}
3 Каковы бы ни были плоские фигуры $F_1 \etc, F_m$, всегда
\eqa{6}{\mu^*\hr{\bigcup\limits_{j=1}^m F_j} \le \suml{j=1}m
\mu^*(F_j)}
\end{stmm}

\begin{stmm}
4 Всякая спрямляемая плоская кривая имеет нулевую площадь.
\end{stmm}

\begin{proof}
Пусть $\Lc$ --- спрямляемая кривая и $\hm{\Lc}$ --- её длина.
Разобьём эту кривую с помощью $n+1$ точек на части длины
$\frac1n\hm{\Lc}$. Примем каждую из этих $n+1$ точек за центр
квадрата со стороной $\frac2n\hm{\Lc}$. Объединение всех таких
квадратов представляет собой многоугольную фигуру, описанную
(очевидно?) вокруг $\Lc$, и площадью, не превосходящей суммы
площадей составляющих её квадратов; то есть, не большей числа
$\frac4{n^2} \hm{\Lc}^2(n+1)$. Так как $\hm{\Lc}$ --- фиксировано, а
$n$ можно выбрать произвольно большим, то число $\frac4{n^2}
\hm{\Lc}^2(n+1)$ может быть сделано меньше любого заданного числа
$\ep>0$; то есть, кривую $\Lc$ можно заключить в многоугольную
фигуру сколь угодно малой площади.
\end{proof}

\begin{stmm}
5 График $\Ga_f$ функции $f$, интегрируемой по Риману на отрезке
$[a,b]$ действительной оси, имеет нулевую площадь.
\end{stmm}

\begin{proof}
Рассмотрим произвольное число $\ep>0$. Согласно критерию
интегрируемости функции на отрезке, найдётся такое разбиение $T$
отрезка $[a,b]$, у которого верхняя $S(f;T)$ и нижняя $s(f;T)$ суммы
Дарбу связаны отношениями $0\le S(f;T) - s(f;T) < \ep$.
Геометрически разность $S(f;T)-s(f;T)$ есть площадь некоторой
ступенчатой многоугольной фигуры $Q$, содержащей $\Ga_f$ (материал
второго семестра), и $пл.\, Q<\ep$. По определению, $пл.\, \Ga_f=0$.
\end{proof}

В частности, график $\Ga_f$ любой непрерывной на отрезке $[a,b]$
функции $f$ есть фигура нулевой площади.

\subsubsection{Критерий квадрируемости плоской фигуры}\label{114}

\begin{theorem}
Плоская фигура $F$ квадрируема тогда и только тогда, когда её
граница $\partial F$ имеет площадь нуль.
\end{theorem}

\begin{proof}
\textbf{Необходимость}. Пусть $F$ --- квадрируемая плоская фигура и
$\ep>0$ --- произвольное число. Находим многоугольные фигуры $P$ и
$Q$, $P\subset F\subset Q$, чтобы $\mu(Q)-\mu(P)<\ep$ (по теореме
1). Так как $\mu([Q])=\mu(Q)$ и $\mu(]P[)=\mu(P)$ и $[Q] = ]P[ \cup
([Q] \backslash ]P[)$, то $\ep> \mu([Q]) - \mu(]P[)=\mu([Q]
\backslash ]P[)$. Поскольку многоугольная фигура $[Q]\backslash ]P[$
содержит $\partial F$, то согласно определению 2 и утверждению 2,
$\mu(\partial F)=0$.

\textbf{Достаточность}. Впишем плоскую фигуру $F$ в квадрат $R$ со
сторонами, параллельными координатным осям на плоскости, и прямыми,
параллельными этим осям, разобьём $R$ на элементарные квадраты со
стороной $h$. Это разбиение квадрата $R$ условимся называть
\emph{сеткой с шагом $h$}.

\begin{lem}
1 Если граница $\partial F$ фигуры $F$ содержится в многоугольной
фигуре площади, меньшей $\ep$, то можно выбрать такой шаг $h$ сетки,
что $\partial F$ содержится в объединении элементарных квадратов
сетки, общая площадь которой меньше $32\ep$.
\end{lem}

\begin{proof}
Любая многоугольная фигура площади, меньшей $\ep$, есть объединение
конечного числа треугольников, не имеющих общих внутренних точек;
каждый треугольник равен объединению двух прямоугольных
треугольников (без общих внутренних точек); каждый прямоугольный
треугольник содержится во вдвое б\emph{о}льшем по площади
прямоугольнике; каждый прямоугольник содержится в объединении не
более, чем вдвое б\emph{о}льшей площади конечного числа квадратов;
каждый квадрат содержится во вдвое б\emph{о}льшем по площади
квадрате со сторонами, параллельными осям координат.

Итак, любая многоугольная фигура площади, меньшей $\ep$, содержится
в объединении конечного числа квадратов со сторонами, параллельными
осям координат, и общей площади, меньшей $8\ep$.

Из указанного конечного числа квадратов выберем квадрат с наименьшей
стороной и возьмём шаг $h$ сетки, равным половине длины стороны
этого квадрата. При таком выборе $h$ каждый указанный квадрат будет
содержаться в объединении элементарных квадратов сетки, общая
площадь которых не больше учетверённой площади данного квадрата.
Поэтому вся многоугольная фигура площади, меньшей $\ep$, содержится
в объединении элементарных квадратов сетки, общая площадь которых
меньше $32\ep$.
\end{proof}

Согласно этой лемме, если граница $\partial F$ плоской фигуры $F$
имеет площадь нуль, то для любого $\ep>0$ выбираем такой шаг $h$
сетки, что вся $\partial F$ будет содержаться в объединении
элементарных квадратов сетки, общая площадь которых меньше $32\ep$.

Объединение всех элементарных квадратов, состоящих только из
внутренних точек $]F[$ фигуры $F$, представляет собой многоугольную
фигуру $P$, содержащуюся в $F$, а объединение этой фигуры $P$ со
всеми элементарными квадратами, содержащими точки границы $\partial
F$ фигуры $F$, представляет собой многоугольную фигуру $Q$,
содержащую $F$, и при этом, $\mu(Q)-\mu(P) < 32\ep$. Доказательство
заканчивается применением теоремы 1.
\end{proof}

\begin{imp}
Объединение, пересечение и разность двух квадрируемых фигур ---
квадрируемые фигуры.
\end{imp}

\begin{proof}
Пусть $F$ есть $F_1\cup F_2$, $F_1\cap F_2$ или $F_1\backslash F_2$.
По утверждению 1, во всех трёх случаях $гр.\, F \subset гр. \, F_1
\cup гр. \, F_2$. Но если $F_1$ и $F_2$ квадрируемы, то, по теореме
3, $гр.\,F_1$ и $гр.\,F_2$ --- фигуры нулевой площади. Значит, также
$гр.\,F_1 \cup гр.\,F_2$, а с ним и $гр.\,F$ --- фигуры нулевой
площади (утверждение 3), так что, снова по теореме 3, $F$
квадрируема.
\end{proof}

\begin{note}
Утверждение следствия 1 распространяется по индукции на объединения
и пересечения любых конечных систем квадрируемых фигур.
\end{note}

\begin{imp}
Всякая плоская фигура, граница которой состоит из одной или
нескольких спрямляемых кривых, квадрируема.
\end{imp}

\begin{proof}
Согласно утверждениям 4 и 3, граница такой плоской фигуры имеет
нулевую площадь , и фигура квадрируема по теореме 3.
\end{proof}

\subsubsection{Свойства квадрируемых фигур}

\begin{points}{-2}

\item \emph{Свойство аддитивности}. Если $F_1$ и $F_2$ ---
квадрируемые фигуры без общих внутренних точек и $F=F_1\cup F_2$, то
$F$ квадрируема и $\mu(F)=\mu(F_1)+\mu(F_2)$.

\begin{proof}
Квадрируемость фигуры $F$ доказана в следствии 1 к теореме 3.
Рассмотрим произвольное число $\ep>0$ и такие многоугольные фигуры
$P_i$ и $Q_i$, $i=1,2$, что $P_i\subset F_i\subset Q_i$, $i=1,2$, и
$\mu(Q_i)-\mu(P_i) < \frac{\ep}2$, $i=1,2$ (что возможно по теореме
1). Тогда $P_1$ и $P_2$ не имеют общих внутренних точек и для
$P=P_1\cup P_2$ справедливо $\mu(P)=\mu(P_1)+\mu(P_2)$.
Многоугольные фигуры $Q_1$ и $Q_2$, возможно пересекающиеся,
образуют $Q=Q_1\cup Q_2$ и $\mu(Q) \le \mu(Q_1)+\mu(Q_2)$. Поскольку
$P\subset F\subset Q$, то \eqa{7}{\mu(P_1) + \mu(P_2) = \mu(P) \le
\mu(F) \le \mu(Q) \le \mu(Q_1)+\mu(Q_2).} С другой стороны, имеем
$\mu(P_i)\le \mu(F_i) \le \mu(Q_i), \; i=1,2$, и
\eqa{8}{\mu(P_1)+\mu(P_2) \le \mu(F_1) + \mu(F_2) \le \mu(Q_1) +
\mu(Q_2).}

Объединяя (7) и (8), получим \equ{\hm{\mu(F) - [\mu(F_1) +
\mu(F_2)]} \le [\mu(Q_1) + \mu(Q_2)] - [\mu(P_1) + \mu(P_2)] =
\mu(Q_1) - \mu(P_1) + \mu(Q_2) - \mu(P_2) <
\frac{\ep}2+\frac{\ep}2=\ep,} откуда, в силу произвольности числа
$\ep>0$, следует равенство $\mu(F)=\mu(F_1)+\mu(F_2)$.
\end{proof}

\item \emph{Свойство монотонности}. Если $F_1\subset F_2$ и $F_i$,
$i=1,2$ --- квадрируемые, то $\mu(F_1) \le \mu(F_2)$.

\begin{proof}
Так как $F_2 = F_1 \cup (F_2\backslash F_1)$ и, по свойству
$1^{\circ}$, $\mu(F_2) = \mu(F_1) + \mu(F_2\backslash F_1)$, то
$\mu(F_2)\ge \mu(F_1)$ в силу $\mu(F_2\backslash F_1)\ge0$.
\end{proof}
\end{points}

Введённое выше понятие площади называют понятием площади по Жордану
или (плоской) \textbf{мерой Жордана}. Мера Жордана инвариантна
относительно ортогональных преобразований плоскости. Она обладает
также свойством конечной аддитивности.

\begin{theorem}
Объединение $F$ любого конечного числа не имеющих попарно общих
внутренних точек квадрируемых фигур $F_j$, $j=\ol{1,m}$, $m\in\N$,
образует квадрируемую фигуру и $\mu(F) = \sum\limits_{j=1}^m
\mu(F_j)$.
\end{theorem}

\subsubsection{Квадрируемые компакты}
\begin{theorem}
Замыкание $[F]$ квадрируемой фигуры $F$ квадрируемо и имеет ту же
площадь, что и $F$.
\end{theorem}

\begin{proof}
Так как $F$ квадрируема, то, по теореме 3, и $гр.\, F$ квадрируема и
$\mu(гр.\, F)=0$. Но, согласно утверждению 2, $[F] = F\cup гр.\, F$.
Поэтому, по следствию 1 к теореме 3, и $[F]$ квадрируемо.
Следовательно, на основании (6), \equ{пл.\, F \le пл.\, [F] \le
пл.\, F + пл.\,гр.\, F=пл.\, F;} то есть, $пл.\, [F] = пл.\, F$.
\end{proof}

\begin{theorem}
Замыкание квадрируемой фигуры есть квадрируемый компакт.
\end{theorem}

\begin{proof}
Как замыкание оно замкнуто, а как замыкание ограниченного множества
ограничено. Но замкнутое и ограниченное множество на плоскости ---
компакт. Он квадрируем по теореме 5.
\end{proof}

\subsubsection{Кубируемые тела в $\R^3$}

Произвольное ограниченное множество $\Ph$ в $\R^3$ назовём
\textbf{телом}. \textbf{Многогранным телом} называют объединение
конечного числа ограниченных многогранников в $\R^3$. Каждый
многогранник в $\R^3$ имеет объём, который обладает (как и площадь
многоугольника) свойствами аддитивности, инвариантности и
монотонности.

Рассмотрим произвольное тело $\Ph$, а также всевозможные
многогранные тела $P$, содержащиеся в $\Ph$ $(]P[\subset \Ph)$ и
всевозможные многогранные тела $Q$, содержащие $\Ph$ ($\Ph\subset
Q\subset [Q]$). Число $\mu^*=\mu^*(\Ph)=\inf\hc{\mu(Q) \; | ;
\Ph\subset Q}$, где $\mu(Q)$ обозначает объём многогранного тела
$Q$, назовём \textbf{верхним объёмом тела} $\Ph$; число
$\mu_*=\mu_*(\Ph) = \sup\hc{\mu(P) \; | \; ]P[\subset\Ph}$ ---
\textbf{нижним объёмом тела} $\Ph$, так что $0\le\mu_*\le\mu^*$.

\begin{dfn}
2 Тело $\Ph$ называют кубируемым (или имеющим объём), если
$\mu_*=\mu^*$. При этом, число $\mu=\mu(\Ph)=\mu^*=\mu_*$ называют
объёмом тела $\Ph$. Другое обозначение: $\mu(\Ph)=об.\,\Ph$.
\end{dfn}

В полной аналогии с теоремой 1.1 доказывается следующая теорема.

\begin{theorem}
Для кубируемости тела $\Ph$ необходимо и достаточно, чтобы для
любого числа $\ep>0$ нашлись такое содержащееся в $\Ph$ многогранное
тело $P$ и такое содержащее $\Ph$ многогранное тело $Q$, для которых
$об.\, Q-об.\,P<\ep$.
\end{theorem}

\begin{dfn}
3 Множество точек пространства $\R^3$ назовём множеством объёма
нуль, если это множество содержится в многогранном теле сколь угодно
малого объёма.
\end{dfn}

Аналогично теореме 1.3 доказывается нижеследующая теорема.

\begin{theorem}
Тело $\Ph$ кубируемо тогда и только тогда, когда его граница имеет
нулевой объём.
\end{theorem}

\begin{stmm}
5 Всякая плоская фигура $F$ в пространстве кубируема и имеет нулевой
объём.
\end{stmm}

\begin{proof}
Пусть фигура $F$ лежит на плоскости $\Pi$ в $\R^3$. Тогда $F$
содержится в некотором прямоугольнике $Q$ на плоскости $\Pi$, а
значит, и в содержащем $Q$ прямоугольном параллелепипеде произвольно
малой высоты, и следовательно, произвольно малого объёма.
\end{proof}

\subsubsection{Некоторые классы кубируемых тел}

\textbf{Цилиндрическим телом} называют тело, ограниченное
цилиндрической поверхностью с образующими, параллельными некоторой
оси, и двумя плоскостями, перпендикулярными этой оси.

Пусть $F$ --- произвольная фигура на координатной плоскости $xOy$ и
$[m,M]$ --- произвольный отрезок на оси $Oz$. Тогда \equ{F\times
[m,M] = \hc{(x,y,z)\in\R^3 \; | \; (x,y)\in F, \; m\le z\le M}}
является цилиндрическим телом.

\begin{theorem}
\label{thm179}Если фигура $F$ квадрируема, то цилиндрическое тело
$\Ph=F\times[m,M]$ кубируемо и $об.\,\Ph=(M-m)пл.\,F$.
\end{theorem}

\begin{proof}
Поскольку плоская фигура $F$ квадрируема, то для любого числа
$\ep>0$ можно указать такие вписанную и описанную многоугольные
фигуры $P$ и $Q$, $P\subset F\subset Q$, что
$пл.\,Q-пл.\,P<\frac{\ep}{M-m}$. Цилиндрические тела
$\Ph_P=P\times[m,M]$ и $\Ph_Q=Q\times[m,M]$ имеют $об.\, \Ph_P =
(M-m)пл.\, P$ и $об.\, \Ph_Q = (M-m)пл.\, Q$, и при этом
$\Ph_P\subset\Ph\subset\Ph_Q$. Так как $об.\,\Ph_Q-об.\,\Ph_P =
(M-m)(пл.\,Q-пл.\,P)<\ep$, то, по теореме 6, тело $\Ph$ кубируемо.
Поскольку, далее, \equ{об.\,\Ph_P = (M-m)пл.\,P \le об.\,\Ph \le
(M-m)пл.\,Q=об.\,\Ph_Q} и $пл.\,P$, $пл.\,Q$ стремятся к $пл.\, F$,
то $об.\,\Ph = (M-m)пл.\,F$.
\end{proof}

\begin{dfn}
4 Ступенчатым телом называют объединение конечного числа
цилиндрических тел, попарно не имеющих общих внутренних точек.
\end{dfn}

На основании предыдущего заключаем, что справедлива теорема:

\begin{theorem}
Если для любого числа $\ep>0$ можно указать такое содержащее тело
$\Ph$ ступенчатое тело $\Ph_2$, и такое содержащееся в $\Ph$
ступенчатое тело $\Ph_1$, что $об.\, \Ph_2 - об.\, \Ph_1<\ep$, то
тело $\Ph$ кубируемо.
\end{theorem}

\newpage
\subsection{Понятие двойного интеграла}

Определённый интеграл --- интеграл по отрезку. Двойной интеграл ---
интеграл по квадрируемому компакту. Все рассматриваемые далее
квадрируемые компакты предполагаются непустыми; некоторые из них
могут иметь нулевую площадь.

\subsubsection{Разбиение квадрируемых компактов}

\emph{Разбиением} $T$ квадрируемого компакта $\Kc$ на координатной
плоскости $\Pi\colon xOy$ назовём всякое представление этого
компакта в виде объединения конечного семейства квадрируемых
компактов $\Kc=\bigcup\limits_{k=1}^m \si_k$, у которых $]\si_j[ \;
\cap\; ]\si_i[=\es$, $i\ne j$. Компакты $\si_k$, $k=\ol{1,m}$,
называются \emph{ячейками} разбиения $T$; площадь ячейки $\si_k$
обозначим $\De\si_k$. Наибольший из диаметров $\diam\si_k$ ячеек
$\si_k$, $k=\ol{1,m}$, назовём \emph{диаметром} разбиения $T$ и
обозначим символом $d(T)$.

На множестве всех разбиений квадрируемого компакта $\Kc$ можно
ввести \textbf{отношение порядка}, считая, что $T'\le T$, если
каждая ячейка разбиения $T'$ является либо ячейкой разбиения $T$,
либо объединением конечного числа ячеек разбиения $T$. Говорят
также, что разбиение $T$ получено \emph{дроблением} разбиения $T'$.

\begin{stmm}
1 Для любых разбиений $T'$ и $T''$ компакта $\Kc$ существует такое
его разбиение $T$, что $T'\le T$ и $T''\le T$.
\end{stmm}

\begin{proof}
Пусть $\si'_k, \; k=\ol{1,m}$ --- ячейки разбиения $T'$ и $\si''_l$,
$l=\ol{1,n}$ --- ячейки разбиения $T''$. Рассмотрим множества
$A=\hc{(k,l) \; | \; \si'_k\cap \si''_l \ne \es}$, $A_k = \hc{l \; |
\; (k,l)\in A}, \; 1\le k\le m$, и обозначим $\si_{kl} =
\si'_k\cap\si''_l$, если $(k,l)\in A$. Тогда все $\si_{kl}\ne\es$ и
каждое $\si_{kl}$ --- квадрируемый компакт (как пересечение
квадрируемых компактов). При этом справедливы свойства:

\begin{points}{-2}
\item $\Kc = \cupl{k=1}m \si'_k \cap \cupl{l=1}n \si''_l =
\cupl{k=1}m \cupl{l=1}n (\si'_k\cap\si''_l)=\cups{(k,l)\in A}
\si_{kl}$;

\item если $(k_1,l_1)$ и $(k_2,l_2)$ --- различные пары в $A$, то
$]\si_{k_1 l_1}[ \; \cap \; ]\si_{k_2l_2}[=\es$.


Действительно, пусть, например, $l_1\ne l_2$. Тогда $]\si_{k_1l_1}[
\;\cap \; ]\si_{k_2l_2}[ \; \subset \; ]\si''_{l_1}[ \; \cap \;
]\si''_{l_2}[=\es$.

Итак, совокупность $\hc{\si_{kl}}, \; (k,l)\in A$, образует
некоторое разбиение $T$ компакта $\Kc$.

\item $T'\le T$ и $T''\le T$.

Действительно, $\si'_k=\si'_k\cap\Kc=\si'_k \cap \cupl{l=1}n \si''_l
= \cupl{l=1}n (\si'_k\cap\si''_l) = \cups{l\in A_k} \si_{kl}$ для
любого $k$, $1\le k\le m$, и значит, $T'\le T$. Аналогично
доказывается, что $T''\le T$.
\end{points}
\end{proof}

Разбиение $T$ в доказанном утверждении принято обозначать в форме
$T=T'\vee T''$.

\subsubsection{Размеченные разбиения квадрируемого компакта}

Рассмотрим произвольное разбиение $T$ квадрируемого компакта $\Kc$
на ячейки $\si_k, \; k=\ol{1,m}$. Выбирая в каждой ячейке $\si_k$
произвольную точку $\ze_k=(\xi_k,\eta_k)$, получаем набор $\ze$
точек $\hc{\ze_1\sco\ze_m}, \; \ze_k\in\si_k, \; k=\ol{1,m}$.
Разбиение $T$ квадрируемого компакта $\Kc$, к которому присоединён
некоторый набор $\ze=\hc{\ze_1\sco\ze_m}$ называют
\textbf{размеченным разбиением} компакта $\Kc$ и обозначают символом
$T_{\ze}$.

На множестве $\Ps$ всех размеченных разбиений квадрируемого компакта
$\Kc$ рассмотрим семейство $\hc{B_{\de}}$ множеств $B_{\de} =
\{T_{\ze}\in\Ps \; | \; d(T_{\ze}) < \de\}, \; \de>0$, и покажем,
что семейство $\{B_{\de}\}$ образует базу на $\Ps$.

Проверим выполнение первого свойства базы; то есть, что каждое
множество $B_{\de}, \; \de>0$, не пусто. Согласно свойству
архимедовости множества действительных чисел, для любого числа
$\de>0$ существует такое натуральное число $n$, что
$2^n>\frac{\sqrt2}{\de}$. Рассмотрим на координатной плоскости $\Pi$
сетку шага $h=\frac1{2^n}$ и обозначим $B_n(\Kc)$ объединение всех
элементарных квадратов $Q_k$, $k=\ol{1,m}$, этой сетки, у которых
$]Q_k[ \; \cap \; \Kc\ne\es$. Тогда многоугольная фигура
$B_n(\Kc)\supset\Kc$. Обозначим $\si_k=\Kc\cap Q_k, \; k=\ol{1,m}$.
Так как $]Q_i[ \; \cap \; ]Q_j[=\es, \; i\ne j$, то $]\si_i[ \; \cap
\; ]\si_j[=\es, \; i\ne j$. Таким образом, компакты $\si_k, \;
k=\ol{1,m}$, служат ячейками некоторого разбиения $T$ компакта
$\Kc$, у которого $d(T)\le \diam(Q_k)=\frac{\sqrt2}{2^n}<\de, \;
k=\ol{1,m}$, и следовательно, множество $B_{\de}\ne\es$ для любого
$\de>0$.

Прежде, чем проверить выполнение второго свойства базы, заметим, что
$B_{\de_1}\subset B_{\de_2}$ для любых $0<\de_1\le\de_2$, поскольку
для каждого элемента $T_{\ze}\in B_{\de_1}$ имеем
$d(T_{\ze})<\de_1\le\de_2$, и следовательно, $T_{\ze}\in B_{\de_2}$.
Теперь, для произвольных $\de_1>0$ и $\de_2>0$ выберем
$\de_3=\min(\de_1,\de_2), \; \de_3>0$, и, по предыдущему,
$B_{\de_3}\subset B_{\de_1}\cap B_{\de_2}$.

База $\hc{B_{\de}}$ на $\Ps$ имеет специальное обозначение
$d(T)\ra0$.

\subsubsection{Интегральные суммы Римана}

Рассмотрим на квадрируемом компакте $\Kc$ произвольную функцию
$f(x,y)$. Для любого размеченного разбиения $T_{\ze}$ компакта $\Kc$
с ячейками $\si_k, \; k=\ol{1,m}$, и набором $\ze=(\ze_1\sco\ze_m)$
точек $\ze_k=(\xi_k,\eta_k)\in\si_k, \; k=\ol{1,m}$, число
\eqa{1}{\si(f;T_{\ze})=\sumkum f(\xi_k,\eta_k)\De \si_k} называют
\textbf{интегральной суммой} Римана функции $f(x,y)$, отвечающей
размеченному разбиению $T_{\ze}$ компакта $\Kc$.

Таким образом, произвольная функция $f(x,y)$, определённая на
квадрируемом компакте $\Kc$, порождает некоторое отображение
$\Ph_f\colon \Ps\ra\R$, определяемое условием
$\Ph_f(T_{\ze})=\si(f;T_{\ze}), \; T_{\ze}\in\Ps$, в котором числа
$\si(f:T_{\ze})$ задаются формулой (1).

\subsubsection{Двойной интеграл Римана по квадрируемому компакту}
\label{124}
\begin{dfn}
1 Число $I\in\R$ называют двойным интегралом функции $f(x,y)$ по
квадрируемому компакту $\Kc$, если $I=\liml{d(T)\ra0}\Ph_f$.
Обозначения: \equ{I=\iints{\Kc}f(x,y)d\si=\iints{\Kc}
f(x,y)\,dx\,dy=\liml{d(T)\ra0}\si(f;T_{\ze}).}
\end{dfn}

\begin{dfn}
{1'} Число $I\in\R$ называют двойным интегралом функции $f(x,y)$ по
квадрируемому компакту $\Kc$, если для любого числа $\ep>0$
существует такое число $\de>0$, что для всякого размеченного
разбиения $T_{\ze}$ компакта $\K$, диаметр $d(T_{\ze})$ которого
меньше $\de$, справедливо неравенство \eqa{2}{\hm{I-\si(f;T_{\ze})}
= \hm{I - \sumkun f(\xi_k,\eta_k)\De \si_k}<\ep.}
\end{dfn}

По своей форме определения 1 и 1' двойного интеграла аналогичны
определению одномерного интеграла Римана на отрезке. Но в одномерном
случае из условия, аналогичного условию (2), следует свойство
ограниченность интегрируемой функции на отрезке. В многомерном
случае ситуация сложнее. Имеет место следующее утверждение.

\begin{stmm}
1 Если компакт $\Kc$ имеет нулевую площадь, то для любой функции
$f(x,y)$, определённой на $\Kc$, двойной интеграл
$\iints{\Kc}f(x,y)\,dx\,dy=0$.
\end{stmm}

\begin{proof}
Рассмотрим произвольное размеченное разбиение $T_{\ze}$ компакта
$\Kc$ с ячейками $\si_k, \; k=\ol{1,m}$, и набором $\ze$ точек
$\ze_k=(\xi_k,\eta_k)\in\si_k, \; k=\ol{1,m}$. Так как
$\Kc=\cupl{k=1}m\si_k$ и $пл.\,\Kc=0$, то $пл.\,\si_k=\De\si_k=0, \;
k=\ol{1,m}$. Поэтому интегральная сумма $\si(f;T_{\ze})=\sumkun
f(\xi_k,\eta_k)\De\si_k=0$ для любого $T_{\ze}\in\Ps$, так что
$\Ph_f\colon\Ps\ra\R$, $\Ph_f(T_{\ze})=0,$ $T_{\ze}\in\Ps$, и
следовательно, $\liml{d(T)\ra0}\Ph_f=0$.
\end{proof}

Отмеченное обстоятельство заставляет нас в дальнейшем изучении
сузить множество функций, имеющих двойные интегралы по квадрируемым
компактам, дополнительным условием их ограниченности на компактах.

\begin{dfn}
2 Функцию $f(x,y)$ называют интегрируемой по Риману на квадрируемом
компакте $\Kc$, если она ограничена на $\Kc$ и обладает двойным
интегралом по $\Kc$ (называемым в этом случае двойным интегралом
Римана).
\end{dfn}

Так как, по определению, двойной интеграл есть предел по базе
некоторого отображения (функции), то всякая функция, интегрируема на
квадрируемом компакте, обладает единственным двойным интегралом
Римана.

\subsubsection{Вычисление площади квадрируемого компакта}
\label{125} Рассмотрим произвольный квадрируемый компакт $\Kc$.
Интегральные суммы $\si(f_0;T_{\ze})$ функции $f_0(x,y)=1$,
$(x,y)\in\Kc$, для любого $T_{\ze}\in\Ps$ с ячейками $\si_k, \;
k=\ol{1,m}$, имеют вид $\si(f_0,T_{\ze})=\sumkun
f_0(\xi_k,\eta_k)\De\si_k=\sumkun\De\si_k$. Так как $]\si_i[ \; \cap
\; ]\si_j[=\es, \; i\ne j$, и $\Kc=\cupl{k=1}m \si_k$, то по теореме
1.4, $\sumkum\De\si_k=пл.\,\Kc$, и значит,
$\si(f_0;T_{\ze})=пл.\,\Kc$ для всех $T_{\ze}\in\Ps$. Таким образом,
отображение $\Ph_{f_0}\colon\Ps\ra\R$,
$\Ph_{f_0}(T_{\ze})=пл.\,\Kc$, $T_{\ze}\in\Ps$ --- постоянное, и
поэтому,имеет $\liml{d(T)\ra0}\Ph_{f_0}=пл.\,\Kc$. Итак,
\equ{пл.\,\Kc = \iints{\Kc}1\cdot\,dx\,dy=\iints{\Kc}\,dx\,dy.}

Аналогично доказывается, что для любой постоянной функции
$f(x,y)=c$, $c\in\R$, на квадрируемом компакте $\Kc$ существует
двойной интеграл $\iints{\Kc}c\,dx\,dy=c\cdot пл.\,\Kc$; в
частности, $\iints{\Kc}0\,dx\,dy=0$.

\subsection{Теория Дарбу двойного интеграла Римана}
\subsubsection{Суммы Дарбу}

Пусть функция $f(x,y)$ определена и ограничена на квадрируемом
компакте $\Kc$ плоскости $\Pi\colon xOy$. Рассмотрим произвольное
разбиение $T$ компакта $\Kc$ на квадрируемые ячейки $\si_k$,
$k=\ol{1,m}$. Тогда существуют числа
\eqa{1}{m_k=\infl{(x,y)\in\si_k}f(x,y), \;
M_k=\supl{(x,y)\in\si_k}f(x,y), \; k=\ol{1,m}} и числа
\equ{s(f;T)=\sumkum m_k\De\si_k, \; S(f;T)=\sumkum M_k\De\si_k,}
называемые, соответственно, нижней и верхней суммами Дарбу функции
$f$, отвечающими разбиению $T$. Поскольку $m_k\le M_k$ и
$\De\si_k\ge0, \; k=\ol{1,m}$, то $s(f;T)\le S(f;T)$ для любого
разбиения $T$ компакта $\Kc$.

Известно (см. материал первого семестра), что разность $M_k-m_k\ge0$
равна колебанию $\om(f;\si_k)$ функции $f$ на множестве $\si_k$.
Поэтому \eqa{2}{0\le S(f;T)-s(f;T)=\sumkum(M_k-m_k)\De\si_k=\sumkum
\om(f;\si_k)\De\si_k.}

Рассмотрим произвольный набор $\ze=(\ze_1\sco\ze_m)$ точек
$\ze_k=(\xi_k,\eta_k)\in\si_k$, $k=\ol{1,m}$, и интегральную сумму
$\si(f;T_{\ze})$. Поскольку $m_k\le f(\xi_k,\eta_k)\le M_k$ для
любого $k, \; 1\le k\le m$, то заключаем, что
\eqa{3}{s(f;T)\le\si(f;T_{\ze})\le S(f;T)} для любого разбиения $T$
компакта $\Kc$ и любого набора точек $\ze=(\ze_1\sco\ze_m)$.

\subsubsection{Свойство монотонности сумм Дарбу}

\begin{theorem}
Если разбиение $T'$ квадрируемого компакта $\Kc$ получено дроблением
его разбиения $T$; то есть, $T\le T'$, то $s(f;T')\ge s(f;T)$ и
$S(f;T')\le S(f;T)$.
\end{theorem}

\begin{proof}
Утверждение теоремы достаточно проверить в случае, когда разбиение
$T'$ получено путём дробления какой-то единственной ячейки $\si_k$,
$1\le k\le m$, разбиения $T$ компакта $\Kc$ на две ячейки $\si_k^1$
и $\si_k^2$, $\si_k=\si_k^1\cup\si_k^2$, $]\si_k^1[ \; \cap \;
]\si_k^2[=\es$. Тогда числа (1) и
\equ{m_k^i=\infl{(x,y)\in\si_k^i}f(x,y), \;
M_i^k=\supl{(x,y)\in\si_k^i}f(x,y), \; i=1,2,} связаны отношениями
\eqa{4}{m_k^i\ge m_k \mbox{ и } M_k^i\le M_k, \; i=1,2.} Понятно,
что для $S(f;T')$, $s(f;T')$ и $S(f;T), \; s(f;T)$ справедливы
соотношения \eqa{5}{S(f;T')-S(f;T) = M_k^1\De\si_k^1 +
M_k^2\De\si_k^2 - M_k\De\si_k} и \eqa{6}{s(f;T')-s(f;T) =
m_k^1\De\si_k^1 + m_k^2\De\si_k^2 - m_k\De\si_k,} в которых
$\De\si_k^i=пл.\,\si_k^i$, $i=1,2$, и
$\De\si_k=\De\si_k^1+\De\si_k^2$.

На основании (4)--(6) и свойства $\De\si_k^i\ge0$ заключаем, что
\equ{S(f;T')-S(f;T)\le M_k\De\si_k^1+M_k\De\si_k^2-M_k\De\si_k=0} и
\equ{s(f;T')-s(f;T)\ge m_k\De\si_k^1 + m_k\De\si_k^2 -
m_k\De\si_k=0.}
\end{proof}

\subsubsection{Свойство отделимости сумм Дарбу}
\begin{theorem}
Для любых разбиений $T_1$ и $T_2$ квадрируемого компакта $\Kc$
справедливо неравенство $s(f;T_1)\le S(f;T_2)$.
\end{theorem}

\begin{proof}
В силу утверждения 1 пункта 3.1, для разбиения $T=T_1\vee T_2$ имеем
$T_i\le T, \; i=1,2$. По теореме предыдущего пункта и (3),
$s(f;T_1)\le s(f;T)\le S(f;T)\le S(f;T_2)$.
\end{proof}

\subsubsection{Основное свойство сумм Дарбу}
\begin{theorem}
Пусть функция $f(x,y)$ определена и ограничена на квадрируемом
компакте $\Kc$. Тогда, для любого фиксированного разбиения $T$
компакта $\Kc$ её верхняя (нижняя) сумма $S(f;T)$ ($s(f;T)$) равна
точной верхней (точной нижней) грани множества $\hc{\si(f;T_{\ze})}$
интегральных сумм функции $f$, отвечающих этому разбиению при
всевозможных наборах точек $\ze$; то есть,
$S(f;T)=\supl{\ze}\si(f;T_{\ze})$ и
$s(f;T)=\infl{\ze}\si(f;T_{\ze})$.
\end{theorem}

\begin{proof}
Если $пл.\,\Kc=0$, то, по определению,
$s(f;T)=\si(f;T_{\ze})=S(f;T)=0$ для любого размеченного разбиения
$T_{\ze}$ компакта $\Kc$, и утверждение теоремы доказано. Пусть
теперь $пл.\,\Kc>0$. Дальнейшее доказательство проведём для верхней
суммы $S(f;T)$. Обозначим ячейки фиксированного разбиения $T$ через
$\si_k$, $k=\ol{1,m}$. Из неравенств (3) следует, что число $S(f;T)$
служит верхней гранью для множества $\hc{\si(f;T_{\ze})}$ при
всевозможных наборах $\ze=(\ze_1\sco\ze_m)$, $\ze_k\in\si_k, \;
k=\ol{1,m}$.

Рассмотрим теперь произвольное число $\ep>0$. Так как
$M_k=\supl{\si_k}f$, $k=\ol{1,m}$, то на каждой $\si_k$ существует
такая точка $\ol{\ze}_k=(\ol{\xi}_k,\ol{\eta}_k)\in\si_k$, что $M_k
- \frac{\ep}{пл.\,\Kc} < f(\ol{\xi}_k,\ol{\eta}_k)\le M_k, \;
k=\ol{1,m}$. Получим набор $\ol{\ze}$ точек $(\ol{\ze}_1\sco
\ol{\ze}_m)$, в которых $M_k < f(\ol{\xi}_k, \ol{\eta}_k) +
\frac{\ep}{пл.\,\Kc}, \; k=\ol{1,m}$. Для размеченного разбиения
$T_{\ol{\ze}}$ интегральная сумма $\si(f;T_{\ol{\ze}}) = \sumkum
f(\ol{\xi}_k,\ol{\eta}_k)\De\si_k$ связана с верхней суммой
$S(f;T)=\sumkum M_k\De\si_k$ отношениями
\equ{S(f;T)\le\si(f;T_{\ol{\ze}}) + \frac{\ep}{пл.\,\Kc}
\sumkum\De\si_k = \si(f;T_{\ol{\ze}}) +
\frac{\ep}{пл.\,\Kc}пл.\,\Kc=\si(f;T_{\ol{\ze}})+\ep,} или
$S(f;T)-\ep<\si(f;T_{\ol{\ze}})$. Таким образом,
$S(f;T)=\supl{\ze}\si(f;T_{\ze})$.

Аналогично доказывается, что $s(f;T)=\infl{\ze}\si(f;T_{\ze})$.
\end{proof}

\subsubsection{Нижний и верхний интегралы Дарбу}

Рассмотрим произвольную функцию $f(x,y)$, ограниченную на
квадрируемом компакте $\Kc$. Согласно свойству отделимости сумм
Дарбу, множества $\hc{s(f;T)}$ и $\hc{S(f;T)}$ всех нижних и верхних
сумм функции $f(x,y)$ удовлетворяют принципу отделяющего отрезка, по
которому существуют $\sup\hc{s(f;T)}=\ul{I}$ и
$\inf\hc{S(f;T)}=\ol{I}$ и $\ul{I}\le\ol{I}$. Числа $\ul{I}$ и
$\ol{I}$ называют соответственно нижним и верхним интегралами Дарбу
функции $f$ на компакте $\Kc$. Понятно, что \equ{s(f;T) \le \ul{I}
\le \ol{I} \le S(f;T)} для любого разбиения $T$ компакта $\Kc$.
Кроме того, в силу принципа отделяющего отрезка, числа $\ul{I}$ и
$\ol{I}$ равны в том и только в том случае, когда для любого числа
$\ep>0$ можно указать такие разбиения $T_1$ и $T_2$ компакта $\Kc$,
для которых разность $S(f;T_1)-s(f;T_2)<\ep$. Рассматривая теперь
разбиение $T=T_1\vee T_2$, так что $T_1\le T$ и $T_2\le T$,
заключаем на основании свойства монотонности сумм Дарбу, что
$S(f;T_1)\ge S(f;T)$ и $s(f;T_2)\le s(f;T)$, и значит, $0\le
S(f;T)-s(f;T)\le S(f;T_1)-s(f;T_2)<\ep$.

Таким образом, $\ul{I}=\ol{I}$ тогда и только тогда, когда для
любого числа $\ep>0$ можно указать такое разбиение $T$ компакта
$\Kc$, что \eqa{7}{0\le S(f;T)-s(f;T)<\ep.}

\subsubsection{Критерии интегрируемости функции на компакте}
\label{136} Для функции $f(x,y)$, определённой и ограниченной на
квадрируемом компакте $\Kc$, рассмотрим отображения
$\psi_f\colon\Ps\ra\R$, $\psi_f(T_{\ze})=s(f;T)$, $T_{\ze}\in\Ps$ и
$\Psi_f\colon\Ps\ra\Kc, \; \Psi_f(T_{\ze})=S(f;T), \;
T_{\ze}\in\Ps$. Неравенство (3) принимает вид
\eqa{3'}{\psi_f(T_{\ze})\le\Ph_f(T_{\ze})\le \Psi_f(T_{\ze}), \;
T_{\ze}\in\Ps,} где, напомним, $\Ph_f\colon \Ps\ra\R$,
$\Ph_f(T_{\ze})=\si(f;T_{\ze})$.

Условимся обозначать $\liml{d(T)\ra0}\psi_f = \liml{d(T)\ra0}
s(f;T)$ и $\liml{d(T)\ra0} \Psi_f = \liml{d(T)\ra0} S(f;T)$, если
пределы существуют.

\begin{theorem}
\label{thm361} Функция $f(x,y)$, определённая и ограниченная на
квадрируемом компакте $\Kc$, интегрируема на $\Kc$ тогда и только
тогда, когда $\liml{d(T)\ra0} [S(f;T)-s(f;T)]=0$. При этом,
$$\iints{\Kc} f(x,y)\,dx\,dy=I=\liml{d(T)\ra0} S(f;T) =
\liml{d(T)\ra0} s(f;T)=\ul{I}=\ol{I}$$.
\end{theorem}

\begin{proof}
\textbf{Достаточность}. В силу свойства отделимости сумм Дарбу,
существует такое число $I$, что $s(f;T_1)\le I\le S(f;T_2)$ для
любых разбиений $T_1$ и $T_2$ компакта $\Kc$. В частности,
\eqa{8}{s(f;T)\le I \le S(f;T)} для любого разбиения $T$ компакта
$\Kc$. Присоединяя к (8) неравенства (3), получим
$\hm{\si(f;T_{\ze})-I} \le S(f;T)-s(f;T)$ для любого
$T_{\ze}\in\Ps$. Так как, по условию, $\liml{d(T)\ra0}
[S(f;T)-s(f;T)]=0$, то $\liml{d(T)\ra0}\hm{\si(f;T_{\ze})-I}=0$, или
$I=\liml{d(T)\ra0} \si(f;T_{\ze})=\iints{\Kc}f(x,y)\,dx\,dy$.

Далее, из (8) следуют неравенства $\hm{I-s(f;T)}\le S(f;T)-s(f;T)$ и
$\hm{I-S(f;T)} \le S(f;T)-s(f;T)$ для всех $T_{\ze}\in\Ps$. Как и
выше, заключаем, что $I=\liml{d(T)\ra0} s(f;T)$ и $I=\liml{d(T)\ra0}
S(f;T)$.

\textbf{Необходимость}. По условию, существует $\liml{d(T)\ra0}
\si(f;T_{\ze})=\iints{\K} f(x,y)\,dx\,dy=I$. Рассмотрим произвольное
число $\ep>0$. По определению предела функции по базе, существует
такое число $\de>0$, что для любого размеченного разбиения $T_{\ze}$
компакта $\Kc$ с диаметром $d(T_{\ze})<\de$ справедливо неравенство
$\hm{\si(f;T_{\ze})-I}<\frac{\ep}2$, или $I-\frac{\ep}2 <
\si(f;T_{\ze}) < I+\frac{\ep}2$, откуда, на основании свойств сумм
Дарбу, имеем \eqa{9}{I-\frac{\ep}2 \le \infl{\ze}
\si(f;T_{\ze})=s(f;T) \le \ul{I} \le \ol{I} \le S(f;T) = \supl{\ze}
\si(f;T_{\ze}) \le I+\frac{\ep}2} для всех $T_{\ze}\in\Ps$,
$d(T_{\ze})<\de$. Поэтому, $0\le S(f;T)-s(f;T)\le\ep$ для всех
$T_{\ze}\in\Ps$, $d(T_{\ze})<\de$, или $\liml{d(T)\ra0}
[S(f;T)-s(f;T)]=0$. Кроме того, на основании (9), оценки
$\hm{I-s(f;T)}\le\frac{\ep}2<\ep$, $\hm{I-S(f;T)}\le\frac{\ep}2 <
\ep$ справедливы для всех $T_{\ze}\in\Ps$, $d(T_{\ze})<\de$, и
следовательно, $I=\liml{d(T)\ra0}s(f;T)=\liml{d(T)\ra0} S(f;T)$.
\end{proof}

\begin{theorem}
\label{thm362} Для того, чтобы функция $f(x,y)$, определённая и
ограниченная на квадрируемом компакте $\Kc$, была интегрируемой на
$\Kc$, необходимо и достаточно, чтобы для любого числа $\ep>0$
нашлось такое разбиение $T$ компакта $\Kc$, для которого $0\le
S(f;T)-s(f;T)<\ep$.
\end{theorem}

\begin{proof}
\textbf{Необходимость} доказана в предыдущей теореме \ref{thm361}.
Для доказательства \textbf{достаточности} отметим, что условие
теоремы совпадает с условием (7), согласно которому $\ul{I}=\ol{I}$
и общее значение $I=\ul{I}=\ol{I}$ равно двойному интегралу функции
$f(x,y)$ по компакту $\Kc$.
\end{proof}

\subsubsection{Геометрический смысл двойного интеграла}
\begin{imp}
Если функция $f(x,y)\ge0$ на квадрируемом компакте $\Kc$
координатной плоскости $\Pi\cln xOy$ и $f(x,y)$ интегрируема на
$\Kc$, то её подграфик $\Ph_f=\hc{(x,y,z) \bvl 0\le z\le f(x,y), \;
(x,y)\in \Kc}$ кубируема и $об.\,\Ph_f=\iints{\Kc}f(x,y)\,dx\,dy$.
\end{imp}

\begin{proof}
Рассмотрим произвольное разбиение $T$ компакта $\Kc$ на ячейки
$\si_k, \; k=\ol{1,m}$, и обозначим $$0\le
m_k=\infl{(x,y)\in\si_k}f(x,y) \le \supl{(x,y)\in\si_k} f(x,y)=M_k,
k=\ol{1,m}.$$ Согласно теореме \ref{thm179}, число $m_k\De\si_k$
равно объёму цилиндрического тела $\Ph_k=\si_k\times[0,m_k]$ с
основанием $\si_k$, к упирающегося снизук в график $\Ga_f$
функции $f(x,y)$. При этом, если $k\ne j$, то $\Ph_k$ и $\Ph_j$ не
имеют общих внутренних точек, поскольку проекции их внутренних точек
на плоскость $xOy$ являются внутренними точками их оснований (ячеек
$\si_k$ и $\si_j$), а последние общих внутренних точек не имеют.
Следовательно, к ступенчатое телок $A_T=\cupl{k=1}m
(\si_k\times[0,m_k])$, к вписанноек в подграфик $\Ph_f$,
кубируемо и $s(f,T)=\sumkum m_k\De\si_k$ --- его объём. Совершенно
аналогично $S(f;T)$ есть объём к ступенчатого телак
$B_T=\cupl{k=1}m (\si_k\times[0,M_k])$, к описанногок около
$\Ph_f$. Но, согласно теореме \ref{thm362}, для каждого $\ep>0$
существует такое разбиение $T$ компакта $\Kc$, что $0\le
S(f;T)-s(f;T)<\ep$. Тем самым для каждого $\ep>0$ существуют такие
кубируемые тела $A (=A_T)$ и $B (=B_T)$, что $A\subset \Ph_f\subset
B$ и $об.\, B-об.\,A<\ep$. А это ещё означает, что $\Ph_f$
кубируемо. Так как при этом $s(f;T)=об.\,A\le об.\, B=S(f;T)$ и
$s(f;T) \le \iints{\Kc} f(x,y)\,dx\,dy\le S(f;T)$, то $\hm{об.\,
\Ph_f - \iints{\Kc} f(x,y)\,dx\,dy} \le S(f;T) - s(f;T)<\ep$, и
поскольку это верно для всех $\ep>0$, то $\iints{\Kc}
f(x,y)\,dx\,dy=об.\,\Ph_f$.
\end{proof}

\begin{imp}
График $\Ga_f$ функции $f(x,y)$, интегрируемой на квадрируемом
компакте $\Kc\subset\Pi$, кубируем и имеет нулевой объём.
\end{imp}

\begin{proof}
Будучи интегрируемой, функция $f(x,y)$ ограничена на $\Kc$. Пусть
$T$ --- произвольное разбиение компакта $\Kc$ с ячейками $\si_k, \;
k=\ol{1,m}$, и $m_k=\infl{\si_k}f(x,y), \; M_k=\supl{\si_k}f(x,y),
\; k=\ol{1,m}$, и $s(f;T), \; S(f;T)$ --- нижняя и верхняя суммы для
разбиения $T$. Рассмотрим тело $\Fc_T = \cupl{k=1}m V_k$, где
$V_k=\si_k\times[m_k,M_k], \; k=\ol{1,m}$. Очевидно,
$\Fc_T\supset\Ga_f$. Согласно теореме \ref{thm179}, цилиндрическое
тело $V_k, \; k=\ol{1,m}$, кубируемо и $об.\,V_k=(M_k-m_k)\De\si_k$.
Поэтому, $\Fc_T$ кубируемо и $об.\,\Fc_T\le \sumkum об.\,V_k =
S(f;T) - s(f;T)$. По теореме \ref{thm361}, $\liml{d(T)\ra0}
[S(f;T)-s(f;T)]=0$. Таким образом, $\Ga_f$ содержится в кубируемых
телах произвольно малого объёма; то есть, является кубируемым телом
нулевого объёма.
\end{proof}

\subsubsection{Интегрируемость непрерывной функции}

\begin{theorem}
\label{thm381}Всякая непрерывная функция на квадрируемом компакте
интегрируема.
\end{theorem}

\begin{proof}
Пусть $\Kc$ --- квадрируемый компакт. Если $пл.\,\Kc=0$, то на $\Kc$
интегрируема любая функция и интеграл равен нулю. Пусть теперь
$пл.\,\Kc>0$, и $f(x,y)$ --- непрерывная функция на $\Kc$. Тогда
$f(x,y)$ равномерно непрерывна на $\Kc$, так что для произвольного
числа $\ep>0$ существует такое число $\de>0$, что для любых точек
$\ze,\ta\in\Kc$, расстояние $d(\ze,\ta)$ между которыми меньше
$\de$, справедливо неравенство $\hm{f(\ze)-f(\ta)} <
\frac{\ep}{пл.\,\Kc}$. Рассмотрим произвольное разбиение $T$
компакта $\Kc$ с диаметром $d(T)<\de$ и с ячейками $\si_k$,
$k=\ol{1,m}$. Так как все $\si_k$ --- компакты, а $f(x,y)$
непрерывна на $\si_k$, то по теореме Вейерштрасса для каждого $k$,
$1\le k\le m$, существуют такие точки $\ze'_k,\ze''_k\in\si_k$, что
$m_k=\infl{\si_k} f(x,y) = f(\ze'_k)$ и $M_k = \supl{\si_k}
f(x,y)=f(\ze''_k), \; k=\ol{1,m}$. При этом, $d(\ze'_k,\ze''_k)\le
\diam\si_k\le d(T)< \de$. Следовательно, \equ{S(f;T)-s(f;T) =
\sumkum (M_k-m_k)\De\si_k = \sumkum \hm{f(\ze''_k) -
f(\ze'_k)}\De\si_k < \frac{\ep}{пл.\,\Kc} \sumkum\De\si_k=\ep} и
остаётся применить теорему \ref{thm361}.
\end{proof}

\subsubsection{Интегрируемость функций, точки разрыва которых
образуют множества нулевой площади}

\begin{theorem}
\label{thm391} Ограниченная на квадрируемом компакте $\Kc$ функция
$f(x,y)$, точки разрыва которой образуют фигуру нулевой площади,
интегрируема на $\Kc$.
\end{theorem}

\begin{proof}
Нужно рассмотреть только случай, когда $пл.\,\Kc>0$ и множество
$F\subset\Kc$ точек разрыва функции $f(x,y)$ не пусто, так что
$f(x,y)$ непрерывна на $\Kc\backslash F$. Кроме того, колебание
$\om=\om(f;\Kc)$ функции $f(x,y)$ на $\Kc$ конечно и положительно.
Рассмотрим произвольное число $\ep>0$. Согласно лемме 1 пункта
\ref{114}, найдётся сетка с шагом $h$, что $F\subset ]Q[$, где $Q$
--- объединение элементарных квадратов $Q_j$, $j=\ol{1,n}$ сетки и
$пл.\,Q<\min\hr{\frac{\ep}{2\om},пл.\,\Kc}$. Тогда множество $\Kc_1
= \Kc\backslash ]Q[$ будет квадрируемым компактом с $пл.\,\Kc_1>0$,
на котором $f(x,y)$ непрерывна (поскольку $\Kc_1\subset\Kc\backslash
F$). Как и в доказательстве теоремы предыдущего пункта, для $\ep>0$
находим такое разбиение $T_1$ компакта $\Kc_1$ с ячейками $\si_i, \;
i=\ol{1,m}$, что $\om(f;\si_i) < \frac{\ep}{2пл.\,\Kc}$,
$i=\ol{1,m}$. Объединение ячеек $\si_i$, $i=\ol{1,m}$, и компактов
$Q_j\cap \Kc\ne\es$, $j=\ol{1,n}$, образует некоторое разбиение $T$
исходного компакта $\Kc$, для которого \equ{0\le S(f;T)-s(f;T) =
\sumium \om(f;\si_i)\De\si_i + \suml{j=1}n \om(f;Q_j\cap\Kc) пл.\,
(Q_j\cap\Kc).}

При этом, \equ{\sumium \om(f;\si_i)\De\si_i < \sumium
\frac{\ep}{2пл.\,\Kc} \De\si_i = \frac{\ep}{2пл.\,\Kc}
\sumium\De\si_i \le \frac{\ep}{2пл.\,\Kc} пл.\,\Kc = \frac{\ep}2} и
\equ{\suml{j=1}n \om(f;Q_j\cap\Kc)пл.\,(Q_j\cap\Kc) \le
\suml{j=1}n\om\cdot пл.\,(Q_j\cap\Kc) \le \om\cdot пл.\,Q < \om\cdot
\frac{\ep}{2\om} = \frac{\ep}2,} так что $0\le
S(f;T)-s(f;T)<\frac{\ep}2+\frac{\ep}2=\ep$. В силу критерия (пункт
\ref{136}), функция $f(x,y)$ интегрируема на $\Kc$.
\end{proof}

\newpage

\subsection{Основные свойства двойного интеграла}
\subsubsection{Свойство линейности}

\begin{theorem}
Если функции $f_i(x,y)$, $i=1,2$, интегрируемы на квадрируемом
компакте $\Kc$ на плоскости $\Pi\cln xOy$, то для любых
$\la_i\in\R$, $i=1,2$, функция $h(x,y)=\la_1 f_1(x,y) + \la_2
f_2(x,y)$ интегрируема на $\Kc$ и \eqa{1}{\iints{\Kc} (\la_1f_1(x,y)
+ \la_2f_2(x,y))\,dx\,dy = \la_1 \iints{\Kc} f_1(x,y)\,dx\,dy +
\la_2 \iints{\Kc}f_2(x,y)\,dx\,dy.}
\end{theorem}

\begin{proof}
Для любого размеченного разбиения $T_{\ze}$ компакта $\Kc$ имеем
\equ{\si(h,T_{\ze})=\la_1\si(f_1;T_{\ze}) + \la_2\si (f;T_{\ze}),}
откуда, на основании свойства линейности предела по базе $d(T)\ra0$,
получаем утверждение теоремы и формулу (1).
\end{proof}

\subsubsection{Свойство монотонности}

\begin{theorem}
Если функции $g(x,y)$ и $h(x,y)$ интегрируемы на квадрируемом
компакте на плоскости $\Pi\cln xOy$ и $g(x,y)\le h(x,y)$ для всех
$(x,y)\in\Kc$, то \eqa{2}{\iints{\Kc}g(x,y)\,dx\,dy \le \iints{\Kc}
h(x,y)\,dx\,dy.}
\end{theorem}

\begin{proof}
Для любого размеченного разбиения $T_{\ze}$ компакта $\Kc$ имеем
$\si(g;T_{\ze})\le\si(h;T_{\ze})$, откуда, на основании свойства
монотонности предела по базе $d(T)\ra0$ получаем формулу (2).
\end{proof}

\subsubsection{Теоремы о среднем значении для двойного интеграла}
\begin{theorem}
\label{thm431} Если функция $f(x,y)$ интегрируема на квадрируемом
компакте $\Kc$ с ненулевой площадью и $m\le f(x,y)\le M$ для всех
$(x,y)\in\Kc$, то \eqa{3}{m\le \frac1{пл.\,\Kc} \iints{\Kc}
f(x,y)\,dy\,dy\le M} (интегральное среднее функции заключено в тех
же границах, что и функция).
\end{theorem}

\begin{proof}
Согласно свойствам линейности и монотонности, \equ{m\cdot пл.\,\Kc =
\iints{\Kc}m\cdot dx\,dy \le \iints{\Kc} f(x,y)\,dx\,dy \le
\iints{\Kc} M\,dx\,dy = M\cdot пл.\,\Kc} и деление на $пл.\,\Kc>0$
даёт (3).
\end{proof}

\begin{theorem}
\label{thm432} (\ul{о среднем значении}). Интегральное среднее
непрерывной функции $f(x,y)$ на связном квадрируемом компакте $\Kc$
с ненулевой площадью равно её значению в некоторой точке этого
компакта.
\end{theorem}

\begin{proof}
Так как $f(x,y)$ непрерывна на компакте $\Kc$, то по теореме
Вейерштрасса она ограничена на $\Kc$ и существуют $(x_i,y_i)\in\Kc$,
$i=1,2$, что $f(x_1,y_1)=\infl{\Kc} f(x,y)=m$ и $f(x_2,y_2) =
\supl{\Kc} f(x,y)=M$. Поскольку $m\le f(x,y)\le M$, то по теореме
\ref{thm431} справедливо (3). В силу связности компакта $\Kc$,
существует точка $(\xi,\eta)\in\Kc$, в которой
$f(\xi,\eta)=\frac1{пл.\,\Kc} \iints{\Kc} f(x,y)\,dx\,dy.$
\end{proof}

\subsubsection{Интегрируемость модуля и оценка двойного интеграла}
\begin{theorem}
\label{thm441}Если функция $f(x,y)$ интегрируема на квадрируемом
компакте $\Kc$, то $\hm{f(x,y)}$ также интегрируема на $\Kc$ и
справедлива оценка \eqa{4}{\hm{\iints{\Kc}f(x,y)\,dx\,dy} \le
\iints{\Kc} \hm{f(x,y)}\,dx\,dy.}
\end{theorem}

\begin{proof}
Интегрируемая на $\Kc$ функция $f(x,y)$ ограничена на $\Kc$; то
есть, $\hm{f(x,y)}\le C$, $C>0$, $(x,y)\in\Kc$. Так как для
произвольного множества $E\subset\Kc$ справедливо
$\om(\hm{f},E)\le\om(f;E)$, то для любого разбиения $T$ компакта
$\Kc$ с ячейками $\si_k$, $k=\ol{1,m}$, справедливы оценки
\equ{S(\hm{f};T) - s(\hm{f};T) = \sumkum \om(\hm{f};\si_k) \De\si_k
\le \sumkum \om(f;\si_k)\De\si_k = S(f;T) - s(f;T).}

Поэтому, на основании критерия интегрируемости, из интегрируемости
на $\Kc$ функции $f(x,y)$ следует интегрируемость на $\Kc$ функции
$\hm{f(x,y)}$. Так как $-\hm{f(x,y)} \le f(x,y) \le \hm{f(x,y)}$ для
$(x,y)\in\Kc$, то на основании свойств линейности и монотонности,
\equ{-\iints{\Kc} \hm{f(x,y)}\,dx\,dy \le \iints{\Kc} f(x,y)\,dx\,dy
\le \iints{\Kc} \hm{f(x,y)}\,dx\,dy,} что равносильно (4).
\end{proof}

\subsubsection{Свойство аддитивности двойного интеграла}

\begin{theorem}
\label{thm451} Если функция $f(x,y)$ интегрируема на квадрируемых
компактах $\Kc_1$ и $\Kc_2$, не имеющих общих внутренних точек, и
$\Kc=\Kc_1\cup\Kc_2$, то $f(x,y)$ интегрируема на $\Kc$ и
справедлива формула \eqa{5}{\iints{\Kc} f(x,y)\,dx\,dy =
\iints{\Kc_1} f(x,y)\,dx\,dy + \iints{\Kc_2} f(x,y)\,dx\,dy.}
\end{theorem}

\begin{proof}
Как объединение двух квадрируемых фигур, $\Kc$ --- квадрируемая
фигура, а как объединение двух компактов --- компакт. Рассмотрим
произвольное число $\ep>0$. Так как $f$ интегрируема на $\Kc_1$ и
$\Kc_2$, существуют такое разбиение $T'$ компакта $\Kc_1$ с ячейками
$\si'_i$, $i=\ol{1,n'}$, $n'\in\N$, что
$S(f;T')-s(f;T')<\frac{\ep}2$, и такое разбиение $T''$ компакта
$\Kc_2$ с ячейками $\si_j''$, $j=\ol{1,n''}$, $n''\in\N$, что
$S(f;T'')-s(f;T'')<\frac{\ep}2$. Поскольку
$\Kc_1=\bigcup\limits_{i=1}^{n'} \si_i'$ и $\Kc_2 =
\bigcup\limits_{j=1}^{n''} \si_j''$, то $\Kc =
\bigcup\limits_{i=1}^{n'}\si_i'\cup\bigcup\limits_{j=1}^{n''}
\si_j'' = \bigcup\limits_{k=1}^n\si_k, \; n\in\N$, и при этом
$]\si'_i[ \; \cap \; ]\si''_j[ \; \subset \; ]\Kc_1[ \; \cap \;
]\Kc_2[ = \es$, $i=\ol{1,n'}$, $j=\ol{1,n''}$. Таким образом,
получим некоторое разбиение $T$ компакта $\Kc$, для которого
$S(f;T)=S(f;T')+S(f;T'')$ и $s(f;T)=s(f;T')+s(f;T'')$, так что
\equ{S(f;T)-s(f;T)=S(f;T')-s(f;T')+S(f;T'')-s(f;T'') <
\frac{\ep}2+\frac{\ep}2=\ep.} По критерию, функция $f$ интегрируема
на $\Kc$.

Чтобы доказать формулу (5), обозначим $I=\iints{\Kc}
f(x,y)\,dx\,dy$, $I_i = \iints{\Kc_i} f(x,y)\,dx\,dy$, $i=1,2$, и
опять рассмотрим произвольное число $\ep>0$. Согласно определению
двойного интеграла, существует такое $\de_1>0$, что
\eqa{6}{\hm{I-\si(f;T_{\ze})}<\frac{\ep}3} для всех размеченных
разбиений $T_{\ze}$ компакта $\Kc$ с диаметрами $d(T_{\ze})<\de_1$;
существует такое $\de_2>0$, что \eqa{7}{\hm{I_1 -
\si(f;T_{\ze'}')}<\frac{\ep}3} для всех размеченных разбиений
$T'_{\ze'}$ компакта $\Kc_1$ с диаметрами $d(T_{\ze'}')<\de_2$ и
существует такое $\de_3>0$, что \eqa{8}{\hm{I_2 -
\si(f;T_{\ze''}'')} < \frac{\ep}3} для всех размеченных разбиений
$T''_{\ze''}$ компакта $\Kc_2$ с диаметрами $d(T''_{\ze''})<\de_3$.
Положим $\de=\min(\de_1,\de_2,\de_3)$, $\de>0$, и рассмотрим такое
размеченное разбиение $T_{\ze'}'$ компакта $\Kc_1$ с
$d(T'_{\ze'})<\de \le \de_2$ и такое размеченное разбиение
$T''_{\ze''}$ компакта $\Kc_2$ с $d(T''_{\ze''})<\de \le\de_3$, у
которых наборы $\ze'$ и $\ze''$ не содержат общих точек (это всегда
возможно сделать в случае, когда внутренности $]\Kc_i[\ne\es, \;
i=1,2$; случай, когда $]\Kc_i[=\es$ для некоторого $i, \; 1\le
i\le2$, обсудим ниже). Объединение разбиений $T'_{\ze'}$ и
$T''_{\ze''}$ образует некоторое размеченное разбиение $T_{\ze}$
компакта $\Kc$ с $d(T_{\ze})<\de\le\de_1$, для которого справедлива
формула $\si(f;T_{\ze})=\si(f;T'_{\ze'}) +
\si(f;T''_{\ze''})$\footnote{Эта формула остаётся справедливой (и
даже упрощается) в случае, когда $]\Kc_i[=\es$ для какого-то $i, \;
1\le i\le2$ (скажем, для $\Kc_2$), поскольку тогда
$\Kc_2=гр.\,\Kc_2$, $пл.\,\Kc_2 = пл.\,(гр.\,\Kc_2)=0$ и
$\si(f;T_{\ze''}'')=0$ для всех $T''_{\ze''}$.}. Поэтому, с учётом
неравенств (6)--(8), имеем оценки \mla{9}{\hm{I-(I_1+I_2)} = \hm{I -
\si(f;T_{\ze}) - (I_1+I_2 - \si(f;T_{\ze}))} \le \hm{I -
\si(f;T_{\ze})} + \hm{I_1+I_2-\si(f;T_{\ze})} = \hm{I -
\si(f;T_{\ze})} + \\ + \hm{I_1 - \si(f;T'_{\ze'}) + I_2 -
\si(f;T''_{\ze''})} \le \hm{I - \si(f;T_{\ze})} + \hm{I_1 -
\si(f;T'_{\ze'})} + \hm{I_2 - \si(f;T''_{\ze''})} <
\frac{\ep}3+\frac{\ep}3+\frac{\ep}3=\ep.} В силу произвольного
выбора $\ep>0$, число, стоящее в левой части (9), равно нулю; то
есть, $I=I_1+I_2$, что равносильно формуле (5).
\end{proof}

\begin{theorem}
\label{thm452} Функция $f(x,y)$, интегрируемая на квадрируемом
компакте $\Kc$, будет интегрируема на любом квадрируемом компакте
$\Kc_1\subset\Kc$.
\end{theorem}

\begin{proof}
Рассмотрим произвольное $\ep>0$ и такое разбиение $T$ компакта $\Kc$
на ячейки $\si_k$, $k=\ol{1,n}$, чтобы $S(f;T)-s(f;T)=\sumkun
\om(f;\si_k)\De\si_k<\ep$. Положим $A = \hc{ l \bvl \si_l \cap
\Kc_1\ne\es}$ и $\si'_l = \si_l\cap\Kc_1$ для всех $l\in A$. Ячейки
$\si'_k$ квадрируемы (как пересечение двух квадрируемых фигур) и
являются компактами (как пересечение компактов). Далее,
$\Kc_1=\Kc_1\cap\Kc = \Kc_1\cap \cupl{k=1}{n}\si_k = \cupl{k=1}n
(\si_k\cap\Kc_1)=\cups{l\in A}\si'_l$. Наконец, так как
$]\si'_{l_1}[ \; \cap \; ]\si'_{l_2}[ \; \subset \; ]\si_{l_1}[ \;
\cap \; ]\si_{l_2}[$, то $]\si'_{l_1}[ \; \cap \; ]\si'_{l_2}[ =
\es$ при $l_1\ne l_2$. Таким образом, $\si'_l$, $l\in A$, образуют
некоторое разбиение $T'$ компакта $\Kc_1$, для которого $S(f;T') -
s(f;T') = \sums{l\in A} \om(f;\si'_l)\De\si'_l$. Поскольку
$0\le\om(f;\si'_l)\le \om(f;\si_l)$ и $0\le \De \si'_l \le \De
\si_l$ и $A\subset\ol{1,n}$, то $S(f;T')-s(f;T') \le
S(f;T)-s(f;T)<\ep$, и следовательно, функция $f$ интегрируема на
$\Kc_1$ (по критерию интегрируемости).
\end{proof}

\begin{imp}
Пусть $\Kc_1 \sco \Kc_n$ --- квадрируемые компакты, попарно без
общих внутренних точек. Функция $f(x,y)$ интегрируема на их
объединении $\Kc$ тогда и только тогда, когда она интегрируема на
каждом $\Kc_j$, $j=\ol{1,n}$, причём тогда \equ{\iints{\Kc}
f(x,y)\,dx\,dy = \suml{j=1}n \iints{\Kc_j} f(x,y)\,dx\,dy.}
\end{imp}

\begin{proof}
Прямое следствие теорем \ref{thm451} и \ref{thm452}.
\end{proof}

\begin{imp}
Если функция $f(x,y)\ge0$ и интегрируема на квадрируемом компакте
$\Kc$, то для любого квадрируемого компакта $\Kc_1\subset\Kc$
справедливо неравенство \equ{\iints{\Kc_1}f(x,y)\,dx\,dy \le
\iints{\Kc} f(x,y)\,dx\,dy.}
\end{imp}

\begin{proof}
Прямое следствие теорем \ref{thm451} и \ref{thm452} и свойства
монотонности двойного интеграла.
\end{proof}

\subsubsection{Интегрируемость произведения интегрируемых функций}

\begin{theorem}
Если функции $f(x,y)$ и $g(x,y)$ интегрируемы на квадрируемом
компакте $\Kc$, то их произведение интегрируемо на $\Kc$.
\end{theorem}

\begin{proof}
Рассмотрим сначала случай, когда $g=f$; то есть докажем, что
интегрируемость $f$ влечёт интегрируемость $f^2$. Так как
$f^2=\hm{f}^2$, а согласно теореме \ref{thm441}, интегрируемость $f$
влечёт интегрируемость $\hm{f}$, то без ограничения общности можно
считать, что $f=\hm{f}$; то есть, что $f(x,y)\ge0$, $(x,y)\in\Kc$.
Обозначим теперь через $M'_E$ и $m'_E$ точные верхнюю и нижнюю грани
на множестве $E\subset\Kc$ для функции $f^2(x,y)$ (а $M_E$ и $m_E$
по--прежнему точные верхняя и нижняя грани для $f(x,y)$). Так как
$0\le m_E\le f(x,y)$ для всех $(x,y)\in E$, то $m^2_E\le f^2(x,y)$
для всех $(x,y)\in E$, и следовательно, $m^2_E\le m'_E$. Аналогично
убеждаемся, что $M'_E\le M_E^2$ (на самом деле здесь, как и выше,
равенство, но для дальнейшего доказательства это не имеет значения).
Поэтому \equ{M'_E-m'_E \le M^2_E-m^2_E = (M_E+m_E)(M_E-m_E) \le
2M(M_E-m_E),} где $M$ --- какая-нибудь верхняя грань для $f(x,y)$ на
$\Kc$ (и $E\subset\Kc$). Отсюда для произвольного разбиения $T$
компакта $\Kc$ на ячейки $\si_1\sco\si_m$ имеем \equ{ S(f^2;T) -
s(f^2;T) = \sumkum (M'_k-m'_k)\De\si_k \le 2M \sumkum(M_k-m_k)\De
\si_k = 2M (S(f;T) - s(f;T)),} и так как в силу доказанного
неравенства левую его часть можно сделать сколь угодно малой, взяв
достаточно малой правую, то вместе с $f$ функция $f^2$ интегрируема
на компакте $\Kc$.

Наконец, для любых $f$ и $g$ справедливо \eqa{8}{fg=\frac14 [(f+g)^2
- (f-g)^2].} Согласно свойству линейности двойного интеграла,
интегрируемость функций $f(x,y)$ и $g(x,y)$ влечёт интегрируемость
функций $f(x,y)\pm g(x,y)$, а значит, по доказанному, и
интегрируемость их квадратов. Утверждение доказываемой теоремы
следует теперь из (8) и свойства линейности двойного интеграла.
\end{proof}

\subsection{Вычисление двойного интеграла повторным интегрированием}

Вычисление двойных интегралов для некоторых областей интегрирования
может быть сведено к двукратному вычислению определённых интегралов.

\subsubsection{Вычисление двойного интеграла по прямоугольнику со
сторонами, параллельными координатным осям}

\begin{theorem}
\label{thm511} Пусть $\Kc$ --- замкнутый прямоугольник, ограниченный
слева и справа прямыми $x=a$ и $x=b$, а снизу и сверху прямыми $y=c$
и $y=d$; то есть, $\Kc=[a,b]\times[c,d]$. Если функция $f(x,y)$
интегрируема на $\Kc$ и для каждого фиксированного $x\in[a,b]$
интегрируема как функция от $y$ на отрезке $[c,d]$, то функция
$g(x)=\intl{c}{d} f(x,y)\,dy$ интегрируема на $[a,b]$ и
\eqa{1}{\iints{\Kc} f(x,y)\,dx\,dy = \intl{a}b \hr{\intl{c}d
f(x,y)\,dy}\,dx.}
\end{theorem}

% рисунок + замечание о геометрическом смысле

\begin{proof}
Пусть $T_1$ и $T_2$ --- разбиения отрезков $[a,b]$ и $[c,d]$ точками
$T_1\cln a=x_0<x_1 < \ldots < x_{m-1}<x_m=b$; $T_2\cln
c=y_0<y_1<\ldots y_{n-1}<y_n=d$ и отрезками $\De_i = [x_{i-1},x_i]$,
$i=\ol{1,m}$; $\De_j = [y_{j-1},y_j]$, $j=\ol{1,n}$. Обозначим
$T_{12}$ разбиения прямоугольника $\Kc$ на прямоугольники
$\si_{ij}=\De_i\times\De_j$, $i=\ol{1,m}$, $j=\ol{1,n}$, и пусть
$m_{ij} = \infl{\si_{ij}} f(x,y)$, $M_{ij} = \supl{\si_{ij}}
f(x,y)$, $m_i = \infl{\De_i}g(x)$, $M_i=\supl{\De_i}g(x)$. Так как
$m_{ij}\le f(x,y)\le M_{ij}$ для всех $(x,y)\in\si_{ij}$, то
\equ{m_{ij} \De y_j \le \intl{y_{j-1}}{y_j} f(x,y)\,dy \le M_{ij}
\De y_j} для всех $x\in[x_{i-1},x_i]$, $i=\ol{1,m}$, где $\De y_j =
\hm{\De_j} = y_j-y_{j-1}$, $j=\ol{1,n}$. Поэтому, \equ{\suml{j=1}n
m_{ij} \De y_j \le \suml{j=1}n \intl{y_{j-1}}{y_j} f(x,y)\,dy =
\intl{y_0}{y_n} f(x,y)\,dy = g(x) \le \suml{j=1}n M_{ij} \De y_j}
для всех $x\in[x_{i-1},x_i]$, $i=\ol{1,m}$, откуда \equ{\suml{j=1}n
m_{ij} \De y_j \le m_i \le M_i \le \suml{j=1}n M_{ij} \De y_j, \;
i=\ol{1,m}.} Умножив эти неравенства почленно на $\De
x_i=\hm{\De_i}=x_i-x_{i-1}$ и просуммировав по $i=\ol{1,m}$, получим
\eqa{2}{s(f;T_{12})=\suml{i=1}m \hr{\suml{j=1}n m_{ij} \De y_j} \De
x_i \le s(g;T_1) \le S(g;T_1) \le \suml{i=1}m \hr{\suml{j=1}n M_{ik}
\De y_j} \De x_i = S(f;T_{12}).}

Так как функция $f(x,y)$ интегрируема на $\Kc$, то, по критерию
интегрируемости, для любого числа $\ep>0$ существует такое число
$\de>0$, что для всех разбиений $T$ прямоугольника $\Kc$ с диаметром
$d(T)<\de$ справедливо неравенство $S(f;T)-s(f;T)<\ep$. Но $\diam
T_{12} = \maxl{i,j} \sqrt{\De x_i^2+\De y_j^2}<\de$, если
$d(T_1)<\frac{\de}{\sqrt2}$ и $d(T_2)<\frac{\de}{\sqrt2}$.
Следовательно, тогда $S(f;T_{12})-s(f;T_{12})<\ep$ и в силу (2) тем
более $S(g;T_1)-s(g;T_1)<\ep$. В силу произвольности $\ep>0$ отсюда
вытекает, что функция $g(x)$ интегрируема на $[a,b]$; причём, в силу
неравенств (2), \equ{s(f;T_{12})\le \intl{a}b g(x)\,dx \le
S(f;T_{12}).}

Так как и \equ{s(f;T_{12}) \le \iints{\Kc} f(x,y)\,dx\,dy \le
S(f;T_{12}),} то \equ{\hm{\iints{\Kc} f(x,y)\,dx\,dy - \intl{a}b
g(x)\,dx} \le S(f;T_{12}) - s(f;T_{12}) < \ep} для всех $\ep>0$,
откуда, в силу произвольности $\ep$, следует равенство (1).
\end{proof}

Совершенно аналогично доказывается следующая теорема.

\begin{theorem}
Пусть $\Kc$ --- замкнутый прямоугольник, ограниченный слева и справа
прямыми $x=a$ и $x=b$, а снизу и сверху --- прямыми $y=c$ и $y=d$;
то есть, $\Kc=[a,b]\times[c,d]$. Если функция $f(x,y)$ интегрируема
на $\Kc$ и для каждого фиксированного $y\in[c,d]$ интегрируема как
функция от $x$ на отрезке $[a,b]$, то функция $h(y) = \intl{a}b
f(x,y)\,dx$ интегрируема на $[c,d]$ и
\eqa{3}{\iints{\Kc}f(x,y)\,dx\,dy = \intl{c}d \hr{\intl{a}b
f(x,y)\,dx}\,dy.}
\end{theorem}

\begin{imp}
Если функция $f(x,y)$ непрерывна на замкнутом прямоугольнике
$\Kc=[a,b]\times[c,d]$, то справедливы обе формулы (1) и (3).
\end{imp}

\begin{proof}
Непрерывная функция $f(x,y)$ интегрируема на прямоугольнике $\Kc$, и
по свойства собственных интегралов, зависящих от параметра, функция
$g(x)=\intl{c}d f(x,y)\,dy$ непрерывна (и значит, интегрируема) на
$[a,b]$, а функция $h(y) = \intl{a}b f(x,y)\,dx$ непрерывна (и
интегрируема) на $[c,d]$.
\end{proof}

\subsubsection{Вычисление двойного интеграла по площади, заключённой
между двумя графиками} \label{152}
\begin{theorem}
\label{thm522} Пусть $\ph_1(x)$ и $\ph_2(x)$ --- непрерывные функции
на отрезке $[a,b]$, причём $\ph_1(x)\le\ph_2(x)$ для всех
$x\in[a,b]$. Тогда $\Kc = \hc{(x,y)\in\R^2 \bvl a\le x\le b, \;
\ph_1(x)\le y\le\ph_2(x)}$ --- квадрируемый компакт, и если функция
$f(x,y)$ интегрируема на $\Kc$, а для каждого фиксированного
$x\in[a,b]$ как функция от $y$ интегрируема на отрезке
$[\ph_1(x),\ph_2(x)]$, то функция $$g(x)=\intl{\ph_1(x)}{\ph_2(x)}
f(x,y)\,dy$$ интегрируема на $[a,b]$ и \eqa{4}{\iints{\Kc}
f(x,y)\,dx\,dy = \intl{a}b \hr{\intl{\ph_1(x)}{\ph_2(x)}
f(x,y)\,dy}\,dx.}
\end{theorem}

%рисунки

\begin{proof}
Множество $\Kc$ квадрируемо (и в частности, ограничено), поскольку
его граница имеет площадь нулю (кривые $\Lc_1$ и $\Lc_2$, задаваемые
уравнениями $y=\ph_1(x)$ и $y=\ph_2(x)$ для $x\in[a,b]$, имеют
нулевые площади как графики интегрируемых функций). Кроме того,
$\Kc$ --- замкнуто. В самом деле, если $(x_n,y_n)\in\Kc$ и
$(x_n,y_n)\ra(x,y)$; то есть, $x_n\ra x$, $y_n\ra y$, то из свойства
$a\le x_n\le b$, $n\in\N$, следует утверждение $a\le x\le b$ и в
силу непрерывности функций $\ph_1$ и $\ph_2$, из свойства
$\ph_1(x_n)\le y_n \le \ph_2(x_n)$ следует утверждение $\ph_1(x)\le
y\le\ph_2(x)$, так что $(x,y)\in\Kc$. Таким образом, $\Kc$ ---
квадрируемый компакт.

Далее, в силу своей непрерывности, $\ph_1$ и $\ph_2$ ограничены на
$[a,b]$. Пусть $c$ и $d$ --- какие--нибудь числа, удовлетворяющие
условиям $c<\ph_1(x)$ и $d>\ph_2(x)$ для всех $x\in[a,b]$. Положим
$\Kc^*=[a,b]\times[c,d]$ и $$f^*(x,y)=\case{f(x,y) \mbox{ на } \Kc,
\\ 0 \mbox{ на } \Kc^*\backslash\Kc.}$$

Тогда $\Kc^*$ и $f^*(x,y)$ удовлетворяют условиям предыдущей теоремы
\ref{thm511}; то есть, $f^*(x,y)$ интегрируема на $\Kc^*$ и
$f^*(x,y)$ при каждом фиксированном $x\in[a,b]$ интегрируема по $y$
на $[c,d]$. Действительно, положим $\Kc'=[\Kc^*\backslash \Kc]$,
$\Kc'$ --- квадрируемый компакт, состоящий из двух кусков:
\equ{\Kc_1'=\hc{(x,y)\in\R^2 \bvl a\le x\le b, \; c\le
y\le\ph_1(x)},\; \Kc_2'=\hc{(x,y)\in\R^2 \bvl a\le x \le b, \;
\ph_2(x)\le y\le d}.}

Функция $f^*(x,y)$ интегрируема на $\Kc$, поскольку совпадает там с
$f(x,y)$, и интегрируема на $\Kc'$, поскольку может отличаться от
нуля там только на графиках функций $\ph_1(x)$ и $\ph_2(x)$, а
объединение этих графиков есть фигура нулевой площади. При этом, в
силу результатов пунктов \ref{124} и \ref{125} и свойства
аддитивности двойного интеграла, \equ{\iints{\Kc_i} f^*(x,y)
\,dx\,dy=0, \; i=1,2, \; \iints{\Kc^*} f^*(x,y)\,dx\,dy =
\iints{\Kc} f^*(x,y)\,dx\,dy + \iints{\Kc'} f^*(x,y)\,dx\,dy =
\iints{\Kc} f(x,y)\,dx\,dy.}

Наконец, для каждого $x\in[a,b]$ имеем
$[c,d]=[c,\ph_1(x)]\cup[\ph_1(x),\ph_2(x)]\cup[\ph_2(x),d]$, причём
$f^*(x,y)$ может быть отличной от нуля на $[c,\ph_1(x)]$ только в
точке $y=\ph_1(x)$ и на $[\ph_2(x),d]$ --- только в точке
$y=\ph_2(x)$, а на отрезке $[\ph_1(x),\ph_2(x)]$, совпадая с
$f(x,y)$, интегрируема как функция от $y$. Следовательно, по
свойствам определённого интеграла, $f^*(x,y)$ для каждого
$x\in[a,b]$ как функция от $y$ интегрируема на $[c,d]$ и
\equ{\intl{c}d f^*(x,y)\,dy = \intl{c}{\ph_1(x)} f^*(x,y)\,dy +
\intl{\ph_1(x)}{\ph_2(x)} f^*(x,y)\,dy + \intl{\ph_2(x)}d f^*(x,y)
\,dy = \intl{\ph_1(x)}{\ph_2(x)} f^*(x,y)\,dy =
\intl{\ph_1(x)}{\ph_2(x)} f(x,y)\,dy.}

Применяя к $f^*(x,y)$ и $\Kc^*$ теорему \ref{thm511}, получаем
\equ{\iints{\Kc} f(x,y)\,dx\,dy = \iints{\Kc^*} f^*(x,y)\,dx\,dy =
\intl{a}b \hr{\intl{c}d f^*(x,y)\,dy}\,dx = \intl{a}b
\hr{\intl{\ph_1(x)}{\ph_2(x)} f(x,y)\,dy}\,dx.}
\end{proof}

Аналогично доказывается следующая теорема.

\begin{theorem}
 Пусть $\psi_1(y)$ и $\psi_2(y)$ --- непрерывные
функции на отрезке $[c,d]$, причём $\psi_1(y) \le \psi_2(y)$ для
всех $y\in[c,d]$. Тогда $\Kc=\hc{(x,y)\in\R^2\bvl c\le y\le d, \;
\psi_1(y)\le x\le \psi_2(y)}$ --- квадрируемый компакт и если
функция $f(x,y)$ интегрируема на $\Kc$ и для каждого фиксированного
$y\in[c,d]$ как функция от $x$ интегрируема на отрезке
$[\psi_1(y),\psi_2(y)]$, то функция $h(y) =
\intl{\psi_1(y)}{\psi_2(y)}f(x,y)\,dx$ интегрируема на $[c,d]$ и
\equ{\iints{\Kc} f(x,y)\,dx\,dy = \intl{c}d
\hr{\intl{\psi_1(y)}{\psi_2(y)}f(x,y)\,dx}\,dy.}
\end{theorem}

\subsection{Тройной интеграл}
\subsubsection{Разбиения кубируемого тела}

Рассмотрим произвольный кубируемый компакт $\Vc\in\R^3\cln Oxyz$.
\emph{Разбиением} $T$ компакта $\Vc$ назовём всякое представление
этого компакта в виде объединения конечного семейства $\hc{\nu_k}$,
$k=\ol{1,n}$, кубируемых компактов, никакие два из которых не имеют
общих внутренних точек; то есть, $\Vc=\cupl{k=1}n \nu_k$, $]\nu_i[
\; \cap \; ]\nu_j[ = \es$, $i\ne j$. Компакты $\nu_k$, $k=\ol{1,n}$
назовём ячейками разбиения $T$; обозначим $об.\,\nu_k=\De \nu_k$,
$k=\ol{1,n}$. Если на каждом компакте $\nu_k$ выбрать некоторую
точку $p_k\in\nu_k$, $p_k=(\xi_k,\eta_k,\ze_k)$, $k=\ol{1,n}$, то
ячейки $\nu_k$, $1\le k\le n$, вместе с набором $p=\hc{p_k}$ точек
$p_k$, $k=\ol{1,n}$, назовём \emph{размеченным разбиением} $T_p$
компакта $\Vc$. Множество всех размеченных разбиений компакта $\Vc$
обозначим символом $\Ps$. Число $d(T_p)=\maxl{1\le k\le
n}(\diam\nu_k)$ называют \emph{диаметром} размеченного разбиения
$T_p$ компакта $\Vc$.

Для произвольного числа $\de>0$ символом $B_{\de}$ обозначим
множество $\hc{T_p\in\Ps \bvl d(T_p)<\de}$. Как и в двумерном
случае, доказывается, что семейство множеств $\hc{B_{\de} \bvl
\de>0}$ образует базу на $\Ps$, которую обозначим $d(T)\ra0$.

\subsubsection{Интегральные суммы и определение тройного интеграла}

Пусть функция $f(x,y,z)$ определена на кубируемом компакте
$\Vc\in\R^3$. Для произвольного размеченного разбиения $T_P$ тела
$\Vc$ с ячейками $\nu_k$, $k=\ol{1,n}$, и набором $p=\hc{p_k}$ точек
$p_k=(\xi_k,\eta_k,\ze_k)\in\nu_k$, $k=\ol{1,n}$, число
\eqa{1}{\si(f;T_p)=\sumkun f(\xi_k,\eta_k,\ze_k)\De \nu_k} называют
\emph{интегральной суммой} функции $f$, отвечающей размеченному
разбиению $T_p$ тела $\Vc$. Рассмотрим отображение (функцию)
$\Ph_f\cln \Ps\ra\R$, задаваемую формулой $\Ph_f(T_p)=\si(f;T_p)$
для всех $T_p\in\Ps$, где число $\si(f;T_p)$ определяется формулой
(1).

\begin{dfn}
1 Число $I=\liml{d(T)\ra0}\Ph_f$ (если предел существует) называют
тройным интегралом функции $f(x,y,z)$ по кубируемому компакту
$\Vc\subset\R^3$ и обозначают $\iiints{\Vc} f(x,y,z)\,d\nu$. Другое
обозначение: $I=\iiints{\Vc} f(x,y,z)\,dx\,dy\,dz$, так что
\eqa{2}{\iiints{\Vc} f(x,y,z)\,d\nu =
\iiints{\Vc}f(x,y,z)\,dx\,dy\,dz = \liml{d(T)\ra0}\Ph_f =
\liml{d(T)\ra0}\si(f;T_p).}
\end{dfn}

\begin{dfn}
2 Функцию $f(x,y,z)$ называют интегрируемой по Риману на кубируемом
компакте $\Vc\in\R^3\cln Oxyz$, если $f$ определена и ограничена на
$\Vc$ и существует тройной интеграл (2).
\end{dfn}

Теория тройного интеграла строится совершенно аналогично теории
двойного интеграла и тройной интеграл обладает всеми аналогичными
свойствами двойного интеграла. В частности,
$\iiints{\Vc}\,dx\,dy\,dz=об.\,\Vc$ для любого кубируемого компакта
$\Vc$ и $\iiints{\Vc} f(x,y,z)\,dx\,dy\,dz = 0$ для любой функции
$f(x,y,z)$, определённой на компакте $\Vc$ нулевого объёма. Тройной
интеграл обладает свойствами линейности, монотонности, аддитивности
и допускает оценку своего модуля. Критерий интегрируемости функции
$f(x,y,z)$ на кубируемом компакте $\Vc$ в терминах её нижних и
верхних сумм Дарбу $s(f;T)=\sumkun m_k\De\nu_k$, $S(f;T)=\sumkun
M_k\De\nu_k$, где $m_k=\infl{\nu_k} f(x,y,z)$,
$M_k=\supl{\nu_k}f(x,y,z)$, $k=\ol{1,n}$, совершенно аналогичны
соответствующим критериям для двойного интеграла.

\subsubsection{Вычисление тройного интеграла по цилиндрическому
телу}

\begin{theorem}
\label{thm631} Пусть $\Kc$ --- квадрируемый компакт на плоскости
$xOy$ и $[c_1,c_2]$ --- отрезок оси $z$. Если функция $f(x,y,z)$
интегрируема на цилиндрическом теле $\Vc=\Kc\times[c_1,c_2]$ и для
каждой фиксированной точки $(x,y)\in\Kc$ как функция от $z$
интегрируема на отрезке $[c_1,c_2]$, то функция
\equ{g(x,y)=\ints{c_1}^{c_2} f(x,y,z)\,dz} интегрируема на $\Kc$ и
\equ{\iiints{\Vc}
f(x,y,z)\,dx\,dy\,dz=\iints{\Kc}g(x,y)\,dx\,dy=\iints{\Kc}
\hr{\ints{c_1}^{c_2} f(x,y,z)\,dz}\,dx\,dy.}
\end{theorem}

\begin{proof}
Пусть $T_1$ --- разбиение компакта $\Kc$ на ячейки $\si_i$,
$i=\ol{1,m}$, и $T_2$ --- разбиение отрезка $[c_1,c_2]$ точками
$z_j$, $j=\ol{0,n}$, так что $c_1=z_0<z_1<\ldots<z_{n-1}<z_n=c_2$.
Положим $\nu_{ij}=\si_i\times[z_{j-1},z_j]$, $i=\ol{1,m}$,
$j=\ol{1,n}$. Доказано (теорема \ref{thm179}), что цилиндрические
тела $\nu_{ij}$ кубируемы и $\De\nu_{ij}=об.\, \nu_{ij} = пл.\,\si_i
\cdot (z_j-z_{j-1})$, а их компактность непосредственно следует из
компактности множеств $\si_i$ и $[z_{j-1},z_j]$. Ясно также, что
тела $\nu_{ij}$ не имеют общих внутренних точек и вместе составляют
$\Vc$. Таким образом, $\nu_{ij}$ --- ячейки некоторого разбиения
$T_{12}$ тела $\Vc$. Пусть, наконец, $m_{ij} = \infl{\nu_{ij}} f$,
$M_{ij} = \supl{\nu_{ij}} f$, $m_i = \infl{\si_i}g$,
$M_i=\supl{\si_i}g$. Так как $m_{ij}\le f(x,y,z)\le M_{ij}$ для всех
$(x,y,z)=\nu_{ij}$, то \equ{m_{ij}\cdot \De z_j \le
\ints{z_{j-1}}^{z_j} f(x,y,z)\,dz \le M_{ij} \cdot \De z_j} для всех
$(x,y)\in\si_i$, $\De z_j = z_j-z_{j-1}$. Поэтому, \equ{\suml{j=1}n
m_{ij} \cdot \De z_j \le \suml{j=1}n \ints{z_{j-1}}^{z_j}
f(x,y,z)\,dz = \ints{c_1}^{c_2} f(x,y,z)\,dz = g(x,y) \le
\suml{j=1}n M_{ij}\cdot\De z_j} для всех $(x,y)\in\si_i$, откуда
следует, что \equ{\suml{j=1}n m_{ij} \cdot \De z_j \le m_i \le M_i
\le \suml{j=1}n M_{ij} \cdot \De z_j, \; i=\ol{1,m}.} Умножив эти
неравенства почленно на $\De\si_i\ge0$ и просуммировав по $i$, $1\le
i\le m$, получим \eqa{3}{S(f;T_{12})=\sumium \suml{j=1}n m_{ij} \De
z_j\De\si_i \le s(g;T_1)\le S(g;T_1)\le \sumium\suml{j=1}n M_{ij}
\De z_j\De\si_i = S(f;T_{12}).}

Так как $f$ интегрируема на $\Vc$, то для произвольного числа
$\ep>0$ существует такое $\de>0$, что для всех разбиений $T$ тела
$\Vc$ с $d(T) <\de$ имеем $0\le S(f;T)-s(f;T)<\ep$. Но
$d(T_{12})=\maxl{i,j}\sqrt{(\diam\si_i)^2+\De z_j^2}<\de$, если
$d(T_1)<\frac{\de}{\sqrt2}$ и $d(T_2)<\frac{\de}{\sqrt2}$.
Следовательно, тогда $S(f;T_{12})-s(f;T_{12})<\ep$, а значит, в силу
(3), тем более $S(g;T_1)-s(g;T_1)<\ep$. Поскольку $\ep$ произвольно,
это означает, что функция $g(x,y)$ интегрируема на квадрируемом
компакте $\Kc$, причём вследствие неравенства (3), \equ{s(f;T_{12})
\le \iints{\Kc} g(x,y)\,dx\,dy\le S(f;T_{12}).}

Так как справедливы также неравенства \equ{s(f;T_{12})\le
\iiints{\Vc} f(x,y,z)\,dx\,dy\,dz \le S(f;T_{12}),} то заключаем,
что $$\hm{\iiints{\Vc}f(x,y,z)\,dx\,dy\,dz - \iints{\Kc}
g(x,y)\,dx\,dy}<\ep$$ для всех $\ep>0$; то есть, что $$\iiints{\Vc}
f(x,y,z)\,dx\,dy\,dz = \iints{\Kc} g(x,y)\,dx\,dy.$$
\end{proof}

\begin{thn}{1.30'}
Пусть $\Kc$ --- квадрируемый компакт на плоскости $xOy$ и
$[c_1,c_2]$ --- отрезок оси $z$. Если функция $f(x,y,z)$
интегрируема на цилиндрическом теле $\Vc=\Kc\times[c_1,c_2]$ и для
каждого фиксированного $z\in[c_1,c_2]$ как функция точки $(x,y)$
интегрируема на $\Kc$, то функция
\equ{h(z)=\iints{\Kc}f(x,y,z)\,dx\,dy} интегрируема на отрезке
$[c_1,c_2]$ и \equ{\iiints{\Vc}f(x,y,z)\,dx\,dy\,dz =
\intl{c_1}{c_2} \hr{\iints{\Kc}f(x,y,z)\,dx\,dy\,dz}\,dz.}
\end{thn}

\begin{proof}
Доказательство совершенно аналогично доказательству теоремы
\ref{thm631}, но теперь \equ{m_j=\infl{[z_{j-1},z_j]}h,\;
M_j=\supl{[z_{j-1},z_j]}h, \; j=\ol{1,n},} остальные обозначения те
же. Так как $m_{ij}\le f(x,y,z) \le M_{ij}$ для всех $(x,y,z)\in
\nu_{ij}$, то \equ{m_{ij}\cdot\De\si_i \le \iints{\si_i}f(x,y,z)
\,dx\,dy \le M_{ij}\cdot\De\si_i} для всех $z\in[z_{j-1},z_j]$,
откуда \equ{\sumium m_{ij}\cdot\De\si_i \le h(z) \le \sumium
M_{ij}\cdot\De\si_i} для всех $z\in[z_{j-1},z_j]$, и следовательно,
\equ{\sumium m_{ij}\cdot\De\si_i\le m_j\le M_j\le \sumium
M_{ij}\cdot\De\si_i, \; j=\ol{1,n}.} Умножая эти неравенства
почленно на $\De z_j$ и суммируя по $j$, получим \equ{s(f;T_{12})\le
s(h;T_2)\le S(h;T_2)\le S(f;T_{12}),} откуда, как и в доказательстве
теоремы \ref{thm631}, следует, что интегрируемость $f$ на $\Vc$
влечёт интегрируемость $h$ на $[c_1,c_2]$ вместе с совпадением их
интегралов.
\end{proof}

Очевидно, непрерывная функция $f(x,y,z)$ на $\Vc$ удовлетворяет
условиям теорем \ref{thm631} и \ref{thm631}'.

\subsubsection{Вычисление тройного интеграла по объёму, заключённому
между двумя графиками}

\begin{theorem}
\label{thm642} Пусть $\psi_1$ и $\psi_2$ --- непрерывные функции на
квадрируемом компакте $\Kc$ в плоскости $xOy$, причём
$\psi_1(x,y)\le\psi_2(x,y)$ для всех $(x,y)\in\Kc$. Тогда тело
\equ{\Vc=\hc{(x,y,z)\in\R^3\bvl (x,y)\in\Kc, \; \psi_1(x,y)\le z\le
\psi_2(x,y)}} --- кубируемый компакт, и если функция $f(x,y,z)$
интегрируема на $\Vc$, а для каждой фиксированной точки
$(x,y)\in\Kc$ как функция от $z$ интегрируема на отрезке
$[\psi_1(x,y),\psi_2(x,y)]$, то функция
\eqa{4}{g(x,y)=\ints{\psi_1(x,y)}^{\psi_2(x,y)} f(x,y,z)\,dz}
интегрируема на $\Vc$ и \eqa{5}{\iiints{\Vc} f(x,y,z)\,dx\,dy\,dz =
\iints{\Kc}g(x,y)\,dx\,dy=\iints{\Kc} \hr{\ints{\psi_1(x,y)}^
{\psi_2(x,y)} f(x,y,z)\,dz}\,dx\,dy.}
\end{theorem}

\begin{proof}
Как непрерывные функции на компакте, $\psi_1$ и $\psi_2$ ограничены
на $\Kc$. Пусть $c$ и $d$ --- какие-нибудь числа, удовлетворяющие
неравенствам $c<\psi_1(x,y)$ и $d>\psi_2(x,y)$ для всех
$(x,y)\in\Kc$. Положим $\Vc^*=\Kc\times[c,d]$. По теореме
\ref{thm179}, $\Vc^*$ --- кубируемое тело, так что его граница имеет
объём нуль. Следовательно, тем же свойством обладает и её часть
\equ{\Fc=\hc{(x,y,z)\in\R^3 \bvl (x,y)\in гр.\,\Kc, \;
\psi_1(x,y)\le z \le\psi_2(x,y)}.} Но $гр.\,\Vc$ есть объединение
множества $\Fc$ и графиков функций $\psi_1$ и $\psi_2$. Так как эти
функции интегрируемы на $\Vc$, то их графики --- тела нулевого
объёма. Поэтому и $гр.\,\Vc$ имеет нулевой объём, а значит, $\Vc$
кубируемо. Кроме того, из непрерывности функций $\psi_1$ и $\psi_2$
и замкнутости $\Kc$ (так же, как в аналогичном случае в
доказательстве теоремы \ref{thm522}) следует, что $\Vc$ замкнуто.
Таким образом, $\Vc$
--- кубируемый компакт. Положив \equ{f^*(x,y,z)=\case{f(x,y,z)
\mbox{ на } \Vc, \\ 0 \mbox{ на } \Vc^*\backslash\Vc}} и
$\Vc'=[\Vc^*\backslash\Vc]$, так что $\Vc'=\Vc_1'\cup\Vc_2'$, где
\equ{\Vc_1'=\hc{(x,y,z)\in\R^3 \bvl (x,y)\in\Kc, \; c\le z \le
\psi_1(x,y)},} \equ{\Vc_2'=\hc{(x,y,z)\in\R^3 \bvl (x,y)\in\Kc, \;
\psi_2(x,y)\le z\le d},} мы рассуждениями, аналогичными проведённым
в доказательстве теоремы \ref{thm522}, убедимся в том, что $f^*$
удовлетворяет условиям теоремы \ref{thm631} (с заменой $f$ на $f^*$
и $\Vc$ на $\Vc^*$), и \equ{\iiints{\Vc}
f(x,y,z)\,dx\,dy\,dz=\iiints{\Vc^*}
f^*(x,y,z)\,dx\,dy\,dz=\iints{\Kc} \hr{\ints{c}^d
f^*(x,y,z)\,dz}\,dx\,dy = \iints{\Kc}
\hr{\ints{\psi_1(x,y)}^{\psi_2(x,y)} f(x,y,z)\,dz}\,dx\,dy.}
\end{proof}

Теорема \ref{thm642} сводит вычисление тройного интеграла к
вычислению сначала простого интеграла, затем двойного.

Как мы увидим, при некоторых условиях значение тройного интеграла
можно найти путём последовательного вычисления трёх простых
интегралов.

Будем обозначать точки $(x,y)\in\R^2$ одной буквой $u$ и
соответственно вместо $(x,y,z)$ писать $(u,z)$.

\begin{lem}
1 Пусть $\psi_1$ и $\psi_2$ --- непрерывные функции на компакте
$\Kc$ в $xOy$, причём $\psi_1(u)\le\psi_2(u)$ для всех $u\in\Kc$.
Если тогда $f$ --- непрерывная функция, заданная на теле
\equ{\Vc=\hc{(u,z)\in\R^3\bvl u\in\Kc,\psi_1(u)\le z\le \psi_2(u)},}
то функция \equ{g(u)=\intl{\psi_1(u)}{\psi_2(u)}f(u,z)\,dz}
(определённая для всех $u\in\Kc$) непрерывна на $\Kc$.
\end{lem}

\begin{proof}
Как и в доказательстве теоремы \ref{thm642} этого пункта, выберем
какие-либо числа $c$ и $d$, удовлетворяющие неравенствам
$c<\psi_1(u)$ и $d>\psi_2(u)$ для всех $u\in\Kc$, и положим
$\Vc^*=\Kc\times[c,d]$, так что $\Vc\subset\Vc^*$. По условию, $f$
--- непрерывная функция на $\Vc$. Продолжим её до непрерывной
функции $F$ на $\Vc^*$. Это можно сделать многими способами;
например, для каждой точки $(u,z)\in\Vc^*$ положим
\equ{F(u,z)=\case{f(u,\psi_1(u)),\mbox{ если } z<\psi_1(u), \\
f(u,z),\mbox{ если } \psi_1(u)\le z\le\psi_2(u) \;(\mbox{то есть,
если }(u,z)\in\Vc^*),\\ f(u,\psi_2(u)),\mbox{ если } z>\psi_2(u).}}
Действительно, как легко проверить, тогда
\equ{F(u,z)=f(u,\min(\max(z,\psi_1(u)),\psi_2(u))),} так что $F$ как
сложная функция, образованная с помощью непрерывных функций
$\psi_1,\psi_2,\max,\min,f$, непрерывна\footnote{Непрерывность
функций $\max$ и $\min$ следует из формул
$\max(f;g)=\frac12[f+g+\hm{f-g}], \;
\min(f;g)=\frac12[f+g-\hm{f-g}$].}. Пусть теперь $u_0$ ---
фиксированная, а $u$ --- переменная точки компакта $\Kc$. Так как
$F$ совпадает с $f$ на $\Vc$, то \eqa{6}{g(u)-g(u_0) =
\intl{\psi_1(u)}{\psi_2(u)} F(u,z)\,dz -
\intl{\psi_1(u_0)}{\psi_2(u_0)}F(u_0,z)\,dz =
\intl{\psi_1(u_0)}{\psi_2(u_0)} [F(u,z)-F(u,z_0)]\,dz +
\intl{\psi_1(u)}{\psi_1(u_0)} F(u,z)\,dz +
\intl{\psi_2(u_0)}{\psi_2(u)}F(u,z)\,dz,} где последние три
интеграла имеют смысл, поскольку пределы интегрирования лежат в
$[c,d]$ и потому точка $(u,z)$ в каждом подинтегральном выражении
принадлежит $\Vc^*$ (хотя может и не принадлежать $\Vc$), а $F$ (в
отличии от $f$) определена и непрерывна на всём $\Vc^*$. Так как при
этом $\Vc^*$ --- компакт, то $F$ равномерно непрерывна на $\Vc^*$.
Поэтому, при заданном $\ep>0$ существует такое $\de_1>0$, что если
$\hn{u-u_0}<\de_1$, то \equ{\hm{F(u,z)-F(u_0,z)} <
\frac{\ep}{3[\psi_2(u_0)-\psi_1(u_0)+\ep]}} для всех
$z\in[\psi_1(u_0),\psi_2(u_0)]$. Кроме того, $F$ ограничена на
$\Vc^*$, $\hm{F(u,z)}\le C$ для всех $(u,z)\in\Vc^*$. Поскольку
$\psi_1$ и $\psi_2$ непрерывны, при том же $\ep$ существует такое
$\de_2>0$, что если $\hn{u-u_0}<\de_2$, то
$\hm{\psi_1(u)-\psi_1(u_0)}<\frac{\ep}{3C}$ и
$\hm{\psi_2(u)-\psi_2(u_0)}<\frac{\ep}{3C}$.

Но тогда из (6) следует, что если $\hn{u-u_0}<\min(\de_1,\de_2)$, то
\ml{\hm{g(u)-g(u_0)}\le\intl{\psi_1(u_0)}{\psi_2(u_0)}
\hm{F(u,z)-F(u_0,z)}\,dz + \hm{\intl{\psi_1(u)}{\psi_1(u_0)}
\hm{F(u,z)}\,dz} + \hm{\intl{\psi_2(u_0)}{\psi_2(u)}
\hm{F(u,z)}\,dz} \le\\\le \frac{\ep [\psi_2(u_0)-\psi_1(u_0)]}{3
[\psi_2(u_0)-\psi_1(u_0)+\ep]} + C\hm{\psi_1(u)-\psi_1(u_0)} +
C\hm{\psi_2(u)-\psi_2(u_0)} < \frac{\ep}3+C\frac{\ep}{3C} +
C\frac{\ep}{3C}=\ep,} так что $g$ непрерывна в (каждой) точке $u_0$
компакта $\Kc$.
\end{proof}

\begin{theorem}
Пусть $\ph_1$ и $\ph_2$ --- непрерывные функции на отрезке $[a,b]$,
причём $\ph_1(x)\le\ph_2(x)$ для всех $x\in[a,b]$, и пусть
$\psi_1,\psi_2$ --- непрерывные функции на множестве
$\Kc=\hc{(x,y)\in\R^2 \bvl a\le x\le b, \; \ph_1(x)\le
y\le\ph_2(x)}$, причём $\psi_1(x,y)\le\psi_2(x,y)$ для всех
$(x,y)\in\Kc$. Тогда тело \equ{\Vc=\hc{(x,y,z)\in\R^3 \bvl a\le x\le
b, \; \ph_1(x)\le y\le\ph_2(x), \; \psi_1(x,y)\le z\le \psi_2(x,y)}}
--- кубируемый компакт и для всякой непрерывной функции $f(x,y,z)$
на $\Vc$ справедлива формула
\eqa{7}{\iiints{\Vc}f(x,y,z)\,dx\,dy\,dz = \intl{a}b
\hr{\intl{\ph_1(x)}{\ph_2(x)}\hr{\intl{\psi_1(x,y)}{\psi_2(x,y)}f(x,y,z)\,dz}\,dy}\,dz.}
\end{theorem}

\begin{proof}
Так как $\Kc$, по теореме \ref{thm522} пункта \ref{152} ---
квадрируемый компакт, то $\Vc$, по теореме \ref{thm642} этого
пункта, --- кубируемый компакт. Будучи непрерывной на $\Vc$, функция
$f$ удовлетворяет условиям теоремы \ref{thm642} и потому справедлива
формула (5). При этом, согласно лемме, функция $g$, определяемая
формулой (4), непрерывна и потому удовлетворяет условиям, наложенным
в теореме \ref{thm522} на функцию $f$. Следовательно, в силу этой
теоремы, \equ{\iints{\Kc}g(x,y)\,dx\,dy=\intl{a}b
\hr{\intl{\ph_1(x)}{\ph_2(x)}g(x,y)\,dy}\,dx,} что в соединении с
формулами (4) и (5) даёт (7).
\end{proof}

\subsection{Замена переменных в двойном и тройном интегралах}

Рассмотрим вначале задачу преобразования двойного интеграла
$\iints{\Kc}F(x,y)\,dx\,dy$ с помощью замены переменных вида
$x=\ph(u,v)$, $y=\psi(u,v)$, где функции $\ph$ и $\psi$ служат
компонентами отображения некоторого открытого подмножества плоскости
с координатами $(u,v)$ в координатную плоскость $xOy$.

\subsubsection{Понятие регулярного отображения}
\begin{dfn}
1 Отображение $f$ из $\R^n$ в $\R^n$, $n\ge2$, определённое на
непустом открытом множестве $\Dc$, $x=f(w)$, $w=(w^1 \sco
w^n)\in\Dc$, $x=(x^1\sco x^n)\in\R^n$, $x^i=f^i(w)=f^i(w^1\sco w^n),
\; i=\ol{1,n}$, называется регулярным, если:
\begin{points}{-2}
\item $f$ непрерывно дифференцируемо в $\Dc$; то есть, каждая
функция $f^i(w)=f^i(w^1\sco w^n), \; i=\ol{1,n}$, имеет непрерывные
всюду в $\Dc$ все частные производные первого порядка;

\item якобиан $\frac{D(f^1\sco f^n)}{D(w^1\sco w^n)}$ отображения
$f$ не равен нулю во всех точках $w\in\Dc$;

\item отображение $f$ инъективно в $\Dc$; то есть, его значения
различны в различных точках $w\in\Dc$ (или, если $f(w_1)=f(w_2)$, то
$w_1=w_2\in\Dc$).
\end{points}
\end{dfn}

Из условий $1^{\circ}$ и $2^{\circ}$ на основании теоремы о
локальном диффеоморфизме следует, что для любой точки $w\in\Dc$
существует такая окрестность $\Uc(w)\subset\Dc$, в которой
непрерывно дифференцируемое отображение $f$ биективно, так что на
открытом множестве $f(\Uc(w))$ существует обратное непрерывно
дифференцируемое отображение $f^{-1}$. Таким образом, отображение
$f$, удовлетворяющее только условиям $1^{\circ}$ и $2^{\circ}$
определения 1, инъективно в $\Uc(w)$; то есть, $f$ --- локально
инъективно в $\Dc$.

\begin{ex}
Отображение $f(r,\ph)$ открытого множества $\Dc_0=\hc{(r,\ph)\in\R^2
\bvl r>0}$, задаваемое компонентами $x=f^1(r,\ph)=r\cos\ph$,
$y=f^2(r,\ph)=r\sin\ph$, имеет в $\Dc_0$ непрерывные производные
$x'_r=(f^1)'_r(r,\ph) = \cos\ph$,
$x'_{\ph}=(f^1)'_{\ph}(r,\ph)=-r\sin\ph$,
$y'_r=(f^2)'_r(r,\ph)=\sin\ph$,
$y'_{\ph}=(f^2)'_{\ph}(r,\ph)=r\cos\ph$ и удовлетворяет в $\Dc_0$
условиям $1^{\circ}$ и $2^{\circ}$ из определения 1, поскольку
\equ{\frac{D(f^1,f^2)}{D(r,\ph)}=\mbmat{\cos\ph && \sin\ph \\
-r \sin\ph && r \cos\ph}=r>0,} но $f$ не удовлетворяет условию
$3^{\circ}$, так как $f(r,\ph)=f(r,\ph+2\pi)$, и следовательно, $f$
не является регулярным отображением в $\Dc_0$.
\end{ex}

Этот пример показывает, что условие $3^{\circ}$ в определении 1 не
следует из его условий $1^{\circ}$ и $2^{\circ}$. С другой стороны,
из условия $3^{\circ}$ следует, что на открытом множестве
$f(\Dc)\subset\R^2$ существует отображение $g\cln f(\Dc)\ra\R^2$,
обратное к $f$, $g=f^{-1}$; то есть, $g\cln f(\Dc)\ra\Dc$.

\subsubsection{Основные свойства регулярных отображений}

\begin{theorem}
\label{thm721} Если отображение $f\cln \Dc\ra\R^n$ непустого
открытого множества $\Dc\subset\R^n$ является регулярным
отображением, то
\begin{points}{-2}
\item образ каждого открытого множества, содержащегося в $\Dc$,
является открытым множеством; в частности, $f(\Dc)$ --- открытое
множество;

\item отображение, обратное к $f$ и определённое на открытом
множестве $f(\Dc)\subset\R^n$ --- также регулярное;

\item образ границы каждого компакта, содержащегося в $\Dc$,
является границей образа этого компакта, а образ внутренности ---
внутренностью.
\end{points}
\end{theorem}

\begin{proof}
Утверждение (1) доказано во втором семестре для любого непрерывно
дифференцируемого отображения.

$2^{\circ}$. Рассмотрим на открытом множестве $f(\Dc)=\Dc_g$
обратное отображение $f^{-1}=g$. В силу теоремы о дифференцируемости
обратного отображения, отображение $g$ непрерывно дифференцируемо в
каждой точке $x\in\Dc_g$. Кроме того, якобиан $\frac{D(g^1\sco
g^n)}{D(x^1\sco x^n)}=\frac1{\frac{D(f^1\sco f^n)}{D(w^1\sco
w^n)}}\ne0$ в каждой точке $x\in\Dc_g$; то есть, выполнены условия
(1) и (2) в определении 1. Далее, если $g(x_1)=g(x_2)$, то $(f\circ
g)(x_1)=(f\circ g)(x_2)$ и $x_1=x_2$; то есть, отображение $g$
инъективно в $\Dc_g$. Итак, выполнены все три условия в определении
1 регулярного отображения.

$3^{\circ}$. Пусть $\Kc$ --- произвольный компакт, содержащийся в
$\Dc$. Так как $f(]\Kc[)\subset f(\Kc)$ и $f(]\Kc[)$ --- открытое
множество (утверждение (1)), то $f(]\Kc[)\subset]f(\Kc)[$. Далее,
так как $f$ непрерывно, то (см. материал второго семестра) $f(\Kc)$
--- компакт (замкнутое и ограниченное множество в $\R^n$). Поэтому,
$гр.\,f(\Kc)\subset f(\Kc)$. Но $\Kc=гр.\,\Kc \cup ]\Kc[$.
Следовательно, \equ{гр.\,f(\Kc)\subset f(\Kc) = f(гр.\,\Kc)\cup
f(]\Kc[)\subset f(гр.\,\Kc)\cup ]f(\Kc)[.}

Поскольку $гр.\,f(\Kc)$ не пересекается с $]f(\Kc)[$, то заключаем,
что \eqa{1}{гр.\,f(\Kc)\subset f(гр.\,\Kc).}

С другой стороны, $\Kc=g(f(\Kc))$. Так как $f(\Kc)$ --- компакт, а
$g$ (по утверждению (2)) регулярно, то по только что доказанному (но
примененному к $f(\Kc)$ вместо $\Kc$ и к $g$ вместо $f$),
$гр.\,\Kc=гр.\,g(f(\Kc))\subset g(гр.\,f(\Kc))$, и поэтому
\eqa{2}{f(гр.\,\Kc)\subset f(g(гр.\,f(\Kc)))=гр.\,f(\Kc).}

Сопоставляя (1) и (2), заключаем, что $f(гр.\,\Kc)=гр.\,f(\Kc)$.
Наконец, так как \equ{f(гр.\,\Kc)\cup f(]\Kc[)=f(гр.\,\Kc \cup
]\Kc[)=f(\Kc)=гр.\,f(\Kc)\cup ]f(\Kc)[,} причём
$f(гр.\,\Kc)=гр.\,f(\Kc)$, $f(]\Kc[)\subset]f(\Kc)[$ и $гр.\,f(\Kc)$
не пересекаются с $]f(\Kc)[$, то $]f(\Kc)[=f(]\Kc[)$.
\end{proof}
\begin{note}
При выполнении условий теоремы \ref{thm721}, образ $f(Q)$ каждой
области $Q\subset\Dc$ --- область. Действительно, по теореме, $f(Q)$
вместе с $Q$ --- открытое множество. Так как при этом $Q$ ещё и
связано, то, в силу непрерывности $f$, $f(Q)$ тоже связано (см.
материал второго семестра). Таким образом, $f(Q)$ --- область. В
частности, если $\Dc$ --- область, то и $f(\Dc)$ --- область.
\end{note}

\subsubsection{Площадь параллелограмма}

На координатной плоскости $uOv$, $w=(u,v)$, с правой системой
координат рассмотрим параллелограмм $P$, образованный упорядоченной
парой векторов $\ol{r}_1=\rbmat{u_1 \\ v_1}$ и $\ol{r}_2=\rbmat{u_2
\\ v_2}$. Из линейной алгебры известно, что $P$ --- квадрируемая
фигура и его площадь \eqa{3}{пл.\,P=\mbmat{ u_1 & u_2 \\ v_1 &
v_2}.}

\subsubsection{Изменение площади параллелограмма при аффинном
отображении} \label{174}

 Аффинное отображение $A$ координатной
плоскости $uOv$ в координатную плоскость $xOy$ задаётся формулами
вида \eqa{4}{\mat{x=a_{11} u + a_{12} v + a_1 \\ y=a_{21} u + a_{22}
v + a_2.}}

Так как $a_{11}=\frac{\pd x}{\pd u}$, $a_{12}=\frac{\pd x}{\pd v}$,
$a_{21}=\frac{\pd y}{\pd u}$, $a_{22}=\frac{\pd y}{\pd v}$, то
матрица Якоби отображения $A$ имеет вид $A=\rbmat{a_{11} & a_{12} \\
a_{21} & a_{22}}$ и якобиан $\frac{D(x,y)}{D(u,v)}$ отображения $A$
совпадает с определителем $\det A$ матрицы $A$. Таким образом,
поскольку $\det A\ne0$, отображение $A$, задаваемое формулами (4),
регулярное.

\begin{theorem}
\label{thm742} Для любого параллелограмма $P$ и любого аффинного
отображения $A$, задаваемого формулами (4), параллелограмм $A(P)$
имеет площадь $пл.\,A(P)=\det A\cdot пл.\,P$.
\end{theorem}

\begin{proof}
Не ограничивая общности, считаем векторы $\ol{r}_1$ и $\ol{r}_2$
отложенными от начала координат (поскольку свойство квадрируемости и
величина площади параллелограмма не зависит от параллельного
переноса в плоскости $uOv$). Тогда при отображении $A$ получаем
векторы $A(\ol{r}_1)$ и $A(\ol{r}_2)$, отложенные от точки
$O(a_1,a_2)$ и координатами $x_i - a_1 = a_{11}u_i + a_{12}v_i$,
$y_i-a_2 = a_{21} u_i + a_{22} v_i$, $i=1,2$. Следовательно,
согласно (3), \equ{пл.\,A(P) = \mbmat{ x_1-a_1 & x_2-a_1 \\ y_1-a_2
& y_2-a_2} = \mbmat{a_{11}u_1 + a_{12}v_1 & a_{11}u_2+a_{12}v_2 \\
a_{21}u_1 + a_{22}v_1 & a_{21}u_2+a_{22}v_2} = \mbmat{a_{11} &
a_{12} \\ a_{21} & a_{22}}\cdot \mbmat{u_1 & u_2 \\ v_1 & v_2} =
\det A\cdot пл.\,P,} где использовано правило умножения
определителей.
\end{proof}

По доказанной теореме, $\hm{пл.\,A(P)}=\hm{\det A}\cdot
\hm{пл.\,P}$. В дальнейшем пары векторов, на которые натянуты
параллелограммы, не будут считаться упорядоченными, и поэтому под
площадью параллелограмма всегда будет пониматься абсолютная
величина.

\subsubsection{Изменение площади квадрируемых фигур при аффинном
отображении} \label{175}
\begin{theorem}
\label{thm753} Если $\Fc$ --- квадрируемая фигура на плоскости
$\Pi\cln uOv$ и $A$ --- аффинное отображение, задаваемое формулами
(4), то образ $A(\Fc)$ --- квадрируемая фигура и
$пл.\,A(\Fc)=\hm{\det A}\cdot пл.\,\Fc$.
\end{theorem}

\begin{proof}
Рассмотрим произвольное число $\ep>0$. Согласно лемме 1, пункт
\ref{114}, на $\Pi$ существует такая сетка с шагом $h$, с помощью
объединения конечного числа элементарных квадратов $q_n$, $n\in\N$,
которой можно получить многоугольные фигуры $P_{\Fc}$ и $Q_{\Fc}$,
чтобы $P_{\Fc}\subset\Fc\subset Q_{\Fc}$ и
$пл.\,Q_{\Fc}-пл.\,P_{\Fc} < \frac{\ep}{\hm{\det A}}$. Образами
квадратов $q_n$, $n\in\N$, при отображении $A$ будут параллелограммы
$A(q_n)$, $n\in\N$, попарно без общих внутренних точек, получающиеся
друг из друга параллельным переносом, и в совокупности накрывающие
$\Pi$. Тогда $A(P_{\Fc})\subset A(\Fc) \subset A(Q_{\Fc})$ и фигуры
$A(P_{\Fc})$, $A(Q_{\Fc})$ квадрируемы, так как они состоят из
конечного числа параллелограммов без общих внутренних точек. Кроме
того, по теореме \ref{thm742} и свойству аддитивности площади,
$пл.\, A(P_{\Fc})=\hm{\det A}пл.\,P_{\Fc}$ и
$пл.\,A(Q_{\Fc})=\hm{\det A} пл.\,Q_{\Fc}$. Поэтом, $пл.\,
A(Q_{\Fc})-пл.\,A(P_{\Fc}) = \hm{\det A} (пл.\, Q_{\Fc} - пл.\,
P_{\Fc})<\hm{\det A} \frac{\ep}{\hm{\det A}}=\ep$. Последнее
означает, что фигура $A(\Fc)$ квадрируема и её площадь
$$пл.\,A(\Fc)=\liml{h\ra0} пл.\,A(Q_{\Fc})=\hm{\det A} \liml{h\ra0}
 пл.\,A(Q_{\Fc})=\hm{\det A} пл.\,\Fc = \iints{\Fc}
\hm{\det A}\,du\,dv.$$
\end{proof}

\subsubsection{Изменение объёма параллелепипеда и кубируемой фигуры
в $\R^n$, $n>1$, при аффинных отображениях}

Рассмотрим в $\R^n$, $n\ge2$, набор из $n$ векторов
$\ol{r}_k=(a_k^1,a_k^2\sco a_k^n)$, $k=\ol{1,n}$. Из линейной
алгебры известно, что объём параллелепипеда $\Ps$, натянутого на
векторы $\ol{r}_k$, $k=\ol{1,n}$, вычисляется по формуле \equ{\De_n
= \mbmat{a_1^1 & \hdotsfor{1} & a_1^n \\ \hdotsfor{3} \\
\hdotsfor{3} \\ a_n^1 & \ldots & a_n^n}.}

Аффинное отображение $A$ пространства $\R^n$ задаётся некоторой
матрицей $A=(a_{ij})$, $i=\ol{1,n}$, $j=\ol{1,n}$, с $\det A\ne0$ и
некоторым вектором $\ol{a}=(a_1\sco a_n)$. Как и в пункте \ref{174},
проверяется, что параллелепипед $A(\Ps)$ кубируем и его объём
вычисляется по формуле \equ{об.\,A(\Ps)=\hm{\det A} об.\,\Ps =
\hm{\frac{D(x^1,x^2\sco x^n)}{D(u^1,u^2\sco u^n)}}\cdot об.\,\Ps.}

Аналогично теореме \ref{thm753}, утверждение справедливо для
произвольного кубируемого тела $\Vc$ в $\R^n$; то есть,
$об.\,A(\Vc)=\hm{\det A} об.\,\Ps$.

\subsubsection{Сохранение квадрируемости (кубируемости) при
регулярных отображениях}

Рассмотрим произвольное регулярное отображение $f\cln D\ra \Pi_1$,
$\Pi_1\cln xOy$, $D\subset\Pi$, $D\ne\es$ --- открытое множество на
$\Pi\cln uOv$, задаваемое компонентами $f(u,v)=(f^1(u,v),f^2(u,v))$,
$x=f^1(u,v)$, $y=f^2(u,v)$, $(u,v)\in D$. Символом $J_f=J_f(u,v)$
обозначим якобиан отображения $f$. По определению,
\equ{J_f(u,v)=\mbmat{\frac{\pd f^1}{\pd u} & \frac{\pd f^1}{\pd v}
\\ \frac{\pd f^2}{\pd u} & \frac{\pd f^2}{\pd v}}.}

Имеет место следующая теорема.

\begin{theorem}
\label{thm774} Для любого квадрируемого компакта $\Kc$,
содержащегося в открытом множестве $D$ его образ $f(\Kc)$ при
регулярном отображении $f$ с областью определения $D_f=D$ ---
квадрируемый компакт и справедлива формула
\eqa{5}{пл.\,f(\Kc)=\iints{\Kc} \hm{J_f(u,v)}\,du\,dv.}
\end{theorem}

В частном случае, когда отображение $f$ аффинное, $f(u,v)=A(u,v)$,
формула (5) доказана в пункте \ref{175}, теорема \ref{thm753}:
\equ{пл.\,A(\Kc)=\hm{\det A}\cdot пл.\,\Kc = \hm{J_A(u,v)}\cdot
пл.\,\Kc = \iints{\Kc} \hm{J_A(u,v)}\,du\,dv.}


Для изложения доказательства в общем случае у меня нет времени:
помимо причин, указанных в самом начале главы, волею календаря у
меня пропадают три лекции, приходящиеся на праздничные дни.

Пусть теперь $x=f(w)$ --- произвольное регулярное отображение
непустого открытого множества $\Dc$ пространства $\R^n$, $n\ge2$,
 в себя, задаваемое компонентами $(f_1(w)\sco f_n(w))$, $w=
(u^1\sco u^n)\in \Dc$, $x^i=f_i(w)=f_i(u^1\sco u^n)$, $i=\ol{1,n}$,
и $\Kc$ --- произвольный кубируемый компакт, содержащийся в $\Dc$.
Как обычно, $\Jc_f(u^1\sco u^n)=\frac{D(f_1\sco f_n)}{D(u^1\sco
u^n)}$
--- якобиан отображения $f$. Аналогом теоремы \ref{thm774} служит
нижеследующая теорема.

\begin{theorem}
Для любого кубируемого компакта $\Kc\subset\Dc$ его образ $f(\Kc)$
--- кубируемый компакт и справедлива формула
\equ{об.\,f(\Kc)=\idotsints{\Kc}\hm{\frac{D(f_1\sco f_n)}{D(u^1\sco
u^n)}}\,du^1\ldots du^n.}
\end{theorem}

\subsubsection{Основная теорема о замене переменных в двойном
интеграле}

\begin{theorem}
Пусть $f(w)=f(u,v)$ --- произвольное регулярное отображение
непустого открытого множества $\Dc$ на координатной плоскости $uOv$
и $\Kc$ --- квадрируемый компакт, содержащийся в открытом множестве
$f(\Dc)$ на координатной плоскости $xOy$. Функция $F(x,y)$
интегрируема на $\Kc$ тогда и только тогда, когда функция $G(u,v) =
(F\circ f)(u,v)\cdot \hm{\Jc_f(u,v)}$ интегрируема на $f^{-1}(\Kc)$
и справедлива формула \eqa{6}{\iints{\Kc}F(x,y)\,dx\,dy =
\iints{f^{-1}(\Kc)}(F\circ f)(u,v)\hm{\Jc_f(u,v)}\,du\,dv.}
\end{theorem}

\begin{proof}
Доказательство проведём для частного случая $f(w)=A(w)=A(u,v)$
аффинного отображения плоскости $uOv$, задаваемого матрицей
$A=(a_{ij})$, $i,j=1,2$, и вектором $\begin{rbmat}{a_1 \\ a_2}
\end{rbmat}$. В этом случае $\hm{\Jc_f(u,v)} = \hm{\det A}>0$ ---
некоторое число и $$\hm{\Jc_{f^{-1}}(x,y)} = \hm{\det A^{-1}} =
\frac1{\hm{\det A}}.$$
\end{proof}

\subsection{Формула Эйлера, связывающая гамма-- и бета--функции}
\subsubsection{}

\eqa{1}{\Ga(s) = \intl0{+\bes}t^{s-1}e^{-t}\,dt =
2\intl0{+\bes}x^{2s-1}e^{-x^2}\,dx=\liml{R\ra+\bes}\intl0R
2x^{2s-1}e^{-x^2}\,dx=\liml{R\ra+\bes}\Ga(s;R), \; s>0,} где
произведена замена переменной $t=x^2$, и \eqa{2}{B(p,q)=\intl01
x^{p-1} (1-x)^{q-1}\,dx = \intl0{\frac{\pi}2} 2\sin^{2p-1}\ph \cdot
\cos^{2q-1}\ph\,d\ph, \; p>0,q>0,} где произведена замена переменной
$x=\sin^2\ph$.

Известно, что $\Ga(s+1)=s\Ga(s)$, $s>0$, а также, что
$B(p,q)=B(q,p)$, что $p>0,q>0$, и справедлива формула приведения
$B(p,q)=\frac{p+q}p B(p+1,q)$, $p>0,q>0$, которая доказывается
интегрированием по частям в несобственном интеграле (2).

\subsubsection{}

На плоскости $\R^2$ с декартовыми координатами $Oxy$ рассмотрим
квадрат $$Q(R) = \hc{(x,y)\in\R^2\bvl 0\le x\le R, \;0\le \;y\le
R},$$ $R>0$, и вписанный в него круговой сектор
$S(R)=\hc{(x,y)\in\R^2\bvl x^2+y^2\le R^2, \;x\ge0,\; y\ge0}$. Тогда
квадрат $Q\hr{\frac{R}2}$ вписан в сектор $S(R)$, так что
$Q\hr{\frac{R}2} \subset S(R)\subset Q(R)$.

\subsubsection{Теорема Эйлера}

$$B(p,q)=\frac{\Ga(p)\Ga(q)}{\Ga(p+q)},\;p>0,\;q>0.$$

\begin{proof}
Доказательство теоремы разобьём на два случая.
\subsubsection{Случай 1: $p\ge1,\;q\ge1$}

Рассмотрим на плоскости $\R^2$ непрерывную функцию
$f(x,y)=4x^{2p-1}y^{2q-1} e^{-x^2}e^{-y^2}$. Поскольку $f(x,y)\ge0$
для всех $(x,y)\in Q(R)$, то согласно свойствам монотонности и
аддитивности двойного интеграла \eqa{3}{\iints{Q(R/2)}
f(x,y)\,dx\,dy \le \iints{S(R)}f(x,y)\,dx\,dy\le
\iints{Q(R)}f(x,y)\,dx\,dy, \; R>0.} Представляя двойной интеграл в
виде повторных и используя обозначения из формулы (1), получим
\eqa{4}{\iints{Q(R)}f(x,y)\,dx\,dy = \intl0R 2x^{2p-1}e^{-x^2}\,dx
\intl0R 2y^{2q-1}e^{-x^2}\,dx = \Ga(p;R)\cdot\Ga(q;R), \; R>0.}
Аналогично,
\eqa{5}{\iints{Q(R/2)}f(x,y)\,dx\,dy=\Ga\hr{p;\frac{R}2}\Ga\hr{q;\frac{R}2}.}

В среднем интеграле формулы (3) переходим к полярным координатам
$x=r\cos\ph$, $y=r\sin\ph$, и сводим его к повторным интегралам,
зная, что модуль якобиана отображения равен $r$. С учётом
обозначений в формулах (1) и (2) и свойства симметричности
$B$--функции, получим \eqa{6}{\iints{S(R)}f(x,y)\,dx\,dy =
\intl0{\pi/2} 2\sin^{2q-1}\ph \cos^{2p-1}\ph\,d\ph \intl0R
2r^{2(p+q)-1} e^{-r^2}\,dr = B(q,p)\cdot \Ga(p+q;R) = B(p,q)\cdot
\Ga(p+q;R).} Подставляя выражения (4), (5) и (6) в неравенства (3),
получим \eqa{7}{\Ga\hr{p;\frac{R}2}\,\Ga\hr{q;\frac{R}2} \le
B(p,q)\,\Ga(p+q;R) \le \Ga(p;R)\,\Ga(q;R),\;R>0.} Переходя в
неравенстве (7) к пределу при $R\ra+\bes$ и используя формулу (1),
получим \equ{\Ga(p)\,\Ga(q)\le B(p,q)\,\Ga(p+q)\le\Ga(p)\,\Ga(q),}
откуда \equ{B(p,q) = \frac{\Ga(p)\,\Ga(q)}{\Ga(p+q)},\;
p\ge1,\,q\ge1.}

\subsubsection{Случай 2: $p>0,\,q>0$ --- произвольные}
Используя формулы приведения для $B$--функции, функциональное
уравнение для $\Ga$--функции и утверждение из предыдущего пункта,
получим \ml{B(p,q) = \frac{p+q}p B(p+1,q)=\frac{p+q}p \,
\frac{p+q+1}q \, B(p+1,q+1) = \frac{p+q}p\frac{p+q+1}q
\frac{\Ga(p+1)\Ga(q+1)}{\Ga(p+q+2)}=\\=\frac{p+q}p\frac{p+q+1}q
\frac{p\,\Ga(p)\,
q\,\Ga(q)}{(p+q+1)(p+q)\Ga(p+q)}=\frac{\Ga(p)\,\Ga(q)}{\Ga(p+q)}, \,
p>0, \, q>0.}

\end{proof}

\subsection{Кратные несобственные интегралы}
\subsubsection{Основные понятия}

Пусть $D$ --- произвольное непустое открытое множество в
пространстве $\R^m$, $m=2,3$.

\begin{dfn}
1 Совокупность (систему) компактов $\hc{\Kc_n}$, $n\in\N$, назовём
исчерпыванием множества $D$, если: \begin{points}{-2} \item $\Kc_n
\subset ] \Kc_{n+1} [ \subset \Kc_{n+1} \subset D$ для всех
$n\in\N$;
\item $\cupl{n=1}{\bes}\Kc_n=D$.
\end{points}
\end{dfn}

Из этого определения, в частности, следует, что для любой точки
$x_0\in D$ существует такой индекс $n\in\N$, что $x_0\in\Kc_n$ и
компакт $\Kc_n$ содержит некоторую окрестность точки $x_0$ (содержит
некоторый шар $\Uc(x_0,\de)$, $\de>0$).

\begin{ex}
Для произвольного открытого множества $D\subset\R^m$, $m=2,3$,
множества $$\Kc_l=\hc{x\in D\bvl d_m(0,x)\le l, \;
d_m(x,гр.\,D)\ge\frac1l},\; l\in\N,$$ являются компактами (как
замкнутые и ограниченные множества в $\R^m$), а система $\hc{\Kc_l;
\; l\in\N}$ образует исчерпывание множества $D$.
\end{ex}

\begin{dfn}
2 Систему компактов $\hc{\Kc_n;\;n\in\N}$ назовём квадрируемым
(кубируемым) исчерпыванием открытого множества $D\subset\R^m$,
$m=2,3$, если эта система исчерпывает $D$ и каждый компакт $\Kc_n$,
$n\in\N$, квадрируем (кубируем).
\end{dfn}

Пусть на открытом множестве $D\subset\R^m$ задана функция
$f(x)=f(x^1\sco x^m)$, интегрируемая на любом квадрируемом
(кубируемом) компакте $\Kc\subset D$, так что существует
$m$--кратный интеграл $\ints{\Kc}f(x)\,dx$ функции $f$ по компакту
$\Kc$.

\begin{dfn}
3 Если для каждого квадрируемого (кубируемого) исчерпывания
$\hc{\Kc_n; \; n\in\N}$ открытого множества $D\subset\R^m$ числовая
последовательность \eqa{1}{a_n = \ints{\Kc_n}f(x)\,dx, \; n\in\N,}
имеет $\liml{n\ra+\bes}a_n=I$ и число $I$ не зависит от выбора
исчерпывания $\hc{\Kc_n;\;n\in\N}$, то $I$ называют несобственным
интегралом функции $f$ на открытом множестве $D$ и обозначают
$I=\ints{D} f(x)\,dx$, где \eqa{2}{\ints{D}f(x)\,dx =
\iints{D}f(x,y)\,dx\,dy \mbox{ или }
\ints{D}f(x)\,dx=\iiints{D}f(x,y,z)\,dx\,dy\,dz.} При этом
несобственный интеграл (2) называют сходящимся.
\end{dfn}

Если для некоторой последовательности вида (1) не существует
предела, или некоторые две последовательности вида (1) имеют
различные пределы, то несобственный интеграл (2) называют
расходящимся.

Из определений 1--3 и линейного свойства предела последовательности
непосредственно следует, что несобственный интеграл (2) обладает
свойством линейности: если существуют несобственные интегралы
$\ints{D}f(x)\,dx$ и $\ints{D}g(x)\,dx$, то для любых чисел
$\la_1,\la_2\in\R$ существует несобственный интеграл $\ints{D}
(\la_1f(x)+\la_2g(x))\,dx$ и \equ{\ints{D}(\la_1f(x)+\la_2g(x))\,dx
= \la_1\ints{D}f(x)\,dx+\la_2\ints{D}g(x)\,dx.}

\subsubsection{Критерий сходимости несобственного интеграла от
неотрицательной функции}

\begin{theorem}\label{thm931}
Если функция $f(x)=f(x^1\sco x^m)$ интегрируема на каждом
квадрируемом (кубируемом) компакте, принадлежащем открытому
множеству $D\subset\R^m$, $m=2,3$, и $f(x)\ge0$ для всех точек $x\in
D$, то несобственный интеграл $\ints{D}f(x)\,dx$ сходится в том и
только в том случае, когда существует хотя бы одно квадрируемое
(кубируемое) исчерпывание $\hc{\Kc_n;\,n\in\N}$ множества $D$, для
которого числовая последовательность (1) ограничена.
\end{theorem}

\begin{proof}
\textbf{Необходимость}. Пусть существует $\ints{D}f(x)\,dx$. Тогда
$\ints{D}f(x)\,dx=\liml{n\ra+\bes}\ints{\Kc_n}f(x)\,dx$ для любого
квадрируемого (кубируемого) исчерпывания $\hc{\Kc_n;\,n\in\N}$
множества $D$, так что числовые последовательности вида (1)
сходящиеся, а следовательно, и ограниченные.

\textbf{Достаточность}. Пусть существует некоторое квадрируемое
(кубируемое) исчерпывание $\hc{\Kc_n;\,n\in\N}$ открытого множества
$D\subset\R^m$, $m=2,3$, для которого последовательность $(a_n)$,
$a_n=\ints{\Kc_n}f(x)\,dx$, $n\in\N$, ограничена. Так как
$\Kc_n\subset\Kc_{n+1}$, $n\in\N$, и $f(x)\ge0$, $x\in D$, то
\equ{0\le a_n=\ints{\Kc_n}f(x)\,dx \le \ints{\Kc_{n+1}}f(x)\,dx =
a_{n+1},} $n\in\N$, так что $(a_n)$ возрастает. Будучи ограниченной,
она имеет $\liml{n\ra+\bes}a_n=I=\sup\hc{a_n\bvl n\in\N}.$

Покажем, что число $I$ не зависит от выбора квадрируемого
(кубируемого) исчерпывания множества $D$. Для этого рассмотрим любое
другое квадрируемое (кубируемое) исчерпывание $\hc{\Kc'_m,
\,m\in\N}$ множества $D$ и числовую последовательность $(b_m)$,
\equ{b_m=\ints{\Kc'_m}f(x)\,dx, \,m\in\N.} По доказанному,
$(b_m)\uparrow$ и $b_m\ge0$, $m\in\N$.

Фиксируем произвольное $m\in\N$ и компакт $\Kc'_m$, и покажем, что
существует $n\in\N$, для которого $\Kc_n\supset\Kc'_m$. Если это не
так, то для любого $k\in\N$ найдётся такая точка $x_k\in\Kc'_m$, что
$x_k\in\Kc_n$. Поскольку все точки последовательности $(x_k)$
принадлежат замкнутому и ограниченному множеству $\Kc'_m$, то
$(x_k)$ содержит некоторую сходящуюся подпоследовательность,
предельная точка $x_0$ которой обязана принадлежать $\Kc'_m$. Не
ограничивая общности, считаем, что сходится сама $(x_k)$ и
$x_0=\liml{k\ra+\bes} x_k$, $x_0\in\Kc'_m$. Согласно замечанию к
определению 1, найдётся компакт $\Kc_{n_0}$, который содержит $x_0$
вместе с некоторой её окрестностью $\Uc(x_0)$. Поэтому можно указать
некоторый индекс $k_0\in\N$, что $x_k\in\Uc(x_0)\subset\Kc_{n_0}$
для всех $k\ge k_0$, что противоречит выбору $(x_k)$.

Итак, $\Kc'_m\subset\Kc_n$ для некоторого $n\in\N$, и $$0\le
b_m=\ints{\Kc'_m}f(x)\,dx\le \ints{\Kc_n}f(x)\,dx=a_n\le
I=\sup\hc{a_n\bvl n\in\N}$$ для любого $m\in\N$. Таким образом,
ограниченная возрастающая подпоследовательность $(b_m)$ имеет
$\liml{m\ra+\bes}b_m=I'\le I$. Меняя местами исчерпывания
$\hc{\Kc'_m;\,m\in\N}$ и $\hc{\Kc_n;\,n\in\N}$, заключаем, что $I\le
I'$, и окончательно, $I'=I$.
\end{proof}

\subsubsection{Общий признак сравнения несобственных интегралов}

\begin{theorem}
Пусть $0\le f(x)\le g(x)$ для всех точек $x\in D\subset\R^m$,
$m=2,3$, и функции $f(x)$ и $g(x)$ интегрируемы на любом
квадрируемом (кубируемом) компакте, содержащемся в открытом
множестве $D$. Тогда из сходимости несобственного интеграла
$\ints{D}g(x)\,dx$ следует сходимость несобственного интеграла
$\ints{D}f(x)\,dx$ (и следовательно, из расходимости несобственного
интеграла $\ints{D}f(x)\,dx$ вытекает расходимость несобственного
интеграла $\ints{D}g(x)\,dx$).
\end{theorem}

\begin{proof}
Рассмотрим произвольное квадрируемое (кубируемое) исчерпывание
$\hc{\Kc_n;\,n\in\N}$ множества $D$. Тогда \eqa{3}{0\le
a_n=\ints{\Kc_n} f(x)\,dx \le \ints{\Kc_n} g(x)\,dx = b_n,\;n\in\N,}
и последовательности $(a_n)$ и $(b_n)$ возрастают.

Если сходится $\ints{D}g(x)\,dx$, то по теореме предыдущего пункта
последовательность $(b_n)$ ограничена сверху, и согласно (3),
ограничена сверху также $(a_n)$. Применяя опять теорему
\ref{thm931}, заключаем, что сходится $\ints{D}f(x)\,dx$.
\end{proof}

\subsubsection{Эталонные интегралы}

Для произвольной точки $x=(x^1\sco x^m)\in\R^m$ обозначим
$\hm{x}=\sqrt{(x^1)^2\spl(x^m)^2}$ и рассмотрим открытые множества
$D_1=\hc{x\in\R^m \bvl \hm{x}>a>0}$ и $D_2=\hc{x\in\R^m \bvl
a+\frac1n \le \hm{x} \le n}$, $n\in\N$, --- компакты,
удовлетворяющие определению 1 для множества $D_1$.

\begin{ex}
Найдём все значения $p\in\R$, для которых сходится несобственный
интеграл $$\ints{D_1}\hm{x}^{-p}\,dx.$$
\end{ex}

\begin{proof}
Рассмотрим случай $m=3$; для точек в $\R^3$ удобнее использовать
запись $(x,y,z)\in\R^3$. В этом случае \equ{D_1=\hc{(x,y,z)\bvl
x^2+y^2+z^2>a^2} \mbox{ и } \Kc_n = \hc{(x,y,z)\bvl \hr{a+\frac1n}^2
\le x^2+y^2+z^2 \le n^2}, \; n\in\N,} так что компакты $\Kc_n$
кубируемы и $\hc{\Kc_n;\,n\in\N}$ --- кубируемое исчерпывание
множества $D_1$. Найдём \equ{a_n = \ints{\Kc_n} \hm{x}^{-p}\,dx =
\iiints{\hr{a+\frac1n}^2 \le x^2+y^2+z^2 \le n^2}
\hr{\sqrt{x^2+y^2+z^2}}^{-p}\,dx\,dy\,dz,\;n\in\N,} переходом к
сферическим координатам $x=r\cos\psi\cos\ph$, $y=r\cos\psi\sin\ph$,
$z=r\sin\psi$, $r\ge0$, $-\frac{\pi}2 \le \psi \le \frac{\pi}2$,
$0\le\ph\le2\pi$. Тогда \equ{a_n = \intl0{2\pi}\,d\ph
\intl{-\frac{\pi}2}{\frac{\pi}2}d\psi \intl{a+\frac1n}n r^{-p}
r^2\cos\psi\,dr = 2\pi \intl{-\frac{\pi}2}
{\frac{\pi}2}\cos\psi\,d\psi \intl{a+\frac1n}n r^{2-p}\,dr =
4\pi\intl{a+\frac1n}n r^{2-p}\,dr, \; n\in\N.}

Поэтому $\liml{n\ra+\bes}a_n=4\pi\intl{a}{+\bes}r^{2-p}\,dr$ и
несобственный интеграл сходится тогда и только тогда, когда
$2-p<-1$, или $p>3$.
\end{proof}

Аналогично проверяется, что несобственный интеграл
$\ints{D_2}\hm{x}^{-p}\,dx$ в случае $m=3$ сходится тогда и только
тогда, когда $p<3$.

\subsubsection{Абсолютная сходимость кратных несобственных
интегралов}

\begin{dfn}
4 Несобственный интеграл $\ints{D}f(x)\,dx$ сходится абсолютно, если
функция $f$ интегрируема на каждом квадрируемом (кубируемом)
компакте $\Kc$ из открытого множества $D$ и сходится несобственный
интеграл $\ints{D}\hm{f(x)}\,dx.$
\end{dfn}

\begin{theorem}
Для несобственных $m$--кратных интегралов $(m\ge2)$ свойства
сходимости и абсолютной сходимости эквивалентны.
\end{theorem}

\begin{proof}
На открытом множестве $D$ рассмотрим неотрицательные функции
\eqa{4}{f_+(x)=\frac{\hm{f(x)}+f(x)}2, \;
f_-(x)=\frac{\hm{f(x)}-f(x)}2.} Тогда \equ{f_+(x)=\case{f(x),\mbox{
если } f(x)\ge0, \\ 0, \mbox{ если } f(x)<0.} \;
f_-(x)=\case{-f(x),\mbox{ если } f(x)\le0, \\ 0,\mbox{ если }
f(x)>0,}} и \eqa{5}{0\le f_+(x)\le\hm{f(x)}, \; 0\le
f_-(x)\le\hm{f(x)}, \; x\in D,} \eqa{6}{f(x)=f_+(x)-f_-(x), \;
\hm{f(x)}=f_+(x)+f_-(x), \; x\in D.}

Так как функции $f(x)$ и $\hm{f(x)}$ интегрируемы на любом
квадрируемом (кубируемом) компакте $\Kc\subset D$, то согласно (4),
каждая из функций $f_+(x)$ и $f_-(x)$ обладает этим же свойством.

Пусть, сначала, несобственный интеграл $\ints{D}f(x)\,dx$ сходится
абсолютно, то есть, сходится интеграл $\ints{D}\hm{f(x)}\,dx$. Тогда
на основании (5) и признака сравнения несобственных интегралов,
убеждаемся в сходимости несобственных интегралов
$\ints{D}f_+(x)\,dx$ и $\ints{D}f_-(x)\,dx$, откуда в силу (6) и
свойства линейности несобственных интегралов следует сходимость
интеграла \equ{\ints{D}f(x)\,dx =
\ints{D}f_+(x)-\ints{D}f_-(x)\,dx.} Обратно, пусть сходится
несобственный интеграл $\ints{D}f(x)\,dx$ и расходится интеграл
$\ints{D}\hm{f(x)}\,dx$. Поскольку $\hm{f(x)}\ge0$, то последнее
предположение, согласно признаку сравнения несобственных кратных
интегралов, означает, что
\equ{\liml{n\ra+\bes}\ints{\Kc_n}\hm{f(x)}\,dx=+\bes} для любого
квадрируемого (кубируемого) исчерпывания $\hr{\Kc_n;\,n\in\N}$
множества $D$. Поэтому, не ограничивая общности, можно считать, что
для квадрируемого (кубируемого) исчерпывания $\hc{\Kc_n\;\,n\in\N}$
множества $D$ справедливо свойство \eqa{7}{\ints{\Kc_{n+1}}
\hm{f(x)}\,dx > 3 \ints{\Kc_n}\hm{f(x)}\,dx + 2n+2, n\in\N.}

Обозначим через $P_n$ замыкание множества $\Kc_{n+1}\backslash
]\Kc_n[$. Тогда $\Kc_{n+1}=\Kc_n\cup P_n$, $]\Kc_n[\;\cap\;
]P_n[=\es$ и, в силу свойства аддитивности кратных интегралов,
неравенство (7) переходит в неравенство
\eqa{8}{\ints{P_n}\hm{f(x)}\,dx>2
\ints{\Kc_n}\hm{f(x)}\,dx+2n+2,\;n\in\N.} Кроме того, на основании
(6), имеем \eqa{9}{\ints{P_n}\hm{f(x)}\,dx = \ints{P_n}f_+(x)\,dx +
\ints{P_n}f_-(x)\,dx.}

Фиксируем $n\in\N$ и предположим, что $$\ints{P_n}
f_+(x)\,dx\ge\ints{P_n}f_-(x)\,dx.$$ Тогда, на основании (8) и (9),
имеем, \eqa{10}{\ints{P_n}f_+(x)|,dx>\ints{\Kc_n}\hm{f(x)}\,dx+n+1.}

Разобьём компакт $P_n$ на конечное число ячеек $P^i_n$,
$i=\ol{1,m}$, чтобы \eqa{11}{0\le\ints{P_n}f_+(x)\,dx - \sumium
m_i\De\si_i\le1,} где $m_i=\infl{P^i_n}f_+(x)$ и
$\De\si_i=пл.\,P^i_n \; (=об.\,P^i_n), \; i=\ol{1,m}$. Объединяя (1)
и (11), получаем \eqa{12}{\sumium m_i\De\si_i > \ints{\Kc_n}
\hm{f(x)}\,dx+n.}

Так как $m_i\ge0$, то оставляем в $\sumium m_i\De\si_i$ лишь те
слагаемые, в которых $m_i>0$. Объединение соответствующих ячеек
$P^i_n$ обозначим $P'_n$. На компакте $P'_n$ функция $f_+(x)>0$, и
значит, $f_+(x)=f(x)>0$, $x\in P'_n$, а неравенство (12) принимает
вид \eqa{13}{\ints{P'_n}f(x)\,dx\ge\sumium m_i\De\si_i>\ints{\Kc_n}
\hm{f(x)}\,dx+n.}

Положим $\Kc^*_n=\Kc_n\cup P'_n$, $]P'_n[ \; \subset \; ]P_n[$ и
$]\Kc_n[ \; \cap \; ]P'_n[ \; \subset \;]\Kc_n[ \;\cap\; ]P_n[ =
\es$. Так как $f(x)\ge-\hm{f(x)}$ и
\equ{\ints{\Kc_n}f(x)\,dx\ge-\ints{\Kc_n}\hm{f(x)}\,dx,} то на
основании (13) и свойства аддитивности кратных интегралов получаем
неравенство \eqa{14}{\ints{\Kc_n^*}f(x)\,dx = \ints{\Kc_n}f(x)\,dx +
\ints{P'_n}f(x)\,dx >
\ints{\Kc_n}f(x)\,dx+\ints{\Kc_n}\hm{f(x)}\,dx+n>n.}

Если предположить, что $\ints{P_n}f_-(x)\,dx\ge\ints{P_n}
f_+(x)\,dx$, то, поскольку $f_-(x)=-f(x)$ для всех $x\in P'_n$,
получаем неравенство \eqa{15}{\ints{\Kc^*_n}f(x)\,dx<-n.} Объединяя
(14) и (15), получаем \eqa{16}{\hm{\ints{\Kc^*_n}f(x)\,dx}>n, \;
n\in\N.}

Поскольку система $\hc{\Kc_n; \,n\in\N}$ образует исчерпывание
открытого множества $D$ и $\Kc^*_n=\Kc_n\cup P'_n$, $]P_n'[\subset
]P_n[$, $P_n=[\Kc_{n+1}\backslash \Kc_n]$, то $]\Kc_n[ \; \cap \;
]P'_n[=\es$, и поэтому $\Kc^*_n\subset ]\Kc^*_{n+1}[$. Кроме того,
$\cupl{n=1}{\bes}\Kc^*_n=D$, так как $\cupl{n=1}{\bes}\Kc_n=D$.
Таким образом, система $\hc{\Kc^*_n; \, n\in\N}$ образует
квадрируемое (кубируемое) исчерпывание множества $D$, и поэтому
неравенство (16) показывает, что несобственный интеграл
$\ints{D}f(x)\,dx$ расходится. Последнее противоречит сделанному
выше предположению, и теорема доказана полностью.
\end{proof}



\newpage
\section{Криволинейные интегралы}
\subsection{Пути в $\R^n$}
\subsubsection{}

Непрерывное отображение $f\cln I\ra\R^n$, $n\ge1$, невырожденного
отрезка $I\subset\R$, $I=[a,b]$, $a<b$, называют дугой Жордана $f$ в
$\R^n$. Тогда $f=(f_1\sco f_n)$ и $f_i=f_i(t)$, $t\in[a,b]$,
$i=\ol{1,n}$.

Пусть $j\cln J\ra I$ --- строго возрастающее отображение отрезка
$J=[c,d]$ на отрезок $[a,b]=I$. Так как $I$ связное множество и
функция $j$ строго возрастает на отрезке $J=[c,d]$, то $j$ ---
непрерывная функция на $[c,d]$ и существует обратная функция
$j^{-1}\cln [a,b]\ra[c,d]$, $j^{-1}$ строго возрастает и непрерывна
на $[a,b]$. Отображение $g=f\circ j$, $g\cln J\ra\R^n$, определяет
дугу Жордана $g$ в $\R^n$, которую называют эквивалентной дуге $f$.
Обозначение: $g\sim f$. Проверим выполнение свойств отношения
эквивалентности:

\begin{nums}{-2}
\item $f\sim f$ (рефлексивность);

\item из $g\sim f$ следует $f\sim g$ (симметричность);

\item из $h\sim g$ и $g\sim f$ следует $h\sim f$ (транзитивность).
\end{nums}

\begin{proof}
\begin{nums}{-2}

\item Рассмотрим $j=i\cln I\ra I$ --- тождественное отображение.
Тогда $i$ строго возрастает на $I$ и $f=f\circ i$; то есть, $f\sim
f$.

\item Если $f\sim g$, то $g=f\circ j$, где функция $j$ строго
возрастает на отрезке $J$ и $j\cln J\ra I$. Тогда обратное
отображение $j^{-1}$ строго возрастает на отрезке $I$, $j^{-1}\cln
I\ra J$ и $f=g\circ j^{-1}$; так что $g\sim f$.

\item Если $h\sim g$ и $g\sim f$, то $h=g\circ k$, $g=f\circ j$, так
что $h=f\circ (j\circ k)$ и $(j\circ k)$ строго возрастает на
некотором отрезке $\Kc=[p,q]$, так что $h\sim f$.
\end{nums}
\end{proof}

Множество эквивалентных дуг Жордана называется непрерывным путём $L$
в $\R^n$. Каждая эквивалентная дуга Жордана $f$, $f\cln I\ra \R^n$,
$f(t)=(f_1(t)\sco f_n(t))$, $t\in I$, называется параметризацией
пути $L$. Если $g\sim f$, то дуга $g$, $g\cln J\ra \R^n$,
$g(\tau)=(g_1(\tau)\sco g_n(\tau))$ является другой параметризацией
пути $L$.

\begin{lem}
1 Каждый непрерывный путь можно параметризовать дугой, параметр
которой пробегает любой заданный невырожденный отрезок на $\R$.
\end{lem}

\begin{proof}
Пусть $f$ --- некоторая параметризация пути $L$, $f\cln
[a,b]\ra\R^n$ и $[c,d]$ --- произвольный невырожденный отрезок на
$\R$. Отображение $t=j(\tau)= a + \frac{\tau-c}{d-c}(b-a)$ называют
аффинной параметризацией; оно строго возрастает, непрерывно на
$[c,d]$ и отображает этот отрезок на отрезок $[a,b]$. Тогда
$g=f\circ j$ непрерывна на $[c,d]$ и так как $a=j(c)$, $b=j(d)$, то
$f(a)=g(c)$ и $f(b)=g(d)$ и $f\sim g$, так что $g$ также будет
параметризацией пути $L$.
\end{proof}

Пусть $f$ и $g$ --- любые две параметризации пути $L$ в $\R^n$;
$f\cln [a,b]\ra\R^n$, $g\cln[c,d]\ra\R^n$. Так как $f\sim g$, то
существует строго возрастающее отображение $j\cln[c,d]\ra[a,b]$, что
$g=f\circ j$ (при этом отображение $j$ непрерывно по теореме о
гомеоморфизме отрезков). Поскольку $a=j(c)$, $b=j(d)$, то
$f(a)=g(c)$ и $f(b)=g(d)$. Точка $f(a)=(f_1(a)\sco f_n(a))\in L$
называется началом пути $L$, точка $f(b)=(f_1(b)\sco f_n(b))$ ---
концом пути $L$. Если $f(a)=f(b)$, путь $L$ называют замкнутым.

\begin{ex}
Для произвольных точек $A=(A^1\sco A^n)$ и $B=(B^1\sco B^n)$ путь
$\Lc=[A,B]$, определяемый параметрическими уравнениями $x^k = (1-t)
A^k + t B^k$, $t\in[0,1]$, $k=\ol{1,n}$, называют прямолинейным
отрезком $[A,B]$ в $\R^n$ с начальной точкой $A$ и конечной точкой
$B$.
\end{ex}

\subsubsection{Противоположные пути}
Обозначим $\si_{u,v}$ --- симметрию отрезка $[u,v]\subset\R$,
относительно его середины $\frac{u+v}2$. Симметрия задаётся формулой
\equ{\si_{u,v}(t)=u+v-t, \; t\in[u,v].}

В частности, $\si_{u,v} \hr{\frac{u+v}2}=\frac{u+v}2$,
$\si_{u,v}(u)=v$ и $\si_{u,v}(v)=u$. Отображение $\si_{u,v}
\downarrow\downarrow$ на $[u,v]$. При этом $\si^2_{u,v} = \si_{u,v}
\circ \si_{u,v} = 1_{u,v}$ --- тождественное отображение, и
следовательно, $\si_{u,v}^{-1} = \si_{u,v}$.

Пусть $f$ и $g$ --- параметризации пути $\Lc$ в $\R^n$; $f\cln
[a,b]\ra\R^n$, $g\cln[c,d]\ra\R^n$ и $g=f\circ j$,
$j\cln[c,d]\ra[a,b]$, $j\uparrow\uparrow$. Тогда
\equ{g\circ\si_{c,d} = f\circ j\circ \si_{c,d} = (f\circ
\si_{a,b})\circ (\si_{a,b}\circ j\circ \si_{c,d}).} Отображение
$\si_{a,b}\circ j\circ\si_{c,d}\uparrow\uparrow$ на $[c,d]$ и
отображает этот отрезок на $[a,b]$. Поэтому $g\circ\si_{c,d} \sim
f\circ\si_{a,b}$; то есть, $f\circ \si_{a,b}$ и $g\circ\si_{c,d}$
--- параметризации одного и того же пути. Назовём этот путь
противоположным пути $L$ и обозначим через $-L$. Началом пути $-L$
служит конец $f(b)=g(d)$ пути $L$, а концом пути $-L$ служит начало
$f(a)=g(c)$ пути $L$. Так как $\si_{a,b}^2=1_{a,b}$, то $-(-L)=L$.

\subsubsection{Простые пути}
Дуга Жордана может иметь точки самопересечения или даже
самоналожения. То же относится и к путям в $\R^n$. Незамкнутый путь
называется простым, если он допускает биективную параметризацию.
Очевидно, что тогда все параметризации простого пути биективны.
Пример: прямолинейный отрезок $[A,B]$ в $\R^n$.

\begin{theorem}
\label{thm131} Множество точек простого незамкнутого пути в $\R^n$
гомеоморфно невырожденному отрезку (на $\R$). При этом каждое
множество точек из $\R^n$, гомеоморфное невырожденному отрезку (на
$\R$) является множеством точек ровно двух (взаимно противоположных)
простых незамкнутых путей.
\end{theorem}

В частности, направленный отрезок из $\R^2$ служит множеством точек
единственного простого незамкнутого пути с тем же началом. Поэтому
мы будем отождествлять график непрерывной функции $g(x)$ на
невырожденном отрезке $[a,b]$, направленный слева направо, с простым
незамкнутым путём $L$, заданным параметрическими уравнениями.
\equ{x=t, \; y=g(t)\; t\in[a,b].}

Замкнутый путь называют простым, если он допускает параметризацию
$h$ с невырожденным отрезком $[a,b]$, обладающую свойством, что если
$h(x)=h(y)$, и $x<y$, то $x=a,\;y=b$. Очевидно, все остальные
параметризации обладают этим свойством.

\begin{ex}
Окружность $\Ga$ на $\R^2$, задаваемая параметрическими уравнениями:
$x=R\cos t$, $y=R\sin t$, $R>0$, $t\in[0,2\pi]$ --- простой
незамкнутый путь в $\R^2$.
\end{ex}

\begin{theorem}
\label{thm132} Множеством точек простого незамкнутого пути на $\R^2$
гомеоморфно окружности. При этом, каждое множество точек плоскости,
гомеоморфное окружности, является множеством точек ровно двух
(взаимно противоположных) простых замкнутых путей с заданным
началом, которым может служить любая точка множества.
\end{theorem}

\subsubsection{Композиция путей}
Пусть конец пути $L$ служит началом пути $M$. Тогда из $L$ и $M$
можно образовать составной путь $N$ следующим образом. Пусть $f$ и
$\ol{g}$ --- какие-либо параметризации путей $L$ и $M$,
соответственно; то есть, $f\cln[a,b]\ra\R^n$, $\ol{g}\cln
[\ol{a},\ol{b}]\ra\R^n$, и $f(b)=\ol{g}(\ol{a})$. Рассмотрим
произвольное $c>b$, и положим $g=\ol{g}\circ j$, где
$j\cln[b,c]\ra[\ol{a},\ol{b}]$ строго возрастающее (например,
аффинное) отображение; то есть,
\equ{j(t)=\ol{a}+\frac{t-b}{c-b}(\ol{b} - \ol{a}), \; t\in[b,c].}

Тогда $g$ является параметризацией пути $M$, $g\cln[b,c]\ra\R^n$, и
$f(b)=g(b)$. Отображения $f$ и $g$ можно простейшим образом
<<склеить>>, положив \eqa{1}{h(t)=\case{f(t),\mbox{ если }
t\in[a,b]; \\ g(t), \mbox{ если } t\in[b,c].}}

Так как $f$ непрерывно слева в $t=b$, а $g$ непрерывна справа в
$t=b$ и $f(b)=g(b)$, то $h(t)$ непрерывно в точке $t=b$. Значит,
$h\cln[a,c]\ra\R^n$ является параметризацией некоторого пути $N$.
Этот же путь $N$ получится при любом другом выборе параметризаций
путей $L$ и $M$. Путь $N$ назовём композицией путей $L$ и $M$ и
будем обозначать $L+M=N$.

Рассмотрим произвольный конечный набор путей $\hc{L_i}$,
$i=\ol{1,m}$, в котором конец пути $L_i$ совпадает с началом пути
$L_{i+1}$ для всех $i=\ol{1,m-1}$. Если уже определён смысл
композиции $m-1$ путей $L_1\sco L_{m-1}$, то под композицией
$L_1\sco L_{m-1},L_m$ понимается путь $(L_1\spl L_{m-1})+L_m$.

Если пути $L_i$, $i=\ol{1,m}$ --- прямолинейные отрезки, то их
композиция называется ломаной линией.

\subsubsection{Пути класса $C^1$}

Путём класса $C^1$ будем называть путь, допускающий непрерывно
дифференцируемую параметризацию. Примером пути класса $C^1$ служит
прямолинейный отрезок.

\begin{theorem}
\label{thm153} Композиция путей класса $C^1$ есть путь класса $C^1$.
\end{theorem}

\begin{proof}
Пусть $L$ и $M$ класса $C^1$ имеют непрерывно дифференцируемые
параметризации $f\cln[a,b]\ra\R^n$,
$\ol{g}\cln[\ol{a},\ol{b}]\ra\R^n$ и $f(b)=\ol{g}(\ol{a})$. Если
$c>b$ и $j\cln[b,c]\ra[\ol{a},\ol{b}]$ --- аффинное отображение, то
$g=\ol{g}\circ j$ --- непрерывно дифференцируемо на $[b,c]$. Однако,
отображение $h$, определяемое формулой (1), может не быть
дифференцируемым в точке $t=b$ (хотя отображение $h(t)=(h^1(t)\sco
h^n(t))$ имеет левую и правую производные в точке $t=b$, но
производные могут не совпадать). Поэтому склеим дуги $f$ и $\ol{g}$
более гладко. Положим \equ{\ol{h}(t)=\case{f\circ j_{a,b}(t), \mbox{
если } t\in[a,b]; \\ g\circ j_{b,c}(t), \mbox{ если } t\in[b,c],}}
где $j_{u,v}(t)=u+\frac{v-u}2 \hr{1-\cos\frac{\pi(t-u)}{v-u}}, \;
t\in[u,v]$. Поскольку $j'_{u,v}(t)=\frac{\pi}2\sin
\frac{\pi(t-u)}{v-u}, \; t\in[u,v]$, и \equ{j'_{u,v}(t)=\case{0,
\mbox{ если } t=u; \\ >0, \mbox{ если } t\in(u,v); \\ 0, \mbox{ если
} t=v,}} то $j_{u,v}$ строго возрастает на $[u,v]$.

Отображения $f\circ j_{a,b}$ и $g\circ j_{a,b}$ являются
параметризациями путей $L$ и $M$. Отображение $\ol{h}(t)$ непрерывно
дифференцируемо на полуинтервалах $[a,b)$ и $(b,c]$ (как композиция
непрерывно дифференцируемых отображений). Пусть
$\ol{h}(t)=(\ol{h}^1(t)\sco \ol{h}^n(t))$, и $$\ol{h}^i(t) =
\case{f^i\circ j_{a,b}(t), \mbox{ если } t\in[a,b]; \\ g^i \circ
j_{b,c}(t), \mbox{ если } t\in[b,c],} i=\ol{1,n}.$$

Так как $j'_{a,b}(b)=j'_{b,c}(b)=0$ и $(\ol{h}^i)'_{лев.}(b) =
(f^i)'_{лев}(b) j'_{a,b}(b)=0$,
$(\ol{h}^i)'_{прав}(b)=(g^i)'_{прав}(b) j'_{b,c}(b)=0$, то
$(\ol{h}^i)'(b)=0$, $i=\ol{1,n}$. Далее, при $t\ra b-0$ имеем
\equ{(\ol{h}^i)'(t) = (f^i)' (j_{a,b}(t)) j'_{a,b}(t) \ra (f^i)'(b)
\cdot 0 = (h^i)'(b), \; i=\ol{1,n},} а при $t\ra b+0$ имеем
\equ{(\ol{h}^i)'(t) = (g^i)'(j_{b,c}(t)) j'_{b,c}(t) \ra
(g^i)'(b)\cdot 0 = (\ol{h}^i)'(b), \; i=\ol{1,n}.}

Поэтому, производные всех функций $(\ol{h}^i)'(t)$, $i=\ol{1,n}$,
непрерывны в точке $t=b$.

Итак, производное отображение $\ol{h}'(t)$ непрерывно на $[a,c]$, и
параметризация $\ol{h}(t)\cln[a,c]\ra\R^n$ задаёт композицию путей
$L+M=N$ класса $C^1$.
\end{proof}

\begin{imp}
Всякая ломаная --- путь класса $C^1$.

Так как симметрия отрезка относительно своей середины является
непрерывно дифференцируемым отображением, то получаем:
\end{imp}

\begin{imp}
Если $L$ есть путь класса $C^1$, то и противоположный ему путь $-L$
будет путём класса $C^1$.
\end{imp}

\subsubsection{Длина пути}

\begin{theorem}
\label{thm164} Все параметризации одного и того же пути имеют
одинаковую длину.
\end{theorem}

\begin{proof}
Пусть $f$ и $g$ --- параметризации пути $L$, так что $g=f\circ j$,
$f\cln [a,b]\ra\R^n$, $g\cln[c,d]\ra\R^n$, $j\cln[c,d]\ra[a,b]$,
отображение $j$ строго возрастает, и $f=(f_1\sco f_n)$, $g=(g_1\sco
g_n)$. Рассмотрим множество $\Ps$ и $\Ps'$ всех разбиений отрезков
$[a,b]$ и $[c,d]$, соответственно. Тогда $j$ порождает биекцию
множеств $\Ps$ и $\Ps'$. Действительно, рассмотрим произвольное
разбиение $T'\in\Ps'$ с точками $c=\tau_0 < \tau_1 < \ldots <
\tau_{m-1} < \tau_m=d$ и положим $t_k = j(\tau_k)$, $k=\ol{0,m}$.
Тогда $a=t_0<t_1<\ldots < t_{m-1}<t_m=b$ есть некоторое разбиение
$T$ отрезка $[a,b]$. Поскольку обратное отображение $j^{-1}$ также
строго возрастает, то заключаем, что каждое разбиение $T$ отрезка
$[a,b]$ порождает некоторое разбиение $T'$ отрезка $[c,d]$. Для
периметров $\si(T',g)$ и $\si(T;f)$ вписанных в $L$ ломаных имеем
\equ{\si(T';g)=\sumkum \sqrt{\sumiun(g_i(\tau_k)-g_i(\tau_{k-1}))^2}
= \sumkum \sqrt{\sumiun (f_i(t_k) - f_i(t_{k-1}))^2} = \si(T;f),}
так что $\sup\hc{\si(T';g) \bvl T'\in\Ps'} = \sup\hc{\si(T;f) \bvl
T\in\Ps}$, и следовательно, длины дуг Жордана $f$ и $g$ совпадают.
\end{proof}

\subsection{Понятие и основные свойства криволинейного интеграла}
\subsubsection{Дифференциальные 1--формы в $\R^3$}

\textbf{Дифференциальной 1--формой} на открытом множестве
$D\subset\R^3$ называется семейство однородных линейных функций на
$\R^3$, зависящее от параметра $u=(x,y,z)$, пробегающего $D$.

Всякая однородная линейная функция на $\R^3$ имеет вид $P\cdot h +
Q\cdot k + R\cdot l$, где $P,Q,R$ --- постоянные и $(h,k,l)\in\R^3$.
Поэтому дифференциальная 1--форма на $D$ задаётся тремя функциями
$P(x,y,z)$, $Q(x,y,z)$, $R(x,y,z)$, определёнными на $D$, и имеет
вид \eqa{1}{\Lc(x,y,z)(h,k,l)=P(x,y,z)\cdot h + Q(x,y,z)\cdot k +
R(x,y,z)\cdot l.}

Если функция $F(x,y,z)$ дифференцируема в $D$, то её дифференциал
$dF(x,y,z)$ есть дифференциальная 1--форма, задаваемая формулой
$$dF(x,y,z)(h,k,l) = F'_x(x,y,z)\cdot h + F'_y(x,y,z)\cdot k +
F'_z(x,y,z)\cdot l.$$

В частности, для координатных функций $F(x,y,z)=x$, $F(x,y,z)=y$,
$F(x,y,z)=z$ имеем соотношения \equ{dx(h,k,l)=h, \; dy(h,k,l)=k, \;
dz(h,k,l)=l,} подставляя которые в (1), получим формулу
\equ{\Lc(x,y,z)(h,k,l)=P(x,y,z)\,dx(h,k,l) + Q(x,y,z)\,dy(h,k,l) +
R(x,y,z)\,dz(h,k,l)} для всех $(h,k,l)\in\R^3$, и значит,
\eqa{2}{\Lc(x,y,z)=P(x,y,z)\,dx+Q(x,y,z)\,dy+R(x,y,z)\,dz.} ---
общий вид дифференциальной 1--формы на открытом множестве в $\R^3$.

Поэтому, в частности,
$$dF(x,y,z)=F'_x(x,y,z)\,dx+F'_y(x,y,z)\,dy+F'_z(x,y,z)\,dz.$$

\subsubsection{Понятие криволинейного интеграла (второго рода)}

Пусть даны:

\begin{nums}{-2}
\item дифференциальная 1--форма $\Lc=\Lc(x,y,z)=P(x,y,z)\,dx +
Q(x,y,z)\,dy + R(x,y,z)\,dz$ на открытом множестве $D\in\R^3$ и
функции $P$, $Q$, $R$ --- непрерывны в $D$;

\item путь $L$ в $D$ с параметризацией $f\cln [a,b]\ra D$,
$D\subset\R^3$, и $f=(f^1,f^2,f^3)$.
\end{nums}

Рассмотрим множество $\Ps$ всех размеченных разбиений отрезка
$[a,b]$ и базу $d(T)\ra0$ на $\Ps$. Пусть $T_{\tau}\in\Ps$ задаётся
точками $a=t_0<t_1<\ldots<t_{n-1}<t_n=b$ и набором $\tau$ точек
$\tau=(\tau_1\sco \tau_n)$, $\tau_k\in[t_{k-1},t_k]$, $k=\ol{1,n}$.
Составим интегральную сумму \eqa{3}{\si_f(\Lc;T_{\tau})=\sumkun
[P(\xi_k,\eta_k,\ze_k)\De x_k + Q(\xi_k,\eta_k,\ze_k)\De y_k +
R(\xi_k,\eta_k,\ze_k)\De z_k],} в которой $\xi_k=f^1(\tau_k)$,
$\eta_k=f^2(\tau_k)$, $\ze_k=f^3(\tau_k)$, $k=\ol{1,n}$, и $\De x_k
= f^1(t_k)-f^1(t_{k-1})$, $\De y_k = f^2(t_k) - f^2(t_{k-1})$, $\De
z_k = f^3(t_k) - f^3(t_{k-1})$, $k=\ol{1,n}$.

\begin{dfn}
1 Если существует $\liml{d(T)\ra0} \si_f(\Lc,T_{\tau})=I$, то число
$I$ называют интегралом дифференциальной 1--формы $\Lc$ вида (2) по
пути $L$ и обозначают \eqa{4}{I=\ints{L}
P(x,y,z)\,dx+Q(x,y,z)\,dy+R(x,y,z)\,dz,} или короче,
\equ{\ints{L}P\,dx+Q\,dy+R\,dz.}
\end{dfn}

Интеграл (4) называют также криволинейным интегралом второго рода от
функций $P,Q,R$ по пути $L$.

\begin{theorem}
Криволинейный интеграл (4) не зависит от выбора параметризации пути
$L$.
\end{theorem}

\begin{proof}
Рассмотрим две параметризации пути $L$ --- $f\cln[a,b]\ra D$ и
$g\cln[c,d]\ra D$. Тогда $g=f\circ j$, где $j\cln[c,d]\ra[a,b]$ ---
аффинная параметризация.

Рассмотрим произвольное размеченное разбиение $T'_{\tau'}$ отрезка
$[c,d]$ точками $c=t'_0<t'_1<\ldots < t'_{n-1} < t'_n=d$ и набором
$\tau'=(\tau'_1 \sco \tau'_n)$, $\tau_k'\in[t'_{k-1},t'_k]$,
$k=\ol{1,n}$. Тогда точки $t_k=j(t'_k)$, $k=\ol{0,n}$, и набор
$\tau=(\tau_1\sco\tau_n)$, $\tau_k=j(\tau'_k)$, $k=\ol{1,n}$,
образуют размеченное разбиение $T_{\tau}$ отрезка $[a,b]$, для
которого интегральная сумма \ml{\si_f(\Lc,T_{\tau}) = \sumkun
\hs{P(\xi_k,\eta_k,\ze_k)\De x_k + Q(\xi_k,\eta_k,\ze_k)\De y_k +
R(\xi_k,\eta_k,\ze_k)\De z_k} = \\=\sumkun
[P(f^1(\tau_k),f^2(\tau_k),f^3(\tau_k))(f^1(t_k)-f^1(t_{k-1})) +
Q(f^1(\tau_k),f^2(\tau_k),f^3(\tau_k))(f^2(t_k)-f^2(t_{k-1})) +\\+
R(f^1(\tau_k),f^2(\tau_k),f^3(\tau_k))(f^3(t_k)-f^3(t_{k-1}))] =
\sumkun
[P(g^1(\tau'_k),g^2(\tau'_k),g^3(\tau'_k))(g^1(t'_k)-g^1(t'_{k-1}))
+\\+
Q(g^1(\tau'_k),g^2(\tau'_k),g^3(\tau'_k))(g^2(t'_k)-g^2(t'_{k-1})) +
R(g^1(\tau'_k),g^2(\tau'_k),g^3(\tau'_k))(g^3(t'_k)-g^3(t'_{k-1}))]
= \si_g(\Lc,T'_{\tau'}).}

Если $\liml{d(T)\ra0} \si_f(\Lc,T_{\tau})=I$, то для произвольного
числа $\ep>0$ существует такое число $\de>0$, что для любого
размеченного разбиения $T_{\tau}$ с $d(T_{\tau})<\de$ справедливо
$\hm{I-\si_f(\Lc;T_{\tau})}<\ep$.

Так как $j\cln[c,d]\ra[a,b]$ непрерывна, то она равномерно
непрерывна на $[c,d]$, и следовательно, для числа $\de>0$ существует
такое число $\eta>0$, то $\hm{j(t')-j(t'')}<\de$ для любых
$t',t''\in[c,d]$, $\hm{t'-t''}<\eta$. Так как $j(t'_k)=t_k$,
$k=\ol{0,n}$, то $\hm{t_k-t_{k-1}}<\de$, если
$\hm{t'_k-t'_{k-1}}<\eta$, $k=\ol{0,n}$. Следовательно, $\hm{I -
\si_g(\Lc;T'_{\tau'})}<\ep$ для любого размеченного разбиения
$T'_{\tau'}$ с $d(T'_{\tau'})<\eta$; то есть, $I=\liml{d(T')\ra0}
\si_g(\Lc;T'_{\tau'})$.
\end{proof}

\begin{theorem}
Дифференциальная форма $P\,dx+Q\,dy+R\,dz$ интегрируема по пути $L$
тогда и только тогда, когда она интегрируема по противоположному
пути $-L$. При этом \eqa{5}{\ints{L}
P\,dx+Q\,dy+R\,dz=-\ints{-L}P\,dx+Q\,dy+R\,dz.}
\end{theorem}

\begin{proof}
Пусть $f$ --- какая--либо параметризация пути $L$ и $f\cln[a,b]\ra
D$. Рассмотрим произвольное размеченное разбиение $T_{\tau}$ отрезка
$[a,b]$ точками $a=t_0<t_1<\ldots<t_{n-1}<t_n=b$ и набором
$\tau=(\tau_1 \sco \tau_n)$, $\tau_k\in[t_{k-1},t_k]$, $k=\ol{1,n}$.
Симметрия $\si_{a,b}$ преобразует размеченное разбиение $T_{\tau}$ в
размеченное разбиение $\ol{T}_{\ol{\tau}}$ отрезка $[a,b]$ точками
$a=\ol{t}_0 < \ol{t}_1 < \ldots < \ol{t}_{n-1} < \ol{t}_n = b$, где
$\ol{t}_{n-k}=\si_{a,b}(t_k)$, $k=\ol{0,n}$, и $\ol{\tau}_{n-k} =
\si_{a,b}(\tau_k)$, $k=\ol{1,n}$.

Отображение $\ol{f}=f\circ\si_{a,b}$ является, по определению,
параметризацией пути $-L$. Для соответствующих интегральных сумм
имеем соотношение \equ{\si_f(\Lc;T_{\tau}) = -\si_{\ol{f}}
(\Lc;\ol{T}_{\ol{\tau}}),} так как $\De x_k = -\De \ol{x}_k$, $\De
y_k = -\De \ol{y}_k$, $\De z_k = -\De \ol{z}_k$, $k=\ol{1,n}$.
Согласно свойству линейности предела функции по базе, существует
\equ{\liml{d(T)\ra0} \si_{\ol{f}}(\Lc;\ol{T}_{\ol{\tau}}) = -
\liml{d(T)\ra0} \si_f(\Lc;T_{\tau}) = -\ints{L} P\,dx+Q\,dy+R\,dz,}
что равносильно (5). Так как $-(-L)=L$, то верно и обратное
утверждение.
\end{proof}

\begin{theorem}
\label{thm223} Дифференциальная форма $\Lc=P\,dx+Q\,dy+R\,dz$,
интегрируемая по пути $L$, интегрируема по каждой части пути $L$.
\end{theorem}

\begin{proof}
Рассмотрим параметризацию $f\cln[a,b]\ra D$ пути $L$ и рассмотрим
произвольную часть $\os{\wedge}L$ пути $L$. Тогда путь
$\os{\wedge}L$ определяется сужением $\os{\wedge}f\cln[c,d]\ra D$
отображения $f$ на $[c,d]\subset[a,b]$.

Допустим, что дифференциальная форма $\Lc=P\,dx+Q\,dy+R\,dz$ не
интегрируема по пути $\os{\wedge}L$; то есть, что не существует
$\ints{\os{\wedge}L}P\,dx+Q\,dy+R\,dz$ как
$\liml{d(T)\ra0}\si_{\os{\wedge}f}
\hr{\Lc;\os{\wedge}T_{\os{\wedge}\tau}}$.

Согласно критерию Коши существования предела по базе, найдётся число
$\ep_0>0$ и для любого числа $\de>0$ существуют размеченные
разбиения $\os{\wedge}T^{i\de}_{\os{\wedge}\tau_i}$, $i=1,2$,
отрезка $[c,d]$ с диаметрами
$d\hr{\os{\wedge}T^{i\de}_{\os{\wedge}\tau_i}}<\de$, $i=1,2$, что
$\hm{\si_{\os{\wedge}f} {\hr{\Lc;
\os{\wedge}T^{1\de}_{\os{\wedge}\tau_1}} - \si_{\os{\wedge}f}
\hr{\Lc; \os{\wedge}T^{2\de}_{\os{\wedge}\tau_2}}}}\ge\ep_0.$

Продолжим размеченные разбиения
$\os{\wedge}T^{i\de}_{\os{\wedge}\tau_i}$, $i=1,2$, отрезка $[c,d]$,
до размеченных разбиений $T^{i\de}_{\tau_i}$, $i=1,2$, отрезка
$[a,b]$ таким образом, чтобы диаметры $d\hr{T^{i\de}_{\tau_i}} =
d\hr{\os{\wedge}T^{i\de}_{\os{\wedge}\tau_i}}<\de$, $i=1,2$, и вне
отрезка $[c,d]$ разбиения
$\hr{\os{\wedge}T^{i\de}_{\os{\wedge}\tau_i}}$, $i=1,2$, имели бы
одинаковые точки разбиения и одинаковые точки в наборах $\tau_i$,
$i=1,2$. Тогда \equ{\hm{\si_f(\Lc;T^{1\de}_{\tau_1}) - \si_f(\Lc;
T_{\tau_2}^{2\de})} = \hm{\si_{\os{\wedge}f}
\hr{\Lc;\os{\wedge}T^{1\de}_{\os{\wedge}\tau_1}} -
\si_{\os{\wedge}f} \hr{\Lc;\os{\wedge}T^{2\de}_{\os{\wedge}\tau_2
}}}\ge\ep_0,} и согласно критерию Коши, не существует
$\liml{d(T)\ra0} \si_f(\Lc;T_{\tau}).$ Последнее противоречит с
условием теоремы и завершает доказательство.
\end{proof}

\begin{theorem}
(Свойство аддитивности). Пусть пути $L_1$ и $L_2$ принадлежат
открытому множеству $D$ в $\R^3$ и конец пути $L_1$ совпадает с
началом пути $L_2$. Если дифференциальная форма $\Lc \equiv P\,dx +
Q\,dy + R\,dz$ интегрируема по каждому пути $L_i$, $i=1,2$, то она
интегрируема по композиции $L=L_1+L_2$ этих путей и \eqa{6}{\ints{L}
P\,dx+Q\,dy+R\,dz = \sums{i=1}^2 \ints{L_i} P\,dx+Q\,dy+R\,dz.}
\end{theorem}

\begin{proof}
По определению, путь $L$ обладает непрерывной параметризацией
$f\cln[a,c]\ra D$ и существует такая точка $b$, $a<b<c$, что
отображение $f_1\cln[a,b]\ra D$ параметризует путь $L_1$, а
отображение $f_2\cln[b,c]\ra D$ параметризует путь $L_2$. По
условию, существует \equ{I_i=\ints{L_i} P\,dx+Q\,dy+R\,dz, \;
i=1,2.}

Рассмотрим произвольное число $\ep>0$. Существуют такие числа
$\de_i>0$, $i=1,2$, что для любого размеченного разбиения
$T^i_{\tau_i}$ отрезка $\De_i$, $i=1,2$, с диаметром
$d\hr{T^i_{\tau_i}}<\de_i$ для соответствующих интегральных сумм
выполняются неравенства \eqa{7}{\hm{\si_i(\Lc;T^i_{\tau_i})-I_i} <
\frac{\ep}3,\; i=1,2.}

Рассмотрим отображение $F\cln D\ra\R^3$ с компонентами $(P,Q,R)$.
Так как отображение $F$ непрерывно в $D$ и непрерывно отображение
$f\cln[a,c]\ra D$, то композиция $F\circ f$ непрерывна на $[a,c]$, а
следовательно ограничена на $[a,c]$, то есть, существует такое число
$C>0$, что норма $\hn{F\circ f(t)}\le C$ для всех $t\in[a,c]$. Кроме
того, отображение $f$, непрерывное на $[a,c]$, равномерно непрерывно
на $[a,c]$; то есть, существует такое число $\de_3>0$, что
$\hn{f(t')-f(t'')} < \frac{\ep}{9C}$ для всех $t',t''\in[a,c]$, для
которых $\hm{t'-t''}<\de_3$.

Положим $\de=\min(\de_1,\de_2,\de_3)$. Рассмотрим произвольное
размеченное разбиение $T_{\tau}$ отрезка $[a,c]$ с
$d(T_{\tau})<\de$. Возможны два случая:

\begin{points}{-2}
\item Точка $b$ --- точка разбиения $T_{\tau}$. Тогда
\equ{\si_f(\Lc;T_{\tau}) = \si_{f_1}(\Lc;T^1_{\tau_1}) + \si_{f_2}
(\Lc;T^2_{\tau_2})} для некоторых размеченных разбиений
$T^i_{\tau_i}$ отрезков $\De_i$ и $d(T^i_{\tau_i})\le d(T_{\tau}) <
\de \le\de_i$, $i=1,2$. Поэтому, в силу (7),
\equ{\hm{\si_f(\Lc;T_{\tau})-(I_1+I_2)} \le
\hm{\si_{f_1}(\Lc;T^1_{\tau_1}) - I_1} + \hm{\si_{f_2}
(\Lc;T^2_{\tau_2})-I_2} < \frac{\ep}3+\frac{\ep}3<\ep.}

\item Существует такое $k$, что $t_{k-1}<b<t_k$. Определим
размеченное разбиение $T^1_{\tau_1}$ отрезка $[a,b]$ с точками
деления $a=t_0<t_1<\ldots <t_{k-1}<t_k=b$. Тогда $d(T^1_{\tau_1})\le
d(T_{\tau})<\de\le\de_1$. Определим размеченное разбиение
$T^2_{\tau_2}$ отрезка $[b,c]$ с точками деления
$c=t_0^2<t_1^2=t_k<\ldots<t^2_{n-k}=t_n=c$. Тогда
$d(T^2_{\tau_2})\le d(T_{\tau}) < \de \le \de_2$. Для
соответствующих интегральных сумм имеем соотношение
\ml{\si_f(\Lc;T_{\tau}) - [\si_{f_1}(\Lc;T^1_{\tau_1}) +
\si_{f_2}(\Lc;T_{\tau_2}^2)] =\\= P(\xi_k,\eta_k,\ze_k)\De x_k +
Q(\xi_k,\eta_k,\ze_k)\De y_k + R(\xi_k,\eta_k,\ze_k)\De z_k -
\sums{i=1}^2 [P(\xi_k^i,\eta_k^i,\ze_k^i)\De x^i_k +
Q(\xi_k^i,\eta_k^i,\ze_k^i)\De y^i_k +
R(\xi_k^i,\eta_k^i,\ze_k^i)\De z^i_k]} и
\eqa{8}{\hm{\si_f(\Lc;T_{\tau}) - \sums{i=1}^2 \si_{f_i}
(\Lc;T_{\tau_i}^i)} \le 3C \cdot \max \hc{\hn{f(t_k) - f(t_{k-1})},
 \hn{f(b)-f(t_{k-1})},  \hn{f(b)-f(t_k)}} < 3C\cdot \frac{\ep}{9C}
= \frac{\ep}3,} так как $d(T^i_{\tau_i})<\de\le\de_i, \; i=1,2.$
\end{points}

Итак, в обоих случаях $\hm{\si_f(\Lc;T_{\tau})-(I_1+I_2)}<\ep$ для
любого размеченного разбиения $T_{\tau}$ отрезка $[a,c]$ с
$d(T_{\tau})<\de$; то есть, $I_1+I_2=\liml{d(T)\ra0}
\si_f(\Lc;T_{\tau}),$ что равносильно (6).
\end{proof}

\subsubsection{Интегралы по координатам}

Если $Q=R=0$ на $D$, то вместо $\ints{L} P\,dx+0\,dy+0\,dz$ пишут
$\ints{L}P\,dx$. Аналогично определяются $\ints{L} Q\,dy$ и
$\ints{L}R\,dz$.

Рассмотрим произвольную параметризацию $f\cln[a,b]\ra D$ пути $L$.
Для произвольного размеченного разбиения $T_{\tau}$ отрезка $[a,b]$
точками деления $a=t_0<t_1<\ldots<t_{n-1}<t_n=b$ и набором
$\tau=(\tau_1\sco\tau_n)$ точек $\tau_k\in[t_{k-1},t_k]$, тогда, по
определению, \equ{\ints{L} P\,dx = \liml{d(T)\ra0} \sumkun
P(\xi_k,\eta_k,\ze_k)\De x_k,} \equ{\ints{L} Q\,dx = \liml{d(T)\ra0}
\sumkun Q(\xi_k,\eta_k,\ze_k)\De y_k,} \equ{\ints{L} R\,dx =
\liml{d(T)\ra0} \sumkun R(\xi_k,\eta_k,\ze_k)\De z_k.}

В силу свойства линейности предела функции по базе, \equ{\ints{L}
P\,dx+Q\,dy+R\,dz = \ints{L} P\,dx + \ints{L} Q\,dy +
\ints{L}R\,dz.}

\subsubsection{Интегралы по замкнутому пути}

Пусть $L$ --- замкнутый путь в $D$ и $f\cln[a,b]\ra D$ ---
какая--либо его параметризация. Рассмотрим произвольную $c$, $a \le
c \le b$. Тогда $L=L_{[a,c]} + L_{[c,b]}$, где $L_{[a,c]}$ --- часть
пути $L$, соответствующая сужению функции $f$ на $[a,c]$; аналогично
определяется $L_{[c,b]}$. Так как $L$ --- замкнутый путь, то
$f(a)=f(b)$. Поэтому существует также путь $L_c = L_{[c,b]} +
L_{[a,c]}$. Если $f(c)\ne f(a)$ (то есть, если $a<c<b$), то пути $L$
и $L_c$ имеют различные начала, и значит, это --- разные пути.
Однако, если дифференциальная форма $\Lc=P\,dx+Q\,dy+R\,dz$
интегрируема по $L$, то она интегрируема и по $L_c$, причём
$$\ints{L} P\,dx+Q\,dy+R\,dz=\ints{L_c} P\,dx+Q\,dy+R\,dz.$$

\begin{proof}
Согласно теореме \ref{thm223}, дифференциальная форма $\Lc$
интегрируема на $L_{[a,c]}$ и на $L_{[c,b]}$, причём, согласно
свойству аддитивности,
$$\ints{L}P\,dx+Q\,dy+R\,dz=\ints{L_{[a,c]}}P\,dx+Q\,dy+R\,dz +
\ints{L_{[c,b]}}P\,dx+Q\,dy+R\,dz.$$ Но тогда, снова по свойству
аддитивности, дифференциальная форма $\Lc$ интегрируема и по пути
$L_c$ с тем же интегралом \equ{\ints{L_c}P\,dx+Q\,dy+R\,dz =
\ints{L_{[c,b]}}
P\,dx+Q\,dy+R\,dz+\ints{L_{[a,c]}}P\,dx+Q\,dy+R\,dz.}
\end{proof}

\subsection{Существование и вычисление криволинейных интегралов}
\subsubsection{Существование криволинейного интеграла второго рода}

Пусть дифференциальная 1--форма
$\Lc=\Lc(x,y,z)=P(x,y,z)\,dx+Q(x,y,z)\,dy+R(x,y,z)\,dz$ определена
на открытом множестве $D$ пространства $\R^3$, путь $L$ лежит в $D$
и имеет параметризацию $f\cln[a,b]\ra D$, $[a,b]\subset\R^1$, и
$f(\ph,\psi,\chi)$.

\begin{theorem}
\label{thm311} Если функции $P,Q,R$ непрерывны в $D$, а путь $L$
непрерывен и спрямляем (то есть, имеет длину), то дифференциальная
форма $P\,dx+Q\,dy+R\,dz$ интегрируема по пути $L$; то есть,
существует $\ints{L}P\,dx+Q\,dy+R\,dz$.
\end{theorem}

\begin{proof}
Согласно условию теоремы и теореме Жордана, функции $\ph,\psi,\chi$
непрерывны на отрезке $[a,b]$ и имеют на нём ограниченное изменение.
Рассмотрим множество $\Ps$ всех размеченных разбиений отрезка
$[a,b]$ и базу $d(T)\ra0$ на $\Ps$. Пусть $T_{\tau}\in\Ps$ задаётся
точками разбиения $a=t_0<t_1<\ldots<t_{n-1}<t_n=b$ и набором $\tau$
точек $\tau=(\tau_1\sco\tau_n)$, $\tau_k=[t_{k-1},t_k]$,
$k=\ol{1,n}$. Интегральную сумму \eqa{1}{\si_f(\Lc;T_{\tau}) =
\sumkun [P(\xi_k,\eta_k,\ze_k)\De x_k + Q(\xi_k,\eta_k,\ze_k)\De y_k
+ R(\xi_k,\eta_k,\ze_k)\De z_k],} в которой $\xi_k=\ph(\tau_k)$,
$\eta_k=\psi(\tau_k)$, $\ze_k=\chi(\tau_k)$, $k=\ol{1,n}$, и $\De
x_k=\ph(t_k)-\ph(t_{k-1})$, $\De y_k = \psi(t_k)-\psi(t_{k-1})$,
$\De z_k= \chi(t_k) - \chi(t_{k-1}))$, $k=\ol{1,n}$, можно
рассматривать как сумму трёх интегральных сумм Стилтьеса для
композиций функций $P\circ f$, $Q\circ f$, и $R\circ f$,
соответственно, по функциям $\ph,\psi,\chi$, отвечающих размеченному
разбиению $T_{\tau}$ отрезка $[a,b]$. Так как функции $P\circ
f,Q\circ f,R\circ f$ непрерывны на $[a,b]$ как композиции
непрерывных функций $P,Q,R$ и непрерывного отображения $f$, то они
интегрируемы по Стилтьесу по функциям ограниченной вариации
$\ph,\psi,\chi$, соответственно. Поэтому, существуют
\eqa{2}{\liml{d(T)\ra0} \sumkun P(\xi_k,\eta_k,\ze_k)\De x_k =
\intl{a}b (P\circ f)(t)d\ph(t),} \eqa{3}{\liml{d(T)\ra0} \sumkun
Q(\xi_k,\eta_k,\ze_k)\De y_k = \intl{a}b (Q\circ f)(t)d\psi(t),}
\eqa{4}{\liml{d(T)\ra0} \sumkun R(\xi_k,\eta_k,\ze_k)\De z_k =
\intl{a}b (R\circ f)(t)d\chi(t),} и согласно (1) и свойству
линейности предела функции по базе, существует $\liml{d(T)\ra0}
\si_f(\Lc;T_{\tau})$, с одной стороны, по определению равный
$\ints{L}P\,dx+Q\,dy+R\,dz$, а, с другой стороны, в силу (2)--(4),
равный сумме \equ{\intl{a}b (P\circ f)(t)d\ph(t) + \intl{a}b (Q\circ
f)(t)d\psi(t) + \intl{a}b (R\circ f)d\chi(t).}

Таким образом, при выполнении условий теоремы \ref{thm311},
\eqa{5}{\ints{L}P\,dx+Q\,dy+R\,dz=\intl{a}b(P\circ f)(t)d\ph(t) +
\intl{a}b(Q\circ f)(t)d\psi(t) + \intl{a}b(R\circ f)(t)d\chi(t).}
\end{proof}

\subsubsection{Вычисление криволинейного интеграла второго рода}

\begin{theorem}
\label{thm322} Пусть функции $P(x,y,z),Q(x,y,z),R(x,y,z)$ непрерывны
на открытом множестве $D$ пространства $\R^3$, путь $L$, лежащий в
$D$, принадлежит классу $\Cc^1$ и $f\cln[a,b]\ra D$ --- непрерывно
дифференцируемая параметризация пути $L$ с компонентами
$f=(\ph,\psi,\chi)$. Тогда дифференциальная форма
$P\,dx+Q\,dy+R\,dz$ интегрируема по $L$ и
\eqa{6}{\ints{L}P\,dx+Q\,dy+R\,dz = \intl{a}b
[P(\ph(t),\psi(t),\chi(t)) \ph'(t) +
Q(\ph(t),\psi(t),\chi(t))\psi'(t) +
R(\ph(t),\psi(t),\chi(t))\chi'(t)]\,dt.}
\end{theorem}

\begin{proof}
Для произвольной параметризации $f\cln[a,b]\ra D$ пути класса
$\Cc^1$ её компоненты $\ph,\psi,\chi$ имеют на $[a,b]$ ограниченное
изменение (в силу ограниченности на $[a,b]$ производных этих функций
и теоремы Лагранжа и конечных приращениях). Поэтому, согласно
теореме \ref{thm311} справедлива формула (5). Более того, интегралы
Стилтьеса в формуле (5) можно вычислить через интегралы Римана (как
это доказано в теории интеграла Стилтьеса) в виде \mla{7}{\intl{a}b
(P\circ f)(t)d\ph(t)+\intl{a}b(Q\circ f)(t)d\psi(t)+\intl{a}b
(R\circ f)(t)d\chi(t) =\\= \intl{a}b
P(\ph(t),\psi(t),\chi(t))\ph'(t)\,dt + \intl{a}b
Q(\ph(t),\psi(t),\chi(t))\psi'(t)\,dt + \intl{a}b
R(\ph(t),\psi(t),\chi(t))\chi'(t)\,dt.}

Формула (6) теперь следует из формул (5) и (7) и свойства линейности
определённого интеграла.
\end{proof}

\subsubsection{Криволинейные интегралы первого рода}

Рассмотрим путь $L$ класса $\Cc^1$, лежащий в открытом множестве $D$
пространства $\R^3$ и пусть $f\cln[a,b]\ra D$ --- его непрерывно
дифференцируемая параметризация с компонентами $f=(\ph,\psi,\chi)$.

Длина $\hm{L}$ пути $L$ вычисляется по формуле \equ{\hm{L}=\intl{a}b
\sqrt{\hm{\ph'(t)}^2 + \hm{\psi'(t)}^2 + \hm{\chi'(t)}^2}\,dt} и не
зависит от выбора параметризации $f$. Более того, спрямляемая любая
часть $L_{[a,\tau]}$ пути $L$, параметризуемая сужением отображения
$f$ на отрезок $[a,\tau]\subset[a,b]$. Таким образом, определена
функция \equ{l(\tau)=\intl{a}{\tau}
\sqrt{\hm{\ph'(t)}^2+\hm{\psi'(t)}^2 + \hm{\chi'(t)}^2}\,dt, \;
\tau\in[a,b],} и $l(\tau)\ge0$, $\tau\in[a,b]$, $l(\tau)\up\up$ на
$[a,b]$ и $l\cln[a,b]\ra[0,\hm{L}]$. Более того, функция $l(\tau)$
непрерывно дифференцируемая на $[a,b]$ и
\eqa{8}{dl(t)=\sqrt{\hm{\ph'(t)}^2+\hm{\psi'(t)}^2+\hm{\chi'(t)}^2}\,dt.}

Обратная функция $t=j(l)$, $l\in[0,\hm{L}]$, непрерывно
дифференцируема и строго возрастает на $[0,\hm{L}]$, и отображает
этот отрезок на $[a,b]$. Поэтому, композиция $g=f\circ j$,
$g\cln[0,\hm{L}]\ra D$, также является непрерывно дифференцируемой
параметризацией исходного пути $L$. Её компоненты $g=(g_1,g_2,g_3)$
имеют вид $g_1(l)=\ph(j(l))$, $g_2(l)=\psi(j(l))$,
$g_3(l)=\chi(j(l))$, $l=[0,\hm{L}]$.

\begin{dfn}
1 Пусть функция $F(x,y,z)$ определена в $D$. Криволинейным
интегралом первого рода $\ints{L}F(x,y,z)\,dt$ от функции $F(x,y,z)$
по кривой $L$ называют определённый интеграл $$\intl{0}{\hm{L}}
F(g_1(l),g_2(l),g_3(l))\,dl$$ (если он существует), то есть, по
определению, \eqa{9}{\ints{L}F(x,y,z)\,dl = \intl{0}{\hm{L}}
F(g_1(l),g_2(l),g_3(l))\,dl.}
\end{dfn}

\begin{theorem}
Если функция $F(x,y,z)$ непрерывна в точках множества пути $L$
класса $\Cc^1$, то \eqa{10}{\ints{L} F(x,y,z)\,dl = \intl{a}b
F(\ph(t),\psi(t),\chi(t)) \sqrt{\hm{\ph'(t)}^2+\hm{\psi'(t)}^2 +
\hm{\chi'(t)}^2}\,dt.}
\end{theorem}

\begin{proof}
Так как функция $F(x,y,z)$ непрерывна на множестве точек пути $L$,
то функция $F(g_1(l),g_2(l),g_3(l))$ непрерывна на отрезке
$[0,\hm{L}]$ и интегрируема на нём, так что существует определённый
интеграл в правой части формулы (9) и, по определению 1, существует
криволинейный интеграл в левой части формулы (9), и справедливо само
равенство (9). Проведя в определённом интеграле в правой части (9)
замену переменной интегрирования $t=j(l)$, получим равенство
\eqa{11}{\intl{0}{\hm{L}} F(g_1(l),g_2(l),g_3(l))\,dl=\intl{a}b
F(\ph(t),\psi(t),\chi(t))dl(t),} в правой части которого стоит
интеграл Стилтьеса для функции $F(\ph(t),\psi(t),\chi(t))$ по
функции $l(t)$. Этот интеграл, с учётом выражения (8), вычисляется
по формуле \eqa{12}{\intl{a}b F(\ph(t),\psi(t),\chi(t))dl(t) =
\intl{a}b F(\ph(t),\psi(t),\chi(t))\sqrt{\hm{\ph'(t)}^2 +
\hm{\psi'(t)}^2 + \hm{\chi'(t)}^2}\,dt.}

Объединяя формулы (9), (11) и (12), получим формулу (10).
\end{proof}

\section{Поверхностные интегралы}
\subsection{Элементы теории поверхностей}
\subsubsection{Векторная функция двух переменных}

Пусть на плоскости $\R^2$ задана декартова прямоугольная система
координат $uOv$ и непустое открытое множество $E\subset \R^2$. Если
каждой точке $M(u,v)\in E$ поставлен в соответствие некоторый вектор
$\ol{r}(M)=\ol{r}(u,v)\in\R^3$, то говорят, что на множестве $E$
задана \textbf{векторная функция двух переменных}.

Рассмотрим базу $\B$ на $E\subset \R^2$. \textbf{Постоянный вектор}
$\ol{a}\in\R^3$ называется \textbf{пределом векторной функции}
$\ol{r}(M)=\ol{r}(u,v)$ по базе $\B$, если для любого числа $\ep>0$
существует такой элемент $B_{\ep}$ базы $\B$, что для всех точек
$M(u,v)\in B_{\ep}$ справедливо неравенство $\hm{\ol{r}(u,v) -
\ol{a}}<\ep$. Обозначение: $\ol{a} = \liml{\B} \ol{r}(u,v) =
\liml{\B} \ol{r}(M)$.

Символом $(u,v)\ra(u_0,v_0)$ (или $M\ra M_0$) обозначим базу
проколотых окрестностей $\hc{\os{\circ}{\Uc}(M_0)}$ точки
$M_0(u_0,v_0)$. Если $\liml{(u,v)\ra(u_0,v_0)} \ol{r}(u,v) =
\ol{r}(u_0,v_0)$, то векторную функцию $\ol{r}(u,v)$ называют
непрерывной в точке $(u_0,v_0)\in E$. Если $M_0(u_0,v_0)$ --- точка
множества $E$, $(u_0,v_0)\in E$, в которой существует $$\liml{\De
u\ra0} \frac{\ol{r}(u_0+\De u,v_0) - \ol{r}(u_0,v_0)}{\De u},$$ то
этот предел (вектор) называют частной производной векторной функции
$\ol{r}(u,v)$ в точке $(u_0,v_0)$ по переменному $u$ и обозначают
$\ol{r}'_u (u_0,v_0)$ или $\frac{\pd}{\pd u} \ol{r}(u_0,v_0)$.
Аналогично, $$\ol{r}'_v = \liml{\De v\ra0} \frac{\ol{r}(u_0,v_0+\De
v)-\ol{r}(u_0,v_0)}{\De v} = \frac{\pd}{\pd v} \ol{r}(u_0,v_0).$$

Подобно скалярному случаю определяются частные производные векторной
функции высших порядков. Если в пространстве $\R^3$ введена
декартова система координат $Oxyz$, то
$\ol{r}(u,v)=(x(u,v),y(u,v),z(u,v))$ и $$\frac{\pd^n
\ol{r}(u,v)}{\pd^k u \,\pd^{n-k} v} = \hr{\frac{\pd^n x(u,v)}{\pd
u^k\, \pd v^{n-k}}, \frac{\pd^n y(u,v)}{\pd u^k \,\pd v^{n-k}},
\frac{\pd^n z(u,v)}{\pd u^k \,\pd v^{n-k}}}.$$

Векторная функция $\ol{r}(u,v)$ называется дифференцируемой в точке
$(u_0,v_0)\in E$, если существуют такие постоянные векторы $\ol{a}$,
$\ol{b}$ (в $\R^3$) и такая векторная функция $\ol{\ep}(\De u,\De
v)$, что $$\De \ol{r} = \ol{r}(u,v) - \ol{r}(u_0,v_0) = \ol{a} \cdot
\De u + \ol{b} \cdot \De v + \ol{\ep}(\De u,\De v)\cdot\rho,$$ где
$\De u=u-u_0$, $\De v=v-v_0$, $\rho=\sqrt{(\De u)^2 + (\De v)^2}$ и
$\liml{(\De u,\De v)\ra(0,0)} \ol{\ep}(\De u,\De v)=\ol{0} =
\liml{\rho\ra0}{\ol{\ep}}(\De u,\De v) = \ol{\ep}(0,0),$ так что
векторная функция $\ol{\ep}(\De u,\De v)$ непрерывна в точке
$(0,0)$.

Аналогично скалярному случаю убеждаемся, что $\ol{a} =
\ol{r}_u(u_0,v_0)$, $\ol{b} = \ol{r}_v(u_0,v_0)$ и таким образом
$$\De \ol{r} = \frac{\pd \ol{r}}{\pd u}\,du + \frac{\pd \ol{r}}{\pd
v}\,dv + \ol{\ep}\cdot\rho,$$ где $du=\De u$, $dv=\De v$, и
$\liml{\rho\ra0}\ol{\ep}=\ol{0}.$

Векторная функция $$d\ol{r} = \frac{\pd \ol{r}}{\pd u}\,du +
\frac{\pd \ol{r}}{\pd v}\,dv,$$ линейная относительно $du$ и $dv$,
называется дифференциалом векторной функции $\ol{r}(u,v)$. По
определению, $$\De \ol{r} = d\ol{r} + \ol{\ep}\cdot\rho, \;
\liml{\rho\ra0}{\ol{\ep}}=\ol{0}.$$

Подобно скалярному случаю, для векторной функции $\ol{r}(u,v)$
определяются дифференциалы высших порядков $d^k\ol{r}$, $k\in\N$, и
имеет место формула Тейлора: $$\ol{r}(u,v) = \sumkzn \frac1{k!} d^k
\ol{r}(u_0,v_0) + \ol{0}(\rho^n), \; \rho\ra0,$$ где $d^k
\ol{r}(u_0,v_0)=\hr{du\cdot \frac{\pd}{\pd u} + dv\cdot
\frac{\pd}{\pd v}}^k \ol{r}(u_0,v_0)$, $k=\ol{0,n}$, $d^0
\ol{r}(u_0,v_0) = \ol{r}(u_0,v_0)$ и $\ol{0}(\rho^n)=\ol{\ep}(\De
u,\De v)\cdot \rho^n$, $\liml{\rho\ra0}\ol{\ep}=0$.


\subsubsection{Поверхность в $\R^3$}

Рассмотрим на $\R^2$ произвольную ограниченную область $D$ (т.е.,
связное открытое множество в $\R^2$) и её замыкание $[D]$.

\begin{dfn}
1 Непрерывное отображение $f\cln [D]\ra\R^3$ называют элементарной
поверхностью в $\R^3$. Обозначаем $f(M)\subset\R^3$, $M\in[D]$ и
образ $f([D])$ называют носителем этой поверхности при
рассматриваемом отображении.
\end{dfn}

Точка носителя поверхности, являющаяся образом по крайней мере двух
различных точек замкнутой области $[D]$, называется кратной точкой
поверхности (или её точкой самопересечения).

Если элементарная поверхность  не имеет кратных точек; то есть,
отображение $f$ биективное, то элементарная поверхность называется
простой.

Если на $\R^2$ введена декартова система координат $(u,v)$, то, по
определению, отображение $f\cln[D]\ra\R^3$ определяет некоторую
векторную функцию $\ol{r}(u,v)$, $(u,v)\in[D]$, графиком которой
служит введённая элементарная поверхность.

Переменные $(u,v)$ называются координатам поверхности или её
параметрами.

\begin{dfn}
2 Элементарная поверхность $f(M)\in\R^3$, $M\in[D]\subset \R^2$,
называется эквивалентной элементарной поверхности $f_1(M)\in\R^3$,
$M\in[D_1]\subset\R^2$, если существует квазирегулярное отображение
$j\cln[D_1]\ra[D]$, для которого $f_1(M)=(f\circ j)(M)$,
$M\in[D_1]$. Обозначение: $f_1\sim f$.
\end{dfn}

Согласно этому определению, поскольку существует обратное
квазирегулярное отображение $j^{-1}\cln[D]\ra[D_1]$, отношение
$f_1\sim f$ действительно является отношением эквивалентности.

\begin{dfn}
3 Класс эквивалентных элементарных поверхностей называют
(непрерывной) поверхностью $\Phi$ в $\R^3$, а любое отображение
$f(M)\in\R^3$, $M\in[D]\subset\R^2$, её определяющее, называют
параметризацией поверхности $\Ph$. Векторная функция
$\ol{r}(u,v)=f(M)$ называется векторной параметризацией поверхности
$\Ph$. Любое квазирегулярное отображение $j\cln[D_1]\ra[D]$ называют
допустимым преобразованием параметров плокости. Если $j=(\ph,\psi)$,
то $u=\ph(u_1,v_1)$, $v=\psi(u_1,v_1)$.
\end{dfn}

\begin{dfn}
4 Общий носитель всех элементарных поверхностей, составляющих данную
поверхность $\Ph$, называют носителем поверхности $\Ph$.
\end{dfn}

\begin{dfn}
5 Точка $(M,f(M))$ элементарной поверхности $f\cln[D]\ra\R^3$
называется эквивалентной точке $(M_1,f_1(M_1))$ элементарной
поверхности $f_1\cln[D_1]\ra\R^3$, если $M_1=j(M)$.
\end{dfn}

\begin{dfn}
6 Класс эквивалентных между собой точек элементарных поверхностей
называется точкой поверхности $\Ph$, а их общий носитель называется
носителем этой точки поверхности $\Ph$.
\end{dfn}

Если $M\in ]D[$, то точка $(M,f(M))$ называется внутренней точкой
элементарной поверхности, если $M\in \pd D$, то точка $(M,f(M))$
называется краевой точкой элементарной поверхности $\Ph$. Так как
при квазирегулярных отображениях внутренние точки области  переходят
о внутренние точки, а граничные точки области --- в граничные точки,
то все эквивалентные точки для точки, внутренней для некоторой
элементарной поверхности, будут внутренними точками эквивалентных
элементарных поверхностей, и порождают внутреннюю точку поверхности
$\Ph$. Аналогичное утверждение справедливо для краевых точек
поверхности $\Ph$. Множество всех краевых точек поверхности $\Ph$
называют её краем.

Совокупность всех носителей точек поверхности составляет носитель
поверхности. Точка носителя поверхности, являющаяся носителем по
крайней мере двух различных точек поверхности, называется кратной
точкой или точкой самопересечения поверхности. Если поверхность не
имеет кратных точек, то она называется простой поверхностью.

\begin{dfn}
7 Поверхность $\Ph$ принадлежит классу $\Cc^1$, если она имеет хотя
бы одну параметризацию класса $\Cc^1$.
\end{dfn}

\begin{ex}
Если $\Ph$ --- график непрерывной дифференцируемой функции
$z=f(x,y)$, $(x,y)\in[D]\subset\R^2$, то $\Ph$ --- простая
поверхность класса $\Cc^1$, задаваемая параметризацией $\ol{r}(u,v)
= (x(u,v),y(u,v),z(u,v))$, $x=x(u,v)=u$, $y=y(u,v)=v$,
$z=z(u,v)=f(u,v)$, $(u,v)\in[D]\subset\R^2$.
\end{ex}

\begin{dfn}
8 Если поверхность $\Ph$ задана параметризацией $f\cln[D]\ra\R^3$,
$[D]\subset\R^2$, то для любого множества $E\subset[D]$ сужение
$f\evn{E}\cln E\ra\R^3$ называется частью поверхности $\Ph$.
\end{dfn}

Если $\Uc\subset[D]$ --- открытое множество, то отображение $f\cln
\Uc\ra\R^3$ называют открытом частью поверхности $\Ph$. Эта открытая
часть поверхности $\Ph$ называется окрестностью на поверхности $\Ph$
любой её точки $f(M)$, где $M\in\Uc$.

\subsubsection{Касательная плоскость и нормаль к поверхности}

Рассмотрим произвольную поверхность $\Ph$ в $\R^3$ класса $\Cc^1$ и
пусть $\ol{r}=\ol{r}(u,v)$, $(u,v)\in[D]$ --- некоторая её
параметризация класса $\Cc^1$. Фиксируем произвольную точку
$(u_0,v_0)\in[D]$. Пусть в $\R^3$ задана декартова система координат
$(x,y,z)$, так что $\ol{r}(u,v)=(x(u,v),y(u,v),z(u,v))$, $(u,v)\in
D$.

Рассмотрим пересечение множества $[D]$ с прямой $v=v_0$. Тогда
$\ol{r}=\ol{r}(u,v_0)$ определяет некоторый путь $L_u$ класса
$\Cc^1$ в $\R^3$. Вектор $\ol{r}_u = \frac{\pd \ol{r}}{\pd u} =
(x'_u,y'_u,z'_u)$ является касательным вектором к пути $L_u$.
Аналогично определяется координатный путь $L_v$ класса $\Cc^1$, для
которого вектора $\ol{r}_v=\frac{\pd \ol{r}}{\pd
v}=(x'_v,y'_v,z'_v)$ является вектором касательной.

\begin{dfn}
9 Точка $\ol{r}(u,v)$ поверхности $\Ph$ называется неособой при
данной параметризации поверхности, если векторы $\ol{r}_u$ и
$\ol{r}_v$ не коллинеарны (то есть, линейно независимы), и точка
$\ol{r}(u,v)$ поверхности $\Ph$ называется особо точкой для данной
параметризации поверхности, если векторы $\ol{r}_u$ и $\ol{r}_v$
коллинеарны.
\end{dfn}

По определению, точка поверхности $\Ph$ класса $\Cc^1$ будет
неособой для данной параметризации $\ol{r}(u,v)$ тогда и только
тогда, когда векторное произведение
$\ol{r}_u\times\ol{r}_v\ne\ol{0}$; в частности, $\ol{r}_u\ne\ol{0}$,
$\ol{r}_v\ne\ol{0}$.

Пусть в $\R^3$ задана декартова система координат $Oxyz$. Тогда
$\ol{r}(u,v)=(x(u,v),y(u,v),z(u,v))$ и $\ol{r}_u=(x'_u,y'_u,z'_u)$,
$\ol{r}_v=(x'_v,y'_v,z'_v)$.

\begin{ex}
Рассмотрим график $\Ga$ непрерывно дифференцируемой функции
$z=f(x,y)$, $(x,y)\in[D]$, $D$ --- ограниченная область в $\R^2$.
Тогда $x=u,y=v,z=f(u,v)$, $\ol{r}=\ol{r}(u,v) = (u,v,f(u,v))$,
$(u,v)\in[D]$ и $\ol{r}_u=(1,0,f'_u)$, $\ol{r}_v=(0,1,f'_v)$.
Поэтому $\hm{\ol{r}_u\times\ol{r}_v}=\sqrt{1+(f'_u)^2+(f'_v)^2}>0$,
так что все точки простой поверхности $\Ga$ неособые.
\end{ex}

Рассмотрим произвольное отображение $f\cln[a,b]\ra D$ класса
$\Cc^1$, определяющее в $D$ путь $l$. Если $f=(\ph,\psi)$, то
$u=\ph(t)$, $v=\psi(t)$, $t\in[a,b]$. Тогда векторная функция
$\ol{r}=\ol{r}(\ph(t),\psi(t))$, $t\in[a,b]$, определяет на
поверхности $\Ph$ некоторый путь $L$ класса $\Cc^1$. При этом,
$d\ol{r} = \ol{r}_u\cdot du + \ol{r}_v\cdot dv$, где $du =
\ph'(t)\,dt$, $dv=\psi'(t)\,dt$. Если точка $\ol{r}(u,v)$
поверхности $\Ph$ не особая, то получаем, что вектор $d\ol{r}$
находится в плоскости векторов $\ol{r}_u$ и $\ol{r}_v$, являясь
одновременно вектором касательной к пути $L$ в точке
$\ol{r}(\ph(t),\psi(t))$.

\begin{dfn}
{10} Касательной плоскостью $\Pi$ к поверхности $\Phi$ в её точке
$\ol{r}(u_0,v_0)$ называют такую проходящую через $\ol{r}(u_0,v_0)$
плоскость, в которой лежат все касательные к кривым, расположенным
на поверхности и проходящим через $\ol{r}(u_0,v_0)$.
\end{dfn}

\begin{stm}
Если данная точка поверхности не особая, то в ней существует
касательная плоскость к поверхности.
\end{stm}

\begin{proof}
Эта плоскость $\Pi$ определяется условием: векторы $d\ol{r} =
\ol{r}_u\cdot du + \ol{r}_v\cdot dv$ и $\ol{r}-\ol{r}_0 =
\ol{r}(u,v) - \ol{r}(u_0,v_0)$ лежат в $\Pi$. Поэтому их смешанное
произведение $(\ol{r}-\ol{r}_0, \ol{r}_u, \ol{r}_v)=0$. Если
$\ol{r}-\ol{r}_0 = (x-x_0,y-y_0,z-z_0)$,
$\ol{r}_u=(x'_u,y'_u,z'_u)$, $\ol{r}_v=(x'_v,y'_v,z'_v)$, то
уравнение касательной плоскости имеет вид \equ{\mbmat{ x-x_0 & y-y_0
& z - z_0 \\ x'_u & y'_u & z'_u \\ x'_v & y'_v & z'_v}=0} и
коэффициентами при $(x-x_0)$, $(y-y_0)$, $(z-z_0)$ служат миноры
матрицы $\rbmat{x'_u & y'_u & z'_u \\ x'_v & y'_v & z'_v}$.
\end{proof}

\begin{ex}
График $\Ga$ непрерывно дифференцируемой функции $z=f(x,y)$,
$(x,y)\in[D]$, $D$ --- ограниченная область в $\R^2$, имеет
$\ol{r}=(u,v,f(u,v))$, $(u,v)\in[D]$, и в каждой точке
$(x_0,y_0,z_0)\in \Ga$, где $(x_0,y_0)\in D$, $z_0=f(x_0,y_0)$,
имеет касательную плоскость с уравнением $z-z_0=f'_x(x_0,y_0)(x-x_0)
+ f'_y(x_0,y_0)(y-y_0).$
\end{ex}

\begin{dfn}
{11} Прямая, проведённая через точку касания поверхности с
касательной плоскостью, и перпендикулярная касательной плоскости,
называется нормалью к поверхности в точке $\ol{r}(u_0,v_0)$.
Уравнение нормали \equ{ \frac{x-x_0}{\mbmat{y'_u & z'_u \\ y'_v &
z'_v}} = \frac{y-y_0}{\mbmat{z'_u & x'_u \\ z'_v & x'_v}} =
\frac{z-z_0}{\mbmat{x'_u & y'_u \\ x'_v & y'_v}}} и по крайней мере
один из определителей не равен нулю, так как
$\ol{r}_u\times\ol{r}_v\ne\ol{0}$.
\end{dfn}

В векторной форме единичным вектором нормали $\ol{\nu}$ к
поверхности $\Ph$ в неособой точке $\ol{r}(u,v)$ служит вектор
$\frac{\ol{r}_u\times\ol{r}_v}{\hm{\ol{r}_u\times\ol{r}_v}}=\ol{\nu}$.

\begin{theorem}
Неособая (особая) при данной параметризации точка поверхности класса
$\Cc^1$ будет неособой (особой) и при любой другой параметризации
этой поверхности, а плоскость, касательная к поверхности в неособой
точке при одной параметризации поверхности, будет касательной и при
другой её параметризации.
\end{theorem}

\begin{proof}
Пусть $\ol{r}(u,v)$, $(u,v)\in[D]$, и $\ol{p}(u_1,v_1)$,
$(u_1,v_1)\in[D_1]$, суть две параметризации поверхности $\Ph$
класса $\Cc^1$. Тогда существует допустимое отображение
(квазирегулярное) \eqa{1}{u_1=\ph(u,v), \; v_1=\psi(u,v)} компакта
$[D]$ на компакт $[D_1]$, что \eqa{2}{\ol{r}(u,v) =
\ol{p}(\ph(u,v),\psi(u,v)), \; (u,v)\in[D].} Продифференцировав
тождество (2), получим \eqa{3}{\mat{\ol{r}'_u=\ph'_u \; \ol{p}_{u_1}
+ \psi'_u \;\ol{p}_{v_1} \\ \ol{r}_v = \ph'_v \;\ol{p}_{u_1} +
\psi'_v\; \ol{p}_{v_1}.}}

Преобразование (3) векторов $\ol{p}_{u_1},\ol{p}_{v_1}$ в векторы
$\ol{r}_u,\ol{r}_v$ не вырождено в $D$, так как определитель
\equ{\mbmat{\ph'_u & \psi'_u \\ \ph'_v & \psi'_v} = \mbmat{\ph'_u &
\ph'_v \\ \psi'_u & \psi'_v} = \frac{D(\ph,\psi)}{D(u,v)}} и якобиан
$\frac{D(\ph,\psi)}{D(u,v)} \ne0$, $(u,v)\in[D]$. Из (3) следует,
что \eqa{4}{\ol{r}_u\times\ol{r}_v = (\ph'_u \; \ol{p}_{u_1} +
\psi'_u \;\ol{p}_{v_1}) \times (\ph'_v \;\ol{p}_{u_1} + \psi'_v\;
\ol{p}_{v_1}) = \frac{D(\ph,\psi)}{D(u,v)} (\ol{p}_{u_1} \times
\ol{p}_{v_1}).}

Поскольку $\frac{D(\ph,\psi)}{D(u,v)} \ne0$, то одновременно
$\ol{r}_u\times\ol{r}_v\ne\ol{0}$ и $\ol{p}_{u_1}\times\ol{p}_{v_1}
\ne \ol{0}$ и плоскость $\Pi$, образованная векторами $\ol{r}_u$ и
$\ol{r}_v$, совпадает с плоскостью $\Pi_1$, образованной векторами
$\ol{p}_{u_1}$ и $\ol{p}_{v_1}$.
\end{proof}

\begin{dfn}
{12} Поверхность $\Ph$ класса $\Cc^1$, допускающая параметризацию
класса $\Cc^1$, относительно которой все точки поверхности $\Ph$
неособые, называется гладкой поверхностью.
\end{dfn}

\begin{ex}
График $\Ga$ непрерывно дифференцируемой функции $z=f(x,y)$ на
компакте $[D]$ является простой гладкой поверхностью.
\end{ex}

\subsubsection{Явные представления поверхности}

Пусть $\Ph$ --- поверхность класса $\Cc^1$ с параметризацией
$\ol{r}(u,v) = (x(u,v),y(u,v),z(u,v))$, $(u,v)\in[D]$, и
$(u_0,v_0)\in D$ --- внутренняя неособая точка; то есть,
$\ol{r}_u\times \ol{r}_v\ne\ol{0}$ в $(u_0,v_0)$. Тогда $\rbmat{x'_u
& y'_u & z'_u \\ x'_v & y'_v & z'_v}$ и \equ{\hm{\ol{r}_u \times
\ol{r}_v} = \sqrt{\mbmat{y'_u & z'_u \\ y'_v & z'_v}^2 + \mbmat{z'_u
& x'_u \\ z'_v & x'_v}^2 + \mbmat{x'_u & y'_u \\ x'_v & y'_v}^2}>0}
в $(u_0,v_0)$ и хотя бы один из определителей под знаком корня
отличен от нуля. Пусть $\mbmat{x'_u & y'_u \\ x'_v & y'_v}\ne0$ в
точке $(u_0,v_0)$. По теореме о неявной функции для отображения
\eqa{5}{x=x(u,v), \; y=y(u,v)} существуют такие окрестности $\Uc$ и
$\Vc$ соответственно точек $(u_0,v_0)$ и $(x_0,y_0)$,
$x_0=x(u_0,v_0)$, $y_0=y(u_0,v_0)$, что отображение (5) будет
диффеоморфизмом областей $\Uc$ и $\Vc$ и в $\Vc$ существует
непрерывно дифференцируемая биекция $\Vc$ на $\Uc$, задаваемая
формулами $u=u(x,y)$, $v\ne v(x,y)$, $(x,y)\in \Vc$. Тогда функция
$z=z(u,v)=z(u(x,y),v(x,y))$ непрерывно дифференцируема в области
$\Vc$ и её график составляет часть $\Fc$ поверхности $\Ph$,
соответствующую окрестности $\Uc$ точки $(u_0,v_0)\in D$. Другими
словами, часть $\Fc$ поверхности $\Ph$ допускает явное задание в
виде графика некоторой функции $z=z(u(x,y),v(x,y))$, $(x,y)\in\Vc$.

\begin{lem}
1 Пусть функция $z=f(x,y)$ непрерывно дифференцируема в некоторой
окрестности $\Uc$ точки $(x_0,y_0)$ и пусть $z_0=f(x_0,y_0)$. Тогда
у точки $(x_0,y_0,z_0)$ существует такая окрестность $\Wc$ в
пространстве $\R^3\cln Oxyz$ и такая окрестность $\Vc$ на
касательной плоскости $\Pi$ к графику $\Ga$ функции $z=f(x,y)$ в
точке $(x_0,y_0,z_0)$, что пересечение графика $\Ga$ с окрестностью
$\Wc$ взаимно однозначно проектируется на окрестность $\Vc$ при
проектировании в направлении, перпендикулярном плоскости $\Pi$.
\end{lem}

\begin{proof}
Обозначим $f^0_x = f'_x(x_0,y_0)$, $f^0_y=f'_y(x_0,y_0)$. Если
$f^0_x=f^0_y=0$, то касательная плоскость $\Pi$ к графику $\Ga$ в
точке $(x_0,y_0,z_0)$ параллельна координатной плоскости $xOy$, и мы
имеем $\Wc=\Uc\times[z_0-\de,z_0+\de]$, $\de>0$, и
$\Vc=\hc{(x,y,z)\bvl(x,y)\in\Uc, z=z_0}$.

Пусть $f^{0^2}_x + f^{0^2}_y>0$ и, для определённости, $f^0_x\ne0$.
Вектор $\ol{\nu}=(f^0_x,f^0_y,-1)$ параллелен нормали к поверхности
$\Ga$ в точке $(x_0,y_0,z_0)$. Поэтому уравнение произвольной
прямой, параллельной нормали к $\Ga$ в точке $(x_0,y_0,z_0)$ имеет в
векторной форме вид \eqa{1}{\ol{r}=\ol{r}_0+\ol{a}+\ol{\nu}t, \;
t\in(-\bes,+\bes), \; \ol{r}=(x,y,z),} где $\ol{r}_0=(x_0,y_0,z_0)$
и $\ol{a}$ --- вектор, параллельный касательной плоскости $\Pi$.
Вектор $\ol{a}$ можно записать в виде $\ol{a}=(\al,\be,\al f^0_x+\be
f^0_y)$, где $\al,\be\in\R$, поскольку $\ol{a}\ol{\nu} = \al f^0_x +
\be f^0_y - (\al f^0_x + \be f^0_y)=0$.

Исключая параметр $t$ из уравнения (1), получаем в координатной
форме \equ{\mat{x=x_0+\al+f^0_{x'}t \bvl f^0_y \\ y =
y_0+\be+f^0_{y'}t \,\bvl f^0_x\\ z=z_0+(\al f^0_x + \be f^0_y) -
t,}} так что \equ{\mat{(x-x_0-\al)f^0_y = (y-y_0-\be)f^0_x \\
[z-z_0-(\al f^0_x + \be f^0_y)]f^0_x = -[x-x_0-\al]}} или
\eqa{2}{\case{(x-x_0-\al)f^0_y - (y-y_0-\be)f^0_x=0 \\ x-x_0-\al +
[z-z_0 - (\al f^0_x + \be f^0_y)] f^0_x=0.}}

Подставляя в систему (2) значение $z=f(x,y)$, получим систему
уравнений для определения координат $x$ и $y$ точки пересечения
прямой (1) с графиком $\Ga$ функции $z=f(x,y)$. Имеем \eqa{3}{
\case{(x-x_0-\al)f^0_y - (y-y_0-\be)f^0_x=0 \\ x-x_0-\al + [f(x,y) -
z_0 - (\al f^0_x + \be f^0_y)]f^0_x=0}.}

Якобиан $\Jc(x,y)$ системы (3) по переменным $x$ и $y$ имеет вид
\equ{\Jc(x,y) = \mbmat{f^0_y & -f^0_x \\ 1+f'_xf^0_x & f'_yf^0_x} =
(1+f'_xf^0_x + f'_yf^0_y)f^0_x.}

Левые части уравнений системы (3) являются непрерывно
дифференцируемыми функциями по переменным $x,y,\al,\be$. Значения
$x=x_0,y=y_0$ и $\al=\be=0$ удовлетворяют уравнениям системы (3).
Кроме того, $\Jc(x_0,y_0)=(1+f^{0^2}_x + f^{0^2}_y)f^0_x\ne0$.
Согласно теореме о локальном диффеоморфизме, существует такая
окрестность $\Uc_0$ точки $(x_0,y_0)$, $\Uc_0\subset\Uc$, и такая
окрестность $\Vc_0$ точки $(0,0)$ на плоскости параметров $\al$ и
$\be$, что система уравнений (3) имеет единственное решение
\eqa{4}{\mat{x=x(\al,\be) \\ y= y(\al,\be),}} которое образует
непрерывно дифференцируемую биекцию окрестностей $\Uc_0$ и $\Vc_0$.

Можно считать, что $\Vc_0$ --- круг радиуса $\ep>0$; то есть, $\Vc_0
= \hc{(\al,\be)\bvl \al^2+\be^2 < \ep^2}$, $\ep>0$. Тогда для любого
вектора $\ol{a}$, у которого $\al^2+\be^2<\ep^2$, прямая (1) имеет,
и при том единственную, точку пересечения с графиком сужения функции
$f$ на окрестность $\Uc_0$ точки $(x_0,y_0)$. Координаты $x$ и $y$
этой точки пересечения находятся по формуле (4), а координата
$z=f(x,y)$.

Так как \equ{\hm{\ol{a}}=\sqrt{\al^2+\be^2+(\al f^0_x+\be f^0_y)^2}
\ge \sqrt{\al^2+\be^2},} то из условия $\hm{\ol{a}}<\ep$ следует
условие $\al^2+\be^2<\ep^2$, и следовательно, прямая (1) пересекает
график $\Ga$ функции $z=f(x,y)$ в единственной точке.
\end{proof}

\subsection{Ориентируемые поверхности}
\subsubsection{Ориентация гладкой поверхности}

В этом параграфе предполагается, что в пространстве $\R^3$ выбрана
правая система декартовых координат. Это значит, что в пространстве
точек $(x,y,z)$ рассматриваются только такие упорядоченные базисы
$\ol{e}_1,\ol{e}_2,\ol{e}_3$, которые получаются из упорядоченного
базиса $\ol{i}=(1,0,0), \ol{j}=(0,1,0), \ol{k}=(0,0,1)$ только
такими ортогональными преобразованиями \equ{\ol{e}_m = c_{m1}\ol{i}
+ c_{m2} \ol{j} + c_{m3}\ol{k}, \; m=1,2,3,} у которых определитель
\equ{\mbmat{c_{11} & c_{12} & c_{13} \\ c_{21} & c_{22} & c_{23} \\
c_{31} & c_{32} & c_{33}}=+1.}

Рассмотрим произвольную гладкую поверхность $\Ph\subset\R^3$ и
некоторую её параметризацию $\ol{r}=\ol{r}(u,v)$, $(u,v)\in[D]$,
класса $\Cc^1$. Тогда $\ol{r}_u\times\ol{r}_v\ne\ol{0}$ для всех
$(u,v)\in[D]$, и значит, в каждой точке поверхности $\Ph$ определён
единичный нормальный вектор \eqa{1}{\ol{\nu} =
\frac{\ol{r}_u\times\ol{r}_v}{\hm{\ol{r}_u\times\ol{r}_v}},} который
является непрерывной функцией аргументов $(u,v)\in[D]$; то есть, на
поверхности $\Ph$ существует непрерывная единичная нормаль.

\begin{dfn}
1 Всякая непрерывная единичная нормаль $\ol{\nu}=\ol{\nu}(u,v)$;
$(u,v)\in[D]$, гладкой поверхности $\Ph=\hc{\ol{r}(u,v) \bvl
(u,v)\in[D]}$ называется ориентацией поверхности $\Ph$.
\end{dfn}

Понятно, что если вектор $\ol{\nu}$ является ориентацией поверхности
$\Ph$, то и вектор $-\ol{\nu}$ является ориентацией поверхности
$\Ph$, и других ориентаций на $\Ph$ нет. Одна из двух ориентаций
$\ol{\nu}$ или $-\ol{\nu}$ (произвольно выбранная) называется
положительной, а другая --- отрицательной.

В пространстве $\R^3$ с правой системой координат принято для
гладкой поверхности $\Ph$, заданной параметризацией
$\ol{r}=\ol{r}(u,v)$, $(u,v)\in[D]$, класса $\Cc^1$, считать
положительной ориентацией нормальный единичный вектор (1).

Поверхность $\Ph$ с положительной ориентацией будем обозначать
символом $\Ph^+$, поверхность $\Ph$ с отрицательной ориентацией
обозначим через $\Ph^-$. Поверхность, у которой фиксирована одна из
ориентаций, называется ориентированной поверхностью.

\subsubsection{Сохранение ориентации при допустимых отображениях}

Пусть положительно ориентированная гладкая поверхность $\Ph$ задана
параметризацией $\ol{r}=\ol{r}(u,v)$, $(u,v)\in[D]$, класса $\Cc^1$,
и единичный нормальный вектор определяется формулой (1). Рассмотрим
произвольное допустимое преобразование параметров $u_1=\ph(u,v)$,
$v_1=\psi(u,v)$, $(u,v)\in[D]$, $(u_1,v_1)\in[D_1]$. Обозначим
$\ol{p}(u_1,v_1)=\ol{p}(\ph(u,v),\psi(u,v)) = \ol{r}(u,v)$. Тогда
\equ{\ol{r}_u\times\ol{r}_v =
\frac{D(\phi,\psi)}{D(u,v)}(\ol{p}_{u_1} \times \ol{p}_{v_1}).}

Следовательно, векторы $\ol{r}_u\times\ol{r}_v$ и
$\ol{p}_{u_1}\times\ol{p}_{v_1}$ направлены в одну сторону только
тогда, когда якобиан $\frac{D(\ph,\psi)}{D(u,v)}>0$ во всех точках
$(u,v)\in D$.

Таким образом, для поверхностей, у которых выбрана ориентация,
допустимыми преобразованиями параметров будем считать только такие
квазирегулярные отображение компактов из $\R^2$, у которых якобианы
положительные.

\subsubsection{Ориентация графика функции двух переменных}

Рассмотрим график $\Ga$ функции $z=f(x,y)$, непрерывно
дифференцируемой на компакте $[D]$, $D$ --- область в $\R^2$. Тогда
$\ol{r}=\ol{r}(x,y) = (x,y,f(x,y))$, $\ol{r}_x = (1,0,f'_x)$,
$\ol{r}_y=(0,1,f'_y)$ и \equ{\ol{r}_x\times\ol{r}_y = \mbmat{\ol{i}
& \ol{j} & \ol{k} \\ 1 & 0 & f'_x \\ 0 & 1 & f'_y} = -f'_x
\cdot\ol{i} - f'_y\cdot \ol{j} + \ol{k},} и
\equ{\ol{\nu}=\hr{\frac{-f'_x}{\sqrt{1+f'^2_x+f'^2_y}},
\frac{-f'_y}{\sqrt{1+f'^2_x+f'^2_y}},
\frac1{\sqrt{1+f'^2_x+f'^2_y}}}.} Следовательно,
\equ{\cos(\ol{\nu},\ol{k})=\frac1{\sqrt{1+f'^2_x+f'^2_y}}>0} для
всех $(x,y)\in D$.


Таким образом, положительная ориентация на графике $\Ga$ функции
$z=f(x,y)$ образует острый угол с осью $\ol{Oz}$. Поэтому $\Ga^+$
называется верхней стороной поверхности $\Ga$, а $\Ga^-$ --- нижней
стороной поверхности $\Ga$.

\subsubsection{}

Определение ориентации не переносится на негладкие поверхности.

\begin{ex}
Конус $z=\sqrt{x^2+y^2}$, $x^2+y^2\le a^2$, $a>0$. Векторное
представление имеет вид $\ol{r}(u,v)=(u,v,\sqrt{u^2+v^2})$,
$u^2+v^2\le a^2$, и $$\ol{r}_u=\hr{1,0,\frac{u}{\sqrt{u^2+v^2}}}, \;
\ol{r}_v=\hr{1,0,\frac{v}{\sqrt{u^2+v^2}}}, \; (u,v)\ne(0,0).$$

$$\ol{r}_u\times\ol{r}_v = \hr{-\frac{u}{\sqrt{u^2+v^2}}, -
\frac{v}{\sqrt{u^2+v^2}}, 1}, \;
\hm{\ol{r}_u\times\ol{r}_v}=\sqrt2,$$ и $$\ol{\nu}(u,v) =
\hr{-\frac{u}{\sqrt2 \sqrt{u^2+v^2}}, -
\frac{v}{\sqrt2\sqrt{u^2+v^2}}, \frac1{\sqrt2}}.$$

Так как не существует $$\liml{(u,v)\ra(0,0)}
\frac{u}{\sqrt{u^2+v^2}}, \; \liml{(u,v)\ra(0,0)}
\frac{v}{\sqrt{u^2+v^2}},$$ то не существует $\liml{(u,v)\ra(0,0)}
\ol{\nu}(u,v)$.

Значит, на конусе нельзя выбрать нормаль, непрерывную на $[D] =
\hc{(u,v)\bvl u^2+v^2\le a^2}$.
\end{ex}

\begin{ex}
Любая часть любого ненулевого двугранного угла является примером
негладкой поверхности, на которой нормаль (при любом её выборе)
имеет целую прямую точек разрыва.
\end{ex}

\subsubsection{Склеивание поверхностей}
Будем говорить, что у поверхности $\Ph=\hc{\ol{r}=\ol{r}(u,v), \;
(u,v)\in[D]}$, ($D$ --- область в $\R^2$) её край $\pd \Ph$ является
кривой, если граница $\pd D$ области $D$ является кривой (точнее,
носителем кривой) и $\pd D = \hc{(u(t),v(t)) \bvl t\in[a,b]}$; то
есть, область $D\subset\R^2$ односвязная, ограниченная замкнутой
кривой.

В этом случае, край $\pd \Ph$ можно рассматривать в виде кривой
$L=\pd\Ph=\hc{\ol{r}(u(t),v(t))\bvl t\in[a,b]}.$

Пусть заданы поверхности $\Ph_i=\hc{\ol{r}(u_i,v_i) \bvl
(u_i,v_i)\in[D_i]}$, $i=1,2\sco m$, у которых $$\pd D_i =
\hc{(u_i(t_i),v_i(t_i))\bvl t_i\in[a_i,b_i]},\;i=\ol{1,m}.$$ Значит,
края $$\pd \Ph_i=L_i=\hc{\ol{r}_i(u_i(t_i),v_i(t_i))\bvl
t_i\in[a_i,b_i]}, \;i=\ol{1,m}.$$

Предположим, что для некоторых пар $(i,j)$, $i,j=\ol{1,m}$, $i\ne
j$, задано конечно число $n_{ij}=n_{ji}$ отрезков
$[a_{ij}^k,b^k_{ij}]\subset[a_i,b_i]$, $a^k_{ij}\le b^k_{ij}$, и
отрезков $[a^k_{ji}]\subset[a_j,b_j]$, $a^k_{ji}\le b^k_{ji}$, и
$k=1,2\sco n_{ij}=n_{ji}$, причём как отрезки $[a^k_{ij},b^k_{ij}]$,
так и отрезки $[a^k_{ji},b^k_{ji}]$ попарно не имеют общих
внутренних точек, и пусть $\ph^k_{ij}\cln
[a^k_{ij},b^k_{ij}]\ra[a^k_{ji},b^k_{ji}]$ --- строго возрастающие
гомеоморфизмы.

Гомеоморфизмы $\ph^k_{ij}$ называются склеивающими гомеоморфизмами,
если для любого $t_i\in[a^k_{ij},b^k_{ij}]$ имеет место условие
склеивания \eqa{2}{\ol{r}_i(u_i(t_i),v_i(t_i)) =
\ol{r}_j(u_j(\ph^k_{ij} (t_i)), v_j(\ph^k_{ij} (t_i))) = \ol{r}_j
(u_j(t_j),v_j(t_j)).}

Рассмотрим кривые $L^k_{ij}$, \equ{L^k_{ij} =
\hc{\ol{r}_i(u_i(t_i),v_i(t_i)) \bvl t_i\in[a^k_{ij},b^k_{ij}]}.}

Векторная функция $\ol{r}=\ol{r}_j(u_j(t_j),v_j(t_j))$,
$t_j\in[a_{ji}^k,b_{ji}^k]$, в силу равенства (2) также является
параметризацией кривой $L^k_{ij}$, так как гомеоморфизм $\ph^k_{ij}$
есть допустимое преобразование параметра кривой $L^k_{ij}$.

Будем предполагать также, что при $j'\ne j$ отрезки
$[a^k_{ij},b^k_{ij}]$ и $[a^l_{ij'},b^l_{ij'}]$, $k=1,2\sco n_{ij}$;
$l=1,2\sco n_{ij'}$, не имеют общих внутренних точек, и
следовательно, каждый конец отрезка $[a^k_{ij},b^k_{ij}]$ может
принадлежать ещё не более, чем одному отрезку
$[a^l_{ij'},b^l_{ij'}]$. Это условие означает, что каждая кривая
склейки $L^k_{ij}$ является частью только двух кривых $L_i$ и $L_j$,
образующих края поверхностей $\Ph_i$ и $\Ph_j$.

Поверхности $\Ph_i$ и $\Ph_j$ называются соседними, если они
склеиваются по крайней мере по одной кривой $L_{ij}$. Система
склеивающих гомеоморфизмов $\ph^k_{ij}$ называется связной, если для
любых поверхностей $\Ph_p$ и $\Ph_q$ из рассматриваемой системы
поверхностей в этой системе существуют такие поверхности
$\Ph_{i_1}\sco\Ph_{i_r}$, что $\Ph_{i_1}=\Ph_p$, $\Ph_{i_r} =
\Ph_q$, и каждая поверхность $\Ph_{i_\nu}$ является соседней с
$\Ph_{i_{\nu+1}}$; то есть, склеена с ней по одной или по нескольким
кривым с помощью соответствующих склеивающих гомеоморфизмов
$\ph_{i_{\nu},i_{\nu+1}}$ для $\nu=1,2\sco r-1$.

\begin{dfn}
2 Система поверхностей $\Ph_1,\Ph_2\sco\Ph_m$ со связной системой
склеивающих гомеоморфизмов $\ph^k_{ij}$ называется поверхностью,
склеенной из поверхностей $\Ph_1\sco\Ph_m$ по кривым $L^k_{ij}$ и
обозначается $\Ph=\hc{\Ph_i}$.
\end{dfn}

Если $\Ph=\hc{\Ph_i}$ --- склеенная поверхность, то совокупность
всех дуг, являющихся такими частями кривых $L_i=\pd \Ph_i$, что
никакие точки этих частей, кроме, быть может, концевых, не
склеиваются ни с какими точками других кривых $\pd \Ph_i$,
называется краем $\pd \Ph$ склеенной поверхности $\Ph$. Край
склеенной поверхности состоит из конечного числа замкнутых контуров.

\begin{ex}
Сфера $S\cln x^2+y^2+z^2=1$ получается склеиванием двух полусфер
$S_1\cln z=\sqrt{1-x^2-y^2}$ и $S_2\cln z=-\sqrt{1-x^2-y^2}$,
$x^2+y^2\le1$.
\end{ex}

\begin{proof}
Действительно, поверхность $S_1$ задаётся параметризацией
$\ol{r}_1(u_1,v_1)=(u_1,v_1,\sqrt{1-u_1^2-v_1^2})$,
$u_1^2+v_1^2\le1$, а поверхность $S_2$ --- параметризацией
$\ol{r}_2(u_2,v_2)=(u_2,v_2,-\sqrt{1-u_2^2-v_2^2})$,
$u_2^2+v_2^2\le1$.

Граница $\pd D_i$ области $D_i$, $i=1,2$, задаётся параметризацией
$u_i=\cos t_i$, $v_i=\sin t_i$, $t_i\in[0,2\pi]$. Кривые
$L_i=\hc{\ol{r}(u_i(t_i),v_i(t_i)) \bvl t_i\in[0,2\pi]}$, $i=1,2$.
Отрезки $[a^k_{ij},b^k_{ij}]=[0,2\pi] = [a^k_{ij},b^k_{ij}]$, и
гомеоморфизм $\ph^k_{ij} = \ph\cln[0,2\pi]\ra[0,2\pi]$ ---
тождественный; то есть, $t_1=t_2$. Условие (2) склеивания выполнено,
так как \equ{\ol{r}_1(u_1(t_1),v_1(t_1))=(\cos t_1,\sin
t_1,\sqrt{1-\cos^2 t_1-\sin^2 t_1}=0) = (\cos t_2,\sin t_2, 0 =
\sqrt{1-\cos^2t_2-\sin^2t_2})=\ol{r}_2(u_2(t_2),v_2(t_2)).}
Склеенная поверхность $S$ имеет пустую границу.
\end{proof}

\subsubsection{Ориентация склеенной поверхности}

Поверхность $\Ph=\hc{\Ph_i}$, склеенная из гладких поверхностей
$\Ph_1\sco \Ph_m$, называется кусочно--гладкой поверхностью.

Пример поверхности двугранного угла показывает, что на
кусочно--гладкой поверхности нельзя, вообще говоря, ввести понятие
ориентируемости в терминах непрерывной нормали.

Более того, при склеивании поверхностей даже к гладким образомк
у склеенных поверхностей могут возникнуть качественно новые
особенности. Пример служит лист Мёбиуса.

Опишем другой подход к понятию ориентации, основанный на склеивании
поверхностей, края которых суть замкнутые кривые.

Пусть $\Ph=\hc{\ol{r}(u,v)\bvl (u,v)\in[D]}$ --- гладкая
поверхность, краем которой является кривая. Тогда границей $\pd D$
области $D$ на $\R^2$ является замкнутая кривая $\pd
D=\hc{(u(t),v(t))\bvl t\in[a,b]}$. На плоскости $\R^2$ задана правая
система декартовых координат. Будем считать положительным на кривой
$\pd D$ направление против часовой стрелки. Положительная ориентация
кривой $\pd D$, в силу отображения $\ol{r}(u(t),v(t))$, $t\in[a,b]$,
порождает вполне определённую ориентацию края $\pd \Ph$ поверхности
$\Ph$. Эта ориентация края $\pd \Ph$ поверхности $\Ph$ называется
согласованной с ориентацией
$\ol{\nu}=\frac{\ol{r}_u\times\ol{r}_v}{\hm{\ol{r}_u\times\ol{r}_v}}.$

Естественность такого определения поясним следующим образом.
Рассмотрим поверхность $\Ph=\Ga$ --- график непрерывно
дифференцируемой функции $z=f(x,y)$, $(x,y)\in[D]$. Тогда $$\ol{\nu}
= \hr{\frac{-f'_x}{\sqrt{1+f'^2_x+f'^2_y}},
\frac{-f'_y}{\sqrt{1+f'^2_x+f'^2_y}},
\frac1{\sqrt{1+f'^2_x+f'^2_y}}}$$ и $\cos\ol{\nu}\ol{k} =
\frac1{\sqrt{1+f'^2_x+f'^2_y}}>0$; то есть, вектор нормали
$\ol{\nu}$ образует с осью $Oz$ острый угол и с концевой точки
вектора нормали $\ol{\nu}$ ориентация края $\pd\Ph=L$ поверхности
$\Ph=\Ga$ является положительной.

Очевидно, что если ориентация $\ol{\nu}$ рассматриваемой гладкой
поверхности $\Ph$ согласована с ориентацией её края $\pd\Ph$, то
ориентация $-\ol{\nu}$ согласована с противоположной ориентацией
кривой $\pd\Ph$.

Таким образом, задание ориентации $\ol{\nu}$ гладкой поверхности
$\Ph$ равносильно заданию ориентации кривой $\pd\Ph$, являющейся
краем поверхности $\Ph$. Поэтому ориентированный край $\pd\Ph$
гладкой поверхности $\Ph$ будем называть ориентацией поверхности
$\Ph$.

Пусть $\Ph_1$ и $\Ph_2$ --- две гладкие поверхности, у которых края
$\pd\Ph_1$ и $\pd\Ph_2$ --- кривые и пусть эти поверхности склеены
по кривым $\ga_1\sco\ga_m$, являющимся частями краёв $\pd\Ph_1$ и
$\pd\Ph_2$ поверхностей $\Ph_1$ и $\Ph_2$. Ориентации краёв $\pd
\Ph_1$ и $\pd\Ph_2$ (поверхностей $\Ph_1$ и $\Ph_2$) называются
согласованными, если каждая из них порождает на склеивающихся кривых
$\ga_1\sco\ga_m$ противоположные ориентации.

\begin{dfn}
3 Поверхность $\Ph$, склеенная из поверхностей $\Ph_1\sco\Ph_m$
называется ориентируемой, если существуют такие ориентации краёв
$\pd\Ph_1\sco \pd\Ph_m$ поверхностей $\Ph_1\sco\Ph_m$, что для любых
двух соседних поверхностей $\Ph_i$ и $\Ph_j$ их ориентации $\pd
\Ph_i$ и $\pd\Ph_j$ согласованы.
\end{dfn}

\begin{ex}
Двугранный угол --- ориентируемая поверхность.
\end{ex}

Если указанной в определении 3 совокупности ориентаций $\pd\Ph_i$ не
существует, то поверхность $\Ph$ называется неориентируемой. Пример
--- лист Мёбиуса.

Край ориентированной склеенной поверхности $\Ph=\hc{\Ph_i}$, как
край всякой склеенной поверхности, состоит из конечно числа
замкнутых контуров. При этом, заданная согласованная ориентация
склеенной ориентируемой поверхности $\Ph=\hc{\Ph_i}$ порождает
определённые ориентации на указанных кривых. Эти ориентации, вместе
взятые, составляют ориентацию края $\pd\Ph$ склеенной поверхности
$\Ph$.

Можно показать, что всякая кусочно--гладкая поверхность, являющаяся
границей некоторой области трёхмерного пространства, ориентируема.

\begin{ex}
Сфера $S\cln x^2+y^2+z^2=1$ имеет внешнюю и внутреннюю стороны в
зависимости от направления нормали (наружу или внутрь).
\end{ex}

\subsection{Поверхностные интегралы}
\subsubsection{Первая квадратичная форма поверхности}

Рассмотрим гладкую поверхность $\Ph$, заданную параметризацией
$\ol{r}=\ol{r}(u,v)$, $(u,v)\in[D]$, $D$ --- область в $\R^2$.
Векторы $\ol{r}_u$ и $\ol{r}_v$ порождают касательную плоскость
$\Pi$ в каждой точке поверхности $\Ph$ и векторы $\ol{r}_u$,
$\ol{r}_v$ образуют базис на $\Pi$. Векторы, лежащие на $\Pi$,
обозначим $d\ol{r}$, а координаты вектора $d\ol{r}$ в базисе
$\ol{r}_u,\ol{r}_v$ обозначим через $du,$ $dv$. Таким образом,
$$d\ol{r}=\ol{r}_u\,du + \ol{r}_v\,dv.$$

Тогда \equ{\hm{dr}^2 = (\ol{r}_u\,du + \ol{r}_v\,dv)^2 =
\ol{r}^2_u\,du^2 + 2 \ol{r}_u\ol{r}_v\,du\,dv + \ol{r}^2_v\,dv^2.}

Обозначим $E=\ol{r}^2_u$, $F=\ol{r}_u\ol{r}_v$, $G=r^2_v$.
Квадратичная форма \eqa{1}{\hm{dr}^2 = E\,du^2 + 2F\,du\,dv +
G^2\,dv^2} называется первой квадратичной формой поверхности $\Ph$.

По определению, первая квадратичная форма положительно определена, и
значит, $EG-F^2>0$.

Рассмотрим квазирегулярное отображение $u_1=\ph(u,v)$,
$v_1=\psi(u,v)$ компакта $[D]$ на компакт $[D_1]$, и пусть
$\ol{r}(u,v) = \ol{p}(\ph(u,v),\psi(u,v))$. Тогда базисы в
касательной плоскости $\Pi$ преобразуются с помощью матрицы
$\rbmat{\ph'_u & \psi'_u \\ \ph'_v & \psi'_v}$. Следовательно,
координаты векторов преобразуются с помощью транспонированной
матрицы $\rbmat{\ph'_u & \ph'_v \\ \psi'_u & \psi'_v}=\Jc$; то есть,
$\Jc$ --- матрицы Якоби отображения $u_1=\ph(u,v)$, $v_1=\psi(u,v)$.

Если матрицу первой квадратичной формы для представления
$\ol{r}=\ol{r}(u,v)$ обозначить через $A$, а для представления
$\ol{r}=\ol{p}(u_1,v_1)$ через $A_1$, то есть, \equ{\mat{ A =
\rbmat{ E & F \\ F & G}, \; E=\ol{r}^2_u, \; F =
\ol{r}_u\cdot\ol{r}_v, \; G=\ol{r}^2_v \\ A_1 = \rbmat{ E_1 & F_1 \\
F_1 & G_1}, \; E_1=\ol{p}^2_u, \; F_1 = \ol{p}_u\cdot \ol{p}_v, \;
G_1=\ol{p}^2_v,}} то, как известно из курса линейной алгебры,
$A=\Jc^* A_1 \Jc$, где $\Jc^*$ --- транспонированная матрица к
$\Jc$. Значит, для соответствующих определителей имеем равенство

\eqa{1}{\mbmat{ E & F \\ F & G} = \mbmat{E_1 & F_1 \\ F_1 & G_1}
\cdot \mbmat{\ph'_u & \ph'_v \\ \psi'_u & \psi'_v }^2 \mbox{ или }
EG-F^2 = (E_1G_1-F_1^2) \hm{\frac{D(u_1,v_1)}{D(u,v)}}^2.}

\subsubsection{Площадь гладкой поверхности}

Длину дуги мы определяем как точную верхнюю грань длин вписанных
ломанных. Однако определить по аналогии площадь поверхности как
точную верхнюю грань площадей вписанных многогранников нельзя,
поскольку последняя может быть бесконечной даже для простейших
кривых поверхностей, как, например, для поверхности прямого
кругового цилиндра (этот пример предложен немецким математиком
Шварцем, XIX в., и он обычно приводится в любом учебном пособии по
многомерному математическому анализу). Но для гладких кривых длину
можно определить также как точную нижнюю грань (или предел) длин
описанных ломанных. Этот принцип положен в основу определения
сложного понятия площади гладкой поверхности. Качественно опишем
этот процесс.

Рассмотрим гладкую поверхность $\Ph$, заданную параметризацией
$\ol{r}=\ol{r}(u,v)$, $(u,v)\in[D]\subset D$, где $D$ --- область в
$\R^2(u,v)$ и $[D]$ --- квадрируемый компакт. Рассмотрим на
$\R^2(u,v)$ сетку шага $h=\frac1{2^n}$, $n\in\N$, с квадратами
$Q_k$, $k\in\N$, $пл.\,Q_k=h^2=\frac1{4^n}$, $k\in\N$, и выберем
$N\in\N$ таким, чтобы для любого $n\ge N$ части $\Ph_k$ поверхности
$\Ph$, получаемые отображением $\ol{r}$ компактов $]D[ \cap Q_k$,
можно проектировать на касательные плоскости, построенные в каждой
точке на $\Ph_k$. Последнее возможно в силу результатов, изложенных
в пунктах 3.1.3 и 3.1.4. Рассмотрим теперь разбиение $T$ компакта
$[D]$ на ячейки $\si_k = [D]\cap Q_k$, $k=\ol{1,m}$, и выберем на
каждой $\si_k$ произвольную точку $(u_k,v_k)$, которой соответствует
точка $P_k(u_k,v_k)\in\Ph_k$. Пусть $\Fc_k$, $k=\ol{1,m}$,
обозначает проекцию части $\Ph_k$ на касательную плоскость $\Pi_k$ к
$\Ph_k$ в точке $P_k$. Так как $\ol{r}(u+h,v)-\ol{r}(u,v) = \ol{r}_u
h + o(h)$, $\ol{r}(u,v+h)-\ol{r}(u,v) = \ol{r}_v h + o(h)$, $h\ra0$,
то с точностью до бесконечно малых порядка $o(h^2)$, $h\ra0$, можно
читать, что площадь $пл.\Fc_k$ пластинки $\Fc_k$ равна площади
параллелограмма на $\Pi_k$, натянутого на векторы $\ol{r}_u h$ и
$\ol{r}_v h$ в точке $P_k(u_k,v_k)$. Последняя по определению равна
$\hm{\ol{r}_u h \times \ol{r}_v h} = \hm{\ol{r}_u\times\ol{r}_v}
(u_k,v_k) h^2$, так что можно считать, что $пл.\,\Fc_k \cong
\hm{\ol{r}_u\times\ol{r}_v} (u_k,v_k)\cdot пл.\,\si_k$,
$k=\ol{1,m}$. Таким образом, суммарная площадь пластинок $\Fc_k$,
$k=\ol{1,m}$, образующих к чешуюк, покрывающую поверхность
$\Ph$, приблизительно совпадает с интегральной суммой \equ{\sumkum
\hm{\ol{r}_u\times\ol{r}_v}(u_k,v_k)\De\si_k} интеграла $\iints{[D]}
\hm{\ol{r}_u\times\ol{r}_v}\,du\,dv$, и значит, стремится к нему при
стремлении к нулю диаметра разбиения $T$ (при $n\ra+\bes$). Этот
предел и принимают по определению за площадь поверхности $\Ph$:
\equ{пл.\,\Ph = \iints{[D]}\hm{\ol{r}_u\times\ol{r}_v}\,du\,dv.}

Известно, что $\hm{\ol{a}\times\ol{b}}^2 + \hm{\ol{a}\ol{b}}^2 =
\hm{\ol{a}}^2\hm{\ol{b}}^2$. Поэтому $\hm{\ol{r}_u\times\ol{r}_v}^2
= \ol{r}_u^2\ol{r}_v^2 - (\ol{r}_u\ol{r}_v)^2 = EG-F^2$ и
\eqa{2}{пл.\,\Ph = \iints{[D]} \sqrt{EG-F^2}\,du\,dv.}

Формула (2) не зависит от выбора параметризации поверхности $\Ph$.
Действительно, пусть $\ol{p}(u_1,v_1)$, $(u_1,v_1)\in[D_1]$ ---
другая параметризация поверхности $\Ph$. Тогда \equ{пл.\,\Ph_1 =
\iints{[D_1]} \sqrt{E_1G_1-F_1^2}\,du_1\,dv_1 = \iints{[D_1]}
\sqrt{EF-G^2} \hm{\frac{D(u_1,v_1)}{D(u,v)}} \,du_1\,dv_1 =
\iints{D} \sqrt{EF-G^2}\,du\,dv=пл.\,\Ph.}

\subsubsection{Поверхностные интегралы первого рода}

Пусть $\Ph$ --- гладкая поверхность в $\R^3$, $\ol{r}=\ol{r}(u,v)$
--- её параметризация класса $\Cc^1$, $(u,v)\in[D]\subset\R^2$, $D$
--- область в $\R^2$ и $[D]$ --- квадрируемый компакт. Пусть функция
$F(M)$ определена в точках $M$ поверхности $\Ph$, так что на $[D]$
определена функция $(F\circ\ol{r})(u,v)$. Рассмотрим множество $\Ps$
всех размеченных разбиений компакта $[D]$ и базу $d(T)\ra0$ на
$\Ps$. Для произвольного размеченного разбиения $T_{\ze}\in\Ps$ с
ячейками $D_k$, $k=\ol{1,n}$, $\cupl{k=1}n D_k=[D]$, и набором $\ze$
точек $\ze_k = (\xi_k,\eta_k)\in D_k$, $k=\ol{1,n}$, поверхность
$\Ph$ разбивается на части $\Ph_k$, $k=\ol{1,n}$, $\cupl{k=1}n\Ph_k
= \Ph$, и каждая $\Ph_k$ задаётся параметризацией
$\ol{r}=\ol{r}(u,v)$, $(u,v)\in D_k$. При этом существуют площади
$пл.\,\Ph_k = \iints{D_k} \sqrt{EG-F^2}\,du\,dv.$

Рассмотрим интегральную сумму \eqa{3}{\si(F;T_{\ze}) = \sumkun
(F\circ \ol{r})(\xi_k,\eta_k) пл.\,\Ph_k.}

\begin{dfn}
1 Число $I=\liml{d(t)\ra0} \si(F; T_{\ze})$ (если предел существует)
называется интегралом первого рода функции $F$ на гладкой
поверхности $\Ph$. Обозначение: $I=\iints{\Ph} F\,ds$.
\end{dfn}

\begin{theorem}
Если функция $F(M)$ непрерывна на гладкой поверхности $\Ph$ и
$\ol{r}(u,v)$ --- некоторая векторная параметризация поверхности
$\Ph$ класса $\Cc^1$, $(u,v)\in[D]\subset\R^2$, $D$ --- область и
$[D]$ --- квадрируемый компакт, то поверхностный интеграл
$\iints{\Ph}F\,ds$ существует и \eqa{4}{\iints{\Ph}F\,ds =
\iints{[D]} (F\circ\ol{r})(u,v)\sqrt{EG-F^2}\,du\,dv.}
\end{theorem}

\begin{proof}
Рассмотрим произвольное размеченное разбиение $T_{\ze}$ компакта
$[D]$ с ячейками $D_k$, $k=\ol{1,n}$, $\cupl{k=1}n D_k=[D]$ и с
набором $\ze=(\ze_1\sco\ze_n)$ точек $\ze_k=(\xi_k,\eta_k)\in D_k$,
$k=\ol{1,n}$. В силу свойства аддитивности двойного интеграла, число
\eqa{5}{I=\iints{[D]} (F\circ\ol{r})(u,v)\sqrt{EG-F^2}\,du\,dv =
\sumkun \iints{D_k} (F\circ\ol{r})(u,v)\sqrt{EG-F^2}\,du\,dv} и
число $I$ существует, так как подинтегральная функция непрерывна на
$[D]$ по условию теоремы. Для каждого $k$, $1\le k\le n$,
\equ{пл.\,\Ph_k = \iints{D_k}\sqrt{EG-F^2}\,du\,dv} и следовательно,
по определению (3) и свойству линейности двойного интеграла,
\eqa{6}{\si(F;T_{\ze})=\sumkun (F\circ\ol{r})(\xi_k,\eta_k)
\iints{D_k} \sqrt{EG-F^2}\,du\,dv = \sumkun \iints{D_k} (F\circ
\ol{r})(\xi_k,\eta_k)\sqrt{EG-F^2}\,du\,dv.}

На основании (5) и (6) имеем оценки \mla{7}{\hm{\si(F;T_{\ze})-I} =
\hm{\sumkun \iints{D_k} [(F\circ\ol{r})(\xi_k,\eta_k) -
(F\circ\ol{r}) (u,v)] \sqrt{EG-F^2}\,du\,dv} \le \\ \le \sumkun
\iints{D_k} \hm{(F\circ \ol{r})(\xi_k,\eta_k) - (F\circ
\ol{r})(u,v)}\sqrt{EG-F^2}\,du\,dv  \le \sumkun
\om(F\circ\ol{r};D_k) \iints{D_k} \sqrt{EG-F^2}\,du\,dv=\\ = \sumkun
\om(F\circ\ol{r};D_k) пл.\,\Ph_k \le \om(F\circ\ol{r};T_{\ze})
\sumkun пл.\,\Ph_k = \om(F\circ\ol{r};T_{\ze})\cdot пл.\,\Ph,}

в которых $\om(F\circ\ol{r};D_k)$, $k=\ol{1,n}$, обозначает
колебание непрерывной функции $(F\circ\ol{r})(u,v)$ на компакте
$D_k$, а $\om(F\circ\ol{r};T_{\ze}) = \maxl{1\le k\le n}
\om(F\circ\ol{r};D_k)$. В силу непрерывности функции
$(F\circ\ol{r})(u,v)$, $\liml{d(T)\ra0}\om(F\circ\ol{r};T_{\ze})=0$,
и согласно (7), существует $\liml{d(T)\ra0} \si(F;T_{\ze})=I$.
Последнее, по определению 1, равносильно утверждению (4) теоремы.
\end{proof}

Доказанная теорема позволяет нам ввести определение 2, эквивалентное
эквивалентное определению 1 для непрерывных функций.

\begin{dfn}
2 Пусть гладкая поверхность $\Ph\subset\R^3\cln Oxyz$ задана
параметризацией $$\ol{r}=\ol{r}(u,v)=(x(u,v),y(u,v),z(u,v))$$ класса
$\Cc^1$ на квадрируемом компакте $[D]$, $D$ --- область в $\R^2\cln
uOv$. Если на поверхности $\Ph$ задана непрерывная функций
$F=F(x,y,z)$, так что непрерывна функция $(F\circ\ol{r})(u,v) =
F(x(u,v),y(u,v),z(u,v))$, $(u,v)\in[D]$, то поверхностный интеграл
$\iints{\Ph}F\,ds$ определяется формулой
\equ{\iints{\Ph}F(x,y,z)\,ds = \iints{[D]} F(x(u,v),y(u,v),z(u,v))
\sqrt{EG-F^2}(u,v)\,du\,dv.}
\end{dfn}

\subsubsection{Поверхностный интеграл второго рода}

Пусть $\ol{i},\ol{j},\ol{k}$ --- единичные координатные векторы
правой декартовой системы координат в $\R^3\cln Oxyz$. Для гладкой
поверхности $\Ph\subset\R^3$ с векторной параметризацией
$\ol{r}=\ol{r}(u,v)$, $(u,v)\in[D]\subset\R^2$ класса $\Cc^1$, где
компакт $[D]$ квадрируем, рассмотрим её положительную ориентацию
$\Ph^+$, определяемую нормалью $\ol{n}=\ol{r}_u\times\ol{r}_v$.
Тогда $\ol{r}(u,v)=(x(u,v),y(u,v),z(u,v))$, $(u,v)\in[D]$, и
\equ{\ol{n} = \ol{r}_u\times\ol{r}_v = \mbmat{\ol{i} & \ol{j} &
\ol{k} \\ x'_u & y'_u & z'_u \\ x'_v & y'_v & z'_v } =
\frac{D(y,z)}{D(u,v)} \ol{i} + \frac{D(z,x)}{D(u,v)} \ol{j} +
\frac{D(x,y)}{D(u,v)} \ol{k}.}

Рассмотрим единичный нормальный вектор $\ol{\nu} =
\frac{\ol{n}}{\hm{\ol{n}}}$ и его углы $\al,\be,\ga$ с ортами
$\ol{i},\ol{j},\ol{k}$, соответственно, так что
$\ol{\nu}=(\cos\al,\cos\be,\cos\ga)$.

Пусть на поверхности $\Ph$ задана векторная функция $\ol{a} =
(\ol{a}\circ\ol{r})(u,v)=(P,Q,R)$ и $P=P(x(u,v),y(u,v),z(u,v))$,
$Q=Q(x(u,v),y(u,v),z(u,v))$, $R=R(x(u,v),y(u,v),z(u,v))$.

\begin{dfn}
3 Поверхностным интегралом $\iints{\Ph^+}\ol{a}\,ds$ второго рода
вектора $\ol{a}$ по ориентируемой поверхности $\Ph^+$ называют
интеграл \eqa{8}{\iints{\Ph^+} \ol{a}\,ds=\iints{\Ph}
(\ol{a}\cdot\ol{\nu})\,ds,} где $(\ol{a}\cdot\ol{\nu})$ ---
скалярное произведение векторов $\ol{a}$ и $\ol{\nu}$.
\end{dfn}

Формулу (8) называют векторной записью поверхностного интеграла
второго рода. Его координатной записью называют формулу
\eqa{9}{\iints{\Ph^+}\ol{a}\,ds = \iints{\Ph^+}P\,dy\,dz + Q\,dz\,dz
+ R\,dx\,dy.}

Запись (9) представляет собой формулу интегрирования по
ориентируемой поверхности дифференциальной 2--формы в $\R^3$,
получающейся из дифференциальной 1--формы посредством операции
внешнего дифференцирования (но только в случае $\R^3$).
Интегрирование дифференциальных форм более высокого порядка, чем 1,
относят к общим объектам изучения в курсах многомерного
математического анализа и дифференциальной геометрии. На отделении
механики этот материал излагается в параллельно читаемом курсе
дифференциальной геометрии (чем, кстати, объясняется уменьшение
числа академических часов в курсе математического анализа).

Укажем способ вычисления интегралов (8) и (9). Согласно определениям
(8) и (9), \eqa{10}{\iints{\Ph^+} P\,dy\,dz+Q\,dz\,dx+R\,dx\,dy =
\iints{\Ph^+} \ol{a}\,ds = \iints{\Ph} (\ol{a}\cdot\ol{\nu})\,ds =
\iints{\Ph} (P\cos\al+Q\cos\be+R\cos\ga)\,ds.}

Так как \equ{\ol{\nu} =
\frac{\ol{r}_u\times\ol{r}_v}{\hm{\ol{r}_u\times\ol{r}_v}} =
\frac1{\sqrt{EG-F^2}} \mbmat{\ol{i} & \ol{j} & \ol{k} \\ x'_u & y'_u
& z'_u \\ x'_v & y'_v & z'_v } = \frac1{\sqrt{EG-F^2}}
\hs{\mbmat{y'_u & z'_u \\ y'_v & z'_v}\ol{i} + \mbmat{z'_u & x'_u \\
z'_v & x'_v}\ol{j} + \mbmat{x'_u & y'_u \\ x'_v & y'_v} \ol{k}},} то
\equ{\cos\al = \ol{\nu}\cdot\ol{i} = \frac1{\sqrt{EG-F^2}}
\mbmat{y'_u & z'_u \\ y'_v & z'_v},}
\equ{\cos\be=\ol{\nu}\cdot\ol{j} = \frac1{\sqrt{EG-F^2}} \mbmat{z'_u
& x'_u \\ z'_v & x'_v},} \equ{\cos\ga=\ol{\nu}\cdot\ol{k} =
\frac1{\sqrt{EG-F^2}} \mbmat{x'_u & y'_u \\ x'_v & y'_v},} и значит,
с учётом (10) и (4),

\ml{\iints{\Ph^+}\ol{a}\,ds = \iints{[D]} \left[
P(x(u,v),y(u,v),z(u,v)) \frac1{\sqrt{EG-F^2}} \mbmat{y'_u & z'_u
\\ y'_v & z'_v} + Q(x(u,v),y(u,v),z(u,v)) \frac1{\sqrt{EG-F^2}}
\mbmat{z'_u & x'_u \\ z'_v & x'_v} \right.+
\\+\left.R(x(u,v),y(u,v),z(u,v)) \frac1{\sqrt{EG-F^2}} \mbmat{x'_u & y'_u
\\ x'_v & y'_v} \right] \sqrt{EG-F^2}\,du\,dv = \iints{[D]} \mbmat{ P & Q & R \\
\frac{\pd x}{\pd u} & \frac{\pd y}{\pd u} & \frac{\pd z}{\pd u} \\
\frac{\pd x}{\pd v} & \frac{\pd y}{\pd v} & \frac{\pd z}{\pd v}}
\,du\,dv = \iints{[D]} (\ol{a},\ol{r}_u,\ol{r}_v)\,du\,dv,} где
$(\ol{a},\ol{r}_u,\ol{r}_v)$
--- смешанное произведение векторов $\ol{a}$, $\ol{r}_u$,
$\ol{r}_v$.

Если $Q=R=0$, $(u,v)\in[D]$, получаем поверхностный интеграл
$\iints{\Ph^+} P(x,y,z)\,dy\,dz$. Аналогично определяются интегралы
$\iints{\Ph^+} Q(x,y,z)\,dz\,dx$ и $\iints{\Ph^+} R(x,y,z)\,dx\,dy.$

Рассмотрим множество $\Ps$ всех размеченных разбиений компакта $[D]$
и базу $d(T)\ra0$ на $\Ps$. Для произвольного $T_{\ze}\in\Ps$
рассмотрим интегральную сумму \equ{\si_1(P;T_{\ze}) = \sumkun
P(x(\xi_k,\eta_k),y(\xi_k,\eta_k),z(\xi_k,\eta_k)) \cos\al_k \cdot
пл.\,\Ph_k.} Можно доказать, что $$\iints{\Ph^+} P(x,y,z)\,dy\,dz =
\liml{d(T)\ra0} \si_1(P;T_{\ze}).$$

Аналогично можно получить записи двух других интегралов.

\subsubsection{Поверхностные интегралы по кусочно--гладким
поверхностям}

Пусть поверхность $\Ph$ склеена из $m$ гладких поверхностей
$\Ph_1\sco\Ph_m$ и пусть $\Ph^+$ --- та ориентация поверхности
$\Ph$, которая получается склеиванием ориентируемых поверхностей
$\Ph_1^+\sco\Ph_m^+$.

Пусть на $\Ph$ заданы непрерывна функция $F(M)$, $M\in\Ph$, и
непрерывна векторная функция $\ol{a}=(P,Q,R)$. По определению,
$$\iints{\Ph} F\,ds = \sumkum \iints{\Ph_k} F\,ds \mbox{ и }
\iints{\Ph^+} \ol{a}\,ds = \sumkum \iints{\Ph_k^+}\ol{a}\,ds.$$

\section{Основные операции теории поля}
\subsection{Инварианты линейного оператора}
\subsubsection{Взаимные базисы векторов в $\R^3$}

В этом и двух последующих пунктах мы напоминаем известный материал
из курса линейной алгебры.

Обозначим символом $\Mg$ множество всех векторов в $\R^3$. Векторы
$\ol{r}_i$, $i=1,2,3$, образуют базис в $\R^3$, если
$(\ol{r}_1\ol{r}_2\ol{r}_3)\ne0$. Базис $\ol{r}^k$, $k=1,2,3$,
называют взаимным к базису $\ol{r}_i$, $i=1,2,3$, если $$\ol{r}_i
\ol{r}^k = \de^k_i = \case{1,i=k \\ 0, i\ne k.}$$

\begin{theorem}
Для произвольного базиса $\ol{r}_i$, $i=1,2,3$, существуют
единственный взаимный базис $\ol{r}^k$, $k=1,2,3$, и \equ{\ol{r}^1 =
\frac{\ol{r}_2\times\ol{r}_3}{(\ol{r}_1\ol{r}_2\ol{r}_3)}, \;
\ol{r}^2 =
\frac{\ol{r}_3\times\ol{r}_1}{(\ol{r}_1\ol{r}_2\ol{r}_3)}, \;
\ol{r}^3 =
\frac{\ol{r}_1\times\ol{r}_2}{(\ol{r}_1\ol{r}_2\ol{r}_3)}.}
\end{theorem}

Рассмотрим матрицы $\hn{g_{ki}}$ и $\hn{g^{ki}}$, где $g_{ki} =
\ol{r}_k\ol{r}_i$, $g^{ki} = \ol{r}^k\ol{r}^i$. Матрицы взаимно
обратны и симметричны (из-за симметрии скалярного произведения).
Кроме того \eqa{1}{\mat{\ol{r}^k = g^{k1}\ol{r}_1 + g^{k2}\ol{r}_2 +
g^{k3} \ol{r}_3 = g^{ki}\ol{r}_i, \; k=1,2,3 \\ \ol{r}_k = g_{k1}
\ol{r}^1 + g_{k2} \ol{r}^2 + g_{k3} \ol{r}^3 = g_{ki} \ol{r}^i, \;
k=1,2,3.}}

\subsubsection{Преобразование базисов в $\R^3$}

Рассмотрим взаимные базисы $\ol{r}_i$, $\ol{r}^i$, $i=1,2,3$, и
$\ol{r}_{i'}$, $\ol{r}^{i'}$, $i'=1,2,3$ и формулы перехода от
старого базиса $\ol{r}_i$ к новому базису $\ol{r}_{i'}$, и обратно:
\eqa{2}{\ol{r}_{i'} = b^i_{i'} \ol{r}_i, \; \ol{r}_i = b^{i'}_i
\ol{r}_{i'}.}

Тогда матрицы $\hn{b^i_{i'}}$ и $\hn{b^{i'}_i}$ взаимно обратные.

Рассмотрим формулы перехода для взаимных базисов:
\eqa{3}{\ol{r}^{i'} = \tilde b^{i'}_i \ol{r}^i, \; \ol{r}^i = \tilde
b^i_{i'} \ol{r}^{i'}} и матрицы $\hn{\tilde b^{i'}_i}$, $\hn{\tilde
b^i_{i'}}$ взаимно обратные.

\begin{theorem}
$$\hn{b^i_{i'}}=\hn{\tilde b^i_{i'}} \mbox{ и } \hn{b^{i'}_i}=\hn{\tilde
b^{i'}_i}.$$
\end{theorem}

На основании этой теоремы и формул (2), (3), заключаем, что
\eqa{4}{\mat{\ol{r}_{i'} = b^i_{i'} \ol{r}_i, \; \ol{r}_i = b^{i'}_i
\ol{r}_{i'}, \\ \ol{r}^{i'} = b^{i'}_i \ol{r}^i, \; \ol{r}^i=
b^i_{i'}\ol{r}^{i'},}} и матрицы $\hn{b^i_{i'}}$, $\hn{b^{i'}_i}$
взаимно обратные.

\subsubsection{Преобразование координат}

Пусть $\ol{r}_i$, $\ol{r}^i$ --- взаимные базисы и $\ol{x}\in\Mg$.
Тогда \eqa{5}{\mat{\ol{x} = x_1\ol{r}^1 + x_2\ol{r}^2+x_3\ol{r}^3 =
x_i\ol{r}^i, \\ \ol{x} = x^1 \ol{r}_1 + x^2\ol{r}_2+x^3\ol{r}_3=
x^i\ol{r}_i.}}

Числа $(x_1,x_2,x_3)$ называют ковариантными координатами вектора
$\ol{x}$, а числа $(x^1,x^2,x^3)$ --- контравариантными координатами
вектора $\ol{x}$. По определению и свойству линейности скалярного
произведения, \eqa{6}{\mat{\ol{x} \cdot \ol{r}_k = x_i \ol{r}^i
\ol{r}_k = x_i \de^i_k=x_k, \; k=1,2,3, \\ \ol{x} \cdot \ol{r}^k =
x^i \ol{r}_i \ol{r}^k = x^i \de^k_i= x^k, \; k=1,2,3,}} и значит,
формулы (5) принимают вид \eqa{формулы \;Гиббса}{\ol{x}=(\ol{x}\cdot
\ol{r}_i)\ol{r}^i, \; \ol{x}=(\ol{x} \cdot \ol{r}^i)\ol{r}_i.}

Рассмотрим новые взаимные базисы $\ol{r}_{i'}$, $\ol{r}^{i'}$,
связанные со старыми базисами формулами (4). Тогда, в силу (4) и (6)
и свойства линейности скалярного произведения, \equ{x_{i'} =
\ol{x}\cdot\ol{r}_{i'} = b^i_{i'} (\ol{x}\cdot\ol{r}_i) = b^i_{i'}
x_i, \; i'=1,2,3} (то есть, ковариантные координаты вектора
изменяются с той же матрицей $\hn{b^i_{i'}}$, что и базисы), а
\equ{x^{i'} = \ol{x} \cdot \ol{r}^{i'} = b^{i'}_i (\ol{x} \cdot
\ol{r}^i)=b^{i'}_i x^i, \; i'=1,2,3} (то есть, контравариантные
координаты вектора изменяются с обратной матрицей по отношению к
базисам).

\subsubsection{Дивергенция линейного оператора}

Рассмотрим произвольный линейный оператор $A\cln \Mg\ra\Mg$; то
есть, такое отображение $A$ из $\Mg$ в $\Mg$, что \equ{A(\la_1\ol{x}
+ \la_2\ol{y}) = \la_1 A\ol{x} + \la_2 A\ol{y}} для любых
$\la_1,\la_2\in\R$ и $\ol{x},\ol{y}\in\Mg$.

\begin{theorem}
\label{thm141'}
$$\ol{r}^i\cdot A\ol{r}_i = \ol{r}_i\cdot A\ol{r}^i$$ для любого
линейного оператора $A\cln\Mg\ra\Mg$ и любых взаимных базисов
$\ol{r}_i$, $\ol{r}^i$, $i=1,2,3$.
\end{theorem}

\begin{proof}
По определению, $\ol{r}^i A\ol{r}_i = \ol{r}^1 A\ol{r}_1 + \ol{r}^2
A \ol{r}_2 + \ol{r}^3 A\ol{r}_3$. Аналогично для $\ol{r}_i\cdot
A\ol{r}^i$. Согласно формулам (1), $$\ol{r}^k=g^{ki}\ol{r}_i, \;
\ol{r}_k=g_{kj} \ol{r}^j.$$ Поэтому, в силу свойств линейности
оператора $A$ и скалярного произведения, имеем \equ{\ol{r}^k
A\ol{r}_k = g^{ki} \ol{r}_i A(g_{kj} \ol{r}^j) = g^{ki} g_{kj}
\ol{r}_i A r^j = \de^i_j \ol{r}_i A\ol{r}^j = \ol{r}_j A\ol{r}^j =
\ol{r}_k A\ol{r}^k,} где использовано также утверждение, что матрицы
$\hn{g^{ki}}$ и $\hn{g_{kj}}$ взаимно обратные.
\end{proof}

\begin{theorem}
\label{thm142'} (Инвариантность относительно преобразования базисов
и координат векторов). $$\ol{r}^i A \ol{r}_i = \ol{r}^{i'} A
\ol{r}_{i'}$$ для любого линейного оператора $A\cln\Mg\ra\Mg$ и для
любых взаимных базисов $\ol{r}_i$, $\ol{r}^i$ и $\ol{r}_{i'}$,
$\ol{r}^{i'}$.
\end{theorem}

\begin{proof}
Согласно формулам (4), $\ol{r}_i=b^{i'}_i \ol{r}_{i'}$, $\ol{r}^i=
b^i_{k'} \ol{r}^{k'}$, и следовательно, как и в доказательстве
теоремы \ref{thm141'}, \equ{\ol{r}^i A\ol{r}_i = b^i_{k'}
\ol{r}^{k'} A(b^{i'}_i \ol{r}_{i'})= b^i_{k'} b^{i'}_i \ol{r}^{k'} A
\ol{r}_{i'} = \de^{i'}_{k'} \ol{r}^{k'} A \ol{r}_{i'} = \ol{r}^{i'}
A \ol{r}_{i'}.}
\end{proof}

\begin{dfn}
1 Сумма $$\ol{r}^i \cdot A \ol{r}_i = \ol{r}^1 A \ol{r}_1 + \ol{r}^2
A \ol{r}_2 + \ol{r}^3 A \ol{r}_3$$ называется дивергенцией линейного
оператора $A\cln\Mg\ra\Mg$ и обозначается $\Div A = \ol{r}^i A
\ol{r}_i = \ol{r}_i A \ol{r}^i$ (теорема \ref{thm141'}).

Характеристика $\Div A$ не зависит от выбора базиса (теорема
\ref{thm142'}); то есть, есть инвариант относительно преобразования
базисов и координат векторов.
\end{dfn}

\subsubsection{Ротор линейного оператора}

\begin{theorem}
\label{thm153'} $$\ol{r}^i\times A \ol{r}_i = \ol{r}_i \times
A\ol{r}^i$$ для любого линейного оператора $A\cln\Mg\ra\Mg$ и любых
взаимных базисов $\ol{r}_i$, $\ol{r}^i$, $i=1,2,3$.
\end{theorem}

\begin{theorem}
\label{thm154'} $$\ol{r}^i\times A\ol{r}_i = \ol{r}^{i'}\times
A\ol{r}_{i'}$$ для любых взаимных базисов $\ol{r}_i$, $\ol{r}^i$,
$i=1,2,3,$ и $\ol{r}_{i'}$, $\ol{r}^{i'}$, $i'=1,2,3$.
\end{theorem}

Доказательства теорем \ref{thm153'} и \ref{thm154'} дословно
повторяют доказательства теорем \ref{thm141'} и \ref{thm142'},
соответственно, с единственной заменой знака скалярного произведения
$(\cdot)$ на знак $(\times)$ векторного произведения.

\begin{dfn}
2 Вектор $$\ol{r}^i\times A\ol{r}_i = \ol{r}^1 \times A r_1 +
\ol{r}^2 \times A \ol{r}_2 + \ol{r}^3 \times A \ol{r}_3$$ называется
ротором линейного оператора $A\cln\Mg\ra\Mg$ и имеет обозначение
$$\rot A = \ol{r}^i\times A\ol{r}_i = \ol{r}_i \times A\ol{r}^i.$$
\end{dfn}

Характеристика $\rot A$ является инвариантом относительно
преобразований координат и векторов в $\R^3$.

\subsubsection{Координатная запись $\Div A$ и $\rot A$}

Рассмотрим ортонормированный базис $\ol{i},\ol{j},\ol{k}$ в $\R^3$.
Ему взаимный взаимный базис --- также $\ol{i},\ol{j},\ol{k}$.
Линейный оператор $A\cln\Mg\ra\Mg$ можно задать в виде матрицы
$\os{\circ}A$ следующим образом:

\equ{\rbmat{\ol{i} A\ol{i} = a_{11} & \ol{i} A \ol{j} = a_{12} &
\ol{i} A \ol{k} =a_{13} \\ \ol{j} A \ol{i} = a_{21} & \ol{j} A
\ol{j} = a_{22} & \ol{j} A \ol{k} = a_{23} \\ \ol{k} A \ol{i} =
a_{31} & \ol{k} A \ol{j}= a_{32} & \ol{k} A \ol{k} = a_{33}} =
\os{\circ}A.}

Тогда $\Div A = \ol{r}^i A \ol{r}_i = \ol{i}A\ol{i} + \ol{j}A\ol{j}
+ \ol{k} A \ol{k} = a_{11}+a_{22} + a_{33}$ --- след матрицы
$\os{\circ}A$. Ротор $\rot A=\ol{r}^i\times A\ol{r}_i=\ol{i} \times
A \ol{i} + \ol{j}\times A\ol{j} + \ol{k}\times A\ol{k}.$ При этом,

\equ{\ol{i}\times A\ol{i} = \ol{i}\times( a_{11}\ol{i} + a_{21}
\ol{j} + a_{31} \ol{k}) = a_{21}(\ol{i}\times\ol{j}) + a_{31}
(\ol{i}\times\ol{k}) = a_{21}\ol{k}-a_{31}\ol{j}} \equ{\ol{j}\times
A\ol{j} = \ol{j}\times (a_{12} \ol{i} + a_{22}\ol{j} + a_{32}\ol{k})
= a_{12} (\ol{j}\times\ol{i}) + a_{32} (\ol{j}\times\ol{k}) =
-a_{12}\ol{k}+a_{32}\ol{i} = a_{32}\ol{i}-a_{12}\ol{k}}
\equ{\ol{k}\times A\ol{k} = \ol{k}\times (a_{13}\ol{i} + a_{23}
\ol{j} + a_{33}\ol{k}) = a_{13} (\ol{k}\times\ol{i}) + a_{23}
(\ol{k}\times\ol{j}) = a_{13}\ol{j} - a_{23}\ol{i}.}

Поэтому \equ{\rot A = (a_{32}-a_{23})\ol{i} + (a_{13}-a_{31})\ol{j}
+ (a_{21}-a_{12})\ol{k}.}

\subsection{Характеристики дифференцируемых полей}
\subsubsection{Дифференцируемые скалярные поля}

Пусть $D$ --- непустое открытое множество в $\R^3$. Отображение
(функцию) $u\cln D\ra\R$ назовём скалярным полем, определённым в
$D$, и используем обозначение $u(M)$, $M\in D$, где $M$ ---
произвольная точка множества $D$. Отображение $\ol{p}\cln D\ra\Mg$
называют векторным полем, определённым в $D$. Отображение
$\ol{p}\cln D\ra\Mg$ называют векторным полем, определённым в $D$, и
обозначают $\ol{p}(M)$, $M\in D$.

Для произвольной фиксированной точки $M\in D$ обозначим через
$\ol{\De r}$ вектор $\ol{MM'}=\ol{\De r}$, $M'\in D$. Таким образом,
каждой точке $M\in D$ сопоставлено векторное поле $\ol{\De r}$.
Фиксированный вектор $g\in\Mg$ порождается линейную форму
$L_g(\ol{\De r}) = \ol{g}\cdot\ol{\De r}$. Это определение полностью
согласуется с соответствующим понятием в случае, когда на $\R^3$
задана некоторая декартова система координат $(x,y,z)$, в которой
$\ol{\De r}=(\De x,\De y,\De z)$ и $\ol{g}=(A,B,C)$; тогда
$\ol{g}\cdot\ol{\De r} = A\De x+B\De y+C\De z$ --- линейная функция
в $\R^3$ относительно $(\De x,\De y,\De z)$.

\begin{dfn}
1 Скалярное поле $u(M)$ называют дифференцируемым во внутренней
точке $M\in D$, если существует такой вектор $\ol{g}\in\Mg$ (выбор
которого зависит только от $M$), что в некоторой окрестности
$\Uc(M)\subset D$ для любой точки $M'\in\Uc(M)$ приращение $\De u =
u(M')-u(M)$ можно представить в виде \eqa{1}{\De u =
\ol{g}\cdot\ol{\De r} + o(\rho), \; \rho\ra+0,} где
$\rho=\hm{\ol{\De r}}$ --- длина вектора $\ol{\De r}$.
\end{dfn}

\begin{stm}
Если представление (1) существует, то оно единственное.
\end{stm}

\begin{proof}
Допустим, что для двух векторов $\ol{g},\ol{h}\in\Mg$ справедливы
представления \equ{\De u = \ol{g}\cdot\ol{\De r}+o(\rho), \; \De u =
\ol{h}\ol{\De r} + o(\rho), \; \rho\ra+0.}

Тогда \eqa{2}{(\ol{g}-\ol{h})\ol{\De r} \cdot \ol{\De r} =
o(\rho)\mbox{ и } (\ol{g}-\ol{h})\ol{e} = \frac{o(\rho)}{\hm{\ol{\De
r}}} = \frac{o(\rho)}{\rho}, \; \rho\ra+0,} где $\ol{e}$ ---
единичный вектор вектора $\ol{\De r}$. Переходя в (2) к пределу при
$\rho\ra+0$, получим $(\ol{g}-\ol{h})\ol{e}=0$ для любого единичного
вектора $\ol{e}$. Следовательно, $\ol{g}-\ol{h}=0$ и
$\ol{g}=\ol{h}$.
\end{proof}

Скалярное поле называют дифференцируемым на открытом множестве
$D\subset\R^3$, если оно дифференцируемо в каждой его точке.

\begin{dfn}
2 Вектор $\ol{g}=\ol{g}(M)$ в разложении (2) называют градиентом
дифференцируемого скалярного поля $u(M)$ и обозначают $\grad u$.
\end{dfn}

Введённое понятие не зависит от выбора системы координат в $\R^3$ и
$\grad u$ --- векторное поле в $D$, если скалярное поле $u(M)$
дифференцируемо в открытом множестве $D$.

На основании определения (1) несложно доказать свойства градиента:
\equ{\mat{\grad(u\pm v)=\grad u\pm\grad v; \; \grad(uv)=u\grad v +
v\grad u; \\ \grad c=\ol{0}; \; \grad cu=c\grad u, \;
\grad\hr{\frac{u}v}=\frac{v\grad u-u\grad v}{v^2}, \; v\ne0.}}

\subsubsection{Производная по направлению скалярного поля}

Рассмотрим скалярное поле $u(M)$, определённое на открытом множестве
$D\subset\R^3$. Фиксируем точку $M\in D$ и окрестность
$\Uc(M)\subset D$. Фиксируем единичный вектор $\ol{e}\in\Mg$.
Рассмотрим такую произвольную точку $M'\in\Uc(M)$, для которой
вектор $MM'$ коллинеарен вектору $\ol{e}$, так что $\ol{MM'}=\ol{\De
r}=t\ol{e}$ и $\hm{t}=\hm{\ol{MM'}}=\hm{\ol{\De r}}$. Обозначим
$\De_e u = u(M')-u(M)$.

\begin{dfn}
3 Если существует $\liml{t\ra0}\frac{\De_e u}{t}=l$, то число $l$
называют производной скалярного поля $u$ в точке $M$ по направлению
$\ol{e}$ и обозначают символом $\frac{\pd u}{\pd \ol{e}}$. Итак,
$\frac{\pd u}{\pd \ol{e}}(M) = \liml{t\ra0} \frac{\De_e u}{t}$,
$MM'=t\ol{e}$.
\end{dfn}

\begin{stm}
Если скалярное поле $u(M)$ дифференцируемо в точке $M\in D$, то
$\frac{\pd u}{\pd \ol{e}}=\grad u\cdot\ol{e}$ для любого направления
$\ol{e}$.
\end{stm}

\begin{proof}
Согласно определению 1, $$\De u = \grad u\cdot \ol{\De r} + o(\rho),
\; \rho\ra+0.$$ Согласно определению 3, $\ol{\De r}=t\ol{e}$.
Поэтому, $\De_e u = (\grad u\cdot\ol{e})t + o(\hm{t})$, $t\ra 0$, и
$$\liml{t\ra+0} \frac{\De_e u}{t} = \liml{t\ra0} \hs{(\grad
u \cdot \ol{e}) + \frac{o(\hm{t})}{t}} = \grad u \cdot \ol{e};$$ то
есть, $\frac{\pd u}{\pd \ol{e}} = \grad u \cdot \ol{e}$.
\end{proof}

\subsubsection{Выражение для градиента в декартовой системе
координат}

Рассмотрим базис $\ol{i},\ol{j},\ol{k}$, порождающий правую систему
координат в $\R^3\cln Oxyz$. Тогда $u=u(M)=u(x,y,z)$ и $$\grad u =
\ol{i} (\grad u \cdot \ol{i}) + \ol{j} (\grad u \cdot \ol{j}) +
\ol{k} (\grad u\cdot\ol{k}),$$ так что $$\frac{\pd u}{\pd \ol{i}} =
\frac{\pd u}{\pd x}, \; \frac{\pd u}{\pd \ol{j}} = \frac{\pd u}{\pd
y}, \; \frac{\pd u}{\pd \ol{k}}=\frac{\pd u}{\pd z}.$$ Поэтому
$$\grad u = \frac{\pd u}{\pd x} \ol{i} + \frac{\pd u}{\pd y} \ol{j}
+ \frac{\pd u}{\pd z} \ol{k}.$$

\subsubsection{Дифференцируемые векторные поля}

\begin{dfn}
4 Векторное поле $\ol{p}(M)$ называется дифференцируемым во
внутренней точке $M$ открытого множества $D\subset\R^3$, если
существует такая окрестность $\Uc(M)\subset D$ и такой линейный
оператор $A\cln\Mg\ra\Mg$, что для любой точки $M'\in\Uc(M)$
справедливо представление \eqa{3}{\ol{\De p} = A\ol{\De r} + \ol{o}
\hr{\hm{\ol{\De r}}}, \; \hm{\ol{\De r}} = \rho\ra+0,} в котором
$\ol{\De p} = \ol{p}(M') - \ol{p}(M)$, $\ol{\De r}=MM'$, и
$\liml{\rho\ra+0} \frac{\ol{o}(\rho)}{\rho}=\ol{0}.$
\end{dfn}

Выбор линейного оператора $A$ зависит только от точки $M\in D$.

\begin{stm}
Если представление (3) существует, то оно единственное.
\end{stm}

\begin{proof}
Пусть $\ol{\De p}=A\ol{\De r}+\ol{o}\hr{\hm{\ol{\De r}}}$ и $\ol{\De
p}=B\ol{\De r}+\ol{o}\hr{\hm{\ol{\De r}}}$, $\hm{\ol{\De r}}\ra0$.
Тогда $(A-B)\ol{\De r}=\ol{o}\hr{\hm{\ol{\De r}}}$, $\hm{\ol{\De
r}}\ra0$. Если $\ol{\De r}=\rho\ol{e}$, $\rho = \hm{\ol{\De r}}$, то
$(A-B)\ol{e} = \frac{\ol{o}(\rho)}{\rho}$, $\rho\ra0$, и после
перехода к пределу при $\rho\ra0$, получим, что $(A-B)\ol{e}=\ol{0}$
для любого единичного вектора $\ol{e}\in\Mg$. Значит, $A=B$.
\end{proof}

Если векторное $\ol{p}(M)$ дифференцируемо в каждой точке открытого
$D$, то $\ol{p}(M)$ называют дифференцируемым в $D$.

\begin{dfn}
5 Положим $\Div A=\Div \ol{p}$ и $\rot A=\rot \ol{p}$.
\end{dfn}

\subsubsection{Производная векторного поля по направлению}

Пусть векторное поле $\ol{p}(M)$ по определено на открытом множестве
$D\subset\R^3$ и точка $M\in D$ --- произвольная. Пусть
$\ol{e}\in\Mg$ --- произвольный единичный вектор. В окрестности
$\Uc(M)\subset D$ рассмотрим такие точки $M'\in\Uc(M)$, для которых
вектор $\ol{MM'}$ коллинеарен вектору $\ol{e}$, так что
$\ol{MM'}=t\ol{e}$, и обозначим $\De\ol{p}_{\ol{e}}(M) = \ol{p}(M')
- \ol{p}(M)$, $\hm{MM'}=\hm{t}$.

\begin{dfn}
6 Если существует $\liml{t\ra0} \frac{\De
\ol{p}_{\ol{e}}(M)}{t}=\ol{l}$, то вектор $\ol{l}$ называют
производной векторного поля $\ol{p}(M)$ в точке $M\in D$ по
направлению $\ol{e}$. Обозначение: $\ol{l}=\frac{\pd
\ol{p}}{\pd\ol{e}}$.
\end{dfn}

\begin{theorem}
Если векторное поле $\ol{p}(M)$ дифференцируемо в точке $M\in D$, то
для любого единичного вектора $\ol{e}$ в точке $M$ существует
$\frac{\pd \ol{p}}{\pd \ol{e}}$ и $\frac{\pd \ol{p}}{\pd\ol{e}} =
A\ol{e}$, где линейный оператор $A$ определяется разложением (3).
\end{theorem}

\begin{proof}
По условию теоремы разложение (3) \equ{\De \ol{p} = A\ol{\De r} +
\ol{o}\hr{\hm{\ol{\De r}}}, \; \hm{\ol{\De r}} = \rho\ra0}
справедливо для любой точки $M'\in\Uc(M)$. Для тех точек
$M'\in\Uc(M)$, для которых $MM' = t\ol{e}$ имеем $\ol{\De r} =
t\ol{e}$ и \equ{\De \ol{p} = A(t\ol{e}) + \ol{o}(\hm{t}) = t A
\ol{e} + \ol{o}(\hm{t}), \; t\ra0,} откуда \equ{\liml{t\ra0}
\frac{\De \ol{p}}{t} = \liml{t\ra0} \hr{A\ol{e} +
\frac{\ol{o}(\hm{t})}{t}} = A\ol{e} \mbox{ и } \frac{\pd
\ol{p}}{\pd\ol{e}}=A\ol{e}.}
\end{proof}

\subsubsection{Инварианты векторного поля в декартовой системе
координат}

Рассмотрим в $\R^3$ декартову систему координат $(x,y,z)$,
определяемую правым ортонормированным базисом
$\ol{i},\ol{j},\ol{k}$. Пусть векторное поле $\ol{p}(M)$ имеет в
этом базисе координаты $\ol{p} = (P(x,y,z),Q(x,y,z),R(x,y,z))$.
Предположим, что $\ol{p}(M)$ дифференцируемо в точке $M(x,y,z)$
открытого множества $D$ в $\R^3$. Тогда \equ{\frac{\pd \ol{p}}{\pd
\ol{i}} = \frac{\pd \ol{p}}{\pd x}=A\ol{i}, \; \frac{\pd \ol{p}}{\pd
\ol{j}} = \frac{\pd \ol{p}}{\pd y}=A\ol{j}, \; \frac{\pd \ol{p}}{\pd
\ol{k}} = \frac{\pd \ol{p}}{\pd z}=A\ol{k},} и значит, матрица
$\os{\circ}A$ линейного оператора $A$ имеет вид
\equ{\rbmat{\frac{\pd P}{\pd x} & \frac{\pd P}{\pd y} & \frac{\pd
P}{\pd z} \\ \frac{\pd Q}{\pd x} & \frac{\pd Q}{\pd y} & \frac{\pd
Q}{\pd z} \\ \frac{\pd R}{\pd x} & \frac{\pd R}{\pd y} & \frac{\pd
R}{\pd z}} = \os{\circ}A.}

Поэтому $$\Div\ol{p}=\Div AA=\frac{\pd P}{\pd x} + \frac{\pd Q}{\pd
y} + \frac{\pd R}{\pd z}$$ и \equ{\rot\ol{p} = \rot A =
\hr{\frac{\pd R}{\pd y} - \frac{\pd Q}{\pd z}}\ol{i} + \hr{\frac{\pd
P}{\pd z} - \frac{\pd Q}{\pd x}}\ol{j} + \hr{\frac{\pd Q}{\pd x} -
\frac{\pd P}{\pd y}}\ol{k} = \mbmat{\ol{i} & \ol{j} & \ol{k} \\
\frac{\pd}{\pd x} & \frac{\pd}{\pd y} & \frac{\pd}{\pd z} \\ P & Q &
R} = \ol{\nabla}\times\ol{p},} где формальный вектор $\ol{\nabla} =
\hr{\frac{\pd}{\pd x},\frac{\pd}{\pd y},\frac{\pd}{\pd z}}$ читается
к вектор наблак.

\subsubsection{}

Непосредственно проверяется, что справедливы следующие утверждения.

\begin{theorem}
\equ{\rot\grad u=\ol{0}, \; \Div\grad u=\De u = \frac{\pd^2 u}{\pd
x^2} + \frac{\pd^2 u}{\pd y^2} + \frac{\pd^2 u}{\pd z^2}, \; \grad
\rot \ol{p}=\ol{0}, \; \Div\rot\ol{p}=0.}
\end{theorem}

Рассмотрим частный случай, когда $\ol{p}=(P,Q,0)$, $P=P(x,y)$,
$Q=Q(x,y)$. В этом случае, $\rot\ol{p}=\hr{\frac{\pd Q}{\pd x} -
\frac{\pd P}{\pd y}}\ol{k}.$

\subsection{Формула Грина}
\subsubsection{Формулировка теоремы и её частный случай}

Пусть $D$ --- ограниченная конечносвязная область на плоскости
$\Pi\cln Oxy$, граница $\pd d=\Ga$ которой состоит из $n+1$
замкнутых кривых $L_0,L_1\sco L_n$ класса $\Cc^1$, причём все
$\Lc_i$, $i=\ol{1,n}$, находятся внутри $L_0$ ($L_i\subset\Int L_0,
\;i=\ol{1,n}$) и для всех $i\ne j$, $i,j=\ol{1,n}$, кривая $L_i$
расположена во внешности кривой $L_j$ ($L_i\subset\Ext L_j,\; i\ne
j$).

%мб рисунок

\begin{theorem}
Если функции $P(x,y)$ и $Q(x,y)$ непрерывны на замыкании $[D]=D\cup
гр.\,D$, имеют непрерывные частные производные первого порядка в $D$
(то есть, $P,Q\in\Cc^1(D)$) и сходятся несобственные интеграл
\equ{\iints{D} \frac{\pd Q}{\pd x}\,dx\,dy, \; \iints{D} \frac{\pd
P}{\pd y}\,dx\,dy,} то несобственный интеграл \eqa{1}{\iints{D}
\hr{\frac{\pd Q}{\pd x} - \frac{\pd P}{\pd y}} \,dx\,dy =
\oints{\Ga^+} P\,dx+Q\,dy = \sumkun \ints{L_k^-}P\,dx+Q\,dy +
\ints{L_0^+} P\,dx+Q\,dy.}
\end{theorem}

\begin{proof}
Случай I: область $D$ односвязная и стандартная относительно обеих
координатных осей.

Это значит, что $\pd D=\Ga=L_0$ и $L_0=\La_1^x \cup \La_2^x$, где
$\La_1^x,\La_2^x$ --- графики некоторых непрерывно дифференцируемых
функций $y=\ph_1(x)$, $y=\ph_2(x)$, $x\in[a,b]$, $\ph_1(x)<\ph_2(x)$
для $x\in(a,b)$ и $\ph_1(x)=\ph_2(x)$ только при $x=a$ и $x=b$, а
также $L_0=\La_1^y\cup\La_2^y$, где $\La_1^y,\La_2^y$ --- графики
некоторых непрерывно дифференцируемых функций
$x=\psi_1(y),x=\psi_2(y)$, $y\in[c,d]$, $\psi_1(y)<\psi_2(y)$ для
$y\in(c,d)$ и $\psi_1(y)=\psi_2(y)$ только при $y=c$ и $y=d$.

Мы получим формулу (1), если докажем, что \eqa{2}{\iints{D}
\frac{\pd Q}{\pd x}\,dx\,dy= \ints{L_0^+} Q\,dy \mbox{ и } \iints{D}
\hr{-\frac{\pd P}{\pd y}}\,dx\,dy = \ints{L_0^+} P\,dx.}

Установим справедливость второй формулы в (2). По условию,
$L_0=\La_1^x\cup\La_2^x$. Кривые$\La_i^x$, $i=1,2$, спрямляемы и
поэтому $пл.\,\Ga=пл.\,L_0=0$, так что область $D$ квадрируема.
Построим некоторое (специального вида) исчерпывание области $D$
квадрируемыми компактами $\hc{D_n}$, $n\in\N$, $n\ge N=\frac2{b-a}$.
Для этого рассмотрим отрезки $[a_n,b_n]$, $a_n=a+\frac1n$,
$b_n=b-\frac1n$, $n>N$, и для фиксированного $n>N$ рассмотрим такое
число $\ep_n>0$, $\ep_n<\frac1n$, чтобы функции
$y=\ph_1^n(x)=\ph_1(x)+\ep_n$ и $y=\ph_2^n(x)=\ph_2(x)-\ep_n$,
$x\in[a_n,b_n]$, обладали свойством \equ{\ph_1(x)+\ep_n <
\ph_2(x)-\ep_n, \; x\in[a_n,b_n],} так что графики $\La_1^n,\La_2^n$
функций $y=\ph_1^n(x)$ и $\ph_2^n(x)$, $x\in[a_n,b_n]$, не
пересекались. Обозначим через $D_n$ замыкание области в $D$,
ограниченной кривыми $\La_1^n,\La_2^n$ и прямыми $x=a_n,x=b_n$.
Выбираем $\ep_{n+1}>0$, чтобы $\ep_{n+1}<\ep_n$ и
$\ep_{n+1}<\frac1{n+1}$. Тогда \equ{\ph_1(x)+\ep_{n+1} < \ph_1(x) +
\ep_n < \ph_2(x) -\ep_n < \ph_2(x) - \ep_{n+1}, \;
x\in[a_{n+1},b_{n+1}].}

Поэтому $D_n\subset ]D_{n+1}[$, $n\in\N$, и $\cupl{n=1}{\bes}D_n=D$.
Поскольку граница $\pd D_n$ имеет $пл.\,\pd D_n=0$, то все $D_n$ ---
квадрируемые компакты, исчерпывающие открытое множество $D$.

Согласно определению компактов $D_n$, \mla{3}{\iints{D_n} \frac{\pd
P}{\pd y}\,dx\,dy = \intl{a_n}{b_n}\,dx
\intl{\ph_1(x)+\ep_n}{\ph_2(x)-\ep_n} \frac{\pd P}{\pd y}\,dy =
\intl{a_n}{b_n} \hs{P(x,\ph_2(x)-\ep_n)-P(x,\ph_1(x)+\ep_n)}\,dx =
\intl{a_n}{b_n} P(x,\ph_2(x))\,dx -\\- \intl{a_n}{b_n}
P(x,\ph_1(x))\,dx + \intl{a_n}{b_n}
[P(x,\ph_2(x)-\ep_n)-P(x,\ph_2(x))]\,dx + \intl{a_n}{b_n}
[P(x,\ph_1(x)) - P(x,\ph_1(x)+\ep_n)]\,dx.}

Последние два интеграла в правой части (3) обозначим $I_1^n$ и
$I_2^n$, соответственно, и докажем, что $\liml{n\ra+\bes} I_1^n =
\liml{n\ra+\bes} I_2^n=0$.

Рассмотрим произвольное число $\ep>0$. Так как $P(x,y)$ непрерывна
на замыкании $[D]$, то она равномерно непрерывна на $[D]$, и значит,
существует такое $\de>0$, $\de=\de(\ep)$, что
$\hm{P(x',y')-P(x'',y'')} < \frac{\ep}{b-a}$ для всех $(x',y')$,
$(x'',y'')\in[D]$, у которых $\hm{x'-x''}<\de$, $\hm{y'-y''}<\de$.
Выберем $x'=x''=x$, $y'=\ph_2(x)$, $y''=\ph_2(x)-\ep_n$. Тогда
$\hm{x'-x''}=0<\de$ и $\hm{y'-y''} = \ep_n$. Так как
$\liml{n\ra+\bes}\ep_n=0$, то для $\de>0$ существует
$N_{\de}=N_{\ep}\in\N$, что $0<\ep_n<\de$ для всех $n>N_{\ep}$.

Следовательно, \eqa{4}{\hm{P(x,\ph_2(x)-\ep_n) - P(x,\ph_2(x))} <
\frac{\ep}{b-a}} для всех $n\ge N$, $n\ge N_{\ep}$. Значит, на
основании (4), \equ{\hm{I_1^n} = \hm{\intl{a_n}{b_n}
[P(x,\ph_2(x)-\ep_n)-P(x,\ph_2(x))]\,dx} \le
\frac{\ep}{b-a}(b_n-a_n)<\ep} для всех $n>\max(N,N_{\ep})$; то есть,
$\liml{n\ra+\bes}I_1^n=0$.

Аналогично, $\liml{n\ra+\bes}I_2^n=0$.

Учитывая эти утверждения, получаем из (3), что \ml{\iints{D}
\frac{\pd P}{\pd y}\,dx\,dy = \liml{n\ra+\bes} \iints{D_n} \frac{\pd
P}{\pd y}\,dx\,dy = \liml{n\ra+\bes} \hr{\intl{a_n}{b_n}
P(x,\ph_2(x))\,dx - \intl{a_n}{b_n} P(x,\ph_1(x))\,dx} =\\=
\intl{a}b P(x,\ph_2(x))\,dx - \intl{a}b P(x,\ph_1(x))\,dx =
-\ints{\Ga^+} P\,dx} и вторая формула в (2) доказана. Аналогично
доказывается и первая формула в (2). Таким образом, формула (1)
доказана в частном случае I.
\end{proof}

\subsubsection{Инвариантная запись формулы (1) в случае I}

Рассмотрим векторное поле $\ol{p} = (P(x,y),Q(x,y),0)$. Тогда
$\rot\ol{p}=\hr{\frac{\pd Q}{\pd x} - \frac{\pd P}{\pd y}}\ol{k}$ и
$$\iints{D} \hr{\frac{\pd Q}{\pd x} - \frac{\pd P}{\pd y}}\,dx\,dy =
\iints{D} \rot\ol{p}\cdot \ol{k}\,d\si.$$ Если обозначить через
$\ol{t}$ единичный вектор касательной к кривой $\Ga$, то
\equ{\ints{\Ga^+} P\,dx+Q\,dy = \ints{\Ga} (\ol{p}\cdot\ol{t})\,dl}
и формула (1) примет вид \eqa{5}{\iints{D}
\rot\ol{p}\cdot\ol{k}\,d\si = \ints{\Ga^+} (\ol{p}\cdot\ol{t})\,dl.}

\subsubsection{Случай II}

Область $D$ можно разбить с помощью конечного числа гладких кривых
$\La_k$ на конечно число областей $D_k$, $k=\ol{1,m}$, каждая из
которых удовлетворяет случаю $I$ в некоторой декартовой системе
координат, своей для каждой $D_k$, и $D=\cupl{k=1}m D_k$.

\begin{proof}
В каждой квадрируемой области $D_k$, $k=\ol{1,m}$, по предыдущему,
справедлива формула \eqa{6}{\iints{D_k} \hr{\frac{\pd Q}{\pd x} -
\frac{\pd P}{\pd y}}\,dx\,dy =
\iints{D_k}\rot\ol{p}\cdot\ol{nu}\,d\si = \ints{\pd D_k^+}
\hr{\ol{p}\cdot\ol{t}}\,dl =\ints{\pd D_k^+} P\,dx+Q\,dy,} где
$\ol{\nu}=\ol{k}$.

В силу свойства аддитивности кратных несобственных интегралов, имеем
\eqa{7}{\sumkum \iints{D_k} \hr{\frac{\pd Q}{\pd x}-\frac{\pd P}{\pd
y}}\,dx\,dy = \iints{D}\hr{\frac{\pd Q}{\pd x} - \frac{\pd P}{\pd
y}}\,dx\,dy.}

Каждая кривая $\Ga_k=\pd D_k$, $k=\ol{1,m}$, состоит из конечного
числа частей $\La_k^j$ кривых $\La_k$ и частей границы $\pd D=\Ga$.
При этом каждая $\La_k^j$ встречается в формулах (6) дважды для
соседних областей $D_k$ и $D_{k+1}$ и эти участки границ областей
$D_k$ и $D_{k+1}$ обходятся в противоположных направлениях. Поэтому,
используя теорему об интегрировании 1--формы по противоположным
путям и свойство аддитивности криволинейного интеграла первого рода,
\eqa{8}{\sumkum \ints{\pd D_k^+} P\,dx+Q\,dy =
\ints{\Ga^+}P\,dx+Q\,dy,} где $\Ga=\pd D$.

Объединяя формулы (7) и (8), получим \equ{\iints{D} \hr{\frac{\pd
Q}{\pd x} - \frac{\pd P}{\pd y}}\,dx\,dy = \ints{\Ga^+}P\,dx+Q\,dy}
и случай II доказан.
\end{proof}

Доказательство теоремы в общем случае изложено в учебнике В. А.
Ильин, Э.Г. Позняк. Основы математического анализа, часть II: М.,
изд. к Наукак, 1973, 447 с.

\subsection{Формула Стокса}
\subsubsection{Формулировка теоремы}

Рассмотрим произвольную кусочно--гладкую поверхность $\Ph$ в
$\R^3\cln Oxyz$, $\pd\Ph=\Ga$ которой состоит из $m$ замкнутых
гладких кривых $L_1\sco L_m$. Пусть $G$ --- область в $\R^3$,
содержащая $\Ph$ ($\Ph\subset G$). Пусть функции $P(x,y,z)$,
$Q(x,y,z)$, $R(x,y,z)$ непрерывно дифференцируемы в $G$. Эти функции
можно рассматривать как компоненты некоторого векторного поля
$\ol{p}=(P,Q,R)$, непрерывно дифференцируемого в $G$. Тогда
\equ{\rot\ol{p} = \mbmat{\ol{i}&\ol{j}&\ol{k} \\ \frac{\pd}{\pd x} &
\frac{\pd}{\pd y} & \frac{\pd}{\pd z} \\ P & Q & R} = \hr{\frac{\pd
R}{\pd y} - \frac{\pd Q}{\pd z}}\ol{i} + \hr{\frac{\pd P}{\pd z} -
\frac{\pd R}{\pd x}}\ol{j} + \hr{\frac{\pd Q}{\pd x}-\frac{\pd
P}{\pd y}}\ol{k}.}

\begin{thn}
{(формула Стокса)} \eqa{1}{\iints{\Ph}\rot\ol{p}\cdot\ol{\nu}\,ds =
\iints{\Ph^+} \hr{\frac{\pd R}{\pd y} - \frac{\pd Q}{\pd z}}\,dy\,dz
+ \hr{\frac{\pd P}{\pd z} - \frac{\pd R}{\pd x}}\,dz\,dx +
\hr{\frac{\pd Q}{\pd x}-\frac{\pd P}{\pd y}}\,dx\,dy = \ints{\Ga^+}
P\,dx+Q\,dy+R\,dz,} где \equ{\ints{\Ga^+} P\,dx+Q\,dy+R\,dz=\sumkun
\ints{L_k} P\,dx+Q\,dy+R\,dz} и направление на каждой $L_k$,
$k=\ol{1,m}$, выбрано согласованным с положительной ориентацией
$\Ph^+$ поверхности $\Ph$.
\end{thn}

\subsubsection{Случай I}
Поверхность $\Ph$ гладкая, край $\pd\Ph=\Ga$ поверхности $\Ph$
состоит из одной замкнутой гладкой кривой $L$ и поверхность $\Ph$
однозначно проектируется на все три координатные плоскости.

\begin{proof}
В этом случае положительная ориентация $\Ph^+$ поверхности $\Ph$
задаётся нормалью $\ol{n}=\ol{r}_u\times\ol{r}_v$, единичный вектор
$\ol{nu}$ которой имеет компоненты
$\ol{nu}=(\cos\al,\cos\be,\cos\ga)$. Формула (1) принимает вид
\eqa{1'}{\iints{\Ph}\hs{ \hr{\frac{\pd R}{\pd y} - \frac{\pd Q}{\pd
z}}\cos\al + \hr{\frac{\pd P}{\pd z} - \frac{\pd R}{\pd x}}\cos\be +
\hr{\frac{\pd Q}{\pd x}-\frac{\pd P}{\pd
y}}\cos\ga}\,ds=\ints{L^+}P\,dx+Q\,dy+R\,dz} и формула (1') будет
доказана, если будут доказаны формулы \eqa{2}{\iints{\Ph}
\hr{\frac{\pd P}{\pd z}\cos\be - \frac{\pd P}{\pd
y}\cos\ga}\,ds=\ints{L^+}P\,dx,} \eqa{3}{\iints{\Ph}\hr{\frac{\pd
Q}{\pd x}\cos\ga - \frac{\pd Q}{\pd z}\cos\al}\,dx =
\ints{L^+}Q\,dy,} \eqa{4}{\iints{\Ph}\hr{\frac{\pd R}{\pd y}\cos\al
- \frac{\pd R}{\pd x}\cos\be}\,ds=\ints{L^+}R\,dz.}

Поскольку в рассматриваемом случае все формулы однотипны, то
установим справедливость формулы (2).

По предположению, поверхность $\Ph$ является графиком некоторой
непрерывно дифференцируемой функции $z=f(x,y)$, определённой в
односвязной области $D=пр._{Oxy}\Ph$ и граница $\pd D=\La$ ---
замкнутая кривая, причём $\La=пр._{Oxy}\pd\Ph=пр._{Oxy}L$. Имеем
$$\ol{n}=\hr{-\frac{\pd f}{\pd x},-\frac{\pd f}{\pd y},1}$$ и
\equ{\cos\be = -\frac{f'_y}{\sqrt{1+f'^2_x+f'^2_y}}, \; \cos\ga =
\frac{1}{\sqrt{1+f'^2_x+f'^2_y}}.}

Следовательно, левая часть в (2) принимает вид \mla{5}{I =
\iints{\Ph} \hr{\frac{\pd P}{\pd z}\cos\be - \frac{\pd P}{\pd
y}\cos\ga}\,ds = \iints{\Ph} \hr{-\frac{\pd P}{\pd z}f'_y -
\frac{\pd P}{\pd y}}\cos\ga\,ds =\\= -\iints{D}\hr{\frac{\pd P}{\pd
y} + f'_y\frac{\pd P}{\pd z}}\,dx\,dy = -\iints{D} \frac{\pd}{\pd y}
\hs{P(x,y,f(x,y))}\,dx\,dy,} так как \equ{\frac{\pd}{\pd
y}P(x,y,f(x,y)) = \frac{\pd P}{\pd y} + \frac{\pd P}{\pd z}f'_y.}

Согласно формуле Грина и (5), имеем \equ{-\iints{D}\frac{\pd}{\pd y}
P(x,y,f(x,y))\,dx\,dy = \ints{\La^+}P(x,y,f(x,y))\,dx.}

Поскольку $\La=пр._{Oxy}L$ и положительное направление $\La^+$
соответствует положительному направлению $L^+$, то, по определению,
\eqa{7}{\ints{\La^+} P(x,y,f(x,y))\,dx=\ints{L^+}P(x,y,z)\,dx.}

Объединяя формулы (5), (6) и (7), получим формулу (2). Формулы (3) и
(4) доказываются аналогично.
\end{proof}

\subsubsection{Инвариантная запись формулы Стокса}

Рассмотрим на гладкой кривой $L=\pd\Ph$ единичный вектор касательной
$\ol{t} = (\cos\ta_1,\cos\ta_2,\cos\ta_3)$. Тогда
\equ{\ints{L^+}P\,dx+Q\,dy+R\,dz = \ints{L} \ol{p}\cdot\ol{t}\,dl,
\; \ol{p}=(P,Q,R),} и \equ{\iints{\Ph^+} \hr{\frac{\pd R}{\pd
y}-\frac{\pd Q}{\pd z}}\,dy\,dz + \hr{\frac{\pd P}{\pd z}-\frac{\pd
R}{\pd x}}\,dz\,dx + \hr{\frac{\pd \Ph}{\pd x} - \frac{\pd P}{\pd
y}}\,dx\,dy = \iints{\Ph}\ol{\nu}\cdot\rot\ol{p}\,ds,} где
$\ol{\nu}=(\cos\al,\cos\be,\cos\ga)$ --- единичная нормаль на $\Ph$.
Формула Стокса принимает вид
\eqa{8}{\iints{\Ph}{\rot\ol{p}\cdot\ol{\nu}\,ds=\ints{L}\ol{p}\cdot\ol{t}\,dl.}}

\subsubsection{}

\begin{lem}
1 Для любой гладкой поверхности $\Ph$ существует такое число
$\de>0$, что любая связная часть поверхности $\Ph$ диаметра $\de$
однозначно проектируется на координатные плоскости некоторой
декартовой системы координат (вообще говоря, своей для каждой части
поверхности $\Ph$).
\end{lem}

\begin{proof}
Проверим справедливость утверждения леммы для некоторой окрестности
любой точки $M$ поверхности $\Ph$, заданной векторной
параметризацией $\ol{r}=\ol{r}(u,v)$, $(u,v)\in[D]$ класса $\Cc^1$.
Так как $\ol{n}(M)\ne\ol{0}$, то выберем такую декартову систему
координат $Oxyz$, в которой $\ol{\nu}(M)=(\cos\al,\cos\be,\cos\ga)$
имеет $\cos\al>0,\cos\be>0,\cos\ga>0$. Тогда $\ol{r}=\ol{r}(u,v) =
(x(u,v),y(u,v),z(u,v))$ и $$\ol{n}(M) = \hr{\mbmat{y'_u & z'_u \\
y'_v & z'_v}, \mbmat{z'_u & x'_u \\ z'_v & x'_v}, \mbmat{x'_u & y'_u
\\ x'_v & y'_v}}$$ и все определители не равны нулю. Выше (пункт
3.1.4, лемма 1) установлено, что существуют такие окрестности
$\Uc_1(M)$, $\Uc_2(M)$, $\Uc_3(M)$ на $\Ph$, которые являются
графиками функций переменных $(x,y), (y,z), (x,z)$. Окрестность
$\Uc(M)=\capl{i=1}3 \Uc_i(M)$ искомая.

Допустим теперь, что утверждение леммы в общем случае не верно.
Согласно допущению, для каждого числа $\de_n=\frac1n>0$, $n\in\N$,
можно указать такую связную часть $\Ph_n$ поверхности $\Ph$,
$\diam\Ph_n=\frac1n$, которая не проектируется на все координатные
плоскости ни в одной декартовой системе координат. На каждой $\Ph_n$
выберем точку $M_n\in\Ph_n$. Последовательность $(M_n)$ точек
компакта $\Ph$ имеет сходящуюся подпоследовательность $(M_{n_k})$,
$\liml{k\ra+\bes} M_{n_k}=M_0$ и $M_0\in\Ph$. Будем считать, что
сама $(M_n)$ имеет $\liml{n\ra+\bes}M_n=M_0\in\Ph$.

По предыдущему, существует такая окрестность $\Uc(M_0)$ на $\Ph$,
которая однозначно проектируется на все три координатные плоскости в
некоторой декартовой системе координат. Так как
$\liml{n\ra+\bes}\de_n=0$, то существует индекс $N\in\N$, что для
всех $n\ge N$ справедливо $\Ph_n\cap\Uc(M_0)\ne\es$. Противоречие со
свойствами частей $\Ph_n\subset\Ph$, $n\in\N$.
\end{proof}

\subsubsection{Случай II}
Поверхность $\Ph$ --- кусочно--гладкая.

По условию, $\Ph=\cupl{k=1}m \Ph_k$ и каждая поверхность $\Ph_k$,
$k=\ol{1,m}$ --- гладкая. Согласно лемме из предыдущего пункта, для
каждой $\Ph_k$, $k=\ol{1,m}$, существует такое число $\de_k>0$, что
справедливо утверждение леммы. Выберем
$\de=\min(\de_1,\de_2\sco\de_m)$, $\de>0$. Разобьём каждую
поверхность $\Ph_k=\cupl{j=1}{n_k} \Ph^j_k$, $\diam\Ph^j_k=\de$,
чтобы каждая часть $\Ph^j_k$ однозначно проектировалась на все
координатные плоскости в некоторой декартовой системе координат.

Тогда $\Ph=\cupl{k=1}m \cupl{k=1}{n_k}\Ph^j_k$, и согласно свойству
аддитивности поверхностного интеграла второго рода и предыдущему
случаю, \ml{\iints{\Ph^+} \hr{\frac{\pd R}{\pd y} - \frac{\pd Q}{\pd
z}}\,dy\,dz + \hr{\frac{\pd P}{\pd z}-\frac{\pd R}{\pd x}}\,dz\,dx +
\hr{\frac{\pd Q}{\pd x}-\frac{\pd P}{\pd y}}\,dx\,dy =\\= \sumkum
\suml{j=1}{n_k} \iints{\Ph_k^{j+}} \hr{\frac{\pd R}{\pd y} -
\frac{\pd Q}{\pd x}}\,dy\,dz + \hr{\frac{\pd P}{\pd z}-\frac{\pd
R}{\pd x}}\,dz\,dx + \hr{\frac{\pd Q}{\pd x}-\frac{\pd P}{\pd
y}}\,dx\,dy =\\= \sumkum\suml{j=1}{n_k} \iints{\Ph_k^{j+}}\rot\ol{p}
\cdot \ol{\nu}\,ds \os{случай\;I}{=} \sumkum\suml{j=1}{n_k}
\ints{\pd \Ph_k^{j+}} \ol{p}\cdot\ol{t}\,dl = \sumkum
\ints{L_k}P\,dx+Q\,dy+R\,dz,} поскольку криволинейные интегралы по
общим участкам границ $\pd \Ph^i_k$, $j=\ol{1,n_k}$, $k=\ol{1,m}$,
проходятся по противоположным направлениям и криволинейный интеграл
первого рода обладает свойством аддитивности.

\subsubsection{Замечание. Понятие внешнего дифференцирования
дифференциальной формы в $\R^3$}

Положим $dx\wedge dx=0$, $dy\wedge dy=0$, $dz\wedge dz=0$, $dy\wedge
dx = -dx\wedge dy$, $dz\wedge dy = -dy\wedge dz$, $dz\wedge dx =
-dx\wedge dz$ и $d(P\,dx+Q\,dy+R\,dz)=dP\wedge dx + dQ\wedge dy +
dR\wedge dz$. Тогда \ml{d(P\,dx+Q\,dy+R\,dz)=dP\wedge dx + dQ\wedge
dy + dR\wedge dz = \hr{\frac{\pd P}{\pd x}\,dx + \frac{\pd P}{\pd
y}\,dy + \frac{\pd P}{\pd z}\,dz}\wedge dx + \hr{\frac{\pd Q}{\pd
x}\,dx + \frac{\pd Q}{\pd y}\,dy + \frac{\pd Q}{\pd z}\,dz}\wedge dy
+\\+ \hr{\frac{\pd R}{\pd x}\,dx + \frac{\pd R}{\pd y}\,dy +
\frac{\pd R}{\pd z}\,dz}\wedge dz = -\frac{\pd P}{\pd y}\,dx\wedge
dy + \frac{\pd P}{\pd z}\,dz\wedge dx + \frac{\pd Q}{\pd
x}\,dx\wedge\,dy - \frac{\pd Q}{\pd z}\,dy\wedge dz - \frac{\pd
R}{\pd x}\,dz\wedge dx + \frac{\pd R}{\pd y}\,dy\wedge dz =\\=
\hr{\frac{\pd R}{\pd y}-\frac{\pd Q}{\pd z}}\,dy\wedge dz +
\hr{\frac{\pd P}{\pd z}-\frac{\pd R}{\pd x}}\,dz\wedge dx +
\hr{\frac{\pd Q}{\pd x}-\frac{\pd P}{\pd y}}\,dx\wedge dy.}

Таким образом, формула Стокса принимает вид \equ{\ints{\Ph^+} d\om =
\ints{\pd \Ph}\om,} где $$\om=P\,dx+Q\,dy+R\,dz.$$

\subsection{Формула Остроградского}
\subsubsection{}
\label{p51} Пусть $\Om$ --- такая ограниченная область в $\R^3$,
которая является кубируемым цилиндроидом относительно всех трёх
координатных плоскостей в $\R^3$ с декартовой системой координат
$Oxyz$, и граница --- замкнутая поверхность $\pd\Om=\Ph$ принадлежит
классу $\Cc^1$.

\begin{theorem}
Если функции $P(x,y,z),Q(x,y,z),R(x,y,z)$ непрерывны на замкнутой
ограниченной области $[\Om]$, непрерывно дифференцируемы внутри
$\Om$ и сходятся несобственные интегралы \equ{\iiints{\Om}\frac{\pd
P}{\pd x}\,dx\,dy\,dz, \; \iiints{\Om}\frac{\pd Q}{\pd
y}\,dx\,dy\,dz, \; \iiints{\Om}\frac{\pd R}{\pd z}\,dx\,dy\,dz,} то
\eqa{1}{\iiints{\Om} \hr{\frac{\pd P}{\pd x} + \frac{\pd Q}{\pd y} +
\frac{\pd R}{\pd z}}\,dx\,dy\,dz = \iints{\Ph^+}
(P\,dy\,dz+Q\,dz\,dx+R\,dx\,dy),} где $\Ph^+$ обозначает внешнюю
сторону замкнутой поверхности $\Ph=\pd\Om$.
\end{theorem}

\begin{proof}
В силу симметричных условий на область $\Om$ нужно доказать одно из
равенств \eqa{2}{\iiints{\Om}\frac{\pd P}{\pd x}\,dx\,dy\,dz =
\iints{\Ph^+}P\,dy\,dz; \; \iiints{\Om}\frac{\pd Q}{\pd
y}\,dx\,dy\,dz = \iints{\Ph^+}Q\,dz\,dx, \; \iiints{\Om}\frac{\pd
R}{\pd z}\,dx\,dy\,dz=\iints{\Ph^+}R\,dx\,dy.}

Остальные равенства в (2) доказываются аналогично и равенство (1)
следует из равенств (2) на основании свойства линейности
несобственных кратных интегралов и поверхностных интегралов второго
рода.

Докажем третье из равенств (2). По условию, поверхности $\Ph$
является объединением графиков $\Ph_1$ и $\Ph_2$ функций
$z=f_1(x,y)$, $z=f_2(x,y)$, определённых в некоторой области
$D=пр._{Oxy}\Ph$ в $\R^2$, и $f_1(x,y)\le f_2(x,y)$ при
$(x,y)\in[D]$. Область $D$ квадрируема, а ориентация $\Ph^+$
определяется на графике $\Ph_2$ функции $z=f_2(x,y)$ верхней
стороной $\Ph^+_2$, а на графике $\Ph_1$ функции $z=f_1(x,y)$ ---
нижней стороной $\Ph^-_1$.

Рассмотрим произвольное исчерпывание $\hc{D_n}$ области $D$
квадрируемыми компактами $D_n$, $n\in\N$, и выберем
последовательность $(\ep_n)$, $\ep_n>0$, $n\in\N$, таким образом,
чтобы $\ep_{n+1}<\ep_n<\frac1n$, $n\in\N$, и \equ{f_1(x,y) +
\ep_{n+1} < f_1(x,y)+\ep_n < f_2(x,y)-\ep_n < f_2(x,y)-\ep_{n+1}, \;
(x,y)\in[D_n].} Тогда, по условию теоремы и определению кратного
несобственного интеграла, \eqa{3}{\iiints{\Om}\frac{\pd R}{\pd
z}\,dx\,dy\,dz = \liml{n\ra+\bes} \iiints{\Om_n}\frac{\pd R}{\pd
z}\,dx\,dy\,dz,} где $\Om_n$, $n\in\N$, обозначает цилиндроид
$\hc{(x,y,z)\in\R^3\bvl f_1(x,y)+\ep_n \le z \le f_2(x,y)-\ep_n, \;
(x,y)\in D_n}$, $n\in\N$.

Используя формулу повторного интегрирования для кратного интеграла,
имеем \mla{4}{\iiints{\Om_n} \frac{\pd R}{\pd z}\,dx\,dy\,dz =
\iints{[D_n]} \,dx\,dy \intl{f_1(x,y)+\ep_n}{f_2(x,y)-\ep_n}
\frac{\pd R}{\pd z}\,dz = \iints{[D_n]} [R(x,y,f_2(x,y)-\ep_n) -
R(x,y,f_1(x,y)+\ep_n)]\,dx\,dy =\\= \iints{[D_n]} [R(x,y,f_2(x,y)) -
R(x,y,f_1(x,y))]\,dx\,dy + \iints{[D_n]} [R(x,y,f_2(x,y)-\ep_n) -
R(x,y,f_2(x,y))]\,dx\,dy +\\+ \iints{[D_n]}
[R(x,y,f_1(x,y))-R(x,y,f_1(x,y)+\ep_n)]\,dx\,dy = I_n+I_n^1+ I_n^2.}

Докажем, что \eqa{5}{\liml{n\ra+\bes}I_n^i=0, \; i=1,2,} проверим
это утверждение для $i=1$.

Рассмотрим произвольное число $\ep>0$. Так как функция $R(x,y,z)$
равномерно непрерывна на компакте $[\Om]$, то существует такое число
$\de=\de(\ep)>0$, что для любых точек
$(x',y',z'),(x'',y'',z'')\in[\Om]$, для которых $\hm{x'-x''}<\de$,
$\hm{y'-y''}<\de$, $\hm{z'-z''}<\de$, выполняется неравенство
$\hm{R(x',y',z')-R(x'',y'',z'')}<\frac{\ep}{пл.\,D}$. Выберем
$x'=x''=x$, $y'=y''=y$, $(x,y)\in D$, и $z'=f_2(x,y)$,
$z''=f_2(x,y)-\ep_n$. Тогда $\hm{z'-z''}=\ep_n$, $n\in\N$. Так как
$\liml{n\ra+\bes}\ep_n=0$, то существует такой индекс $N\in\N$,
$N=N_{\de}=N_{\ep}$, что для всех $n\in\N$, $n\ge N_{\ep}$,
справедливо неравенство $0<\ep_n<\de$. Поэтому \equ{\hm{R(x,y,
f_2(x,y))-R(x,y,f_2(x,y)-\ep_n)}<\frac{\ep}{пл.\,D}} для всех
$(x,y)\in[D]$ и всех $n\in\N$, $n\ge N_{\ep}$.

Значит, \equ{\hm{I_n^1} \le\iints{[D_n]}
\hm{R(x,y,f_2(x,y))-R(x,y,f_2(x,y)-\ep_n)}\,dx\,dy \le
\frac{\ep}{пл.\,D} \cdot пл.\,D_n<\ep, \; n\in\N, \; n\ge N_{\ep},}
или $\liml{n\ra+\bes} I^1_n=0$. Итак, утверждение (5) доказано.
Согласно свойствам двойного несобственного интеграла,
\eqa{6}{\liml{n\ra+\bes} I_n = \liml{n\ra+\bes} \iints{[D_n]}
[R(x,y,f_2(x,y)) - R(x,y,f_1(x,y))]\,dx\,dy = \iints{D}
R(x,y,f_2(x,y))\,dx\,dy - \iints{D} R(x,y,f_1(x,y))\,dx\,dy.} Далее,
по определению поверхностных интегралов второго рода,
\eqa{7}{\iints{D} R(x,y,f_2(x,y))\,dx\,dy =
\iints{\Ph_2^+}R(x,y,z)\,dx\,dy} и \eqa{8}{\iints{D}
R(x,y,f_1(x,y))\,dx\,dy = -\iints{\Ph_1^-}R(x,y,z)\,dx\,dy.}
Объединяя формулы (3)--(8), получим \equ{\iiints{\Om} \frac{\pd
R}{\pd z}\,dx\,dy\,dz = \iints{\Ph_2^+}R(x,y,z)\,dx\,dy +
\iints{\Ph_1^-}R(x,y,z)\,dx\,dy = \iints{\Ph^+}R(x,y,z)\,dx\,dy,} то
есть, третью формулу в (2). Теорема доказана.
\end{proof}

\subsubsection{Инвариантная запись формулы Остроградского}
Считаем функции $P(x,y,z),Q(x,y,z),R(x,y,z)$ компонентами векторного
поля $\ol{p}=(P,Q,R)$, определённого на $[\Om]$. Тогда $\frac{\pd
P}{\pd x}+\frac{\pd Q}{\pd y}+\frac{\pd R}{\pd z}=\Div\ol{p}$. По
предположению, в каждой точке гладкой поверхности $\Ph=\pd\Om$
существует непрерывная нормаль, единичный вектор которой имеет
компоненты $\ol{\nu}(M)=(\cos\al,\cos\be,\cos\ga)$. Тогда
\equ{\iints{\Ph^+}P\,dy\,dz+Q\,dz\,dx+R\,dx\,dy = \iints{\Ph}
(P\cos\al+Q\cos\be+R\cos\ga)\,ds=\iints{\Ph}\ol{p}\cdot\ol{\nu}\,ds}
есть поток векторного поля $\ol{p}$ через поверхность $\Ph$. Кроме
того, \equ{\iiints{\Om}\hr{\frac{\pd P}{\pd x} + \frac{\pd Q}{\pd y}
+ \frac{\pd R}{\pd z}}\,dx\,dy\,dz = \iiints{\Om}\Div\ol{p}\, dv.}

Поэтому формула (1) принимает инвариантный вид
\eqa{9}{\iiints{\Om}\Div\ol{p}\,dv =
\iints{\Ph=\pd\Om}\ol{p}\cdot\ol{\nu}\,ds.}

\subsubsection{}
\begin{theorem}
Формула (1) остаётся справедливой, если область $\Om\subset\R^3$
можно с помощью гладких поверхностей разбить на конечное число
областей $\Om_k$, $k=\ol{1,m}$, каждая из которых имеет вид,
указанный в условиях теоремы пункта \ref{p51} в некоторой декартовой
системе координат (своей для каждого $k$, $k=\ol{1,m}$).
\end{theorem}

\begin{proof}
\ml{\iiints{\Om}\hr{\frac{\pd P}{\pd x}+\frac{\pd Q}{\pd
y}+\frac{\pd R}{\pd z}}\,dx\,dy\,dz = \sumkum \iiints{\Om_k}
\hr{\frac{\pd P}{\pd x}+\frac{\pd Q}{\pd y}+\frac{\pd R}{\pd
z}}\,dx\,dy\,dz =\\= \sumkum \iiints{\Om_k}\Div\ol{p}\cdot dv
\os{(9)}{=} \sumkum \suml{j=1}{n_k} \iints{\Ph^j_k}
\ol{p}\cdot\ol{\nu}\,ds = \iints{\pd
\Om^+}P\,dy\,dz+Q\,dz\,dx+R\,dx\,dy,} и последнее равенство имеет
место, так как поверхностные интегралы по общим кускам границы
соседних областей $\Om_k$, $\Om_{k+1}$ берутся по противоположным
ориентациям, и значит, их суммы равны нулю, а также в силу свойства
аддитивности поверхностных интегралов второго рода.
\end{proof}

\subsubsection{Замечание. О дифференцировании 2--формы в $\R^3$}
\ml{d(P\,dy\wedge dz+Q\,dz\wedge dx + R\,dx\wedge dy) = dP\wedge
dy\wedge dz + dQ\wedge dz\wedge dx + dR\wedge dx\wedge dy =
\hr{\frac{\pd P}{\pd x}\,dx + \frac{\pd P}{\pd y}\,dy+\frac{\pd
P}{\pd z}\,dz}\wedge dy\wedge dz + \\
+ \hr{\frac{\pd Q}{\pd x}\,dx + \frac{\pd Q}{\pd y}\,dy+\frac{\pd
Q}{\pd z}\,dz}\wedge dz\wedge dx + \hr{\frac{\pd R}{\pd x}\,dx +
\frac{\pd R}{\pd y}\,dy+\frac{\pd R}{\pd z}\,dz}\wedge dx\wedge dy =
\frac{\pd P}{\pd x}\,dx\wedge dy\wedge dz + \frac{\pd Q}{\pd
y}\,dy\wedge dz\wedge dx +\\+ \frac{\pd R}{\pd z}\,dz\wedge dx\wedge
dy = \frac{\pd P}{\pd x}\,dx\wedge dy\wedge dz + \frac{\pd Q}{\pd
y}\,dx\wedge dy\wedge dz + \frac{\pd R}{\pd z}\,dx\wedge dy\wedge dz
= \hr{\frac{\pd P}{\pd x}+\frac{\pd Q}{\pd y}+\frac{\pd R}{\pd
z}}\,dx\wedge dy\wedge dz} и формула Остроградского принимает вид
\equ{\ints{\Om}d\om=\ints{\pd\Om^+}\om,} где $$\om=P\,dy\wedge
dz+Q\,dz\wedge dx + R\,dx\wedge dy.$$

\subsection{Точные и замкнутые дифференциальные формы}
\subsubsection{}

Пусть дифференциальная форма $P\,dx+Q\,dy$ задана в области
$D\subset\R^2$ и $P(x,y),Q(x,y)$ непрерывны в $D$.

\begin{dfn}
1 Интегралы дифференциальной формы $P\,dx+Q\,dy$ в области $D$
зависят только от начала и конца пути интегрирования, если для любых
двух точек $A,B$ области $D$ интегралы этой формы по всем путям
класса $\Cc^1$ в $D$ с началом $A$ и концом $B$ равны между собой.
\end{dfn}

\begin{theorem}
Интегралы дифференциальной формы в области $D$ зависят только от
начала и конца пути интегрирования тогда и только тогда, когда её
интегралы по всем замкнутым путям класса $\Cc^1$ в $D$ равны нулю.
\end{theorem}

\begin{proof}
\textbf{Необходимость}. Обозначим для краткости дифференциальную
форму $P\,dx+Q\,dy=\om$. Пусть $\ints{L_1}\om=\ints{L_2}\om$ для
любых путей $L_1,L_2$ класса $\Cc^1$ в $D$ с общим началом и общим
концом. Пусть $L$ --- произвольный замкнутый путь класса $\Cc^1$ в
$D$ с непрерывно дифференцируемой параметризацией $f\cln[a,b]\ra D$,
$f(a)=f(b)$.

Если путь $L$ одноточечный, то есть $f=\const$, то $\ints{L}\om=0$
(по определению). Пусть $L$ --- неодноточечный путь. Тогда
существует $c\in(a,b)$, что $f(c)\ne f(a)$. Пусть пути $L_1$ и $L_2$
определяются параметризациями $f\cln[a,c]\ra D$ и $f\cln[c,b]\ra D$.
Тогда пути $L_1$ и $-L_2$ будут из класса $\Cc^1$, имеют общее
начало $f(a)=f(b)$ и общий конец $f(c)$. Согласно предположению,
$$\ints{L_1}\om=\ints{-L_2}\om.$$ Но $L=L_1+L_2$, и значит,
\equ{\ints{L}\om=\ints{L_1}\om+\ints{L_2}\om=\ints{L_1}\om-\ints{-L_2}\om=0.}

\textbf{Достаточность}. Пусть интегралы дифференциальной формы $\om$
по любым замкнутым путям класса $\Cc^1$ в $D$ равны нулю. И пусть
$L_1$ и $L_2$ --- произвольные пути класса $\Cc^1$ в $D$, имеющие
общее начало и общий конец. Тогда $-L_2$ тоже путь класса $\Cc^1$ в
$D$. Так как $L=L_1+(-L_2)$ --- замкнутый путь класса $\Cc^1$ в $D$,
то \equ{\ints{L_1}\om-\ints{L_2}\om= \ints{L_1}\om+\ints{-L_2}\om=
\ints{L}\om=0,} и значит, \equ{\ints{L_1}\om=\ints{L_2}\om.}
\end{proof}

\subsubsection{Точные дифференциальные формы}
\begin{dfn}
2 Дифференциальную форму $P\,dx+Q\,dy$ в области $D$ называют
точной, если в $D$ существует такая дифференцируемая функция
$F(x,y)$, что $P\,dx+Q\,dy=dF$ в $D$. Тогда $$P=\frac{\pd F}{\pd x},
\; Q=\frac{\pd F}{\pd y}.$$

Функцию $F$ называют первообразной дифференциальной формы
$P\,dx+Q\,dy$ в области $D$

Функцию $F$ называют также потенциалом векторного поля
$\ol{p}=(P,Q)$ в $D$ и тогда $\ol{p}=\grad F$.
\end{dfn}

\begin{theorem}
Интегралы дифференциальной формы $P\,dx+Q\,dy$, заданной в области
$D$, зависят только от начала и конца пути интегрирования класса
$\Cc^1$ в $D$ тогда и только тогда, когда эта форма --- точная в
$D$.
\end{theorem}

\begin{proof}
\textbf{Необходимость}. Фиксируем некоторую точку $M_0(x_0,y_0)$
области $D$ и рассмотрим произвольную точку $M(x,y)\in D$. Так как
открытое множество $D$ --- область, то есть связное множество, то
существует ломаная $\ol{M_0M}$ с началом $M_0$ и концом в $M$,
которая целиком лежит в $D$. Положим
\equ{F(x,y)=F(M)=\ints{\ol{M_0M}}P\,dx+Q\,dy.} Рассмотрим точку
$M_1(x+h,y)$ и добавим к ломанной $\ol{M_0M}$ новое звено
$\ol{MM_1}$. Согласно свойству аддитивности криволинейного интеграла
и теореме о среднем для определённого интеграла от непрерывной
функции, \equ{\frac{F(x+h,y)-F(x,y)}h=\frac1h
\ints{\ol{M_0M_1}}P\,dx+Q\,dy-\frac1h\ints{\ol{M_0M}}P\,dx+Q\,dy =
\frac1h\ints{\ol{MM_1}}P\,dx = \frac1h\ints{\ol{MM_1}}P\,dx =
\frac1h\intl{x}{x+h}P(t,y)\,dt=P(\xi,y),} где $\xi$ лежит между $x$
и $x+h$ и $dy=0$ на $\ol{MM_1}$.

Так как $\liml{h\ra0}P(\xi,y)=P(x,y)$, то существует
\equ{\liml{h\ra0} \frac{F(x+h,y)-F(x,y)}h = P(x,y)} и
$P(x,y)=\frac{\pd F}{\pd x}.$

Аналогично доказывается формула $Q(x,y)=\frac{\pd F}{\pd y}$.

\textbf{Достаточность}. Пусть $P\,dx+Q\,dy=dF$ в $D$ и $L$ ---
произвольный путь класса $\Cc^1$ в $D$ с началом в точке $A$ и
концом в точке $B$. Рассмотрим непрерывно дифференцируемую
параметризацию $f\cln[a,b]\ra D$ пути $L$; $f(a)=A$, $f(b)=B$, и
$f=(\ph,\psi)$. Тогда, по определению, \ml{\ints{L}P\,dx+Q\,dy =
\intl{a}b [P(f(t))\ph'(t) + Q(f(t))\psi'(t)]\,dt = \intl{a}b [F'_x
(f(t))\ph'(t) + F'_y (f(t))\psi'(t)]\,dt = \intl{a}b (F\circ
f)'(t)\,dt =\\= (F\circ f)(b)-(F\circ f)(a) =
F(f(b))-F(f(a))=F(B)-F(A).}
\end{proof}

\begin{imp}
Дифференциальная форма $P\,dx+Q\,dy$, заданная в области $D$, точна
тогда и только тогда, когда её интегралы по всем замкнутым путям
класса $\Cc^1$ в $D$ равны нулю.
\end{imp}

\subsubsection{Замкнутые дифференциальные формы}
\begin{dfn}
3 Дифференциальную форму $P\,dx+Q\,dy$ в области $D$ называют
замкнутой, если она локально точна; то есть, каждая точка области
$D$ обладает окрестностью, в которой эта форма точная.
\end{dfn}

\begin{theorem}
Пусть $P(x,y)$ и $Q(x,y)$ --- непрерывные функции с непрерывными
частными производными $\frac{\pd P}{\pd y},\frac{\pd Q}{\pd x}$ в
области $D$. Для того, чтобы дифференциальная форма $P\,dx+Q\,dy$
была замкнутой, необходимо и достаточно, чтобы $\frac{\pd P}{\pd
y}=\frac{\pd Q}{\pd x}$ во всех точках области $D$.
\end{theorem}

\begin{proof}
\textbf{Необходимость}. Пусть форма $P\,dx+Q\,dy$ замкнута в $D$.
Тогда каждая точка $M(x,y)\in D$ обладает кругом $\Uc\subset D$, в
котором эта форма имеет первообразную $F(x,y)$, и значит, $F'_x=P$,
$F'_y=Q$. Так как, по условию, существуют непрерывные $\frac{\pd
P}{\pd y}$ и $\frac{\pd Q}{\pd x}$, то следовательно, функция $F$
обладает непрерывными вторыми производными $F''_{xy}=\frac{\pd
P}{\pd y}$ и $F''_{yx}=\frac{\pd Q}{\pd x}$, для которых, в силу
теоремы Шварца, $F''_{xy}=F''_{yx}$; то есть, $\frac{\pd P}{\pd
y}=\frac{\pd Q}{\pd x}.$

\textbf{Достаточность}. Пусть $M_0(x_0,y_0)$ --- произвольная точка
области $D$. Существует круг $\Uc(M_0;\rho)$, $\rho>0$, что
$\Uc(M_0,\rho)\subset D$. Вместе с каждой точкой $M(x,y)$ круга
$\Uc(M_0;\rho)$ в нём лежит и прямоугольник $\Kc=M_0M_1MM_2$.
Согласно формуле Грина,
\equ{\ints{\ol{M_0M_1MM_2}}P\,dx+Q\,dy=\iints{\Kc} \hr{\frac{\pd
Q}{\pd x}-\frac{\pd P}{\pd y}}\,dx\,dy=0,} следовательно,
\equ{\ints{\ol{M_0M_1M}}P\,dx+Q\,dy=\ints{\ol{M_0M_2M}}P\,dx+Q\,dy=F(x,y).}
Тогда \eqa{1}{F(x,y)=\ints{M_0M_2}Q\,dy+\ints{M_2M}P\,dx=\intl{y_0}y
Q(x_0,\eta)\,d\eta+\intl{x_0}x P(\xi,y)\,d\xi} (так как $dx=0$ на
$M_0M_2$ и $dy=0$ на $M_2M$), и \eqa{2}{F(x,y) =
\ints{M_0M_1}P\,dx+\ints{M_1M}Q\,dy = \intl{x_0}x P(\xi,y_0)\,d\xi +
\intl{y_0}y Q(x,\eta)\,d\eta} (так как $dy=0$ на $M_0M$ и $dx=0$ на
$M_1M$).

Согласно теореме о дифференцируемости интеграла по верхнему пределу,
из (1) следует, что \equ{\frac{\pd F}{\pd x}(x,y)=P(x,y),} а из (2)
следует, что \equ{\frac{\pd F}{\pd y}(x,y)=Q(x,y).}

Так как функции $P(x,y)$ и $Q(x,y)$ непрерывны, то функция $F(x,y)$
дифференцируема и $$dF=F'_x\,dx+F'_y\,dy=P\,dx+Q\,dy.$$
\end{proof}

\subsubsection{Потенциальные и соленоидальные векторные поля}
Пусть в некоторой области $\Om\subset\R^3$ задано непрерывно
дифференцируемое векторное поле $\ol{p}(M)=\ol{p}(x,y,z)$,
$M(x,y,z)\in\Om$.

\textbf{Циркуляцией} векторного поля $\ol{p}$ по замкнутой кривой
$L$ класса $\Cc^1$, расположенной в области $\Om$, называют интеграл
$\ints{L}\ol{p}\cdot\ol{t}\,dl$, где $\ol{t}$ --- единичный вектор
касательной к $L$ и $dl$ --- дифференциал длины кривой $L$.

\textbf{Потоком} векторного поля $\ol{p}$ через ориентированную
кусочно--гладкую поверхность $\Ph$, расположенную в $\Om$,
называется интеграл $\iints{\Ph} \ol{p}\cdot\ol{\nu}\,ds$, где
$\ol{\nu}$ --- единичный вектор нормали к $\Ph$, определяющий
ориентацию на $\Ph$, и $ds$ --- элемент площади поверхности на
$\Ph$.

\begin{dfn}
4 Векторное поле $\ol{p}$ называют потенциальным в области $\Om$,
если циркуляция этого поля по любой замкнутой кривой класса $\Cc^1$,
расположенной в $\Om$, равна нулю.
\end{dfn}

\begin{dfn}
5 Векторное поле $\ol{p}$ называют соленоидальным в области $\Om$,
если поток этого поля через любую кусочно--гладкую замкнутую
поверхность (несамопересекающуюся), расположенную в $\Om$ и
представляющую собой границу некоторой ограниченной подобласти
области $\Om$, равен нулю.
\end{dfn}

Трехмерная область $\Om$ называется поверхностно--односвязной, если
для любой замкнутой кривой $L$ класса $\Cc^1$, расположенной в
$\Om$, можно указать такую ориентируемую кусочно--гладкую
поверхность $\Ph$, расположенную в $\Om$, границей которой служит
$L$. Отметим, что упомянутая поверхность $\Ph$ удовлетворяет формуле
Стокса.

\begin{theorem}
Пусть в поверхностно--односвязной области $\Om$ задано непрерывно
дифференцируемое векторное поле $\ol{p}=(P,Q,R)$. Тогда эквивалентны
следующие утверждения:
\begin{points}{-2}
\item Векторное поле $\ol{p}=\ol{p}(M)$ потенциальное;

\item В области $\Om$ существует потенциальная функция
$u(M)=u(x,y,z)$; то есть, такая функция, что $\ol{p}=\grad u$, или,
что то же, $du=P\,dx+Q\,dy+R\,dz$. В этом случае для любых точек $A$
и $B$ области $\Om$ и для произвольной кривой $\os{\smile}{AB}$
класса $\Cc^1$, соединяющей эти точки и расположенной в $\Om$,
\equ{\ints{\os{\smile}{AB}}\ol{p}\cdot\ol{t}\,dl=u(B)-u(A);}

\item Векторное поле $\ol{p}=\ol{p}(M)$ безвихревое, то есть
$\rot\ol{p}=\ol{0}$ в $\Om$.
\end{points}

Очевидно, утверждение 3 эквивалентно соотношениям \equ{\frac{\pd
P}{\pd y}=\frac{\pd Q}{\pd x}, \; \frac{\pd Q}{\pd z}=\frac{\pd
R}{\pd y}, \; \frac{\pd R}{\pd x}=\frac{\pd P}{\pd z}.}

Таким образом, каждое из условий 2 и 3 представляет собой
необходимое и достаточное условие потенциальности непрерывно
дифференцируемого векторного поля $\ol{p}$.
\end{theorem}
\begin{proof}
Применим схему $\begin{CD} 1@>>> 2\\ @AAA @VVV \\ @<<< 3\end{CD}$.
Утверждения $\textbf{1}^{\circ} \Rightarrow \textbf{2}^{\circ}$ и
$\textbf{2}^{\circ} \Rightarrow \textbf{3}^{\circ}$ справедливы без
предположения о поверхностной односвязности области $\Om$ и
доказываются в полной аналогии с соответствующими утверждениями
теорем пунктов 4.6.1--4.6.3. Докажем утверждение $\textbf{3}^{\circ}
\Rightarrow \textbf{1}^{\circ}$. Пусть $L$ --- произвольная
замкнутая кривая класса $\Cc^1$, расположенная в $\Om$. Так как
область $\Om$ поверхностно односвязная, то в $\Om$ существует
кусочно--гладкая поверхность $\Ph$, границей которой служит $L$. По
формуле Стокса имеем \equ{\ints{L}\ol{p}\cdot \ol{t}\,dl =
\iints{\Ph} \rot\ol{p}\cdot\ol{\nu}\,ds.} Отсюда и из условия
$\rot\ol{p}=\ol{0}$, получаем \equ{\ints{L}\ol{p}\cdot\ol{t}\,dl=0;}
то есть, $\ol{p}$ является потенциальным полем.
\end{proof}

Пространственная область $\Om$ называется объёмно--односвязной, если
любая замкнутая, кусочно--гладкая несамопересекающаяся
ориентированная поверхность, расположенная в $\Om$, служит границей
некоторой области, расположенной в $\Om$.

\begin{theorem}
Для того, чтобы непрерывно дифференцируемое векторное поле $\ol{p}$
было соленоидальным в объёмно--односвязной области $\Om$, необходимо
и достаточно, чтобы во всех точках $\Om$ выполнялось равенство
$\Div\ol{p}=0$.
\end{theorem}

\begin{proof}
\textbf{Необходимость}. Пусть в некоторой точке
$M_0(x_0,y_0,z_0)\in\Om$ $\Div\ol{p}\ne0$, и пусть, для
определённости, $\Div\ol{p}(M_0)=c>0$. В силу непрерывности функции
$\Div\ol{p}(M)$, существует такая шаровая окрестность
$B(M_0,\rho_0)$, $\rho_0>0$, в которой $\Div\ol{p}(M)>\frac{c}2$ для
всех $M\in B(M_0,\rho_0)$. Если $S_0$ обозначает сферу--границу шара
$\Uc(M_0,\rho_0)$, то, по теореме Остроградского,
\equ{\iiints{\Uc(M_0,\rho_0)}\Div\ol{p}\,dv =
\iints{S_0}\ol{p}\cdot\ol{\nu}\,ds \ge \frac{c}2\frac43 \pi\rho_0^3
> 0,} что противоречит свойству соленоидальности поля $\ol{p}(M)$,
согласно которому обязано $$\iints{S_0}\ol{p}\cdot\ol{\nu}\,ds=0.$$

\textbf{Достаточность}. Рассмотрим произвольную замкнутую,
кусочно--связную, несамопересекающуюся, ориентируемую поверхность
$\Ph$, расположенную в $\Om$. Так как $\Om$ --- объёмно односвязная
область, то $\Ph$ является границей некоторой подобласти $\Om_0$
области $\Om$. Применяя к $\Om_0$ и векторному полю $\ol{p}$ формулу
Остроградского, получим соотношение
\equ{\iiints{\Om_0}\Div\ol{p}\,dv =
\iints{\Ph}\ol{p}\cdot\ol{\nu}\,ds,} из которого и из условия
$\Div\ol{p}=0$ следует соотношение
\equ{\iints{\Ph}\ol{p}\cdot\ol{\nu}\,ds=0.} Последнее равенство,
согласно определению означает соленоидальность поля $\ol{p}$ в
$\Om$.
\end{proof}



\end{document}
