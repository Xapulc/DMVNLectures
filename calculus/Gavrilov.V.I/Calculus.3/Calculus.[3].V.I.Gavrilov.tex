\documentclass[a4paper]{article}
\usepackage[utf]{dmvn}

\tocsubsectionparam{2.7em} \tocsubsubsectionparam{3.7em}
\newenvironment{exx}{\par\vskip\theoremskip\textbf{Пример.}}{\par\vskip\theoremskip}

\renewenvironment{dfn}[1]{\par\vskip\theoremskip\textbf{Определение~{#1}.}}{\par\vskip\theoremskip}
\newenvironment{thh}{\par\vskip\theoremskip\textbf{Теорема.}\normalfont \itshape}{\par\vskip\theoremskip}
\newenvironment{imp*}{\par\vskip\theoremskip\textbf{Следствие.}\normalfont \itshape}{\par\vskip\theoremskip}
\newenvironment{stm*}{\par\vskip\theoremskip\textbf{Утверждение.}\normalfont \itshape}{\par\vskip\theoremskip}
\newenvironment{lem}[1]{\par\vskip\theoremskip\textbf{Лемма~{#1}.}\normalfont \itshape}{\par\vskip\theoremskip}
\newenvironment{lem*}{\par\vskip\theoremskip\textbf{Лемма.}\normalfont \itshape}{\par\vskip\theoremskip}
\newenvironment{thn}[1]{\par\vskip\theoremskip\textbf{Теорема~{#1}.}\normalfont \itshape}{\par\vskip\theoremskip}


\newcommand{\emd}[1]{\emph{#1}}

\begin{document}
\dmvntitle{Курс лекций по}{математическому анализу}{Лектор
Валериан Иванович Гаврилов} {II курс, 3 семестр, поток
механиков}{Москва, 2006 г.}

\pagebreak


\dmvntrail
\medskip
\medskip
\medskip
\tableofcontents \pagebreak


\hrule\begin{center}\LARGE \bf Часть 5.\hfill Р я д
ы\end{center}\hrule

\section{Числовые ряды}

\subsection{Нижний и верхний пределы числовой последовательности}

\subsubsection{Множество частичных пределов ограниченной числовой
последовательности}

Рассмотрим произвольную ограниченную числовую последовательность
$(a_n)$ и множество $A=\{a_n \bvl n\in \N\}$ её значений. Множество
$A$ значений ограничено и, следовательно, имеет точную нижнюю грань
$\inf A = m$ и точную верхнюю грань $\sup A=M$, $m\le a_n \le M, \;
n\in \N$. По теореме Больцано, ограниченная последовательность
$(a_n)$ содержит сходящиеся подпоследовательности $(a_{n_k})$,
$\liml{k\ra +\bes} a_{n_k}=l_1$. Так как $m \le a_{n_k} \le M, \;
k\in\N$, то в силу свойства монотонности предела последовательности,
$m\le \liml{k\ra +\bes}a_{n_k}=l_1\le M$. Обозначим символом $\Lc$
множество всех частичных пределов последовательности $(a_n)$,
$\Lc\ne\es$ ($l_1\in\Lc$). Следовательно, множество $\Lc$ ограничено
и $\Lc \subset [m,M]$. Поэтому существуют $\inf\Lc=p$ и
$\sup\Lc=q\colon m\le p \le q \le M$.

\begin{note}
Числа $m,p,q,M$ --- различные. У последовательности $(a_n)\colon
a_n=(-1)^n \cdot \hr{1+\frac1n}, \; n\in\N$, множество
$\Lc=\{-1,1\}$, и $m=-2,p=-1,q=1,M=3/2$.
\end{note}

\begin{theorem}
Множество частичных пределов ограниченной числовой
последовательности обладает наименьшим и наибольшим элементами.
\end{theorem}

\begin{proof}
Рассмотрим произвольную ограниченную последовательность $(a_n)$ и
$\Lc\ne\es$. По определению, существует $q=\sup\Lc, p=\inf\Lc$.
Покажем, что $q,p \in \Lc$. Докажем, что $q\in\Lc$. Рассмотрим
произвольное $\ep>0$, так как $q-\ep<q=\sup\Lc$, то, по критерию
точной верхней грани, существует $l_{\ep} \in \Lc \colon q-\ep <
%l малое goes here
l_{\ep} \le q$. Пусть $\de = l_{\ep} - (q-\ep)>0$. Так как $l_{\ep}
= \liml{\nu\ra +\bes} a_{n_\nu}$ для некоторой подпоследовательности
$(a_{n_{\nu}}) \subset (a_n)$, то для $\de>0 \; \exists \;
N\in\N\colon \hm{a_{n_{\nu}}-l_{\ep}} < \de$ для всех $\nu \ge N$,
или $(q-\ep) = l_{\ep}-\de < a_{n_\nu} < l_{\ep}+\de = 2 l_{\ep} -
(q-\ep), \; \nu \ge N$. Таким образом, каждый интервал
$(q-\ep,q+\ep)$ содержит элементы последовательности $(a_n)$ с
бесконечным множеством индексов. Выберем $\ep_k = \frac1k > 0, \;
k\in\N$. Для каждого числа $k\in\N$ интервал
$\hr{q-\frac1k,q+\frac1k}$ содержит элементы последовательности
$(a_n)$ с бесконечным множеством индексов, следовательно, для
каждого $k\in\N$ выбираем $a_{n_k}\in (a_n) \colon a_{n_k} \in
\hr{q-\frac1k,q+\frac1k}$ и $n_k > n_{k-1}, \; k\ge2$. Получим
подпоследовательность $(a_{n_k}) \colon \liml{k\ra+\bes} a_{n_k}=q$,
так как $\hm{a_{n_k}-q} < \frac1k, \; \forall k\in\N$, то есть
$q\in\Lc$, следовательно, $q=\sup \Lc = \max \Lc$.

Аналогичное доказательство для числа $p=\inf\Lc = \min\Lc$.
\end{proof}

\subsubsection{Определение нижнего и верхнего предела числовой
последовательности}

Анализируя доказательство теоремы 1, заключаем, что доказано более
сильное утверждение:

\begin{thn}{1.1'}
Если числовая последовательность ограничена сверху, и множество её
частичных пределов не пусто, то оно содержит наибольший элемент.

Если числовая последовательность ограничена снизу, и множество её
частичных пределов не пусто, то оно содержит наименьший элемент.
\end{thn}

\begin{note}
Условие непустоты множества частичных пределов существенно!
Действительно, отрицательная бесконечно большая последовательность
$(a_n)$ ограничена сверху
%скобочки \hr goes here
$\hr{\liml{n\ra+\bes}a_n=-\bes}$, но не имеет верхнего предела, так
как $\Lc=\es$. Аналогично, положительная бесконечно большая
последовательность $(a_n)$
%and here
$\hr{\liml{n\ra+\bes}a_n=+\bes}$ ограничена снизу, но не имеет
нижнего предела, так как $\Lc=\es$.
\end{note}

На основании теоремы 1.1' и замечания, введём определение.

\begin{df}
Если последовательность $(a_n)$ ограничена сверху и множество $\Lc$
её частичных пределов не пусто, то $q=\sup\Lc=\max\Lc$ ---
\textbf{верхний предел последовательности $(a_n)$}. Обозначение:
$q=\uliml{n\ra+\bes}a_n.$

Если последовательность $(a_n)$ ограничена снизу и множество $\Lc
\ne \es$, то $p=\inf\Lc=\min\Lc$ --- \textbf{нижний предел
последовательности $(a_n)$}. Обозначение: $p=\lliml{n\ra+\bes}a_n$.

\end{df}

Условимся об обозначениях:

$\uliml{n\ra+\bes}a_n = q < +\bes$ --- ограниченная сверху
последовательность $(a_n)$ с верхним пределом $q$.

$\uliml{n\ra+\bes}a_n = +\bes$ --- неограниченная сверху
последовательность.

$\lliml{n\ra+\bes}a_n = p>-\bes$ --- ограниченная снизу
последовательность $(a_n)$ с нижним пределом $p$.

$\lliml{n\ra+\bes}a_n = -\bes$ --- неограниченная снизу
последовательность.

\subsubsection{Эквивалентные определения нижнего и верхнего предела
числовой последовательности}

Рассмотрим произвольную последовательность $(a_n)$ и числовые
множества $A_k=\{a_n \bvl n\ge k\}$, $k\in\N$, $A_1=A$. По
определению, $A_{k+1} \subset A_{k} \subset A_1 = A, \; \forall
k\in\N$. Если последовательность $(a_n)$ ограничена сверху, то
ограничены сверху все $A_k$, $k\in\N$, и существует $M_k=\sup A_k$,
$k\in\N$. Так как $A_{k+1} \subset A_k$, то $M_{k+1} \le M_k$ и
последовательность $(M_k)$ убывает. По теореме Вейерштрасса, либо
существует $\liml{k\ra+\bes}M_k=q^{*}$ (если $(M_k)$ ограничена
снизу), либо $\liml{k\ra+\bes}M_k=-\bes$ (если $(M_k)$ неограничена
снизу).

В последнем случае сама $(a_n)$ обязана быть отрицательной
бесконечно большой последовательностью, то есть
$\liml{n\ra+\bes}a_n=-\bes$. Действительно, так как
$\liml{k\ra+\bes}M_k = -\bes$, то для любого $E>0 \; \exists N\in\N,
\; N=N_E$, что $M_k<-E$ для $k=N$ и, следовательно, $a_n \le
M_k=\sup A_k < -E$ для любого $n\ge k=N$, то есть
$\liml{n\ra+\bes}a_n=-\bes$.

Если $(a_n)$ ограничена снизу, то ограничены снизу все $A_k, \;
k\in\N$, и, следовательно, существуют точные нижние грани $m_k =
\inf A_k, \; k\in\N$. Так как $A_{k+1}\subset A_k$, то $m_{k+1}\ge
m_k$. Так что последовательность ($m_k$) возрастает. По теореме
Вейерштрасса, либо существует $\liml{k\ra+\bes} m_k=p^{*}$ (если
($m_k$) ограничена сверху), либо $\liml{k\ra+\bes}m_k=+\bes$
(($m_k$) --- положительная бесконечно большая последовательность)
(если ($m_k$) неограничена сверху). В последнем случае, сама
последовательность $(a_n)$ \emph{обязана} быть положительной
бесконечно большой последовательностью, то есть $\liml{n\ra+\bes}a_n
= +\bes$. Действительно, для любого $E>0 \; \exists N\in\N \colon
m_k>E$ для $k=N$, $N=N_E$, и $a_n\ge m_k > E$ для всех $n\ge k=N$,
то есть $\liml{n\ra+\bes}a_n = +\bes$.

\begin{theorem} Для произвольной последовательности
$(a_n)$, ограниченной сверху, справедливо
$q^*=q=\uliml{n\ra+\bes}a_n$. Для произвольной последовательности
$(a_n)$, ограниченной снизу, справедливо
$p^*=p=\lliml{n\ra+\bes}a_n$.
\end{theorem}

\begin{lem}{1}
Если $(a_n)$ ограничена сверху, и множество её частичных пределов
$\Lc\ne\es$, то для любого $\ep>0 \; \exists N\in\N\, \;
N=N_{\ep}\colon a_n<q^*+\ep$ для всех $n\ge N$.

Если $(a_n)$ ограничена снизу, и множество её частичных пределов
$\Lc\ne\es$, то для любого $\ep>0 \; \exists N\in\N\, \;
N=N_{\ep}\colon a_n>p^*+\ep$ для всех $n\ge N$.
\end{lem}

\begin{proof}
(Для $q^*$). Так как, по определению и по теореме Вейерштрасса,
$q^*=\liml{k\ra+\bes}M_k=\inf \{M_k \bvl k\in\N\}$, и $q^*+\ep>q^*$,
то, по критерию точной нижней грани, существует некий элемент $M_N$
такой, что $q^* \le M_N < q^* + \ep$. Так как $M_N=\sup A_N$, то
$a_n \le M_N < q^* + \ep$ для любого $n\ge N$.
\end{proof}

\begin{lem}{2}
Если $(a_n)$ ограничена сверху и множество $\Lc\ne\es$, то для
любого $\ep>0$ и любого $N\in\N$ $\exists \; n'\in\N, \; n'>N \colon
a_{n'} > q^*-\ep$. Если $(a_n)$ ограничена снизу, и множество
$\Lc\ne\es$, то для любого $\ep>0$ и любого $N\in\N$ существует
$n'>N\in\N \colon a_{n'} < p^*+\ep$.
\end{lem}

\begin{proof}
(Для $q^*$). Так как $M_k \ge q^* = \inf \{M_k \bvl k\in\N \} > q^*
- \ep$ для любого $k\in\N$ и $M_k = \sup A_k > q^* - \ep$, то
существует $a_{n'} \in A_k, \; n'\ge k$, для которого $q^* - \ep <
a_{n'}$. Полагаем $k=N+1>N$ и находим $n' \ge k > N$, для которого
$a_{n'} > q^* - \ep$. Аналогично проверяется утверждение для $p^*$.
\end{proof}

\begin{proof}
Переходим непосредственно к доказательству теоремы 1.2 и покажем,
что $q^*=q=\uliml{n\ra+\bes}a_n$. Проверим сначала, что $q^* \in
\Lc$. Рассмотрим последовательность $(\ep_k)$, $\ep_k>0$,
$\liml{k\ra+\bes} \ep_k=0$, в которой считаем $\ep_1>0$ и $\ep_1 > 2
\hm{q^*-a_{n_1}}$ с каким-либо индексом $n_1\in\N$, так что
$\hm{q^*-a_{n_1}}<\ep_1$. Остальные числа подпоследовательности
$(a_{n_k})$ выберем по индукции. Считаем, что выбраны
$a_{n_1},a_{n_2},\ldots,a_{n_{k-1}}, \; n_1<n_2<\ldots<n_{k-1}$,
такие, что $\hm{q^*-a_{n_{\nu}}}<\ep_{\nu}, \; \nu=\ol{1,k-1}$. Для
числа $\ep_k>0$, по лемме 1, существует такой индекс $N^1_k \in \N$,
что $a_n<q^*+\ep_k$ для всех $n\ge N^1_k$. Рассмотрим
$N_k=\max(n_{k-1}, N^1_k)$. Согласно лемме 2, существует $n'>N_k\ge
n_{k-1}$, что $a_{n'}>q^*-\ep_k$. Полагаем $n_k=n'$, так что $n_k >
n_{k-1}$ и $a_{n_k}>q^* - \ep_k$. С другой стороны, $n_k=n' > N_k
\ge N^1_k$, и следовательно, $a_{n_k}<q^*+\ep_k$. Таким образом,
$\hm{q^*-a_{n_k}}<\ep_k$ для всех $k\in\N$ и $q^*=\liml{k\ra+\bes}
a_{n_k}, \; q^*\in\Lc$.

Рассмотрим теперь $\ep>0$ и произвольное $l_1\in\Lc$, так что
$l_1=\liml{\nu\ra+\bes} a_{n_{\nu}}$ для некоторой
подпоследовательности $(a_{n_{\nu}})$ последовательности $(a_n)$.
Согласно лемме 1, существует $N_{\ep}\in\N$, что $a_{n_\nu} < q^* +
\ep$ для всех $n_{\nu} \ge N_{\ep}$. По свойству сохранения
неравенства для предела сходящейся последовательности, $l_1 =
\liml{\nu\ra+\bes} a_{n_{\nu}} \le q^* + \ep$. Итак, любой элемент
$l_1\in\Lc$ удовлетворяет неравенству $l_1 \le q^* + \ep$ для всех
$\ep>0$, и значит, $l_1 \le q^*$ для всех $l_1\in\Lc$. Поэтому
$q^*=\max\Lc=q$.

Аналогично проверяется, что $p^*=\min\Lc=p$.
\end{proof}

Объединяя утверждения теоремы 1.2 и леммы 1, получим
\textbf{основное свойство верхнего и нижнего пределов}.

\begin{theorem}
Если последовательность $(a_n)$ имеет конечный верхний предел
$\uliml{n\ra+\bes}a_n = q<+\bes$, то для любого $\ep>0$ существует
такой индекс $N\in\N, \; N=N_{\ep}$, что $a_n<q+\ep$ справедливо для
всех $n\in\N, \; n\ge N$. Если последовательность $(a_n)$ имеет
конечный нижний предел $\lliml{n\ra+\bes}a_n = p>-\bes$, то для
любого $\ep>0$ существует такой индекс $N\in\N, \; N=N_{\ep}$, что
$a_n>p-\ep$ справедливо для всех $n\in\N, \; n\ge N$.
\end{theorem}

В математической литературе встречаются обозначения
$\uliml{n\ra+\bes}a_n = \liml{n\ra+\bes} \sup a_n$ и
$\lliml{n\ra+\bes}a_n = \liml{n\ra+\bes}\inf a_n$, объяснение
которых содержится в теореме 1.2.

\subsubsection{Критерий сходящейся последовательности}

\begin{theorem}
Числовая последовательность $(a_n)$ сходится и имеет
$\liml{n\ra+\bes}a_n=l$ тогда и только тогда, когда 1) $(a_n)$
ограничена и 2) $\lliml{n\ra+\bes}a_n=\uliml{n\ra+\bes}a_n = l$.
\end{theorem}

\begin{proof}
\textbf{Необходимость}. По условию, существует $\liml{n\ra+\bes}
a_n=l$. Тогда $(a_n)$ ограничена и множество $\Lc$ всех её частичных
пределов состоит из единственного числа $l$, $\Lc=\{ l \}$. Поэтому,
$p=\min \Lc = l = \max\Lc = q$.

\textbf{Достаточность}. Согласно условиям 1) и 2) и теоремам 1.1,
1.2, множество $\Lc\ne\es$, $\lliml{n\ra+\bes}a_n = p = \min\Lc$,
$\uliml{n\ra+\bes}a_n = q = \max\Lc$ и $p=q=l$. Рассмотрим
произвольное число $\ep>0$. По теореме 1.3, существуют
$N^i_{\ep}\in\N$, $i=1,2$, что $a_n > p-\ep = l-\ep$ для всех $n\ge
N^1_{\ep}$ и $a_n<q+\ep=l+\ep$ для всех $n\ge N^2_{\ep}$. Обозначим
$N=N_{\ep}=\max (N^1_{\ep},N^2_{\ep})$. Тогда $l-\ep < a_n < l+\ep$
для всех $n\ge N$, то есть $l=\liml{n\ra+\bes} a_n$.
\end{proof}

\subsubsection{Некоторые свойства нижнего и верхнего предела
последовательности}

\begin{theorem}
Если числовая последовательность $(a_n)$ имеет конечный
$\uliml{n\ra+\bes}a_n=q < +\bes$ $\hr{\lliml{n\ra+\bes}a_n=p
>-\bes}$, то для произвольной сходящейся последовательности $(b_n)$,
предел которой $\liml{n\ra+\bes} b_n = b \ge 0$, справедливо
$\uliml{n\ra+\bes} (a_n b_n)=b \uliml{n\ra+\bes}a_n$
$\hr{\lliml{n\ra+\bes}(a_nb_n)=b \lliml{n\ra+\bes}a_n}$.
\end{theorem}

\begin{proof}
Доказательство проведём для верхнего предела. Рассмотрим множество
$\Lc$ всех частичных пределов последовательности $(a_n)$;
$\Lc\ne\es$ --- по условию. Если через $\Lc_1$ обозначить множество
всех частичных пределов последовательности $(a_nb_n)$, то $\Lc_1 =
\{ l_1\in\R \bvl l_1=l'\cdot b, \; l'\in\Lc \}$, так как все
частичные пределы сходящейся последовательности $(b_n)$ совпадают с
$b$. Так как число $b\ge0$, то $\max\Lc_1=b\max\Lc$, так что
$\uliml{n\ra+\bes} (a_nb_n)=b\uliml{n\ra+\bes}a_n$. Аналогично,
$\min\Lc_1=b\min\Lc$ и $\lliml{n\ra+\bes}(a_nb_n)=b
\lliml{n\ra+\bes}a_n$.
\end{proof}

\begin{imp*}
Если существует $\uliml{n\ra+\bes}a_n=q<+\bes$
$\hr{\lliml{n\ra+\bes}a_n=p>-\bes}$, то для произвольного $b\ge0$
справедливо $\uliml{n\ra+\bes} (a_n b)=b \uliml{n\ra+\bes}a_n$
$\hr{\lliml{n\ra+\bes}(ba_n)=b \lliml{n\ra+\bes}a_n}$.
\end{imp*}

\begin{theorem}
Если существует $\uliml{n\ra+\bes}a_n<+\bes$ и
$\uliml{n\ra+\bes}b_n<+\bes$ ($\lliml{n\ra+\bes}a_n>-\bes$ и
$\lliml{n\ra+\bes}b_n>-\bes$) и $a_n \le b_n$ для всех $n\ge N$ и
некоторого $N\in\N$, то $\uliml{n\ra+\bes}a_n \le
\uliml{n\ra+\bes}b_n$ ($\lliml{n\ra+\bes}a_n \le
\lliml{n\ra+\bes}b_n$).
\end{theorem}

\begin{proof}
Доказательство приведём для верхних пределов. Рассмотрим
произвольное число $\ep>0$. Согласно основному свойству верхнего
предела (теорема 1.3), существует $N_{\ep} \in \N$ такое, что $b_n <
\uliml{n\ra+\bes}b_n + \ep$ для всех $n\ge N_{\ep}$. Считая
$N_{\ep}$ выбранным $N_{\ep} \ge N$, где число $N$ фигурирует в
условии теоремы, заключаем, что $a_n \le b_n < \uliml{n\ra+\bes} b_n
+ \ep$ для всех $n\ge N_{\ep}$, откуда $\uliml{n\ra+\bes}a_n \le
\uliml{n\ra+\bes}b_n + \ep$ для любого $\ep>0$, и следовательно,
$\uliml{n\ra+\bes}a_n \le \uliml{n\ra+\bes}b_n$.
\end{proof}

\subsection{Начальные сведения о числовых рядах}
\subsubsection{Понятие числового ряда. Сходящиеся ряды}

Рассмотрим произвольную числовую последовательность $(a_n)$.
Складывая один за другим её члены, получаем последовательность сумм
$s_1=a_1$, $s_2=a_1+a_2$, \ldots, $s_n=a_1+\ldots+a_n=\sumkun a_k$,
\ldots, так что $s_n=s_{n-1}+a_n$ для всех $n>1$. Поэтому процесс
образования этих сумм можно представить в виде к бесконечно
развертывающейся суммык $a_1+a_2+\ldots+a_n+\ldots$. Это не
алгебраическая сумма (в алгебре определены лишь суммы конечного
числа слагаемых), а запись процесса образования последовательности
сумм $(s_n)$.

Формальное выражение \eqa{1}{a_1+a_2+\ldots+a_n+\ldots,} порождаемое
числовой последовательностью $(a_n)$, называют \textbf{числовым
рядом}, $a_1,a_2,\ldots$ --- его членами: первым, вторым; $a_n$ ---
$n$--ым или общим членом ряда; $s_1,s_2$,\ldots ---
\textbf{частными} (или \textbf{частичными}) \textbf{суммами} ряда.
Будем ряд (1) обозначать символом $\sum a_n$. Иногда нумерацию
членов ряда начинают не с 1, а с 0.

Если последовательность $(s_n)$ частных сумм ряда $\sum a_n$
сходится к некоторому числу $s$, то этот ряд называют
\textbf{сходящимся}, а $s$ --- его \textbf{суммой}, и пишут
$s=\sumnui a_n$, или (допуская вольность в обозначениях)
$s=a_1+a_2+\ldots+a_n+\ldots$. Часто символом $\sumnui a_n$
обозначают сам ряд (1). Если последовательность $(s_n)$ не имеет
предела, ряд (1) называют \textbf{расходящимся}.

По определению, для произвольных $n,p\in\N$ справедливо
\eqa{2}{s_{n+p} = \sum\limits_{k=1}^{n+p} a_k = \sumkun a_k +
\sum\limits_{k=n+1}^{n+p} a_k = s_n + (a_{n+1} + \ldots + a_{n+p}).}

\begin{theorem}
(Критерий Коши сходимости числового ряда). Ряд (1) сходится тогда и
только тогда, когда для произвольного $\ep>0$ существует
$N_{\ep}\in\N$, что \eqa{3}{\hm{a_{n+1} + \ldots+a_{n+p}}<\ep}
справедливо для всех $n,p\in\N, \; n\ge N_{\ep}$.
\end{theorem}

\begin{proof}
Согласно (2), $\hm{a_{n+1}+\ldots+a_{n+p}}=\hm{s_{n+p}-s_{n}}$ и
свойство (3) равносильно критерию Коши сходимости последовательности
$(s_n)$, что, в свою очередь, равносильно свойству сходимости ряда
(1).
\end{proof}

\begin{imp*}
(Необходимое условие сходимости ряда). У сходящегося ряда его члены
образуют бесконечно малую последовательность; то есть, если ряд
$\sum a_n$ сходится, то $\liml{n\ra+\bes}a_n=0$.
\end{imp*}

\begin{proof}
В условии (3) выбираем $p=1$. Тогда $\hm{a_{n+1}}<\ep$ для любого
$\ep>0$ и всех $n\in\N, \; n\ge N_{\ep}$; то есть, $\liml{n\ra+\bes}
a_n=0$.
\end{proof}

\begin{ex}
Ряд $\sumnui \frac1{\sqrt{n+1}+\sqrt{n}}$ имеет
$a_n=\frac1{\sqrt{n+1}+\sqrt{n}} = \sqrt{n+1}-\sqrt{n}, \; n\in\N$,
и $s_n=a_1+a_2 + \ldots + a_n = \sqrt2 - 1 + \sqrt3 -\sqrt2 + \ldots
+ \sqrt{n+1}-\sqrt{n}=\sqrt{n+1}-1, \; n\in\N$. Ряд расходится,
поскольку $\liml{n\ra+\bes}s_n=+\bes$, но $\liml{n\ra+\bes}a_n=0$.
\end{ex}

Поэтому, необходимое условие сходимости ряда не является
достаточным.

Числовой ряд $a+aq+aq^2+\ldots+aq^{n-1} + aq^n + \ldots = \sumnui
aq^{n-1} = \sumnzi aq^n, \; a\ne0$, и $0^0=1$, если $q=0$, называют
\textbf{бесконечной геометрической прогрессией}. Его частные суммы
\equ{s_n=\sumkun aq^{n-1} = \lcmat{a\frac{1-q^n}{1-q}, \mbox{ если }
q\ne1,\\an, \mbox{ если } q=1.}}

Так как $\liml{n\ra+\bes}q^n=0$, если $\hm{q}<1$, то
$\liml{n\ra+\bes} s_n = \frac{a}{1-q}$ при $\hm{q}<1$. Если
$\hm{q}$>1, то $(q^n)$ --- бесконечно большая последовательность, и
следовательно, ряд расходится. Если $\hm{q}=1$, то общий член ряда
\equ{a_n=aq^n=\lcmat{a \mbox{ для } q=1,\\ (-1)^na \mbox{ для }
q=-1,}} и так как $a\ne0$, последовательность $(a_n)$ не является
бесконечно малой, и ряд расходится, согласно необходимому условию
сходимости ряда.

\begin{theorem}
(Линейное свойство сходящихся рядов). Если ряды $\sum a_n$ и $\sum
b_n$ сходятся к суммам $s$ и $t$, соответственно, то для любых
$\la_1,\la_2\in\R$ ряд $\sum(\la_1a_n+\la_2b_n)$ --- линейная
комбинация $\la_1\sum a_n+\la_2\sum b_n$ исходных рядов --- сходится
к сумме $\la_1s+\la_2t$.
\end{theorem}

\begin{proof}
Рассмотрим частные суммы $s_n=\sum\limits_{k=1}^m a_k, \;
t_n=\sumkun b_k$ и $\si_n = \sumkun (\la_1a_k+\la_2b_k), \; n\in\N$.
Тогда $\si_n=\la_1s_n+\la_2t_n, \; n\in\N$, и так как , по условию,
существует $\liml{n\ra+\bes}s_n=s$ и $\liml{n\ra+\bes}t_n=t$, то по
свойству линейности предела последовательности, существует
$\liml{n\ra+\bes}\si_n = \la_1\liml{n\ra+\bes}s_n +
\la_2\liml{n\ra+\bes}t_n=\la_1s+\la_2t=\si$ и число $\si$ есть сумма
ряда $\sum (\la_1a_n+\la_2b_n) = \la_1\sum a_n + \la_2 \sum b_n$.
\end{proof}

Для $k\in\N$, $k$--ым \textbf{остаточным рядом} ($k$--ым
\textbf{остатком}) ряда $\sum a_n$ называют числовой ряд
\eqa{4}{a_{k+1} + a_{k+2} + \ldots = \sumnui a_{k+n}.}

\begin{theorem}
Каждый остаточный ряд сходится или расходится одновременно с
исходным рядом и в случае сходимости \eqa{5}{\sumnui a_{k+n} =
\sumnui a_n - s_k.}
\end{theorem}

\begin{proof}
Обозначим $s_n, \; t^{(k)}_n$ --- частные $n$--ые суммы рядов (1) и
(4), соответственно. Тогда $t^{(k)}_n=a_{k+1} + \ldots + a_{k+n} =
\sums{l=1}^n a_{k+l}$ и \eqa{6}{s_{k+n} = a_1+a_2+\ldots + a_k +
a_{k+1} + \ldots + a_{k+n} = s_k+t^{(k)}_n, \; n\in\N.}

Числа $s_{k+n}, \; n\in\N$, $k$ --- фиксированное, образуют
подпоследовательность $(s_{k+n})$ последовательности $(s_n)$.

Если сходится ряд (1), то существует $\liml{n\ra+\bes}s_n=s$ и
$s=\liml{n\ra+\bes} s_{n+k}$ для любой подпоследовательности
$(s_{n+k})$. Согласно (6), существует $\liml{n\ra+\bes} t^{(k)}_n =s
- s_k, \; k\in\N$; то есть сходится любой остаток (4) ряда (1) и
справедлива формула (5). Числа $r_k = \liml{n\ra+\bes} t^{(k)}_n, \;
k\in\N$, называют суммами остатков ряда (1) и формула (5) принимает
вид $s-s_k = r_k, \; k\in\N$.

Обратно, если сходится некоторый $k$--ый остаток (4) ряда (1). то
есть, существует $\liml{n\ra+\bes} t^{(k)}_n=r_k$, то в силу формулы
(6) существует $\liml{n\ra+\bes} s_{n+k} = s_k+r_k$, так что для
произвольного $\ep>0$ существует такой индекс $N\in\N$, что
$\hm{s_{n+k} - (s_k+r_k)}<\ep$ для всех $n\ge N$. Но последнее
означает, что $\hm{s_n - (s_k+r_k)}<\ep$ для всех $n\ge N+k$, то
есть, что существует $\liml{n\ra+\bes} s_n = s = s_k+r_k$ и ряд
$\sum a_n$ сходится.
\end{proof}

\begin{imp}
Ряд сходится тогда и только тогда, когда суммы его остатков образуют
бесконечно малую последовательность.
\end{imp}

\begin{proof}
Выше отмечено, что формула (5) имеет эквивалентный вид $s=s_k+r_k,
\; k\in\N$. Поэтому $\liml{k\ra+\bes} s_k = s \; \Leftrightarrow \;
\liml{k\ra+\bes} r_k =0$.
\end{proof}

\begin{imp}
(Свойство локальности для рядов). Изменение или отбрасывание любого
конечного числа членов ряда не нарушает его сходимости или
расходимости.
\end{imp}

\begin{proof}
Поскольку изменилось или отброшено лишь конечное число членов
исходного ряда $\sum a_n$, то некоторый остаточный ряд ряда $\sum
a_n$ является остаточным (может быть, с другим номером) и для
полученного ряда $\sum b_n$. Поэтому сходимость (расходимость) этого
остаточного ряда в силу теоремы равносильна сходимости
(расходимости) и ряда $\sum a_n$ и ряда $\sum b_n$.
\end{proof}

\subsubsection{Суммирование монотонной последовательности}

\begin{theorem}\label{thm1.10}
Если неотрицательная функция $f(x)\ge0$ убывает на промежутке
$[a,+\bes)$ и имеет $\liml{x\ra+\bes} f(x)=0$, то для любого
$n\in\N$ числа $s_n = f(a)+f(a+1) + \ldots + f(a+n-1)$ и $\si_n =
\int\limits_a^{a+n} f(x)\,dx$ связаны отношением \eqa{7}{s_n = \si_n
+ c - \al_n, \; n\in\N,} в котором $0 \le c \le f(a)$ и $\al_n$ ---
бесконечно малая последовательность, $0\le \al_n \le f(a+n), \;
n\in\N$. Если, дополнительно, функция $f$ выпукла вниз, то $\frac12
f(a) \le c \le f(a)$ и $\frac12f(a+n) \le \al_n \le f(a+n)$.
\end{theorem}

\begin{proof}
Нарисуем чертёж.

\epsfbox{pictures.10}

Так как функция $f$ убывает на $[a,+\bes)$, то $f$ интегрируема на
каждом $[a,a+n]$, и следовательно числа $\si_n =
\int\limits_{a}^{a+n} f(x)\,dx$ существуют. Так как $f\ge0$ на
$[a,+\bes)$, то $\si_n$ --- площадь криволинейного трапеции
(подграфика) функции $f$ над отрезком $[a,a+n]$. Число $f(a+k+1)$,
$k\in\N$, равно площади прямоугольника со стороной $[a+k-1,a+k]$ на
оси $OX$ и высотой $f(a+k-1)$, и потому число $s_n = f(a) + f(a+1) +
\ldots + f(a+n-1)$ равно площади прямоугольной ступенчатой фигуры,
описывающей подграфик функции $f$ над $[a,a+n]$. Так что $s_n \ge
\si_n$ и $c_n = s_n - \si_n \ge 0$, $n\in\N$. Число $c_n$ равно
сумме площадей заштрихованных "криволинейных" прямоугольников (как
на рисунке). Совершим параллельный перенос их вдоль оси $OX$ влево в
прямоугольник $\Pi$ со стороной $[0,1]$ и высотой $f(a)$ и получим
оценку $0\le c_n \le f(a) = \mbox{площадь }\Pi$. Далее, $c_{n+1}\ge
c_n$, $n\in\N$, то есть $(c_n)\uparrow$ и ограничена сверху. Она
имеет $\liml{n\ra+\bes}c_n=c$ и $0\le c \le f(a)$. Так как $c=\sup
\hc{c_n \bvl n\in\N}$, то $c\ge c_n, \; n\in\N$, и $c_n=c-\al_n, \;
n\in\N$, где $(\al_n)\downarrow$ и $\liml{n\ra+\bes}\al_n=0$. Более
того, $0\le \al_n \le \mbox{площадь }\Pi_n$, где $\Pi_n$ ---
прямоугольник со стороной $[0,1]$ на оси $OX$ и высотой $f(a+n)$,
так что $0\le \al_n \le f(a+n)$, $n\in\N$, и формула (7) доказана.

Если, дополнительно, функция $f$ выпукла вниз (как на рисунке), то
заштрихованного в прямоугольниках $\Pi$ и $\Pi_n$ не меньше, чем
незаштрихованного, и следовательно, справедливы противоположные
равенства $\frac12 f(a) \le c \le f(a)$ и $\frac12 f(a+n) \le \al_n
\le f(a+n), \; n\in\N$.
\end{proof}

\subsubsection{Гармонический ряд}
Так называют ряд $1 + \frac12+\frac13 + \ldots + \frac1n + \ldots =
\sum \frac1n = \sumnui\frac1n$. Его общий член $a_n = \frac1n =
f(n), \; n\in\N$, где $f(x)=\frac1x$, $x\ge 1$, и согласно формуле
(7), \eqa{8}{s_n = 1+\frac12 + \ldots + \frac1n =
\int\limits_1^{n+1} \frac{dx}{x} + c - \al_n = \ln(n+1) + c - \al_n,
\; n\in\N,} где $\frac12 = \frac12 f(1) < c < f(1) = 1$ и
$\frac1{2(n+1)} < \frac12 f(n+1) < \al_n < f(n+1) = \frac1{n+1}, \;
n\in\N$.

Число $c$, $\frac12 < c < 1$, в формуле (8), называется константой
Эйлера (до настоящего времени не известна её природа: является ли
число $c$ рациональным, иррациональным, трансцендентным и т.п.). Так
как $\liml{n\ra+\bes} \ln(n+1)=+\bes$, то на основании (8)
заключаем, что гармонический ряд расходится. С другой стороны,
$\liml{n\ra+\bes} a_n = \liml{n\ra+\bes}\frac1n = 0.$

\subsubsection{Эталонный ряд $\sum\frac1{n^p}$}

Если $p=0$, то $a_n = \frac1{n^p} = 1, \; n\in\N$, так что
последовательность $(a_n)$ --- постоянная и не бесконечно малая, и
следовательно, ряд расходится. Если $p<0$, то $a_n =
\frac1{n^p}\ra+\bes$ при $n\ra+\bes$ и ряд расходится (в силу
необходимого признака сходимости ряда). Если $p>0$, то по теореме
пункта 2.2, в которой $a_n=\frac1{n^p}=f(n), \; n\in\N$, и
$f(x)=\frac1{x^p}$, $p>0$, $x\ge1$, имеем \equ{s_n = 1 + \frac1{2^p}
+ \frac1{3^p} + \ldots + \frac1{n^p} = \int\limits_1^{n+1}
\frac{dx}{x^p} + c_p - \al_n = \case{\frac1{p-1} - \frac1{p-1}
\hr{\frac1{n+1}}^{p-1} + c_p - \al_n, \; p\ne1, \\ \ln(n+1) + c -
\al_n, \; p=1.},} где $\liml{n\ra+\bes} \al_n=0$. Отсюда
\equ{\liml{n\ra+\bes} s_n = \case{\frac1{p-1} + c_p, \mbox{ если }
p>1,
\\ +\bes, \mbox{ если } 0 < p \le 1.}}

Таким образом, ряд $\sum \frac1{n^p}$ сходится при $p>1$ и
расходится при $p\le1$.

Если $p>1$, то по теореме пункта 2.2, $\frac1{2(n+1)^p} < \al_n <
\frac1{(n+1)^p}$ и сумма ряда $s=\frac1{p-1} + c_p$. Поэтому для
сумм $r_n, \; n\in\N$, остатков ряда имеем оценки \equ{r_n = s - s_n
= \frac1{p-1} \frac1{(n+1)^{p-1}} + \al_n \mbox{ и } \frac1{p-1}
\frac1{(n+1)^{p-1}} < r_n < \frac1{(n+1)^{p-1}} \hs{\frac1{p-1} +
\frac1{n+1}} < \frac{p}{p-1} \frac1{(n+1)^{p-1}}, \; n\in\N,}
являющиеся оценками скорости сходимости эталонного ряда
$\sum\frac1{n^p}.$

\subsubsection{Интегральный признак (Коши--Маклорена) сходимости
положительного ряда}

\begin{theorem}
Если неотрицательная функция $f(x)$ убывает на $[a,+\bes)$ и
$\liml{n\ra+\bes} f(x) = 0$, то числовой ряд $\sumnui f(a+n-1)$
сходится или расходится одновременно с несобственным интегралом
$\int\limits_a^{+\bes} f(x)\,dx$.
\end{theorem}

\begin{proof}
Согласно теореме пункта 2.2, частные суммы ряда $s_n =
\int\limits_{a}^{a+n} f(x)\,dx + c - \al_n, \; n\in\N$, где
$\liml{n\ra+\bes}\al_n=0$. Поэтому, ряд $\sum f(a+n-1)$ сходится
тогда и только тогда, когда существует $\liml{n\ra+\bes}b_n=b$, где
$b_n = \int\limits_a^{a+n} f(x)\,dx, \; n\in\N$. Так как $f(x)\ge0$,
$x\in[a,+\bes)$, то все $b_n\ge0$ и $b_{n+1}\ge b_n$, $n\in\N$. По
теореме Вейерштрасса, возрастающая последовательность $(b_n)$
сходится тогда и только тогда, когда $(b_n)$ ограничена сверху.

На $[a,+\bes)$ определена функция $F(t) = \int\limits_a^t f(x)\,dx$
и $F(t)\ge0$. $F\uparrow$ на $[0,+\bes)$. Согласно обозначениям,
$b_n = F(n+a), \; n\in\N$. Несобственный интеграл
$\int\limits_a^{+\bes} f(x)\,dx$ сходится $\Leftrightarrow$
существует $\liml{t\ra+\bes} F(t) = l \; \Leftrightarrow \; F(x)$
ограничена сверху на $[a,+\bes) \; \Leftrightarrow \; (b_n), \; b_n
= F(a+n), \; n\in\N$, ограничена сверху. Действительно, пусть $F$
ограничена сверху, то есть существует $C>0$, что $0 \le F(t) \le C$
для всех $t\in[a,+\bes)$. Рассмотрим произвольное $n\in\N$, $\exi t,
\; t>a+n$ и $F(t)\ge F(a+n)=b_n$ или $b_n \le F(t) \le C, \;
n\in\N$, то есть, $(b_n)$ ограничена сверху. Если $(b_n)$ ограничена
сверху, то есть, $0\le b_n \le C, \; n\in\N$, то для произвольного
$t\in[a,+\bes)$ существует $n\in\N$, что $n+a>t$ и $F(t) \le F(a+n)
= b_n \le C$, то есть, $F$ ограничена сверху на $[a,+\bes)$.

Итак, несобственный интеграл $\int\limits_a^{+\bes} f(x)\,dx$
сходится $\Leftrightarrow$ функция $F(t)$ ограничена сверху на
$[a,+\bes) \; \Leftrightarrow \; (b_n)$ ограничена сверху
$\Leftrightarrow$ сходится ряд $\sum f(a+n-1)$.
\end{proof}

\subsubsection{Примеры}
Ряд $\sum\limits_{n=2}^{\bes} \frac1{n\ln^qn}$ сходится при $q>1$ и
расходится при $q\le1$.

\begin{proof}
Общий член $a_n = \frac1{n\ln^qn}, \; n\in\N$, порождён функцией
$f(x) = \frac1{x\ln^qx}, \; x\in[2,+\bes)$. Так как $f'(x) =
-\frac1{x^2\ln^{q+1}x} (\ln x + q)<0$, если $x > e^{-q}$, то
$f\downarrow\downarrow$ на $[2,+\bes) \cap (e^{-q},+\bes)$, причём
$\liml{x\ra+\bes} \frac1{x\ln^qx}=0$ для любого $q\in\R$.
Следовательно, по теореме предыдущего пункта, ряд сходится или
расходится одновременно с несобственным интегралом
$\int\limits_2^{+\bes} \frac{dx}{x\ln^qx}$. Функция $F(t) =
\int\limits_2^t \frac{dx}{x\ln^qx} = \case{\frac1{1-q} (\ln t)^{1-q}
- \frac{(\ln 2)^{1-q}}{1-q}, \; q\ne1, \\ \ln \ln t - \ln \ln 2, \;
q=1}$ имеет $\liml{t\ra+\bes} F(t) = +\bes$, если $q\le1$, и
$\liml{t\ra+\bes} F(t) = \frac{(\ln 2)^{1-q}}{q-1}$, если $q>1$.
\end{proof}

\subsection{Знакопостоянные ряды}
Числовой ряд $\sum u_n$ называется знакопостоянным, если его члены
сохраняют постоянный знак. Так как ряды $\sum u_n$ и $-\sum u_n$
сходятся или расходятся одновременно, то достаточно рассматривать
только ряды с $u_n\ge0, \; n\in\N$.

\begin{df}
Числовой ряд $\sum u_n$ называется положительным, если все его члены
неотрицательны.
\end{df}

\subsubsection{Критерий сходимости положительного ряда}

\begin{theorem}\label{thm1.12}
Положительный ряд $\sum u_n, \; u_n\ge0$, сходится тогда и только
тогда, когда последовательность $(s_n)$ его частных сумм, $s_n =
\sumkun u_k, \; k\in\N$, ограничена сверху.
\end{theorem}

\begin{proof}
$s_{n+1} = s_n + u_{n+1} \ge s_n \ge 0, \; n\in\N$, то есть,
$(s_n)\uparrow$ и, следовательно, по теореме Вейерштрасса,
$\liml{n\ra+\bes} s_n = s$ существует (то есть, ряд сходится) тогда
и только тогда, когда $(s_n)$ ограничена сверху. В случае
сходимости, $s_n \le s = \sup \hc{s_n \bvl n\in\N}, \; n\in\N.$
\end{proof}

Поскольку ограниченность сверху любой возрастающей
последовательности равносильна ограниченности сверху некоторой её
подпоследовательности, то теорема \ref{thm1.12} равносильна
следующей теореме.

\begin{theorem}
Положительный ряд сходится в том и только в том случае, когда
некоторая подпоследовательность его частных сумм ограничена сверху.
\end{theorem}

\subsubsection{Общий признак сравнения положительных рядов}
\begin{theorem}
Если положительные ряды $\sum u_n$ и $\sum v_n$ обладают свойством,
что $u_n\le v_n$ для всех $n$, начиная с некоторого индекса, то из
сходимости ряда $\sum v_n$ следует сходимость ряда $\sum u_n$ (и,
следовательно, из расходимости ряда $\sum u_n$ следует расходимость
ряда $\sum v_n$).
\end{theorem}

\begin{proof}
По условию, существует такой номер $k\in\N$, что $u_{k+n} \le
v_{k+n}$ для всех $n\in\N$. Пусть $(s^{(k)}_n)$ и $(t^{(k)}_n)$
--- последовательности частных сумм рядов $\sum u_{k+n}$ и $\sum
v_{k+n}$, соответственно, так что $s^{(k)}_n \le t^{(k)}_n$ для всех
$n\in\N$. Если ряд $\sum v_n$ сходится, то его $k$--ый остаточный
ряд $\sum v_{k+n}$ тоже сходится, так что, по теореме \ref{thm1.12},
последовательность $(t^{(k)}_n)$ его частных сумм ограничена сверху.
Но тогда и последовательность $(s^{(k)}_n)$ ограничена, и значит, на
основании теоремы \ref{thm1.12}, ряд $\sum u_{k+n}$ сходится.
Следовательно, и ряд $\sum u_n$ сходится.
\end{proof}

Эта теорема позволяет иногда устанавливать сходимость или
расходимость положительного ряда путём подбора надлежащего к ряда
сравненияк, в качестве которого часто выступают эталонный ряд и
бесконечная геометрическая прогрессия.

\begin{imp}\label{imp1.3}
Если положительный ряд $\sum u_n$ сходится и $v_n = a_n u_n$,
$n\in\N$, где $0 \le a_n \le A$ для всех $n\in\N$, начиная с
некоторого номера, то сходится и ряд $\sum v_n$.
\end{imp}

\begin{imp}\label{imp1.4}
Если положительный ряд $\sum u_n$ расходится и $v_n = b_n u_n$,
$n\in\N$, где $b_n \ge b > 0$ для всех $n\in\N$, начиная с
некоторого номера, то расходится и ряд $\sum v_n$.
\end{imp}

\begin{imp}
(Предельная форма общего признака сравнения рядов). Если
положительные ряды $\sum u_n$ и $\sum v_n$ обладают свойством
$\liml{n\ra+\bes} \frac{u_n}{v_n}=l, \; 0<l<+\bes$, то ряды сходятся
или расходятся одновременно.
\end{imp}

\begin{proof}
Рассмотрим $\ep=\frac{l}2>0$ и найдём $N\in\N, \; N=N_{\ep}$, что
$\hm{\frac{u_n}{v_n} - l}<\ep$ для всех $n\ge N$, или $\frac{l}2 = l
- \ep < \frac{u_n}{v_n} < l+\ep = \frac{3l}2, \; n\ge N$, и
воспользуемся следствиями \ref{imp1.3} и \ref{imp1.4}.
\end{proof}

\subsubsection{Основной признак сравнения положительных рядов}
\begin{theorem}
Если положительные ряды $\sum u_n$ и $\sum v_n$ обладают свойством
$\frac{u_{n+1}}{u_n} \le \frac{v_{n+1}}{v_n}$ для всех $n\ge n_0$ с
некоторого $n_0\in\N$, то из сходимости ряда $\sum v_n$ следует
сходимость ряда $\sum u_n$ (и из расходимости ряда $\sum u_n$
следует расходимость ряда $\sum v_n$).
\end{theorem}

\begin{proof}
По условию, $u_n > 0, \; v_n>0$ для всех $n\ge n_0$ и
$\frac{u_n}{v_n} \ge \frac{u_{n+1}}{v_{n+1}}$ для всех $n\ge n_0$,
или $\frac{u_{n_0}}{v_{n_0}} \ge \frac{u_{n_0+n}}{v_{n_0+n}}$ для
всех $n\in\N$. Полагая $\frac{u_{n_0}}{v_{n_0}} = \rho > 0$, получим
$u_{n_0+n} \le \rho v_{n_0+n}$ и $v_{n_0+n} \ge \frac1{\rho}
u_{n_0+n}$ для всех $n\in\N$. Доказательство завершается применением
следствий \ref{imp1.3} и \ref{imp1.4} и теоремы предыдущего пункта
3.2.
\end{proof}

\subsubsection{Признак Даламбера сходимости положительных рядов}

\begin{theorem}
Если положительный ряд $\sum u_n$ обладает свойством, что числовая
последовательность $(d_n)$, $d_n = \frac{u_{n+1}}{u_n}$, имеет
$\ul{d}=\lliml{n\ra+\bes}d_n$ и $\ol{d} = \uliml{n\ra+\bes}d_n$, то
ряд сходится, когда $\ol{d}<1$ и расходится, когда $\ul{d}>1$.
Случай, когда $\ol{d} \le 1$ и $\ul{d}\ge1$, требует дополнительных
исследований.
\end{theorem}

\begin{proof}
Предположим сначала, что $\ol{d}<1$, и рассмотрим $\ep_1 =
\frac{1-\ol{d}}2>0$. Согласно основному свойству верхнего предела,
существует $n_1\in\N$, что $d_n < \ol{d} + \ep_1 = \frac{1+\ol{d}}2
= q<1$ для всех $n\ge n_1$. Тогда $\frac{u_{n+1}}{u_n} <
\frac{q^{n+1}}{q^n}=q, \; n\ge n_1$, и т.к. ряд $\sum q^n$ сходится,
то сходится и $\sum u_n$.

Если теперь $\ul{d}>1$ и $\ep_2 = \frac{\ul{d}-1}2>0$, то согласно
основному свойству нижнего предела, существует $n_2\in\N$, что $d_n
> \ul{d} - \ep_2 = \frac{\ul{d}+1}2>1$ для всех $n\ge n_2$, или
$\frac{u_{n+1}}{u_n} > 1$, $u_{n+1}>u_n>0$, $n\ge n_2$, так что
$(u_n)$ не может быть бесконечно малой последовательностью. Согласно
необходимому признаку, ряд $\sum u_n$ расходится.

Приводимые ниже примеры иллюстрируют последнее утверждение теоремы.
\end{proof}

\begin{imp}
(Предельная форма признака Даламбера). Если положительный ряд $\sum
u_n$ обладает свойством, что числовая последовательность $(d_n), \;
d_n = \frac{u_{n+1}}{u_n}, \; n\in\N$, имеет $\liml{n\ra+\bes} d_n =
d$, то ряд сходится, когда $d<1$, и расходится, когда $d>1$. Случай
$d=1$ требует дополнительного исследования.
\end{imp}

\begin{proof}
Так как существует $\liml{n\ra+\bes}d_n=d$, то $d=\ul{d}=\ol{d}$.
\end{proof}

\begin{ex}
$\sum\frac1n$ расходится. $u_n=\frac1n, \;
\frac{u_{n+1}}{u_n}=\frac{n}{n+1}\ra1, \; n\ra+\bes.$
\end{ex}

\begin{ex}
$\sum\frac1{n^2}$ сходится, $u_n=\frac1{n^2}, \; \frac{u_{n+1}}{u_n}
= \frac{n^2}{(n+1)^2} \ra 1, \; n\ra+\bes$.
\end{ex}

\begin{ex}
$a+b^2+a^3+b^4+\ldots, \; 0<a<b<1$, сходится как сумма двух
бесконечных геометрических прогрессий. Так как $u_{2k}=b^{2k}, \;
u_{2k-1}=a^{2k-1}, \; k\in\N$, то $\frac{u_{2k+1}}{u_{2k}} =
a\hr{\frac{a}{b}}^{2k} \ra 0$ при $k\ra+\bes$ и
$\frac{u_{2k+2}}{u_{2k+1}} = b\hr{\frac{b}{a}}^{2k+1} \ra +\bes$ при
$k\ra+\bes$, и следовательно, $\ul{d}=0$ и $\ul{d}=+\bes$.
\end{ex}

\begin{ex}
Рассмотрим $u_n = \case{\frac1{\sqrt{n}}, \mbox{ если } n=k^2, \;
k\in\N,\\ \frac1n, \mbox{ если } n\in\N, \; n\ne k^2.}$ Тогда ряд
$\sum u_n$ расходится и $\frac{u_{k^2+1}}{u_{k^2}} =
\frac{k}{k^2+1}\ra0$ при $k\ra+\bes$, $\frac{u_{k^2}}{u_{k^2-1}} =
\frac{k^2-1}{k} \ra +\bes$ при $k\ra+\bes$. Таким образом,
$\ul{d}=0$ и $\ol{d}=+\bes$.
\end{ex}

\subsubsection{Алгебраический признак Коши сходимости положительных
рядов}

\begin{theorem}
Рассмотрим положительный ряд $\sum u_n$ и для последовательности
$(c_n), \; c_n=\sqrt[n]{u_n}, \; n\in\N$, обозначим
$\ol{c}=\uliml{n\ra+\bes}c_n$ (возможно, $\ol{c}=+\bes$). Если
$\ol{c}<1$, ряд сходится; если $\ol{c}>1$ (в частности,
$\ol{c}=+\bes$), ряд расходится. Случай $\ol{c}=1$ требует
дополнительных исследований.
\end{theorem}

\begin{proof}
Пусть сначала $\ol{c}<1, \; 0\le\ol{c}<1$. Для числа $\ep =
\frac{1-\ol{c}}2>0$, согласно основному свойству верхнего предела,
существует $n_{\ep}\in\N$, что $c_n < \ol{c} + \ep =
\frac{1+\ol{c}}2 = q < 1$ для всех $n\ge n_{\ep}$. Таким образом,
$u^{\frac1n}_n = c_n < q, \; u_n<q^n$ для всех $n\ge n_{\ep}$. Так
как $0\le q < 1$, то ряд $\sum q^n$ сходится и, по общему признаку
сравнения положительных рядов, сходится ряд $\sum u_n$.

Пусть теперь $\ol{c}>1$ (в частности, $\ol{c}=+\bes$). Так как
$\ol{c}$ --- частичный предел числовой последовательности $(c_n)$,
то $\ol{c} = \liml{k\ra+\bes} c_{n_k}$ для некоторой
подпоследовательности $(c_{n_k})$ последовательности $(c_n)$
$(\liml{k\ra+\bes} c_{n_k} = +\bes$, если $\ol{c}=+\bes$). Поскольку
$\ol{c}>1$, существует $K\in\N$, что $c_{n_k}>1$ для всех $k\ge K$
(неравенство сохранится и в случае $\liml{k\ra+\bes} c_{n_k} =
+\bes$). Следовательно, $u^{\frac1{n_k}}_{n_k} = c_{n_k} > 1, \;
u_{n_k} > 1$ для всех $k\ge K$. Таким образом, подпоследовательность
$(u_{n_k})$ и сама последовательность $(u_n)$ не могут быть
бесконечно малыми, и ряд расходится в силу необходимого признака
сходимости ряда.

Последнее утверждение теоремы иллюстрируется примерами ниже.
\end{proof}

\begin{imp}
(Предельная форма признака Коши). Если положительный ряд $\sum u_n$
обладает свойством, что последовательность $(c_n), \;
c_n=\sqrt[n]{u_n}, \; n\in\N$ имеет $\liml{n\ra+\bes}c_n=c$, то ряд
сходится, когда $c<1$, и расходится, когда $c>1$. Случай $c=1$
требует дополнительных исследований.
\end{imp}

\begin{proof}
Так как существует $\liml{n\ra+\bes}c_n=c$, то $c=\ol{c}$.
\end{proof}

\begin{ex}
$\sum\frac1n$ расходится; $u_n=\frac1n, \; c_n=\sqrt[n]{u_n} \ra 1,
\; n\ra+\bes$.
\end{ex}

\begin{ex}
$\sum\frac1{n^2}$ сходится, $u_n=\frac1{n^2}, \;
c_n=\sqrt[n]{u_n}=\frac1{\sqrt[n]{n^2}}\ra1, \; n\ra+\bes$.
\end{ex}

\subsubsection{Признак Раабе}

\begin{theorem}
Предположим, что для положительного ряда $\sum u_n$ существует
последовательность $(p_n)$, $p_n = n\hr{\frac{u_n}{u_{n+1}}-1}, \;
n\in\N$. Если существуют число $p>1$ и индекс $n_0\in\N$, что
$p_n>p$ для всех $n\ge n_0$, то ряд сходится. Если $p_n \le 1$ для
всех $n\ge n_1$ и некоторого $n_1\in\N$, то ряд расходится.
\end{theorem}

\begin{proof}
Пусть, сначала, $p_n>p>1, \; n\ge n_0$. Тогда
\eqa{1}{\frac{u_n}{u_{n+1}} > 1 + \frac{p}n, \; n\ge n_0.} Выберем
$\al, \; 0<\al<p-1$, так что ряд $\sum v_n, \; v_n =
\frac1{n^{1+\al}}, \; n\in\N$, сходится. Имеем
\eqa{2}{\frac{v_n}{v_{n+1}} = \hr{\frac{n+1}{n}}^{1+\al} =
\hr{1+\frac1n}^{1+\al} = 1 + \frac{1+\al}{n} + o\hr{\frac1n}, \;
n\ra+\bes.} Поэтому, на основании (1) и (2),
\eqa{3}{\frac{u_n}{u_{n+1}} - \frac{v_n}{v_{n+1}} >
\frac{p-\al-1}{n} + o\hr{\frac1n} = \frac{p-\al-1}{n} \hs{1+o(1)},
\; n\ra+\bes,} и так как $p-\al-1>0$ и $\liml{n\ra+\bes}
[1+o(1)]=1>0$, то (по свойству сохранения знака предела) правая
часть в (3) положительна для всех $n\ge N$ и некоторого $N\in\N$
(выбираем $N\ge n_0$). Итак, $\frac{u_n}{u_{n+1}} -
\frac{v_n}{v_{n+1}} > 0, \; n\ge N$, и $\frac{u_{n+1}}{u_n} <
\frac{v_{n+1}}{v_n}$ для всех $n\ge N$, и следовательно, ряд $\sum
u_n$ сходится по основному признаку сравнения рядов.

Пусть теперь $p_n \le1$ для всех $n\ge n_1$. Тогда
$\frac{u_n}{u_{n+1}} \le 1 + \frac1n = \frac{n+1}n, \; n\ge n_1$, и
$\frac{u_{n+1}}{u_n} \ge \frac{n}{n+1} = \frac{\frac1n +
1}{\frac1n}$ для всех $n\ge n_1$. Так как ряд $\sum\frac1n$
расходится, то расходится и ряд $\sum u_n$.
\end{proof}

\begin{imp}
Если $(p_n)$ в теореме имеет $\lliml{n\ra+\bes} p_n = \ul{p}>1$, то
ряд $\sum u_n$ сходится. Если $\uliml{n\ra+\bes}p_n = \ol{p} < 1$,
то ряд $\sum u_n$ расходится.
\end{imp}

\begin{proof}
Для $\ep_1 = \frac{\ul{p}-1}2 > 0$, согласно основному свойству
нижнего предела, существует $n_1\in\N$, что $p_n > \ul{p} - \ep_1 =
\frac{1+\ul{p}}2 > 1$ для всех $n\ge n_1$, и ряд сходится по
теореме. Для $\ep_2 = \frac{1-\ol{p}}2>0$, согласно основному
свойству верхнего предела, существует $n_2\in\N$, что $p_n < \ol{p}
+ \ep_2 = \frac{1+\ol{p}}2 < 1$ для всех $n\ge n_2$, и ряд
расходится по теореме.
\end{proof}

\begin{imp}
(Предельная форма признака Раабе). Если для положительного ряда
$\sum u_n$ последовательность $(p_n)$,
$p_n=n\hr{\frac{u_n}{u_{n+1}}-1}, \; n\in\N$, имеет
$\liml{n\ra+\bes} p_n = p$, то ряд сходится, когда $p>1$, и
расходится, когда $p<1$; случай $p=1$ требует дополнительных
исследований.
\end{imp}

\begin{proof}
По критерию существования предела последовательности,
$p=\liml{n\ra+\bes}p_n = \ul{p} = \lliml{n\ra+\bes}p_n = \ol{p} =
\uliml{n\ra+\bes}p_n$, и применяем предыдущее следствие.
\end{proof}

\begin{ex}
Ряд $\sum\frac1n$ расходится, $u_n=\frac1n, \; p_n =
n\hr{\frac{u_n}{u_{n+1}}-1}=1, \; n\in\N$, и $\liml{n\ra+\bes}p_n =
1.$
\end{ex}

\begin{ex}
Ряд $\sum \frac1{n^2\ln^2n}$ сходится, но $\liml{n\ra+\bes}p_n=1$,
где $p_n = n\hr{\frac{u_n}{u_{n+1}}-1}, \; u_n = \frac1{n\ln^2n}, \;
n\ge2$.
\end{ex}

\begin{note}
Если существует $\liml{n\ra+\bes}p_n = \liml{n\ra+\bes}
n\hr{\frac{u_n}{u_{n+1}}-1}=p$ и так как последовательность $(n)$
--- бесконечно большая, то $\liml{n\ra+\bes}
\hr{\frac{u_n}{u_{n+1}}-1}=0$, и следовательно, $\liml{n\ra+\bes}
\frac{u_{n+1}}{u_n} = 1$. Поэтому, в случае $p\ne1$ признак Раабе
даёт ответ о сходимости или расходимости ряда $\sum u_n$.
\end{note}

\subsubsection{Признак Гаусса}
\begin{theorem}
Предположим, что положительный ряд $\sum u_n$ обладает свойством
\eqa{4}{\frac{u_n}{u_{n+1}} = \la + \frac{\mu}{n} +
\frac{\ta_n}{n^{1+\si}}, \; n\in\N,} где $\la,\mu,\si\in\R$ и
$\si>0$, а последовательность $(\ta_n)$ ограничена. Тогда
\begin{nums}{-3}
\item если $\la>1$, ряд сходится, и если $\la<1$ --- расходится;
\item если $\la=1$, то ряд сходится при $\mu>1$ и расходится при
$\mu<1$;
\item при $\la=\mu=1$ ряд расходится.
\end{nums}
\end{theorem}

\begin{proof}
Так как $(\ta_n)$ ограничена, то $\liml{n\ra+\bes}
\frac{u_n}{u_{n+1}} = \la$, и утверждение 1 --- предельная форма
признака Даламбера. Если $\la=1$, то (4) принимает вид
\eqa{5}{\frac{u_n}{u_{n+1}} = 1 + \frac{\mu}{n} +
\frac{\ta_n}{n^{1+\si}}, \; n\in\N,} и следовательно,
$\liml{n\ra+\bes}n \hr{\frac{u_n}{u_{n+1}} - 1} = \mu$, так что
утверждение 2 --- предельная форма признака Раабе. Пусть, наконец,
$\la=\mu=1$, то есть (5) переходит в \eqa{6}{\frac{u_n}{u_{n+1}} = 1
+ \frac1n + \frac{\ta_n}{n^{1+\si}}, \; n\in\N.}

Рассмотрим расходящийся ряд $\sum\frac1{n\ln n}$ и обозначим $v_n =
\frac1{n\ln n}, \; n\ge 2.$ Тогда \mla{7}{\frac{v_n}{v_{n+1}} =
\frac{n+1}n \frac{\ln(n+1)}{\ln n} = \hr{1+\frac1n} \frac{\ln n +
\ln \hr{1 + \frac1n}}{\ln n} = \hr{1+\frac1n} \frac{\ln n + \frac1n
+ o\hr{\frac1n}}{\ln n} = \\ = \hr{1+\frac1n} \hs{1 + \frac1{n\ln n}
+ o\hr{\frac1{n\ln n}}} = 1 + \frac1n + \frac1{n\ln n} +
\frac1{n^2\ln n} + o\hr{\frac1{n\ln n}} = 1+\frac1n + \frac1{n\ln n}
+ o\hr{\frac1{n\ln n}}, \; n\ra+\bes.} На основании (6) и (7), имеем
\eqa{8}{\frac{u_n}{u_{n+1}} - \frac{v_n}{v_{n+1}} = -\frac1{n\ln n}
+ \frac{\ta_n}{n^{1+\si}} + o\hr{\frac1{n\ln n}} = -\frac1{n\ln n}
\hs{1 - \frac{\ta_n \ln n}{n^{\si}} + o(1)}, \; n\ra+\bes.} Так как
$\liml{n\ra+\bes}\hs{1 - \frac{\ta_n \ln n}{n^{\si}} + o(1)}=1>0$
(последовательность $(\ta_n)$ ограничена, а
$\liml{n\ra+\bes}\frac{\ln n}{n^{\si}}=0, \; \si>0$), то правая
часть в (8) отрицательна для всех $n\ge n_{\ep}$ с некоторым
$n_{\ep}\in\N$. Таким образом, $\frac{u_n}{u_{n+1}} <
\frac{v_n}{v_{n+1}}, \; n\ge n_{\ep}$, или $\frac{u_{n+1}}{u_n} \ge
\frac{v_{n+1}}{v_n}, \; n\ge n_{\ep}$, и следовательно, ряд $\sum
u_n$ расходится.
\end{proof}

\subsection{Абсолютно и условно сходящиеся числовые ряды}
\subsubsection{Понятие абсолютно сходящегося ряда}

\begin{dfn}{1}
Числовой ряд $\sum a_n$ называют абсолютно сходящимся, если сходится
ряд $\sum \hm{a_n}$ из абсолютных величин $\hm{a_n}$ членов $a_n$
ряда.
\end{dfn}

Любой сходящийся знакопостоянный ряд сходится абсолютно.

\begin{theorem}
  Любой абсолютно сходящийся ряд сходится.
\end{theorem}

\begin{proof}
Рассмотрим произвольный абсолютно сходящийся ряд $\sum a_n$, то
есть, сходящийся ряд $\sum\hm{a_n}$, и произвольное число $\ep>0$.
По критерию Коши для сходящегося ряда $\sum\hm{a_n}$, существует
$n_{\ep}\in\N$, что $\hm{\hm{a_{n+1}} + \ldots + \hm{a_{n+p}}} <
\ep$ для всех $n,p\in\N$, $n\ge n_{\ep}$. Но тогда $\hm{a_{n+1} +
\ldots + a_{n+p}} \le \hm{a_{n+1}} + \ldots + \hm{a_{n+p}} =
\hm{\hm{a_{n+1}} + \ldots + \hm{a_{n+p}}} < \ep$ для всех $n,p
\in\N, \; n\ge n_{\ep}$, и согласно критерию Коши, сходится ряд
$\sum a_n$.
\end{proof}

\subsubsection{Линейное свойство абсолютно сходящихся рядов}
\begin{theorem}
Сумма абсолютно сходящихся рядов образует абсолютно сходящийся ряд.
\end{theorem}

\begin{proof}
Пусть ряды $\sum a_n$ и $\sum b_n$ сходятся абсолютно и число
$\ep>0$
--- произвольное. Так как сходятся ряды $\sum \hm{a_n}$ и $\sum
\hm{b_n}$, то по критерию Коши, существует $n_{\ep}\in\N$, что
$\hm{a_{n+1}} + \ldots + \hm{a_{n+p}} < \frac{\ep}2$ и $\hm{b_{n+1}}
+ \ldots + \hm{b_{n+p}} < \frac{\ep}2$ для всех $n,p\in \N, \; n\ge
n_{\ep}$. Поэтому, $\hm{a_{n+1} + b_{n+1}} + \ldots + \hm{a_{n+p} +
b_{n+p}} \le \hm{a_{n+1}} + \hm{b_{n+1}} \spl \hm{a_{n+p}} +
\hm{b_{n+p}} < \frac{\ep}2 + \frac{\ep}2 = \ep$ для всех $n,p\in\N,
\; n\ge n_{\ep}$, так что, по критерию Коши, сходится ряд $\sum
\hm{a_n+b_n}$, то есть, ряд $\sum(a_n+b_n)$
--- сумма рядов $\sum a_n$ и $\sum b_n$ --- сходится абсолютно.
\end{proof}

\begin{theorem}
Если ряд $\sum a_n$ абсолютно сходится и последовательность $(M_n)$
ограничена, то ряд $\sum M_n a_n$ сходится абсолютно.
\end{theorem}

\begin{proof}
По условию теоремы, существует такое число $C>0$, что $\hm{M_n} \le
C$ справедливо для всех $n\in N_0$. Поэтому, $\hm{M_{n+1} a_{n+1}} +
\ldots + \hm{M_{n+p}a_{n+p}} = \hm{M_{n+1}} \hm{a_{n+1}} + \ldots +
\hm{M_{n+p}} \hm{a_{n+p}} \le C \hr{\hm{a_{n+1}} + \ldots +
\hm{a_{n+p}}}$ справедливо для всех $n,p\in \N$. Рассмотрим
произвольное $\ep>0$. Так как по условию теоремы сходится ряд
$\sum\hm{a_n}$, то существует такое $n_{\ep} \in\N$, что
$\hm{a_{n+1}} + \ldots + \hm{a_{n+p}} < \frac{\ep}{C}$ для всех
$n,p\in\N, \; n\ge n_{\ep}$. Следовательно, $\hm{M_{n+1}a_{n+1}} +
\ldots + \hm{M_{n+p}a_{n+p}} < C \frac{\ep}{C} = \ep$ для всех $n,p
\in\N, \; n\ge n_{\ep}$, и по критерию Коши, сходится ряд $\sum
\hm{M_n a_n}$, то есть, ряд $\sum M_n a_n$ сходится абсолютно.
\end{proof}

\subsubsection{Признаки абсолютной сходимости числовых рядов}
\begin{theorem}
(Признак Даламбера). Если числовой ряд $\sum a_n$ обладает
свойством, что существует $\lliml{n\ra+\bes}
\hm{\frac{a_{n+1}}{a_n}} = \ul{\la}$ и $\uliml{n\ra+\bes} =
\ol{\la}$, то ряд сходится абсолютно, когда $\ol{\la}<1$, и ряд
расходится, когда $\ul{\la}>1$.
\end{theorem}

\begin{proof}
Так как $\hm{\frac{a_{n+1}}{a_n}} = \frac{\hm{a_{n+1}}}{\hm{a_n}},
\; n\in\N$, то из условия $\ol{\la}<1$, по признаку Даламбера
следует, что сходится ряд $\sum\hm{a_n}$, то есть ряд $\sum a_n$
сходится абсолютно. Если $\ul{\la}>1$, то в доказательстве признака
Даламбера было установлено, что $\hm{a_{n+1}}>\hm{a_n}$ для всех
$n\in\N, \; n\ge n_1$ и некоторого $n_1\in\N$, так что
последовательности $(\hm{a_n})$ и $(a_n)$ не могут быть бесконечно
малыми, и поэтому ряды $\sum\hm{a_n}$ и $\sum a_n$ расходятся.
\end{proof}

\begin{theorem}
(Признак Коши). Если для числового ряда $\sum a_n$ рассмотреть
$\uliml{n\ra+\bes} \sqrt[n]{\hm{a_n}} = \ol{\mu}$, то ряд абсолютно
сходится, когда $\ol{\mu}<1$, и ряд расходится, когда $\ol{\mu}>1$
(в частности, если $\ol{\mu}=+\bes$).
\end{theorem}

\begin{proof}
Рассмотрим положительный ряд $\sum \hm{a_n}$. Согласно признаку
Коши, этот ряд сходится, если $\ol{\mu}<1$, то есть, ряд $\sum a_n$
абсолютно сходится при $\ol{\mu}<1$. Если $\ol{mu}>1$ (в частности,
если $\ol{\mu}=+\bes)$, то в доказательстве признака Коши было
отмечено, что последовательность $(\hm{a_n})$ не может быть
бесконечно малой, а вместе с ней и $(a_n)$ не является бесконечно
малой последовательностью, так что по необходимому признаку, ряд
$\sum a_n$ расходится.
\end{proof}

\begin{imp*}
(Предельная форма признаков Даламбера и Коши). Если у числового ряда
$\sum a_n$ существуют $\liml{n\ra+\bes}
\hm{\frac{a_{n+1}}{a_n}}=\la$ или
$\liml{n\ra+\bes}\sqrt[n]{\hm{a_n}}=\mu$, то ряд абсолютно сходится,
когда $\la<1$ или $\mu<1$, и ряд расходится, когда $\la>1$ или
$\mu>1$; случай $\la=1$ или $\mu=1$ требует дополнительных
исследований.
\end{imp*}

\begin{proof}
Согласно условию и критерию существования предела
последовательности, $\la = \ul{\la} = \ol{\la} = \uliml{n\ra+\bes}
\hm{\frac{a_{n+1}}{a_n}} =
\lliml{n\ra+\bes}\hm{\frac{a_{n+1}}{a_n}}$ и $\mu=\ol{\mu} =
\uliml{n\ra+\bes} \sqrt[n]{\hm{a_n}}$. Применяем предыдущие теоремы.
\end{proof}

\subsubsection{Условно сходящиеся ряды}

Так называют числовые ряды $\sum a_n$, которые сходятся, но не
сходятся абсолютно; то есть, расходится положительный ряд
$\sum\hm{a_n}$.

\begin{theorem}
(Тождество Абеля). Рассмотрим последовательности $(a_n), \; (\al_n)$
и $(s_n)$, $s_0=0, \; s_1=a_1, \ldots, s_n = \sumkun a_k, \;
n\in\N$. Тогда для произвольных $m,n\in\N, \; m>n$, справедливо
тождество \equ{\sum\limits_{k=n}^m \al_k a_k =
\sum\limits_{k=n}^{m-1} s_k (\al_k - \al_{k+1}) + \al_m s_m - \al_n
s_{n-1}.}
\end{theorem}

\begin{proof}
\ml{\sum\limits_{k=n}^m \al_k a_k = \sum\limits_{k=n}^m \al_k (s_k -
s_{k-1}) = \al_n (s_n - s_{n-1}) + \al_{n+1} (s_{n+1}-s_n) + \ldots
+ \al_m (s_m - s_{m-1}) = s_n (\al_n - \al_{n+1}) +\\+ s_{n+1}
(\al_{n+1} - \al_{n+2}) + \ldots + s_{m-1} (\al_{m-1} - \al_m) +
\al_m s_m - \al_n s_{n-1} = \sum\limits_{k=n}^{m-1} s_k (\al_k -
\al_{k+1}) + \al_m s_m - \al_n s_{n-1}.}
\end{proof}

\subsubsection{Признак Дирихле сходимости числового ряда}
\begin{theorem}
Если
\begin{nums}{-2}
\item последовательность $(s_n)$ частных сумм ряда $\sum a_n$
ограничена;
\item последовательность $(\al_n)$ убывает и бесконечно мала, то
есть $\liml{n\ra+\bes} \al_n =0$,
\end{nums} то ряд $\sum \al_n a_n$ сходится.
\end{theorem}

\begin{proof}
Проверим выполнение критерия Коши для ряда $\sum\al_n a_n$.
Рассмотрим произвольное $\ep>0$ и произвольные $m,n\in\N, \; m>n.$
Согласно тождеству Абеля, \equ{\hm{\sum\limits_{k=n}^m \al_k a_k}
\le \sum\limits_{k=n}^{m-1} \hm{s_k} \hm{\al_k - \al_{k+1}} +
\hm{s_m} \hm{\al_m} + \hm{s_{n-1}}\hm{\al_n}.} По условию 1 теоремы,
существует такое $M>0$, что $\hm{s_k} \le M$ для всех $k\in\N$, и
поэтому, \eqa{1}{\hm{\sum\limits_{k=n}^m \al_k a_k} \le M
\hs{\sum\limits_{k=n}^{m-1} \hm{\al_k - \al_{k+1}} + \hm{\al_m} +
\hm{\al_n}}.} Так как последовательность $(\al_n)$ убывает, то
$\hm{\al_k - \al_{k+1}} = \al_k - \al_{k+1} \ge 0, \; k\in\N$, и
неравенство (1) принимает вид \mla{2}{\hm{\sum\limits_{k=n}^m \al_k
a_k} \le M \hs{\sum\limits_{k=n}^{m-1} (\al_k - a_{k+1}) +
\hm{\al_m} + \hm{\al_n}} = M \hs{\al_n - \al_{n+1} + \al_{n+1} -
\al_{n+2} + \ldots + \al_{m-1} - \al_m + \hm{\al_m} + \hm{\al_n}}=\\
= M (\al_n + \hm{\al_n} - \al_m + \hm{\al_m}) \le 2M(\hm{\al_n} +
\hm{\al_m}).}

Так как $\liml{n\ra+\bes}\al_n=0$, то для $\ep>0$ существует такой
$n_{\ep}\in\N$, что $\hm{\al_n} < \frac{\ep}{4M}$ для всех $n\in\N,
\; n\ge n_{\ep}$. Поэтому, $\hm{\al_m} < \frac{\ep}{4M}$ для всех
$m>n \ge n_{\ep}$, что является критерием Коши сходимости ряда $\sum
\al_n a_n.$
\end{proof}

\subsubsection{Признак Абеля сходимости числового ряда}
\begin{theorem}
Если
\begin{nums}{-2}
\item ряд $\sum a_n$ сходится и
\item последовательность $(\al_n)$ монотонна и ограничена,
\end{nums}
то ряд $\sum \al_n a_n$ сходится.
\end{theorem}

\begin{proof}
Согласно условию 1 теоремы, последовательность $(s_n)$ частных сумм
ряда $\sum a_n$ ограничена (как сходящаяся последовательность,
$\liml{n\ra+\bes}s_n=s$ --- сумма ряда $\sum a_n$). Пусть, для
определенности, последовательность $(\al_n)$ возрастает. Согласно
теореме Вейерштрасса, существует $\liml{n\ra+\bes}\al_n=\al$ и
$\al_n \le \al_{n+1} \le \al$ для всех $n\in\N$. Следовательно,
$\be_n = \al - \al_n \ge 0, \; \be_{n+1} = \al - \al_{n+1} \le \al -
\al_n = \be_n, \; n\in\N$, и $\liml{n\ra+\bes}\be_n=0$. Согласно
признаку Дирихле, сходится ряд $\sum \be_n a_n = \sum (\al-\al_n)
a_n$. Согласно условию 1 теоремы, сходится ряд $\sum \al a_n$, и
следовательно, в силу свойства линейности сходящихся рядов, сходится
ряд $\sum \al a_n - \sum (\al - \al_n) a_n = \sum \al_n a_n$.
\end{proof}

\subsubsection{Знакочередующиеся ряды}
Так называют ряды вида $\sum (-1)^{n+1} u_n = u_1 - u_2 + u_3 - u_4
+ \ldots$, у которых $u_n \ge0, \; n\in\N$.

\begin{theorem}
(Признак Бернулли--Лейбница). Знакочередующийся ряд $\sum (-1)^{n+1}
u_n$ сходится, если последовательность $(u_n)$ убывает и бесконечно
мала. При этом, сумма каждого остатка ряда по абсолютной величине не
превосходит абсолютной величины первого своего члена и совпадает с
ним по знаку; то есть $\hm{r_n} \le u_{n+1}, \; n\in\N$, и $\sgn r_n
= (-1)^{n+1}$. В частности, для суммы $s$ ряда $\sum (-1)^{n+1} u_n$
справедлива оценка $0 \le s \le u_1$.
\end{theorem}

\begin{proof}
Ряд $\sum (-1)^{n+1} u_n$ сходится по признаку Дирихле, в котором
$\al_n=u_n$ и $a_n = (-1)^{n+1}, \; n\in\N$ (частные суммы ряда
$\sum (-1)^{n+1}$ равны 1 или 0; то есть, образуют ограниченную
последовательность). Обозначим его сумму через
$s=\sum\limits_{n=1}^{\bes} (-1)^{n+1} u_n$ и докажем неравенство $0
\le s \le u_1$. На рисунке
%рисунок...
видно, что числа $u_1 - u_2, \; u_3-u_4, \ldots, u_{2n-1} - u_{2n}$
равны длинам отрезков, взаимно друг на друга не налагаемых и
содержащихся в отрезке $[0,u_1]$ длины $u_1$. Поэтому, $0\le s_{2n}
= u_1-u_2+u_3-u_4 + \ldots + u_{2n-1} - u_{2n} \le u_1$ и так как
$s=\liml{n\ra+\bes} s_n = \liml{n\ra+\bes} s_{2n}$, то $0\le s \le
u_1$.

Каждый $n$--ый остаток ряда $\sum (-1)^{n+1}u_n$ является
знакочередующимся рядом, удовлетворяющим условиям теоремы, и по
доказанному модуль $\hm{r_n}$ его суммы удовлетворяет неравенству
$\hm{r_n} \le \hm{(-1)^{n+1} u_{n+1}} = u_{n+1}$.
\end{proof}

\begin{ex}
Найдём сумму ряда Лейбница $\sum (-1)^{n+1} \frac1n = 1-\frac12 +
\frac13 - \frac14 + \ldots.$
\end{ex}

\begin{proof}
Ряд сходится по признаку Бернулли--Лейбница. Кроме того, имеем
\ml{s_{2n} = 1 - \frac12 + \frac13 - \frac14 + \ldots + \frac1{2n-1}
- \frac1{2n} = \hr{1+\frac12+\frac13+\ldots+\frac1{2n}} -
2\hr{\frac12+\frac14+\ldots+\frac1{2n}} = \ln(2n+1)+c-\\-\al_{2n} -
\hr{1+\frac12+\ldots+\frac1n} = \ln(2n+1) + c - \al_{2n} - \ln(n+1)
- c + \al_n = \ln \frac{2n+1}{n+1} + \be_n, \; \be_n = \al_n -
\al_{2n},} где использована формула суммирования из теоремы
\ref{thm1.10}. Поскольку $\liml{n\ra+\bes} \be_n = \liml{n\ra+\bes}
(\al_n - \al_{2n})=0$, то $s=\liml{n\ra+\bes} s_n = \liml{n\ra+\bes}
s_{2n} = \liml{n\ra+\bes} \hr{\ln\frac{2n+1}{n+1} + \be_n} = \ln2.$
Итак, $\sum\limits_{n=1}^{\bes} (-1)^{n+1} \frac1n = \ln2.$
\end{proof}

Отметим, что ряд Лейбница сходится условно, он не сходится
абсолютно, поскольку расходится гармонический ряд $\sum\frac1n$.

\subsection{Действия над рядами}

Напомним, что ряд $\sum a_n$ сходится тогда и только тогда, когда
сходятся ряды $\sum \la a_n$ для любого $\la\ne0$.

\subsubsection{Сочетательное свойство сходящихся рядов}

\begin{theorem}
Если числовой ряд \eqa{1}{\sum a_n = a_1+a_2+\ldots+a_n+\ldots}
сходится к сумме $s$, то для произвольной возрастающей
последовательности $(n_k), \; n_k\in\N, \; k\in\N,$ ряд
\eqa{2}{(a_1+a_2+\ldots+ a_{n_1})+(a_{n_1+1}+\ldots+a_{n_2}) +
\ldots + (a_{n_k+1} + \ldots + a_{n_k+1}) + \ldots} сходится к сумме
$s$. Если для возрастающей последовательности $(n_k)$ все слагаемые
в каждой из скобок ряда (2) имеют одинаковые знаки; то есть,
$a_m\cdot a_l\ge0, \; n_k+1\le m,l\le n_{k+1}, \; k\in\N$, и ряд (2)
сходится, то ряд (1) сходится к сумме ряда (2).
\end{theorem}

\begin{proof}
Рассмотрим $s_n=\sumkun a_k, \; n\in\N$; тогда
$s=\liml{n\ra+\bes}s_n$. Обозначим $\si_k = (a_1+a_2+\ldots+
a_{n_1}) + \ldots + (a_{n_{k-1}+1} + \ldots + a_{n_k}), \; k\in\N$.
Тогда $\si_k = s_{n_k}, \; k\in\N$, и следовательно,
$s=\liml{n\ra+\bes} s_n = \liml{k\ra+\bes} s_{n_k} =
\liml{k\ra+\bes} \si_k$; то есть, ряд (2) сходится к сумме $s$.

Предположим теперь, что сходится ряд (2); то есть, существует
$\liml{k\ra+\bes} \si_k=\si$. Рассмотрим произвольное $n\in\N, \;
n>n_1$. Тогда существует единственное число $k\in\N$ такое, что $n_k
\le n \le n_{k+1}$. Поэтому, $s_n = \si_k + (a_{n_k+1} + \ldots +
a_n)$ и $a_m\cdot a_l \ge 0$ для всех $n_k+1\le m, l \le n \le
n_{k+1}$. Следовательно, \eqa{3}{\hm{s_n - \si_k} = \hm{a_{n_k+1} +
\ldots + a_n} \le \hm{a_{n_k+1} + \ldots + a_{n_k+1}} =
\hm{\si_{k+1} - \si_k}.} Так как $\liml{k\ra+\bes} \si_k =
\liml{k\ra+\bes} \si_{k+1} = \si$, то $\liml{k\ra+\bes}
\hm{\si_{k+1} - \si_k}=0$ и для произвольного $\ep>0$ существует
$k_{\ep}\in\N$, что $\hm{\si_k - \si}<\frac{\ep}2$ и $\hm{\si_{k+1}
- \si_k} < \frac{\ep}2$ для всех $k>k_{\ep}$, то есть всех $k+1\ge
k_{\ep}$. Положим $N_{\ep} = n_{k_{\ep}}$. Тогда для любого $n\ge
N_{\ep}$ и такого $k\in\N$, что $n_k \le n \le n_{k+1}$, имеем
$n_{k+1} \ge n \ge N_{\ep} = n_{k_{\ep}}$ и $k+1\ge k_{\ep}$.
Поэтому, с учётом (3), \equ{\hm{s_n-\si} = \hm{s_n - \si_k + \si_k -
\si} \le \hm{s_n - \si_k} + \hm{\si_k - \si} \le \hm{\si_{k+1} -
\si_k} + \hm{\si_k - \si} < \frac{\ep}2 + \frac{\ep}2 = \ep} для
любого $n\ge N_{\ep}$, так что $\si = \liml{n\ra+\bes} s_n$.
\end{proof}

\subsubsection{Переместительный закон для абсолютно сходящихся
рядов}
\begin{theorem}
(О. Коши) Если ряд $\sum a_n$ абсолютно сходится к сумме $s$, то
любая перестановка его членов приводит к ряду, абсолютно сходящемуся
к $s$.
\end{theorem}

\begin{proof}
Рассмотрим вначале случай, когда все $a_n\ge0$, и рассмотрим
произвольную перестановку $(1,2,\ldots,n,\ldots) \ra
(n_1,n_2,\ldots,n_k,\ldots)$, которая приводит к ряду $\sum
a_{n_k}$. Рассмотрим произвольное $k\in\N$ и частную сумму
$\sum\limits_{j=1}^k a_{n_j}$. Положим $m=\max(n_1,\ldots,n_k)$.
Тогда $\sum\limits_{j=1}^k a_{n_j} \le \sum\limits_{l=1}^m a_l \le
\sum\limits_{n=1}^{\bes} a_n = s$ на основании неотрицательности
членов $a_n$ и критерия сходимости положительных рядов. На том же
основании сходится ряд $\sum\limits_{j=1}^{\bes} a_{n_j}$ и его
сумма $s'<s$ (то есть, при перестановке слагаемых сумма не
увеличивается). Но ряд $\sum a_n$ получается из $\sum a_{n_k}$
обратной перестановкой и поэтому $s\le s'$, так что $s'=s$.

В общем случае заметим, что $a_n = (\hm{a_n} + a_n) - \hm{a_n}$, и
так как $0 \le \hm{a_n} + a_n \le 2\hm{a_n}$, то по признаку
сравнения из сходимости ряда $\sum \hm{a_n}$ (по условию теоремы)
следует сходимость положительного ряда $\sum (\hm{a_n} + a_n)$.
Тогда \equ{\sum\limits_{n=1}^{\bes} a_n = \sum\limits_{n=1}^{\bes}
(\hm{a_n} + a_n) - \sum\limits_{n=1}^{\bes} \hm{a_n} =
\sum\limits_{j=1}^{\bes} (\hm{a_{n_j}} + a_{n_j}) -
\sum\limits_{j=1}^{\bes} \hm{a_{n_j}} = \sum\limits_{j=1}^{\bes}
a_{n_j}.}
\end{proof}

\subsubsection{Свойства членов условно сходящихся рядов}
\begin{theorem}\label{thm1.31}
Всякий условно сходящийся ряд содержит бесконечно много
положительных и отрицательных членов.
\end{theorem}

\begin{proof}
Рассмотрим произвольный условно сходящийся ряд $\sum a_n$, так что
$\sum a_n$ сходится, а $\sum \hm{a_n}$ расходится. Если ряд содержит
конечное множество отрицательных членов, то все $a_n\ge0$, начиная с
некоторого $N\in\N$, и поэтому $a_n=\hm{a_n}$ для всех $n\ge N$. Так
как ряд $\sum a_n$ сходится, то сходится его остаток
$\sum\limits_{n=N}^{\bes} a_n = \sum\limits_{n=N}^{\bes} \hm{a_n}$,
что противоречит расходимости ряда $\sum \hm{a_n}$. Далее, ряд $\sum
(-a_n)$ сходится условно, так как $\hm{-a_n} = \hm{a_n}, \; n\in\N$,
и по доказанному имеет бесконечное множество отрицательных членов,
которые являются положительными членами исходного ряда $\sum a_n$.
\end{proof}

Пусть ряд $\sum a_n$ сходится условно. Обозначим $u_k = a_{n_k}
\ge0$ и $v_l = a_{n_l} < 0$. Согласно теореме \ref{thm1.31},
последовательности $(u_k)$ и $(v_l)$ содержат бесконечно много
различных членов.

\begin{theorem}\label{thm1.32}
Если ряд $\sum a_n$ сходится условно, то знакопостоянные ряды $\sum
u_k$ и $\sum v_l$ расходятся.
\end{theorem}

\begin{proof}
По условию теоремы, $\liml{n\ra+\bes} s_n = s$ и $\liml{n\ra+\bes}
\si_n = +\bes$, где $s_n = \sumkun a_k, \; \si_n = \sumkun \hm{a_k},
\; n\in\N$. Рассмотрим $p_n = \frac12 (\si_n + s_n), \; q_n =
\frac12 (s_n - \si_n), \; n\in\N$. Тогда $p_n, \; n\in\N$ --- сумма
всех неотрицательных членов $a_k$, входящих в частную сумму $s_n$
ряда $\sum a_n$, а $q_n, \; n\in\N$ --- сумма всех отрицательных
членов $a_k$, входящих в $s_n$. Таким образом, последовательности
$(p_n)$ и $(q_n)$ будут некоторыми подпоследовательностями частных
сумм рядов $\sum u_k$ и $\sum v_l$, соответственно. Так как, по
определению, $\liml{n\ra+\bes} p_n = +\bes, \; \liml{n\ra+\bes} q_n
= -\bes$, то ряды $\sum u_k$ и $\sum v_l$ расходятся.
\end{proof}

\begin{theorem}
(Риман). Если числовой ряд сходится условно, то изменяя порядок его
членов, можно заставить ряд иметь своей суммой любое действительное
число или расходиться.
\end{theorem}

\begin{proof}
Рассмотрим произвольный условно сходящийся ряд $\sum a_n$ и
последовательности $(p_n)$ и $(q_n)$ его неотрицательных и
отрицательных членов, которые бесконечны по теореме \ref{thm1.31}.
Рассмотрим сначала произвольное число $C$, для определённости,
$C\ge0$. По теореме \ref{thm1.32}, последовательность $(A_n), \;
A_n=p_1+p_2+\ldots+p_n, \; n\in\N$, частных сумм ряда $\sum p_n$ ---
положительная бесконечно большая; $\liml{n\ra+\bes} A_n = +\bes$.
Следовательно, найдётся такой номер $n_1\in\N$, что будут верны
неравенства $A_{n_1} > C \ge A_{n_1} - p_{n_1}$. Последовательность
$(B_m), \; B_m = A_{n_1} + q_1 + q_2 + \ldots + q_m, \; m\in\N$, по
той же теореме \ref{thm1.32}, является отрицательной бесконечно
большой, $\liml{m\ra+\bes}B_m=-\bes$. Следовательно, найдётся такой
номер $m_1\in\N$, что $B_{m_1} < C \le B_{m_1} - q_{m_1}$.
Последовательность $(A_{n_1+n})$, $A_{n_1+n} = B_{m_1} + p_{n_1+1} +
\ldots + p_{n_1+n}, \; n\in\N$, как и выше по теореме \ref{thm1.32},
является положительной бесконечно большой; $\liml{n\ra+\bes}
A_{n_1+n} = +\bes$, и следовательно, найдётся такой номер
$n_2\in\N$, что будут верны неравенства $A_{n_2} > C \ge A_{n_2} -
p_{n_2}$. Продолжая этот процесс поочерёдного присоединения групп
неотрицательных и отрицательных слагаемых, получим ряд
\eqa{3}{(p_1+p_2 + \ldots + p_{n_1}) + (q_1+q_2 + \ldots + q_{m_1})
+ (p_{n_1+1} + \ldots + p_{n_2}) + \ldots.}

Покажем, что его сумма равна $C$. Для частных сумм ряда (3) имеем:
\ml{s_1 = (p_1+p_2+p_{n_1}) = A_{n_1}, \; \hm{A_{n_1} - C} \le
p_{n_1}, \\ s_2 = B_{m_1}, \; \hm{B_{m_1} - C} \le -q_{m_1}, \\ s_3
= A_{n_2}, \; \hm{A_{n_2}-C} \le p_{n_2}, \ldots.}

Так как ряд $\sum a_n$ сходится, то $\liml{n\ra+\bes} a_n=0$, и
поэтому $\liml{k\ra+\bes} p_{n_k}=0$ и $\liml{k\ra+\bes} q_{n_k}=0$.
Отсюда и из предыдущих соотношений следует, что $\liml{k\ra+\bes}
s_k=C$. Поскольку слагаемые в каждой скобке ряда (3) либо
положительны, либо отрицательны, то скобки можно открыть. Ряд
$p_1+p_2+\ldots + p_{n_1} + q_1 + q_2 + \ldots + q_{m_1} + p_{n_1+1}
+ \ldots$ получен перестановкой исходного ряда $\sum a_n$ и его
сумма равна $C$.

Случай, когда перестановочный ряд расходится, доказывается проще.
Действительно, рассмотрим сначала сумму $p_1 + p_2 + p_{n_1}>1$;
далее --- сумму $(p_1+p_2+\ldots+p_{n_1}) + (q_1+q_2+\ldots+q_{m_1})
< -1$, затем --- $(p_1+\ldots+p_{n_1}) + (q_1+\ldots+q_{m_1}) +
(p_{n_1+1} + \ldots + p_{n_2}) > 2$, $(p_1+\ldots + p_{n_1}) + (q_1
+ \ldots + q_{m_1}) + (p_{n_1+1} + \ldots + p_{n_2}) + (q_{m_1+1} +
\ldots + q_{m_2}) < -2$, \ldots.

Частные суммы $(s_k)$ ряда \eqa{4}{(p_1+\ldots+ p_{n_1}) + (q_1 +
\ldots + q_{m_1}) + (p_{n_1+1} + \ldots + p_{n_2}) + \ldots.}
удовлетворяют неравенствам $s_{2n-1}>n$ и $s_{2n}<-n, \; n\in\N$, и
следовательно, ряд (4) расходится. Числа $s_k, \; k\in\N$, образуют
подпоследовательность в последовательности частных сумм ряда
\eqa{5}{p_1  + \ldots + p_{n_1} + q_1 + \ldots + q_{m_1} + p_{n_1+1}
+ \ldots + p_{n_2} + q_{m_1+1} + \ldots q_{m_2} + \ldots,}
полученного из ряда (4) после раскрытия скобок и являющегося
перестановкой исходного ряда $\sum a_n$. Так как
подпоследовательность не имеет предела, то не имеет предела и вся
последовательность частных сумм ряда (5), то есть ряд (5)
расходится.
\end{proof}

\subsection{Последовательности и ряды с комплексными членами}

\subsubsection{Поле комплексных чисел}

Множество всех упорядоченных пар $(x,y)\in\R^2$ обозначим символом
$\Cbb$ и введём на $\Cbb$ операции сложения и умножения в форме
\eqa{1} { (x_1,y_1) + (x_2,y_2) = (x_1+x_2,y_1+y_2),}
\eqa{2}{(x_1,y_1)\cdot (x_2,y_2) = (x_1x_2-y_1y_2, x_1y_2 + y_1
x_2),} так что пары $0=(0,0)$ и $1=(1,0)$ являются нейтральными
элементами относительно введённых операций, соответственно.
Непосредственно проверяется, что обе операции обладают свойствами
(аксиомами) соответствующих операций для $\R$, и, следовательно,
$\Cbb$ образует поле.

Поскольку $(x_1,0)+(x_2,0) = (x_1+x_2,0)$ и $(x_1,0)\cdot(x_2,0) =
(x_1x_2,0)$, то отображение $\R\owns x \corr{} \Cbb = \hc{(x,y) \; |
\; x,y\in\R}$ является вложением поля $\R$ в поле $\Cbb$, и поэтому
считаем $(x,0)=x\in\R$. Пара (0,1) имеет обозначение $i=(0,1)$, и
\eqa{2'}{i^2=(0,1)\cdot(0,1)=(-1,0)=-1.}

Так как $iy=(0,1)\cdot(y,0)=(0,y)$ и $(x,y)=(x,0)+(0,y)$, то
$(x,y)=x + iy, \; x,y\in\R$. Если обозначить $x+iy=z\in\Cbb$, то
$x=\Re z, \; y=\Im z$. Если $y=0$, то комплексное число $z$ называют
вещественным (или действительным) числом; если $y\ne0$, то $z$
называют мнимым комплексным числом; если $x=0, y\ne0$, то $z=iy$
называют чисто мнимым числом. Согласно формулам (1), (2) и (2'),
\ml{z_1+z_2 = (x_1 + iy_1) + (x_2+iy_2) = (x_1+x_2)+i(y_1+y_2),\\
z_1\cdot z_2 = (x_1 + iy_1) \cdot (x_2 + iy_2) = x_1x_2 + iy_1x_2 +
ix_1y_2 + i^2y_1y_2 = (x_1x_2-y_1y_2) + i(x_1y_2+x_2y_1).}

Для $z=x+iy$ комплексное число $x-iy=\ol{z}$ называют комплексно
сопряжённым для $z$ и $z\cdot\ol{z} = x^2+y^2 \ge0$. Число
$\sqrt{z\cdot\ol{z}} = \sqrt{z^2+y^2}=\hm{z}$ --- модуль
комплексного числа $z$; если $z=x$ (то есть, $y=0$), то $\ol{z}=x$ и
$\hm{z} = \sqrt{x^2}=\hm{x}$. Из определения модуля непосредственно
вытекают следующие свойства.

\begin{nums}{-2}
\item $\hm{z}=0 \; \Lra \; z=0;$
\item $\hm{z_1z_2}=\hm{z_1}\hm{z_2};$
\item (неравенство треугольника) $\hm{z_1+z_2} \le
\hm{z_1}+\hm{z_2}$.
\end{nums}

Понятно также, что \eqa{3}{\max(\Re z, \Im z) \le \hm{z} \le \hm{\Re
z} + \hm{\Im z}.}

С помощью модуля комплексного числа на множестве $\Cbb$ вводится
метрика $\rho(z_1,z_2) = \hm{z_1-z_2}, \; z_i\in\Cbb, \; i=1,2$ ---
расстояние между $z_1$ и $z_2$. Действительно, оба свойства метрики
$\rho(z_1,z_2)$ ---
\begin{nums}{-2}
\item $\rho(z_1,z_2)=0 \; \Lra \; z_1=z_2$;
\item $\rho(z_1,z_2) \le \rho(z_1,z_3) + \rho(z_2,z_3)$ (неравенство
треугольника)
\end{nums}
--- имеют вид
\begin{nums}{-2}
\item $\hm{z_1-z_2}=0 \; \Lra \; z_1=z_2;$
\item $\hm{z_1-z_2} = \hm{z_1-z_3+z_3-z_2} \le \hm{z_1-z_3} +
\hm{z_3-z_2} = \hm{z_1-z_3} + \hm{z_2-z_3}$,
\end{nums}
и непосредственно следуют из свойств модуля комплексного числа.

Таким образом, поле $\Cbb$ с метрикой $\rho(z_1,z_2) = \hm{z_1-z_2}$
образует метрическое пространство, в котором, в частности, можно
изучать процесс сходимости последовательностей.

Отметим ещё, что в отличии от линейно упорядоченного поля $\R$ поле
$\Cbb$ упорядочить нельзя.

\subsubsection{Предел последовательности комплексных чисел}
Любое отображение $f\colon \N\ra\Cbb$ порождает в $\Cbb$
последовательность $(z_n)$, $z_n=f(n) \in \Cbb, \; n\in\N$, которую
называют комплексной последовательностью.

\begin{df}
Комплексное число $z$ называют пределом комплексной
последовательности $(z_n)$, если числовая последовательность
$(\rho(z,z_n)), \; \rho(z,z_n) = \hm{z-z_n}, \; n\in\N$ ---
бесконечно малая, то есть, $\liml{n\ra+\bes} \hm{z-z_n}=0$.
Обозначение: $z=\liml{n\ra+\bes}z_n$ (как и в метрическом
пространстве).
\end{df}

Рисунок иллюстрирует расположение точек комплексной
последовательности $(z_n)$ и её предела $z=\liml{n\ra+\bes}z_n$.

\begin{theorem}\label{thm1.34}
Пусть $z=x+iy, \; z_n = x_n+iy_n, \; n\in\N$. Тогда
$z=\liml{n\ra+\bes} z_n \; \Lra \; x=\liml{n\ra+\bes} x_n, \;
y=\liml{n\ra+\bes} y_n$.
\end{theorem}

\begin{proof}
Согласно (3), $\max(\hm{x-x_n}, \; \hm{y-y_n}) \le \hm{z-z_n} \le
\hm{x-x_n} + \hm{y-y_n}, \; n\in\N$, и используем свойства
бесконечно малых числовых последовательностей.
\end{proof}

\begin{imp*}
\begin{nums}{-2}
\item сходящаяся комплексная последовательность $(z_n)$ ограничена,
то есть, $\hm{z_n}\le C$ для всех $n\in\N$ и некоторого $C>0$;
\item сходящаяся комплексная последовательность $(z_n)$ обладает
свойством локальности; то есть, изменение или отбрасывание любого
конечного множества элементов последовательности не влияет на её
сходимость и величину предела.
\end{nums}
\end{imp*}

\begin{proof}
Следствие предыдущей теоремы и аналогичных свойств числовых
последовательностей.
\end{proof}

\begin{theorem}
(Дальнейшие свойства сходящихся комплексных последовательностей).
Предположим, что последовательности $(z_n)$ и $(w_n)$,
$z_n=x_n+iy_n, \; w_n=u_n+iv_n, \; n\in\N$, имеют
$\liml{n\ra+\bes}z_n=z, \; \liml{n\ra+\bes}w_n=w$, где $z=x+iy, \;
w=u+iv$. Тогда
\begin{nums}{-2}
\item для любых $\la_1,\la_2\in\Cbb$ последовательность $(\om_n), \;
\om_n=\la_1z_n+\la_2w_n, \; n\in\N$, имеет
$\liml{n\ra+\bes}\om_n=\la_1\liml{n\ra+\bes}z_n +
\la_2\liml{n\ra+\bes}w_n$;
\item последовательность $(z_nw_n)$ имеет
$\liml{n\ra+\bes}z_nw_n=\liml{n\ra+\bes}z_n\cdot\liml{n\ra+\bes}w_n=zw$;
\item если $w\ne0$, то существует $N\in\N$, что $w_n\ne0$ для всех
$n\ge N$;
\item если $w\ne0$, то существует
$\liml{n\ra+\bes}\frac{z_n}{w_n}=\frac{z}{w}$.
\end{nums}
\end{theorem}

\begin{proof}
Докажем, для примера, утверждение 2. Имеем $z_n\cdot w_n = (x_n +
iy_n)(u_n+iv_n) = x_nu_n-y_nv_n + i(x_nv_n+y_nu_n), \; n\in\N$. Так
как существует $\liml{n\ra+\bes}z_n = z = x+iy, \;
\liml{n\ra+\bes}w_n = w = u+iv$, то согласно теореме \ref{thm1.34},
существуют $\liml{n\ra+\bes}x_n=x, \; \liml{y_n}=y, \;
\liml{n\ra+\bes} u_n=u, \; \liml{n\ra+\bes}v_n=v$, и поэтому,
существуют $\liml{n\ra+\bes} (x_nu_n-y_nv_n)=xu-yv$,
$\liml{n\ra+\bes} (x_nv_n+y_nu_n)=xv+yu$. Согласно теореме
\ref{thm1.34}, существует $\liml{n\ra+\bes}
z_nw_n=(xu-yv)+i(xv+yu)=zw$.
\end{proof}

\subsubsection{Критерий Коши сходимости комплексной
последовательности}
\begin{theorem}
Комплексная последовательность $(z_n)$ сходится тогда и только
тогда, когда $(z_n)$ --- фундаментальная последовательность; то
есть, для любого числа $\ep>0$ существует такое индекс
$n_{\ep}\in\N$, что $\hm{z_n-z_m}<\ep$ для всех $m,n\ge n_{\ep}$.
\end{theorem}

\begin{proof}
Согласно (3), для $z_n = x_n+iy_n, \; n\in\N$, имеем
\eqa{4}{\max(\hm{x_m-x_n}, \; \hm{y_m-y_n}) \le \hm{z_m-z_n} \le
\hm{x_m-x_n} + \hm{y_m-y_n}, \; m,n\in\N.}

Поэтому, на основании (4): $(z_n)$ --- фундаментальная $\Lra$
$(x_n)$ и $(y_n)$ --- сходящиеся последовательности,
$\liml{n\ra+\bes} x_n = x, \; \liml{n\ra+\bes}y_n=y \; \Lra$
(теорема \ref{thm1.34}) существует $\liml{n\ra+\bes}z_n=x+iy=z$.
\end{proof}

\subsubsection{Ряды с комплексными членами (комплекснозначные ряды)}

Рассмотрим ряд \eqa{5}{z_1+z_2+\ldots+z_n+\ldots = \sum z_n, \;
z_n\in\Cbb, \; n\in\N,} и его частные суммы $s_n=\sumkun z_k, \;
n\in\N$. Ряд (5) называют сходящимся к сумме $s\in\Cbb$, если
$s=\liml{n\ra+\bes} s_n$; ряд (5) называют расходящимся, если
$(s_n)$ не имеет предела (не сходится). Если $z_n=x_n+iy_n, \;
n\in\N$, то $s_n = \sumkun x_k + i\sumkun y_k, \; n\in\N$, и
$\sumkun x_k, \; \sumkun y_k, \; n\in\N$ --- частные суммы числовых
рядов $\sum x_n$ и $\sum y_n$, соответственно. На основании
вышеизложенного, ряд (5) сходится тогда и только тогда, когда
сходятся числовые ряды $\sum x_n$ и $\sum y_n$ и
$s=\sum\limits_{n=1}^{\bes} z_n = \sum\limits_{n=1}^{\bes} x_n +
i\sum\limits_{n=1}^{\bes} y_n$.

\subsubsection{Абсолютная сходимость ряда с комплексными членами}

Ряд (5) называют абсолютно сходящимся, если сходится положительный
ряд $\sum \hm{z_n}$.

\begin{theorem}
Ряд $\sum z_n, \; z_n=x_n+iy_n, \; n\in\N$, абсолютно сходится тогда
и только тогда, когда абсолютно сходятся числовые ряды $\sum x_n$ и
$\sum y_n$.
\end{theorem}

\begin{proof}
Предположим сначала, что абсолютно сходится ряд $\sum z_n$, то есть
сходится числовой ряд $\sum\hm{z_n}$. Так как
$0\le\hm{x_n}\le\hm{z_n}, \; 0\le\hm{y_n}\le\hm{z_n}, \; n\in\N$, то
по признаку сравнения положительных рядов, сходятся ряды
$\sum\hm{x_n}$ и $\sum\hm{y_n}$, то есть, ряды $\sum x_n$ и $\sum
y_n$ сходятся абсолютно. Обратно, пусть абсолютно сходятся ряды
$\sum x_n$ и $\sum y_n$; то есть, сходятся ряды $\sum\hm{x_n}$ и
$\sum\hm{y_n}$. Поскольку $0\le\hm{z_n}\le\hm{x_n}+\hm{y_n}, \;
n\in\N$ и сходится ряд $\sum(\hm{x_n}+\hm{y_n})$, то сходится ряд
$\sum\hm{z_n}$, то есть, ряд $\sum z_n$ сходится абсолютно.
\end{proof}

\begin{imp*}
Всякий абсолютно сходящийся комплексный ряд сходится.
\end{imp*}

\begin{proof}
Рассмотрим абсолютно сходящийся ряд $\sum z_n, \; z_n=x_n+iy_n, \;
n\in\N$. По предыдущей теореме, абсолютно сходятся ряды $\sum x_n$ и
$\sum y_n$, и следовательно, оба ряда сходятся. Но тогда обязан
сходиться и ряд $\sum z_n = \sum x_n+i\sum y_n$.
\end{proof}

\subsubsection{Переместительный закон для абсолютно сходящихся рядов
с комплексными членами}

\begin{theorem}
(Коши--Абель). Любая перестановка членов абсолютно сходящегося ряда
с комплексными членами приводит к абсолютно сходящемуся ряду, сумма
которого равна сумме исходного ряда.
\end{theorem}

\begin{proof}
Рассмотрим произвольный абсолютно сходящийся ряд $\sum z_n, \;
z_n=x_n+iy_n, \; n\in\N$, и произвольный перестановочный ряд $\sum
z_{n_k}$, $z_{n_k}=x_{n_k} + iy_{n_k}, \; k\in\N$. По теореме
предыдущего пункта и следствия к ней, $\sum z_n = \sum x_n + i\sum
y_n$ и числовые ряды $\sum x_n$ и $\sum y_n$ абсолютно сходятся. Так
как ряды $\sum x_{n_k}$ и $\sum y_{n_k}$ получены из рядов $\sum
x_n$ и $\sum y_n$ некоторой перестановкой членов, по по теореме Коши
для абсолютно сходящихся числовых рядов, ряды $\sum x_{n_k}$ и $\sum
y_{n_k}$ абсолютно сходятся и их суммы $\sum\limits_{k=1}^{\bes}
x_{n_k} = \sum\limits_{n=1}^{\bes} x_n$, $\sum\limits_{k=1}^{\bes}
y_{n_k} = \sum\limits_{n=1}^{\bes} y_n$. Следовательно, по теореме
предыдущего пункта, абсолютно сходится ряд $\sum z_{n_k} = \sum
x_{n_k} + i\sum y_{n_k}$ и его сумма $\sum\limits_{k=1}^{\bes}
z_{n_k} = \sum\limits_{k=1}^{\bes} x_{n_k} +
i\sum\limits_{k=1}^{\bes} y_{n_k} = \sum\limits_{n=1}^{\bes} x_n +
i\sum\limits_{n=1}^{\bes} y_n = \sum\limits_{n=1}^{\bes} z_n$.
\end{proof}

\subsection{Произведение рядов}

\subsubsection{}

Рассмотрим ряды с комплексными членами \eqa{1}{\sum a_n =
a_1+a_2+\ldots+a_n+\ldots,} \eqa{2}{\sum b_n =
b_1+b_2+\ldots+b_n+\ldots,} и образуем бесконечную матрицу из
элементов $a_i,b_k, \; i,k\in\N$, вида \eqa{3}{\rbmat{a_1b_1 &
a_2b_1 & a_3b_1 & \ldots\ldots & a_ib_1 & \ldots\ldots & \\ a_1b_2 &
a_2b_2 & a_3b_2 & \ldots\ldots & a_ib_2 & \ldots\ldots & \\ a_1b_3 &
a_2b_3 & a_3b_3 & \ldots\ldots & a_ib_3 & \ldots \ldots & \\
\ldots & \ldots&\ldots&\ldots\ldots&\ldots&\ldots\ldots &\\ a_1b_k &
a_2b_k & a_3b_k & \ldots\ldots & a_ib_k & \ldots\ldots & \\ \ldots &
\ldots&\ldots&\ldots\ldots&\ldots&\ldots\ldots &}.}

Так как множество элементов матрицы (3) не более, чем счётно, то её
элементы можно перенумеровать в одну последовательность бесконечно
многими способами $a_{i_p} b_{k_p}, \; p\in\N$, и получить
бесконечно много произведений $\sum a_{i_p} b_{k_p}$ исходных рядов.
Из них выделяют два произведения.

\textbf{I.} $a_1b_1; \; a_1b_2,a_2b_1; \; a_1b_3,a_2b_2,a_3b_1; \;
\ldots$ (нумерация и суммирование идут по диагоналям в матрице (3)).
Образованный после этого ряд
\eqa{4}{a_1b_1+(a_1b_2+a_2b_1)+(a_1b_3+a_2b_2+a_3b_1)+\ldots=\sum
c_k,} в котором $c_k=\sum\limits_{j=1}^k a_j b_{k+1-j}, \; k\in\N$,
называют произведением по Коши рядов $\sum a_n$ и $\sum b_n$.

\textbf{II.} \eqa{3}{\rbmat{a_1b_1 & a_2b_1 & a_3b_1 & \ldots\ldots
& a_nb_1 & \ldots\ldots & \\ a_1b_2 & a_2b_2 & a_3b_2 & \ldots\ldots
& a_nb_2 & \ldots\ldots & \\ a_1b_3 &
a_2b_3 & a_3b_3 & \ldots\ldots & a_nb_3 & \ldots \ldots & \\
\ldots & \ldots&\ldots&\ldots\ldots&\ldots&\ldots\ldots &\\ a_1b_n &
a_2b_n & a_3b_n & \ldots\ldots & a_nb_n & \ldots\ldots & \\ \ldots &
\ldots&\ldots&\ldots\ldots&\ldots&\ldots\ldots &}.}

Нумерация и суммирование идут по квадратам в матрице (3) и
образовавшийся после этого ряд имеет вид
\eqa{5}{a_1b_1+(a_1b_2+a_2b_2+a_2b_1)+(a_1b_3+a_2b_3+a_3b_3+a_3b_2+a_3b_1)+\ldots.}

\subsubsection{Теорема Коши о произведении абсолютно сходящихся
рядов}

\begin{theorem}
Если ряды $\sum a_n$ и $\sum b_n$ сходятся абсолютно к суммам $A$ и
$B$, соответственно, то их произведение, состоящее из элементов
матрицы (3) в любом порядке, также абсолютно сходится и имеет своей
суммой число $AB$.
\end{theorem}

\begin{proof}
По условию, $A=\sumnui a_n, \; B=\sumnui b_n$ и сходятся ряды
$\sum\hm{a_n}, \; \sum\hm{b_n}$ к суммам $A^*, \; B^*$,
соответственно. Рассмотрим произвольный ряд $\sum a_{i_p} b_{k_p}$ и
ряд $\sum \hm{a_{i_p} b_{k_p}}=\sum\hm{a_{i_p}}\cdot\hm{b_{k_p}}$.
Обозначим $\si_p=\sum\limits_{l=1}^p \hm{a_{i_l}b_{k_l}}, \;
p\in\N$, и $\nu=\max(i_1,\ldots,i_p,k_1,\ldots,k_p)$.
Последовательность $(\si_p)$ возрастает, и в силу критерия
сходимости положительных рядов, \equ{\si_p =
\hm{a_{i_1}}\hm{b_{k_1}} + \hm{a_{i_2}}\hm{b_{k_2}} + \ldots +
\hm{a_{i_p}}\hm{b_{k_p}} \le (\hm{a_1}+\hm{a_2} + \ldots +
\hm{a_{\nu}}) \cdot (\hm{b_1}+\hm{b_2} + \ldots + \hm{b_{\nu}}) \le
A^* B^*, \; p\in\N.} По тому же критерию, существует
$\liml{p\ra+\bes}\si_p = \si \le A^*B^*$ и ряд $\sum a_{i_p}b_{k_p}$
сходится абсолютно. По теореме Коши--Абеля, его сумма $s$ не зависит
от перестановок его членов. Поэтому представим ряд $\sum
a_{i_p}b_{k_p}$ в виде произведения типа II по квадратам, частные
суммы которого обозначим $s'_n, \; n\in\N$, так что \equ{s'_n =
(a_1+a_2+\ldots+a_n)(b_1+b_2+\ldots+b_n) = A_nB_n, \; n\in\N,} где
$A_n$ и $B_n$ --- частные суммы рядов $\sum a_n$ и $\sum b_n$,
соответственно. Так как $\liml{n\ra+\bes}A_n=A, \;
\liml{n\ra+\bes}B_n=B$, то $\liml{n\ra+\bes}s'_n=AB=s'$ и число $s'$
равно сумме $s$ ряда $\sum a_{i_p}b_{k_p}$ по теореме Коши--Абеля.
Поэтому, $\sum\limits_{p=1}^{\bes} a_{i_p}b_{k_p}=AB.$
\end{proof}

\subsubsection{Произведение рядов по Коши}

Ряды вида (4) $\sum c_n, \; c_n=\sum\limits_{j=1}^n a_j b_{n+1-j},
\; n\in\N$, широко и плодотворно используются в случае степенных
рядов (для которых Коши и ввёл понятие произведения рядов). Ряды
вида $\sumnzi a_nz^n=a_0+a_1z+ \ldots + a_nz^n + \ldots, \; \sumnzi
b_nz^n = b_0+b_1z+\ldots+b_nz^n+\ldots, \; z\in\Cbb$, называются
степенными рядами. Их произведение по Коши имеет вид $\sumnzi
c_nz^n, \; c_n=\sum\limits_{j=0}^n a_j b_{n-j}, \; n\in\N_0 =
\N\cup\hc{0}$. Подробно степенные ряды будут изучены в следующей
главе.

\begin{theorem}
(Мертенс). Произведение по Коши (4) рядов (1) и (2) сходится, если
сходятся оба ряда и один из них сходится абсолютно. При этом, сумма
произведения равна произведению сумм сомножителей.
\end{theorem}

\begin{proof}
Предположим, что ряд $\sum a_n$ сходится абсолютно к сумме $A$, а
ряд $\sum b_n$ сходится к сумме $B$. Тогда ряд $\sum\hm{a_n}$
сходится к некоторой сумме $A^*>0$. Рассмотрим \ml{C_n = \sumkun c_k
= c_1+c_2+\ldots+c_n = a_1b_1 + (a_1b_2+a_2b_1) + \ldots + (a_1b_n +
a_2b_{n-1} + \ldots + a_{n-1}b_2+a_nb_1)=\\=a_1(b_1+b_2+\ldots+b_n)
+ a_2(b_1+b_2+\ldots+b_{n-1}) + \ldots +
a_nb_1=a_1B_n+a_2B_{n-1}+\ldots+a_nB_1 = \sumkun a_k B_{n+1-k},} где
$B_k=\sum\limits_{j=1}^k b_j, \; k\in\N$, и $\liml{k\ra+\bes}
B_k=B$. Следовательно, \eqa{6}{C_n-A_nB = \sumkun a_k B_{n+1-k} -
\sumkun a_kB = \sumkun a_k(B_{n+1-k}-B) = \sumkun a_k \be_{n+1-k} =
\ga_n, \; n\in\N,} где $\be_n=B_n-B, \; n\in\N$, и
$\liml{n\ra+\bes}\be_n=0$.

Последовательность $(\ga_n)$ есть преобразование Тёплица
последовательности $(\be_n)$, в котором последовательность $(a_n)$
обладает свойствами \begin{nums}{-2}
\item $\liml{n\ra+\bes}a_n=0$ (так как ряд $\sum a_n$ сходится);
\item существует $c>0$, что $\hm{a_1}+\ldots + \hm{a_n}\le c, \;
n\in\N$ (число $c=A^*$, так как ряд $\sum\hm{a_n}$ сходится).
\end{nums}
Так как $\liml{n\ra+\bes}\be_n=0$, то по теореме Тёплица,
последовательность $(\ga_n)$ бесконечно малая,
$\liml{n\ra+\bes}\ga_n=0$. Согласно (6),
\equ{0=\liml{n\ra+\bes}\ga_n = \liml{n\ra+\be} (C_n-A_nB) =
\liml{n\ra+\bes}C_n - \liml{n\ra+\bes} A_nB =
\liml{n\ra+\bes}C_n-AB,} откуда $AB=\liml{n\ra+\bes}C_n$.
\end{proof}

\subsection{Бесконечные произведения}
\subsubsection{Основные определения и обозначения}

Рассмотрим произвольную последовательность $(b_n)$, $b_n\in\Cbb, \;
n\in\N$, и образуем новую последовательность $(p_n)$, $p_1=b_1,\;
p_2=b_1\cdot b_2,\;\ldots,\;p_n=b_1\cdot b_2\cdot\ldots\cdot b_n, \;
p_n=\prods{k=1}^n b_k, \; n\in\N$, которую называют
последовательностью частичных произведений бесконечного произведения
$\prods{n=1}^{\bes} b_n$, числа $b_n$ --- члены бесконечного
произведения $\prods{n=1}^{\bes}b_n$.

\begin{df}
Бесконечное произведение $\prods{n=1}^{\bes}b_n$ называют сходящимся
к числу $p$, если $p=\liml{n\ra+\bes}p_n$ и $p\ne0$. Обозначение:
$p=\prods{n=1}^{\bes}b_n$, $p$ --- значение бесконечного
произведения $\prods{n=1}^{\bes}b_n$. Если $(p_n)$ не имеет предела,
то бесконечное произведение $\prods{n=1}^{\bes}b_n$ называют
расходящимся; если $\liml{n\ra+\bes}p_n=0$, то
$\prods{n=1}^{\bes}b_n$ называют расходящимся к нулю.
\end{df}
Итак, у сходящегося бесконечного произведения
$\prods{n=1}^{\bes}b_n$ все члены $b_n$ отличны от нуля.

\begin{ex}
(Эйлер). $\prods{n=1}^{\bes} \frac{e^{\frac1n}}{1+\frac1n}=e^c$, где
$c$ --- константа Эйлера.
\end{ex}

\begin{proof}
Так как $b_n=\frac{n}{n+1}e^{\frac1n}, \; n\in\N$, то, с учётом
формулы суммирования (теорема \ref{thm1.10}), $p_n=b_1\cdot
b_2\cdot\ldots\cdot b_n = \frac{1\cdot2\cdot3\cdot\ldots\cdot(n-1)
n}{2\cdot3\cdot\ldots\cdot n(n+1)} e^{1+\frac12+\ldots+\frac1n} =
\frac1{n+1} e^{\ln(n+1)+c-\al_n}=e^c e^{-\al_n}$, где
$\liml{n\ra+\bes}\al_n=0$. Поэтому, существует
$\liml{n\ra+\bes}p_n=\liml{n\ra+\bes}e^c e^{-\al_n}=e^c$.
\end{proof}

\subsubsection{Необходимый признак сходящегося бесконечного
произведения}

\begin{theorem}
Последовательность членов любого сходящегося бесконечного
произведения сходится к 1.
\end{theorem}

\begin{proof}
Рассмотрим произвольное сходящееся бесконечное произведения
$\prods{n=1}^{\bes}b_n, \; b_n\in\Cbb, \; n\in\N$, так что
существует $\liml{n\ra+\bes}p_n=p\ne0$, где $p_n=\prods{k=1}^n b_k$,
и следовательно, $b_n = \frac{p_n}{p_{n-1}}, \; n\ge2$. Так как
$\liml{n\ra+\bes} p_{n-1} = \liml{n\ra+\bes}p_n=p\ne0$, то
$\liml{n\ra+\bes} b_n = \liml{n\ra+\bes} \frac{p_n}{p_{n-1}}=1$.
\end{proof}

\subsubsection{Бесконечные произведения с положительными членами}
\begin{theorem}\label{thm1.42}
Если все числа $a_n>0, \; n\in\N$, то $\prods{n=1}^{\bes}a_n$
сходится тогда и только тогда, когда сходится ряд $\sum\ln a_n$. В
случае сходимости, если $p$ и $s$ суть значение бесконечного
произведение и сумма ряда, то $p=e^s$.
\end{theorem}

\begin{proof}
Рассмотрим $s_n=\sumkun \ln a_k = \ln \hr{\prods{k=1}^n a_k}=\ln
p_n, \; n\in\N$, откуда $p_n=e^{s_n}, \; n\in\N$. В силу
непрерывности функций $\exp$ и $\ln$, пределы
$\liml{n\ra+\bes}p_n=p\ne0$, $\liml{n\ra+\bes}s_n=s$ существуют или
не существуют одновременно, и в случае существования справедливо
$p=e^s$.
\end{proof}

\begin{theorem}\label{thm1.43}
Если $a_n=a+u_n, \; -1<u_n\ne0, \; n\in\N$, и все числа $u_n$
знакопостоянны (то есть, все $u_n>0$ или все $u_n<0$), то
$\prods{n=1}^{\bes} a_n = \prods{n=1}^{\bes} (1+u_n)$ сходится тогда
и только тогда, когда сходится один из рядов $\sum u_n, \;
\sum\hm{u_n}, \; \sum\ln(1+u_n), \; \sum\hm{\ln(1+u_n)}$ (и
сходимость каждого из рядов влечёт сходимость трёх остальных).
\end{theorem}

\begin{proof}
Поскольку, согласно необходимым признакам, бесконечное произведение
и ряды сходятся при условии $\liml{n\ra+\bes}u_n=0$, считаем, что
$\liml{n\ra+\bes}u_n=0$.

Так как все $u_n$ знакопостоянные, то $\sgn \ln(1+u_n)=\sgn u_n, \;
n\in\N$, а так как $u_n\ne0, \; n\in\N$, то существует
\eqa{1}{\liml{n\ra+\bes} \frac{\ln(1+u_n)}{u_n}= 1 =
\liml{n\ra+\bes} \frac{\hm{\ln(1+u_n)}}{\hm{u_n}}.}

На основании (1) и предельной формы общего признака сравнения
положительных рядов заключаем, что ряды $\sum u_n$, $\sum\hm{u_n}$,
$\sum\ln(1+u_n)$, $\sum\hm{\ln(1+u_n)}$ сходятся или расходятся
одновременно. Сходимость ряда $\sum\ln(1+u_n)$, по теореме
\ref{thm1.42}, равносильна сходимости бесконечного произведения
$\prods{n=1}^{\bes} a_n=\prods{n=1}^{\bes} (1+u_n)$.
\end{proof}

\begin{theorem}
Если $a_n = 1+u_n, \; -1<u_n\ne0, \; n\in\N$, и последовательность
$(u_n)$ не знакопостоянная, то бесконечное произведение
$\prods{n=1}^{\bes}(1+u_n)$ сходится, если сходятся ряды $\sum u_n$
и $\sum u_n^2$.
\end{theorem}

\begin{proof}
Так как сходится ряд $\sum u_n$, то $\liml{n\ra+\bes}u_n=0$.
Согласно локальной формуле Тейлора, $\ln(1+u_n)= u_n - \frac{u_n^2}2
+ o(u_n^2), \; n\ra+\bes$, откуда $\liml{n\ra+\bes}\frac{u_n -
\ln(1+u_n)}{u_n^2}=\frac12>0$. В силу предельной формы общего
признака сравнения положительных рядов и условия сходимости ряда
$\sum u_n^2$, заключаем, что сходится $\sum (u_n - \ln(1+u_n))$,
откуда по линейному свойству следует, что сходится ряд $\sum
\ln(1+u_n)$. Последнее, по теореме \ref{thm1.42}, равносильно
сходимости бесконечного произведения $\prods{n=1}^{\bes}(1+u_n)$.
\end{proof}

\subsubsection{Абсолютная сходимость бесконечного произведения}
\begin{df}
Бесконечное произведение $\prods{n=1}^{\bes}(1+u_n), \; -1<u_n\ne0,
\; n\in\N$, называют абсолютно сходящимся, если сходится
$\prods{n=1}^{\bes}(1+\hm{u_n})$, где $\hm{u_n}$ --- модуль числа
$u_n$.
\end{df}

Согласно теореме \ref{thm1.43}, $\prods{n=1}^{\bes}(1+\hm{u_n})$
сходится тогда и только тогда, когда сходится один из рядов
$\sum\hm{u_n}, \; \sum\ln(1+\hm{u_n})$. Если
$\liml{n\ra+\bes}\hm{u_n} = \liml{n\ra+\bes}u_n=0$ и $u_n\ne0,\;
n\in\N$, то \eqa{2}{\liml{n\ra+\bes}\frac{\ln(1+\hm{u_n})}{\hm{u_n}}
= \liml{n\ra+\bes} \hm{\frac{\ln(1+u_n)}{u_n}} = \liml{n\ra+\bes}
\frac{\hm{\ln(1+u_n)}}{\hm{u_n}}=1,} и на основании (2) заключаем,
что ряды $\sum\hm{u_n}, \; \sum\ln(1+\hm{u_n}), \;
\sum\hm{\ln(1+u_n)}$ сходятся или расходятся одновременно. Таким
образом, $\prods{n=1}^{\bes}(1+u_n)$ сходится абсолютно $\Lra$
$\prods{n=1}^{\bes}(1+\hm{u_n})$ сходится $\Lra$
$\sum\ln(1+\hm{u_n})$ сходится $\Lra$ $\sum\hm{\ln(1+u_n)}$ сходится
$\Lra$ $\sum\ln(1+u_n)$ сходится абсолютно.

\begin{theorem}
Любое абсолютно сходящееся $\prods{n=1}^{\bes}(1+u_n), \;
-1<u_n\ne0, \; n\in\N$, сходится.
\end{theorem}

\begin{proof}
Так как $\prods{n=1}^{\bes}(1+u_n), \; -1<u_n\ne0, \; n\in\N$,
сходится абсолютно, то абсолютно сходится ряд $\sum\ln(1+u_n)$ и по
теореме \ref{thm1.42}, сходится $\prods{n=1}^{\bes}(1+u_n)$; сумма
$s=\sumnui\ln(1+u_n)$ связана со значением
$p=\prods{n=1}^{\bes}(1+u_n)$ равенством $p=e^s$.
\end{proof}

\begin{theorem}
Значение абсолютно сходящегося бесконечного произведения не зависит
от порядка его сомножителей.
\end{theorem}

\begin{proof}
Если абсолютно сходится $\prods{n=1}^{\bes}(1+u_n)=p$, то абсолютно
сходится ряд $\sumnui\ln(1+u_n) = s$ и $p=e^s$. Так как сумма $s$ не
зависит от перестановки членов ряда $\sum\ln(1+u_n)$, то и число $p$
не зависит от перестановки членов бесконечного произведения
$\prods{n=1}^{\bes}(1+u_n)$.
\end{proof}

\subsubsection{Гамма--функция Эйлера}

\eqa{3}{\Ga(x)=\frac1x \prods{n=1}^{\bes}
\frac{\hr{1+\frac1n}^x}{1+\frac{x}n}, \; x\in\R, \; x\ne-n, \;
n\in\N\cup \hc{0}.}

По локальной формуле Тейлора при фиксированном $x\in\R, \; x\ne-n,
\; n\in\N\cup\hc{0}$, имеем \ml{\frac{\hr{1+\frac1n}^x}{1+\frac{x}n}
= \\ =\hr{1+\frac{x}{n}}^{-1} \hr{1+\frac{x}{n} +
\frac{x(x-1)}{2n^2} + o\hr{\frac1{n^2}}} = \hr{1-\frac{x}n +
\frac{x^2}{n^2} + o\hr{\frac1{n^2}}} \hr{1+\frac{x}{n} +
\frac{x(x-1)}{2n^2} + o\hr{\frac1{n^2}}} = \\ = 1 + \frac1{n^2}
\hr{x^2 + \frac{x(x-1)}2 - x^2} + o\hr{\frac1{n^2}} = 1 +
\frac{x(x-1)}2\frac1{n^2} + o\hr{\frac1{n^2}}, \; n\ra+\bes.}

Ряд $\sum \hs{\frac{x(x-1)}2\frac1{n^2} + o\hr{\frac1{n^2}}}$
абсолютно сходится при всех $x\in\R$. Поэтому, на основании
результатов предыдущего пункта, бесконечное произведение (3)
абсолютно сходится для всех $x\in\R, \; x\ne-n, \;
n\in\N\cup\hc{0}$. В частности, $\Ga(1)=1$.

Рассмотрим частичные произведения \equ{p_n(x) = \frac1x
\prods{k=1}^n \frac{\hr{1+\frac1k}^x}{1+\frac{x}k} = \frac1x
\prods{k=1}^n \frac{k}{k+x} \frac{(k+1)^x}{k^x} = \frac{1\cdot2\sd
n\cdot 2^x \cdot 3^x \sd n^x \cdot (n+1)^x}{x(x+1) \sd (x+n) \cdot
1^x \cdot 2^x \sd n^x} = \frac{n! (n+1)^x}{x(x+1)\sd(x+n)}.}

Так как $\liml{n\ra+\bes} \frac{(n+1)^x}{n^x} = \liml{n\ra+\bes}
\hr{1+\frac1n}^x = 1,\; x\in\R$, то \eqa{4}{\Ga(x) =
\liml{n\ra+\bes} p_n(x) = \liml{n\ra+\bes} \frac{(n+1)^x}{n^x}
\frac{n!n^x}{x(x+1)\sd(x+n)} = \liml{n\ra+\bes}
\frac{n!n^x}{x(x+1)\sd (x+n)}, \; x\ne-n, \; n\in\N \cup \hc{0}}
--- формула Эйлера--Гаусса.

\subsubsection{Функциональное уравнение для гамма-функции}
Записав формулу (4) для $\Ga(x+1)$, имеем
\equ{\frac{\Ga(x+1)}{\Ga(x)} = \liml{n\ra+\bes}
\frac{n!n^{x+1}}{(x+1)\sd(x+n)(x+n+1)} \cdot
\frac{x(x+1)\sd(x+n)}{n!n^x} = \liml{n\ra+\bes} \frac{nx}{x+n+1}=x.}

Итак, \eqa{5}{\Ga(x+1)=x\Ga(x), \; x\ne-n, \; n\in\N\cup \hc{0}.}

Полагая в (5) $x=n\in\N$, получим \equ{\Ga(n+1) = n\Ga(n) = n(n-1)
\Ga(n-1) = \ldots = n(n-1)\sd1\cdot\Ga(1)=n!} (так как $\Ga(1)=1$).
Поэтому, \eqa{5'}{\Ga(n+1)=n!, \; n\in\N.}

Заметим, что $\Ga(n+1)$ имеет смысл и при $n=0, \; \Ga(1)=1$.
Поэтому из (5') следует, что $0!=1$ (формула, принятая нами за
аксиому в первом семестре).

\subsubsection{Формула Вейерштрасса}
Так как $e^c = \prods{n=1}^{\bes} \frac{e^{\frac1n}}{1+\frac1n}$,
$c$ --- константа Эйлера, то $e^cx = \prods{n=1}^{\bes}
\frac{e^{\frac{x}n}}{\hr{1+\frac1n}^x}$. Поскольку $\Ga(x+1)=x\Ga(x)
= \prods{n=1}^{\bes} \frac{\hr{1+\frac1n}^x}{1+\frac{x}n}$, то $e^cx
\Ga(x+1) = \prods{n=1}^{\bes} \frac{e^{\frac{x}n}}{1+\frac{x}n}$ и
\eqa{6}{\frac1{\Ga(x+1)} = e^{cx} \prods{n=1}^{\bes}
\hr{1+\frac{x}n} e^{-\frac{x}{n}}, \; x\in\R, \; x\ne-in, \;
n\in\N.} (формула Вейерштрасса).

\subsubsection{}

\begin{theorem}
Если последовательность $(a_n)$, $a_n>0$, обладает свойством, что
$\frac{a_{n+1}}{a_n}=1+\frac{s}n + \frac{\ta_n}{n^{1+\si}}, \; n\in\
N$, где $s,\si\in\R, \; \si>0$, и последовательность $(\ta_n)$
ограничена, то существует $c>0$, что $a_n = cn^s \hs{1+o(1)}, \;
n\ra+\bes.$
\end{theorem}

\begin{proof}
Рассмотрим $b_n = \frac{a_n}{n^s}, \; n\in\N$. Тогда
\ml{\frac{b_{n+1}}{b_n} = \frac{a_{n_1}}{a_n} \frac{n^s}{(n+1)^s} =
\hs{1+\frac{s}n + \frac{\ta_n}{n^{1+\si}}} \hr{1+\frac1n}^{-s} =
\hs{1+\frac{s}n + \frac{\ta_n}{n^{1+\si}}} \hs{1 - \frac{s}n +
\frac{(-s)(-s-1)}2 \frac1{n^2} + o\hr{\frac1{n^2}}} = \\ = 1 +
\frac{\ta_n}{n^{1+\si}} + \frac{s(s+1)}2 \frac1{n^2} -
\frac{s^2}{n^2} + o\hr{\frac1{n^2}} = 1 + \frac{\ta_n}{n^{1+\si}} -
\frac{s(s-1)}2 \frac1{n^2} + o\hr{\frac1{n^2}}, \; n\ra+\bes.}

Ряд $\sum \hs{\frac{ta_n}{n^{1+\si}} + \frac{s}2\frac1{n^2} +
o\hr{\frac1{n^2}}}$ сходится абсолютно, так как $\si>0$ и
$\hm{\ta_n} \le M, \; n\in\N$, с некоторой $M>0$. Следовательно,
абсолютно сходится бесконечное произведение $\prods{n=1}^{\bes}
\frac{b_{n+1}}{b_n}=p$ и \ml{p = \liml{n\ra+\bes} \prods{k=1}^n
\frac{b_{k+1}}{b_k} = \liml{n\ra+\bes} \prods{k=1}^n
\frac{a_{k+1}}{a_k} \frac{k^s}{(k+1)^s} = \liml{n\ra+\bes} \frac
{a_2\cdot a_3 \sd a_n \cdot a_{n+1}}{a_1\cdot a_2 \sd a_n} \cdot
\frac{1^s \cdot 2^s \sd n^s}{2^s \sd n^s \cdot (n+1)^s} = \\
= \liml{n\ra+\bes} \frac{a_{n+1}}{a_1\cdot (n+1)^s} = \frac1{a_1}
\liml{n\ra+\bes} \frac{a_n}{n^s}.} Итак, существует
$\liml{n\ra+\bes} \frac{a_n}{n^s}=a_1p=c$ и поэтому $\frac{a_n}{n^s}
= c\hs{1+o(1)}, \; n\ra+\bes$, откуда $a_n=cn^s \hs{1+o(1)}, \;
n\ra+\bes$.
\end{proof}

\section{Функциональные последовательности и функциональные ряды}

\subsection{Сходимость функциональных последовательностей и
функциональных рядов}

\subsubsection{Основные определения}

\begin{dfn}{1}
Функция $F(x,n), \; (x,n)\in\R\times\N$, порождает функциональную
последовательность $(f_n(x))$, $f_n(x) = F(x,n), \;n\in\N$. При этом
область определения $D_{(f_n)}=\mbox{проекция}_{\R} D_F$. Обозначим
$D_{(f_n)}= E\subset\R$.
\end{dfn}

\begin{dfn}{2}
Функциональная последовательность $\hr{f_n(x)}$, $x\in E$,
называется сходящейся на множестве $E$, если для каждого $x\in E$
сходится числовая последовательность $(f_n(x))$, то есть, существует
$\liml{n\ra+\bes} f_n(x)=f(x), \; x\in E$.
\end{dfn}

Таким образом, на $E$ определена новая числовая функция $f(x)$,
называемая предельной функцией функциональной последовательности
$(f_n(x))$. Обозначение: $f(x) = \liml{n\ra+\bes} f_n(x), \; x\in
E$, или $f_n\ra f$ на $E$.

\begin{dfn}{1'}
Функция $F(z,n), \; (z,n)\in\Cbb\times\N$, порождает комплексную
функциональную последовательность $(f_n(z)), \; f_n(z) = F(z,n), \;
z\in E \subset\Cbb, \; n\in\N$.
\end{dfn}

\begin{dfn}{2'}
Если для каждого $z\in E\subset\Cbb$ существует $\liml{n\ra+\bes}
f_n(z) = f(z)$, то на $E$ определена новая комплексная функция
$f(z)$, называемая предельной функцией комплексной функциональной
последовательности $(f_n(z))$.
\end{dfn}

Итак, $f(z)=\liml{n\ra+\bes} f_n(z), \; z\in E$ тогда и только
тогда, когда для произвольной $z\in E$ и любого $\ep>0$ существует
$N\in\N$, $N=N(\ep; z)$, что $\hm{f_n(z)-f(z)}<\ep$ для всех $n\ge
N$.

Обозначение: $f_n\ra f$ на $E$.

\subsubsection{Равномерно сходящиеся функциональные
последовательности}

\begin{dfn}{3}
Комплексная функциональная последовательность $(f_n(z))$ равномерно
сходится на множестве $E\subset\Cbb$, если \begin{points}{-2} \item
она сходится на $E$ к некоторой предельной функции $f(z), \;
f(z)=\liml{n\ra+\bes} f_n(z), \; z\in E$, \item для произвольного
$\ep>0$ существует $N\in\N$ ($N=N(\ep)$, выбор числа $N$ не зависит
от точек $z\in E$), что\end{points} \eqa{1}{\hm{f_n(z)-f(z)}<\ep}
для всех $n\ge N$ и всех $z\in E$.
\end{dfn}

Обозначение: $f_n\rra f$ на $E$.

Если $E=\hc{z_1\sco z_k}\subset \Cbb$ --- конечное множество и
$f_n\ra f$ на $E$, то понятно, что $f_n\rra f$ на $E$, поскольку для
любого $\ep>0$ и каждого $z_j\in E, \; j=\ol{1,k}$, согласно
определению 2', существует $N_j\in\N, \; N_j=N(\ep,z_j), \; 1\le
j\le k$, что $\hm{f_n(z_j) - f(z_j)} < \ep$ для всех $n\ge
N(\ep,z_j)$, $1\le j \le k$, и выбирая $N=\max(N_1,N_2 \sco N_k), \;
N=N(\ep)$, заключаем, что (1) выполнено для всех $n\ge N$ и всех
$z\in E$.

\begin{ex}
$f_n(x) = \frac{x}{1+n^2x^2}, \; x\in E=\R, \; n\in\N$. Тогда
$\liml{n\ra+\bes} f_n(x)=0=f(x), \; x\in\R$, и для любого $x\in\R$
имеем \equ{\hm{f_n(x)-f(x)} = \frac{\hm{x}}{1+n^2x^2} = \frac1{2n}
\hm{\frac{2nx}{1+n^2x^2}} \le \frac1{2n}, \; x\in\R, \; n\in\N.}
Поэтому неравенство (1) справедливо для всех $n\ge N_{\ep} =
\hs{\frac1{2\ep}}+1 > \frac1{2\ep}$ (тогда $\frac1{2n} < \ep$) и
всех $x\in\R$, то есть, $f_n \rra f$ на $\R$.
\end{ex}

\subsubsection{Критерий Коши равномерной сходимости функциональной
последовательности}

\begin{theorem}
Функциональная последовательность $(f_n(z))$ равномерно сходится на
множестве $E\subset\Cbb$ тогда и только тогда, когда для
произвольного числа $\ep>0$ существует $N\in\N, \; N=N(\ep)$, что
\eqa{2}{\hm{f_{n+p}(z)-f_n(z)}<\ep} для всех $z\in E$ и всех
$n,p\in\N, \; n\ge N=N(\ep)$; \eqa{2'}{\hm{f_m(z)-f_n(z)}<\ep} для
всех $z\in E$ и всех $m,n\ge N=N(\ep)$.
\end{theorem}

\begin{proof}
\textbf{Необходимость}. Пусть $f_n\rra f$ на $E$ и $\ep>0$ ---
произвольное. Согласно определению 3, существует $N\in\N, \;
N=N(\ep)$, что неравенство $\hm{f(z)-f_n(z)} < \frac{\ep}2$
справедливо для всех $z\in E$ и всех $n\ge N=N(\ep)$. Тогда
$\hm{f_n(z) - f_{n+p}(z)} \le \hm{f(z) - f_n(z)} + \hm{f(z) -
f_{n+p}(z)} < \frac{\ep}2+ \frac{\ep}2 = \ep$ для всех $z\in E$ и
всех $n,p\in\N, \; n\ge N=N(\ep)$.

\textbf{Достаточность}. Пусть для произвольного $\ep>0$ существует
$N=N(\ep)$, что неравенство (2) справедливо для всех $z\in E$ и всех
$n,p\in\N, \; n \ge N=N(\ep)$. Неравенство (2) есть критерий Коши
сходимости последовательности $(f_n(z))$ в каждой точке $z\in E$ к
некоторой предельной функции $f(z) = \liml{n\ra+\bes} f_n(z), \;
z\in E$. В силу свойства локальности предела комплексной
последовательности, $f(z) = \liml{p\ra+\bes} f_{n+p}(z), \; z\in E$,
для любого $n\in\N$. Переходя в неравенстве (2) к пределу при
$p\ra+\bes$, получим $\hm{f(z) - f_n(z)} \le \ep$ для всех $z\in E$
и всех $n\ge N = N(\ep)$. Согласно определению 3, $f_n\rra f$ на
$E$.

Случай с неравенством (2') рассматривается аналогично.
\end{proof}

\subsubsection{Критерий равномерной сходимости функциональной
последовательности}

\begin{theorem}
Функциональная последовательность $f_n\rra f$ на $E$ тогда и только
тогда, когда положительная числовая последовательность $(\al_n), \;
\al_n = \sup\limits_{z\in E} \hm{f(z) - f_n(z)}, \; n\in\N$ ---
бесконечно малая, то есть $\liml{n\ra+\bes}\al_n=0$.
\end{theorem}

\begin{proof}
Если $f_n\rra f$ на $E$, то по определению 3, для произвольного
числа $\ep>0$ существует $N\in\N, \; N=N(\ep)$, что неравенство
$\hm{f(z) - f_n(z)}<\ep$ справедливо для всех $z\in E$ и всех $n\ge
N=N(\ep)$. Тогда $\al_n = \supl{z\in E}\hm{f(z) - f_n(z)}\le\ep$ для
всех $n\ge N=N(\ep)$; то есть, $\liml{n\ra+\bes}\al_n = 0$. Если
теперь $\liml{n\ra+\bes}=0$, то для произвольного $\ep>0$ существует
$N\in\N, \; N=N(\ep)$, что $0\le\al_n < \ep$ для всех $n\ge
N=N(\ep)$. Тогда $\hm{f(z)-f_n(z)}<\ep$ для всех $z\in E$ и всех
$n\ge N=N(\ep)$, и по определению 3, $f_n\rra f$ на $E$.
\end{proof}

\subsubsection{Неравномерно сходящиеся последовательности}

Сходящуюся на множестве $E$ функциональную последовательность, не
сходящуюся при этом на $E$ равномерно, называют неравномерно
сходящейся функциональной последовательностью на $E$. Это
равносильно нижеследующему определению.

\begin{dfn}{4}
Функциональная последовательность $(f_n(z))$ неравномерно сходится
на множестве $E\subset\Cbb$, если \begin{points}{-2} \item она
сходится на $E$ к некоторой предельной функции $f(z)$; \item
существует число $\ep_0>0$, обладающее свойством: для произвольного
$N\in\N$ найдётся $n\in\N, \; n\ge N$, и найдётся точка $z_n\in E$,
чтобы справедливо неравенство $\hm{f(z_n)-f_n(z_n)}\ge\ep_0$.
\end{points}
\end{dfn}

\begin{ex}
$f_n(x)=\frac{nx}{1+n^2x^2}, \; x\in[0,1], \; n\in\N$. Тогда
$f_n(0)=0, \; \liml{n\ra+\bes}f_n(0)=0$, и $f(x) = \liml{n\ra+\bes}
\frac{nx}{1+n^2x^2} = \liml{n\ra+\bes}\frac1n \frac{x}{\frac1{n^2} +
x^2}=0$ для всех $x\in(0,1]$. Таким образом, $f_n\ra f$ на $[0,1]$ и
$f(x)=0, \; x\in[0,1]$. В точках $x_n=\frac1n\in(0,1], \; n\in\N$,
имеем $\hm{f(x_n)-f_n(x_n)} = f_n(x_n)=\frac12=\ep_0>0, \; n\in\N$.
Таким образом, последовательность $(f_n(x))$ не сходится равномерно
на $[0,1]$, то есть, сходится неравномерно.
\end{ex}

\subsubsection{Сходимость функциональных рядов}

Рассмотрим произвольную функциональную последовательность $(a_n(z))$
на множестве $E$ из $\Cbb$ и образуем новую функциональную
последовательность $(s_n(z)), \; s_1(z)=a_1(z), \; s_2(z) =
a_1(z)+a_2(z) \sco s_n(z)=\sumkun a_k(z)$, $n\in\N$. Функции
$s_n(z), \; n\in\N$, называют частными суммами функционального ряда
\eqa{3} {a_1(z) + a_2(z) +\sco+a_n(z)+\ldots = \sum a_n(z).}

Если в каждой точке $z\in E\subset \Cbb$ комплексная
последовательность $(s_n(z))$ сходится; то есть, $(s_n(z))$
поточечно сходится на $E$ к некоторой предельной функции $s(z) =
\liml{n\ra+\bes} s_n(z), \; z\in E$, то $s(z)$ называют суммой ряда
(3), а множестве $E$ --- областью сходимости ряда (3).

Обозначение: $s(z) = \sumnui a_n(z), \; z\in E$.

\begin{theorem}
(Критерий Коши). Ряд (3) сходится на множестве $E\subset\Cbb$ тогда
и только тогда, когда для произвольного $z\in E$ и произвольного
$\ep>0$ существует $N\in\N, \; N=N(z;\ep)$, что неравенство
$\hm{a_{n+1}(z) \spl a_{n+p}(z)} < \ep$ справедливо для всех
$n,p\in\N, \; n\ge N=N(z;\ep)$ (неравенство $\hm{a_n(z) \spl a_m(z)}
< \ep$ справедливо для всех $n,m\ge N=N(z;\ep))$.
\end{theorem}

\begin{proof}
\equ{\hm{a_{n+1}(z) \spl a_{n+p}(z)} = \hm{s_{n+p}(z) - s_n(z)} <
\ep} для всех $n,p\in\N, \; n\ge N=N(z;\ep)$ --- критерий Коши
сходимости функциональной последовательности $(s_n(z))$ на $E$, то
есть $s_n(z) \ra s(z)$ на $E$ и $s(z)$ --- сумма ряда (3).
\end{proof}

\begin{dfn}{5}
Функциональный ряд \eqa{4}{a_{n+1}(z) + \ldots = \sumkui a_{n+k}(z),
\; z\in E,} называется $n$--ым остаточным рядом ряда (3). Для его
суммы $r_n(z) = \sum\limits_{m=1}^{\bes} a_{n+m}(z)$ справедлива
формула \eqa{5}{r_n(z) = s(z) - s_n(z), \; z\in E, \; n\in\N,} тогда
и только тогда, когда ряд (3) сходится.
\end{dfn}

\begin{stm*}
Ряд (3) сходится на $E$ тогда и только тогда, когда функциональная
последовательность $(r_n(z))$ поточечно сходится на множестве $E$ к
нулевой функции; то есть, $r_n(z) \ra0$ на $E$.
\end{stm*}

\begin{dfn}6
Ряд (3) называют абсолютно сходящимся на некотором множестве
$E_1\subset\Cbb$, если на $E_1$ сходится ряд $\sum \hm{a_n(z)}$.
\end{dfn}

Так как всякий абсолютно сходящийся ряд сходится, то $E_1\subset E$.

\subsubsection{Равномерно сходящиеся функциональные ряды}
\begin{dfn}
7 Ряд (3) называют равномерно сходящимся на множестве
$E\subset\Cbb$, если на $E$ равномерно сходится последовательность
$(s_n(z))$ его частных сумм; то есть, $s_n(z)\rra s(z)$ на $E$;
$s(z)$ --- сумма ряда (3).
\end{dfn}

\begin{stm*}
Ряд (3) равномерно сходится на множестве $E$ тогда и только тогда,
когда последовательность $(r_n(z))$ сумм его остатков равномерно
сходится на $E$ к нулевой функции.
\end{stm*}

\begin{proof}
Согласно определению 7 и формуле (5), $s_n(z)\rra s(z)$ на $E$ тогда
и только тогда, когда $r_n(z)\rra0$ на $E$.
\end{proof}

\begin{ex}
Ряд $\sum \frac{(-1)^{n+1}}{x^2+n}$ равномерно сходится на $E=\R$ и
ряд сходится условно на $\R$.
\end{ex}

\begin{proof}
\equ{\hm{r_n(x)}=\hm{\sum\limits_{k=n+1}^{\bes}
\frac{(-1)^{k+1}}{x^2+k}} \le \frac1{x^2+n+1} \le \frac1{n+1}, \;
x\in\R, \; n\in\N} (по признаку Бернулли--Лейбница), так что
$r_n(x)\rra0$ на $\R$; то есть, ряд равномерно сходится на $\R$.
Положительный ряд $\sum\frac1{x^2+n}$ расходится при любом $x\in\R$,
так как для любого $x\in\R$ существует $m\in\N$, что $m>x^2$, и
следовательно, $\frac1{x^2+n} > \frac1{m+n}, \; n\in\N$, и
$\sum\frac1{m+n}$ есть $m$--ый остаточный ряд расходящегося
гармонического ряда.
\end{proof}

\begin{theorem}
Ряд (3) равномерно сходится на $E\subset\Cbb$ тогда и только тогда,
когда для произвольного числа $\ep>0$ существует $N\in\N, \;
N=N(\ep)$ (выбор $N$ зависит только от $\ep>0$ и не зависит от точек
$z\in E$), что неравенство \eqa{6}{\hm{a_{n+1}(z) \spl a_{n+p}(z)} <
\ep} \eqa{6'}{\hm{a_{n(z)} \spl a_m(z)}<\ep} справедливо
одновременно для всех $z\in E$ и всех $n,p\in\N$, $n\ge N=N(\ep)$
(всех $m,n\in\N, \; m,n\ge N=N(\ep)$).
\end{theorem}

\begin{proof}
$\hm{a_{n+1}(z)\spl a_{n+p}(z)}=\hm{s_{n+p}(z)-s_n(z)}$ и
неравенство (6) [(6')] есть критерий Коши равномерной сходимости
последовательности $(s_n(z))$ на множестве $E$.
\end{proof}

\subsubsection{Линейное свойство равномерно сходящихся рядов}

\begin{theorem}
Если ряды $\sum a_n(z)$ и $\sum b_n(z)$ равномерно сходятся на
множестве $E\subset\Cbb$, то их сумма $\sum(a_n(z)+b_n(z))$
равномерно сходится на $E$.
\end{theorem}

\begin{proof}
Рассмотрим произвольное $\ep>0$. Согласно (6), существует $N_1\in\N,
\; N_1=N_1(\ep)$, что $\hm{a_{n+1}(z) \spl a_{n+p}(z)} <
\frac{\ep}2$ для всех $z\in E$ и всех $n,p\in\N$, $n\ge
N_1=N_1(\ep)$, и существует $N_2\in\N, \; N_2=N_2(\ep)$, что
$\hm{b_{n+1}(z) \spl b_{n+p}(z)} < \frac{\ep}2$ для всех $z\in E$ и
всех $n,p\in\N, \; n\ge N_2=N_2(\ep)$. Обозначим $N=\max(N_1,N_2),
\; N=N(\ep)$. Тогда \equ{\hm{a_{n+1}(z) + b_{n+1}(z) \spl a_{n+p}(z)
+ b_{n+p}(z)} \le \hm{a_{n+1}(z)\spl a_{n+p}(z)} + \hm{b_{n+1}(z)
\spl b_{n+p}(z)} < \frac{\ep}2 + \frac{\ep}2 = \ep} для всех $z\in
E$ и всех $n,p\in\N, \; n\ge N=N(\ep)$. Согласно (6), ряд $\sum
(a_n(z)+b_n(z))$ равномерно сходится на $E$.
\end{proof}

\begin{theorem}
Если ряд $\sum a_n(z)$ равномерно сходится на множестве
$E\subset\Cbb$, а функция $v(z)$ ограничена на $E$, то $\sum
v(z)a_n(z)$ равномерно сходится на $E$.
\end{theorem}

\begin{proof}
По условию, существует $M>0$, что $\hm{v(z)}\le M$ для всех $z\in
E$. Пусть $\ep>0$ --- произвольное. Согласно (6), существует
$N\in\N, \; N=N(\ep)$, что $\hm{a_{n+1}(z) \spl a_{n+p}(z)} <
\frac{\ep}M$ для всех $z\in E$ и всех $n,p\in\N, \; n\ge N=N(\ep)$.
Тогда \equ{\hm{v(z)a_{n+1}(z) \spl v(z)a_{n+p}(z)} = \hm{v(z)}
\hm{a_{n+1}(z) \spl a_{n+p}(z)} \le M\hm{a_{n+1}(z) \spl a_{n+p}(z)}
< \frac{\ep}M M = \ep} для всех $z\in E$ и всех $n,p\in\N, \; n\ge
N=N(\ep)$, так что ряд $\sum v(z) a_n(z)$ равномерно сходится на
$E$.
\end{proof}

\subsubsection{Свойство аддитивности равномерно сходящихся рядов}
\begin{theorem}
Если ряд $\sum a_n(z)$ равномерно сходится на множествах $E_1$ и
$E_2$, то он равномерно сходится на множестве $E=E_1\cup E_2$.
\end{theorem}

\begin{proof}
Критерий Коши применяем отдельно на $E_1$ и $E_2$, и следовательно,
он справедлив и на $E=E_1\cup E_2$.
\end{proof}

\subsubsection{Важное замечание}
Всякую функциональную последовательность $(f_n(z))$ можно
рассматривать как последовательность частных сумм функционального
ряда $\sum a_n(z)$, если $a_1(z)=f_1(z), \; a_2(z) = f_2(z)-f_1(z)
\sco a_n(z)=f_n(z)-f_{n-1}(z), \; n\ge2.$

\subsection{Признаки равномерной сходимости функциональных рядов}

\subsubsection{Признак Вейерштрасса}
\begin{theorem}
Если функции $a_n(z), \; n\in\N$, определены на множестве
$E\subset\Cbb$ и $c_n = \supl{z\in E} \hm{a_n(z)}, \; n\in\N$, то из
сходимости положительного ряда $\sum c_n$ следует равномерная
сходимость на $E$ рядов $\sum a_n(z)$ и $\sum \hm{a_n(z)}$.
\end{theorem}

\begin{proof}
Проверим выполнение критерия Коши равномерной сходимости рядов.
Рассмотрим произвольное $\ep>0$. Так как сходится $\sum c_n$, то
существует $N\in\N, \; N=N(\ep)$, что $c_{n+1} \spl c_{n+p} =
\hm{c_{n+1} \spl c_{n+p}} < \ep$ для всех $n,p\in\N, \; n\ge N$.
Поэтому, \equ{\hm{a_{n+1}(z) \spl a_{n+p}(z)} \le \hm{a_{n+1}(z)}
\spl \hm{a_{n+p}(z)} \le c_{n+1} \spl c_{n+p} < \ep} для всех $z\in
E$ и всех $n,p\in\N, \; n\ge N=N(\ep)$; то есть, ряды $\sum a_n(z)$
и $\sum \hm{a_n(z)}$ равномерно сходятся на множестве $E$.
\end{proof}

\subsubsection{Признак Дирихле}

\begin{theorem}
Если функции $a_n(x), \; b_n(x), \; n\in\N$, определены на множестве
$E\subset\R$ и обладают свойствами \begin{points}{-2} \item частные
суммы $B_n(x)=\sumkun b_k(x), \; n\in\N$, равномерно ограничены на
$E$ (то есть, существует $M>0$, что $\hm{B_n(x)}\le M$ для всех
$x\in E$ и всех $n\in\N$); \item для каждого $x\in E$ числовая
последовательность $(a_n(x))$ убывает, а функциональная
последовательность $(a_n(x))$ равномерно сходится на $E$ к нулевой
функции,\end{points} то ряд $\sum a_n(x)b_n(x)$ равномерно сходится
на $E$.
\end{theorem}

\begin{proof}
Рассмотрим произвольные $m,n\in\N, \; m>n$. Совершая преобразование
Абеля, имеем \equ{\sum\limits_{k=n}^m a_k(x) b_k(x) =
\sum\limits_{k=n}^{m-1} B_k(x) (a_k(x)-a_{k+1}(x)) + B_m(x)a_m(x) -
B_{n-1}(x)a_n(x),} где $B_0(x)=0$, и откуда, с учётом условия 1 и
свойства убывания числовых последовательностей $(a_n(x))$ для
каждого $x\in E$, получим оценку \mla{1}{\hm{\sum\limits_{k=n}^m
a_k(x)b_k(x)} \le \sum\limits_{k=n}^{m-1} \hm{B_k(x)} \hm{a_k(x) -
a_{k+1}(x)} + \hm{B_m(x)} \hm{a_m(x)} + \hm{B_{n-1}(x)} \hm{a_n(x)}
\le \\ \le M \hs{ \sum\limits_{k=n}^{m-1} (a_k(x) - a_{k+1}(x)) +
\hm{a_m(x)} + \hm{a_n(x)}} = \\ = M \hs{a_n(x) - a_{n+1}(x) +
a_{n+1}(x) - a_{n+2} \spl a_{m-1} - a_m(x) + \hm{a_m(x)} + a_n(x)}
=\\= M \hs{a_n(x) + \hm{a_n(x)} - a_m(x) + \hm{a_m(x)}} \le 2M
\hs{\hm{a_n(x)} + \hm{a_m(x)}},} справедливо для всех $x\in E$ и
всех $m,n\in\N, \; m>n$.

Так как $a_n(x)\rra0$ на $E$, то для произвольного числа $\ep>0$
существует $N\in\N, \; N=N(\ep)$, что $\hm{a_n(x)} < \frac{\ep}{4M}$
для всех $x\in E$ и всех $n\in\N$, $n\ge N=N(\ep)$. Тогда
$\hm{a_m(x)} < \frac{\ep}{4M}$ для всех $x\in E$ и всех $m\in\N, \;
m>n\ge N=N(\ep)$. Поэтому, на основании (1) заключаем, что
\equ{\hm{\sum\limits_{k=n}^m a_k(x)b_k(x)} < 2M \hr{\frac{\ep}{4M} +
\frac{\ep}{4M}} = \ep} для всех $x\in E$ и всех $m>n \ge N=N(\ep)$;
то есть, справедлив критерий Коши равномерной сходимости ряда $\sum
a_n(x) b_n(x)$ на множестве $E$.
\end{proof}

\subsubsection{Признак Абеля равномерной сходимости функционального
ряда}

\begin{theorem}
Если функциональный ряд $\sum b_n(x)$ равномерно сходится на
множестве $E\subset\R$, а функции $a_n(x), \; x\in E, \; n\in\N$,
обладают свойствами \begin{points}{-2} \item для каждого $x\in E$
числовая последовательность $(a_n(x))$ монотонна и \item
функциональная последовательность $(a_n(x))$ равномерно ограничена
на множестве $E$ (то есть, существует $M>0$, что $\hm{a_n(x)}\le M$
для всех $x\in E$ и всех $n\in\N$), \end{points} то ряд $\sum a_n(x)
b_n(x)$ равномерно сходится на $E$.
\end{theorem}

\begin{proof}
Рассмотрим произвольное число $\ep>0$. В силу равномерной сходимости
на $E$ ряда $\sum b_n(x)$, существует $N\in\N, \; N=N(\ep)$, что
\eqa{2}{\hm{b_{n+1}(x) \spl b_{n+p}(x)}<\ep} для всех $x\in E$ и
всех $n,p\in\N, \; n\ge N$. Рассмотрим $B_0(x)=0, \;
B_1(x)=b_{n+1}(x), \; B_2(x) = b_{n+1}(x) + b_{n+2}(x) \sco B_j(x) =
b_{n+1}(x) \spl b_{n+j}(x), \; j=\ol{1,p}, \; p\in\N$. Тогда
$b_{n+j}(x) = B_j(x) - B_{j-1}(x), \; n,j\in\N$, и в силу
преобразования Абеля \mla{3}{\sum\limits_{k=n+1}^{n+p} a_k(x)b_k(x)
= \sum\limits_{j=1}^p a_{n+j}(x) b_{n+j}(x) = \sum\limits_{j=1}^p
a_{n+j}(x) \hs{B_j(x)-B_{j-1}(x)} = \\ =\sum\limits_{j=1}^{p-1}
B_j(x) \hs{a_{n+j}(x) - a_{n+j+1}(x)} + a_{n+p}(x)B_p(x) -
a_{n+1}(x)B_0(x), \; x\in E.}

Согласно (2), $\hm{B_j(x)}<\ep$ для всех $x\in E$ и всех
$j=\ol{1,p}, \; p\in\N$, и $B_0(x)=0, \; x\in E$. Не ограничивая
общности, считаем, что $(a_n(x))$ убывает в каждой $x\in E$; тогда,
на основании (3), с учётом условия 2 теоремы, получим \mla{4}{
\hm{\sum\limits_{k=n+1}^{n+p} a_k(x) b_k(x)} \le
\sum\limits_{j=1}^{p-1} \hm{B_j(x)} \hm{a_{n+j}(x) - a_{n+j+1}(x)} +
\hm{a_{n+p}(x)} \hm{B_p(x)} < \ep \hs{\sum\limits_{j=1}^{p-1}
(a_{n+j}(x) - a_{n+j+1}(x)) + \hm{a_{n+p}(x)}} < \\ < \ep
\hs{a_{n+1}(x) - a_{n+2}(x) + a_{n+2}(x) - a_{n+3}(x) \spl
a_{n+p-1}(x) - a_{n+p}(x) + \hm{a_{n+p}(x)}} =\\= \ep \hs{a_{n+1}(x)
- a_{n+p}(x) + \hm{a_{n+p}(x)}}  < \ep \hr{\hm{a_{n+1}(x)} +
2\hm{a_{n+p}(x)}} \le \ep \cdot 3M} для всех $x\in E$ и всех $n,p\in
\N, \; n\ge N=N(\ep)$. Утверждение (4) есть критерий Коши
равномерной сходимости ряда $\sum a_n(x)b_n(x)$ на множестве $E$.
\end{proof}

\subsubsection{Признак Дини равномерной сходимости функциональных
рядов}

\begin{theorem}
Если непрерывные и неотрицательные на отрезке $[a,b]$ функции
$a_n(x), \; n\in\N$, образуют ряд $\sum a_n(x)$, сходящийся на
$[a,b]$ к непрерывной функции $s(x)$, то ряд $\sum a_n(x)$
равномерно сходится на $[a,b]$.
\end{theorem}

\begin{proof}
Рассмотри частные суммы $s_n(x) = \sumkun a_k(x), \; n\in\N$, и
суммы $r_n(x), \; n\in\N$, остатков ряда $\sum a_n(x)$. Так как
$a_k(x)\ge0, \; x\in[a,b], \; k\in\N$, то $0\le s_n(x) \le
s_{n+1}(x) \le s(x), \; x\in[a,b], \; n\in\N$, и $r_n(x)=s(x) -
s_n(x), \; n\in\N$, удовлетворяют неравенствам $0\le r_{n+1}(x) \le
r_n(x), \; x\in[a,b], \; n\in\N$, и все $r_n(x)$ непрерывны на
$[a,b]$, поскольку непрерывны на $[a,b]$ все $s_n(x), \; n\in\N$, и
сумма $s(x)$. Кроме того, $\liml{n\ra+\bes}r_n(x)=0$ для всех $x\in
[a,b]$. Утверждение теоремы будет доказано, если установим, что
$r_n(x)\rra0$ на $[a,b]$.

Предположим, напротив, что $(r_n(x))$ сходится на $[a,b]$ к нулевой
функции неравномерно. Это значит, что существует число $\ep_0>0$ и
для произвольного $\nu\in\N$ существует число $n_{\nu}\in\N, \;
n_{\nu} \ge \nu$, и точка $x_{n_{\nu}}\in[a,b]$, для которых
\equ{\hm{r_{n_{\nu}}(x_{n_{\nu}})} = r_{n_{\nu}} (x_{n_{\nu}}) \ge
\ep_0, \; \nu\in\N.} Не ограничивая общности, считаем $n_{\nu} = \nu
= n, \; \nu\in\N$, так что на $[a,b]$ существует последовательность
точек $(x_n)$, в которых $r_n(x_n) \ge \ep_0, \; n\in\N$. По теореме
Больцано, ограниченная последовательность $(x_n)$ содержит
сходящуюся подпоследовательность $(x_{n_k})$, $\liml{k\ra+\bes}
x_{n_k} = x_0$, и так как $a\le x_{n_k} \le b, \; k\in\N$, то $a\le
x_0 \le b$; то есть, $x_0\in[a,b]$. При этом, $r_{n_k} (x_{n_k})
\ge\ep_0, \; k\in\N$.

Для произвольного фиксированного $m\in\N$ существует $n_k\ge m$, и
поэтому, $r_m(x)\ge r_{n_k}(x), \; x\in[a,b]$; в частности,
\eqa{5}{r_m(x_{n_k}) \ge r_{n_k} (x_{n_k}) \ge \ep_0, \; k\in\N, \;
n_k\ge m.}

Так как функция $r_m(x)$ непрерывна на $[a,b]$ и $\liml{k\ra+\bes}
x_{n_k} = x_0\in[a,b]$, то $\liml{k\ra+\bes} r_m(x_{n_k}) =
r_m(x_0)$, и, с учётом (5), $r_m(x_0) \ge \ep_0$ для всех $m\in\N$.
Последнее противоречит свойству $\liml{n\ra+\bes} r_n(x)=0$ для
любого $x\in[a,b]$.

Таким образом, предположение неверно и $r_n(x)\rra0$ на $[a,b]$; то
есть, ряд $\sum a_n(x)$ равномерно сходится на $[a,b]$.
\end{proof}

\subsubsection{Признак Дини равномерной сходимости функциональных
последовательностей}

\begin{theorem}
Если функции $f_n(x), \; n\in\N$, непрерывны на $[a,b]$ и в точках
$x\in[a,b]$ функциональная последовательность $(f_n(x))$ монотонная
одинаковой направленности (то есть, $(f_n(x))$ возрастает в каждой
$x\in[a,b]$, либо $(f_n(x))$ убывает в каждой точке $x\in[a,b]$) и
$(f_n(x))$ сходится на $[a,b]$ к функции $f(x)$, непрерывной на
$[a,b]$, то $(f_n(x))$ равномерно сходится к $f(x)$ на $[a,b]$.
\end{theorem}

\begin{proof}
Рассмотрим функции $a_1(x) = f_1(x), \; a_2(x) = f_2(x) - f_1(x)
\sco a_n(x) = f_n(x) - f_{n-1}(x), \; n\ge2$, все непрерывные на
$[a,b]$. Частные суммы $s_n(x), \; n\in\N$, ряда $\sum a_n(x)$ имеют
вид \equ{s_n(x_n) = a_1(x) + a_2(xx) \spl a_n(x) = f_1(x)+f_2(x) -
f_1(x) \spl f_n(x) - f_{n-1}(x) = f_n(x), \; n\in\N, \; x\in[a,b],}
и поэтому, сумма $s(x)$ ряда, $s(x) = \liml{n\ra+\bes}s_n(x) =
\liml{n\ra+\bes} f_n(x) = f(x)$ непрерывна на $[a,b]$. В силу
свойства монотонности, фигурирующего в условии теоремы, либо ряд
$\sum a_n(x)$, либо ряд $\sum(-a_n(x))$ удовлетворяет всем условиям
теоремы Дини для рядов, согласно которой ряд равномерно сходится на
$[a,b]$. Это означает, что $s_n(x) \rra s(x)$ на $[a,b]$, или
$f_n(x)\rra f(x)$ на $[a,b]$.
\end{proof}

\begin{note}
В двух последних теоремах отрезок можно заменить произвольным
компактом на $\R$.
\end{note}

\subsection{Свойства равномерно сходящихся рядов и
последовательностей}

\subsubsection{Непрерывность суммы равномерно сходящегося
функционального ряда}

\begin{theorem}
Если функции $a_n(z), \; n\in\N$, определены на некотором множестве
$E\subs\Cbb$ и ряд $\sum a_n(z)$ равномерно сходится на $E$ к сумме
$s(z)$, то функция $s(z)$ будет непрерывной в каждой точке $z_0\in
E$, в которой непрерывны все функции $a_n(z), \; n\in\N$.
\end{theorem}

\begin{proof}
Рассмотрим произвольные $\ep>0$ и $s_n(z) = \sumkun a_k(z), \; z\in
E, \; n\in\N$, так что $s(z) = s_n(z) + r_n(z), \; z\in E, \;
n\in\N$, где $r_n(z)$ --- сумма $n$--ого остаточного ряда для $\sum
a_n(z)$. Так как ряд равномерно сходится на $E$, то $r_n(z)\rra0$ на
$E$, и поэтому существует такое $N\in\N, \; N=N(\ep)$, что
$\hm{r_n(z)} < \frac{\ep}3$ для всех $z,z_0\in E$. Поскольку также
$s(z) = s_N(z) + r_N(z), \; z\in E$, то оценка
\mla{1}{\hm{s(z)-s(z_0)} = \hm{s_N(z) + r_N(z) - s_N(z_0) -
r_N(z_0)} \le \hm{s_N(z) - s_N(z_0)} +\\+ \hm{r_N(z)} +
\hm{r_N(z_0)} < \hm{s_N(z) - s_N(z_0)} + \frac{2\ep}3} справедлива
для всех $z,z_0\in E$. Функция $s_N(z) = \sum\limits_{k=1}^N a_k(z)$
непрерывна в точке $z_0\in E$, как сумма непрерывных функций
$a_k(z), \; k=\ol{1,N}$, и следовательно, для $\ep>0$ существует
$\de>0$, что \eqa{2}{\hm{s_N(z) - s_N(z_0)} < \frac{\ep}3} для всех
$z\in E, \; \hm{z-z_0}<\de$.

На основании (1) и (2), получаем оценку $\hm{s(z) - s(z_0)} <
\frac{\ep}3 + \frac{2\ep}3 = \ep$ для всех $z\in E, \;
\hm{z-z_0}<\de$, которая означает непрерывность функции $s(z)$ в
точке $z_0\in E$.
\end{proof}

\begin{imp*}
Равномерно сходящийся ряд непрерывных функций имеет своей суммой
непрерывную функцию.
\end{imp*}

\subsubsection{Непрерывность предельной функции равномерно сходящейся
функциональной последовательности} %в оригинале --- не 3.2, а 3.1'

\begin{theorem}
Если функциональная последовательность $(f_n(z))$ равномерно
сходится на множестве $E\subs\Cbb$ к предельной функции $f(z)$ и
каждая $f_n(z), \; n\in\N$, непрерывна в точке $z_0\in E$, то
предельная функция $f(z)$ непрерывна в $z_0$.
\end{theorem}

\begin{proof}
Рассмотрим функции $a_1(z) = f_1(z), \; a_n(z) = f_n(z) -
f_{n-1}(z), \; n\ge2, \; z\in E$, непрерывные в точке $z_0\in E$.
Функциональный ряд $\sum a_n(z)$ имеет частные суммы \equ{s_n(z) =
\sumkun a_k(z) = f_1(z) + f_2(z) - f_1(z) \spl f_n(z) - f_{n-1}(z) =
f_n(z), \; z\in E, \; n\in\N.} Поэтому, условие $f_n(z)\rra f(z)$ на
$E$ равносильно свойству $s_n(z)\rra f(z)$ на $E$; то есть, свойству
равномерной сходимости ряда $\sum a_n(z)$ к сумме $s(z)=f(z)$. По
теореме предыдущего пункта, функция $f(z)=s(z)$ непрерывна в точке
$z_0\in E$.
\end{proof}

\subsubsection{Предел по базе суммы равномерно сходящегося функционального ряда}

\begin{theorem}
Если функциональный ряд $\sum a_n(x)$ равномерно сходится на
множестве $E\subs\R$ и $\B$ --- база на $E$, по которой существуют
$\liml{\B}a_n(x)=c_n$ для всех $n\in\N$, то числовой ряд $\sum c_n$
сходится и его сумма равна пределу суммы функционального ряда по
базе $\B$; то есть, \equ{\liml{\B}\sumnui a_n(x) = \sumnui\liml{\B}
a_n(x).}
\end{theorem}

\begin{proof}
Обозначим $s(x) = \sumnui a_n(x), \; x\in E$, и рассмотрим
произвольное $\ep>0$. Так как ряд $\sum a_n(x)$ равномерно сходится
на $E$ к $s(x)$, то по критерию Коши, существует $N\in\N, \;
N=N(\ep)$, что \eqa{3}{\hm{a_{n+1}(x) \spl a_{n+p}(x)}<\frac{\ep}3}
для всех $x\in E$ и всех $n,p\in\N, \; n\ge N$. Опираясь на свойства
линейности и монотонности предела функции по базе, а также на
свойство непрерывности функции $\hm{t}, \; t\in\R$, на основании (3)
получим, что \ml{\hm{c_{n+1} \spl c_{n+p}} = \hm{\liml{\B}
a_{n+1}(x) \spl \liml{\B} a_{n+p}(x)} = \hm{\liml{\B} (a_{n+1}(x)
\spl a_{n+p}(x))} =\\= \liml{\B} \hm{a_{n+1}(x) \spl a_{n+p}(x)} \le
\frac{\ep}2 < \ep} справедливо для всех $n,p\in\N, \; n\ge
N=N(\ep)$, что является критерием Коши сходимости числового ряда
$\sum c_n$ к некоторой сумме $c$.

Так как ряд $\sum a_n(x)$ равномерно сходится на $E$ к функции
$s(x)$, то для $\ep>0$ существует $N_1\in\N, \; N_1=N_1(\ep)$, что
$\hm{s(x)-s_n(x)}<\frac{\ep}3$ для всех $n\ge N_1$. Так как $\sumnui
c_n=c$, то существует $N_2\in\N, \; N_2=N_2(\ep)$, что
$\hm{c-s_n}<\frac{\ep}3$ для всех $n\ge N_2$, где $s_n = \sumkun
c_k, \; n\in\N$. Обозначая $N=\max(N_1,N_2), \; N=N(\ep)$,
заключаем, что $\hm{s(x) - s_N(x)} < \frac{\ep}3$ для всех $x\in E$
и $\hm{c-s_n}<\frac{\ep}3$. Поэтому \mla{4}{\hm{s(x)-c} = \hm{s(x) -
s_N(x) + s_N(x) - c + s_N - s_N} = \hm{s_N(x) - s_N + s(x) - s_N(x)
+ s_N - c} \le\\\le \hm{s_N(x) - s_N} + \hm{s(x) - s_N(x)} + \hm{s_N
- c} < \hm{s_N(x) - s_N} + \frac{2\ep}3, \; x\in E.}

По условию и по свойству линейности предела функции по базе,
\equ{s_N= \sum\limits_{k=1}^N c_k = \sum\limits_{k=1}^N \liml{\B}
a_k(x) = \liml{\B} \sum\limits_{k=1}^N a_k(x) = \liml{\B} s_N(x).}
Поэтому, для числа $\ep>0$ существует такой элемент $B_{\ep}$ базы
$\B$, на котором \eqa{5}{\hm{s_N(x) - s_N} < \frac{\ep}3} для всех
$x\in B_{\ep}$. На основании (4) и (5) заключаем, что неравенство
$\hm{s(x)-c}<\frac{\ep}3 + \frac{2\ep}3 = \ep$ справедливо для всех
$x\in B_{\ep}$; то есть, что $c = \liml{\B}s(x)$.
\end{proof}

\subsubsection{Предел по базе предельной функции равномерно
сходящейся функциональной последовательности}
%original 3.2'

\begin{theorem}
Если $f_n(x)\rra f(x)$ на множестве $E\subs\R, \; E\ne\es$, и на
базе $\B$, лежащей в $E$, существуют $\liml{\B}f_n(x) = l_n, \;
n\in\N$, то существует $\liml{\B}l_n=l$ и $l=\liml{\B} f(x)$; то
есть, \equ{\liml{\B} \hr{\liml{n\ra+\bes} f_n(x)} = \liml{n\ra+\bes}
\hr{\liml{\B} f_n(x)}.}
\end{theorem}

\begin{proof}
Доказать самостоятельно, используя ряд $\sum a_n(x), \; a_1=f_1(x),
\; a_n(x) = f_n(x) - f_{n-1}(x), \; n\in\N, \; x\in E$.
\end{proof}

\subsubsection{Почленное интегрирование функционального ряда}
\begin{theorem}
Если функциональный ряд $\sum a_n(x)$ равномерно сходится на
$[a,b]\subs\R$ к сумме $s(x)$ и все $a_n(x), \; n\in\N$, непрерывны
на $[a,b]$, то \eqa{6}{\intl{a}b s(x)\,dx = \sumnui \intl{a}b
a_n(x)\,dx.}
\end{theorem}

\begin{proof}
Согласно следствию к теореме пункта 2.3.1, функция $s(x)$ непрерывна
на $[a,b]$, и следовательно, все интегралы в формуле (6) существуют.
Рассмотрим частные суммы $s_n(x) = \sumkun a_k(x), \; n\in\N$, и
суммы $r_n(x), \; n\in\N$, остатков ряда $\sum a_n(x)$, так что
$s(x) = s_n(x)+r_n(x), \; x\in[a,b], \; n\in\N$, и $r_n(x)\rra0$ на
$[a,b]$ в силу условия равномерной сходимости ряда. Так как функции
$s_n(x), \; n\in\N$, непрерывны на $[a,b]$ как конечные суммы
непрерывных функций, то все функции $s(x), \; s_n(x), \; r_n(x), \;
n\in\N$, непрерывны на $[a,b]$ и \eqa{7}{\intl{a}b s(x)\,dx =
\intl{a}b s_n(x)\,dx + \intl{a}b r_n(x)\,dx = \sumkun
\intl{a}{b}a_k(x)\,dx + \intl{a}{b}r_n(x)\,dx} по линейному свойству
интеграла Римана. Формула (6) следует из (7), если справедливо
утверждение, что \eqa{8}{\intl{a}b r_n(x)\,dx=0.}

Для проверки (8) рассмотрим произвольное число $\ep>0$. Так как
$r_n(x)\rra0$ на $[a,b]$, то существует $N\in\N, \; N=N(\ep)$, что
$\hm{r_n(x)} < \frac{\ep}{2(b-a)}$ для всех $x\in[a,b]$ и всех $n\ge
N$. Поэтому, справедлива оценка \equ{\hm{\intl{a}{b}r_n(x)\,dx} \le
\intl{a}b\hm{r_n(x)}\,dx \le \frac{\ep}{2(b-a)}(b-a) < \ep} для всех
$n\ge N, \; N=N(\ep)$, из которой следует (8).
\end{proof}

\subsubsection{Интегрирование предельной функции равномерно
сходящейся функциональной последовательности}
%orig 3.3'
\begin{theorem}
Если функциональная последовательность $(f_n(x))$ равномерно
сходится на $[a,b]\subs\R$ к предельной функции $f(x)$ и все
$f_n(x), \; n\in\N$, непрерывны на $[a,b]$, то справедлива формула
\eqa{6'}{\intl{a}b f(x)\,dx = \liml{n\ra+\bes}\intl{a}b f_n(x)\,dx.}
\end{theorem}

\begin{proof}
Доказать самостоятельно, используя ряд $\sum a_n(x), \; a_1(x) =
f_1(x), \; a_n(x) = f_n(x) - f_{n-1}(x), \; n\ge 2$, $x\in[a,b]$.
\end{proof}

\begin{ex}
Рассмотри последовательность $(f_n(x)), \; x\in[0,1], \; n\in\N$ и
график функции $f_n(x)$ изображен на рисунке.
%рисунок!
Тогда $f_n(0), \; n\in\N$, и $\liml{n\ra+\bes}f_n(0)=0=f(0)$. Для
любого $x\in(0,1]$ рассмотрим такое $N\in\N$, чтобы $N>\frac1x$, или
$x>\frac1N$. Для всех $n\ge N$ имеем $x>\frac1N \ge \frac1n$, и
следовательно, на основании рисунка, $f_n(x)=0$ для всех $n\ge N$,
так что $\liml{n\ra+\bes}f_n(x)=0$. Итак, $f_n(x)\ra f(x)=0, \;
x\in[0,1]$. Эта сходимость не является равномерной, поскольку для
любого $n\in\N$ выберем $x_n=\frac1{2n}\in[0,1]$, в которой
$\hm{f_n(x_n) - f(x_n)} = f_n(x_n)=2n$, и $\liml{n\ra+\bes}
\hm{f_n(x)-f(x)}=+\bes$. При этом, на основании геометрического
свойства интеграла,
\equ{\intl01f(x)\,dx=0\ne1=\intl01f_n(x)\,dx=\intl0{1/n}f_n(x)\,dx}
для всех $n\in\N$.
\end{ex}

\subsubsection{Почленное дифференцирование функциональных рядов}

\begin{theorem}
Если все функции $a_n(x), \; n\in\N$, непрерывно дифференцируемы на
отрезке $[a,b]$, ряд $\sum a_n(x)$ сходится на $[a,b]$ к сумме
$s(x)$ и ряд $\sum a_n'(x)$ равномерно сходится на $[a,b]$ к сумме
$s^*(x)$, то функция $s(x)$ дифференцируема на $[a,b]$ и её
производная $s'(x) = s^*(x), \; x\in[a,b]$; то есть,
\equ{\hr{\sumnui a_n(x)}' = \sumnui a_n'(x), \; x\in[a,b].}
\end{theorem}

\begin{proof}
Поскольку все производные $a_n'(x), \; n\in\N$, непрерывны на
$[a,b]$, то по теореме предыдущего пункта функция $s^*(x) = \sumnui
a_n'(x)$ непрерывна на $[a,b]$ и для любого $x\in[a,b]$ справедлива
формула \eqa{9}{\intl{a}x s^*(t)\,dt = \sumnui \intl{a}x a'_n(t)\,dt
= \sumnui (a_n(x) - a_n(a)) = s(x) - s(a), \; x\in[a,b].} Так как
функция $s^*$ непрерывна на $[a,b]$, то существует
$\hr{\intl{a}xs^*(t)\,dt}'=s^*(x), \; x\in[a,b]$, и, согласно (9),
$s^*(x) = (s(x)-s(a))', \; x\in[a,b]$. Так как $(s(a))'=0$, то, по
свойству линейности операции дифференцирования, существует $s'(x)$ и
$s'(x)=s^*(x), \; x\in[a,b]$.
\end{proof}

\subsubsection{Дифференцируемость гамма-функции Эйлера}

Согласно формуле Вейерштрасса, \equ{\frac1{\Ga(x+1)} = e^{cx}
\prodl{n=1}{\bes} \hr{1+\frac{x}n} e^{-\frac{x}n}, \; x\notin -\N,}
и бесконечное произведение абсолютно сходится для всех $x\notin
-\N$. Поскольку $\Ga(x+1)=x\Ga(x), \; x\notin -\N_0$,
$\N_0=\N\cup\hc{0}$, то \equ{\frac1{\hm{x}\hm{\Ga(x)}} = e^{cx}
\prods{n=1}^{\bes} \hm{1+\frac{x}{n}} e^{-\frac{x}{n}}, \;
x\notin-\N_0,} и \equ{-\ln\hm{x}-\ln\hm{\Ga(x)} = cx + \sumnui
\hs{\ln\hm{1+\frac{x}n}-\frac{x}n}, \; x\notin-\N_0,}
\eqa{10}{\ln\hm{\Ga(x)}=-\ln\hm{x}-cx + \sumnui \hs{\frac{x}n -
\ln\hm{1+\frac{x}n}}, \; x\notin -\N_0.}

Формальное дифференцирование левой и правой частей в формуле (10)
приводит к \eqa{11}{\frac{\Ga'(x)}{\Ga(x)} = -\frac1x-c+\sumnui
\hs{\frac1n - \frac1{n+x}}, \; x\notin-\N_0,} и формула (11) будет
иметь место, если мы докажем, что ряд в её правой части равномерно
сходится на каждом отрезке $\De\subs\R$, не содержащем точек
множества $-\N_0$ (по теореме предыдущего пункта). На отрезке $\De$
справедлива оценка $\hm{x}\le M$ для всех $x\in\De$ и некоторого
числа $M>0$. Поэтому \equ{\hm{\frac1n - \frac1{n+x}} =
\frac{\hm{x}}{n \hm{n+x}} \le \frac{M}{n(n-M)}} для всех $x\in\De$ и
всех $n\in\N, \; n>M$. Так как положительный ряд $\sums{n>M}
\frac{M}{n(n-M)}$ сходится, то по признаку Вейерштрасса на $\De$
равномерно сходится ряд $\sum \hs{\frac1n - \frac1{n+x}}$, и
следовательно, формула (11) справедлива для всех $x\notin -\N_0$.

Итак, функция $\Ga(x)$ дифференцируема всюду в своей области
определения и справедлива формула (11).

По определению, \ml{\sumnui \hr{\frac1{n+1}-\frac1n} =
\liml{n\ra+\bes} \sumkun \hr{\frac1{k+1} - \frac1k} =
\liml{n\ra+\bes} \hr{\frac1n - 1 + \frac13 - \frac12 \spl
\frac1{n+1} - \frac1n} =\\= \liml{n\ra+\bes} \hr{\frac1{n+1} - 1} =
-1,} или \eqa{12}{1 + \sumnui\hr{\frac1{n+1} - \frac1n}=0.}

На основании (11) и (12) имеем \eqa{13}{\frac{\Ga'(x)}{\Ga(x)} = -c
+ 1 - \frac1x + \sumnui \hs{\frac1{n+1} - \frac1{n+x}} = -c
+\sums{n=0}^{\bes} \hs{\frac1{n+1} - \frac1{n+x}}, \; x\notin -\N_0.
}

Формальное дифференцирование формулы (13) приводит к
\eqa{14}{\hr{\frac{\Ga'(x)}{\Ga(x)}}' = \sums{n=0}^{\bes}
\frac1{(n+x)^2}, \; x\notin-\N_0} и равенство (14) справедливо, если
будет доказано, что ряд в его правой части равномерно сходится на
каждом отрезке $\De\subs\R$, не содержащем точек множества $-\N_0$.
Как и выше, для всех $x\in\De$ справедлива оценка
$\hm{\frac1{(n+x)^2}} < \frac1{(n-M)^2}$ при всех $n\in\N, \; n>M$,
и так как положительный ряд $\sums{n>M} \frac1{(n-M)^2}$ сходится,
по признаку Вейерштрасса ряд $\sums{n=0}^{\bes} \frac1{(n+x)^2}$
равномерно сходится на $\De$, и следовательно, (14) справедливо
всюду на $\R$, кроме $x\in -\N_0$.

Итак, доказано, что функция $\Ga(x)$ обладает второй производной
$\Ga''(x)$ всюду в своей области определения.

Дальнейшее дифференцирование формулы (14) (ряда в её правой части)
также приводит к рядам, равномерно сходящимся на каждом отрезке
$\De\subs\R$, не содержащем точек множества $-\N_0$ (по признаку
Вейерштрасса). Это показывает, что функция $\Ga(x)$ бесконечно
дифференцируема в своей области определения $x\notin-\N_0$.

\subsection{Степенные ряды}

\subsubsection{Область абсолютной сходимости степенного ряда}

Рассмотрим произвольную последовательность $(c_n)$ комплексных чисел
$c_n\in\Cbb, \; n\in\N$, комплексное число $c_0$ и произвольное
комплексное число $z_0\in\Cbb$. Функциональный ряд \eqa{1}{c_0 +
c_1(z-z_0) + c_2(z-z_0)^2 \spl c_n(z-z_0)^n +\ldots = \sumnzi
c_n(z-z_0)^n, \; z\in\Cbb,} называют \textbf{степенным рядом} с
\textbf{центром} в точке $z_0$; числа $c_n, \; n\in\N_0 =
\N\cup\hc{0}$, называют \textbf{коэффициентами} степенного ряда (1).
Изучение свойств степенного ряда (1) сводят к случаю, когда $z_0=0$;
то есть, изучают степенной ряд \eqa{2}{c_0 + c_1z + c_2z^2 \spl
c_nz^n + \ldots = \sumnzi c_nz^n, \; z\in\Cbb.}

\begin{theorem}
Пусть $(c_n)$ --- последовательность комплексных чисел и
\eqa{3}{\uliml{n\ra+\bes} \sqrt[n]{\hm{c_n}}=\rho \mbox{ и }
R=\frac1{\rho},} где в левой части последнего равенства считаем
$R=0$, если $\uliml{n\ra+\bes} \sqrt[n]{\hm{c_n}}=+\bes$, и
$R=+\bes$, если $\uliml{n\ra+\bes} \sqrt[n]{\hm{c_n}}=0$. Тогда ряд
(2) абсолютно сходится, если $\hm{z}<R$, и расходится, если
$\hm{z}>R$; в случае $R=+\bes$ ряд (2) абсолютно сходится для всех
$z\in\Cbb$, в случае $R=0$ ряд (2) сходится только в своём центре
$z_0=0$.
\end{theorem}

\begin{proof}
Обозначим $a_n(z) = c_nz^n, \; n\in\N_0$. Тогда $\uliml{n\ra+\bes}
\sqrt[n]{\hm{a_n(z)}} = \uliml{n\ra+\bes} \hm{z}
\sqrt[n]{\hm{c_n}}$. Если $\uliml{n\ra+\bes} \sqrt[n]{\hm{c_n}} =
+\bes$ (и $R=0$), то $\uliml{n\ra+\bes} \sqrt[n]{\hm{a_n(z)}}$
существует и равен $0$ только когда $z=0$, $\hm{z}=0$, а для любого
$z\in\Cbb, \; z\ne0$, имеем $\hm{z}>0$ и $\uliml{n\ra+\bes} \hm{z}
\sqrt[n]{\hm{c_n}}=+\bes$, так что по радикальному признаку Коши ряд
(2) сходится в точке $z=0$ и для $z\ne0$ положительный ряд $\sumnzi
\hm{a_n(z)}$ расходится, а последовательность его членов
$(\hm{a_n(z)})$ не является бесконечно малой; то есть, не является
бесконечно малой последовательность $(a_n(z))$ для любого $z\ne0$ и
ряд (2) расходится при всех $z\ne0$. Если $\uliml{n\ra+\bes}
\sqrt[n]{\hm{c_n}}=0$ (и $R=+\bes$), то по свойствам верхнего
предела (глава 1, пункт 1.1.1), \equ{\uliml{n\ra+\bes}
\sqrt[n]{\hm{a_n(z)}} = \uliml{n\ra+\bes} \hm{z} \sqrt[n]{\hm{c_n}}
= \hm{z} \uliml{n\ra+\bes} \sqrt[n]{\hm{c_n}}=0} для всех
$z\in\Cbb$, так что по радикальному признаку Коши положительный ряд
$\sumnzi \hm{a_n(z)}$ сходится для всех $z\in\Cbb$; то есть, ряд
$\sumnzi a_n(z)$ абсолютно сходится для всех $z\in\Cbb$.

Пусть теперь $\uliml{n\ra+\bes}=\rho, \; 0<\rho<+\bes$. Тогда, как и
выше, $\uliml{n\ra+\bes} \sqrt[n]{\hm{a_n(z)}}=\hm{z}
\uliml{n\ra+\bes} \sqrt[n]{\hm{c_n}} = \hm{z}\rho$. Согласно
радикальному признаку Коши, положительный ряд $\sumnzi \hm{a_n(z)}$
сходится, если $\hm{z}\rho<1$, или $\hm{z} < \frac1{\rho}=R$, и
расходится, если $\hm{z}\rho>1$, или $\hm{z} > \frac1{\rho}=R$,
причём последовательность его членов $(\hm{a_n(z)})$ в этом случае
не является бесконечно малой, равно как и последовательность
$(a_n(z))$. Другими словами, ряд $\sumnzi a_n(z)$ абсолютно сходится
для всех $z, \; \hm{z}<R$, и ряд расходится для всех $z$,
$\hm{z}>R$.

Итак, областью абсолютной сходимости ряда (2) служит круг $\hm{z}<R$
на $\Cbb$ и ряд (2) расходится при $\hm{z}>R$, где $R$ определяется
формулой (3).
\end{proof}

Формулу (3) называют формулой Коши--Адамара, а величину $R$ в ней
--- \textbf{радиусом} сходимости степенного ряда (2).

\begin{imp}
(Первая теорема Абеля). Если степенной ряд (2) сходится в точке
$z_1\in\Cbb$, $z_1\ne0$, то ряд (2) абсолютно сходится в круге
$\hm{z}<\hm{z_1}$.
\end{imp}

\begin{proof}
Так как ряд (2) сходится в $z_1$, то по теореме $\hm{z_1}\le R$ и
$R\ge\hm{z_1}>0$. Так как $z_1\ne0$, то для всех $z, \;
\hm{z}<\hm{z_1}$, справедливо $\hm{z} < \hm{z_1} \le R, \;
\hm{z}<R$, и по теореме ряд (2) абсолютно сходится в $z$.
\end{proof}

\begin{imp}
Если существует $\liml{n\ra+\bes} \sqrt[n]{\hm{c_n}}=l\ge0$, то ряд
(2) абсолютно сходится при $\hm{z}<\frac1l$ и расходится при
$\hm{z}>\frac1l$.
\end{imp}

\begin{proof}
В нашем случае число $l=\rho = \uliml{n\ra+\bes}\sqrt[n]{\hm{c_n}}.$
\end{proof}

\begin{imp}
Если коэффициенты ряда (2) отличны от нуля, начиная с некоторого
индекса, и существует $\liml{n\ra+\bes} \hm{\frac{c_{n+1}}{c_n}} =
l_1\ge0$, то ряд (2) абсолютно сходится в круге $\hm{z} <
\frac1{l_1}$ и расходится при $\hm{z} > \frac1{l_1}$.
\end{imp}

\begin{proof}
Задача 4 в списке обязательных задач коллоквиума утверждает, что
если существует $\liml{n\ra+\bes} \hm{\frac{c_{n+1}}{c_n}}=l_1$, то
$l_1 = \liml{n\ra+\bes} \sqrt[n]{\hm{c_n}}$, и применяем следствие
2.
\end{proof}

\subsubsection{Степенные ряды в действительной области}
В этом пункте рассмотрим степенной ряд \eqa{2'}{\sumnzi c_nx^n=c_0 +
c_1x \spl c_nx^n + \ldots, \; c_n, \; x\in\R, \; n\in\N_0.}

Его областью абсолютной сходимости служит интервал $(-R,R)$, который
может быть всей числовой прямой $(-\bes,+\bes)$ или вырождаться в
точку $x=0$, а $R$ вычисляется по формуле (3). Областью сходимости
ряда (2') будет промежуток $\ha{-R,R}$.

\begin{ex}
Ряд $\sumnui n^n x^n$ сходится только при $x=0$, так как $c_n=n^n,
\; n\in\N$, и $\uliml{n\ra+\bes} \sqrt[n]{\hm{c_n}} =
\uliml{n\ra+\bes} n = +\bes$.
\end{ex}

\begin{ex}
Ряд $\sumnzi \frac{x^n}{n!} = 1+x+\frac{x^2}2 \spl \frac{x^n}{n!} +
\ldots$ абсолютно сходится на всей числовой прямой, так как
$c_n=\frac1{n!}$ и $\liml{n\ra+\bes} \hm{\frac{c_{n+1}}{c_n}} =
\liml{n\ra+\bes} \frac{n!}{(n+1)!} = \liml{n\ra+\bes} \frac1{n+1}=0$
и $R=+\bes$.
\end{ex}

\begin{ex}
Ряд $\sumnui \frac{(-1)^{n+1} x^n}{n}$ абсолютно сходится в
интервале (-1,1), так как $c_n = \frac{(-1)^{n+1}}n, \;
\hm{c_n}=\frac1n, \; n\in\N$, и $\uliml{n\ra+\bes}
\sqrt[n]{\hm{c_n}} = \liml{n\ra+\bes} \frac1{\sqrt[n]{n}}=1, \;
R=1$. При $x=1$ ряд $\sum \frac{(-1)^{n+1}}{n}$ сходится; при $x=-1$
ряд $\sum \frac{(-1)^{n+1}}{n} (-1)^n = -\sum\frac1n$ расходится, и
областью сходимости ряда служит промежуток $(-1,1]$.
\end{ex}

\begin{ex}
Ряд $\sum \frac{x^n}{n^2}$ имеет $c_n=\frac1{n^2}, \; n\in\N$, и
$\uliml{n\ra+\bes} \sqrt[n]{\hm{c_n}}=\liml{n\ra+\bes}
\frac1{\sqrt[n]{n^2}}=1$, так что областью абсолютной сходимости
служит интервал $(-1,1)$. При $x=\pm1$ имеем $\sum
\frac{(\pm1)^n}{n^2}$ и $\hm{\frac{(\pm1)^n}{n^2}}=\frac1{n^2}$, ряд
$\sum\frac1{n^2}$ сходится. Таким образом, областью сходимости и
абсолютной сходимости будет отрезок $[-1,1]$.
\end{ex}

\begin{ex}
$\sumnzi (-1)^n x^{2n} = 1 - x^2 + x^4 - x^6 \spl (-1)^n x^{2n} +
\ldots = \frac1{1+x^2}, \; \hm{x}<1$. Заметим, что $c_{2k} = (-1)^k,
\; c_{2k-1}=0$, $k\in\N, \; \hm{c_{2k}}=1, \; \hm{c_{2k-1}}=0$ и
$\uliml{n\ra+\bes} \sqrt[n]{\hm{c_n}}=1, \; \lliml{n\ra+\bes}
\sqrt[n]{\hm{c_n}}=0$; следовательно, $R=1$ и ряд абсолютно сходится
в интервале $(-1,1)$. В точках $x_{1,2}=\pm1$ ряд расходится, хотя
его сумма $\frac1{1+x^2}$ в этих точках определена и равна
$\frac12$. Если вместо $x$ взять $z\in\Cbb$, то $\sumnzi (-1)^n z^n
= \frac1{1+z^2}, \; \hm{z}<1$, и функция $\frac1{1+z^2}$ не
существует в точках $z_{1,2}=\pm i, \; (\pm i)^2=-1$, лежащих на
границе круга $\hm{z}<1$ абсолютной сходимости ряда, $\hm{z_{1,2}} =
\hm{\pm i}=1$.
\end{ex}

\subsubsection{Равномерная сходимость степенных рядов}

\begin{theorem}
Если степенной ряд (2) имеет ненулевой радиус сходимости $R$ ($>0$),
то он равномерно сходится на каждом замкнутом круге $\hm{z}\le r$
для всех $r, \; 0<r<R$.
\end{theorem}

\begin{proof}
Точка $z=r$ лежит в круге $\hm{z}<R$ и в ней ряд (2), по теореме
пункта 4.1, сходится абсолютно; то есть, сходится положительный ряд
$\sumnzi \hm{c_n} r^n$. Для любого $z$, \; $\hm{z}\le r$,
справедливо $\hm{a_n(z)} = \hm{c_n z^n} = \hm{c_n} \hm{z}^n \le
\hm{c_n} r^n, \; n\in\N_0$, и по признаку Вейерштрасса ряд $\sumnzi
a_n(z) = \sumnzi c_nz^n$ равномерно сходится на $\hm{z}\le r$.
\end{proof}

\subsubsection{Непрерывность суммы степенного ряда}

\begin{theorem}
Если степенной ряд (2) имеет ненулевой радиус сходимости $R$ ($>0$),
то его сумма $s(z)$ непрерывна в круге $\hm{z}<R$.
\end{theorem}

\begin{proof}
Рассмотрим произвольную точку $z$, $\hm{z}<R$. Существует $r$,
$0<r<R$, что $\hm{z}<r<R$. По теореме предыдущего пункта, ряд (2)
равномерно сходится на $\hm{z}\le r$. Так как каждая функция
$a_n(z)=c_nz^n$ непрерывна на $\Cbb$, то по теореме пункта 3.1, его
сумма $s(z)$ непрерывна в круге $\hm{z}\le r$; в частности, она
непрерывна в точке $z$, $\hm{z}<r$.
\end{proof}

\subsubsection{Вторая теорема Абеля}

\begin{theorem}
Если степенной ряд $(2') \; \sumnzi c_nx^n$ имеет ненулевой радиус
сходимости $R$ ($>0$) и ряд сходится в $x=R$ (в $x=-R$), то он
равномерно сходится на отрезке $[0,R]$ (на $[-R,0]$) и его сумма
$s(x)$ непрерывна на $[0,R]$ (на $[-R,0]$), и в частности, она
непрерывна слева в $x=R$ (непрерывна справа в $x=-R$); то есть,
существует $\liml{x\ra R-0} s(x)=s(R)=\sumnzi c_n R^n$
$\hr{\mbox{существует} \liml{x\ra -R+0} s(x) = s(-R) = \sumnzi
(-1)^n c_nR^n}$.
\end{theorem}

\begin{proof}
Рассмотрим вначале случай $x=R$. Тогда $s(x) = \sumnzi c_nx^n =
\sumnzi c_nR^n \hr{\frac{x}R}^n, \; x\in[0,R]$. Числовой ряд
$\sumnzi c_nR^n$ сходится, и его можно считать равномерно сходящимся
на $[0,R]$ функциональным рядом с постоянными членами, а функции
$f_n(x) = \hr{\frac{x}R}^n, \; n\in\N$, образуют на $[0,R]$
убывающую последовательность в каждой точке $x\in[0,R]$, равномерно
ограниченную на $[0,R]$; так как $1\ge \frac{x}R \ge
\hr{\frac{x}R}^2 \ge \ldots \ge \hr{\frac{x}R}^n \ge \ldots$.
Согласно признаку Абеля, функциональный ряд $\sumnzi c_nx^n$
равномерно сходится на $[0,R]$. Так как каждая функция $c_nx^n$
непрерывна на $\R$, то и сумма ряда $s(x)$ непрерывна на $[0,R]$; в
частности, $s(x)$ непрерывна слева в точке $x=R$.

Пусть теперь ряд $(2')$ сходится в $x=-R$, то есть, сходится
числовой ряд $\sumnzi (-1)^nc_nR^n$. Степенной ряд $\sumnzi (-1)^n
c_n y^n$
%хз, тут x или y
имеет радиус сходимости $R_1 = \frac1{\rho_1}, \; \rho_1 =
\uliml{n\ra+\bes} \sqrt[n]{\hm{(-1)^n c_n}} = \uliml{n\ra+\bes}
\sqrt[n]{\hm{c_n}} = \rho$ и $R_1=R$, и ряд сходится в точке $y=R$.
По доказанному, его сумма $s^*(y)$ непрерывна на $[0,R]$. Так как
для $s(x) = \sumnzi c_nx^n$ справедливо $s(x) = s^*(-y)$,
%тут тоже хз
то $s(x)$ непрерывна на $[-R,0]$.
\end{proof}

\subsubsection{Почленное дифференцирование и интегрирование
степенных рядов}

\begin{theorem}
Если степенной ряд $\sumnzi c_nx^n$ имеет ненулевой радиус
сходимости $R$ ($>0$) и сходится к сумме $f(x) = \sumnzi c_nx^n$,
$x\in(-R,R)$, то в интервале $(-R,R)$ справедливы формулы
\eqa{4}{\intl0x f(t)\,dt = \sumnzi \frac{c_n}{n+1} x^{n+1} = c_0x +
\frac{c_1}2 x^2 \spl \frac{c_n}{n+1} x^{n+1} + \ldots,} и
\eqa{5}{f'(x) = \sumnui nc_n x^{n-1} = c_1 + 2c_2x \spl nc_n x^{n-1}
+ \ldots.}
\end{theorem}

\begin{proof}
Заметим, что степенные ряды $\sumnzi b_nx^n$, $\sumnui b_nx^{n-1}$,
$\sumnzi b_nx^{n+1}$ имеют один и тот же интервал (и радиус)
сходимости, поскольку каждый получается из другого умножением или
делением на функцию $\ph(x)=x$. Поэтому ряды $\sumnui nc_n x^{n-1}$
и $\sumnzi nc_nx^n$ имеют одинаковые радиусы сходимости, а радиус
сходимости $R_1$ последнего, вычисляемый по формуле (3). равен $R_1
= \frac1{\rho_1}$, $\rho_1 = \uliml{n\ra+\bes} \sqrt[n]{\hm{nc_n}} =
\uliml{n\ra+\bes} \sqrt[n]{n} \cdot \sqrt[n]{\hm{c_n}} =
\liml{n\ra+\bes} \sqrt[n]{n} \cdot \uliml{n\ra+\bes}
\sqrt[n]{\hm{c_n}} = \uliml{n\ra+\bes} \sqrt[n]{\hm{c_n}} = \rho$, и
$R_1=R$. Аналогично, ряды $\sumnzi \frac{c_n}{n+1} x^{n+1}$ и
$\sumnzi \frac{c_n}{n+1}x^n$ имеют одинаковые радиусы сходимости, и
радиус сходимости $R_2$ последнего ряда равен $R_2 =
\frac1{\rho_2}$, где $\rho_2 = \uliml{n\ra+\bes}
\sqrt[n]{\frac{\hm{c_n}}{n+1}} = \liml{n\ra+\bes}
\frac1{\sqrt[n]{n+1}} \cdot \uliml{n\ra+\bes} \sqrt[n]{\hm{c_n}} =
\uliml{n\ra+\bes} \sqrt[n]{\hm{c_n}} = \rho$, так что $R_2=R$.

Итак, радиусы сходимости рядов (2'), (4) и (5) одинаковые и
совпадают с радиусом сходимости $R$ ряда (2').

Тогда ряды (2'), (4) и (5) равномерно сходятся на каждом отрезке
$[-r,r], \; 0<r<R$, и так как их члены непрерывны на $[-r,r]$, то на
каждом $[-r,r]$ выполнены все условия теорем о почленном
дифференцировании и интегрировании функциональных рядов, согласно
которым справедливы формулы (4) и (5).
\end{proof}

\begin{imp}
Если степенной ряд $\sumnzi c_nx^n$ имеет ненулевой радиус
сходимости $R$ ($>0$), то его сумма $f(x)$ бесконечно
дифференцируема в интервале $(-R,R)$ и для каждого $k\in\N$
справедлива формула \eqa{6}{f^{(k)}(x) = \sums{n=k}^{\bes} n(n-1)
\sd (n-k+1) c_n x^{n-k}, \; x\in(-R,R).}
\end{imp}

\subsubsection{Единственность разложения в степенной ряд}

\begin{theorem}
Если степенной ряд $\sumnzi c_nx^n$ имеет ненулевой радиус
сходимости $R$ ($>0$) и сумму $f(x)$, то $c_0=f(0), \; c_k =
\frac{f^{(k)}(0)}{k!}, \; k\in\N$.
\end{theorem}

\begin{proof}
Поскольку $f(x) = c_0+c_1x \spl c_nx^n + \ldots, \; x\in(-R,R)$, то
$f(0) = c_0$. Согласно (5), \equ{f'(x) = c_1+2c_2(x) \spl
nc_nx^{n-1} + \ldots, \; x\in(-R,R),} и $f'(0)=c_1$ или $c_1 =
\frac{f'(0)}{1!}$. Согласно (6), \equ{f^{(k)}(x) = k(k-1) \sd
(k-k+1) c_k + (k+1)k \sd (k+1-k+1) c_{k+1}x + \ldots, \;
x\in(-R,R),} и $f^{(k)}(0)=k! c_k$, или $c_k =
\frac{f^{(k)}(0)}{k!}, \; k\in\N$.
\end{proof}

\begin{theorem}
Если степенной ряд $\sumnzi c_n(x-x_0)^n$ имеет ненулевой ряд
сходимости $R$ ($>0$), то в интервале его абсолютной сходимости
$(-R+x_0, R+x_0)$ его сумма $f(x)$ представима в виде \eqa{7}{f(x) =
\sumnzi \frac{f^{(n)}(x_0)}{n!} (x-x_0)^n, \; x\in(-R+x_0,R+x_0).}
\end{theorem}

\begin{proof}
Положив $t=x-x_0$, переведём ряд $\sumnzi c_n(x-x_0)^n$ в ряд
$\sumnzi c_nt^n = g(t), \; t\in(-R,R)$, и $g(x-x_0)=f(x)$. Согласно
предыдущей теореме, $c_n = \frac{g^{(n)}(0)}{n!} =
\frac{f^{(n)}(x_0)}{n!}$, $n\in\N_0 = \N\cup\hc{0}$, так что
справедлива формула (7).
\end{proof}

Ряд (7) называют \textbf{рядом Тейлора} функции $f(x)$.

Итак, \textbf{любой степенной ряд с ненулевым радиусом сходимости
является рядом Тейлора своей суммы}.

\subsection{Ряды Тейлора}
\subsubsection{Предварительные сведения}

Рассмотрим обратную задачу: всегда ли функция $f$, порождающая ряд
Тейлора $\sumnzi \frac{f^{(n)}(x_0)}{n!} (x-x_0)^n$, является его
суммой? Оказывается, нет.

Во--первых, может случиться, что ряд Тейлора сходится только в своём
центре.

\begin{proof}
Французский математик Э. Борель показал, что существуют бесконечно
дифференцируемые функции с любыми произвольно заданными значениями
последовательных производных в фиксированной точке. Таким образом,
каждый степенной ряд, в частности и ряд с нулевым радиусом
сходимости, есть ряд Тейлора некоторой функции.
\end{proof}

Во--вторых, даже если ряд Тейлора функции $f$ сходится, он может
иметь другую сумму.

\begin{proof}
Рассмотрим функцию $f_0(x)$ \equ{f_0(x) =
\case{e^{-\frac1{(x-x_0)^2}}, \mbox{ если } x\ne x_0, \\ 0, \mbox{
если } x=x_0,}} и положим $t=x-x_0$. Функция $g(t)$,
\equ{g(t)=\case{e^{-\frac1{t^2}}, \mbox{ если } t\ne0, \\ 0, \mbox{
если } t=0,}} имеет \equ{g'(t) = \frac{2}{t^3} e^{-\frac1{t^2}}, \;
t\ne0, \; g''(t) = - \frac{2\cdot3}{t^4} e^{-\frac1{t^2}} +
\frac{2\cdot2}{t^6} e^{-\frac1{t^2}} = \hr{\frac4{t^6} -
\frac6{t^4}} e^{-\frac1{t^2}}, \; t\ne0, \; g^{(n)}(t) = P_{3n}
\hr{\frac1t} e^{-\frac1{t^2}}, \; t\ne0,} где $P_{3n}(y)$ ---
некоторый многочлен от $y$ степени $3n$. Так как $\liml{t\ra0}
g^{(n)}(t) = \liml{y\ra+\bes} P_{3n}(\pm\sqrt{y}) e^{-y} = 0, \;
n\in\N,$ то $0=g^{(n)}(0) = f^{(n)}_0 (x_0), \; n\in\N$. Поэтому,
$\sumnzi \frac{f^{(n)}_0(x_0)}{n!}(x-x_0)^n=0, \; x\in\R$.
\end{proof}

Последний пример показывает также, что \textbf{если $\sumnzi
c_n(x-x_0)^n$ --- ряд Тейлора функции $f$, то он же является рядом
Тейлора бесконечного множества других функций вида $f+cf_0, \;
c\in\R$ --- произвольная постоянная} (здесь $(f+cf_0)^{(n)}(x_0) =
f^{(n)}(x_0) + cf^{(n)}_0(x_0) = f^{(n)}(x_0), \;
n\in\N_0=\N\cup\hc{0})$.

\subsubsection{Сходимость ряда Тейлора к производящей функции}

Предположим, что функция $f$ бесконечно дифференцируема в интервале
$I$ и существует точка $x_0\in I$, что промежуток $J$ сходимости её
ряда Тейлора $\sumnzi \frac{f^{(n)}(x_0)}{n!}(x-x_0)^n$ накрывает
$I$ (название условия необходимы для решения задачи,
сформулированной в заголовке пункта, в силу доказанных в предыдущем
параграфе свойств степенных рядов). Частные суммы ряда $s_n(x) =
P_n(x;x_0,f), \; x\in J, \; n\in\N$ --- многочлены Тейлора функции
$f$ в точке $x_0$. Поэтому, сумма ряда $s(x)=\liml{n\ra+\bes} s_n(x)
= \liml{n\ra+\bes} P_n(x;x_0,f)$ для всех $x\in J$ и $s(x)=f(x)$ для
$x\in I\subset J$ тогда и только тогда, когда $\liml{n\ra+bes}
[f(x)-P_n(x;x_0,f)]=0$ для каждого $x\in I$. Поскольку в интервале
$I$ для функции $f$ справедлива формула Тейлора, то $f(x) -
P_n(x;x_0,f) = r_n(x;x_0,f), \; x\in I, \; n\in\N$, где
$r_n(x;x_0,f)$ --- $n$--ый остаточный член в формуле Тейлора функции
$f$ на интервале $I$. Таким образом, справедливо следующее
утверждение.

\begin{stm*}
Если функция $f$ бесконечно дифференцируема в интервале $I$ и точка
$x_0\in I$, то \equ{f(x)=\sumnzi \frac{f^{(n)}(x_0)}{n!}(x-x_0)^n,
\; x\in I,} тогда и только тогда, когда последовательность
$(r_n(x))$ остаточных членов $r_n(x) = r_n(x;x_0,f), \; n\in\N$, в
её формуле Тейлора на $I$ сходится к нулевой функции для всех $x\in
I$.
\end{stm*}

\begin{theorem}
(Достаточное условие разложимости в ряд Тейлора). Если функция $f$
бесконечно дифференцируема на отрезке между $x_0$ и $x$ (на $\R$) и
все её производные $f^{(n)}(t), \; n\in\N$, равномерно ограничены на
этом отрезке, то $f(x)=\sumnzi \frac{f^{(n)}(x_0)}{n!} (x-x_0)^n$,
то есть ряд Тейлора функции $f$ сходится в точке $x$ к сумме $f(x)$.
\end{theorem}

\begin{proof}
Рассмотрим остаточные члены $r_n(x;x_0,f)$ в форме Лагранжа.
\equ{r_n(x;x_0,f)=\frac{f^{(n+1)}(x_0 + \ta(x-x_0))}{(n+1)!}
(x-x_0)^{(n+1)}, \; 0<\ta<1.} По условию теоремы, существует число
$M>0$, что $\hm{f^{(n)}(t)}\le M$ для всех $t$ между $x_0$ и $x$ и
всех $n\in\N$. Поэтому, \eqa{1}{\hm{r_n(x;x_0,f)} \le
\frac{M}{(n+1)!} \hm{x-x_0}^{n+1}, \; x\ne x_0.}

Обозначим $u_n = \frac{M}{n!} \hm{x-x_0}^n, \; x\ne x_0, \; n\in\N$.
Тогда $\frac{u_{n+1}}{u_n} = \frac{\hm{x-x_0}}{n+1}, \; n\in\N$, и
$\liml{n\ra+\bes} \frac{u_{n+1}}{u_n}=0$. Согласно признаку
Даламбера, ряд $\sum u_n$ сходится, и следовательно,
$\liml{n\ra+\bes}u_n=0$. На основании (1), $\liml{n\ra+\bes}
r_n(x;x_0,f)=0$ и применяем предыдущее утверждение.
\end{proof}

\subsubsection{Ряды Тейлора элементарных функций}

\textbf{I.} Функция $f(x)=e^x$, $f^{(n)}(x)=e^x, \; x\in\R, \;
n\in\N$, и $f^{(n)}(0)=f(0)=1, \; n\in\N$. Для произвольного
фиксированного $x\in\R$ справедливо $\hm{f^{(n)}(t)}=e^t \le
e^{\hm{x}}$ для всех $t$ между $0$ и $x$ и всех $n\in\N$. Согласно
теореме предыдущего пункта, \eqa{2}{e^x=1+x+\frac{x^2}{2!} \spl
\frac{x^n}{n!} + \ldots = 1 + \sumnui \frac{x^n}{n!}, \; x\in\R.}
Для сумм $r_n(x)$ остатков ряда (2) справедлива оценка (1), в
которой $M=e^{\hm{x}}$ и $x_0=0$, так что $\hm{r_n(x)} \le
\frac{e^{\hm{x}}}{(n+1)!} \hm{x}^{n+1}, \; x\in\R, \; n\in\N$, и
последняя оценка указывает на скорость сходимости ряда (2) в  точках
$x\in\R$.

\textbf{II.} Функция $f(x)=\sin x, \; x\in\R$, имеет
$f^{(k)}(x)=\sin\hr{x+\frac{\pi}2k}, \; \hm{f^{(k)}(x)}\le1, \;
x\in\R, \; k\in\N$, и следовательно, \eqa{3}{\sin x = x -
\frac{x^3}{3!} \spl \frac{(-1)^{n-1}}{(2n-1)!} x^{2n-1} + \ldots =
\sumnui \frac{(-1)^{n-1}}{(2n-1)!} x^{2n-1}, \; x\in\R.}

Для сумм $r_n(x)$ остатков ряда (3) в силу (1) справедливы оценки
$\hm{r_{2n}(x)} \le \frac1{(2n)!} \hm{x}^{2n}, \; x\in\R, \;
n\in\N$.

\textbf{III.} Функция $f(x)=\cos x, \; x\in\R$, имеет $f^{(k)}(x) =
\cos\hr{x+k\frac{\pi}2}, \; \hm{f^{(k)}(x)}\le1$, $x\in\R, \;
k\in\N$, и следовательно, \eqa{4}{\cos x = 1 - \frac{x^2}{2!} \spl
\frac{(-1)^n}{(2n)!} x^{2n} + \ldots = \sumnzi \frac{(-1)^n}{(2n)!}
x^{2n}, \; x\in\R.} Для сумм $r_n(x)$ остатков ряда (4), на
основании (1), справедливы оценки $\hm{r_{2n+1}(x)} \le
\frac1{(2n+1)!} \hm{x}^{2n+1}, \; x\in\R$, $n\in\N$.

\textbf{IV.} Функция $f(x)=\ln(1+x), \; x>-1$, имеет
\equ{f'(x)=\frac1{1+x}=1-x+x^2-x^3 \spl (-1)^nx^n + \ldots, \;
\hm{x}<1,} и поэтому \equ{f(x)-f(0) = \ints{0}^x f'(t)\,dt =
x-\frac{x^2}2 + \frac{x^3}3 - \frac{x^4}4 \spl \frac{(-1)^n}{n+1}
x^{n+1} + \ldots = \sumnui \frac{(-1)^{n-1}}{n}x^n, \; x\in(-1,1),}
откуда \eqa{5}{\ln(1+x) = x-\frac{x^2}2 \spl \frac{(-1)^{n-1}}{n}
x^n + \ldots = \sumnui \frac{(-1)^{n-1}}{n}x^n, \; x\in(-1,1),}
поскольку $f(0)=\ln1=0$. Для $0<x<1$ суммы $r_n(x)$ остатков
знакочередующегося ряда (5) имеют оценки $\hm{r_n(x)} \le
\frac{\hm{x}^{n+1}}{n+1}, \; 0<x<1, \; n\in\N$. Для $-1<x<0$
равенства \equ{1-x+x^2-x^3 \spl (-1)^n x^n = \frac1{1+x} -
\frac{(-1)^(n+1)x^{n+1}}{1+x}, \; \hm{x}<1, \; n\in\N,} ведут к
оценкам \equ{\hm{r_{n+1}(x)} = \hm{\ints{0}^x \frac{(-1)^{n+1}
t^{n+1}}{1+t}\,dt} \le \frac1{1-\hm{x}} \hm{\ints0^x
\hm{t}^{n+1}\,dt} \le \frac{\hm{x}^{n+2}}{(n+2)(1-\hm{x})}, \;
\hm{x}, \; n\in\N,} откуда \equ{\hm{r_n(x)} \le
\frac1{(n+1)(1-\hm{x})} \hm{x}^{n+1}, \; -1<x<0, \; n\in\N.}

В $x=1$ имеем ряд Лейбница $\sumnui \frac{(-1)^{n-1}}{n}=\ln2$; в
$x=-1$ --- расходящийся ряд $-\sum\frac1n$.

\textbf{V. Биномиальный ряд}

Разложим в степенной ряд функцию $f(x)=(1+x)^{\al}, \; \al\in\R$.
Если $\al\in\N$, то \equ{(1+x)^{\al} = 1+\al x +
\frac{\al(\al-1)}{2!}x^2 \spl x^{\al} = \sum\limits_{k=0}^{\al}
C^k_{\al} x^k, \; x\in\R.} Если $\al=0$, то $f(x)=1, \; x\in\R$.
Пусть теперь $\al\in\R\backslash\N_0, \; \N_0=\N\cup\hc{0}$. Тогда
$f'(x) = \al(1+x)^{\al-1} \sd f^{(n)}(x) = \al(\al-1)\sd
(\al-n+1)(1+x)^{\al-n}$ и при $n>\al$ функция $f^{(n)}$ не
существует в точке $x=-1$. Отсюда следует, что радиус сходимости $R$
ряда для $f$ удовлетворяет условию $R\le1$. Покажем, что
$f(x)=(1+x)^{\al}, \; \al\in\R\backslash\N_0$, разлагается в
степенной ряд в интервале $(-1,1)$. Для этого оценим остаточный член
$r_n(x;0;f)=r_n(x)$ в формуле Тейлора функции $f$, записав его в
интегральной форме: \equ{r_n(x) = \frac1{n!} \ints0^x
f^{(n+1)}(t)(x-t)^n\,dt.} Имеем $0<\hm{x}<1$ и \eqa{6}{r_n(x) =
\frac{\al(\al-1)\sd(\al-n)}{n!}\ints0^x (1+t)^{\al-n-1} (x-t)^n \,
dt,} откуда \eqa{7}{\hm{r_{n+1}(x)} = \frac{\hm{\al(\al-1) \sd
(\al-n)(\al-n-1)}}{(n+1)!} \hm{\ints0^x \hm{x-t}^n (1+t)^{\al-n-1}
\cdot \frac{\hm{x-t}}{1+t}\,dt}.} Замечая, что $\frac{\hm{x-t}}{1+t}
= \frac{\hm{x}-\hm{t}}{1+t} \le \frac{\hm{x}-\hm{t}}{1-t} \le
\hm{x}$, на основании (6) и (7), получим \equ{\hm{r_{n+1}(x)} \le
\frac{\hm{n+1-\al}}{n+1} \hm{x} \hm{r_n(x)}} и
\equ{\liml{n\ra+\bes}{\frac{\hm{n+1-\al}}{n+1}}=1.} Следовательно,
при фиксированном $x$, $0<\hm{x}<1$, и достаточно малом $\ep>0$
существует $n_{\ep}\in\N$, что
\equ{\frac{\hm{r_{n+1}(x)}}{\hm{r_n(x)}} \le \hm{1-\frac{\al}{n+1}}
\hm{x} \le (1+\ep) \hm{x} = q < 1} для всех $n\ge n_{\ep}$.
Последнее означает, что $\hm{r_n(x)}\ra0$ при $n\ra+\bes$ не
медленнее, чем геометрическая прогрессия, и поэтому
\eqa{8}{(1+x)^{\al} = 1 + \sumnui
\frac{\al(\al-1)\sd(\al-n+1)}{n!}x^n, \; x\in(-1,1).} В точке $x=-1$
имеем ряд \eqa{9}{1+\sumnui (-1)^n
\frac{\al(\al-1)\sd(\al-n+1)}{n!};} в точке $x=1$ --- ряд
\eqa{10}{1+\sumnui \frac{\al(\al-1)\sd(\al-n+1)}{n!}.} Отметим, что
ряд (9) знакопостоянный, а ряд (10) --- знакочередующийся.

Так как \equ{\hm{(-1)^n \frac{\al(\al-1)\sd(\al-n+1)}{n!}} =
\hm{\frac{\al(\al-1)\sd(\al-n+1)}{n!}},} то ряды (9) и (10)
абсолютно сходятся для одних и тех же значений
$\al\in\R\backslash\N_0$. Обозначим $u_n = \frac{\hm{\al(\al-1)\sd
(\al-n+1)}}{n!}, \; n\in\N$. Тогда $\frac{u_{n+1}}{u_n} =
\frac{\hm{\al-n}}{n+1} = \frac{n-\al}{n+1}$ для всех $n>\al$.
Согласно локальной формуле Тейлора, \eqa{11}{\frac{u_{n+1}}{u_n} =
\hr{1-\frac{\al}n} \hr{1+\frac1n}^{-1} = \hr{1-\frac{\al}n}
\hr{1-\frac1n + \frac1{n^2} + o\hr{\frac1{n^2}}} = 1 - \frac{\al+1}n
+ \frac{\al+1}{n^2} + o\hr{\frac1{n^2}}, \; n\ra+\bes.} На основании
(11) и теоремы из пункта ?, параграфа ?, главы 1, заключаем, что
существует число $c>0$ и справедливо \eqa{12}{u_n = c\cdot
n^{-(1+\al)} (1+o(1)), \; n\ra+\bes.} Поэтому, ряд $\sum u_n$
сходится, если $1+\al>1, \; \al<0$, и расходится, если $1+\al\le1,
\; \al\le0$. Поскольку, по предположению, $\al\ne0$, окончательно
заключаем, что ряды (9) и (10) абсолютно сходятся при $\al>0$ и
знакопостоянный ряд (9) расходится при $\al<0$.

Так как $\frac{u_{n+1}}{u_n} = \frac{n-\al}{n+1}, \; n>\al$, то
$\frac{u_{n+1}}{u_n}\le1$, когда $\frac{n-\al}{n+1}\le1$ и $n>\al$;
то есть, когда $\al\ge-1$ и $n>\al$. Следовательно,
последовательность $(u_n)$ убывает для всех $n\in\N$ при $\al\ge-1$.
На основании (12) заключаем, что $\liml{n\ra+\bes}u_n=0$, если
$1+\al>0$, $\al>-1$. Поэтому, в силу признака Бернулли--Лейбница,
знакочередующийся ряд (10) сходится при $\al>-1$. Если $\al\le-1$,
то есть, $1+\al\le0$, то из (12) следует, что последовательность
$(u_n)$ не бесконечно малая, и следовательно, ряд (10) расходится.

\textbf{Окончательный вывод:}

$x=-1$; ряд абсолютно сходится при $\al>0$ и расходится при $\al<0$;

$x=1$; ряд абсолютно сходится при $\al>0$, сходится условно при
$-1<\al<0$ и расходится при $\al\le-1$.

\subsubsection{Экспоненциальная функция комплексного переменного}

Степенной ряд $1+z + \frac{z^2}{2!} \spl \frac{z^n}{n!} + \ldots =
\sumnzi \frac{z^n}{n!}$ сходится для всех $z\in\Cbb$, поскольку для
его общего члена $a_n(z) = \frac{z^n}{n!}$ при $z\ne0$ имеем
\equ{\liml{n\ra+\bes} \hm{\frac{a_{n+1}(z)}{a_n(z)}} =
\liml{n\ra+\bes} \frac{\hm{z}}{n+1}=0<1} и применяем признак
Даламбера. Обозначим $\sumnzi \frac{z^n}{n!} = E(z), \; z\in\Cbb$.

\begin{theorem}
$E(z)\cdot E(w) = E(z+w), \; z,w\in\Cbb$.
\end{theorem}

\begin{proof}
Ряды $\sumnzi \frac{z^n}{n!} = E(z), \; \sumnzi \frac{w^n}{n!} =
E(w)$ абсолютно сходятся на $\Cbb$, и, следовательно, по теореме
Коши--Абеля, абсолютно сходятся на $\Cbb$ все их произведения, имея
одинаковые суммы $E(z)\cdot E(w)$. Их произведение по Коши имеет
сумму \equ{\sumnzi \hr{\sumkzn \frac{z^k}{k!} \cdot
\frac{w^{n-k}}{(n-k)!}} = \sumnzi \hr{\frac1{n!} \sumkzn
\frac{n!}{k!(n-k)!} z^k w^{n-k}} = \sumnzi \hr{\frac1{n!} \sumkzn
C^k_n z^k w^{n-k}} = \sumnzi \frac{(z+w)^n}{n!} = E(z+w),} так что
$E(z+w)=E(z)\cdot E(w), \; z,w\in\Cbb$.
\end{proof}

Таким образом, функция $E(z), z\in\Cbb$, удовлетворяет такому же
функциональному уравнению, что и экспоненциальная функция $e^x, \;
x\in\R$. Кроме того, если $z=x$, то $E(x) = \sumnzi \frac{x^n}{n!} =
e^x, \; x\in\R$. Поэтому, по определению, обозначают $E(z) = e^z, \;
z\in\Cbb$, и \eqa{13}{e^z = 1+z+\frac{z^2}{2!} \spl \frac{z^n}{n!} +
\ldots = \sumnzi \frac{z^n}{n!}, \; z\in\Cbb.}

Если в (13) положить $z=ix, \; i^2=-1, \; x\in\R$, и заметить, что
$i^3=-i, \; i^4=1$, то \equ{e^{ix} = \sumnzi \frac{(ix)^n}{n!} =
\sumkzi \frac{(-1)^k}{(2k)!} x^{2k} + i \sumkui
\frac{(-1)^{k-1}}{(2k-1)!} x^{2k-1} = \cos x + i \sin x, \; x\in\R.}
Таким образом, доказана формула Эйлера \equ{e^{ix} = \cos x + i\sin
x, \; x\in\R.}

Она справедлива и для комплексных чисел $z\in\Cbb$ в виде $e^{iz} =
\cos z + i\sin z, \; z\in\Cbb$.

Прямым следствием теоремы и формулы Эйлера будет формула
Муавра--Эйлера: $(\cos x + i\sin x)^n = \cos{nx} + i\sin{nx}, \;
x\in\R, \; n\in\N.$

\begin{proof}
$(e^{ix})^n = e^{inx}, \; x\in\R, \; n\in\N$, и применяем формулу
Эйлера.
\end{proof}

\subsection{Разложение синуса в бесконечное произведение. Формула
дополнения для гамма-функций}

Все замечательные результаты этого параграфа принадлежат Л. Эйлеру.

\subsubsection{Предварительные замечания}

По формуле Муавра--Эйлера, $(\cos z + i\sin z)^m = \cos{mz} +
i\sin{mz}, \; z\in\R, \; m\in\N$, откуда \equ{\sin{mz} =
m\cos^{m-1}z \sin z - \frac{m(m-1)(m-2)}{1\cdot2\cdot3} \cos^{m-3}z
\cdot \sin^3 z + \ldots.} Если $m=2n+1, \; n\in\N$, то, заменяя
чётные степени $\cos^{2k}z = (1-\sin^2 z)^k$, получим
\eqa{1}{\sin(2n+1)z = \sin z \cdot P(\sin^2 x),} где $P(u)$ ---
алгебраический многочлена степени $n$ от переменного $u$.

Если $z$ обращает в ноль $\sin(2n+1)z$ и $\sin z\ne0$, то на
основании (1) заключаем, что $\sin^2 z$ --- корень $P(u)$. Числа
$\frac{\pi}{2n+1}, \; 2 \frac{\pi}{2n+1}, \ldots, n
\frac{\pi}{2n+1}$ все лежат в интервале $\hr{0,\frac{\pi}2}$ и
обращают в ноль левую часть в (1). Поэтому $u_1 = \sin^2
\frac{\pi}{2n+1}, \; u_2 = sin^2 2\frac{\pi}{2n+1}, \ldots, u_n =
\sin^2 n \frac{\pi}{2n+1},$ --- все корни $P(u)$ и других корней
многочлен $P(u)$ степени $n$ не имеет. Следовательно, \equ{P(u) =
a(u-u_1)\sd (u-u_n) = A\hr{1-\frac{u}{u_1}}\sd
\hr{1-\frac{u}{u_n}},} так как все $u_i\ne0$, и, с учётом (1),
\equ{A = \liml{z\ra0} \frac{\sin(2n+1)z}{\sin z} = 2n+1.} Таким
образом, формула (1) принимает вид \equ{\sin(2n+1)z = (2n+1)  \sin z
\hr{1 - \frac{\sin^2z}{\sin^2 \frac{\pi}{2n+1}}} \sd \hr{1 -
\frac{\sin^2z}{\sin^2 \frac{n\pi}{2n+1}}},} откуда при $z =
\frac{x}{2n+1}$ получим \eqa{2}{\sin x = (2n+1)\sin \frac{x}{2n+1}
\prods{j=1}^n \hr{1 - \frac{sin^2 \frac{x}{2n+1}}{\sin^2
\frac{j\pi}{2n+1}}}, \; n\in\N, \; x\in\R.}

\subsubsection{}

Фиксируем $x\ne l\pi, \; l\in\Z$, так что $\sin x\ne0$. Выберем
$k\in\N$ с условием $(k+1)\pi > \hm{x}$ и на основании (2) для
любого $n\in\N, \; n>k$, получим \eqa{3}{\sin x = \Uc_k^{(n)} \cdot
\Vc_k^{(n)},} где \equ{\Uc_k^{(n)} = (2n+1) \sin \frac{x}{2n+1}
\prods{j=1}^k \hr{1 - \frac{\sin^2 \frac{x}{2n+1}}{\sin^2
\frac{j\pi}{2n+1}}},} \equ{\Vc_k^{(n)} = \prods{j=k+1}^{n} \hr{1 -
\frac{\sin^2 \frac{x}{2n+1}}{\sin^2 \frac{j\pi}{2n+1}}}.}

Так как $\liml{n\ra+\bes} (2n+1) \sin\frac{x}{2n+1} = x$ и
$\liml{n\ra+\bes} \frac{\sin^2 \frac{x}{2n+1}}{\sin^2
\frac{j\pi}{2n+1}} = \frac{x^2}{j^2\pi^2}, \; j=\ol{1,k}$, то
существует \eqa{4}{\Uc_k = \liml{n\ra+\bes} \Uc_k^{(n)} = x
\prods{j=1}^k \hr{1 - \frac{x^2}{j^2 \pi^2}}, \; (k+1)\pi > \hm{x}.}

На основании (3) и (4) заключаем, что существует $\liml{n\ra+\bes}
\Vc_k^{(n)} = \Vc_k$, $(k+1)\pi > \hm{x}$.

\subsubsection{Оценка $\Vc_k^{(n)}$.}

Для $0<\ph<\frac{\pi}2$ справедливо $\frac{2}{\pi}\ph < \sin\ph <
\ph$. Поэтому, $\sin^2 \frac{x}{2n+1} < \frac{x^2}{(2n+1)^2}$ и
\equ{\sin^2 \frac{j\pi}{2n+1} > \hr{\frac{2j}{2n+1}}^2 =
\frac{4j^2}{(2n+1)^2}, \; j=\ol{k+1,n},} так что \eqa{5}{1 >
\Vc_k^{(n)} > \prods{j=k+1}^{n} \hr{1 - \frac{x^2}{4j^2}}.}
Бесконечное произведение $\prods{j=j_0}^{\bes} \hr{1 -
\frac{x^2}{4j^2}}$ абсолютно сходится для всех $x$, $x^2 < 4j_0^2$,
так как абсолютно сходится ряд $\sum\limits_{j=1}^{\bes}
\frac{x^2}{4j^2}$. Поэтому для его остаточных произведений
$\prods{j=k+1}^{\bes} \hr{1 - \frac{x^2}{4j^2}}$ справедливо
$\liml{k\ra+\bes} \prods{j=k+1}^{\bes} \hr{1 - \frac{x^2}{4j^2}} =
1$. Следовательно, на основании (5), имеем \equ{1\ge
\liml{n\ra+\bes} \Vc_k^{(n)} = \Vc_k \ge \liml{n\ra+\bes}
\prods{j=k+1}^n \hr{1 - \frac{x^2}{4j^2}} = \prods{j=k+1}^{\bes}
\hr{1 - \frac{x^2}{4j^2}}} и \equ{1 \ge \liml{k\ra+\bes} \Vc_k \ge
\liml{k\ra+\bes} \prods{j=k+1}^{\bes} \hr{1 - \frac{x^2}{4j^2}} =
1,} так что \eqa{6}{\liml{k\ra+\bes} \Vc_k = 1.}

На основании (3), (6) и (4) заключаем, что $\sin x =
\liml{k\ra+\bes} \Uc_k$ и \eqa{7}{\sin x = x \prods{n=1}^{\bes}
\hr{1 - \frac{x^2}{n^2\pi^2}}, \; x\ne n\pi, \; n\in\Z.} Но (7)
справедлива и при $x=n\pi, \; n\in\Z$, так как $\sin n\pi=0$.

\subsubsection{}

Итак, доказана формула Эйлера \eqa{8}{\sin x = x\prods{n=1}^{\bes}
\hr{1 - \frac{x^2}{n^2\pi^2}}, \; x\in\R.}

\subsubsection{Формула дополнения для гамма-функции}

Заменяя в (8) аргумент $x$ на $\pi x$, получим \eqa{9}{\sin\pi x =
\pi x \prods{n=1}^{\bes} \hr{1 - \frac{x^2}{n^2}}, \; x\in\R.} Так
как $\Ga(x) = \frac1x \prods{n=1}^{\bes}
\frac{\hr{1+\frac1n}^x}{1+\frac{x}n}, \; \Ga(x+1) = x\Ga(x), \;
x\in\R\backslash \N_0$, то \equ{\Ga(1-x) = -x\Ga(-x) = (-x)
\frac1{(-x)} \prods{n=1}^{\bes}
\frac{\hr{1+\frac1n}^{-x}}{1-\frac{x}n} = \prods{n=1}^{\bes}
\frac{\hr{1 + \frac1n}^{-x}}{1 - \frac{x}n}} для всех
$x\in\R\backslash (-\N_0)$. Поэтому, с учётом (9), для всех $x\in
\R\backslash \hs{\N_0 \cup (-\N_0)} = \R\backslash\Z$ справедливо
\equ{\Ga(x) \Ga(1-x) = \frac1x \prods{n=1}^{\bes} \frac1{\hr{1 -
\frac{x^2}{n^2}}} = \frac{pi}{\sin\pi x}, \; x\notin \Z.}

Формула $\Ga(x)\Ga(1-x) = \frac{\pi}{\sin\pi x}, \; x\notin\Z$,
носит название формулы дополнения для гамма-функции.

\section{Несобственные интегралы}

\subsection{Признаки сходимости несобственных интегралов}

\subsubsection{Некоторые воспоминания о втором семестре}

Пусть функция $f$ определена на промежутке $[a,b), \; -\bes < a < b
\le + \bes$, и $f\in\Rc[a,t]$ для любого $t, \; a<t<b$. Если функция
$F(t) = \ints{a}^t f(x)\,dx, \; F(a)=0$, непрерывная на $[a,b)$,
имеет $\liml{t\ra b-} F(t) = l_1$, где $t\ra b-$ обозначает либо
базу $t\ra b-0$ (если $b\in\R$), либо базу $t\ra +\bes$ (если
$b=+\bes$), то число $l_1$ называют несобственным интегралом функции
$f$ по промежутку $[a,b)$ и обозначают $l_1 = \ints{a}^b f(x)\,dx$.
Итак, \eqa{1} {\ints{a}^b f(x)\,dx = \liml{t\ra b-0} \ints{a}^t
f(x)\,dx = \liml{t\ra b-0} F(t),} если $b\in\R$, и
\eqa{1'}{\liml{t\ra +\bes} F(t) = \liml{t\ra+\bes} \ints{a}^t
f(x)\,dx = \ints{a}^{+\bes} f(x)\,dx,} если $b=+\bes$.

Пусть теперь функция $f$ определена на промежутке $(a,b]$, $-\bes
\le a < b < +\bes$, и $f\in\Rc[t,b]$ для любого $t$, $a<t<b$. Если
функция $\Ph(t) = \ints{t}^b f(x)\,dx, \; \Ph(b)=0,$ непрерывная на
$(a,b]$, имеет $\liml{t\ra a+} \Ph(t) = l_2$, где $t\ra a+$
обозначает либо базу $t\ra a+0$ (если $a\in\R$), либо базу $t\ra
-\bes$ (если $a = -\bes$), то число $l_2$ называют несобственным
интегралом функции $f$ по промежутке $(a,b]$ и обозначают $l_2 =
\ints{a}^b f(x)\,dx$. Итак, \eqa{2}{\ints{a}^b f(x)\,dx = \liml{t\ra
a+0} \ints{t}^b f(x)\,dx = \liml{t\ra a+0} \Ph(t),} если $a\in\R$, и
\eqa{2'}{\liml{t\ra-\bes} \Ph(t) = \liml{t\ra-\bes} \ints{t}^b
f(x)\,dx = \ints{-\bes}^b f(x)\,dx,} если $a=-\bes$.

В случае существования пределов (1), (1'), (2), (2'),
соответствующие несобственные интегралы называют сходящимися; в
случае несуществования пределов --- расходящимися. Сходящиеся
несобственные интегралы обладают свойством аддитивности; то есть,
существование предела $\liml{t\ra b-} F(t) = \ints{a}^b f(x)\,dx$
$\hs{\liml{t\ra a+} \Ph(t) = \ints{a}^b f(x)\,dx}$ равносильно
сходимости несобственных интегралов $\ints{t}^b f(x)\,dx =
\liml{t'\ra b-} \ints{t}^{t'} f(x)\,dx = r(t)$ $\hs{\ints{a}^t
f(x)\,dx = \liml{t''\ra a+} \ints{t''}^t f(x)\,dx = r(t)}$ для всех
$t, a<t<b$, и справедливости равенств \eqa{3}{\ints{a}^b f(x)\,dx =
\ints{a}^t f(x)\,dx + \ints{t}^b f(x)\,dx, \; a<t<b.}

Несобственные интегралы $r(t)$, $a<t<b$, называются остатками
несобственного интеграла $\ints{a}^b f(x)\,dx$. На основании (1),
(1'), (2), (2') и (3) заключаем, что несобственный интеграл
$\ints{a}^b f(x)\,dx$ сходится тогда и только тогда, когда
$\liml{t\ra b-} r(t) = \liml{t\ra b-} \ints{t}^b f(x)\,dx =0$
$\hs{\liml{t\ra a+} r(t) = \liml{t\ra a+} \ints{a}^t f(x)\,dx=0}$.

\subsubsection{Признаки сходимости несобственных интегралов}

\begin{theorem}
(Признак Абеля). Если функции $f$ и $g$ определены на промежутке
$[a,b)$ (на $(a,b]$) и обладают свойствами: \begin{nums}{-2} \item
сходится несобственный интеграл $\ints{a}^b f(x)\,dx$; \item функция
$g$ монотонна и ограничена на $[a,b)$ (на $(a,b]$),\end{nums} то
сходится несобственный интеграл $\ints{a}^b f(x)g(x)\,dx$.
\end{theorem}

\begin{proof}
Согласно условию 2, существует $L>0$, что $\hm{g(x)}\le L, \;
x\in[a,b)$ ($x\in(a,b]$). Проверим выполнение критерия Коши
сходимости несобственного интеграла. Для этого рассмотрим
произвольное число $\ep>0$ и произвольные точки $t_i, \; a<t_i<b, \;
i=1,2$. Так как $f,g\in\Rc[t_1,t_2]$ и $g$ монотонна на $[t_1,t_2]$,
то согласно второй теореме о среднем значении для интеграла Римана,
\eqa{4}{\ints{t_1}^{t_2} f(x) g(x)\,dx = g(t_1) \ints{t_1}^{\xi}
f(x)\,dx + g(t_2)\ints{\xi}^b f(x)\,dx,} где $\xi$ лежит между $t_1$
и $t_2$. На основании (4) справедливы оценки
\eqa{5}{\hm{\ints{t_1}^{t_2} f(x)g(x)\,dx} \le \hm{g(t_1)}
\hm{\ints{t_1}^{\xi} f(x)\,dx} + \hm{g(t_2)} \hm{\ints{\xi}^b
f(x)\,dx} \le L \hr{\hm{\ints{t_1}^{\xi} f(x)\,dx} +
\hm{\ints{\xi}^{t_2} f(x)\,dx}}, \; t_i\in(a,b), \; i=1,2.}

Согласно условию 1, по критерию Коши сходимости несобственный, для
$\ep>0$ существует такое $b_{\ep}$, $a<b_{\ep}<b$ (существует такое
$a_{\ep}$, $a<a_{\ep}<b$), что для всех $t',t'', b_{\ep}<t', t''<b$
(всех $t',t'', \; a<t', \; t''<a_{\ep}$) справедлива оценка
$\hm{\ints{t'}^{t''} f(x)\,dx} < \frac{\ep}{2L}$. Выбирая $t_i, \;
b_{\ep} < t_i < b$ $\hs{a<t_i<b}, \; i=1,2$, и отмечая, что $\xi$
расположена между $t_1$ и $t_2$, получим оценки
\eqa{6}{\hm{\ints{t_1}^{\xi} f(x)\,dx} < \frac{\ep}{2L} \mbox{ и }
\hm{\ints{\xi}^{t_2} f(x)\,dx}<\frac{\ep}{2L}} для всех выбранных
$t_i, \; i=1,2$.

На основании (5) и (6), имеем оценку \equ{\hm{\ints{t_1}^{t_2}
f(x)g(x)\,dx} < L\hr{\frac{\ep}{2L} + \frac{\ep}{2L}}=\ep,}
справедливую для всех $t_i, \; b_{\ep}<t_i<b$ [всех $t_i, \;
a<t_i<a_{\ep}$], $i=1,2$, что является критерием Коши сходимости
несобственного интеграла $\ints{a}^b f(x)g(x)\,dx$.
\end{proof}

\begin{theorem}
(Признак Дирихле). Если функции $f$ и $g$ определены на промежутке
$[a,b), \; -\bes<a<b<+\bes$, и обладают свойствами:
\begin{nums}{-2}\item функция $F(t) = \ints{a}^t f(x)\,dx$
ограничена на $[a,b)$; \item функция $g$ монотонна на $[a,b)$ и
имеет $\liml{x\ra b-} g(x)=0$,\end{nums} то несобственный интеграл
$\ints{a}^b f(x)g(x)\,dx$ сходится.
\end{theorem}

\begin{proof}
Из условию 2 теоремы следует, что $g\in\Rc[a,t]$ для любого $t,
a<t<b$. Проверим выполнение критерия Коши. Для произвольных $t_i$,
$a<t_i<b$, $i=1,2$, справедливо (4). Рассмотрим произвольное
$\ep>0$. Согласно условию 1, существует такое $C>0$, что
$\hm{F(t)}\le C$ для всех $t\in[a,b)$. Поэтому, с учётом свойства
аддитивности определённых интегралов, \equ{\hm{\ints{t_1}^{\xi}
f(x)\,dx} = \hm{\ints{a}^{\xi} f(x)\,dx - \ints{a}^{t_1} f(x)\,dx} =
\hm{F(\xi) - F(t_1)} \le 2 C} и \equ{\hm{\ints{\xi}^{t_2} f(x)\,dx}
= \hm{\ints{a}^{t_2} f(x)\,dx - \ints{a}^{\xi} f(x)\,dx} =
\hm{F(t_2) - F(\xi)} \le 2C} и равенство (4) ведёт к оценке
\eqa{7}{\hm{\ints{t_1}^{t_2} f(x)g(x)\,dx} \le \hm{g(t_1)}
\hm{\ints{t_1}^{\xi} f(x)\,dx} + \hm{g(t_2)} \hm{\ints{\xi}^{t_2}
f(x)\,dx} \le 2C (\hm{g(t_1)} + \hm{g(t_2)}),} справедливой для
любых $t_i, \; a<t_i<b, \; i=1,2$. Так как $\liml{x\ra b-} g(x)=0$,
то для любого $\ep>0$ существует такое $b_{\ep}$, $a<b_{\ep}<b$, что
$\hm{g(x)} < \frac{\ep}{4C}$ для всех $x,b_{\ep}<x<b$. Поэтому, на
основании (7), имеем оценку \equ{\hm{\ints{t_1}^{t_2} f(x)g(x)\,dx}
< 2C \hr{\frac{\ep}{4C} + \frac{\ep}{4C}} = \ep,} справедливую для
всех $t_i, \; b_{\ep}<t_i<b, \; i=1,2$, что есть критерий Коши
сходимости несобственного интеграла $\ints{a}^b f(x)g(x)\,dx.$
\end{proof}

\subsubsection{Интегральные синус и косинус}

Согласно признаку Дирихле, несобственные интегралы si$x =
-\ints{x}^{+\bes} \frac{\sin t}{t}\,dt$ и ci$x = -\ints{x}^{+\bes}
\frac{\cos t}{t}\,dt$ сходятся для всех $x>0$; кроме того,
существует si$0 = -\ints{0}^{+\bes} \frac{\sin t}{t}\,dt$. Так как
si$x = \ints1^x \frac{\sin t}t \,dt - \ints1^{+\bes} \frac{\sin
t}{t}\,dt$, ci$x = \ints1^x \frac{\cos t}{t}\,dt - \ints1^{+\bes}
\frac{\cos t}t\,dt, \; x>0$, то (si)$'x=\frac{\sin x}x$,
(ci)$'x=\frac{\cos x}x, \; x>0$.

\subsubsection{Интегралы Фруллани}
\begin{theorem}
Если функция $f\in\Cc[0,+\bes)$ и существует $\\liml{x\ra+\bes} f(x)
= f(+\bes)$, то для любых положительных чисел $a,b>0$ справедлива
формула \equ{\ints0^{+\bes} \frac{f(ax)-f(bx)}x\,dx = [f(0) -
f(+\bes)] \ln\frac{b}a.}
\end{theorem}

\begin{proof}
Рассмотрим произвольные $0<\de<\De<+\bes$. Так как
$f\in\Rc[\de,\De]$, то \mla{8}{\ints{\de}^{\De}
\frac{f(ax)-f(bx)}x\,dx=\ints{\de}^{\De} \frac{f(ax)}x\,dx -
\ints{\de}^{\De} \frac{f(bx)}x\,dx = \ints{a\de} \frac{f(t)}t\,dt -
\ints{b\de}^{b\De} \frac{f(t)}t\,dt =\\= \ints{a\de}^{b\de}
\frac{f(t)}t\,dt + \ints{b\de}^{a\De} \frac{f(t)}t\,dt -
\ints{b\de}^{a\De} \frac{f(t)}t\,dt - \ints{a\De}^{b\De}
\frac{f(t)}t\,dt = \ints{a\de}^{b\de} \frac{f(t)}t\,dt -
\ints{a\De}^{b\De}\frac{f(t)}t\,dt.}

Согласно первой теореме о среднем значении для интегралов,
\eqa{9}{\ints{a\de}^{b\de} \frac{f(t)}t\,dt = f(\xi)
\ints{a\de}^{b\de} \frac{dt}t = f(\xi)\ln\frac{b}a,} где $\xi =
a\de+\ta_1(b-a)\de, \; 0<\ta_1<1$, и \eqa{10}{\ints{a\De}^{b\De}
\frac{f(t)}t\,dt = f(\eta) \ints{a\De}^{b\De} \frac{dt}t = f(\eta)
\ln\frac{b}a,} где $\eta = a\De+\ta_2(b-a)\De, \; 0<\ta_2<1$.

Подставляя (9) и (10) в (8), получим \eqa{11}{\ints{\de}^{\De}
\frac{f(ax)-f(bx)}x\,dx = [f(\xi)-f(\eta)]\ln\frac{b}a.}

Так как $\liml{\de\ra+0} f(\xi) = \liml{x\ra+0} f(x) = f(0)$ (в силу
непрерывности функции $f$ в точке $x=0$) и $\liml{\De\ra+\bes}
f(\eta) = \liml{x\ra+\bes} f(x) = f(+\bes)$, то существует предел
при $\de\ra+0$ и $\De\ra+\bes$ правой части формулы (11), и
следовательно, её левой части; то есть, сходится несобственный
интеграл \equ{\ints0^{+\bes} \frac{f(ax)-f(bx)}x\,dx =
\liml{\de\ra0,\De\ra+\bes} \ints{\de}^{\De} \frac{f(ax)-f(bx)}x\,dx
= [f(0)-f(+\bes)]\ln\frac{b}a.}
\end{proof}

\begin{theorem}
Если функция $f\in\Cc[0,+\bes)$ и сходится $\ints{c}^{+\bes}
\frac{f(x)}x\,dx$ для некоторого $c>0$, то для любых положительных
чисел $a,b>0$ \eqa{12}{\ints0^{+\bes} \frac{f(ax)-f(bx)}x\,dx =
f(0)\ln\frac{b}a.}
\end{theorem}

\begin{proof}
Как и выше, справедливы формулы (8) и (9). Согласно критерию Коши
сходимости несобственного интеграла $\ints{c}^{+\bes}
\frac{f(t)}t\,dx$, имеем \equ{\liml{\De\ra+\bes} \ints{a\De}^{b\De}
\frac{f(x)}x\,dx=0,} а в силу непрерывности функции $f$ в точке
$x=0$ имеем $\liml{\de\ra+0} f(\xi) = \liml{x\ra+0} f(x) = f(0)$.
Поэтому, переходя в (8) к пределам по базам $\de\ra+0$ и
$\De\ra+\bes$, с учётом (9), получим формулу (12).
\end{proof}

\begin{theorem}
Если функция $f$ непрерывна в интервале $(0,+\bes)$, сходится
несобственный интеграл $\ints0^c \frac{f(x)}x\,dx$ для некоторого
$c>0$ и существует $\liml{x\ra+\bes}f(x) = f(+\bes)$, то для любых
чисел $a,b>0$ справедлива формула \equ{\ints0^{+\bes}
\frac{f(ax)-f(bx)}x\,dx = f(+\bes)\ln\frac{a}b.}
\end{theorem}

\begin{proof}
Как и выше, справедливы формулы (8) и (10). Согласно критерию Коши
сходимости несобственного интеграла $\ints0^c \frac{f(x)}x\,dx$,
имеем \equ{\liml{\de\ra+0} \ints{a\de}^{b\de} \frac{f(t)}t\,dt=0,} а
согласно условия существования предела $f(+\bes)$ и формуле (10),
имеем $f(+\bes)=\liml{\De\ra+\bes} f(\eta)$ и
\equ{\liml{\De\ra+\bes} \ints{a\De}^{b\De} \frac{f(t)}t\,dt =
f(+\bes)\ln\frac{b}a.}

Поэтому, переходя в (8) к пределам по базам $\de\ra+0$ и
$\De\ra+\bes$, получим \equ{\ints0^{+\bes} \frac{f(ax)-f(bx)}x\,dx =
-f(+\bes)\ln\frac{b}a = f(+\bes)\ln\frac{a}b.}
\end{proof}

\subsubsection{Аналогия с рядами}

\begin{theorem}
  Если функция $f$ определена на $[a,b)$, $-\bes<a<b\le+\bes$,
  функция $f\in\Rc[a,t]$ для всех $t, \; a<t<b$, и $t\ra b-$
  обозначает базу $t\ra b-0$, если $b\in\R$, и базу $t\ra+\bes$,
  если $b=+\bes$, то несобственный интеграл $\ints{a}^b f(x)\,dx$
  сходится тогда и только тогда, когда для любой последовательности
  $(t_n)$, $a=t_0, \; a<t_n<b, \; n\in\N$, и $\liml{n\ra+\bes}
  t_n=n$, сходится ряд $\sum a_n$, в котором $a_n =
  \ints{t_{n-1}}^{t_n} f(x)\,dx, \; n\in\N$.
\end{theorem}

\begin{proof}
Рассмотрим произвольную $(t_n)$, $a=t_0$, $a<t_n<b$, $n\in\N$,
$\liml{n\ra+\bes} t_n=b$ и $a_n = \ints{t_{n-1}}^{t_n} f(x)\,dx$,
$n\in\N$. Частные суммы $s_n, \; n\in\N$, ряда $\sum a_n$ в силу
свойства аддитивности определённого интеграла, вычисляются по
формулам \equ{s_n = \sumkun a_k = \sumkun \ints{t_{k-1}}^{t_k}
f(x)\,dx = \ints{t_0}^{t_1} \spl \ints{t_{n-1}}^{t_n} f(x)\,dx =
\ints{t_0}^{t_n} f(x)\,dx = \ints{a}^{t_n} f(x)\,dx, \; n\in\N.} Так
как на $[a,b)$ определена непрерывная функция $F(t) = \ints{a}^t
f(x)\,dx$, то $s_n = F(t_n), \; n\in\N$.

$\Ra$ По условию, сходится несобственный интеграл $\ints{a}^b
f(x)\,dx$; то есть, существует $\liml{t\ra b-} F(t) = l$. Если
$b\in\R$, то, по определению Гейне, $l=\liml{n\ra+\bes} F(t_n) =
\liml{n\ra+\bes} s_n$, так что $l = \sumnui a_n$. Пусть теперь
$b=+\bes$ и $l=\liml{t\ra+\bes} F(t)$. Рассмотрим произвольное число
$\ep>0$ и произвольную $(t_n)$, $a\le t_n < +\bes$,
$\liml{n\ra+\bes} t_n = +\bes$. Тогда, по определению, существует
такое $t_{\ep}$, $a<t_{\ep}<+\bes$, что $\hm{F(t)-l}<\ep$ для всех
$t$, $t_{\ep}<t<+\bes$. Так как $\liml{n\ra+\bes} t_n = +\bes$, то
существует такое $N\in\N$, $N=N_{\ep}$, что $t_{\ep}<t_n < +\bes$
для всех $n\ge N$, и следовательно, $\hm{F(t_n)-l}<\ep$ для всех
$n\ge N$; то есть, $l = \liml{n\ra+\bes} F(t_n)$, так что и в этом
случае $l = \liml{n\ra+\bes} F(t_n) = \liml{n\ra+\bes} s_n$ и ряд
$\sum a_n$ сходится.

$\La$ Пусть сходятся все ряды $\sum a_n$, $a_n =
\ints{t_{n-1}}^{t_n} f(x)\,dx$, для всех $(t_n)$, $a=t_0$,
$a<t_n<b$, $\liml{n\ra+\bes} t_n=b$, и предположим, что расходится
несобственный интеграл $\ints{a}^b f(x)\,dx$. Тогда, согласно
критерию Коши, существует некоторое $\ep_0>0$ и для любого $b'$,
$a<b'<b$, существуют $t_{b'}^1$, $t_{b'}^2$,
$b'<t_{b'}^1<t_{b'}^2<b$, что \equ{\hm{\ints{t_{b'}^1}^{t_{b'}^2}
f(x)\,dx} \ge \ep_0;} то есть, \equ{\ep_0 \le
\hm{\ints{t_{b'}^1}^{t_{b'}^2} f(x)\,dx} = \hm{F(t_{b'}^2) -
F(t_{b'}^1)}.}

Рассмотрим $b'_n = b-\frac1n$, $n\ge N = \hs{\frac1{b-a}}+1$, если
$b\in\R$, и $b'_n=n$, $n\ge N = [\hm{a}]+1$, если $b=+\bes$ (в обоих
случаях условие $n\ge N$ обеспечивает свойство $a<b'_n<b$). Тогда
для каждого $n\in\N$, $n\ge N$, существуют точки $t^i_n, \; i=1,2$,
$b'_n<t_n^i<b$, в которых $\hm{F(t_n^2) - F(t_n^1)}\ge\ep_0$, $n\ge
N$. Числа $t^1_N$, $t^2_N$, $t^1_{N+1}$, $t^2_{N+1}$, \ldots
перенумеруем в одну последовательность $(t_m)$, положив $t_0=a, \;
t_1=t^1_N$, $t_2=t^2_N$, \ldots. Тогда $\liml{m\ra+\bes} t_m=b$, так
как $\liml{n\ra+\bes} t^i_n=b$ для каждого $i=1,2$, и по условию
сходится ряд $\sum a_m$, где $a_m = \ints{t_{m-1}}^{t_m} f(x)\,dx$,
$m\in\N$. С другой стороны, для любого $n\in\N, \; n\ge N$,
существует единственное $k\in\N$, что $t_n^1 = t_{2k-1}$ и $t^2_n =
t_{2k}$; и обратно, для любого $k\in\N$ существует единственной
$n\in\N$, что $t_{2k-1}=t^1_n$ и $t_{2k}=t^2_n$. Поэтому, \equ{\ep_0
\le \hm{F(t_n^2) - F(t_n^1)} = \hm{F(t_{2k}) - F(t_{2k-1})} =
\hm{s_{2k} - s_{2k-1}} = \hm{a_{2k}}, \; k\in\N,} так что
подпоследовательность $(a_{2k})$ последовательности $(a_m)$ не
является бесконечно малой, и вместе с ней вся последовательность
$(a_m)$ не бесконечно малая и ряд $\sum a_m$ обязан расходиться.
Полученное противоречие завершает доказательство теоремы.
\end{proof}

Аналогичное утверждение справедливо для несобственных интегралов по
промежутке $(a,b]$, $-\bes\le a < b < +\bes$ (сформулируйте и
докажите его).

\begin{tproblem}
Если $\ints{a}^{+\bes} f(x)\,dx$ сходится и существует
$\liml{x\ra+\bes} (xf(x))$, то $\liml{n\ra+\bes} xf(x)=0$.
\end{tproblem}

\begin{note}
В доказанной теореме можно рассматривать только возрастающие
последовательности $(t_n)$.
\end{note}

\begin{note}
Если в условиях теоремы дополнительно $f(x)\ge 0$, то её
справедливость достаточно проверить только для одной
последовательности $(t_n)$.
\end{note}

\begin{proof}
В этом случае функция $F(t) = \ints{a}^t f(x)\,dx$ неотрицательна и
возрастает на $[a,b)$ и $\liml{t\ra b-0} F(t)=l \; \Lra \; l =
\liml{n\ra+\bes} F(t_n)$ для некоторой последовательности $(t_n)$,
$a\le t_n < b$, $\liml{n\ra+\bes} t_n=b$.
\end{proof}

\subsection{Главное значение несобственного интеграла}
\subsubsection{Функции, интегрируемые по Коши на $(-\bes,+\bes)$}

\begin{dfn}1
Функция $f$ называется интегрируемой по Коши на $(-\bes,+\bes)$,
если она определена на $(-\bes,+\bes)$, интегрируема на каждом
$[a,b]$, $-\bes<a<b<+\bes$, и существует $\liml{A\ra+\bes}
\ints{-A}^{A} f(x)\,dx=l$. Число $l$ называют главным значением
несобственного интеграла по $(-\bes,+\bes)$ и обозначают \equ{l=\vp
\ints{-\bes}^{+\bes} f(x)\,dx = \liml{A\ra+\bes} \ints{-A}^{A}
f(x)\,dx.}
\end{dfn}

\begin{theorem}
Если функция $f\in\Rc[a,b]$, $-\bes<a<b<+\bes$, и нечетная, то $\vp
\ints{-\bes}^{+\bes} f(x)\,dx=0$. Если функция $f\in\Rc[a,b]$,
$-\bes<a<b<+\bes$, и чётная, то $\vp \ints{-\bes}^{+\bes} f(x)\,dx$
существует тогда и только тогда, когда сходится несобственный
интеграл $\ints0^{+\bes} f(x)\,dx$ и $\vp \ints{-\bes}^{+\bes}
f(x)\,dx = 2\ints0^{+\bes} f(x)\,dx.$
\end{theorem}

\begin{proof}
Для любого $A>0$ справедливо $\ints{-A}^{A} f(x)\,dx=0$, если $f$
--- нечётная, и $\ints{-A}^{A}f(x)\,dx=2\ints0^{A}f(x)\,dx$, если
$f$ --- чётная; затем используем определение 1 и определение
несобственного интеграла.
\end{proof}

Для любой функции $f$, определённой на $(-\bes,+\bes)$, справедливо
разложение $f(x)=\ph(x)+\psi(x)$, в котором функция $\ph(x) =
\frac{f(x)+f(-x)}2$ --- чётная, а функция $\psi(x) =
\frac{f(x)-f(-x)}2$ --- нечётная. Поэтому, \equ{\vp
\ints{-\bes}^{+\bes} f(x)\,dx = 2\ints{0}^{+\bes} \ph(x)\,dx =
\ints0^{+\bes} [f(x) + f(-x)]\,dx,} если последний несобственный
интеграл сходится.

\begin{ex}
Вычислим $\vp\ints{-\bes}^{+\bes} \frac{1+x}{1+x^2}\,dx$.
\end{ex}

\begin{proof}
\equ{f(x) = \frac{1+x}{1+x^2} = \frac1{1+x^2} + \frac{x}{1+x^2}, \;
x\in(-\bes,+\bes),} и следовательно, \equ{\vp \ints{-\bes}^{+\bes}
\frac{1+x}{1+x^2}\,dx= 2\ints0^{+\bes} \frac{dx}{1+x^2} =
2\liml{t\ra+\bes} \ints0^t \frac{dx}{1+x^2} = 2\liml{t\ra+\bes}
(\arctg t - \arctg 0) = 2\cdot\frac{\pi}2 = \pi.}
\end{proof}

Отметим, что сам несобственный интеграл расходится.

\subsubsection{Главное значение по Коши на промежутке}
Пусть функция $f$ определена на $[a,b]\backslash\hc{c}, \; a<c<b$,
$f$ интегрируема на каждом отрезке $[a,c-\al], \; [c+\al,b]$,
$0<\al<\max \hc{c-a,b-c}$.

\begin{dfn}{2}
Функцию $f$ называют интегрируемой по Коши на $[a,b]$, если
существует \equ{\liml{\al\ra0} \hr{\ints{a}^{c-\al} f(x)\,dx +
\ints{c+\al}^b f(x)\,dx} = \vp \ints{a}^b f(x)\,dx.}
\end{dfn}

\begin{ex}
Вычислим $\vp\ints{a}^b \frac{dx}{x-c}, \; a<c<b$.
\end{ex}

\begin{proof}
По определению 2, \equ{\vp \ints{a}^b \frac{dx}{x-c} =\liml{\al\ra0}
\hs{\ints{a}^{c-\al} \frac{dx}{x-c} + \ints{c+\al}^b \frac{dx}{x-c}}
= \liml{\al\ra0} \hs{\ln\hm{\al} - \ln(c-a) + \ln(b-c) -
\ln\hm{\al}} = \liml{\al\ra0} \ln\frac{b-c}{c-a} =
\ln\frac{b-c}{c-a}.}
\end{proof}

Отметим, что сам несобственный интеграл $\ints{a}^b \frac{dx}{x-c} =
\ints{a}^c \frac{dx}{x-c} + \ints{c}^b \frac{dx}{x-c}$ не существует
(расходится).

\subsubsection{Интегральный логарифм}
\begin{theorem}
Если функция $f$ непрерывна на $[a,b]$, дважды дифференцируема в
точке $c\in(a,b)$ и $f(c)=0$, $f'(c)\ne0$, то существует $\vp
\ints{a}^{b} \frac{dx}{f(x)}.$
\end{theorem}

\begin{proof}
Так как $f\in\Dc^{(2)}(c)$, то по локальной формуле Тейлора, с
учётом $f(c)=0$, имеем \eqa{1}{f(x)=f'(c)(x-c) + \frac12
[f''(c)+\al(x)] (x-c)^2, \; x\ra c,} где $\liml{x\ra c} \al(x)=0$.
Функция $\ph(x) = \frac1{f(x)} - \frac1{f'(c)(x-c)}$ непрерывна на
$[a,b]\subset\hc{c}$ и, на основании (1), имеет \equ{\liml{x\ra c}
\ph(x) = \liml{x\ra c} \frac{-\frac12 [f''(c) + \al(x)] (x-c)^2}
{f'(c) (x-c)^2 \hs{f'(c) + \frac12 (f''(c)+\al(x))(x-c)}} = -\frac12
\frac{f''(c)}{f'^2(c)}.}

Поэтому, несобственный интеграл $\ints{a}^b \ph(x)\,dx = \ints{a}^c
\ph(x)\,dx + \ints{c}^b \ph(x)\,dx$ не только сходится, но и
является интегралом Римана для $\ph(x)$ на $[a,b]$. Как и в примере
2, существует \equ{\vp \ints{a}^b \frac{dx}{f'(c)(x-c)} =
\frac1{f'(c)\ln\frac{b-c}{c-a}}}.

Таким образом, на основании представления $\frac1{f(x)} =
\frac1{f'(c)(x-c)}+\ph(x)$, $x\in[a,b]\backslash\hc{c}$, заключаем,
что существует $\vp\ints{a}^b \frac{dx}{f(x)}$, в то время как
несобственный интеграл $\ints{a}^b \frac{dx}{f(x)}$ расходится, так
как расходится несобственный интеграл $\ints{a}^b
\frac{dx}{f'(c)(x-c)}.$
\end{proof}

\begin{ex}
Интегральный логарифм li\,$x$ определяется формулой li\,$x =
\ints0^x \frac{dt}{\ln t},$ в которой несобственный интеграл
сходится только для $0<x<1$, а для $x>1$ он понимается в смысле
своего главного значения (которое существует по теореме 2).
\end{ex}

\subsection{Равномерное стремление к предельной функции по базе}
\subsubsection{Понятие равномерного стремления к предельной функции
по базе}

Рассмотрим функцию $F(x,y)$, определённую на множестве $D = X\times
Y$, $X,Y\subset\R$, и произвольную базу $\B$ на $Y$. Предположим,
что для любого $x\in X$ существует $\liml{\B} F(x,y)=f(x)$; то есть,
для произвольных $\ep>0$ и $x\in X$ существует элемент $B(\ep;x)$
базы $\B$ (выбор множества $B(\ep;x)$ зависит от $\ep>0$ и от $x\in
X$), на котором $\hm{F(x,y)-f(x)}<\ep$ для всех $y\in B(\ep;x)$.

\begin{dfn}1
Функцию $F(x,y)$ называют равномерно стремящейся к предельной
функции $f(x)$ на множестве $X$ по базе $\B$ из множества $Y$, если
\begin{nums}{-2} \item $f(x) = \liml{\B} F(x,y)$ для всех $x\in x$;
\item для любого $\ep>0$ существует такой элемент $B_{\ep}$ базы
$\B$ (выбор которого зависит только от $\ep>0$ и не зависит от точек
$x\in X$), что $\hm{F(x,y)-f(x)}<\ep$ для всех $x\in X$ и всех $y\in
B_{\ep}$.\end{nums}

Обозначение: $F(x,y)\rra f(x)$ на множестве $X$ по базе $\B$ из $Y$.

\end{dfn}

Если $Y=\N$, то $F(x,y) = F(x,n) = f_n(x), \; x\in X, \; n\in\N$, и
$(f_n(x))$ --- функциональная последовательность на $X$. Поэтому,
$F(x,y)\rra f(x)$ на множестве $X$ по базе $n\ra+\bes \; \Lra \;
f_n\rra f$ на $X$.

\subsubsection{Критерий Коши}

\begin{theorem}
Для того, чтобы функция $F(x,y)$, определённая на $D=X\times Y$,
$X,Y\subset \R$, равномерно стремилась на множестве $X$ к предельной
функции $f(x)$ по базе $\B$ из $Y$, необходимо и достаточно, чтобы
для любого $\ep>0$ существовал такой элемент $B_{\ep}$ базы $\B$,
выбор которого не зависит от точек $x\in X$ и на котором
\eqa{1}{\hm{F(x,y')-F(x,y'')}<\ep} для всех $x\in X$ и всех
$y',y''\in\B_{\ep}.$
\end{theorem}

\begin{proof}
\textbf{Необходимость.} Пусть $F(x,y) \rra f(x)$ на $X$ по базе $\B$
из $Y$. Рассмотрим произвольное $\ep>0$. Согласно определению 1,
существует элемент $B_{\ep}\in\B$ (выбор которого не зависит от
$x\in X$), что $\hm{F(x,y) - f(x)} < \frac{\ep}2$ для всех $x\in X$
и всех $y\in B_{\ep}$. Тогда $\hm{F(x,y')-F(x,y'')} \le
\hm{F(x,y')-f(x)} + \hm{F(x,y'')-f(x)} <
\frac{\ep}2+\frac{\ep}2=\ep$ для всех $x\in X$ и всех $y',y''\in
B_{\ep}$.

\textbf{Достаточность.} По условию, для произвольного $\ep>0$
существует $B_{\ep}\in\B$ (выбор которого не зависит от $x\in X$),
что $\hm{F(x,y)-F(x,y')}<\frac{\ep}2$ для всех $x\in X$ и всех $y,y'
\in B_{\ep}$. Таким образом, для любого $x\in X$ выполнен критерий
Коши существования $\liml{\B} F(x,y) = f(x), \; x\in x$; в
частности, $f(x) = \liml{\B} F(x,y')$, $x\in X$. Фиксируя $y\in
B_{\ep}$ и используя свойства монотонности предела функции по базе и
непрерывности функции $\hm{t}$, $t\in\R$, получим
\equ{\hm{F(x,y)-f(x)} = \hm{F(x,y) - \liml{\B} F(x,y')} = \liml{\B}
\hm{F(x,y) - F(x,y')} \le \frac{\ep}2 < \ep} для всех $x\in X$ и
всех $y\in B_{\ep}$; то есть, получим справедливость определения 1.
\end{proof}

\subsubsection{Сведение к равномерно сходящимся последовательностям}
Пусть $F(x,y)$ определена на $D=X\times Y\subset \R^2$. В
приложениях наиболее употребительны следующие базы $\B$ в $Y$:
\begin{nums}{-2}
\item $Y\owns y\ra y_0$;
\item $y\ra+\bes$;
\item $y\ra -\bes$;
\item $y\ra y_0\pm 0$.
\end{nums}

Этим обстоятельством объясняется выбор формулировок остальных
результатов этого параграфа, хотя они справедливы и в общем случае.

\begin{theorem}\label{thm3.10}
Функция $F(x,y)$ равномерно стремится на $X$ к предельной функции
$f(x)$ по базе $Y\owns y\ra y_0$ [по базе $y\ra+\bes$] тогда и
только тогда, когда для любой последовательности $(y_n)$, $y_n\in
Y$, $y_n\ne y_0$, $n\in\N$, $\liml{n\ra+\bes} y_n=y_0$ [любой
$(y_n), \; y_n\in Y, \; n\in\N, \; \liml{n\ra+\bes} y_n=+\bes$]
функциональная последовательность $(F(x,y_n))$ равномерно сходится к
функции $f(x)$ на $X$.
\end{theorem}

\begin{proof}
\textbf{Необходимость.} По условию, $F(x,y)\rra f(x)$ на $X$ по
$Y\owns y\ra y_0$ [$y\ra+\bes$]. Рассмотрим произвольное $\ep>0$.
Согласно определению 1, существует $\de>0$ [$\De>0$], что
$\hm{F(x,y)-f(x)}<\ep$ для всех $x\in X$ и всех $y\in Y$, $y\ne
y_0$, $\hm{y-y_0}<\de$ [всех $y\in Y, \; y>\De$]. Рассмотрим
произвольную $(y_n)$, $y_n\in Y, \; y_n\ne y_0, \; n\in\N, \;
\liml{n\ra+\bes} y_n = y_0$ [произвольную $(y_n), \; (y_n)\in Y, \;
n\in \N, \; \liml{n\ra+\bes}y_n = +\bes$]. Для числа $\de>0$
[$\De>0$] существует $N\in\N$, что $\hm{y-y_n}<\de$ [$y_n> \De$] для
всех $n\ge N$, и следовательно, $\hm{F(x,y_n)-f(x)}<\ep$ для всех
$x\in X$ и всех $n\ge N$, так что $F(x,y_n) \rra f(x)$ на $X$.

\textbf{Достаточность.} Пусть выполнено утверждение теоремы, но
$F(x,y)$ стремится к $f(x)$ на $X$ неравномерно по базе $Y\owns y\ra
y_0$ [$y\ra+\bes$]. Согласно определению 1, существует такое
$\ep_0>0$, что для любого $\de>0$ [любого $\De>0$] можно указать
такие точки $x_{\de}\in X$ [$x_{\De}\in X$] и $y_{\de} \in Y, \;
y_{\de}\ne y_0$ [$y_{\De}\in Y$], что $0<\hm{y_0 - y_{\de}}<\de$
[$y_{\de} > \De$] и $\hm{F(x_{\de}, y_{\de} - f(x_{\de}))} \ge
\ep_0$ [$\hm{F(x_{\De},y_{\De}) - f(x_{\De})}\ge\ep_0$]. Выбирая
$\de_n=\frac1n$ [$\De_n=n$], $n\in\N$, получим последовательность
$(x_n)$ в $X$ и последовательность $(y_n)$ в $Y$, что $y_n\ne y_0$,
$0<\hm{y_n-y_0}<\frac1n$ [$y_n>n$], $n\in\N$, и
$\hm{F(x_n,y_n)-f(x_n)}\ge \ep_0, \; n\in\N$. Последнее означает,
что функциональная последовательность $(F(x,y_n))$ не сходится
равномерно к $f(x)$ на $X$. Полученное противоречие заканчивается
доказательство теоремы.
\end{proof}

\subsubsection{Интегрируемость (непрерывность) предельной функции}
\begin{theorem}\label{thm3.11}
Если функция $F(x,y))$, определённая на $D=X\times Y\subset\R^2$,
непрерывна в точке $x_0\in X$ при каждом $y\in Y$ и $F(x,y)$
равномерно стремится на $X$ к предельной функции $f(x)$ по базе $\B$
из $Y$, то функция $f(x)$ непрерывна в $x_0\in X$.
\end{theorem}

\begin{theorem}\label{thm3.12}
Если функция $F(x,y)$ определена на $D=[a,b]\times Y\subset\R^2$,
интегрируема на $[a,b]$ для каждого $y\in Y$ и $F(x,y)$ равномерно
стремится к $f(x)$ на $[a,b]$ по базе $\B$ из $Y$, то $f(x)$
интегрируема на $[a,b]$.
\end{theorem}

\begin{proof}
В качестве базы $\B$ в обеих теоремах рассмотрим базы из теоремы
\ref{thm3.10}, и рассмотрим произвольную последовательность $(y_n)$
в $Y$, удовлетворяющую условиям этой теоремы. Тогда $F(x,y_n) =
f_n(x)$, $x\in X$ ($x\in[a,b]$), $n\in\N$, и $F(x,y_n) = f_n(x) \rra
f(x)$ на $X$ (на $[a,b]$). Поскольку каждая $f_n(x), \; n\in\N$ либо
непрерывна в $x_0\in X$ (условие теоремы \ref{thm3.11}), либо
интегрируема на $[a,b]$ (условие теоремы \ref{thm3.12}), то
утверждения теорем следуют теперь на основании соответствующих
свойств равномерно сходящихся последовательностей.
\end{proof}

\subsubsection{Теорема Дини}

\begin{theorem}
Пусть множество $Y\subset\R$ является промежутком $[c,y_0)$,
$-\bes<c<y_0\le+\bes$, а база $\B$ --- базой $y\ra y_0-$. Пусть
функция $F(x,y)$ определена на $D=[a,b]\times[c,y_0)$ и стремится к
предельной функции $f(x)$ на $[a,b]$ по базе $y\ra y_0-$; то есть,
$f(x)=\liml{y\ra y_0-} F(x,y), \; x\in[a,b]$. Если \begin{nums}{-2}
\item $F(x,y)$ непрерывна на $[a,b]$ по $x$ при каждом $y\in
Y=[c,y_0)$;
\item при каждом $x\in[a,b]$ функция $F(x,y)$ возрастает по
$y\in[c,y_0)$; \item $f(x)$ непрерывна на $[a,b]$,\end{nums} то
$F(x,y)$ равномерно стремится к $f(x)$ на $[a,b]$ по базе $y\ra
y_0-$.
\end{theorem}

\begin{proof}
Рассмотрим некоторую строго возрастающую последовательность $(y_n),
\; y_n\in Y, \; c<y_n<y_0 \, (<+\bes)$, имеющую $\liml{n\ra+\bes}
y_n=y_0$ ($\liml{n\ra+\bes}y_n=+\bes$). Функциональная
последовательность $F((x,y_n))$ удовлетворяет на $[a,b]$ условиям
теоремы Дини для функциональных последовательностей, и
следовательно, $F(x,y_n)\rra f(x)$ на $[a,b]$. Поэтому, для
произвольного $\ep>0$ существует $N\in\N$, $N=N(\ep)$, что
$-\ep<F(x,y_n) - f(x)<\ep$ для всех $x\in[a,b]$ и всех $n\ge N$.
Рассмотрим теперь произвольное $y\in Y, \; y\ge y_N$. Согласно
условию 2 теоремы, $F(x,y) \ge F(x,y_N)$ для любого $x\in[a,b]$, так
что $-\ep<F(x,y_N)-f(x)\le F(x,y)-f(x)$ для всех $x\in[a,b]$. Так
как $(y_n)\up\up$ и $\liml{n\ra+\bes} y_n = y_0$, то существует
$m\in\N$, что $y<y_m$, и так как $y_N\le y < y_m$, то $m\ge N$.
Поэтому, с учётом условия 2 теоремы, $F(x,y)-f(x) \le F(x,y_m) -
f(x) < \ep$. Итак, $-\ep< F(x,y) - f(x)<\ep$ для всех $x\in[a,b]$ и
всех $y\ge y_N$, $N=N(\ep)$; то есть, по определению 1, $F(x,y)\rra
f(x)$ на $[a,b]$ по базе $\B$ (вида $y\ra y_0-0$ или $y\ra+\bes$).
\end{proof}

\begin{center}\textbf{Теорема Дини для других баз}\end{center}
\begin{theorem}
Пусть функция $F(x,y)$ определена на множестве $D=[a,b]\times
Y\subset \R^2$, где $Y=(y_0,c]$, $-\bes\le y_0 < c < +\bes$, и
$F(x,y)$ стремится к предельной функции $f(x)$ по базе $\B=y\ra
y_0+$ ($y\ra y_0+0$, если $y_0\in\R$, и $y\ra -\bes$, если
$y_0=-\bes$).

Если \begin{nums}{-2} \item $F(x,y)$ при каждом $y\in Y$ непрерывна
по $x$ на $[a,b]$; \item $F(x,y)$ при каждом $x\in[a,b]$ убывает на
$Y=(y_0,c]$; \item $f(x)$ непрерывна на $[a,b]$,\end{nums} то
$F(x,y)$ равномерно стремится к $f(x)$ на $[a,b]$ по базе $y\ra
y_0+$.
\end{theorem}

\subsubsection{Перестановка двух предельных переходов}
\begin{theorem}
Пусть функция $F(x,y)$ определена на множестве $D=X\times
Y\subset\R^2$, и на базах $\B_X$ из $X$ и $\B_Y$ из $Y$ выполнены
условия: \begin{points}{-2}
          \item существует $\liml{\B_Y} F(x,y) = \ph(x), \; x\in X$;
          \item существует $\liml{\B_X} F(x,y) = \psi(y), \; y\in
          Y$;
          \item $F(x,y)\rra \ph(x)$ на множестве $X$ по базе $\B_Y$.
        \end{points}

Тогда: \begin{nums}{-2}\item существует $\liml{\B_X} \ph(x) =
\liml{\B_X} \hs{\liml{\B_Y} F(x,y)}$; \item существует $\liml{\B_Y}
\psi(y) = \liml{\B_Y} \hs{\liml{\B_X} F(x,y)}$; \item $\liml{\B_X}
\hs{\liml{\B_Y} F(x,y)} = \liml{\B_Y} \hs{\liml{\B_X}
F(x,y)}$.\end{nums}
\end{theorem}

\begin{proof}
Рассмотрим произвольное число $\ep>0$. На основании условия \pt{3},
существует такой элемент $B_{\ep}^Y$ базы $\B_{Y}$, что
\eqa{2}{\hm{F(x,y) - F(x,y')} < \frac{\ep}3} для всех $x\in X$ и
всех $y,y'\in B_{\ep}^Y$ (критерий Коши равномерного стремления
функции по базе). Поэтому, на основании условия \pt{2}, свойства
монотонности предела функции по базе и свойства непрерывности
функции $\hm{t}$, $t\in\R$, имеем \eqa{3}{\hm{\psi(y) - \psi(y')} =
\liml{\B_X} \hm{F(x,y) - F(x,y')} \le \frac{\ep}3 < \ep} для всех
$y,y'\in B_{\ep}^Y$. Неравенство (3) есть критерий Коши
существования $\liml{\B_Y} \psi(y)=l$, и $\liml{\B_Y} \psi(y')=l$.
Фиксируя $y\in B_{\ep}^Y$ в (3), как и для (2), получим
\eqa{4}{\hm{\psi(y) - l} = \liml{\B_Y} \hm{\psi(y) - \psi(y')} \le
\frac{\ep}3} для любого $y\in B_{\ep}^Y$. Так как, согласно условию
\pt1, $\liml{\B_Y} F(x,y')=\ph(x), \; x\in X$, то переходя в обеих
частях неравенства (2) к пределу по базе $\B_Y$ при фиксированном
$y\in B_{\ep}^Y$, получим \eqa{5}{\hm{F(x,y)-\ph(x)} = \liml{\B_Y}
\hm{F(x,y) - F(x,y')} \le \frac{\ep}3} для всех $x\in X$ и всех
$y\in B_{\ep}^Y$. Так как $\liml{\B_X} F(x,y) = \psi(y), \; y\in Y$,
то для $\ep>0$ существует элемент $B_{\ep}^X$ базы $\B_X$ (выбор
множества $B_{\ep}^X$ зависит, вообще говоря, и от $y\in Y$), что
\eqa{6}{\hm{F(x,y)-\psi(y)}<\frac{\ep}3} для всех $x\in B_{\ep}^X$ и
всех $y\in Y$. Фиксируем произвольное $y\in B_{\ep}^Y$. Тогда для
него справедливы (4), (5) и (6), так что \ml{\hm{\ph(x)-l} =
\hm{\ph(x) - F(x,y) + F(x,y) - \psi(y) + \psi(y)- l} \le
\\\le \hm{F(x,y) - \ph(x)} + \hm{F(x,y)-\psi(y)} + \hm{\psi(y) - l} <
\frac{\ep}3 + \frac{\ep}3 + \frac{\ep}3 = \ep} для всех $x\in
B_{\ep}^X$; то есть, $l=\liml{\B_X}\ph(x)$.
\end{proof}

\subsection{Собственные интегралы, зависящие от параметра}
\subsubsection{Определения и обозначения}

Рассмотрим функцию $f(x,y)$, определённую на множестве
$D=[a,b]\times Y$, $a,b\in\R$, $Y$ --- множество в $\R$, и
интегрируемую по аргументу $x$ на $[a,b]$ при каждом фиксированном
значении $y\in Y$. Тогда на $Y$ определена функция
\equ{I(y)=\ints{a}^b f(x,y)\,dx, \; y\in Y,} называемая собственным
интегралом функции $f(x,y)$ на $[a,b]$; $y$ --- параметр
собственного интеграла.

\begin{theorem}\label{thm3.16}
(Предел собственного интеграла по параметру). Если функция $f(x,y)$
определена на множестве $D = [a,b]\times Y\subset\R^2$ и существует
собственный интеграл $I(y)=\ints{a}^b f(x,y)\,dx$, то для любой базы
$\B$ на $Y$, по которой $f(x,y)$ равномерно стремится к предельной
функции $\ph(x)$ на $[a,b]$, справедливо утверждение
\eqa{1}{\liml{\B} I(y) = \liml{\B} \ints{a}^b f(x,y)\,dx =
\ints{a}^b \hs{\liml{\B} f(x,y)}\,dx = \ints{a}^b \ph(x)\,dx.}
\end{theorem}

\begin{proof}
Так как $f(x,y)$ интегрируема по $x\in [a,b]$ для любого $y\in Y$ и
$f(x,y)\rra \ph(x)$ по базе $\B$ из $Y$, то, согласно теореме из
пункта 3.3.4, функция $\ph$ интегрируема на $[a,b]$; то есть,
существует $\ints{a}^b \ph(x)\,dx$. Рассмотрим произвольное число
$\ep>0$. Согласно определению равномерного стремления функции к
предельной функции по базе $\B$, на $\B$ существует такой элемент
$B_{\ep}$, на котором $\hm{f(x,y) - \ph(x)} < \frac{\ep}{2(b-a)}$
для всех $x\in[a,b]$ и всех $y\in B_{\ep}$. Поэтому,
\equ{\hm{\ints{a}^b f(x,y)\,dx - \ints{a}^b \ph(x)\,dx} \le
\ints{a}^b \hm{f(x,y) - \ph(x)}\,dx \le \frac{\ep}{2(b-a)} (b-a) =
\frac{\ep}2 < \ep} для всех $y\in B_{\ep}$; то есть, $\ints{a}^b
\ph(x)\,dx = \liml{\B} \ints{a}^b f(x,y)\,dx$.
\end{proof}

В частности, если $\B=\Oc_Y(y_0), \; y_0\in Y$ --- база окрестностей
точки $y_0$, и $f(x,y)\rra f(x,y_0)$ на $[a,b]$ по базе
$\Oc_Y(y_0)$, то существует $\liml{\Oc_Y(y_0)} I(y_0)$; то есть,
функция $I(y)$ непрерывна в $y_0\in Y$.

\subsubsection{Признак непрерывности собственного интеграла по
параметру}
\begin{theorem}
Если функция $f(x,y)$ непрерывна как функция двух переменных на
$P\colon [a,b]\times[c,d]$, $a,b,c,d\in\R$, то $I(y) = \ints{a}^b
f(x,y)\,dx$ непрерывна на $[c,d]$.
\end{theorem}

\begin{proof}
Собственный интеграл $I(y) = \ints{a}^b f(x,y)\,dx$ существует, так
как функция $f(x,y)$ непрерывна (а следовательно, интегрируема) по
аргументу $x$ на $[a,b]$. Так как $P$ --- компакт в $\R^2$, то
непрерывна функция $f(x,y)$ равномерно непрерывна на $P$, и
следовательно, для любого $\ep>0$ существует такое $\de>0$,
$\de=\de(\ep)$, что $\hm{f(x',y') - f(x'',y'')}<\ep$ для всех
$M'(x',y')$, $M''(x'',y'')\in P$, у которых $\hm{x'-x''}<\de$ и
$\hm{y'-y''}<\de$.

Рассмотрим произвольное $x\in[a,b]$ и фиксируем произвольное
$y_0\in[c,d]$. Выбирая $x'=x''=x$, $y'=y\in[a,b]$ и $y''=y_0$,
получим $\hm{f(x,y)-f(x,y_0)}<\ep$ для всех $y\in[c,d]=\De$,
$\hm{y-y_0}<\de$ и всех $x\in[a,b]$, так что $f(x,y)\rra f(x,y_0)$
на $[a,b]$ по базе $\De\owns y\ra y_0$. Согласно теореме
\ref{thm3.16}, существует \equ{\liml{\De\owns y\ra y_0} I(y) =
\ints{a}^b \hs{\liml{\De\owns y\ra y_0} f(x,y)}\,dx = \ints{a}^b
f(x,y_0)\,dx = I(y_0);} то есть, функция $I(y)$ непрерывна в любой
точке $y_0\in[c,d]$ и непрерывна на отрезке $[c,d]$.
\end{proof}

\subsubsection{Дифференцирование собственного интеграла по
параметру}

\begin{theorem}
(Лейбниц). Если функция $f(x,y)$, определённая на
$P=[a,b]\times[c,d]$, непрерывна по $x$ на $[a,b]$ при каждом
$y\in[c,d]$ и частная производная $f'_y(x,y)$ непрерывна на $P$ как
функция двух переменных, то функция $I(y)=\ints{a}^b f(x,y)\,dx$
дифференцируема на $[a,b]$ и её производная $I'_y$ вычисляется по
формуле \eqa{2}{I'(y) = \ints{a}^b f'_y(x,y)\,dx, \; y\in[c,d].}
\end{theorem}

\begin{proof}
Рассмотрим вначале случай $y_0\in(c,d)$. Выбираем $h\ne0$ таким,
чтобы $(y_0+h)\in(c,d)$, так что $h$ принадлежит некоторой
проколотой окрестности $\os{\circ}{\Oc}(0)$ нуля, и
\eqa{3}{\frac{I(y_0+h)-I(y_0)}{h} = \ints{a}^b \frac{f(x,y_0+h) -
f(x,y_0)}{h}\,dx.}

Функция $F(x,h)=\frac{f(x,y_0+h)-f(x,y_0)}h$ определена на множестве
$D=[a,b]\times \os{\circ}{\Oc}(0)$ и, по формуле Лагранжа о конечном
приращении, $F(x,h)=f'_y(x,y_0+\ta h)$, $0<\ta<1$ (выбор $\ta$
зависит от $x$). Покажем, что $F(x,h)\rra f'_y(x,y_0)$ на $[a,b]$ по
базе $h\ra0$. Так как функция $f'_y(x,y)$ равномерно непрерывна на
$P=[a,b]\times[c,d]$, то для произвольного числа $\ep>0$ существует
такое $\de>0, \; \de=\de(\ep)$, что $\hm{f'_y(x',y') -
f'_y(x'',y'')}<\ep$ для всех точек $M'(x',y')$, $M''(x'',y'')\in P$,
у которых $\hm{x'-x''}<\de$ и $\hm{y'-y''}<\de$.

Рассмотрим произвольное $x\in[a,b]$ и положим $x'=x''=x$ и $y'=y_0$,
$y''=y_0+\ta h$. Тогда $\hm{x'-x''}=0<\de$ и $\hm{y'-y''} = \hm{\ta
h} < \hm{h} < \de$ для любого $h\in\os{\circ}{\Oc}(0)$,
$0<\hm{h}<\de$. Таким образом $\hm{f'_y(x,y_0) - f'_y(x,y_0+\ta
h)}<\ep$ для всех $x\in[a,b]$ и $0<\hm{h}<\de$; то есть,
$\hm{f'_y(x,y_0) - F(x,h)}<\ep$ для всех $x\in[a,b]$ и
$0<\hm{h}<\de$. Другими словами, $F(x,h)\rra f'_y(x,y_0)$ на $[a,b]$
по базе $h\ra0$. Согласно теореме пункта 4.4.1, существует
\equ{\liml{h\ra0} \ints{a}^b \frac{f(x,y_0+h) - f(x,y_0)}{h}\,dx =
\ints{a}^b \hs{\liml{h\ra0} \frac{f(x,y_0+h) - f(x,y_0)}{h}}\,dx =
\ints{a}^b f'_y(x,y_0)\,dx.}

На основании (3) заключаем, что существует $\liml{h\ra0}
\frac{I(y_0+h) - I(y_0)}h = \ints{a}^b f'_y(x,y_0)\,dx$; то есть,
существует $I'(y_0) = \ints{a}^b f'_y(x,y)\,dx$ и формула (2)
доказана для любого $y\in(c,d)$. В концевых точках отрезка $[c,d]$
аналогичным образом рассматриваются односторонние производные.
\end{proof}

\begin{note}
Производная $I'(y)$ в утверждении теоремы является непрерывной
функцией на $(c,d)$ по теореме пункта 3.4.2.
\end{note}

\subsubsection{Интегрирование собственного интеграла по параметру}

\begin{theorem}
Если функция $f(x,y)$ непрерывна на $P=[a,b]\times[c,d]\subset\R^2$
как функция двух переменных, то справедлива формула
\eqa{4}{\ints{c}^d I(y)\,dy = \ints{c}^d \,dy \ints{a}^b f(x,y)\,dx
= \ints{a}^b\,dx \ints{c}^d f(x,y)\,dy.}
\end{theorem}

\begin{proof}
Будет доказана более общая формула \eqa{5}{\ints{c}^{\eta}\,dy
\ints{a}^b f(x,y)\,dx = \ints{a}^b \,dx \ints{c}^{\eta} f(x,y)\,dy}
для любого $\eta\in[c,d]$.

Согласно теореме пункта 3.4.2, функция $I(y) = \ints{a}^b
f(x,y)\,dx$ непрерывна на $[c,d]$. Поэтому, функция
$\ints{c}^{\eta}\,dy \ints{a}^b f(x,y)\,dx = \ints{c}^{\eta}
I(y)\,dy$ непрерывна на $[c,d]$ и дифференцируема в $(c,d)$, причём
\eqa{6}{\hr{\ints{c}^{\eta}\,dy \ints{a}^b f(x,y)\,dy}'_{\eta} =
\hr{\ints{c}^{\eta} I(y)\,dy}'_{\eta} = I(\eta), \; \eta\in(c,d).}
Функция $\ph(x,\eta) = \ints{c}^{\eta} f(x,y)\,dy$ ограничена на
$P=[a,b]\times[c,d]$, непрерывна по $x$ на $[a,b]$ в силу теоремы
пункта 3.4.2 и дифференцируема по аргументу $\eta$ на $(c,d$) (как
интеграл с переменным верхним пределом), причём $\ph'_{\eta}(x,\eta)
= f(x,\eta), \; x\in[a,b]$, $\eta\in[c,d]$. Таким образом, функция
$\ph'_{\eta}(x,\eta)$ непрерывна на $P=[a,b]\times[c,d]$ как функция
двух переменных. Согласно теореме Лейбница (пункт 3.4.3), функция
$\ints{a}^b\ph(x,\eta)\,dx$ дифференцируема по параметру $\eta$ и
\equ{\hr{\ints{a}^b \ph(x,\eta)\,dx}'_{\eta} = \ints{a}^b
\ph'_{\eta}(x,\eta)\,dx = \ints{a}^b f(x,\eta)\,dx = I(\eta), \;
\eta\in(c,d).} Поэтому, \eqa{7}{\hr{\ints{a}^b \,dx \ints{c}^{\eta}
f(x,y)\,dy}'_{\eta} = \hr{\ints{a}^b \ph(x,\eta)\,dx}'_{\eta} =
I(\eta), \; \eta\in(c,d).}

По теореме из второго семестра,две непрерывные на отрезке функции,
имеющие внутри отрезка равные производные, отличаются на постоянную.
Поэтому, на основании (6) и (7) заключаем, что
\eqa{8}{\ints{c}^{\eta}\,dy \ints{a}^b f(x,y)\,dx = \ints{a}^b\,dx
\ints{c}^{\eta} f(x,y)\,dy + C, \; \eta\in[c,d].} Полагая в формуле
(8) $\eta=c$, получим $0=0+C$, откуда $C=0$ и формула (8) переходит
в (5).
\end{proof}

\subsection{Несобственные интегралы, зависящие от параметра}

\subsubsection{Определения и обозначения}

Рассмотрим функцию $f(x,y)$, определённую на множестве
$D_1=[a,b)\times Y$, $-\bes<a<b\le+\bes$, $Y\subset\R$, и для
каждого $y\in Y$ интегрируемую по аргумента $x$ на каждом отрезке
$[a,t]$, $a<t<b$. Функция $F(y,t)$, \eqa{1}{F(y,t)=\ints{a}^t
f(x,y)\,dx} определена на множестве $E_1=Y\times[a,b)$ и непрерывна
по $t$ на $[a,b)$ для каждого $y\in Y$.

\begin{dfn}{1}
Если для любого $y\in Y$ существует $\liml{t\ra b-}F(y,t) = I(y)$,
где $t\ra b-$ есть база $t\ra b-0$, если $b\in\R$, и база
$t\ra+\bes$, если $b=+\bes$, то функцию $I(y)$ называют
несобственным интегралом функции $f(x,y)$ по промежутку $[a,b)$,
зависящим от параметра $y\in Y$. Обозначение: \eqa{2}{I(y) =
\ints{a}^b f(x,y)\,dx = \liml{t\ra b-} \ints{a}^t f(x,y)\,dx =
\liml{t\ra b-} F(y,t), \; y\in Y.}
\end{dfn}

Рассмотрим теперь функцию $f(x,y)$, определённую на множестве
$D_2=(a,b]\times Y$, $-\bes\le a < b < +\bes$, $Y\subset\R$, и для
каждого $y\in Y$ интегрируемую по аргументу $x$ на каждом отрезке
$[t,b]$, $a<t<b$. Функция $\Ph(y,t)$, \eqa{1'}{\Ph(y,t)=\ints{t}^b
f(x,y)\,dx} определена на множестве $E_2=Y\times(a,b]$ и непрерывна
по $t$ на $(a,b]$ для любого $y\in Y$.

\begin{dfn}{1'}
Если для любого $y\in Y$ существует $\liml{t\ra a+} \Ph(y,t)=I(y),$
где $t\ra a+$ есть база $t\ra a+0$, если $a\in\R$, и база $t\ra
-\bes$, если $a=-\bes$, то функцию $I(y)$ называют несобственным
интегралом функции $f(x,y)$ по промежутку $(a,b]$, зависящим от
параметра $y\in Y$, и обозначают \eqa{2'}{I(y) = \ints{a}^b
f(x,y)\,dx = \liml{t\ra a+} \ints{t}^b f(x,y)\,dx = \liml{t\ra a+}
\Ph(y,t), \; y\in Y.}
\end{dfn}

Отметим, что для обоих несобственных интегралов, зависящих от
параметра, имеет место формула \eqa{3}{\ints{a}^b f(x,y)\,dx =
\ints{a}^c f(x,y)\,dx + \ints{c}^b f(x,y)\,dx, \; a<c<b,} в правой
части которой один из интегралов есть собственный интеграл,
зависящий от параметра $y\in Y$, а другой --- несобственный
интеграл, зависящий от параметра $y\in Y$, и значение несобственного
интеграла в левой части формулы (3) не зависит от выбора $c$,
$a<c<b$.

\subsubsection{Абсолютная сходимость}

Несобственный интеграл $\ints{a}^b f(x,y)\,dx$ по промежутку $[a,b)$
или $(a,b]$ называют абсолютно сходящимся на множестве $\os{\circ}Y
\subset\R$, если в каждой точке $y\in\os{\circ}Y$ сходится
несобственный интеграл $\ints{a}^b \hm{f(x,y)}\,dx$. Так как из
абсолютной сходимости несобственного интеграла следует его
сходимость, то $\os{\circ}Y\subset Y$.

\subsubsection{Остаток несобственного интеграла, зависящего от
параметра}

Рассмотрим вначале случай $-\bes<a<b\le+\bes$ и функцию $f(x,y)$,
определённую на $D_1=[a,b)\times Y$ и удовлетворяющую определению 1.
Тогда, в силу формул (3) и (1), имеем представление \eqa{4}{I(y) =
\ints{a}^b f(x,y)\,dx = \ints{a}^t f(x,y)\,dx + \ints{t}^b
f(x,y)\,dx = F(y,t)+R(y,t),} в котором функции $F(y,t)$ и $R(y,t)$
определены на множестве $E_1=Y\times[a,b)$. Несобственный интеграл
\equ{R(y,t) = \ints{t}^b f(x,y)\,dx, \; y\in Y, \; a\le t<b,}
зависящий от параметра $y\in Y$, называют остатком несобственного
интеграла (2).

Аналогично, в случае $-\bes\le a < b < +\bes$ и функции $f(x,y)$,
определённой на множестве $D_2 = (a,b]\times Y$ и удовлетворяющей
определению 1', справедлива формула \eqa{4'}{I(y) = \ints{a}^b
f(x,y)\,dx = \ints{a}^t f(x,y)\,dx + \ints{t}^b f(x,y)\,dx =
R(y,t)+\Ph(y,t),} в которой функции $\Ph(y,t)$ и $R(y,t)$ определены
на множестве $E_2=Y\times(a,b]$ и несобственный интеграл
\equ{R(y,t)=\ints{a}^t f(x,y)\,dx, \; y\in Y, \; a<t\le b} называют
остатком несобственного интеграла (2').

На основании определений 1 и 1' и формул (4) и (4') заключаем, что
несобственный интеграл (2) [(2')] сходится тогда и только тогда,
когда его остаток $R(y,t)$, определённый на множестве
$E_1=Y\times[a,b)$ [на множестве $E_2=Y\times(a,b]$], стремится по
базе $t\ra b-$ [по базе $t\ra a+$] к нулевой предельной функции на
$Y$; то есть, $\liml{t\ra b-} R(y,t)=0$, $y\in Y$ $\hs{\liml{t\ra
a+} R(y,t)=0, \; y\in Y}.$

\subsubsection{Равномерная сходимость несобственного интеграла,
зависящего от параметра}

\begin{dfn}
  2 Пусть функция $f(x,y)$ определена на множестве $D_1=[a,b)\times
  Y$, $-\bes<a<b\le+\bes$, $Y\subset\R$, и на множестве
  $E_1=Y\times[a,b)$ определена функция $F(y,t)$, задаваемая
  формулой (1). Несобственный интеграл (2) называют равномерно
  сходящимся на множестве $Y$, если \begin{nums}{-2}\item интеграл (2) сходится на $Y$
  к функции $I(y)$ в смысле определения 1; \item функция $F(y,t)$
  равномерно стремится к функции $I(y)$ на множестве $Y$ по базе
  $t\ra b-$.\end{nums}
\end{dfn}

\begin{dfn}
{2'} Пусть функция $f(x,y)$ определена на множестве $D_2=(a,b]\times
Y$, $-\bes\le < b < +\bes$, $Y\subset\R$, и на множестве
$E_2=Y\times(a,b]$ определена функция $\Ph(y,t)$, задаваемая
формулой (1'). Несобственный интеграл (2') называют равномерно
сходящимся на множестве $Y$, если \begin{nums}{-2}\item
несобственный интеграл (2') сходится на $Y$ к функции $I(y)$ в
смысле определения 1'; \item функция $\Ph(y,t)$ равномерно стремится
к функции $I(y)$ на множестве $Y$ по базе $t\ra a+$.\end{nums}
\end{dfn}

На основании этих определений, формул (4) и (4') и критерия из
пункта 3.5.3, заключаем, что несобственный интеграл (2) [(2')]
равномерно сходится на множестве $Y$ тогда и только тогда, когда
функция $R(y,t)$ равномерно стремится к нулевой предельной функции
на множестве $Y$ по базе $t\ra b-$ [$t\ra a+$]; то есть,
$R(y,t)\rra0$ на $Y$ по базе $t\ra b-$ [по базе $t\ra a+$].

В терминах неравенств этот критерий имеет вид: несобственный
интеграл (2) [(2')] равномерно сходится на множестве $Y$ тогда и
только тогда, когда для произвольного числа $\ep>0$ существует число
$b_{\ep}$, $a<b_{\ep}<b$ [существует число $a_{\ep}$, $a<a_{\ep}<b$]
(выбор числа $b_{\ep}$ [$a_{\ep}$] зависит только от $\ep>0$), что
для всех $y\in Y$ и всех $t$, $b_{\ep}<t<b$ [всех $t$,
$a<t<a_{\ep}$] справедлива оценка \eqa{5}{\hm{\ints{t}^b
f(x,y)\,dx}<\ep \hspace{30pt} \hs{\hm{\ints{a}^t f(x,y)\,dx}<\ep}.}

Отметим, как просто следствие определений 2 и 2' и приведённого выше
критерия, что равномерно сходящиеся несобственные интегралы также
обладают свойством аддитивности: \begin{nums} {-2} \item если
несобственный интеграл (2) [(2')] равномерно сходится на множестве
$Y$, то он равномерно сходится на каждом его подмножество $Y'$,
$Y'\subset Y$; \item если несобственный интеграл (2) [(2')]
равномерно сходится на множествах $Y_1$ и $Y_2$, то он равномерно
сходится на их объединении $Y=Y_1\cup Y_2$.\end{nums}

Действительно, оценки вида (5) справедливы для множеств $Y_1$ и
$Y_2$ с некоторыми числами $b_{\ep}^i$, $a<b_{\ep}^i<b$, $i=1,2$
[числами $a_{\ep}^i$, $a<a_{\ep}^i<b$, $i=1,2$], и следовательно,
эти оценки останутся справедливыми для множества $Y=Y_1\cup Y_2$ с
числами $b_{\ep}=\max(b_{\ep}^1,b_{\ep}^2)$
[$a_{\ep}=\min(a_{\ep}^1,a_{\ep}^2)$].

\subsubsection{Неравномерная сходимость несобственного интеграла,
зависящего от параметра}

\begin{dfn}
  3 Несобственный интеграл (2) [(2')] называют сходящимся
  неравномерно на множестве $Y$, если он сходится на $Y$ в смысле
  определения 1 [1'], но не имеет места свойство 2 в определении
  2 [2'].
\end{dfn}

\begin{stm*}
Несобственный интеграл $I(y)=\ints{a}^b f(x,y)\,dx$ по промежутку
$[a,b)$ (по промежутку $(a,b]$) неравномерно сходится на множестве
$Y$ тогда и только тогда, когда найдётся некоторое $\ep_0>0$ и для
произвольного $b'$, $a<b'<b$ (для произвольного $a'$, $a<a'<b$)
существуют точка $t_{b'}$, $b'<t_{b'}<b$ ($t_{a'}$, $a<t_{a'}<a'$) и
точка $y_{b'}\in Y$ ($y_{a'}\in Y$), в которых
\eqa{6}{\hm{\ints{t_{b'}}^b f(x,y_{b'})\,dx}\ge\ep_0 \hspace{30pt}
\hr{\hm{\ints{a}^{t_{a'}} f(x,y_{a'})\,dx}\ge\ep_0}.}
\end{stm*}

\begin{proof}
Определение 3 есть отрицание определения 2, а утверждение (6) ---
отрицание утверждения (5).
\end{proof}

\begin{ex}
Исследуем на сходимость, абсолютную сходимость и равномерную
сходимость несобственный интеграл $\ints0^{+\bes} ye^{-xy}\,dx, \;
y\in\R$.
\end{ex}

\begin{proof}
Подинтегральная функция $f(x,y)=ye^{-xy}$ определена для всех
$(x,y)\in D = [0,+\bes)\times\R$. Так как $f(x,y)=0$ при $y=0$ для
всех $x\in[0,+\bes)$, то несобственный интеграл сходится и равен
нулю при $y=0$. Пусть $y\ne0$. Тогда остаток (при $t>0$) \equ{R(y,t)
= \ints{t}^{+\bes} ye^{-xy}\,dx = \hs{-e^{-xy}}^{+\bes}_{t} =
\liml{A\ra+\bes} \hr{-e^{-Ay} + e^{-yt}} = \case{e^{-yt}, \mbox{
если } y>0,\\ -\bes, \mbox{ если } y<0.}} Таким образом, интеграл
сходится при $y\ge0$ и расходится при $y<0$. Так как
$\hm{f(x,y)}=\hm{y} e^{-xy}$ для всех $(x,y)\in D$ и $\hm{f(x,y)} =
f(x,y)$ для $y\ge0$, $x\in[0,+\bes)$, то исследуемый интеграл
абсолютно сходится при всех $y\ge0$; то есть, области его сходимости
и абсолютной сходимости совпадают.
\end{proof}

Если $c>0$ и $y\in[c,+\bes)$, то $\hm{R(y,t)} = e^{-yt} \le
e^{-ct}$, $t\ge0$, и $\liml{t\ra+\bes} e^{-ct}=0$. Следовательно,
$R(y,t)\rra0$ на $[c,+\bes)$ по базе $t\ra+\bes$, и несобственный
интеграл равномерно сходится на каждом промежутке $[c,+\bes)$,
$c>0$.

Если $y\in[0,c]$, $c>0$, то для произвольного $b'$, $0<b'<+\bes$,
выбираем $t_{b'}>0$ таким, чтобы $t_{b'} > \max \hr{b',\frac1c}$; то
есть, $t_{b'}>b'$ и $t_{b'}>\frac1c$, $0<\frac1{t_{b'}}<c$, и
полагаем $y_{b'}=\frac1{t_{b'}}$, $0<y_{b'}<c$. Тогда
$\hm{R(y_{b'},t_{b'})}=e^{-1}=\ep_0>0$, и следовательно,
несобственный интеграл сходится неравномерно на $[0,c]$, $c>0$, и по
свойству аддитивности он сходится неравномерно на всём промежутке
$[0,+\bes)$.

\subsubsection{Критерий Коши равномерной сходимости несобственного
интеграла, зависящего от параметра}

\begin{theorem}
Несобственный интеграл $\ints{a}^b f(x,y)\,dx=I(y)$ по промежутку
$[a,b)$ (по $(a,b]$) равномерно сходится на множестве $Y\subset\R$ в
том и только в том случае, когда для любого числа $\ep>0$ существует
такое число $b_{\ep}$, $a<b_{\ep}<b$ (существует $a_{\ep},
a<a_{\ep}<b$), что для всех $y\in Y$ и всех $t_i$, $b_{\ep}<t_i<b$,
$i=1,2$ (всех $t_i$, $a<t_i<a_{\ep}$, $i=1,2$) справедливо
неравенство \eqa{7}{\hm{\ints{t_1}^{t_2} f(x,y)\,dx}<\ep.}
\end{theorem}

\begin{proof}
Согласно свойству аддитивности определённого интеграла, для любых
$t_i$, $a<t_i<b$, $i=1,2$, имеем \equ{\ints{t_1}^{t_2} f(x,y)\,dx =
\ints{a}^{t_2} f(x,y)\,dx - \ints{a}^{t_1} f(x,y)\,dx =
F(y,t_2)-F(y,t_1)} \equ{\hr{\ints{t_1}^{t_2} f(x,y)\,dx =
\ints{t_1}^b f(x,y)\,dx - \ints{t_2}^b f(x,y)\,dx = \Phi(y,t_1) -
\Phi(y,t_2)}.}

Поэтому, условие (7) равносильно условию
$\hm{F(y,t_2)-F(y,t_1)}<\ep$ для всех $t_i$, $b_{\ep}<t_i<b$,
$i=1,2$ (условию $\hm{\Phi(y,t_2)-\Phi(y,t_1)}<\ep$ для всех $t_i$,
$a<t_i<a_{\ep}$, $i=1,2$) и всех $y\in Y$; то есть, равносильно
критерию Коши равномерного стремления функции $F(y,t)$ (функции
$\Phi(y,t)$) к предельной функции на множестве $Y$ по базе $t\ra b-$
(по базе $t\ra a+$). Последнее равносильно свойству равномерной
сходимости несобственного интеграла по параметру (определения 2 и
2').
\end{proof}

\subsubsection{Признак Вейерштрасса равномерной сходимости
несобственного интеграла}

\begin{theorem}
Пусть функция $f(x,y)$ определена на множестве $D_1 = [a,b)\times Y$
(на множестве $D_2=(a,b]\times Y$) и на множестве $E_1=Y\times[a,b)$
(на множестве $E_2=Y\times(a,b]$) определена функция $F(y,t)$
(функция $\Phi(y,t)$), задаваемая формулой (1) ((1')). Если
$\hm{f(x,y)}\le\ph(x)$ для всех $(x,y)\in D_1$ (всех $(x,y)\in D_2$)
и сходится несобственный интеграл $\ints{a}^b \ph(x)\,dx$ по $[a,b)$
(по $(a,b]$), то несобственные интегралы $\ints{a}^b f(x,y)\,dx$ и
$\ints{a}^b \hm{f(x,y)}\,dx$ сходятся равномерно на множестве $Y$.
\end{theorem}

\begin{proof}
Пусть, для определённости, функция $f(x,y)$ определена на
$D_1=[a,b)\times Y$. Рассмотрим произвольное $\ep>0$. Согласно
критерию Коши сходимости несобственного интеграла $\ints{a}^b
\ph(x)\,dx$, существует такое $b_{\ep}$, $a<b_{\ep}<b$, что для всех
$t_i$, $b_{\ep}<t_i<b$, $i=1,2$, справедливо неравенство
$\hm{\ints{t_1}^{t_2} \ph(x)\,dx}<\ep$. Для простоты считаем
$b_{\ep}<t_1<t_2<b$. Тогда \equ{\hm{\ints{t_1}^{t_2} f(x,y)\,dx} \le
\ints{t_1}^{t_2} \hm{f(x,y)}\,dx \le \ints{t_1}^{t_2} \ph(x)\,dx =
\hm{\ints{t_1}^{t_2} \ph(x)\,dx} < \ep} для всех $y\in Y$ и всех
$t_i$, $b_{\ep}<t_i<b$, $i=1,2$. Согласно критерию Коши,
несобственные интегралы $\ints{a}^b f(x,y)\,dx$ и $\ints{a}^b
\hm{f(x,y)}\,dx$ по промежутку $[a,b)$ равномерно сходятся на
множестве $Y$.

Аналогично рассматривается случай промежутка $(a,b]$.
\end{proof}

\subsubsection{Признак Абеля равномерной сходимости несобственных
интегралов}

\begin{theorem}
Если функции $f(x,y)$, $g(x,y)$ определены на множестве
$D_1=[a,b)\times Y$ (на множестве $D_2=(a,b]\times Y$) и обладают
свойствами: \begin{nums}{-2} \item несобственный интеграл
$\ints{a}^b f(x,y)\,dx$ равномерно сходится на множестве $Y$; \item
функция $g(x,y)$ ограничена на $D_1$ (на $D_2$) и для каждого $y\in
Y$ монотонна по $x$ на $[a,b)$  (на $(a,b]$),\end{nums} то
несобственный интеграл $\ints{a}^b f(x,y)g(x,y)\,dx$ равномерно
сходится на множестве $Y$.
\end{theorem}

\begin{proof}
Доказательство приведём для случая $[a,b)$ и $D_1=[a,b)\times Y$.
Согласно условию 2, существует такое $M>0$, что $\hm{g(x,y)}\le M$
для всех $(x,y)\in D_1$. Кроме того, для любого $y\in Y$ и любых
$t_i$, $a<t_i<b$, $i=1,2$, на $[t_1,t_2]$ выполнены условия второй
теоремы о среднем значении для определённых интегралов (функция
$f(x,y)$ интегрируема на $[t_1,t_2]$ для любого $y\in Y$ в силу
условия 1 теоремы и определения 1 сходимости несобственного
интеграла, зависящего от параметра), так что
\eqa{8}{\ints{t_1}^{t_2} f(x,y)g(x,y)\,dx = g(t_1,y)
\ints{t_1}^{\xi} f(x,y)\,dx + g(t_2,y)\ints{\xi}^{t_2} f(x,y)\,dx,}
где $\xi\in[t_1,t_2]$ (и выбор $\xi$ зависит от $y\in Y$). Из (8)
следует оценка \eqa{9}{\hm{\ints{t_1}^{t_2} f(x,y)g(x,y)\,dx} \le M
\hs{\hm{\ints{t_1}^{\xi} f(x,y)\,dx} + \hm{\ints{\xi}^{t_2}
f(x,y)\,dx}}, \; \xi\in[t_1,t_2].}

Рассмотрим теперь произвольное число $\ep>0$. На основании условия 1
и критерия Коши, для $\ep>0$ существует такое $b_{\ep}$,
$a<b_{\ep}<b$, что $\hm{\ints{t'}^{t''} f(x,y)\,dx} <
\frac{\ep}{2M}$ для всех $y\in Y$ и всех $t^j$, $b_{\ep}<t^j<b$,
$j=1,2$. Поэтому, \eqa{10}{\hm{\ints{t_1}^{\xi} f(x,y)\,dx} <
\frac{\ep}{2M} \mbox { и } \hm{\ints{\xi}^{t_2}
f(x,y)\,dx}<\frac{\ep}{2M}} для всех $t_i$, $b_{\ep}<t_i<b$, $i=1,2$
(заметим, что также $b_{\ep}<\xi<b$). Объединяя неравенства (9) и
(10), получим оценку \equ{\hm{\ints{t_1}^{t_2} f(x,y)g(x,y)\,dx} <
M\hr{\frac{\ep}{2M} + \frac{\ep}{2M}}=\ep,} справедливую для всех
$y\in Y$ и всех $t_i$, $b_{\ep} < t_i < b$, $i=1,2$, и которая есть
критерий Коши равномерной сходимости на множестве $Y$ исследуемого
несобственного интеграла.
\end{proof}

\begin{imp*}
Если функция $f(x)$ определена на $[a,b)$ (на $(a,b]$) и сходится
несобственный интеграл $\ints{a}^b f(x)\,dx$, а функция $g(x,y)$
определена и ограничена на $D_1=[a,b)\times Y$ (на $D_2=(a,b]\times
Y$) и для каждого $y\in Y$ монотонна на $[a,b)$ (на $(a,b]$), то
несобственный интеграл $\ints{a}^b f(x)g(x,y)\,dx$ равномерно
сходится на множестве $Y$.
\end{imp*}

\begin{proof}
Частный случай предыдущей теоремы.
\end{proof}

\begin{ex}
Если сходится $\ints0^{+\bes} f(x)\,dx$, то $\ints0^{+\bes}
f(x)e^{-xy}\,dx$ и $\ints0^{+\bes} f(x) e^{-x^2y}\,dx$ сходятся
равномерно на $Y=[0,+\bes)$.
\end{ex}

\begin{proof}
Функции $g_1(x,y)=e^{-xy}$ и $g_2(x,y)=e^{-x^2y}$ ограничены на
множестве $D=[0,+\bes)\times [0,+\bes)$ и для каждого
$y\in[0,+\bes)$ монотонны по $x$ на $[0,+\bes)$.
\end{proof}

\subsubsection{Признак Дирихле равномерной сходимости несобственных
интегралов}

\begin{theorem}
Если функции $f(x,y)$, $g(x,y)$ определены на множестве
$D_1=[a,b)\times Y$ (на множестве $D_2=(a,b]\times Y$) и обладают
свойствами: \begin{nums}{-2} \item на множестве $E_1=Y\times[a,b)$
(на $E_2=Y\times(a,b]$) формулой (1) (формулой (1')) определена и
ограничена функция $F(x,y)$ (функция $\Phi(x,y)$); \item для каждого
$y\in Y$ функция $g(x,y)$ монотонна по $x\in[a,b)$ (по $x\in(a,b]$)
и $g(x,y)\rra0$ на $Y$ по базе $x\ra b-$ (по базе $x\ra
a+$),\end{nums} то несобственный интеграл $\ints{a}^b f(x,y)g(x,y)
\, dx$ равномерно сходится на множестве $Y$.
\end{theorem}

\begin{proof}
Как и в доказательстве признака Абеля, замечаем, что для любых
$t_i$, $a<t_i<b$, $i=1,2$, справедлива формула (8). Согласно условию
1 теоремы, существует такое число $L>0$, что $\hm{F(y,t)}\le L$
($\hm{\Phi(y,t)}\le L$) для всех $(y,t)\in E_1$ (всех $(y,t)\in
E_2$). С учётом свойства аддитивности определённого интеграла, как и
в доказательстве критерия Коши, заключаем, что
\equ{\hm{\ints{t_1}^{\xi} f(x,y)\,dx} = \hm{F(\xi,y)-F(t_1,y)} =
\hm{\Phi(t_1,y)-\Phi(\xi,y)} \le 2L} и \equ{\hm{\ints{\xi}^{t_2}
f(x,y)\,dx} = \hm{F(t_2,y)-F(\xi,y)} = \hm{\Phi(\xi,y) -
\Phi(t_2,y)} \le 2L,} и следовательно, на основании формулы (8)
имеем оценку \eqa{11}{\hm{\ints{t_1}^{t_2} f(x,y)g(x,y)\,dx} \le
\hr{\hm{g(t_1,y)} + \hm{g(t_2,y)}}2L,} справедливую для любого $y\in
Y$ и любых $t_i$, $a<t_i<b$, $i=1,2$.

Рассмотрим произвольное число $\ep>0$. По определению свойства
равномерного стремления функции $g(x,y)$ по базе $x\ra b-$ (по базе
$x\ra a+$), для числа $\ep>0$ существует такое число $b_{\ep}$,
$a<b_{\ep}<b$ (такое число $a_{\ep}$, $a<a_{\ep}<b$), что
$\hm{g(x,y)} < \frac{\ep}{4L}$ для всех $y\in Y$ и всех $x$,
$b_{\ep}<x<b$ (всех $x$, $a<x<a_{\ep}$). Поэтому, на основании (11),
неравенство \equ{\hm{\ints{t_1}^{t_2} f(x,y)g(x,y)\,dx} < 2L \cdot
\frac{\ep}{4L} + 2L\cdot \frac{\ep}{4L}=\ep} справедливо для всех
$y\in Y$ и всех $t_i$, $b_{\ep}<t_i<b$, $i=1,2$ (всех $t_i$,
$a<t_i<a_{\ep}$, $i=1,2$). Последнее есть критерий Коши равномерной
сходимости на множестве $Y$ исследуемого несобственного интеграла.
\end{proof}

\subsubsection{Связь с рядами}
Выше, в параграфе 1 обсуждалась связь между несобственными
интегралами и порождаемыми ими числовыми рядами. Здесь изучение этой
связи будет углублено.

Пусть функция $f(x,y)$ определена на множестве $D_1=[a,b)\times Y$,
$-\bes<a<b\le+\bes$, $Y\subset\R$, и для каждого $y\in Y$
интегрируема по аргументу $x$ на каждом отрезке $[a,t]$, $a<t<b$,
так что на множестве $E_1=Y\times[a,b)$ определена функция $F(y,t)$,
задаваемая формулой (1).

Рассмотрим произвольную последовательность $(t_n)$, $t_0=a$,
$a<t_n<b$, $\liml{n\ra+\bes}t_n=b$ (возможно, $b=+\bes$), и образуем
функциональный ряд \eqa{12}{\sumnui a_n(y), \; y\in Y,} с общим
членом $a_n(y)$, $n\in\N$, \eqa{13}{a_n(y) = \ints{t_{n-1}}^{t_n}
f(x,y)\,dx, \; y\in Y, \; n\in\N.}

Частные суммы $s_n(y)$, $n\in\N$, ряда (12) имеют вид, с учётом
(13), \eqa{14}{s_n(y) = \sumkun a_k(y) = \sumkun
\ints{t_{n-1}}^{t_n} f(x,y)\,dx = \ints{a}^{t_n} f(x,y)\,dx =
F(y,t_n), \; y\in Y, \; n\in\N.}

Таким же способом, как и в пункте 1.5 параграфа 1 доказывается (на
основании формулы (14)), что несобственный интеграл $\ints{a}^b
f(x,y)\,dx=I(y)$ сходится на множестве $Y$ тогда и только тогда,
когда для любой последовательности $(t_n)$, $t_0=a$, $a<t_n<b$,
$\liml{n\ra+\bes}t_n=b$, функциональный ряд (12) сходится на
множестве $Y$.

Поскольку, на основании теоремы пункта 3.3 параграфа 3, функция
$F(y,t)\rra I(y)$ на множестве $Y$ по базе $t\ra b-$ тогда и только
тогда, когда $F(y,t_n)\rra I(y)$ на множестве $Y$ для любой
последовательности $(t_n)$, $t_0=a$, $a<t_n<b$, $\liml{n\ra+\bes}
t_n=b$, то несобственный интеграл $\ints{a}^b f(x,y)\,dx=I$ по
промежутку $[a,b)$ равномерно сходится на множестве $Y\subset\R$
тогда и только тогда, когда $F(y,t_n)\rra I(y)$ на $Y$ для любой
последовательности $(t_n)$, $t_0=a$, $a<t_n<b$,
$\liml{n\ra+\bes}t_n=b$.

Аналогичные рассуждения можно провести для зависимого от параметра
$y\in Y$ несобственного интеграла $\ints{a}^b f(x,y)\,dx$ по
промежутку $(a,b]$, $-\bes\le a < b < +\bes$.

\subsection{Свойства равномерно сходящихся несобственных интегралов}
\subsubsection{Предельный переход под знак интеграла}

\begin{theorem}
Пусть функция $f(x,y)$, определённая на множестве $D_1=[a,b)\times
Y$, и база $\B$ на $Y$ обладают свойствами: \begin{nums}{-2} \item
несобственный интеграл $\ints{a}^b f(x,y)\,dx = I(y)$ по промежутку
$[a,b)$ равномерно сходится на множестве $Y$; \item для каждого
$x\in[a,b)$ существует $\liml{\B}f(x,y) = \ph(x)$; \item для любого
$t$, $a<t<b$, $f(x,y)\rra \ph(x)$ на отрезке $[a,t]$ по базе
$\B$.\end{nums} Тогда сходится несобственный интеграл $\ints{a}^b
\ph(x)\,dx$ и справедлива формула \eqa{1}{\liml{\B} \ints{a}^b
f(x,y)\,dx = \ints{a}^b \hs{\liml{\B} f(x,y)}\,dx = \ints{a}^b
\ph(x)\,dx.}
\end{theorem}

\begin{proof}
Согласно условию 1 теоремы, $f(x,y)$ интегрируема по $x$ на каждом
отрезке $[a,t]$, $a<t<b$, для любого $y\in Y$. Так как $f(x,y)\rra
\ph(x)$ по базе $\B$ на $[a,t]$, то $\ph(x)$ интегрируема на $[a,t]$
(параграф 3, пункт 3.4). Поэтому, для функции $F(y,t)$, \equ{
F(y,t)= \ints{a}^t f(x,y)\,dx, \; y\in Y, \; t\in[a,b)} существует
\equ{ \liml{\B} F(y,t) = \liml{\B} \ints{a}^t f(x,y)\,dx =
\ints{a}^t \hs{\liml{\B} f(x,y)}\,dx = \ints{a}^t \ph(x)\,dx, \;
t\in[a,b)} (параграф 4, пункт 4.1). Так как, согласно условию 1
теоремы, $F(y,t)\rra I(y)$ на $Y$ по базе $t\ra b-$, то, по теореме
о перестановке двух предельных переходов, существует \equ{\liml{\B}
\hs{\liml{t\ra b-} F(y,t)} = \liml{t\ra b-} \hs{\liml{\B} F(y,t)} =
\liml{t\ra b-} \ints{a}^t \ph(x)\,dx = \ints{a}^b \ph(x)\,dx,} так
что доказаны сходимость несобственного интеграла от функции $\ph(x)$
и формула (1), поскольку $\liml{t\ra b-} F(y,t) = \ints{a}^b
f(x,y)\,dx, \; y\in Y.$
\end{proof}

\subsubsection{Непрерывность несобственного интеграла по параметру}

\begin{theorem}
Если функция $f(x,y)$ непрерывна как функция двух переменных на
множестве $D_1 = [a,b)\times[c,d]$ и несобственный интеграл
$\ints{a}^b f(x,y)\,dx = I(y)$ по промежутку $[a,b)$ равномерно
сходится на $[c,d]$, то функция $I(y)$ непрерывна на $[c,d]$.
\end{theorem}

\begin{proof}
Рассмотрим произвольное $t$, $a<t<b$. Так как $f(x,y)$ непрерывна на
$P_t=[a,t]\times[c,d]$, то $f(x,y)$ равномерно непрерывна на $P_t$ и
для любого $y_0\in[c,d]=\De$ функция $f(x,y)\rra f(x,y_0)$ на
$[a,t]$ по базе $\De \owns y\ra y_0$ (см. доказательство теоремы из
пункта 4.2, параграф 4). Таким образом, по теореме предыдущего
пункта существует \equ{\liml{\De\owns y\ra y_0} I(y) = \liml{\De
\owns y\ra y_0} \ints{a}^b f(x,y)\,dx = \ints{a}^b
\hs{\liml{\De\owns y\ra y_0} f(x,y)}\,dx = \ints{a}^b f(x,y_0)\,dx =
I(y_0).}
\end{proof}

\begin{note}
  Если, дополнительно, непрерывная на $D_1$ функция $f(x,y)\ge0$, то
  справедлива обратная теорема: из непрерывности несобственного
  интеграла $I(y) = \ints{a}^b f(x,y)\,dx$ на $[c,d]$ следует его
  равномерная сходимость на $[c,d]$.
\end{note}

\begin{proof}
Функция $F(y,t) = \ints{a}^t f(x,y)\,dx$, $y\in[c,d]$, $t\in[a,b)$
непрерывна по $y$. Согласно теореме Дини, $F(y,t)\rra I(y)$ на
$[c,d]$ по базе $t\ra b-$.
\end{proof}

\subsubsection{Интегрирование несобственного интеграла по параметру}

\begin{theorem}
Если функция $f(x,y)$ непрерывна на множестве
$D_1=[a,b)\times[c,d]$, и несобственный интеграл $I(y) = \ints{a}^b
f(x,y)\,dx$ равномерно сходится на $[c,d]$, то справедлива формула
\eqa{1}{\ints{c}^d I(y)\,dy = \ints{c}^d\,dy \ints{a}^b f(x,y)\,dx =
\ints{a}^b \,dx \ints{c}^d f(x,y)\,dy,} в которой последний интеграл
--- несобственный по $[a,b)$.
\end{theorem}

\begin{proof}
По теореме предыдущего пункта, функция $I(y)$ непрерывна на $[c,d]$,
а следовательно, интегрируема на $[c,d]$ и первое равенство в
формуле (1) доказано. Рассмотрим произвольную возрастающую
последовательность $(t_n)$, $t_0=a\le t_n < b$, $\liml{n\ra+\bes}
t_n=b$, и образуем ряд \eqa{2}{I(y) = \ints{a}^b f(x,y)\,dx =
\sumnui \ints{t_{n-1}}^{t_n} f(x,y)\,dx = \sumnui a_n(y), \;
y\in[c,d],} который, согласно условиям теоремы и редукции
несобственных интегралов к функциональным рядом (пункт 5.10,
параграф 5), равномерно сходится на $[c,d]$. На каждом
прямоугольнике $P_n = [t_{n-1},t_n]\times[c,d] \subset D_1, \;
n\in\N$, выполнены все условия теоремы пунктов 4.3 и 4.4 параграфа
4, о непрерывности и интегрируемости собственных интегралов,
зависящих от параметра, согласно которым каждая функция $a_n(y)$
непрерывна на $[c,d]$ и справедливы формулы \eqa{3}{\ints{c}^d
a_n(y)\,dy = \ints{c}^d\,dy \ints{t_{n-1}}^{t_n} f(x,y)\,dx =
\ints{t_{n-1}}^{t_n}\,dx \ints{c}^d f(x,y)\,dy, \; n\in\N.}
Поскольку ряд (2) равномерно сходится на $[c,d]$ и все $a_n(y)$,
$n\in\N$, непрерывны на $[c,d]$, он почленно интегрируем на $[c,d]$
и, с учётом формул (3), справедливо равенство \mla{4}{\ints{c}^d
I(y)\,dy = \sumnui \ints{c}^d a_n(y)\,dy = \sumnui
\ints{t_{n-1}}^{t_n}\,dx \ints{c}^d f(x,y)\,dy = \liml{n\ra+\bes}
\sumkun \ints{k-1}^{t_k}\,dx \ints{c}^d f(x,y)\,dy =
\liml{n\ra+\bes} \ints{a}^{t_n}\,dx \ints{c}^d f(x,y)\,dy,} в
котором использовано также свойство аддитивности определённого
интеграла. В силу редукции несобственных интегралов к функциональным
рядам (пункт 5.10, параграф 5), формула (4) означает, что
$\ints{c}^d I(y)\,dy =\ints{a}^b \,dx \ints{c}^d f(x,y)\,dy$, то
есть, что справедливо второе равенство в утверждении (1) теоремы.
\end{proof}

\subsubsection{Дифференцирование несобственного интеграла по
параметру}

\begin{theorem}
Пусть функция $f(x,y)$ определена на множестве
$D_1=[a,b)\times[c,d]\subset\R^2$, $-\bes<a<b\le+\bes$ [на множестве
$D_2=(a,b]\times[c,d]\subset\R^2$, $-\bes\le a < b < +\bes$] и
обладает свойствами: \begin{nums}{-2} \item функция $f$ непрерывна
по аргументу $x$ для каждого $y\in[c,d]$; \item частная производная
$f'_y(x,y)$ непрерывна как функция двух переменных на множестве
$D_1$ [на множестве $D_2$]; \item несобственный интеграл $\ints{a}^b
f(x,y)\,dx = I(y)$ сходится на $[c,d]$; \item несобственный интеграл
$\ints{a}^b f'_y(x,y)\,dx$ равномерно сходится на $[c,d]$.
\end{nums} Тогда функция $I(y)=\ints{a}^b f(x,y)\,dx$
дифференцируема на $[c,d]$ и справедлива формула \eqa{5}{I'(y) =
\ints{a}^b f'_y(x,y)\,dx, \; y\in[c,d].}
\end{theorem}

\begin{proof}
Рассмотрим произвольную возрастающую последовательность $(t_n)$,
$t_0=a\le t_n < b$, $\liml{n\ra+\bes} t_n=b$, и функциональный ряд
(2), который, в силу условию 3 теоремы, сходится к функции $I(y) =
\ints{a}^b f(x,y)\,dx$ на $[c,d]$. Функция $f(x,y)$ на каждом
прямоугольнике $P_n=[t_{n-1},t_n]\times[c,d] \subset D_1$, $n\in\N$,
удовлетворяет теореме Лейбница о дифференцировании собственного
интеграла по параметру, согласно которой \equ{a'_n(y) =
\ints{t_{n-1}}^{t_n} f'_y(x,y)\,dx, \; y\in[c,d], \; n\in\N,} и
следовательно, с учётом условия 4 теоремы, сходится ряд \ml{\sumnui
a'_n(y) = \liml{n\ra+\bes} \sumkun a'_k(y) = \liml{n\ra+\bes}
\sumkun \ints{t_{k-1}}^{t_k} f'_y(x,y)\,dx = \liml{n\ra+\bes}
\ints{a}^{t_n} f'_y(x,y)\,dx = \ints{a}^b f'_y(x,y)\,dx, \;
y\in[c,d]} (использованы также свойство аддитивности определённого
интеграла и редукция несобственного интеграла к функциональным
последовательностям). Так как ряд $\sumnui a_n(y) = \ints{a}^b
f(x,y)\,dx = I(y)$ сходится на $[c,d]$, а ряд $\sumnui a'_n(y) =
\ints{a}^b f'_y(x,y)\,dx$ равномерно сходится на $[c,d]$, то по
теореме о почленном дифференцировании функциональных рядов
\equ{\ints{a}^b f'_y(x,y)\,dx = \hr{\sumnui a_n(y)}'_y = I'(y), \;
y\in[c,d].}
\end{proof}

\subsubsection{Теорема Дини для несобственных интегралов}

\begin{theorem}
Пусть $Y=[c,y_0)$, $-\bes<c<y_0\le + \bes$, и символ $\B$ обозначает
базу $y\ra y_0-0$, если $y_0\in\R$, и базу $y\ra+\bes$, если
$y_0=+\bes$. Пусть функция $f(x,y)$ определена на множестве
$D_1=[a,b)\times Y$ и обладает свойствами: \begin{nums}{-2} \item
$f(x,y)\ge0$ на $D_1$; \item $f(x,y)$ непрерывна по $x$ на $[a,b)$;
\item для каждого $x\in[a,b)$ функция $f(x,y)$ возрастает по $y\in
Y$; \item функция $f(x,y)$ стремится к некоторой предельной функции
$\ph(x)$ на $[a,b)$ по базе $\B$; то есть, $\ph(x) = \liml{\B}
f(x,y)$, $x\in[a,b)$; \item функция $\ph(x)$ непрерывна на $[a,b)$.
\end{nums} Если сходится несобственный интеграл $\ints{a}^b
\ph(x)\,dx$, то несобственный интеграл $I(y) = \ints{a}^b
f(x,y)\,dx$ равномерно сходится на $Y=[c,y_0)$, и поэтому,
\eqa{6}{\liml{\B} I(y) = \ints{a}^b \ph(x)\,dx.}
\end{theorem}

\begin{proof}
На основании условий 2--5 и теоремы Дини из пункта 3.5, параграф 3,
функция $f(x,y)$ равномерно стремится к предельной функции $\ph(x)$
на $[a,b)$ по базе $\B$. В силу условий 1 и 3, для каждого
$x\in[a,b)$ и каждого $y\in Y$ справедливо $0\le f(x,y) \le \ph(x) =
\supl{\B} f(x,y)$ (теорема Вейерштрасса о пределе возрастающей
функции). Поэтому, с учётом сходимости несобственного интеграла
$\ints{a}^b \ph(x)\,dx$ и признака Вейерштрасса (пункт 5.7, параграф
5) заключаем, что несобственных интеграл $I(y) = \ints{a}^b
f(x,y)\,dx$ равномерно сходится на множестве $Y$, а соотношение (6)
справедливо на основании теоремы пункта 6.1.
\end{proof}

Аналогичная теорема справедлива для множестве $Y=(y_0,c]$, $-\bes\le
y_0<c<+\bes$, с заменой в условии 3 предыдущей теоремы свойства
возрастания функции $f(x,y)$ на множестве $Y$ на свойство её
убывания на $Y$.

\subsection{Эйлеровы интегралы}
\subsubsection{Интеграл Эйлера первого рода (по Лежандру);
бета--функция}

Рассмотрим несобственный интеграл \eqa{1}{B(p,q) = \ints0^1 x^{p-1}
(1-x)^{q-1}\,dx,} зависящий от параметров $p,q\in\R$, и представим
его в виде \eqa{2}{B(p,q) = \ints{0}^{\frac12} x^{p-1}
(1-x)^{q-1}\,dx + \ints{\frac12}^1 x^{p-1} (1-x)^{q-1}\,dx =
I_1(p,q) + I_2(p,q),} где $I_1(p,q)$ имеет особенность только в
точке $x=0$, а $I_2(p,q)$ --- только в точке $x=1$.

Так как для любого $q\in\R$ функция $(1-x)^{q-1}$ положительна,
непрерывна и ограничена на отрезке $[0,\frac12]$, то существуют
$C_1^q>0$ и $C_2^q$, что $C_1^q x^{p-1} \le x^{p-1} (1-x)^{q-1} \le
C_2^q x^{p-1}$ для всех $x\in(0,\frac12]$ и всех $p\in\R$. Так как
несобственный интеграл $\ints0^{\frac12} x^{p-1}\,dx$ сходится для
всех $p>0$ и расходится для всех $p\le0$, то по признаку сравнения
несобственных интегралов, заключаем, что несобственный интеграл
$I_1(p,q)$ сходится только при $p>0$ и всех $q\in\R$.

Аналогично, функция $x^{p-1}$ положительна, непрерывна и ограничена
на отрезке $[\frac12,1]$ для любого $p\in\R$, и следовательно,
существуют $C_1^p>0$, $C_2^p>0$, что $C_1^p(1-x)^{q-1}\le x^{p-1}
(1-x)^{q-1} \le C_2^p (1-x)^{q-1}$ для всех $x\in[\frac12,1)$ и всех
$q\in\R$. Так как несобственный интеграл $\ints{\frac12}^1
(1-x)^{q-1}\,dx$ сходится при $q>0$ и расходится при $q\le0$, то как
и выше, заключаем, что несобственный интеграл $I_2(p,q)$ сходится
только при $q>0$ и всех $p\in\R$.

Окончательно, в силу (2), бета--функция Эйлера $B(p,q)$ определена
только для $p>0$ и $q>0$.

\subsubsection{Непрерывность бета--функции}

\begin{theorem}
Функция $B(p,q)$, задаваемая интегралом (1), непрерывна по каждому
аргументу $p$ и $q$ в своей области определения.
\end{theorem}

\begin{proof}
Подинтегральная функция $f(x,p,q) = x^{p-1} (1-x)^{q-1}$ непрерывна
как функция трёх переменных $(x,p,q)$ на множестве $E=(0,1)\times
(0,+\bes) \times (0,+\bes)$ и $f(x,p,q)\ge0$. Для произвольного
$p_0>0$ и $q_0>0$ интеграл $B(p_0,q_0)$ сходится (функция
существует) и для всех $p\ge p_0 > 0$ и $q\ge q_0 > 0$ справедливы
оценки \equ{0< x^{p-1} (1-x)^{q-1} \le x^{p_0-1} (1-x)^{q_0-1}, \;
x\in(0,1).} Согласно признаку Вейерштрасса, несобственный интеграл
(1) равномерно сходится при $p\ge p_0>0$, $q\ge q_0>0$. Поэтому (по
теореме пункта 6.2, параграф 6) функция $B(p,q)$ непрерывна при всех
$p\ge p_0 > 0$ и при всех $q\ge q_0 > 0$.

Рассмотрим теперь произвольные $p>0$ и $q>0$ и выберем такие $p_0>0$
и $q_0>0$, чтобы $p>p_0>0$ и $q>q_0>0$. По предыдущему, $B(p,q)$
непрерывна в точке $p>0$ и в точке $q>0$.
\end{proof}

\subsubsection{Симметричность бета--функции}

\begin{theorem}
$B(p,q)=B(q,p)$, $p>0$, $q>0$.
\end{theorem}

\begin{proof}
В несобственном интеграле (1) для $B(p,q)$ совершим замену
переменной интегрирования $x=1-t$, $t=1-x$, $dx=-dt$. Получим
\equ{B(p,q) = \ints0^1 x^{p-1} (1-x)^{q-1}\,dx = -\ints0^1
(1-t)^{p-1} t^{q-1}\,dt = \ints0^1 t^{q-1} (1-t)^{p-1}\,dt =
B(q,p).}
\end{proof}

\subsubsection{Функциональное уравнение для $B(p,q)$}

Интегрируя (1) по частям, получим \ml{B(p,q+1) = \ints0^1 x^{p-1}
(1-x)^q\,dx = \hs{\frac{x^p}p (1-x)^q}^1_0 + \frac{q}p \ints0^1 x^p
(1-x)^{q-1}\,dx = \hs{x^p \equiv x^{p-1} - x^{p-1}(1-x)} = \\ = 0 +
\frac{q}p \ints0^1 x^{p-1} (1-x)^{q-1}\,dx - \frac{q}p \ints0^1
x^{p-1} (1-x)^q\,dx = \frac{q}p B(p,q) - \frac{q}p B(p,q+1),} откуда
\eqa{2}{B(p,q+1) = \frac{q}{p+q} B(p,q), \; p>0, \; q>0.} В силу
симметричности функции $B(p,q)$ имеем также \eqa{2'}{B(p+1,q) =
\frac{p}{p+q} B(p,q), \; p>0, \; q>0.}

\subsubsection{Частный случай}

Пусть $q=n\in\N$; согласно (2), \equ{B(p,n) = \frac{n-1}{p+n-1}
B(p,n-1) = \frac{n-1}{p+n-1}\cdot \frac{n-2}{p+n-2} B(p,n-2) =
\ldots = \frac{(n-1)(n-2)\sd 2\cdot1}{(p+n-1)(p+n-2)\sd(p+1)}
B(p,1).}

Но \equ{B(p,1) = \ints0^1 x^{p-1}\,dx = \hs{\frac{x^p}p}^1_0 =
\frac1p,} так что \eqa{3}{B(p,n) = \frac{(n-1)!}{p(p+1)\sd(p+n-1)},
p>0, n\in\N.}

Пусть $p=m\in\N$. Тогда \equ{B(m,n) = \frac{(n-1)!}{m(m+1)\sd
(m+n-1)} = \frac{(n-1)! 1\cdot 2 \sd (m-1)}{1\cdot 2 \sd (m-1)m(m+1)
\sd (m+n-1)} = \frac{(m-1)!(n-1)!}{(m+n-1)!}.}

Ранее доказано, что $(n-1)!=\Ga(n)$, $(m-1)!=\Ga(m)$, $m,n\in\N$,
так что \equ{B(m,n) = \frac{\Ga(m)\Ga(n)}{\Ga(m+n)}, \; m,n\in\N.} В
четвёртом семестре мы докажем замечательную формулу Эйлера
\equ{B(p,q) = \frac{\Ga(p)\Ga(q)}{\Ga(p+q)}, \; p>0, \; q>0.}

\subsubsection{Интеграл Эйлера второго рода; гамма-функция}

Рассмотрим несобственный интеграл \equ{I(s) = \ints0^{+\bes} x^{s-1}
e^{-x}\,dx = \ints0^1 x^{s-1} e^{-x}\,dx + \ints1^{+\bes} x^{s-1}
e^{-x}\,dx = I_1(s)+I_2(s).}

Так как $e^{-1} x^{s-1} \le x^{s-1} e^{-x} \le x^{s-1}$ для
$x\in(0,1]$ и всех $s\in\R$, то несобственный интеграл $I_1(s)$
сходится при $s-1>-1$, или $s>0$, и расходится при $s-1\le-1$, или
$s\le0$; то есть, $I_1(s)$ сходится только при $s>0$.

Так как $\liml{x\ra+\bes} \frac{x^{s-1}e^{-x}}{\frac1{x^2}} =
\liml{x\ra+\bes} \frac{x^{s+1}}{e^x}=0$ для любого $s\in\R$ и
$\ints1^{+\bes} \frac{dx}{x^2}$ сходится, то, согласно предельной
форме признака сравнения несобственных интегралов, интеграл $I_29s)$
сходится при всех $s\in\R$.

Окончательно, $I(s)$ сходится только при $s>0$.

Полагая $x=\ln\frac1z$, $z=e^{-x}$, $dx=-\frac1z\,dz = -e^x\,dz$,
$0<z<1$, в интеграле $I(s)$ получим \equ{I(s)= -\ints1^0
\hr{\ln\frac1z}^{s-1} e^{-x}e^x\,dz = \ints0^1
\hr{\ln\frac1z}^{s-1}\,dz, \; s>0.} Известно, что $0\le \ln\frac1z =
\liml{n\ra+\bes} n(1-z^{\frac1n}), \; 0<z\le1.$

Рассмотрим выражение $\frac{1-z^{\al}}{\al}$, $z\in(0,1]$ ---
фиксировано, как функцию аргумента $\al$, $\al>0$, производная
которой равна \equ{\hr{\frac{1-z^{\al}}{\al}}'_{\al} = \frac{-\al
z^{\al} \ln z - (1-z^{\al})}{\al^2} = \frac{z^{\al} (1-\al\ln z) -
1}{\al^2}.} Функция $\ph(a) = z^{\al} (1-\al\ln z)-1$ определена и
непрерывна по переменной $\al$ для $\al\ge0$ и $\ph(0)=0$.
Производная $\ph'(\al) = z^{\al} \ln z(1-\al\ln z) - z^{\al}\ln z =
-\al z^{al} (\ln z)^2 < 0$ для всех $\al>0$. Таким образом, функция
$\ph(\al)$ строго убывает при $\al\ge0$ и $\ph(\al)<\ph(0)=0$ для
$\al>0$.

Итак, функция $F(z,\al) = \frac{1-z^{\al}}{\al}$ строго убывает по
аргументу $\al>0$. Если $\al=\frac1n$, $n\in\N$, то функция $F(z,n)
= n(1-z^{\frac1n})$ определена на множестве $E=(0,1]\times\N$,
непрерывна на $E$ (как функция двух переменных) и $F(z,n)\ge0$, и
$F(z,n)$ строго возрастает по аргументу $n$. Таким образом, функция
$F(z,n)$ удовлетворяет на множестве $E$ всем условиям теоремы Дини
из пункта 6.5, параграфа 6, согласно которой (с учётом свойства
непрерывности степенной функции) \equ{I(s) = \ints0^1
\hr{\ln\frac1z}^{s-1}\,dz = \ints0^1 \hr{\liml{n\ra+\bes}
n(1-z^{\frac1n})}^{s-1}\,dz = \ints0^1 \liml{n\ra+\bes}
\hr{n(1-z^{\frac1n})}^{s-1}\,dz = \liml{n\ra+\bes} \ints0^1 n^{s-1}
(1-z^{\frac1n})^{s-1}\, dz.}

Полагая $z^{\frac1n}=y$, $z=y^n$, $dz=ny^{n-1}\,dy$, $0<y<1$,
получим \equ{ I(s) = \liml{n\ra+\bes} n^s \ints0^1 y^{n-1}
(1-y)^{s-1}\,dy = \liml{n\ra+\bes} n^s B(n,s) = \liml{n\ra+\bes}
\frac{n^s (n-1)!}{s(s+1)\sd (s+n-1)} = \Ga(s),} где использованы
формула (3) и формула Эйлера--Гаусса для $\Ga(s)$. Итак, \equ{\Ga(s)
= \ints0^{+\bes} x^{s-1} e^{-x}\,dx, \; s>0.}

\subsubsection{Дифференцируемость гамма-функции}

В главе 2, параграф 3, пункт 3.5, доказано, что $\Ga(s)$ бесконечно
дифференцируема в своей области определения $D_{\Ga} =
\R\backslash(-\N_0)$, $\N_0 = \N\cup\hc{0}$. В частности, $\Ga(s)$
--- бесконечно дифференцируемая функция для $s>0$. Поэтому,
\equ{\Ga'(s) = \ints0^{+\bes} x^{s-1} \ln x e^{-x}\,dx, \; s>0} и
\equ{\Ga''(s) = \ints0^{+\bes} x^{s-1} e^{-x} (\ln x)^2 \,dx>0, \;
s>0,} так что функция $\Ga(s)$ выпукла вниз в интервале $(0,+\bes)$
и производная $\Ga'(s)$ строго возрастает в $(0,+\bes)$.

\subsubsection{График функции $\Ga(s)$ на интервале $(0,+\bes)$}

По определению, \equ{\Ga(1) = \ints0^{+\bes} e^{-x}\,dx =
\liml{t\ra+\bes} \ints0^t e^{-x}\,dx = \liml{t\ra+\bes}
\hs{-e^{-x}}^t_0 = \liml{t\ra+\bes} (-e^{-t}+1)=1,} и
$\Ga(2)=\Ga(1)=1$. По теореме Ролля, в интервале $(1,2)$ существует
$s_0$, $1<s_0<2$, в которой $\Ga'(s_0)=0$. Так как $\Ga'(s)$ строго
возрастает в интервале $(0,+\bes)$, то $s_0$ --- единственный ноль
производной $\Ga'(s)$ и $s_0$ --- точка строгого минимума выпуклой
вниз функции $\Ga(s)$. Так как $\Ga(s+1) = s\Ga(s)$, $s>0$, то
$\Ga(s) = \frac{\Ga(s+1)}s$, $s>0$, и $\liml{s\ra +0} \Ga(s) =
\liml{s\ra +0} \frac{\Ga(s+1)}s=+\bes$ (поскольку $\liml{s\ra+0}
\Ga(s+1)= \Ga(1) = 1$). Для всех $s>n+1$, $n\in\N$,
$Ga'(s)>\Ga'(s_0)=0$, и следовательно, функция $\Ga(s)$ строго
возрастает при $s>s_0$, так что $\Ga(s) > \Ga(n+1)=n!$, и поэтому
$\liml{s\ra+\bes} \Ga(s) = +\bes$. График функции $\Ga(s)$ изображён
на рисунке.

\subsubsection{Интеграл Эйлера--Пуассона}

По формуле дополнения для гамма-функции, $\Ga(x)\Ga(1-x) =
\frac{\pi}{\sin\pi x}$, $x\notin \Z$. При $x=\frac12$ имеем
$\Ga(\frac12) = \sqrt{\pi}$, так как $\Ga(s)>0$, $s>0$. С другой
стороны, по определению \equ{\Ga\hr{\frac12} = \ints0^{+\bes}
x^{-\frac12} e^{-x}\,dx = \mbmat{ x^{\frac12}=t & dx=2t\,dt \\ x=t^2
& } = \ints0^{+\bes} 2e^{-t^2}\,dt,} откуда \equ{\ints0^{+\bes}
e^{-x^2}\,dx = \frac{\sqrt{\pi}}2} --- интеграл Эйлера--Пуассона.

Кроме того, \equ{\ints{-\bes}^{+\bes} e^{-x^2}\,dx = \ints{-\bes}^0
e^{-x^2}\,dx + \ints0^{+\bes} e^{-x^2}\,dx = \ints0^{+\bes}
e^{-t^2}\,dt + \ints0^{+\bes} e^{-x^2}\,dx = \sqrt{\pi},} где $t=-x$
в первом слагаемом последней суммы.

\subsection{Некоторые способы вычисления несобственных интегралов}
\subsubsection{Интеграл Эйлера}

\equ{I=\ints0^{\frac{\pi}2} \ln\sin x\,dx} вычисляется заменой
переменной интегрирования $x=2t$, $0\le t \le \frac{\pi}4$,
$dx=2\,dt$, и \equ{I = 2\ints0^{\frac{\pi}4} \ln\sin 2t\,dt =
2\ints0^{\frac{\pi}4} [\ln2 + \ln\sin t + \ln\cos t]\,dt =
\frac{\pi}2 \ln2 + 2 \ints0^{\frac{\pi}4} \ln\sin t\,dt +
2\ints0^{\frac{\pi}4} \ln\cos t\,dt.} В последнем интеграле полагаем
$t=\frac{\pi}2 - u$, $\frac{\pi}4 \le u \le \frac{\pi}2$, $dt=-du$,
и приводим к виду \equ{2\ints0^{\frac{\pi}4} \ln\cos t\,dt =
-2\ints{\frac{\pi}2}^{\frac{\pi}4} \ln\cos\hr{\frac{\pi}2 - u}\,du =
2\ints{\frac{\pi}4}^{\frac{\pi}2} \ln\sin u\,du,} так что, в силу
свойства аддитивности определённого интеграла, \equ{I =
\frac{\pi}2\ln2 + 2\ints0^{\frac{\pi}2} \ln\sin t\,dt = \frac{\pi}2
\ln2 + 2I,} откуда $I=-\frac{\pi}2\ln2$. К числу $I$ сводятся
интегралы $\ints0^{\frac{\pi}2} \frac{x}{\tg x}\,dx$ и
$\ints0^{\frac{\pi}2} \frac{\arcsin x}x\,dx$.

\subsubsection{Интеграл Дирихле}

Доказана сходимость интеграла $I=\ints0^{+\bes} \frac{\sin x}x\,dx$.
Для вычисления его значения рассмотрим интеграл \eqa{1}{I(y) =
\ints0^{+\bes} \frac{\sin x}x e^{-xy}\,dx.} Функция $F(x,y) =
\frac{\sin x}x e^{-xy} = f(x)e^{-xy}$ непрерывна, как функция двух
переменных, на множестве $D=[0,+\bes) \times [0,+\bes)$, если
считать $f(0) = 1 = \liml{x\ra0} \frac{\sin x}x$. Согласно следствию
к признаку Абеля равномерной сходимости несобственного интеграла,
зависящего от параметра (см. пункт 5.8, параграф 5, примеры),
несобственный интеграл (1) равномерно сходится по $y\in[0,+\bes)$, а
по теореме о непрерывности несобственного интеграла по параметру
(пункт 6.2, параграф 6), функция $I(y)$ непрерывна в точке $y=0$ и
$I=\liml{y\ra+0} I(y)=I(0)$.

Формальное дифференцирование по $y$ интеграла в правой части формулы
(1) приводит к несобственному интегралу \eqa{2}{-\ints0^{+\bes}
\frac{\sin x}x\cdot x e^{-xy}\,dx = -\ints0^{+\bes} e^{-xy} \sin x
\,dx.}

Рассмотрим несобственный интеграл \eqa{3}{J(y) = \ints0^{+\bes}
e^{-xy} \sin x\,dx.} Для каждого $y>0$ существует $\liml{x\ra+\bes}
g(x,y) = \liml{x\ra+\bes} e^{-xy}=0$. Для произвольного $y_0>0$ и
произвольного $x\in[0,+\bes)$ оценки $0\le g(x,y) = e^{-xy} \le
e^{-xy_0}$ справедливы для всех $y\ge y_0$. Поэтому, для любого
числа $\ep>0$, выбирая $b_{\ep} = \frac1{y_0} \hm{\ln\frac1{\ep}}$,
заключаем, что $0\le g(x,y) \le e^{-xy_0}\le \ep$ одновременно для
всех $y\ge y_0 > 0$ и всех $x> b_{\ep}$. Таким образом, функция
$g(x,y)\rra0$ на множестве $Y_0 = [y_0,+\bes)$, $y_0>0$, по базе
$x+\bes$, и поскольку $\hm{\ints0^t \sin x\,dx} = \hm{-\cos
t+1}\le2$ для всех $t\in[0,+\bes)$, то по признаку Дирихле интеграл
(3) равномерно сходится на $Y_0 = [y_0,+\bes)$ для всякого $y_0>0$.
Следовательно, на каждом множестве $Y_0 = [y_0,+\bes)$, $y_0>0$, с
учётом (1)--(3), справедливо $I'(y) = -J(y)$, $y\in Y_0$.

Для любого $y>0$ существует $y_0>0$, что $y>y_0>0$, и на промежутке
$[y_0,+\bes)$ справедлива формула $I'(y) = -J(y)$; в частности, она
справедлива в точке $y>y_0$. Окончательно, имеем формулу
\eqa{4}{I'(y) = -J(y), \; y>0.} Интегрируя по частям дважды, получим
\ml{J(y) = \hs{-\frac1y \sin x e^{-xy}}^{+\bes}_0 + \frac1y
\ints0^{+\bes} \cos x \cdot e^{-xy}\,dx = \liml{t\ra+\bes}
\hs{-\frac1y \sin x e^{-ty}} + 0 + \frac1y \ints0^{+\bes} \cos x
\hr{-\frac1y de^{-xy}} = \\ \hs{-\frac1{y^2} \cos x
e^{-xy}}^{+\bes}_0 - \frac1{y^2} \ints0^{+\bes} \sin x e^{-xy}\,dx =
\liml{t\ra+\bes} \hs{-\frac1{y^2} \cos x e^{-ty}} + \frac1{y^2} -
\frac1{y^2} J(y) = \frac1{y^2} - \frac1{y^2} J(y), \; y>0,} откуда
$$J(y) = \frac1{1+y^2}, \; y>0,$$ и согласно (4), имеем
\eqa{4'}{I'(y) = -\frac1{1+y^2}, \; y>0.}

Поэтому, \eqa{5}{I(y) = -\arctg y + C, \; y\ge0.}

Оценка \equ{\hm{I(y)} \le \ints0^{+\bes} \hm{\frac{\sin x}x}
e^{-xy}\,dx \le \ints0^{+\bes} e^{-xy}\,dx = \liml{t\ra+\bes}
\hs{-\frac1y e^{-xy}}^t_0 = \liml{t\ra+\bes} \hr{-\frac1y e^{-ty} +
\frac1y} = \frac1y, \; y>0,} показывает, что $\liml{y\ra+\bes} I(y)
= 0$. Поэтому, согласно (5), \equ{0 = \liml{y\ra+\bes} I(y) =
\liml{y\ra+\bes} [-\arctg y + C] = -\frac{\pi}2 + C,} откуда
$C=\frac{\pi}2$ и справедлива формула \eqa{6}{I(y) = \frac{\pi}2 -
\arctg y, \; y\ge0.} Согласно (6), $\liml{y\ra+0} I(y) =
\frac{\pi}2$. Но $\liml{y\ra+0} I(y) =I(0) = I$, и поэтому
$I=\frac{\pi}2$; то есть, \equ{\ints0^{+\bes} \frac{\sin x}x\,dx =
\frac{\pi}2.}

\subsubsection{Разрывный множитель Дирихле}

Рассмотрим несобственный интеграл \equ{I(a) = \ints0^{+\bes}
\frac{\sin ax}x\,dx, \; a\in\R.} Если $a=0$, то $\sin ax=0$,
$x\in[0,+\bes)$, и $I(0)=0$. Если $a>0$, то подстановка $ax=t$
приводит к интегралу \equ{I(a) = \ints0^{+\bes} \frac{\sin ax}x\,dx
= \ints0^{+\bes} \frac{\sin t}{\frac{t}a}a\,dt = \ints0^{+\bes}
\frac{\sin t}t\,dt = \frac{\pi}2, \; a>0.}

Если $a<0$, то $-a>0$ и $I(-a)=-I(a)$, поскольку
$\sin(-ax)=-\sin(ax)$, $x\ge0$. По предыдущему, $I(-a)=\frac{\pi}2$,
и следовательно, $I(a)=-\frac{\pi}2$, $a<0$. Окончательно,
\eqa{7}{\ints0^{+\bes} \frac{\sin ax}x\,dx = \case{\frac{\pi}2,
\mbox{ если } a>0, \\ 0, \mbox{ если } a=0,\\ -\frac{\pi}2, \mbox {
если } a<0.}} или \equ{\ints0^{+\bes} \frac{\sin ax}x\,dx =
\frac{\pi}2 \sgn a,} так что $$\sgn a = \frac2{\pi} \ints0^{+\bes}
\frac{\sin ax}x\,dx.$$

Несобственный интеграл $\ints0^{+\bes} \frac{\sin\al x \cos \be
x}x\,dx$, $\al>0$, $\be>0$, называют разрывным множителем Дирихле.
Согласно (7), \equ{\ints0^{+\bes} \frac{\sin \al x \cos \be x}x\,dx
= \frac12 \hs{\ints0^{+\bes} \frac{\sin(\al+\be)x}x\,dx +
\ints0^{+\bes} \frac{\sin(\al-\be)x}x\,dx} = \case{\frac{\pi}2,
\mbox{ если } \al>\be, \\ \frac{\pi}4, \mbox{ если } \al=\be, \\ 0,
\mbox{ если } \al<\be.}}

\section{Приближение функций тригонометрическими и алгебраическими
многочленами}

\subsection{Классы интегрируемых функций}

\subsubsection{Модуль непрерывности функций}

Рассмотрим произвольную непрерывную функцию $f$ на промежутке
$\ha{a,b}$ и произвольное число $\de>0$. Величина $\om_f(\de) =
\om(\de;f) = \sup\hc{\hm{f(x')-f(x'')} \bvl x',x''\in\ha{a,b}, \;
\hm{x'-x''}\le\de}$ называется \textbf{модулем непрерывности
функции} $f$ на $\ha{a,b}$. По определению, $\om(\de)\ge0$ для всех
$\de>0$. С увеличением числа $\de$ расширяется множество
$\hc{\hm{f(x') - f(x'')} \bvl x',x''\in\ha{a,b}, \;
\hm{x'-x''}\le\de}$ и поэтому его $\sup$ не может уменьшаться; то
есть, $\om(\de)$ --- возрастающая функция аргумента $\de$. Отсюда
следует существование одностороннего предела функции $\om_f(\de)$
при $\de\ra+0\colon \la=\liml{\de\ra+0} \om_f(\de)$.

Если $\la=0$, то $f$ равномерно непрерывна на $\ha{a,b}$, поскольку
для любого $\ep>0$ существует такое $\de>0$, что $\hm{f(x') -
f(x'')} \le \om_f(\de) < \ep$ для всех $x',x''\in\ha{a,b}$,
$\hm{x'-x''}\le\de$.

\begin{exx}
Функция $f(x) = \cos\frac{\pi}x$, $x\in(0,1)$, непрерывна в своей
области определения и для неё $\la\ge2$.
\end{exx}

\begin{proof}
Возьмём две точки $x_n=\frac1n$, $x_{n+1} = \frac1{n+1}$ интервала
$(0,1)$ и положим $\de_n = \frac1n - \frac1{n+1} = \frac1{n(n+1)}$.
Последовательность $(\de_n)$ стремится к нулю,но $$\hm{f(x_n) -
f(x_{n+1})} = \hm{\cos\pi n - \cos(n+1)\pi} = 2,$$ и следовательно,
\equ{\om_f(\de_n) = \sup\hc{\hm{f(x') - f(x'')} \bvl x',x''\in(0,1),
\; \hm{x'-x''}<\de_n}\ge2} и $\la = \liml{\de\ra+0} \om_f(\de)\ge2$.
\end{proof}

Согласно теореме Гейне--Кантора, всякая непрерывна на отрезке
функция равномерно непрерывна на нём и для неё $\la=0$.

\begin{theorem}
$\om_f(n\de) \le n\om_f(\de)$ для всех $\de>0$ и всех $n\in\N$.
\end{theorem}

\begin{proof}
Пусть $f$ непрерывна на $\ha{a,b}$ и $x,y\in\ha{a,b}$, $\hm{x-y} \le
n\de$ и $x<y$. Рассмотрим $x_k = x+k\frac{y-x}n$, $k=\ol{0,n}$.
Тогда $\hm{x_{k+1}-x_k} = \frac{y-x}n\le\de$, $k=\ol{0,n}$, и
\equ{\hm{f(y) - f(x)} = \hm{\sumkun (f(x_k) - f(x_{k-1}))} \le
\sumkun \hm{f(x_k) - f(x_{k-1})} \le n\om(\de).} Поэтом,
$\om_f(n\de) \le n\om_f(\de)$.
\end{proof}

\begin{imp*}
$\om_f(c\de) \le (c+1)\om_f(\de)$ для всех $\de>0$ и всех $c>0$.
\end{imp*}

\begin{proof}
Рассмотрим целую часть $[c]=n\ge0$, $n+1\in\N$. Тогда $n\le c < n+1$
и так как $\om(\de)$ возрастает, то $\om_f(c\de) \le \om_f((n+1)\de)
\le (n+1) \om_f(\de) \le (c+1)\om_f(\de)$.
\end{proof}

\subsubsection{Классы непрерывных функций по модулю непрерывности}

\begin{theorem}
Если функция $f$ непрерывна на $\ha{a,b}$, дифференцируема в $(a,b)$
за исключением некоторого конечного множества $K$ и производная $f'$
ограничена в своей $D_{f'}$; то есть, $\hm{f'(x)}\le M$, $M>0$, для
всех $x\in(a,b)\subset K$, то $\om_f(\de) \le M\de$, $\de>0$.
\end{theorem}

\begin{proof}
Рассмотрим произвольные $x,y\in\ha{a,b}$, $x<y$, и предположим, что
интервал $(x,y)$ содержит $l$ точек $x_j\in K$, $j=\ol{1,l}$. На
каждом отрезке $[x,x_1]$, $[x_1,x_2]$, \ldots, $[x_{l-1},x_l]$,
$[x_l,y]$ выполнены все условия теоремы Лагранжа о конечных
приращениях, согласно которой \ml{\hm{f(y) - f(x)} \le \hm{f(y) -
f(x_l)} + \hm{f(x_l) - f(x_{l-1})} \spl \hm{f(x_2) - f(x_1)} +
\hm{f(x_1) - f(x)} = \hm{f'(\xi_l)} (y-x_l) +\\+ \hm{f'(\xi_{l-1})}
(x_l - x_{l-1}) \spl \hm{f'(\xi_1)} (x_2-x_1) + \hm{f'(\xi)} (x_1 -
x) \le M \hm{y - x_l + x_l - x_{l-1} \spl x_2 - x_1 + x_1 - x} =\\=
M\hm{y-x} \le M\de,} и следовательно $\om_f(\de) \le M\de$, $\de>0$.
\end{proof}

\begin{dfn}
1 Функция $f$ принадлежит классу Липшица на промежутке $\ha{a,b}$,
если существует такое число $M>0$, что $\om_f(\de) \le M\de$,
$\de>0$.
\end{dfn}

Из этого определения непосредственно следует, что если функция $f$
на $\ha{a,b}$ классу Липшица с константой $M>0$, то в любой точке
$x\in(a,b)$, в которой существует производная $f'(x)$, справедлива
оценка $\hm{f'(x)}\le M$. Действительно, для любого $y\in(a,b)$,
$y\ne x$, справедливо неравенство $\hm{\frac{f(y) - f(x)}{y-x}}\le
M$, переходя в котором к пределу при $y\ra x$, получим
$\hm{f'(x)}\le M$.

\begin{exx}
Функция $f(x) = \sqrt[3]{x}$ не принадлежит классу Липшица на
$[-1,1]$, поскольку её производная функция не ограничена в своей
области определения $D_{f'}\subset[-1,1]$. Но можно доказать, что
$\hm{\sqrt[3]x - \sqrt[3]y} \le 2 \hm{x-y}^{\frac13}$ для любых
$x,y\in[-1,1]$ (проверьте это!).
\end{exx}

\begin{dfn}
2 Функция $f$ принадлежит классу Гёльдера порядка $\al$, $0<\al<1$,
на промежутке $\ha{a,b}$, если существует такое число $M>0$, что
$\om_f(\de) \le M\de^{\al}, \; \de>0$.
\end{dfn}

\begin{dfn}
3 Функция $f$ принадлежит классу Дини--Липшица на промежутке
$\ha{a,b}$, если $\liml{\de\ra0} \om_f(\de) \ln\frac1{\de}=0$.
\end{dfn}

Если $\om_f(\de) \le M\de^{\al}$, $0 < \al\le1$, то $\liml{\de\ra0}
\om_f(\de)\ln\frac1{\de}=0$, поскольку $\liml{\de]ra0} \om_f(\de)
\ln\frac1{\de} \le \liml{\de\ra0} M\de^{\al} \ln\frac1{\de}=0$.
Однако и самый широкий из определённых классов --- класс
Дини--Липшица --- не содержит всех непрерывных функций на
промежутке.

\subsubsection{Ортогональные системы функций}

\begin{dfn}
4 Функции $f$ и $g$ называют ортогональными на отрезке $[a,b]$, если
$f,g\in\Rc[a,b]$ и $\ints{a}^b f(x)g(x)\,dx=0$, $\ints{a}^b
f^2(x)\,dx\ne0$, $\ints{a}^b g^2(x)\,dx\ne0$.
\end{dfn}

\begin{dfn}
5 Функциональная последовательность $(f_n(x))$ образует
ортогональную систему функций на $[a,b]$, если $f_n\in\Rc[a,b]$,
$n\in\N$, и $f_i$, $f_j$ ортогональны для всех $i\ne j$.
\end{dfn}

\begin{stm*}
Тригонометрическая система, то есть система функций $1$, $\cos x$,
$\sin x$, $\cos2x$, \ldots, $\cos nx$, $\sin nx$, \ldots, $n\in\N$,
ортогональна на отрезке $[0,2\pi]$ (на $[-\pi,\pi]$).
\end{stm*}

\begin{proof}
$\mat{\ints{-\pi}^{\pi} 1\cdot\cos kx\,dx = \hs{\frac1k\sin
kx}^{\pi}_{-\pi} = \frac1k (\sin k\pi - \sin(-k\pi)) =0, \; k\in\N.
\\ \ints{-\pi}^{\pi} 1\cdot \sin kx\,dx = \hs{-\frac1k \cos
k\pi}^{\pi}_{-\pi} = -\frac1k (\cos k\pi - \cos(-k\pi)) =0, \;
k\in\N, \\ \ints{-\pi}^{\pi} \cos kx \cdot \cos lx\,dx = \frac12
\ints{-\pi}^{\pi} [\cos(k+l)\pi + \cos(k-l)x]\,dx =0, \; k\ne l. \\
\ints{-\pi}^{\pi} \sin kx \sin lx\,dx = \frac12 \ints{-\pi}^{\pi}
[\cos(k-l)x - \cos(k+l)x]\,dx =0, \;k\ne l. \\ \ints{-\pi}^{\pi}
\sin kx \cos lx\,dx = \frac12 \ints{-\pi}^{\pi} [\sin(k+l)x +
\sin(k-l)x]\,dx = 0, \; k,l\in\N.}$
\end{proof}

\subsection{Положительные тригонометрические многочлены}
\subsubsection{Определения}

Для произвольных действительных чисел $A_0$, $A_1$, $B_1$, \ldots,
$A_n$, $B_n$, \ldots, $n\in\N$ \equ{T(x) = A_0 + \sumkun (A_k \cos
kx + B_k \sin kx)} называют тригонометрическим многочленом
(полиномом) степени (порядка) $n\in\N$, если $A_n^2 + B_n^2\ne0$.
Если $T(x)\ge0$ для всех $x\in\R$, то $T(x)$ называют положительным
многочленом. Многочлен $T(x)$ называют чётным, если все $B_k=0$.

В дальнейшем нас будут интересовать положительные многочлены вида
\eqa{1}{u_n(x) = \frac12 + \rho_1^{(n)}\cos x + \rho_2^{(n)}\cos 2x
\spl \rho_n^{(n)} \cos nx.}

\begin{theorem}
Для любого положительного многочлена вида (1) справедливо
$\rho_1^{(n)}<1$, и существуют последовательности положительных
многочленов вида (1), для которых $\liml{n\ra+\bes} \rho_1^{(n)} =
1$.
\end{theorem}

\begin{proof}
Так как многочлены $u_n(x)$ и $1-\cos x$ положительны и непрерывны,
то \equ{ 0 < I_n = \ints{-\pi}^{\pi} (1-\cos x) \hr{\frac12 +
\rho_1^{(n)} \cos x \spl \rho_n^{(n)} \cos nx} \,dx.} Для вычисления
интеграла $I_n$ заметим, что тригонометрические функции ортогональны
и поэтому \equ{I_n = \ints{-\pi}^{\pi} \frac12\,dx - \rho_1^{(n)}
\ints{-\pi}^{\pi} \cos^2 x \,dx = \pi(1 - \rho_1^{(n)}).} Таким
образом, $\pi (1-\rho_1^{(n)})>0$, $\rho_1^{(n)}<1$.

Чтобы доказать существование последовательности положительных
многочленов вида (1), для которых $\rho_1^{(n)}\ra1$, $n\ra+\bes$
напомним, что $\hm{z}^2=z\cdot\ol{z}$ для любого $z\in\Cbb$ и
$\cos\al=\frac12 (e^{i\al} + e^{-i\al})$, $\al\in\R$, согласно
формуле Эйлера.

Рассмотрим положительную $\ph_n(z)$, \equ{\ph_n(x) = \hm{a_0 + a_1
e^{ix} + a_2 e^{2ix} \spl a_n e^{nix}}^2, \; x\in\R,} у которой
$a_0$, $a_1$, $a_2$, \ldots, $a_n$ --- действительные числа, и
покажем, что $\ph_n(x)$ --- чётный тригонометрический многочлен,
вычислив при этом его коэффициенты. Вспомнив, что квадрат модуля
комплексного числа равен произведению этого числа на сопряжённое и
$\ol{e^{ikx}} = \cos kx - i\sin kx = e^{-ikx}$, получим \ml{\ph_n(x)
= (a_0 + a_1 e^{ix} + a_2 e^{2ix} \spl a_n e^{nix}) (a_0 + a_1
\ol{e}^{ix} + a_2 \ol{e}^{2ix} \spl a_n \ol{e}^{nix}) = (a_0^2 +
a_1^2 \spl a_n^2) + (e^{ix} + e^{-ix}) \times\\ \times (a_0a_1 +
a_1a_2 \spl a_{n-1}a_n) + (e^{2ix} + e^{-2ix}) (a_0a_2 + a_1a_3 \spl
a_{n-2}a_n) \spl (e^{nix} + e^{-nix}) a_0a_n.}

Пользуясь формулой Эйлера, найдём \mla{2}{\ph_n(x) = (a_0^2 + a_1^2
\spl a_n^2) + 2(a_0a_1 + a_1a_2 \spl a_{n-1}a_n)\cos x + 2(a_0a_2 +
a_1a_3 \spl \\+a_{n-2}a_n)\cos2x \spl 2a_0a_n\cos nx.}

Положим в равенстве (2) $a_0=a_1=a_2 = \ldots a_n = 1$, получим
\equ{\ph_n(x) = (n+1) + 2n\cos x + 2(n-1)\cos2x \spl 2\cos nx,}
откуда \eqa{3}{\be_n(x) = \frac{\ph_n(x)}{2(n+1)} = \frac12 +
\frac{n}{n+1}\cos x + \frac{n-1}{n+1}\cos2x \spl \frac1{n+1}\cos nx
= \frac12 + \sumkun \frac{n-k+1}{n+1}\cos kx.}

Многочлены $\be_n(x)$ имеют такой же вид, как и многочлены (1), и
для них $\be_1^{(n)} = \frac{n}{n+1}\ra1$, $n\ra+\bes$.
\end{proof}

\subsubsection{Некоторые тождества}

\textbf{Утверждение 1.} \eqa{4}{\frac12 + \cos x + \cos2x \spl \cos
nx = \frac{\sin\hr{n+\frac12}x}{2\sin\frac{x}2}, \; x\in\R, \;
n\in\N.}


\begin{proof}
Пусть $C(x)$ --- сумма левой части в (1). Тогда
\ml{2C(x)\sin\frac{x}2 = \sin\frac{x}2 + 2\sin\frac{x}2 \cos x +
2\sin\frac{x}2\cos2x \spl 2\sin\frac{x}2\cos nx = \sin\frac{x}2 +
\hr{\sin\frac{3x}2 - \sin\frac{x}2} \spl \\+\hr{\sin\hr{n+\frac12}x
- \sin\hr{n-\frac12}x} = \sin\hr{n+\frac12}x} и равенство (4)
доказано для всех $x\in\R$, $n\in\N$, и $\sin\frac{x}2\ne0$.

Если $\sin\frac{x}2=0$, то $x=2k\pi$, $k\in\Z$, и $\liml{x\ra 2k\pi}
\frac{\sin\hr{n+\frac12}x}{2\sin\frac{x}2} = \liml{x\ra2k\pi}
\frac{\hr{n+\frac12}x}x = n+\frac12$ и $C(2k\pi) = n+\frac12$. Таким
образом, равенство (4) доказано полностью.
\end{proof}

\textbf{Утверждение 2.} \eqa{5}{\cos3x + \cos 5x \spl \cos(2n+1)x =
\cos(n+2)x \frac{\sin nx}{\sin x}, \; x\in\R, \; n\in\N.}


\begin{proof}
Обозначим $s(x)$ сумму в левой части (5). Тогда \ml{2s(x) \sin x =
2\cos3x \sin x + 2\cos5x\sin x \spl 2\cos(2n+1)x \sin x =
(\sin4x-\sin2x) \spl (\sin(2n+2)x - \sin2nx) = \\ = \sin(2n+2)x -
\sin2x = 2\cos(n+2)x\sin x, \; x\in\R, \; n\in\N.}
\end{proof}

\subsubsection{}

В формуле (2) для $\ph_n(x)$ положим $a_k = \sin\frac{k+1}{n+2}\pi$,
$k=\ol{0,n}$. Тогда, с учётом утверждения 1, \ml{A_0 = a_0^2 + a_1^2
\spl a_n^2 = \sin^2\frac{\pi}{n+2} \spl \sin^2 \frac{n+1}{n+2}\pi =
\frac12 - \frac12 \cos\frac{2\pi}{n+2} \spl \frac12 -
\cos\frac{2(n+1)}{n+2}\pi =\\= \frac12(n+1) - \frac12 \hr{\frac12 +
\cos \frac{2\pi}{n+2} \spl \cos(n+1) \frac{2\pi}{n+2} - \frac12} =
\frac{n+1}2 - \frac12 \hs{\frac{\sin\hr{n+1+\frac12}
\frac{2\pi}{n+2}}{2\sin\frac{\pi}{n+2}} - \frac12} =\\= \frac{n+1}2
- \frac12 \hs{\frac{\sin\hr{2\pi -
\frac{\pi}{n+2}}}{2\sin\frac{\pi}{n+2}} - \frac12} = \frac{n+1}2 +
\frac12 = \frac{n+2}2.}

Вычислим $A)1 - 2(a_0a_1 + a_1a_2 \spl a_{n-1} a_n)$, используя
утверждение 2. Имеем \ml{A_1 = 2\sin\frac{\pi}{n+2}
\sin\frac{2\pi}{n+2} + 2\sin\frac{2\pi}{n+2} \sin\frac{3\pi}{n+2}
\spl 2\sin\frac{n}{n+2}\pi \sin\frac{n+1}{n+2}\pi =
\cos\frac{\pi}{n+2}- \cos 3\frac{\pi}{n+2} +\\+ \cos \frac{\pi}{n+2}
- \cos 5\frac{\pi}{n+2} \spl \cos\frac{\pi}{n+2} - \cos(2n+1)
\frac{\pi}{n+2} = n\cos\frac{\pi}{n+2} - \frac{\cos(n+2)
\frac{\pi}{n+2} \sin \frac{n\pi}{n+2}}{\sin\frac{\pi}{n+2}} =
n\cos\frac{\pi}{n+2} +\\+ \frac{\sin\hr{\pi -
\frac{2\pi}{n+2}}}{\sin\frac{\pi}{n+2}} = n\cos\frac{\pi}{n+2} +
2\cos\frac{\pi}{n+2} = (n+2)\cos\frac{\pi}{n+2}.}

Таким образом, \equ{\ph_n(x) = \frac{n+2}2 +
(n+2)\cos\frac{\pi}{n+2}\cos x + A_2 \cos2x \spl A_n \cos nx} и
чётный положительный многочлен \eqa{6}{\al_n(x) =
\frac{\ph_n(x)}{n+2} = \frac12 + \cos\frac{\pi}{n+2} \cos x +
\rho_2^{(n)} \cos2x \spl \rho_n^{(n)}\cos nx} имеет такой же вид,
как и многочлен (1).

У положительных многочленов $\al_n(x)$ последовательность
$(\rho^{(n)}_1)$, $\rho_1^{(n)} = \cos\frac{\pi}{n+2}$, стремится к
единице быстрее, чем такая же последовательность у многочленов
$\be_n(x)$.

Действительно, в случае многочленов $\be_n(x)$ \equ{1 - \rho^{(n)}_1
= 1 - \frac{n}{n+1} = \frac1{n+1},} а в случае многочленов
$\al_n(x)$, \equ{1 - \rho_1^{(n)} = 1 - \cos\frac{\pi}{n+2} =
2\sin^2 \frac{\pi}{2(n+2)} < 2 \frac{\pi^2}{4(n+2)^2} <
\frac5{(n+2)^2}.}

Это обстоятельство в дальнейшем окажется весьма полезным.

\subsection{Приближение функций тригонометрическими многочленами}
\subsubsection{Свёртка тригонометрического многочлена и
интегрируемой функции}

Пусть функции $f$ интегрируема на $[-\pi,\pi]$. Функция
\eqa{1}{\si_n(x) = \si_n(f;x) = \frac1{\pi} \ints{-\pi}^{\pi} f(t)
u_n(t-x)\,dt = \frac1{\pi} \ints{-\pi}^{\pi} f(t) \hs{\frac12 +
\sumkun \rho_k^{(n)} \cos k(t-x)}\,dt} называется свёрткой функции
$f$ и тригонометрического многочлена $u_n(x)$, задаваемого формулой
(1) в параграфе 2.

\begin{stm*}
Функция $\si_n(x)$ есть тригонометрический многочлен порядка не выше
$n$ для любой функции $f\in\Rc[a,b]$.
\end{stm*}

\begin{proof}
Согласно (1) и свойству линейности определённого интеграла,
\mla{2}{\si_n = \frac1{2\pi} \ints{-\pi}^{\pi} f(t)\,dt + \sumkun
\frac1{\pi} \ints{-\pi}^{\pi} \rho_k^{(n)} f(t) \cos k(t-x)\,dt =
\frac1{2\pi} \ints{-\pi}^{\pi} f(t)\,dt  + \sumkun\frac1{\pi}
\ints{-\pi}^{\pi} \rho_k^{(n)} [f(t)\cos kt \cos kx + f(t) \sin kt
\sin kx]\,dt =\\= \frac1{2\pi} \ints{-\pi}^{\pi} f(t)\,dt + \sumkun
\rho_k^{(n)} \hs{\cos kx \frac1{\pi} \ints{-\pi}^{\pi} f(t)\cos
kt\,dt + \sin kx \frac1{\pi} \ints{-\pi}^{\pi} f(t)\sin kt\, dt} =
\\ = \frac{a_0}2 + \sumkun \rho^{(n)}_k (a_k\cos kx + b_k \sin kx),}
где числа \eqa{3}{a_0 = \frac1{\pi} \ints{-\pi}^{\pi} f(t)\,dt, \;
a_k = \frac1{\pi} \ints{-\pi}^{\pi} f(t)\cos kt\,dt, \; b_k =
\frac1{\pi} \ints{-\pi}^{\pi} f(t)\sin kt\,dt, \; k\in\N,}
называются коэффициентами Фурье функции $f\in\Rc[-\pi,\pi]$.
\end{proof}

Положим в (1) и (2) все $\rho_k^{(n)}=1$, $k=\ol{1,n}$. Тогда сумма
\eqa{4}{s_n(f;x) = \frac{a_0}2 + \sumkun (a_k\cos kx + b_k\sin kx) =
\frac1{\pi} \ints{-\pi}^{\pi} f(t) \hs{\frac12 + \sumkun \cos
k(t-x)}\,dt = \frac1{\pi} \ints{-\pi}^{\pi} f(t)
\frac{\sin\hr{n+\frac12}(t-x)}{2\sin\frac{t-x}2}\,dt} для любой
функции $f\in\Rc[-\pi,\pi]$, где использовано утверждение 1 из
пункта 2.2, параграф 2.

\subsubsection{Свёртка с периодической функцией}

В этом пункте считаем функцию $f$ определённой на всём $\R$ и
$2\pi$--периодической; предположим также, что $f\in\Rc[-\pi,\pi]$.

\begin{lem}
1 Для любого $x\in\R$ справедливо \eqa{5}{\ints{-\pi+x}^{\pi+x}
f(t)\,dt = \ints{-\pi}^{\pi} f(t)\,dt.}
\end{lem}

\begin{proof}
Из условий леммы и свойства аддитивности определённого интеграла
следует, что $f\in\Rc[a,b]$ для любых $a,b\in\R$ и
\equ{\ints{-\pi+x}^{\pi+x} f(t)\,dt = \ints{-\pi+x}^{\pi} f(t)\,dt +
\ints{-\pi}^{\pi} f(t)\,dt + \ints{\pi}^{\pi+x} f(t)\,dt = I_1 + I_2
+ I_3.} Последнее равносильно (5), поскольку \equ{I_3 =
\ints{\pi}^{\pi+x} f(t)\,dt = \mbmat{t = u+2\pi \\ dt=du} =
\ints{-\pi}^{-\pi+x} f(u+2\pi)\,du = \ints{-\pi}^{-\pi+x} f(u)\,du =
- I_1.}
\end{proof}

Согласно формуле (1) и лемме 1, \equ{\si_n(f;x) = \frac1{\pi}
\ints{-\pi+x}^{\pi+x} f(y+x) u_n(y)\,dy,} где $y=t-x$, и
\eqa{6}{\si_n(f;x) = \frac1{\pi} \ints{-\pi}^{\pi} f(x+y) u_n(y)
\,dy, \; n\in\N.}

Если $f(x)\equiv1$ в (6), то, на основании ортогональности
тригонометрической системы и свойства линейности определённого
интеграла, \eqa{7}{\si_n(1;x) = \frac1{\pi} \ints{-\pi}^{\pi}
u_n(y)\,dy = \frac1{\pi} \ints{-\pi}^{\pi} \hs{\frac12 + \sumkun
\rho_k^{(n)} \cos ky}\,dy = 1,} и следовательно, \eqa{8}{f(x) =
\frac1{\pi} \ints{-\pi}^{\pi} f(x) u_n(y)\,dy, \; x\in\R.}

Так как $u_n(x)\ge0$, $x\in\R$, то из (6) и (7) следует оценка
\eqa{9}{\hm{f(x) - \si_n(f;x)} \le \frac1{\pi} \ints{-\pi}^{\pi}
\hm{f(x+y) - f(x)} u_n(y)\,dy, \; n\in\N.}

\subsubsection{Вспомогательные утверждения}
\begin{lem}
2 Если функции $f,g\in\Rc[a,b]$, то \eqa{10}{\hr{\ints{a}^b
f(x)g(x)\,dx}^2 \le \ints{a}^b f^2(x)\,dx \cdot \ints{a}^b
g^2(x)\,dx.}
\end{lem}

\begin{proof}
Для всех $z\in\R$ справедливо \equ{0\le \ints{a}^b (f(x) -
zg(x))^2\,dx = z^2 \ints{a}^b g^2(x)\,dx - 2z\ints{a}^b f(x)g(x)\,dx
+ \ints{a}^b f^2(x)\,dx,} и квадратный трёхчлен относительно $z$
обязан иметь неположительный дискриминант $D$, \equ{D =
\hr{\ints{a}^b f(x)g(x)\,dx}^2 - \hr{\ints{a}^b g^2(x)\,dx}
\hr{\ints{a}^b f^2(x)\,dx}\le0,} что равносильно (10).
\end{proof}

\begin{lem}
3 Для любого $x$, $0\le x\le \frac{\pi}2$, справедливо $x\le
\frac{\pi}2\sin x$.
\end{lem}

\begin{proof}
Функция $f(x) = x - \frac{\pi}2\sin x$ имеет $f''(x) =
\frac{\pi}2\sin x>0$, $0<x<\frac{\pi}2$, и $f(0) =
f\hr{\frac{\pi}2}=0$. Поэтому $x-\frac{\pi}2\sin x = f(x)\le0$ для
$0\le x \le \frac{\pi}2$.
\end{proof}

\begin{lem}
4 Если $u_n(x) = \frac12 + \sumkun \rho_k^{(n)} \cos kx \ge 0$, то
\eqa{11}{\frac1{\pi} \ints{-\pi}^{\pi} \hm{y} u_n(y)\,dy \le
\frac{\pi}{\sqrt2} \sqrt{1-\rho_1^{(n)}}.}
\end{lem}

\begin{proof}
Используя оценку из леммы 3 и формулы (7), (10), имеем
\ml{\frac1{\pi} \ints{-\pi}^{\pi} \hm{y} u_n(y)\,dy = \frac2{\pi}
\ints{-\pi}^{\pi} \hm{\frac{y}2} u_n(y)\,dy \le \ints{-\pi}^{\pi}
\hm{\sin\frac{y}2} u_n(y)\,dy \le \hr{\ints{-\pi}^{\pi} \sin^2
\frac{y}2 u_n(y)\,dy}^{\frac12} \cdot \hr{\ints{-\pi}^{\pi}
u_n(y)\,dy}^{\frac12} = \\ =\frac{\sqrt\pi}{\sqrt2}
\hr{\ints{-\pi}^{\pi} (1-\cos y) u_n(y)\,dy}^{\frac12}.}
\end{proof}

В доказательстве теоремы 1 пункта 2.1, параграф 2, установлено, что
\equ{ \ints{-\pi}^{\pi} (1-\cos y) u_n(y)\,dy = \pi (1 -
\rho_1^{(n)}),} так что, окончательно, получаем неравенство (11).

\subsubsection{Оценка приближения непрерывных периодических функций
тригонометрическими многочленами}

\begin{theorem}
Если функция $f$ непрерывна и $2\pi$--периодическая на $\R$ и
$\om(\de)$ её модуль непрерывности на $[-2\pi,2\pi]$, то для любого
положительного многочлена $u_n(x) = \frac12 + \sumkun \rho_k^{(n)}
\cos kx$ и любого числа $m>0$ оценка \eqa{12}{\hm{\si_n (f;x) -
f(x)} \le \om\hr{\frac1m} \hr{1+\frac{m\pi}{\sqrt2}
\sqrt{1-\rho_1^{(n)}}}, \; n\in\N,} справедлива для всех
$x\in[-\pi,\pi]$.
\end{theorem}

\begin{proof}
Рассмотрим произвольное $m>0$. Согласно (9) и следствию к теореме 1
пункта 1.1, параграф 1, имеем \ml{\hm{f(x) - \si_n(f;x)} \le
\frac1{\pi} \ints{-\pi}^{\pi} \om(\hm{y}) u_n(y)\,dy = \frac1{\pi}
\ints{-\pi}^{\pi} \om\hr{\frac1m \cdot m\hm{y}} u_n(y)\,dy \le
\frac1{\pi} \ints{-\pi}^{\pi} (m\hm{y} + 1) \om\hr{\frac1m}
u_n(y)\,dy = \\=\frac{m}{\pi} \om\hr{\frac1m} \ints{-\pi}^{\pi}
\hm{y} u_n(y)\,dy + \frac1{\pi} \om\hr{\frac1m} \ints{-\pi}^{\pi}
u_n(y)\,dy,} откуда, с учётом (11) и (7) следует, что \equ{\hm{f(x)
- \si_n(f;x)} \le \frac{m\pi}{\sqrt2} \om\hr{\frac1m}
\sqrt{1-\rho_1^{(n)}} + \om\hr{\frac1m}, \; n\in\N;} то есть,
следует оценка (12).
\end{proof}

\begin{imp*}
В предположениях предыдущей теоремы, \eqa{13}{\hm{f(x) - \si_n(f;x)}
\le \om\hr{\sqrt{1 - \rho_1^{(n)}}} \hr{1+\frac{\pi}{\sqrt2}}, \;
x\in [-\pi,\pi], \; n\in\N.}
\end{imp*}

\begin{proof}
В формуле (12) выбираем $m = \frac1{\sqrt{1 - \rho_1^{(n)}}}$.
\end{proof}

$\hr{m\sqrt{1 - \rho_1^{(n)}}=1}$.

\subsection{Равномерное приближение непрерывных функций
многочленами}

\subsubsection{Теорема Фейера о равномерной аппроксимации
непрерывной периодической функции тригонометрическими многочленами}

Свёртки функции $f\in\Rc[-\pi,\pi]$ с положительными многочленами
$\be_n(x)$, также являющиеся по результатам пункта 3.1
тригонометрическими многочленами, называют многочленами Фейера и
обозначают $F_n(f;x)$. Поскольку \equ{\be_n(x) = \frac12 + \sumkun
\frac{n-k+1}{n+1} \cos kx, \; n\in\N,} то согласно формуле (2),
пункт 3.1, \equ{ F_n(f;x) = \frac{a_0}2 + \sumkun \frac{n-k+1}{n+1}
(a_k\cos kx + b_k \sin kx), \; n\in\N,} где $a_0$, $a_k$, $b_k$,
$k\in\N$ --- коэффициенты Фурье функции $f$, определяемые формулами
(3), пункт 3.1.

\begin{theorem}
Для любой непрерывной и $2\pi$--периодической функции $f$
справедлива оценка \equ{\maxl{x\in[-\pi,\pi]} \hm{f(x) - F_n(f;x)}
\le c \om_f\hr{\frac1{\sqrt{n+1}}}, \; n\in\N,} в которой $0<c<1 +
\frac{\pi}{\sqrt2}$ и $\om_f(\de)$ --- модуль непрерывности функции
$f$ на $[-2\pi,2\pi]$, и следовательно, для произвольного числа
$\ep>0$ существует такое $N\in\N$, $N=N(\ep)$, что
$\maxl{x\in[-\pi,\pi]} \hm{f(x) - F_n(f;x)} < \ep$ для всех $n\ge
N$; то есть, $F_n(f;x) \rra f(x)$ на $[-\pi,\pi]$ (и на всём $\R$).
\end{theorem}

\begin{proof}
Применяем формулу (13) из следствия к теореме пункта 3.4, для
$\si_n(f;x) = F_n(f;x)$, имеющих $\rho_1^{(n)} = \frac{n}{n+1}$, и
замечаем, что, по теореме Гейне--Кантора,
$\om_f\hr{\frac1{\sqrt{n+1}}}\ra0$ при $n\ra+\bes$.
\end{proof}

\subsubsection{Две теоремы Вейерштрасса}

Изложенная выше теорема Фейера была доказана в 1904 г. Однако начало
теории приближений непрерывных функций многочленами положено К.
Вейерштрассом в 1885 г.

\textbf{Первая теорема Вейерштрасса.} Для произвольной непрерывной и
$2\pi$--периодической функции $f$ и любого числа $\ep>0$ существует
такой тригонометрический многочлен $T(x)$, что $\hm{f(x) -
T(x)}<\ep$ для всех $x\in[-\pi,\pi]$ (и всех $x\in\R$).

\begin{proof}
В принятом нами изложении выбираем $T(x) = F_n(f;x)$.
\end{proof}

\textbf{Вторая теорема Вейерштрасса.} Для любой непрерывной функции
$f$ на невырожденном отрезке $[a,b]$, $-\bes<a<b<+\bes$, и любого
числа $\ep>0$ можно указать такой алгебраический многочлен $P(x)$,
для которого $\hm{f(x) - P(x)}<\ep$ справедливо для всех
$x\in[a,b]$.

\begin{proof}
Подстановка $x = a + \frac{x'}{\pi}(b-a)$ переводит $[a,b]$ в
$[0,\pi]$. Так как при этой подстановке алгебраические многочлены
переходят в алгебраические многочлены (с сохранением своих
степеней), то задача сводится к случаю непрерывных функций на
отрезке $[0,\pi]$. Поэтому рассмотрим произвольную функцию
$f\in\Cc[0,\pi]$, и образуем новую функцию $f^*$ посредством условия
$f^*(x) = f(-x)$, если $x\in[-\pi,0]$, и $f^*(x) = f(x)$, если
$x\in[0,\pi]$. Тогда $f^*\in\Cc[-\pi,\pi]$ (в точках $x\ne0$ она
непрерывна как композиция непрерывных функций; в $x=0$ имеем
$\liml{x\ra-0}f^*(x) = \liml{x\ra-0} f(-x) = \liml{x\ra+0} f(x) =
\liml{x\ra+0} f^*(x) = f(0) = f^*(0)$) и $f^*(-\pi) = f^*(\pi)$.
Доопределим, далее, $f^*$ на всю числовую прямую как непрерывную
$2\pi$--периодическую функцию. Рассмотрим произвольное число
$\ep>0$. Согласно первой теореме Вейерштрасса, существует такой
тригонометрический многочлен $T(x)$, что $\hm{f^*(x) - T(x)} <
\frac{\ep}2$ для всех $x\in[-\pi,\pi]$. Если заменить каждую
тригонометрическую функцию, входящую в $T(x)$, её тейлоровским
разложением по степеням $x$, то $T(x)$ представится в виде суммы
сходящегося всюду на $\R$ степенного ряда $\sumnzi c_nx^n = T(x)$,
$x\in\R$. На $[-\pi,\pi]$ этот ряд сходится равномерно, и поэтому,
если выбрать в качестве алгебраического многочлена $P(x)$ некоторую
частную сумму этого ряда с достаточно большим индексом $n$, то
получим неравенство $\hm{T(x) - P(x)}<\frac{\ep}2$ для всех
$x\in[-\pi,\pi]$. Следовательно, $\hm{f^*(x) - P(x)} \le \hm{f^*(x)
- T(x)} + \hm{T(x) - P(x)} < \ep$ для всех $x\in[-\pi,\pi]$; в
частности, $\hm{f(x) - P(x)}<\ep$ для всех $x\in[0,\pi]$.
\end{proof}

\begin{imp*}
Если функция $f\in\Cc[a,b]$, то существует последовательность
алгебраических многочленов $(P_n(x))$, что $P_n(x)\rra f(x)$ на
$[a,b]$ и $f(x) = P_1(x) + \sumnui [P_{n+1}(x) - P_n(x)]$,
$x\in[a,b]$, и ряд равномерно сходится на $[a,b]$.
\end{imp*}

\subsubsection{Наилучшее приближение функций}

Постановка задачи принадлежит П.Л. Чебышёву.

Обозначим $E_n(f) = \infl{P_n} \hc{\maxl{x\in[a,b]} \hm{f(x) -
P_n(x)} \bvl f\in\Cc[a,b], \; \deg P_n \le n}$, $n\in\N$.

\textbf{Теорема Чебышёва.} Существует единственный алгебраический
многочлен $\ol{p}_n(x)$, $\deg p_n = n$, что $E_n(f) =
\maxl{x\in[a,b]} \hm{f(x) - \ol{p}_n(x)}$, $n\in\N$. Более того,
числа $E_n(0) = \inf\hc{\maxl{x\in[-1,1]} \hm{P_n(x)} \bvl \deg P_n
\le n}$ равны $E_n(0) = \frac1{2^{n-1}}$, $n\in\N$, и алгебраические
многочлены $\ol{p}_n(x) = \frac1{2^{n-1}} \cos(n\arccos x)$,
$x\in[-1,1]$.

Аналогично, если $E^*_n(f) = \inf \hc{\maxl{[-\pi,\pi]} \hm{f(x) -
T_n(x)} \bvl f \mbox{ непрерывная и } 2\pi\mbox{--периодическая, }
\deg T_n\le n}$, то существует единственный тригонометрический
многочлен $t_n(x)$, $\det t_n = n$, что $E^*_n(f) =
\maxl{[-\pi,\pi]} \hm{f(x) - t_n(x)}$, $n\in\N$. Согласно теореме
Фейера, $E^*_n(f) \le c \om_f \hr{\frac1{\sqrt{n+1}}}$, $0<c\le 1 +
\frac{\pi}{\sqrt2}$. Существует более точная оценка.

\textbf{Первая теорема Джексона.} $E^*_n(f) \le 6 \om_f
\hr{\frac1{n+2}}$, $n\in\N$.

\begin{proof}
Рассмотрим тригонометрические многочлены \equ{A_n(f;x) = \frac1{\pi}
\ints{-\pi}^{\pi} f(t) \al_n (t-x)\,dt, \; n\in\N,} --- многочлены
Джексона, где \eqa{1}{\al_n(x) = \frac12 + \cos \frac{\pi}{n+2} \cos
x + \rho_2^{(n)} \cos 2x \spl \rho_n^{(n)}\cos nx.} Тогда, на
основании оценки (12) из пункта 3.4 и определения величины
$E^*_n(f)$, имеем \eqa{2}{E^*_n(f) \le \maxl{[-\pi,\pi]} \hm{f(x) -
A_n(f;x)} \le \om_f \hr{\frac1m} \hs{1 + \frac{\pi m}{\sqrt2}
\sqrt{1 - \rho_1^{(n)}}}} для любого $m>0$. Поэтому, с учётом (1),
имеем \equ{1 + \frac{\pi m}{\sqrt2} \sqrt{1 - \rho_1^{(n)}} = 1 +
\pi(n+2) \sin \frac{\pi}{2(n+2)} < 1 + \pi(n+2) \frac{\pi}{2(n+2)}=
1 + \frac{\pi^2}2 < 6}. На основании (2) и (3), в которых $m=n+2$,
получаем утверждение теоремы.
\end{proof}

В частности, если $\om_f(\de) \le M\de^{\al}$, $0 < \al<1$; то есть,
функция $f$ принадлежит классу Гёльдера порядка $\al$, $0<\al<1$, то
$E^*_n(f) \le c \frac1{n^{\al}}$, $c=6M$, $n\in\N$.

\textbf{Теорема С.Н. Бернштейна} \emph{(обратная)}. Если существует
$c>0$, что $E^*_n(f) \le c \frac1{n^{\al}}$, $0<\al<1$, $n\in\N$, то
$\om_f(\de) \le M\de^{\al}$ с некоторой $M>0$.

Таким образом, скорость убывания величины $E^*_n(f)$ в теореме
Джексона неулучшаема.

\pagebreak
\section{Ряды Фурье}
\subsection{Предварительные сведения}
\subsubsection{Определения и примеры}

Рассмотрим произвольную функцию $f$, интегрируемую на $[-\pi,\pi]$,
и числа \eqa{1}{a_n = \frac1{\pi} \ints{-\pi}^{\pi} f(x)\cos nx\,dx,
\; n\in\N_0 = \N\cup\hc{0},} \eqa{2}{b_n = \frac1{\pi}
\ints{-\pi}^{\pi} f(x)\sin nx\,dx, \; n\in\N,} --- коэффициенты
Фурье функции $f$.

Функциональный ряд \eqa{3}{\frac{a_0}2 + \sumnui (a_n\cos nx +
b_n\sin nx), \; x\in\R,} называют тригонометрическим рядом Фурье
функции $f$. Таким образом, каждой функции $f\in\Rc[-\pi,\pi]$,
$x\in[-\pi,\pi]$, ставится в соответствие её ряд Фурье (3),
определённый на всём $\R$.

\begin{stm}
Если тригонометрический ряд \eqa{4}{A_0 + \sumnui (A_n\cos nx +
B_n\sin nx),} у которого $A_0$, $A_n$, $B_n\in\R$, $n\in\N$,
равномерно сходится на $[-\pi,\pi]$ к сумме $f(x)$, то ряд (4)
является рядом Фурье своей суммы $f(x)$.
\end{stm}

\begin{proof}
Поскольку, по условию, \equ{f(x) = A_0 + \sumnui (A_n\cos nx + B_n
\sin nx), \; x\in[-\pi,\pi],} и ряд (4) равномерно сходится на
$[-\pi,\pi]$, то $f$ непрерывна на $[-\pi,\pi]$ и ряд допускает
почленное интегрирование (см. свойства равномерно сходящихся рядов с
непрерывными членами). Тогда, используя свойство ортогональности
тригонометрической системы, имеем \equ{ a_0 = \frac1{\pi}
\ints{-\pi}^{\pi} f(x)\,dx = \frac1{\pi} \ints{-\pi}^{\pi} A_0\,dx +
\sumnui \frac1{\pi} \ints{-\pi}^{\pi} (A_n\cos nx + B_n\sin nx)\,dx
= 2A_0,} откуда $A_0 = \frac{a_0}2$, и \mln{a_k = \frac1{\pi}
\ints{-\pi}^{\pi} f(x)\cos kx\,dx = \frac1{\pi} \ints{-\pi}^{\pi}
A_0\cos kx\,dx + \sumnui \hs{\frac1{\pi} \ints{-\pi}^{\pi} A_n \cos
nx \cos kx\,dx + \frac1{\pi} \ints{-\pi}^{\pi} B_n\sin nx \cos
kx\,dx} = \\ = \frac1{\pi} \ints{-\pi}^{\pi} A_k\cos^2 kx \,dx =
A_k, \; k\in\N.}

Аналогично доказывается, что $B_k=b_k$, $k\in\N$.
\end{proof}

\begin{imp}
Произвольный тригонометрический многочлен $T(x) = \frac{c_0}2 +
\sumkun(c_k\cos kx + d_k\sin kx)$ является рядом Фурье функции
$T(x)$; то есть, $c_k = \frac1{\pi} \ints{-\pi}^{\pi} T(x) \cos
kx\,dx$, $k=\ol{0,n}$, и $d_k = \frac1{\pi} \ints{-\pi}^{\pi} T(x)
\sin kx \,dx$, $k=\ol{1,n}$.
\end{imp}

\begin{proof}
Любой $T(x)$ можно считать тригонометрическим рядом с нулевыми
коэффициентами, начиная с некоторого индекса. Такой ряд равномерно
сходится на всём $\R$.
\end{proof}

\subsubsection{Частные суммы}

Пусть $f\in\Rc[-\pi,\pi]$ и $a_0$, $a_n$, $b_n$, $n\in\N$ --- её
коэффициенты Фурье. В параграфе 3 предыдущей главы 4 доказаны
формулы \equ{ s_n(f;x) = \frac1{\pi} \ints{-\pi}^{\pi} f(t)
\frac{\sin \hr{n+\frac12}(t-x)}{2\sin\frac{t-x}2}\,dt, \; n\in\N;}
если $n=0$, то \equ{s_0(f;x) = \frac1{\pi} \ints{-\pi}^{\pi} f(t)
\frac{\sin\frac{t-x}2}{2\sin\frac{t-x}2}\,dt = \frac1{2\pi}
\ints{-\pi}^{\pi} f(t)\,dt = \frac{a_0}2.}

Пусть теперь ещё $f(-\pi) = f(\pi)$, так что $f$ продолжается на всё
$\R$ как $2\pi$--периодическая функция. Согласно лемме 1 параграфа 3
главы 4, \equ{s_n(f;x) = \frac1{\pi} \ints{-\pi+x}^{\pi-x} f(x+y)
\frac{\sin\hr{n+\frac12}y}{2\sin\frac{y}2}\,dy = \frac1{\pi}
\ints{-\pi}^{\pi} f(x+y) \frac{\sin\hr{n+\frac12}y}{2\sin\frac{y}2}
\,dy} и далее, \mla{5}{s_n(f;x) = \frac1{\pi} \ints{-\pi}^0 f(x+y)
\frac{\sin\hr{n+\frac12}y}{2\sin\frac{y}2}\,dy + \frac1{\pi}
\ints0^{\pi} f(x+y) \frac{\sin\hr{n+\frac12}y}{2\sin\frac{y}2}\,dy =
\frac1{\pi} \ints0^{\pi} \frac{f(x+t) + f(x-t)}2 \frac{\sin
\hr{n+\frac12}t}{\sin\frac{t}2}\,dt, \; n\in\N,} где $y=t$ во втором
слагаемом суммы и $y=-t$ в первом слагаемом, а функция $D(y) =
\frac{\sin\hr{n+\frac12}y}{\sin\frac{y}2}$ --- чётная, $D(-y)=D(y)$.
Функцию $D(t)$ в формуле (5) называют ядром Дирихле.

\subsubsection{Частные суммы ряда Фурье и многочлены Фейера}

Пусть $f\in\Rc[-\pi,\pi]$ и $f(-\pi)=f(\pi)$. Тогда \equ{ a_n\cos nx
+ b_n \sin nx = s_n - s_{n-1}, \; n\in\N,} где $s_n = s_n(f;x)$,
$n\in\N$, и $s_0 = s_0(f;x)$. Для многочленов Фейера $F_n(f;x)$,
$n\in\N$, функции $f$ справедлива формула \equ{F_n(f;x) =
\frac{a_0}2 + \sumkun \frac{n-k+1}{n+1} (a_k \cos kx + b_k \sin kx),
\; n\in\N.} Поэтому \ml{F_n(f;x) = s_0 + \sumkun \frac{n-k+1}{n+1}
(s_k - s_{k-1}) = \frac{(n+1) s_0 + n(s_1-s_0) + (n-1) (s_2-s_1)
\spl (s_n - s_{n-1})}{n+1} = \\ = \frac{s_0+s_1\spl s_n}{n+1} =
\frac{\sums{k=0}^n s_k(f;x)}{n+1}, \; n\in\N} --- среднее
арифметическое частных сумм ряда Фурье.

\subsubsection{Теорема Коши}

\begin{theorem}
Если последовательность $(s_n)$, $n\in\N_0$, имеет $\liml{n\ra+\bes}
s_n = s$, то последовательность $(f_n)$, $f_n = \frac{s_0 + s_1 \spl
s_n}{n+1}$, $n\in\N$, также имеет $\liml{n\ra+\bes} f_n = s$.
\end{theorem}

\begin{proof}
$f_n - s = \frac{(s_0-s) + (s_1 - s) \spl (s_n - s)}{n+1}$,
$n\in\N_0$. Рассмотрим произвольное число $\ep>0$. Так как
$\liml{n\ra+\bes} s_n = s$, то существует такое $N\in\N$, что
$\hm{s_n - s}<\frac{\ep}2$ для всех $n>N$. Поэтому \ml{ \hm{f_n - s}
\le \frac{\hm{s_0-s} \spl \hm{s_N-s}}{n+1} + \frac{\hm{s_{N+1} - s}
\spl \hm{s_n-s}}{n+1} < \frac{\hm{s_0-s} \spl \hm{s_N-s}}{n+1} +
\frac{\ep}2 \frac{n-N}{n+1} <\\< \frac{\hm{s_0 - s} \spl
\hm{s_N-s}}{n+1} + \frac{\ep}2, \; n>N.}

Кроме того, существует $N_2\in\N$, что \equ{\frac{\hm{s_0-s} \spl
\hm{s_N-s}}{n+1} < \frac{\ep}2} для всех $n>N_{\ep}$ и
следовательно, $\hm{f_n - s} < \frac{\ep}2 + \frac{\ep}2 = \ep$ для
всех $n>\max(N,N_{\ep})$; то есть, $s = \liml{n\ra+\bes} f_n$.
\end{proof}


\subsubsection{Необходимое условие сходимости ряда Фурье непрерывной
и $2\pi$--периодической функции}

\begin{theorem}
Если ряд Фурье непрерывной и $2\pi$--периодической функции $f$
сходится в точке $x\in\R$, то его сумма равна $f(x)$.
\end{theorem}

\begin{proof}
Многочлены Фейера $F_n(f;x)$, $n\in\N$, равномерно сходятся к
функции $f(x)$ на $\R$, а последовательность $(s_n(f;x))$ сходится в
точке $x\in\R$. По теореме Коши, $\liml{n\ra+\bes} s_n(f;x) = f(x)$.
\end{proof}

\textbf{Важное методологическое замечание}

В предыдущем параграфе 1 отмечено, что если функция (конечно,
непрерывная) может быть представлена равномерно сходящимся
тригонометрическим рядом, то ряд этот непременно есть её ряд Фурье.
Здесь мы обнаруживаем полную аналогию со степенными рядами.
Степенной ряд есть ряд Тейлора своей суммы и равномерно сходящийся
тригонометрический ряд есть ряд Фурье своей суммы.

С другой стороны, было отмечено, что ряд Тейлора функции $f$ может
сходиться к функции, отличной от $f$. Согласно доказанной теореме,
ряд Фурье непрерывной и периодической функции $f$, если он сходится
в какой--нибудь точке $x\in\R$, имеет сумму, непременно равную
значению $f(x)$ функции $f$ в этой точке.

\subsubsection{Примеры. Коэффициенты Фурье чётных и нечётных
функций}

\begin{lem*}
Если функции $f$ и $g$ интегрируемы на $[-a,a]$, $a>0$, и $f$ ---
чётная, а $g$ --- нечётная, то $\ints{-a}^a f(x)\,dx = 2\ints0^a
f(x)\,dx$, $\ints{-a}^a g(x)\,dx=0$.
\end{lem*}

Поэтому, если функция $f\in\Rc[-\pi,\pi]$ и $f$ --- чётная, то $a_n
= \frac2{\pi} \ints0^{\pi} f(x)\cos nx\,dx$, $n\in\N_0$ и $b_n=0$,
$n\in\N$, а если $f$ --- нечётная, то $a_n=0$, $n\in\N_0$, а $b_n =
\frac2{\pi}\ints0^{\pi} f(x)\sin nx\,dx$, $n\in\N$.

\begin{ex}
Рассмотрим $2\pi$--периодическую функцию $f$, для которой
$f(x)=\hm{x}$, когда $x\in[-\pi,\pi]$; $f(-\pi)=f(\pi)=\pi$. Тогда
все $b_n=0$, $n\in\N$, и \ml{a_0 = \frac2{\pi} \ints0^{\pi} f(x)\,dx
= \frac2{\pi} \ints0^{\pi} x\,dx = \pi, \; a_n = \frac2{\pi}
\ints0^{\pi} f(x) \cos nx\,dx = \frac2{\pi} \ints0^{\pi} x\cos nx\,
dx = \frac2{\pi} \hs{\frac{x\sin nx}n + \frac{\cos nx}{n^2}}^{\pi}_0
= \\ = \frac2{\pi n^2} \hs{(-1)^n - 1}, \; n\in\N.} Поэтому, $a_0 =
\pi$, $a_{2k}=0$, $a_{2k-1}= - \frac4{\pi(2k-1)^2}$, $k\in\N$. Кроме
того, \equ{ \hm{x} = \frac{\pi}2 - \frac4{\pi} \sumkui
\frac{\cos(2k-1) x}{(2k-1)^2}, \; x\in[-\pi,\pi],} поскольку
тригонометрический ряд Фурье в правой части формулы (..) равномерно
сходится на $\R$ по признаку Вейерштрасса $\hr{\mbox{сходится
мажорирующий ряд }\sum \frac1{(2k-1)^2}}$. В $x=0$ имеем \equ{ 0 =
\frac{\pi}2 - \frac4{\pi}\sumkui \frac1{(2k-1)^2} \mbox{ и } \sumkui
\frac1{(2k-1)^2} = \frac{\pi^2}8.}

Поэтому, \equ{A = \sumnui \frac1{n^2} = \sumkui \frac1{(2k-1)^2} +
\sumkui \frac1{(2k)^2} = \frac{\pi^2}8 + \frac14 \sumkui \frac1{k^2}
= \frac{\pi^2}8 + \frac14A,} откуда $A = \frac{\pi^2}6$ и
$\sumnui\frac1{n^2} = \frac{\pi^2}6.$
\end{ex}

\subsection{Равномерная сходимость рядов Фурье}
\subsubsection{Константы Лебега}

Пусть функция $f\in\Rc[-\pi,\pi]$ и $f(-\pi) = f(\pi)$, так что $f$
продолжается на $\R$ как $2\pi$--периодическая функция. Поэтому
частные суммы $s_n(f;x)$, $n\in\N_0$ ряда Фурье функции $f$ имеют
вид \eqa{1}{s_n(f;x) = \frac1{\pi} \ints{-\pi}^{\pi} f(x+y)
\frac{\sin\hr{n+\frac12}y}{2\sin\frac{y}2}\,dy, \; n\in\N_0.} Если
$T_n(x)$ --- тригонометрический многочлен, $\deg T_n\le n$, то
$T_n(x)$ можно считать равномерно сходящимся на $\R$
тригонометрическим рядом, у которого частная сумма $s_n(T_n;x) =
T_n(x)$, $x\in\R$, и, в силу (1), \eqa{2}{T_n(x) = s_n(T_n;x) =
\frac1{\pi} \ints{-\pi}^{\pi} T_n(x+y) \frac{\sin\hr{n+\frac12}y}
{2\sin\frac{y}2}\,dy.}

На основании (1) и (2) имеем \equ{s_n(f;x) - s_n(T_n;x) =
\frac1{\pi} \ints{-\pi}^{\pi} \hs{f(x+y) - T_n(x+y)}
\frac{\sin\hr{n+\frac12}y}{2\sin\frac{y}2}\,dy,} и поскольку
$s_n(f;x) - f(x) = s_n(f;x) - s_n(T_n;x) + T_n(x) - f(x)$, то
получаем оценку \equ{\hm{s_n(f;x) - f(x)} \le \frac1{\pi}
\ints{-\pi}^{\pi} \hm{f(x+y) - T_n(x+y)}
\hm{\frac{\sin\hr{n+\frac12}y}{2\sin\frac{y}2}}\,dy + \hm{T_n(x) -
f(x)}, \; x\in[-\pi,\pi].} Обозначим $\tau_n = \supl{x\in[-\pi,\pi]}
\hm{f(x) - T_n(x)} = \supl{x\in\R} \hm{f(x) - T_n(x)}$. Тогда
\eqa{3}{\hm{ s_n(f;x) - f(x)} \le \tau_n \hs{1 + \frac1{\pi}
\ints{-\pi}^{\pi} \hm{\frac{\sin\hr{n+\frac12}y}{2\sin\frac{y}2}}
\,dy} = \tau_n [1+\Lc_n], \; n\in\N,} где \eqa{4}{\Lc_n =
\frac1{\pi} \ints{-\pi}^{\pi} \hm{\frac{\sin \hr{n+\frac12}y}{2\sin
\frac{y}2}} \,dy, \; n\in\N,} --- константы Лебега.

\subsubsection{Оценка констант Лебега}

Согласно (4) и чётности функции
$\hm{\frac{\sin\hr{n+\frac12}y}{2\sin\frac{y}2}}$, имеем \eqa{5}{
\Lc_n = \frac2{\pi} \ints0^{\pi}
\hm{\frac{\sin\hr{n+\frac12}y}{2\sin\frac{y}2}}\,dy= \frac1{\pi}
\ints0^{\frac1n} \hm{\frac{\sin\hr{n+\frac12}y}{2\sin\frac{y}2}} +
\frac1{\pi} \ints{\frac1n}^{\pi}
\hm{\frac{\sin\hr{n+\frac12}y}{2\sin\frac{y}2}}\,dy = I_1+I_2.}

Так как $\hm{\frac{\sin\hr{n+\frac12}y}{2\sin\frac{y}2}} =
\hm{\frac12 + \sumkun \cos ky} \le n+\frac12, \; y\in\R$, то
\eqa{6}{I_1 \le \frac2{\pi} \ints0^{\frac1n} \hr{n+\frac12}\,dy =
\frac{2n+1}{\pi n} < 1, \; n\in\N.} Согласно лемме 3 из параграфа 4
главы 4 $\hr{\sin x \ge \frac2{\pi}x, \; x\in\hs{0,\frac{\pi}2}}$,
справедлива оценка \eqa{7}{I_2 \le \frac1{\pi} \ints{\frac1n}^{\pi}
\frac{dy}{\sin\frac{y}2} \le \frac1{\pi} \ints{\frac1n}^{\pi}
\frac{dy}{\frac2{\pi} \frac{y}2} = \ints{\frac1n}^{\pi} \frac{dy}y =
\ln\pi + \ln n < 2 + \ln n.}

Объединяя (5)--(7), получим \eqa{8}{\Lc_n < 3 + \ln n, \; n\in\N.}
Точное значение $\Lc_n$ неизвестно до сих пор. На основании (3) и
(8) имеем оценку \eqa{9}{\hm{s_n(f;x) - f(x)} < \tau_n (4+\ln n), \;
n\in\N, \; x\in[-\pi,\pi],} где $\tau_n = \supl{x\in[-\pi,\pi]}
\hm{f(x) - T_n(x)}$ и $T_n(x)$ --- произвольный тригонометрический
многочлен, $\deg T_n \le n$.

\subsubsection{Теорема Дини--Липшица о равномерной сходимости ряда
Фурье}

\begin{theorem}
Если функция $f$ имеет период $2\pi$ и принадлежит классу
Дини--Липшица на отрезке $[-2\pi,2\pi]$ $\hr{\mbox{то есть, }
\liml{\de\ra+0} \om_f(\de) \ln\frac1{\de}=0}$, то её ряд Фурье
равномерно сходится на $[-\pi,\pi]$ (и на всём $\R$) к сумме $f(x)$.
\end{theorem}

\begin{proof}
Применим формулу (9) к $T_n(x) = A_n(f;x)$ --- многочленам Джексона.
Тогда, по первой теореме Джексона, $\tau_n = \maxl{x\in[-\pi,\pi]}
\hm{f(x) - A_n(f;x)} \le 6\om_f \hr{\frac1n}$, и поэтому, \eqa{10}
{\hm{s_n(f;x) - f(x)} \le 6\om_f \hr{\frac1n} [4+\ln n], \; n\in\N,
\; x\in[-\pi,\pi], \; (x\in\R).} Так как \equ{\liml{n\ra+\bes}
6\om_f \hr{\frac1n} (4+\ln n) = \liml{n\ra+\bes} 6\om_f \hr{\frac1n}
\ln n = \liml{\de\ra+0} 6\om_f(\de) \ln \frac1{\de}=0,} то из (10)
следует, что $s_n(f;x)\rra f(x)$ на $[-\pi,\pi]$ (и на всём $\R$).
\end{proof}

Класс Дини--Липшица достаточно широк и в этом сказывается большое
преимущество рядов Фурье перед рядами Тейлора.

Теорема Дини--Липшица содержит \emph{окончательное} (в терминах
модуля непрерывности функции) условие равномерной сходимости ряда
Фурье функции, так как можно построить функцию $f$, удовлетворяющую
условию $f(-\pi) = f(\pi)$ с модулем непрерывности, имеющим на
отрезке $[-2\pi,2\pi]$ порядок $O\hr{1/\ln\hr{\frac1{\de}}}$,
$\de\ra+0$, ряд Фурье которой расходится на множестве точек, всюду
плотном на $[-\pi,\pi]$.

\subsection{Сходимость в среднем ряда Фурье}
\subsubsection{Сходимость в среднем функциональной
последовательности}

\begin{df}
Функциональная последовательность $(f_n(x))$ сходится в среднем к
функции $f(x)$ на отрезке $[a,b]$, $b>a$, если числовая
последовательность $(\al_n)$, \equ{\al_n = \ints{a}^b \hs{f_n(x) -
f(x)}^2\,dx\ge0, \; n\in\N,} --- бесконечно малая; то есть,
$\liml{n\ra+\bes} \al_n=0$.
\end{df}

\begin{stm*}
Если $f_n\in\Rc[a,b]$, $n\in\N$, и $f_n(x) \rra f(x)$ на $[a,b]$, то
$(f_n(x))$ сходится к $f(x)$ в среднем на $[a,b]$.
\end{stm*}

\begin{proof}
В этом случае $f\in\Rc[a,b]$. Рассмотрим произвольное число $\ep>0$
и найдём такое $N\in\N$, $N=N(\ep)$, что $\supl{x\in[a,b]} \hm{f(x)
- f_n(x)} < \sqrt{\frac{\ep}{2(b-a)}}$ для всех $n\ge N$. Тогда
\equ{\al_n = \ints{a}^b [f_n(x) - f(x)]^2 \le \ints{a}^b
\frac{\ep}{2(b-a)}\,dx = \frac{\ep}2 < \ep} для всех $n\ge N$; то
есть, $\liml{n\ra+\bes} \al_n=0$.
\end{proof}

\textbf{Контрпример.} Рассмотрим $f_n(x)=x^n$, $x\in[0,1]$,
$n\in\N$, и $f(x)=0$, $x\in[0,1]$. Тогда \equ{\al_n = \ints0^1
x^{2n}\,dx = \frac1{2n+1}, \; n\in\N,} и $\liml{n\ra+\bes} \al_n=0$.
С другой стороны, $(f_n(x))$ не сходится к функции $f(x)$ на
$[0,1]$, так как $\liml{n\ra+\bes} f_n(1)=1\ne0=f(1)$.

\subsubsection{Вспомогательное утверждение}

\begin{lem*}
Если функция $f$ непрерывна на $[a,b]$, $b>a$, и $\ints{a}^b
f^2(x)\,dx=0$, то $f(x)=0$, $x\in[a,b]$.
\end{lem*}

\begin{proof}
Если $f(x_0)\ne0$, $x_0\in[a,b]$, то $f^2(x_0)>0$ и существует $\De
= [x_0-\de,x_0+\de]$, $\de>0$, что $f^2(x)>0$ для всех $x\in
\De\cap[a,b]$ (поскольку $f^2$ непрерывна на $[a,b]$). Пусть
$\De\cap[a,b] = \De_1$, длина $\hm{\De_1}>0$. Тогда \equ{\ints{a}^b
f^2(x)\,dx = \ints{[a,b]\backslash\De_1} f^2(x)\,dx + \ints{\De_1}
f^2(x)\,dx \ge \ints{\De_1} f^2(x)\,dx = f^2(\xi)\hm{\De_1} > 0.}
\end{proof}

\subsubsection{Свойство минимальности частных сумм ряда Фурье}

Пусть функция $f\in\Rc[-\pi,\pi]$ и $s_n(f;x)$, $n\in\N$ --- частная
сумма ряда Фурье функции $f$, а $T_n(x)$ --- произвольный
тригонометрический многочлен $\deg T_n\le n$. Рассмотрим \mla{1}{I_n
= \ints{-\pi}^{\pi} [f(x) - T_n(x)]^2\,dx = \ints{-\pi}^{\pi} [f(x)
- s_n(f;x) + s_n(f;x) - T_n(x)]^2\,dx = \ints{-\pi}^{\pi} [f(x) -
s_n(f;x)]^2\,dx + \\+2\ints{-\pi}^{\pi} [f(x)-s_n(f;x)] [s_n(f;x) -
T_n(x)]^2\,dx + \ints{-\pi}^{\pi} [s_n(f;x) - T_n(x)]^2\,dx =
I_1+I_2+I_3.} В формуле (1) числа $I_1\ge0$, $I_3\ge0$. Докажем, что
$I_2=0$.

Функции $f$ и $s_n(f;x)$ имеют одинаковые коэффициенты Фурье для
всех $k=\ol{0,n}$ (см. параграф 1, пункт 1.1, следствие 2). Поэтому
\equ{\ints{-\pi}^{\pi} [f(x) - s_n(f;x)] \cos kx\,dx = \pi a_k - \pi
a_k = 0, \; k=\ol{0,n},} \equ{\ints{-\pi}^{\pi} [f(x) - s_n(f;x)]
\sin kx \,dx = \pi b_k - \pi b_k = 0, \; k=\ol{1,n},} и
следовательно, \eqa{2}{\ints{-\pi}^{\pi} \hm{f(x)- s_n(f;x)}
\pi_n(x)\,dx=0} для любого тригонометрического многочлена
$\pi_n(x)$, $\deg \pi_n\le n$. В частности, (2) справедливо для
$s_n(f;x) - T_n(x) = t_n(x)$, $\deg t_n \le n$, и поэтому $I_2=0$.
Таким образом, доказана формула \eqa{3}{I_n = \ints{-\pi}^{\pi}
[f(x)-T_n(x)]^2\,dx = \ints{-\pi}^{\pi} [f(x) - s_n(f;x)]^2\,dx +
\ints{-\pi}^{\pi} [s_n(f;x) - T_n(x)]^2\,dx.}

\begin{thn}
{(минимальное свойство частных сумм ряда Фурье)} Если функция
$f\in\Rc[-\pi,\pi]$ и $T_n(x)$ --- произвольный тригонометрический
многочлен с $\deg T_n\le n$, то \eqa{4}{\ints{-\pi}^{\pi} [f(x) -
s_n(f;x)]^2 \,dx \le \ints{-\pi}^{\pi} [f(x) - T_n(x)]^2\,dx} и
равенство в (4) достигается только для $T_n(x) = s_n(f;x)$.
\end{thn}

\begin{proof}
Неравенство (4) прямо следует из равенства (3). Знак равенства в (4)
достигается только в случае, когда $\ints{-\pi}^{\pi} [s_n(f;x) -
T_n(x)]^2\,dx=0$, что, согласно лемме из пункта 3.2, имеет место
только в случае $s_n(f;x) = T_n(x)$.
\end{proof}

\subsubsection{Формула Бесселя}

Пусть функция $f\in\Rc[-\pi,\pi]$. Вычислим \equ{I =
\ints{-\pi}^{\pi} [f(x) - s_n(f;x)]^2\,dx = \ints{-\pi}^{\pi}
f^2(x)\,dx - 2\ints{-\pi}^{\pi} [f(x) - s_n(f;x)]s_n(f;x)\,dx -
\ints{-\pi}^{\pi} s_n^2(f;x)\,dx = I^1+I^2+I^3.} Как и в пункте 3.3,
убеждаемся, что $I^2=0$. Поэтому, вспоминая, что $s_n(f;x)$ и $f(x)$
имеют одинаковые коэффициенты Фурье для всех $k=\ol{0,n}$, имеем
\ml{0 \le I = \ints{-\pi}^{\pi} f^2(x)\,dx - \ints{-\pi}^{\pi}
\hs{\frac{a_0}2 + \sumkun (a_k\cos kx + b_k\sin kx)} s_n(f;x)\,dx =
\ints{-\pi}^{\pi} f^2(x)\,dx - \hs{\pi \frac{a_0}2 a_0 + \sumkun \pi
(a_k^2 + b_k^2)},} откуда \eqa{5}{\ints{-\pi}^{\pi}
[f(x)-s_n(f;x)]^2\,dx = \ints{-\pi}^{\pi} f^2(x)\,dx - \pi
\hs{\frac{a_0^2}2 + \sumkun (a_k^2 + b_k^2)}.}

\subsubsection{Неравенство Бесселя}

\begin{theorem}
Если функция $f\in\Rc[-\pi,\pi]$, то \eqa{6}{\frac{a_0^2}2 + \sumnui
(a_n^2+b_n^2) \le \frac1{\pi} \ints{-\pi}^{\pi} f^2(x)\,dx.}
\end{theorem}

\begin{proof}
Так как левая часть в равенстве (5) неотрицательна, то
\eqa{7}{\frac{a_0^2}2 + \sumkun(a_k^2+b_k^2) \le \frac1{\pi}
\ints{-\pi}^{\pi} f^2(x)\,dx} для всех $n\in\N$. Слева в (7) стоят
частные суммы положительного ряда из формулы (6), который, в силу
(7) и критерия сходимости положительных рядов, сходится и для его
суммы справедливо неравенство (6).
\end{proof}

\subsubsection{Свойство коэффициентов Фурье интегрируемых функций}

\begin{theorem}
Если функция $f\in\Rc[-\pi,\pi]$, то её коэффициенты Фурье $(a_n)$ и
$(b_n)$ образуют бесконечно малые последовательности; то есть,
$\liml{n\ra+\bes} a_n = \liml{n\ra+\bes} b_n = 0$.
\end{theorem}

\begin{proof}
Сходимость ряда в (6) влечёт свойство $\liml{n\ra+\bes} (a_n^2 +
b_n^2)=0$. Так как $0\le a_n^2 \le a_n^2 + b_n^2$ и $0\le b_n^2 \le
a_n^2 + b_n^2$, то $\liml{n\ra+\bes} a_n^2 = \liml{n\ra+\bes} b_n^2=
0$, откуда $\liml{n\ra+\bes} a_n = \liml{n\ra+\bes} b_n = 0$.
\end{proof}

\subsubsection{Уравнение замкнутости А.М. Ляпунова}

\begin{theorem}
Если функция $f$ непрерывна на $[-\pi,\pi]$ и $f(-\pi)=f(\pi)$, то
\eqa{8}{\frac{a_0^2}2 + \sumnui (a_n^2 + b_n^2) = \frac1{\pi}
\ints{-\pi}^{\pi} f^2(x)\,dx.}
\end{theorem}

\begin{proof}
Согласно (5) и свойству минимальности частных сумм ряда Фурье,
\eqa{9}{0 \le \frac1{\pi} \ints{-\pi}^{\pi} f^2(x)\,dx -
\hs{\frac{a_0^2}2 + \sumkun (a_k^2+b_k^2)} = \frac1{\pi}
\ints{-\pi}^{\pi} [f(x) - s_n(f;x)]^2\,dx \le \frac1{\pi}
\ints{-\pi}^{\pi} [f(x) - T_n(x)]^2\,dx} для любого
тригонометрического многочлена $T_n(x)$, $\deg T_n\le n$.

Выбирая в качестве $T_n(x) = F_n(f;x)$, $n\in\N$, --- многочлены
Фейера функции $f$, и зная, что $F_n(f;x) \rra f(x)$ на
$[-\pi,\pi]$, а следовательно, по утверждению пункта 3.1,
$\liml{n\ra+\bes} \ints{-\pi}^{\pi} [f(x) - F_n(f;x)]^2\,dx=0$,
получаем на основании (9) справедливость неравенства (8).
\end{proof}

\begin{note}
На самом деле, А.М. Ляпунов доказал равенство (8) для значительно
более широкого класса функций, включающего в себя, в частности,
множество $\Rc[-\pi,\pi]$.
\end{note}

\subsubsection{Единственность разложения в ряд Фурье}

\begin{theorem}
Если непрерывные на $[-\pi,\pi]$ функции $f$ и $g$, обладающие
свойством $f(-\pi)=f(\pi)$ и $g(-\pi)=g(\pi)$, имеют одинаковые
коэффициенты Фурье, то $f(x)=g(x)$, $x\in[-\pi,\pi]$.
\end{theorem}

\begin{proof}
Функция $h(x) = f(x) - g(x)$ непрерывна на $[-\pi,\pi]$ и $h(-\pi) =
h(\pi)$, и по условию, все коэффициенты Фурье функции $h$ равны
нулю. Согласно (8), $\ints{-\pi}^{\pi} h^2(x)\,dx=0$, и $h(x)=0$,
$x\in[-\pi,\pi]$ (по лемме пункта 3.2).
\end{proof}

\begin{imp*}
Если у функции $f$, непрерывной на $[-\pi,\pi]$ и $f(-\pi)=f(\pi)$,
все коэффициенты Фурье равны нулю, то $f(x)=0$, $x\in[-\pi,\pi]$.
\end{imp*}

\subsection{Теория Римана рядов Фурье}

\subsubsection{Лемма Римана}

Функцию $f$ называют абсолютно интегрируемой на промежутке
$\ha{a,b}$, $-\bes\le a < b \le +\bes$, если либо $f\in\Rc[a,b]$,
либо абсолютно сходящийся несобственный интеграл $\ints{a}^b
f(x)\,dx$; то есть, сходится несобственный интеграл $\ints{a}^b
\hm{f(x)}\,dx$.

\begin{theorem}
Для любой абсолютно интегрируемой на промежутке $\ha{a,b}$ функции
$f$ справедливо \eqa{1}{\liml{p\ra+\bes} \ints{a}^b f(x)\sin px\,dx
= \liml{p\ra+\bes} \ints{a}^b f(x)\cos px\,dx=0.}
\end{theorem}

\begin{proof}
Пусть сначала $f\in\Rc[a,b]$. Заметим, что для любых $\al,\be\in\R$,
$\al<\be$, и произвольного $p>0$ \eqa{2}{\hm{\ints{\al}^{\be} \cos
px\,dx} = \hm{\frac{\sin p\be - \sin p \al}{p}} \le \frac2p} и
\eqa{3}{\hm{\ints{\al}^{\be} \sin px\,dx} = \hm{\frac{\cos p\al -
\cos p \be}{p}} \le \frac2p.}

Докажем первое утверждение в (1). Прежде всего отметим, что функция
$f$ ограничена на $[a,b]$. Рассмотрим произвольное разбиение
$T\colon a=x_0<x_1<\ldots<x_{n-1}<x_n=b$ отрезка $[a,b]$ отрезками
$[x_{k-1},x_k] = \De_k$ длины $\hm{\De_k} = \De x_k = x_k -
x_{k-1}$, $k=\ol{1,n}$, и обозначим $m_k = \infl{x\in \De_k} f(x)$,
$M_k = \supl{x\in \De_k} f(x)$, $k=\ol{1,n}$. Тогда \equ{\ints{a}^b
f(x)\sin px\,dx = \sumkun \ints{x_{k-1}}^{x_k} f(x)\sin px\,dx =
\sumkun \ints{x_{k-1}}^{x_k} [f(x) - m_k] \sin px \,dx + \sumkun m_k
\ints{x_{k-1}}^{x_k} \sin px\,dx,} где, с учётом неравенства (3),
\equ{\hm{\sumkun m_k \ints{x_{k-1}}^{x_k} \sin px\,dx} \le \sumkun
\hm{m_k} \hm{\ints{x_{k-1}}^{x_k} \sin px\,dx} \le \sumkun \hm{m_k}
\frac2p = \frac2p \sumkun \hm{m_k}, \; p>0,} а также
\equ{\hm{\sumkun \ints{x_{k-1}}^{x_k} [f(x) - m_k] \sin px\,dx} \le
\sumkun \ints{x_{k-1}}^{x_k} \hm{f(x) - m_k} \hm{\sin px}\,dx \le
\sumkun (M_k - m_k)\De x_k,} так что неравенство
\eqa{4}{\hm{\ints{a}^b f(x)\sin px\,dx} \le \sumkun (M_k - m_k)\De
x_k + \frac2p \sumkun\hm{m_k}} справедливо для любого разбиения $T$
отрезка $[a,b]$ и любого $p>0$.

Рассмотрим произвольное число $\ep>0$ и на основании третьего
критерия интегрируемости функции $f$ выберем такое разбиение
$T_{\ep}\colon a=x_0 < x_1 < \ldots < x_{n-1} < x_n=b$, чтобы
\eqa{5}{\hm{\sumkun (M_k - m_k)\De x_k} < \frac{\ep}2,} и
зафиксируем числа $M_k$, $m_k$, $k=\ol{1,n}$, этого разбиения
$T_{\ep}$. Тогда найдётся такое $p_{\ep}>0$, что \eqa{6}{\frac2p
\sumkun \hm{m_k} < \frac{\ep}2 \mbox{ для всех }
p>p_{\ep}=\frac4{\ep} \sumkun \hm{m_k}.} На основании (4)--(6)
заключаем, что \equ{\hm{\ints{a}^b f(x)\sin px\,dx} < \frac{\ep}2 +
\frac{\ep}2 = \ep} для всех $p>p_{\ep}$, или, что
\equ{\liml{p\ra+\bes} \ints{a}^b f(x)\sin px\,dx=0.}

Рассмотрим теперь случай, когда абсолютно сходится несобственный
интеграл $\ints{a}^b f(x)\,dx$, и предположим (для определённости),
что он имеет особенность только на верхнем пределе и $b\in\R$. Тогда
функция $f\in\Rc[a,t]$ для любого $t$, $a<t<b$, и $\liml{t\ra b-0}
\ints{a}^b f(x)\,dx = \ints{a}^b f(x)\,dx$, и $\liml{t\ra b-0}
\ints{a}^t \hm{f(x)}\,dx = \ints{a}^b \hm{f(x)}\,dx.$

Поскольку $0\le\hm{f(x)\sin px} \le \hm{f(x)}$ для всех $x\in[a,b)$
и всех $p\in\R$, то несобственный интеграл $\ints{a}^b f(x)\sin
px\,dx$ также сходится абсолютно для всех $p\in\R$. Рассмотрим
произвольное число $\ep>0$ и укажем такое $\eta$, $0<\eta < b-a$,
чтобы $\ints{b-\eta}^b \hm{f(x)}\,dx < \frac{\ep}2$, так что
неравенство \eqa{7}{\hm{\ints{b-\eta}^b f(x)\sin px\,dx} \le
\ints{b-\eta}^b \hm{f(x)\sin px}\,dx \le \ints{a}^b \hm{f(x)}\,dx <
\frac{\ep}2} справедливо для всех $p\in\R$. Кроме того, по свойству
аддитивности несобственных интегралов, \eqa{8}{\ints{a}^b f(x)\sin
px\,dx = \ints{a}^{b-\eta} f(x)\sin px\,dx + \ints{b-\eta}^b
f(x)\sin px\,dx,} и функция $f\in\Rc[a,b-\eta]$. По доказанному,
$\liml{p\ra+\bes} \ints{a}^{b-\eta} f(x)\sin px\,dx=0$, и поэтому
существует такое $p_{\ep}>0$, что \eqa{9}{\hm{\ints{a}^{b-\eta}
f(x)\sin px\,dx} < \frac{\ep}2} для всех $p>p_{\ep}>0$. На основании
(7)--(9) заключаем, что \equ{\hm{\ints{a}^b f(x)\sin px\,dx} <
\frac{\ep}2 + \frac{\ep}2 = \ep} для всех $p>p_{\ep}>0$, или, что
\equ{\liml{p\ra+\bes} \ints{a}^b f(x)\sin px\,dx=0.}

Аналогично рассматриваются остальные типы несобственных интегралов,
а также доказывается второе утверждение теоремы.

\end{proof}

\subsubsection{Свойство коэффициентов Фурье абсолютно интегрируемых
функций}

Пусть функция $f$ абсолютно интегрируема на промежутке
$\ha{-\pi,\pi}$. Тогда числа \eqa{10}{a_n = \frac1{\pi}
\ints{-\pi}^{\pi} f(x)\cos nx\,dx, \; n\in\N_0 = \N\cup\hc{0},} и
\eqa{11}{b_n = \frac1{\pi} \ints{-\pi}^{\pi} f(x)\sin nx\,dx,
n\in\N,} которые существуют согласно теореме предыдущего пункта,
называются коэффициентами Фурье функции $f$ и, по той же теореме,
$\liml{n\ra+\bes} a_n = \liml{n\ra+\bes} b_n = 0$ (база $n\ra+\bes$
есть подбаза базы $p\ra+\bes$).

\subsubsection{Частные суммы ряда Фурье абсолютно интегрируемых
функций}

Пусть функция $f$ абсолютно интегрируема на $\ha{-\pi,\pi}$.
Согласно (10) и (11), функции $f$ сопоставляется её ряд Фурье \equ{
f(x), \; x\in\ha{-\pi,\pi}, \; \sim\; \frac{a_0}2 + \sumnui (a_n\cos
nx + b_n\sin nx), \; x\in\R,} частные суммы $s_n(f;x)$, $n\in\N$,
которого, как и в параграфах 2 и 3, вычисляется по формулам
\eqa{12}{s_n(f;x) = \frac1{\pi} \ints{-\pi}^{\pi} f(t)
\frac{\sin\hr{n+\frac12}(t-x)}{2\sin\frac{t-x}2} \,dt, \; n\in\N}
(на основании свойства линейности определённых и несобственных
интегралов). Рассмотрим новую функцию $f^*(x)$, $x\in[-\pi,\pi]$,
положив $f^*(x)=f(x)$, $x\in(-\pi,\pi)$, и $f^*(-\pi) = f^*(\pi)$,
если $f\in\Rc[-\pi,\pi]$. Поскольку интеграл Римана не к
замечаетк конечного множества точек отрезка интегрирования, то
\equ{\ints{-\pi}^{\pi} f(x)g(x)\,dx = \ints{-\pi}^{\pi}
f^*(x)g(x)\,dx} для любой функции $g\in\Rc[-\pi,\pi]$, и поэтому, в
частности, $s_n(f;x) = s(f^*;x)$. Поскольку функция $f^*$
продолжается на всё $\R$ как $2\pi$--периодическая функция, то так
же, как и в параграфе 1, доказывается лемма, по которой
\equ{\ints{-\pi+x}^{\pi+x} f^*(t)\,dt = \ints{-\pi}^{\pi}
f^*(t)\,dt} для любого $x\in\R$, и поэтому, аналогично формуле (5)
из параграфа 1, имеем \equ{s_n(f^*;x) = \frac1{\pi}
\ints{-\pi}^{\pi} f^*(x+y) \frac{\sin\hr{n+\frac12}y}{2\sin
\frac{y}2}\,dy = \frac1{\pi} \ints0^{\pi} [f^*(x+t) + f^*(x-t)]
\frac{\sin\hr{n+\frac12}t}{2\sin \frac{t}2}\,dt.} Так как
$s_n(f^*;x) = s_n(f;x)$, $n\in\N_0$, то \eqa{13}{s_n(f;x) =
\frac1{\pi} \ints0^{\pi} [f(x+t) + f(x-t)] \frac{\sin
\hr{n+\frac12}t}{2\sin\frac{t}2}\,dt, \; n\in\N_0.}

\subsubsection{Принцип локализации}

Пусть функция $f$ абсолютно интегрируема на $\ha{-\pi,\pi}$.
Рассмотрим произвольное $\de>0$ и запишем формулу (13) в виде
\eqa{14}{s_n(f;x) = \frac1{\pi} \ints0^{\de} [f(x+t) + f(x-t)]
\frac{\sin\hr{n+\frac12}t}{2\sin\frac{t}2}\,dt + \frac1{\pi}
\ints{\de}^{\pi} [f(x+t) + f(x-t)] \frac{\sin\hr{n+\frac12}t}{2\sin
\frac{t}2}\,dt = \si_n(f;x,\de) + r_n(f;x,\de).}

Функция $\frac1{\sin\frac{t}2}$ непрерывна на $[\de,\pi]$, и
следовательно, функция $\frac{f(x+t) + f(x-t)}2
\frac1{\sin\frac{t}2}$ абсолютно интегрируема по $t$ на
$\ha{\de,\pi}$, $\de>0$, для любого фиксированного $x\in(-\pi,\pi)$.
Таким образом, формула (14) показывает, что для каждого $\de>0$ и
каждого $x\in(-\pi,\pi)$ последовательности $(s_n(f;x))$ и
$(\si_n(f;x,\de))$ сходятся или расходятся одновременно.

\textbf{Теорема Римана.} Для любой абсолютно интегрируемой на
$\ha{-\pi,\pi}$ функции $f$ поведение её ряда Фурье в любой точке
$x\in(-\pi,\pi)$ (то есть, его сходимость или расходимость в точке
$x$) зависит исключительно от значений, принимаемых функцией $f$ в
сколь угодно малой окрестности точки $x$.

\begin{imp*}
Если функция $f$ абсолютно интегрируема на $\ha{-\pi,\pi}$ и
$f(x)=0$ в некотором интервале $(\al,\be)$, содержащемся в
$\ha{-\pi,\pi}$, то её ряд Фурье сходится к $0=f(x)$ в каждой точке
$x\in(\al,\be)$.
\end{imp*}

\begin{proof}
Фиксируем $x\in(\al,\be)$ и в $\ha{-\pi,\pi}$ и $f(t)=0$, $t\in
(x-\de,x+\de)$. Поэтому, $\si_n(f;x,\de)=0$ в формуле (14) и
$\liml{n\ra+\bes} s_n(f;x)=0=f(x)$.
\end{proof}

\begin{imp*}
Если функции $f$ и $g$ абсолютно интегрируемы на $\ha{-\pi,\pi}$ и
$f(x)=g(x)$ на некотором интервале $(\al,\be)$ из $\ha{-\pi,\pi}$,
то ряды Фурье функций $f$ и $g$ сходятся или расходятся
одновременно, и в случае сходимости имеют на $(\al,\be)$ одинаковые
суммы.
\end{imp*}

\begin{proof}
Функция $h(x) = f(x)-g(x)$ абсолютно интегрируема на $\ha{-\pi,\pi}$
на основании свойства линейности определённого и несобственного
интегралов и $h(x)=0$, $x\in(\al,\be)$. На том же основании и по
определению (12), \equ{s_n(f;x) - s_n(g;x) = \frac1{\pi}
\ints{-\pi}^{\pi} [f(x+t) - g(x+t)]
\frac{\sin\hr{n+\frac12}(t-x)}{2\sin\frac{t-x}2} \,dt = s_n(f-g;x) =
s_n(h;x).}

Согласно предыдущему следствию, $\liml{n\ra+\bes} s_n(h;x)=0=h(x)$
для всех $x\in(\al,\be)$, так что $\liml{n\ra+\bes} [s_n(f;x) -
s_n(g;x)]=0$ для всех $x\in(\al,\be)$, и доказательство завершено.
\end{proof}

\subsubsection{Признак Дини сходимости ряда Фурье в точке и его
следствия}

Укажем достаточные условия сходимости ряда Фурье абсолютно
интегрируемой на $\ha{-\pi,\pi}$ функции $f$ в точках интервала
$(-\pi,\pi)$. Для этого фиксируем произвольную точку
$x_0\in(-\pi,\pi)$ и рассмотрим в ней частные суммы $s_n(f;x_0)$
ряда, используя полученные для них в пункте 4.3 интегральные
выражения (13), так что \eqa{13'}{s_n(f;x_0) = \frac1{\pi}
\ints0^{\pi} [f(x_0+t) + f(x_0-t)] \frac{\sin\hr{n+\frac12}t}{2\sin
\frac{t}2}\,dt.} Поскольку у функции $f(x)\equiv1$ все
$s_n(f;x)\equiv1$, то из (13) (или из (13')) следует, что \equ{1 =
\frac2{\pi} \ints0^{\pi} \frac{\sin\hr{n+\frac12}t}{2\sin
\frac{t}2}\,dt.} Умножая обе части этого равенства на постоянное
$s_0$ --- предполагаемую сумму ряда в $x_0$, точное значение которой
мы установим ниже, и вычитая из (13'), найдём \eqa{15}{s_n(f;x_0) -
s_0 = \frac1{\pi} \ints0^{\pi} \ph(t)
\frac{\sin\hr{n+\frac12}t}{2\sin \frac{t}2}\,dt,} где для краткости
положено \eqa{16}{\ph(t) = f(x_0+t) + f(x_0-t)-2s_0.}

Если мы хотим установить, что $s_0$ действительно является суммой
ряда, то для этого нужно доказать, что интеграл в (15) при
$n\ra+\bes$ стремится к нулю.

Обратимся к выбору самого числа $s_0$. В практических приложениях
важны два случая, когда \begin{items}{-2} \item (a) функция $f$
непрерывна в точке $x_0$; \item (б) $f$ имеет в этой точке разрыв
первого рода (скачок), так что оба предела $f(x_0+0)$ и $f(x_0-0)$
существуют.
\end{items} Поэтому, ограничим себя только этими двумя случаями и
положим \begin{items}{-2}\item в случае (а): $s_0=f(x_0)$; \item в
случае (б): $s_0 = \frac{f(x_0+0)+f(x_0-0)}2$.
\end{items}

Впрочем, эти суммы можно не различать, если в точке $x_0$, где
налицо разрыв первого рода, выполнено равенство $f(x_0) =
\frac{f(x_0+0) + f(x_0-0)}2$. Точки, где это условие соблюдено,
иногда называют регулярными.

Отметим, что так как \eqa{а}{\liml{t\ra+0} f(x_0\pm t)=f(x_0),} или
\eqa{б}{\liml{t\ra+0} f(x_0\pm t) = f(x_0\pm0),} смотря по случаю,
то при указанном выборе числа $s_0$ всегда имеет $\liml{t\ra+0}
\ph(t)=0$.

Имея это в виду, сформулируем теперь признак Дини.

\textbf{Признак Дини.} Ряд Фурье функции $f$ в точке $x_0$ сходится
к сумме $s_0$, если при некотором $h>0$ несобственный интеграл
$\ints0^h \frac{\hm{\ph(t)}}t\,dt$ сходится.

\begin{proof}
При этом предположении сходится и интеграл $\ints0^{\pi}
\frac{\hm{\ph(t)}}t\,dt$. Если переписать формулу (15) в виде
\equ{s_n(f;x_0) - s_0 = \frac1{\pi} \ints0^{\pi} \frac{\ph(t)}t
\frac{\frac12t}{\sin\frac{t}2} \sin\hr{n+\frac12}t\,dt,} то
непосредственно по лемме Римана ясно, что при $n\ra+\bes$ интеграл в
правой части стремится к нулю, так как функция $\frac{\ph(t)}t$, а с
нею и $\frac{\ph(t)}t \frac{\frac12t}{\sin\frac{t}2}$, абсолютно
интегрируемы на $\ha{-\pi,\pi}$ $\hr{\mbox{существует }\liml{t\ra0}
\frac{\frac12t}{\sin\frac{t}2}=1}$.
\end{proof}

В развёрнутом виде интеграл Дини записывается в виде: \equ{\mbox{в
случае (а): } \ints0^h \frac{\hm{f(x_0+t) + f(x_0-t) -
2f(x_0)}}t\,dt,} \equ{\mbox{в случае (б): } \ints0^h \frac{\hm{f(x_0
+ t) + f(x_0-t) - f(x_0+0) - f(x_0-0)}}t\,dt,} и следовательно,
достаточно предположить существование порознь интегралов (смотря по
случаю) \eqa{17}{\ints0^h \frac{\hm{f(x_0+t) - f(x_0)}}t\,dt \mbox{
и } \ints0^h \frac{\hm{f(x_0-t)-f(x_0)}}t\,dt} или \equ{\ints0^h
\frac{\hm{f(x_0+t)-f(x_0+0)}}t\,dt \mbox{ и } \ints0^h
\frac{\hm{f(x_0 - t) - f(x_0-0)}}t\,dt.} Отсюда можно получить ряд
частных признаков, используя различные известные признаки
существования интегралов. Ограничиваясь случаем (а), укажем
нижеследующий признак.

\textbf{Признак Липшица.} Ряд Фурье функции $f$ сходится в точке
$x_0$, где она непрерывна, к сумме $f(x_0)$, если для достаточно
малых $t>0$ выполняется неравенство $\hm{f(x_0\pm t) - f(x_0)} \le
L\cdot t^{\al}$, в котором $L$ и $\al$ --- положительные постоянные
и $\al\le1$.

\begin{proof}
В случае $\al=1$ имеем $\hm{\frac{f(x_0\pm t) - f(x_0)}t}\le L$, так
что интегралы (17) существуют как собственный (см. материал второго
семестра). Если же $0<\al<1$, то \equ{\hm{\frac{f(x_0\pm t) -
f(x_0)}t} \le \frac{L}{t^{1-\al}},} и так как справа стоит
интегрируемая функция, то интегралы (17) существуют, как
несобственные.
\end{proof}

В частности, условие Липшица при $\al=1$ заведомо будет выполняться,
если у функции $f$ в точке $x_0$ существует конечная производная
$f'(x_0)$ или, по крайней мере, конечные односторонние производные
\equ{\Dc^+ f(x_0) = \liml{t\ra+0} \frac{f(x_0+t) - f(x_0)}t, \; D^-
f(x_0) = \liml{t\ra+0} \frac{f(x_0-t) - f(x_0)}{-t},} хотя бы и
различные между собой (к угловая точка
к). Таким образом, в точке
$x_0$, где функция дифференцируема, или, по крайней мере, имеет обе
конечные односторонние производные, ряд Фурье сходится, причём сумма
его равна $f(x_0)$.

Нетрудно перефразировать признак Липшица и для случая (б). Как
частное следствие отсюда, получим здесь, что в точке $x_0$ разрыва
первого рода для сходимости ряда Фурье достаточно предположить
существование конечных пределов \equ{\liml{t\ra+0} \frac{f(x_0+t) -
f(x_0+0)}t, \; \liml{t\ra+0} \frac{f(x_0-t) - f(x_0-0)}{-t};} причём
на этот раз суммой ряда будет $\frac{f(x_0+0) + f(x_0-0)}2$.

\subsection{Теория Дирихле рядов Фурье}
\subsubsection{Лемма Дирихле}

\begin{theorem}
Если функция $g$ возрастает и ограничена на $(0,h]$ для некоторого
$h>0$, то \eqa{1}{\liml{p\ra+\bes} \ints0^h g(t) \frac{\sin pt}t\,dt
= g(+0)\frac{\pi}2,} где $g(+0) = \liml{t\ra+0}g(t)$.
\end{theorem}

\begin{proof}
На основании свойств монотонности и ограниченности функции $g$ на
$(0,h]$ следует существование одностороннего предела $\liml{t\ra+0}
g(t) = g(+0) = \infl{t\in(0,h]} g(t)$, а с учётом того, что
$\liml{t\ra0} \frac{\sin pt}{t}=p$ для любого $p\in\R$ --- и
существование интеграла в (1). Представим его в виде суммы двух
интегралов \eqa{2}{g(+0) \ints0^h \frac{\sin pt}t\,dt + \ints0^h
[g(t) - g(+0)] \frac{\sin pt}t\,dt.}

Если первый из них с помощью подстановки $pt=z$, $p>0$,
преобразовать к виду $g(+0) \ints0^{ph} \frac{\sin z}z\,dz$, то
сразу ясно, что при $p\ra+\bes$ он стремится к $\frac{\pi}2g(+0)$,
ибо $\ints0^{+\bes} \frac{\sin z}z\,dz=\frac{\pi}2$. Таким образом,
весь вопрос сводится к доказательству того, что второй из интегралов
(2) стремится к нулю.

Для произвольного числа $\ep>0$ найдём такое $\de>0$ (можно считать
$\de<h$), что $0\le g(t) - g(+0)<\ep$ для $0<t\le\de$. Разобьём
теперь упомянутый только что интеграл на два. \equ{\hr{\ints0^{\de}
+ \ints{\de}^h} [g(t) - g(+0)] \frac{\sin pt}t\,dt = I_1 + I_2.} К
интегралу $I_1$ применим вторую теорему о среднем (в форме
формулировки Бонне) и получим, что \equ{ I_1 = [g(\de) - g(+0)]
\ints{\xi}^{\de} \frac{\sin pt}t\,dt = [g(\de) - g(+0)] \ints{\xi
p}^{\de p} \frac{\sin z}z\,dz,} где $0\le\xi < \de$ и $p>0$. Первый
множитель $<\ep$, а второй равномерно ограничен при всех значениях
$p>0$. Действительно, из сходимости несобственного интеграла
$\ints0^{+\bes} \frac{\sin z}z\,dz$ следует, что непрерывная (при
$x\ge0$) функция $\ints0^x \frac{\sin z}z\,dz$ аргумента $x$,
имеющая при $x\ra+\bes$ конечный предел, будет ограничена при всех
значениях $x$, \equ{\hm{\ints0^x \frac{\sin z}z\,dz} \le L \;
(L=\const),} так что \equ{\hm{\ints{\xi p}^{\de p} \frac{\sin
z}z\\,dz} = \hm{\ints0^{\de p} \frac{\sin z}z\,dz - \ints0^{\xi p}
\frac{\sin z}z\,dz} \le 2L.} Итак, для интеграла $I_1$ имеем
независимо от $p>0$ оценку $\hm{I_1} < 2L\ep$.

Что касается интеграла $I_2$, то при $p\ra+\bes$ (и фиксированном
$\de$) он стремится к нулю по лемме Римана, так как множитель при
$\sin pt$ есть интегрируемая в собственном смысле функция (ведь
$t\ge\de>0$). Этим завершается доказательство теоремы.
\end{proof}

\subsubsection{Признак Дирихле--Жордана сходимости ряда Фурье в
точке}

\begin{theorem}
Если функция $f\in\Rc[-\pi,\pi]$, $f(-\pi) = f(\pi)$, и для точки
$x_0\in(-\pi,\pi)$ можно указать $h>0$, что $[x_0 - h, x_0+h]
\subset(-\pi,\pi)$ и на $[x_0-h,x_0+h]$ функция $f$ имеет
ограниченное изменение, то её ряд Фурье сходится в точке $x_0$ к
сумме $f(x_0)$ (если $f$ непрерывна в $x_0$) и к сумме $\frac12
[f(x_0+0) + f(x_0-0)]$ (если $f$ имеет разрыв в $x_0$).
\end{theorem}

\begin{proof}
По условию теоремы и теореме Жордана, $f(x) = f_1(x) - f_2(x)$ на
$[x_0-\de, x_0+\de]$ и функции $f_i$, $i=1,2$ --- возрастающие.
Поэтому, пределы $f(x_0\pm0)$ существуют в каждой точке $x\in
(x_0-h,x_0+h)$; в частности, существуют $f(x_0\pm0)$. Далее,
\eqa{3}{ s_n(f;x_0) = \frac1{\pi} \ints0^h [f(x_0+t) + f(x_0-t)]
\frac{\sin \hr{n+\frac12}t}{2\sin \frac{t}2}\,dt + \frac1{\pi}
\ints{h}^{\pi} [f(x_0+t) + f(x_0-t)] \frac{\sin\hr{n+\frac12}
t}{2\sin \frac{t}2}\,dt = I_n^1 + I_n^2.}

Согласно лемме Римана, \eqa{4}{\liml{n\ra+\bes} I^2_n =
\liml{n\ra+\bes} \frac1{\pi} \ints{h}^{\de} [f(x_0+t) + f(x_0-t)]
\frac1{2\sin\frac{t}2} \sin\hr{n+\frac12}t\,dt=0.}

Запишем $I_n^1$ в виде \eqa{5}{I_n^1 = \frac1{\pi} \ints0^h
[f(x_0+t) + f(x_0-t)] \frac{\frac{t}2}{\sin\frac{t}2}
\frac{\sin\hr{n+\frac12}t}{t}\,dt.} Функция $f(x_0+t) + f(x_0-t)$
имеет ограниченное изменение на $[0,h]$, как сумма функций
ограниченной вариации. Функция $\frac{\frac{t}2}{\sin\frac{t}2}\ra1$
при $t\ra0$, $\frac{\frac{t}2}{\sin\frac{t}2}\ge1$, $t\in[0,h]$, и
она возрастает и ограничена на $[0,h]$. Поэтому, функция $g(t) =
[f(x_0+t) + f(x_0-t)] \frac{\frac{t}2}{\sin\frac{t}2}$ имеет
ограниченное изменение на $[0,h]$ и $g(t) = g_1(t)-g_2(t)$, где
функции $g_i$, $i=1,2$, возрастают на $[0,h]$. По определению $g(+0)
= g_1(+0) - g_2(+0) = f(x_0+0) + f(x_0-0)$, так как $\liml{t\ra0}
\frac{\frac{t}2}{\sin\frac{t}2} = 1$. Итак, в новых обозначениях
интеграл (5) принимает вид \eqa{6}{I_n^1 = \frac1{\pi} \ints0^h
[g_1(t) - g_2(t)] \frac{\sin\hr{n+\frac12}t}{t}\,dt.} По лемме
Дирихле \equ{\liml{n\ra+\bes} \frac1{\pi} \ints0^h g_i(t)
\frac{\sin\hr{n+\frac12}t}{t}\,dt = \frac1{\pi} \frac{\pi}2 g_i(+0)
= \frac12 g_i(+0), \; i=1,2,} и следовательно, на основании (6)
имеем \eqa{7}{\liml{n\ra+\bes} I_n^1 = \frac12 [g_1(+0) - g_2(+0)] =
\frac12 g(+0) = \frac{f(x_0+0) + f(x_0-0)}2.} Объединяя (3), (4) и
(7), получим, что \equ{\liml{n\ra+\bes} s_n(f;x_0) = \frac{f(x_0+0)
+ f(x_0-0)}2,} откуда и следует утверждение теоремы.
\end{proof}

\subsubsection{Кусочно--монотонные функции}

\begin{df}
Функция $f$, определённая на $[a,b]$, $a<b$, называется
кусочно--монотонной на $[a,b]$, если $[a,b]$ можно разбить на
конечное число промежутков, на каждом из которых она монотонна.
\end{df}

\begin{lem*}
Любая ограниченная кусочно--монотонная на отрезке функция имеет на
нём ограниченное изменение.
\end{lem*}

\begin{proof}
Пусть функция $f$ ограничена и кусочно--монотонна на отрезке
$[a,b]$, так что существует разбиение $a\le \ol{x}_1 < \ol{x}_2 <
\ldots < \ol{x}_m \le b$, что функция $f$ монотонна на промежутках
$\ha{\ol{x}_l, \ol{x}_{l+1}}$, $l=\ol{1,m-1}$, и $\left[ a, \ol{x}_1
\right>$ $\left< \ol{x}_m,b\right]$, а следовательно, имеет на
каждом из них ограниченную вариацию. Рассмотрим произвольное
разбиение $T\colon a=x_0 < x_1 < \ldots < x_{n-1} < x_n = b$ отрезка
$[a,b]$ и образуем $T' = T\cup \hc{\ol{x}_1,\ldots, \ol{x}_m}$.
Тогда вариация $\bigvee(f;T) \le \bigvee(f;T')$ и $\bigvee(f;T')\le
C$, где $C$ равно сумме полных вариаций функции $f$ по всем
указанным выше промежуткам. Поэтому, $\bigvee\limits_a^b f = \sup
\hc{\bigvee(f;T) \bvl T} \le C$.
\end{proof}

Первоначально сформулированные самим Дирихле условия разложимости
функции в ряд Фурье носили более частный характер. Именно, им
доказан нижеследующий признак.

\textbf{Признак Дирихле.} Если функция $f$ периода $2\pi$
кусочно--монотонна на отрезке $[-\pi,\pi]$ и имеет в нём не более,
чем конечное число точек разрыва первого рода, то её ряд Фурье
сходится к сумме $f(x_0)$ в каждой точке непрерывности и к сумме
$\frac{f(x_0+0) + f(x_0-0)}2$ в каждой точке разрыва.

\begin{proof}
Согласно лемме, это --- частный случай теоремы пункта 5.2.
\end{proof}

\subsubsection{Взаимоотношение признаков Дини и Дирихле--Жордана}

Можно сказать, что они несравнимы между собой ; то есть, не вытекают
один из другого. Рассмотрим сначала функцию $f$ на $[-\pi,\pi]$,
заданную в виде \equ{f(x) = \case{\frac1{\ln\frac{\hm{x}}{2\pi}},
\mbox{ если } x\ne0, \\ 0, \mbox{ если } x=0.}} Эта функция
непрерывна и кусочно--монотонна, и значит, удовлетворяет условию
Дирихле. В то же время интеграл Дини, относящийся к точке $x=0$,
\equ{\ints0^h \frac{\hm{f(t) + f(-t) - 2f(0)}}t = 2\ints0^h
\frac{dt}{t\ln\frac{t}{2\pi}}} расходится при любом $h>0$.

С другой стороны, для определённой на $[-\pi,\pi]$ функции \equ{f(x)
= \case{x\cos\frac{\pi}{2x}, \mbox{ если } x\ne0, \\ 0, \mbox{ если
} x=0,}} в точке $x=0$ выполнено условие Липшица $\hm{f(x) - f(0)} =
\hm{x\cos\frac{\pi}{2x}} \le \hm{x},$ а следовательно, и условие
Дини. Однако эта функция ни в какой окрестности точки $x=0$ не имеет
ограниченного изменения.

Действительно, рассмотрим, например, отрезок $[0,1]$. Если
рассмотреть разбиение $T'$ этого отрезка точками $0<\frac1{2n} <
\frac1{2n-1} < \ldots < \frac13 < \frac12 < 1$, то \ml{\bigvee(f;T)
= \hm{\frac1{2n}\cos\pi n- 0} + \hm{\frac1{2n-1}\sin\pi n -
\frac1{2n}\cos\pi n} + \hm{\frac1{2n-2} \cos\pi(n-1) -
\frac1{2n-1}\sin\pi n} \spl \\+ \hm{1\cos\frac{\pi}2 -
\frac12\cos\pi} = \frac1{2n} + \frac1{2n} + \frac1{2n-2} +
\frac1{2n-2} \spl \frac12+\frac12 = 1 + \frac12 \spl \frac1n =
\ln(n+1) + c - \al_n, \; \liml{n\ra+\bes} \al_n=0,} так что
$\bigvee\limits_0^1 f = \sup\hc{\bigvee(f;T) \bvl T}=+\bes$.


\subsection{Ряды Фурье гладких функций}
\subsubsection{Связь между коэффициентами Фурье функции и её
производных}

\begin{df}
Функцию $f$ отнесём к множеству $\Rc^{(k)}[-\pi,\pi]$, $k\in\N$,
если у неё существует производная $f^{(k)}(x)$, $x\in[-\pi,\pi]$,
$f^{(k)}\in\Rc[-\pi,\pi]$ и $f^{(l)}(-\pi) = f^{(l)}(\pi)$ для всех
$l=\ol{0,n-1}$; $f^{(0)}=f$.
\end{df}

Рассмотрим коэффициенты Фурье $a_n$, $b_n$ функции $f$ и обозначим
$a_n^{(l)}$, $b_n^{(l)}$, $n\in\N$ --- коэффициенты Фурье функции
$f^{(l)}$, $l=\ol{1,k}$. Тогда \equ{\pi a_n = \ints{-\pi}^{\pi}
f(x)\cos nx\,dx = \hs{f(x)\frac{\sin nx}{n}}^{\pi}_{-\pi} - \frac1n
\ints{-\pi}^{\pi} f'(x)\sin nx\,dx = -b^{(1)}_n\frac1n\pi;} то есть,
$$a_n = -\frac1n b_n^{(1)}, \; n\in\N_0.$$

Аналогично, с учётом равенства $f(-\pi)=f(\pi)$ и чётности функции
$\cos x$, \equ{\pi b_n = \ints{-\pi}^{\pi} f(x)\sin nx\,dx = \hs{
-f(x)\frac{\cos nx}n}^{\pi}_{-\pi} + \frac1n \ints{-\pi}^{\pi}
f'(x)\cos nx\,dx = \frac1n \pi a_n^{(1)};} то есть, $$b_n=\frac1n
a_n^{(1)}, \; n\in\N.$$

Если применить эти формулы к коэффициентам $a_n^{(1)}$, $b_n^{(1)}$,
$n\in\N$, то получим \equ{a_n = -\frac1{n^2} a_n^{(2)}, \; b_n =
-\frac1{n^2} b_n^{(2)}, \; n\in\N.} Продолжая этот процесс, мы
получим (по индукции) соответствующие формулы, в которых необходимо
различать случаи чётного и нечётного числа $k$. Именно,
\eqa{1a}{\mbox{при } k=2m\colon a_n = (-1)^m \frac{a_n^{(k)}}{n^k},
\; b_n = (-1)^m \frac{b_n^{(k)}}{n^k}, \; n\in\N,}
\eqa{1b}{\mbox{при } k=2m-1\colon a_n = (-1)^m
\frac{b_n^{(k)}}{n^k}, \; b_n = (-1)^{m-1} \frac{a_n^{(k)}}{n^k}, \;
n\in\N.}

\subsubsection{Почленное дифференцирование ряда Фурье}

\begin{lem*}
Коэффициенты Фурье $a_n$, $b_n$ функции $f$, принадлежащей множеству
$\Rc^{(k+1)}[-\pi,\pi]$, $k\in\N$, обладают свойством сходимости
ряда $\sumnui n^k (\hm{a_n} + \hm{b_n})$.
\end{lem*}

\begin{proof}
Согласно формулам $(1a)$, $(1b)$, \equ{\hm{a_n} + \hm{b_n} =
\frac1{n^{k+1}} \hr{\hm{a_n^{(k+1)}} + \hm{b_n^{(k+1)}}}, \;
n\in\N,} откуда \eqa{2}{n^k \hr{\hm{a_n} + \hm{b_n}} =
\frac{\hm{a_n^{(k+1)}}}n + \frac{\hm{b_n^{(k+1)}}}n, \; n\in\N.} Так
как \equ{\frac{\hm{a_n^{(k+1)}}}n \le \frac12 \hr{\hm{a_n^{(k+1)}}^2
+ \frac1{n^2}}, \; \frac{\hm{b_n^{(k+1)}}}n \le \frac12
\hr{\hm{b_n^{(k+1)}}^2 + \frac1{n^2}},} то на основании (2) имеем
\eqa{3}{n^k \hr{\hm{a_n} + \hm{b_n}} \le \frac12 \hr{
\hm{a_n^{(k+1)}}^2 + \hm{b_n^{(k+1)}}^2} + \frac1{n^2}, \; n\in\N.}
Поскольку $f^{(k+1)}\in\Rc[-\pi,\pi]$, то, по неравенству Бесселя,
сходится ряд \equ{\sumnui \hr{\hm{a_n^{(k+1)}}^2 +
\hm{b_n^{(k+1)}}^2} \le \frac1{\pi} \ints{-\pi}^{\pi}
\hm{f^{(k+1)}(x)}^2\,dx,} а также сходится ряд $\sum \frac1{n^2}$.
Поэтому, согласно (3), сходится ряд $\sumnui n^k \hr{\hm{a_n} +
\hm{b_n}}.$
\end{proof}

\begin{theorem}
Ряд Фурье любой функции $f$ из $\Rc^{(k+1)}[-\pi,\pi]$, $n\in\N$,
можно почленно дифференцировать $k$ раз и продифференцированный ряд
абсолютно и равномерно сходится на $[-\pi,\pi]$ к сумме
$f^{(k)}(x)$.
\end{theorem}

\begin{proof}
Любая функция $f$ из $\Rc^{(k+1)}[-\pi,\pi]$, согласно определению,
имеет непрерывную $f'$ на $[-\pi,\pi]$, и следовательно, по теореме
Лагранжа о конечных приращениях, функция $f$ удовлетворяет условию
Липшица на $[-\pi,\pi]$. Поэтому, по теореме Дини--Липшица,
\eqa{4}{f(x) = \frac{a_0}2 + \sumnui (a_n\cos nx + b_n\sin nx), \;
x\in[-\pi,\pi],} и ряд в (4) равномерно сходится на $[-\pi,\pi]$.
Его формальное почленное дифференцирование $k$ раз приводит к ряду
\eqa{5}{\sumnui n^k \hs{a_n\cos \hr{nx+k\frac{\pi}2} + b_n\sin
\hr{nx + k\frac{\pi}2}}.} Поскольку оценка \equ{ \hm{n^k \hs{a_n
\cos \hr{nx+k\frac{\pi}2} + b_n\sin\hr{nx+k\frac{\pi}2}}} \le n^k
\hr{\hm{a_n} + \hm{b_n}}, \; n\in\N,} справедлива для всех
$x\in[-\pi,\pi]$ (всех $x\in\R$) и, согласно лемме, сходится ряд
$\sumnui n^k \hr{\hm{a_n}+\hm{b_n}}$, то по признаку Вейерштрасса
ряд (5) сходится абсолютно и равномерно на $[-\pi,\pi]$. Поэтому, по
теореме о почленном дифференцировании функциональных рядов, суммой
ряда (5) будет функция $f^{(k)}(x)$.
\end{proof}

\subsubsection{Почленное интегрирование ряда Фурье}

\begin{theorem}
Если функция $f$ непрерывна на $[-\pi,\pi]$ и \eqa{6}{\frac{a_0}2 +
\sumnui(a_n\cos nx + b_n\sin nx)} --- её ряд Фурье, то
\mla{7}{\ints0^t f(x)\,dx = \ints0^t \frac{a_0}2\,dt + \sumnui
\ints0^t (a_n\cos nx + b_n\sin nx)\,dx = \frac{a_0}2t + \sumnui
\hr{\frac{a_n}n\sin nt + \frac{b_n}n(1-\cos nt)}} для любого
$t\in[-\pi,\pi]$ и ряд (7) равномерно сходится на $[-\pi,\pi]$.
\end{theorem}

\begin{proof}
Рассмотрим функцию $F(t)$, \eqa{8}{F(t) = \ints0^t \hr{f(x) -
\frac{a_0}2}\,dx, \; t\in[-\pi,\pi].} Функция $F(t)$ непрерывна на
$[-\pi,\pi]$, имеет производную $F'(t)=f(t)-\frac{a_0}2$,
непрерывную на $[-\pi,\pi]$, и следовательно, функция $F(t)$
принадлежит классу Липшица на $[-\pi,\pi]$ (по теореме Лагранжа о
конечных приращениях). Кроме того, \equ{F(\pi) - F(-\pi) =
\ints{-\pi}^{\pi} \hr{f(x) - \frac{a_0}2}\,dx = \pi a_0 - \pi a_0 =
0;} то есть, $F(-\pi)=F(\pi)$.

Поэтому, по теореме Дини--Липшица, \eqa{9}{F(t) = \frac{A_0}2 +
\sumnui (A_n\cos nt + B_n\sin nt), \; t\in[-\pi,\pi],} где
\mla{10}{A_n = \frac1{\pi} \ints{-\pi}^{\pi} F(t)\cos nt\,dt =
\hs{\frac1{\pi n} F(t)\sin nt}^{\pi}_{\pi} - \frac1{\pi n}
\ints{-\pi}^{\pi} F'(t)\sin nt\,dt = -\frac1{\pi n}
\ints{-\pi}^{\pi} \hs{f(t) - \frac{a_0}2}\sin nt\,dt = -\frac1n b_n,
\; n\in\N,} и, с учётом равенства $F(-\pi)=F(\pi)$ и чётности
функции $\cos x$, \eqa{11}{B_n = \frac1{\pi} \ints{-\pi}^{\pi}
F(t)\sin nt\,dt = \hs{-\frac1{\pi n} F(t)\cos nt}^{\pi}_{\pi} +
\frac1{\pi n} \ints{-\pi}^{\pi} \hs{f(t) - \frac{a_0}2}\cos nt\,dt =
\frac1n a_n, \; n\in\N,} а ряд (9) равномерно сходится на
$[-\pi,\pi]$.

Положив $t=0$ в (9), получим, с учётом (8) и (10), что \equ{0 =
\frac{A_0}2 + \sumnui A_n = \frac{A_0}2 - \sumnui \frac{b_n}n,}
откуда \eqa{12}{\frac{A_0}2 = \sumnui \frac{b_n}n.}

Подставляя формулы (10)--(12) в представление (9), приходим к
разложению \equ{F(t) = \sumnui \hs{\frac{a_n}n\sin nt + \frac1n
(1-\cos nt)}, \; t\in[-\pi,\pi],} которое с учётом (8) равносильно
(7).
\end{proof}

\subsubsection{Поведение коэффициентов ряда Фурье функции
ограниченной вариации}

\begin{theorem}
Если функция $f$ имеет ограниченную вариацию на $[-\pi,\pi]$, то для
её коэффициентов Фурье справедливы асимптотики $a_n =
O\hr{\frac1n}$, $n\ra+\bes$, и $b_n = O\hr{\frac1n}$, $n\ra+\bes$.
\end{theorem}

\begin{proof}
Так как $f=f_1-f_2$ и $f_i$, $i=1,2$, возрастают на $[-\pi,\pi]$, то
$f\in\Rc[-\pi,\pi]$. Поэтому \mla{13}{a_n = \frac1n
\ints{-\pi}^{\pi} f(x)\cos nx\,dx = \frac1{\pi} \ints{-\pi}^{\pi}
f(x) d\hr{\frac{\sin nx}x} = \hs{\frac1{\pi} f(x) \frac{\sin
nx}n}^{\pi}_{-\pi} - \frac1{\pi} \ints{-\pi}^{\pi} \frac{\sin nx}n
d(f(x)) = -\frac1{\pi n} \ints{-\pi}^{\pi} \sin nx d(f(x)),} где
последний интеграл рассматривается как интеграл Стилтьеса.
Аналогично, \mla{14}{b_n = \frac1{\pi} \ints{-\pi}^{\pi} f(x)\sin
nx\,dx = \frac1{\pi} \hs{-f(x) \frac{\cos nx}n}^{\pi}_{-\pi} +
\frac1{\pi n} \ints{-\pi}^{\pi} \cos nx d(f(x)) = \frac{(-1)^n}{\pi
n} [f(-\pi) - f(\pi)] + \frac1{\pi n} \ints{-\pi}^{\pi} \cos nx
df(x).}

Согласно оценке модуля интеграла Стилтьеса и на основании (13) и
(14), справедливы оценки \eqa{15a}{\hm{a_n} \le \frac1{\pi n}
\maxl{x\in[-\pi,\pi]} \hm{\sin nx} \bigvee\limits_{-\pi}^{\pi} f =
\frac1{\pi n} \bigvee\limits_{-\pi}^{\pi} f,} и \eqa{15b}{\hm{b_n}
\le \frac1{\pi n} \hm{f(\pi) - f(-\pi)} + \frac1{\pi n}
\maxl{x\in[-\pi,\pi]} \hm{\cos nx} \bigvee\limits_{-\pi}^{\pi} f =
\frac1{\pi n} \hm{f(\pi) - f(-\pi)} + \frac1{\pi n}
\bigvee\limits_{-\pi}^{\pi} f.}

Таким образом, неравенства (15a) и (15b) означают, что $a_n =
O\hr{\frac1n}$, $n\ra+\bes$, и $b_n = O\hr{\frac1n}$, $n\ra+\bes$.
\end{proof}

\begin{note}
Если дополнительно предположить, что $f(-\pi) = f(\pi)$, то
$\hm{b_n} \le \frac1{\pi n} \bigvee\limits_{-\pi}^{\pi} f, \;
n\in\N$.
\end{note}

\subsubsection{Оценка остатка ряда Фурье}

\begin{theorem}
Если функция $f\in\Rc^{(k)} [-\pi,\pi]$, $f^{(k)}(-\pi) =
f^{(k)}(\pi)$, $k\in\N$, и произвольная $f^{(k)}(x)$ имеет
ограниченное изменение на $[-\pi,\pi]$, то остаток $R_n(f;x)$ ряда
Фурье функции $f$ имеет равномерную оценку $\hm{R_n(f;x)} \le
\frac{v_k}{n^k}$ для всех $x\in[-\pi,\pi]$, в которой $v_k =
\bigvee\limits_{-\pi}^{\pi} f^{(k)}$.
\end{theorem}

\begin{proof}
Так как всякая функция ограниченной вариации на отрезке ограничена
на нём, то функция $f^{(k)}$ ограничена на $[-\pi,\pi]$, и
следовательно, как ив доказательстве теоремы пункта 6.2, заключаем,
что $f$ удовлетворяет условию Липшица на $[-\pi,\pi]$. Поэтому, по
теореме Дини--Липшица, ряд Фурье функции $f$ равномерно сходится на
$[-\pi,\pi]$ к сумме $f(x)$ и справедлива формула (4). Тогда
\equ{R_n(f;x) = f(x) - s_n(f;x) = \sums{m=n+1}^{\bes} (a_m\cos mx +
b_m\sin mx), \; x\in[-\pi,\pi], \; n\in\N,} и \eqa{16}{\hm{R_n(f;x)}
\le \sums{m=n+1}^{\bes} \hr{\hm{a_m} + \hm{b_m}}, \; x\in[-\pi,\pi],
\; n\in\N.}

Если $a_n^{(k)}$, $b_n^{(k)}$ --- коэффициенты Фурье функции
$f^{(k)}(x)$, то согласно формулам (1a) и (1b), \eqa{17}{\hm{a_m} +
\hm{b_m} = \frac1{m^k} \hr{\hm{a_m^{(k)}} + \hm{b_m^{(k)}}}.}
Воспользуемся теперь оценками (15a) и (15b). Поскольку $f^{(k)}(x)$
имеет ограниченную вариацию на $[-\pi,\pi]$ и $f^{(k)}(-\pi) =
f^{(k)}(\pi)$, то, с учётом замечания из пункта 6.4,
\eqa{18}{\hm{a_m^{(k)}} \le \frac1{\pi m}
\bigvee\limits_{-\pi}^{\pi} f^{(k)} = \frac1{\pi m} v_k, \;
\hm{b^{(k)}_m} \le \frac1{\pi m} \bigvee\limits_{-\pi}^{\pi} f^{(k)}
= \frac1{\pi m} v_k, \; m\in\N.}

Согласно (16)--(18), \equ{\hm{R_m(f;x)} \le \frac1{\pi} v_k
\sums{m=n+1}^{\bes} \hr{\frac1{m^{k+1}} + \frac1{m^{k+1}}} =
\frac2{\pi} v_k \sums{m=n+1}^{\bes} \frac1{m^{k+1}} < v_k
\sums{m=n+1}^{\bes} \frac1{m^{k+1}}.} Так как $$\sums{m=n+1}^{\bes}
\frac1{m^{k+1}} < \frac1{n^k},$$ то окончательно, $$\hm{R_n(f;x)} <
\frac{v_k}{n^k}, \; v_k = \bigvee\limits_{-\pi}^{\pi} f^{(k)}.$$
\end{proof}

\end{document}
