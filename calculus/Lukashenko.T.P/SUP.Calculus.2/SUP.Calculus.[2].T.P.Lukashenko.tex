\documentclass[unicode,10pt]{article}
\usepackage[xe]{dmvn}
\usepackage{polyglossia}
\usepackage{unicode-math}
\usepackage{fontspec}
\usepackage{makeidx}
\usepackage{sh57}
\defaultfontfeatures{Scale=MatchLowercase, Mapping=tex-text}
\setmainfont{CMU Serif}
\setsansfont{CMU Sans Serif}
\setmonofont{CMU Typewriter Text}
\setmathfont{xits-math.otf}

\title{Математический анализ --- формулировки}
\author{Лектор --- Т.\,П.\,Лукашенко}
\date{I курс, 2 семестр, поток математиков}

\newcommand{\билет}[1]{\par\medskip\noindent{\large \textsf{Билет #1.}}\par}

\begin{document}
\maketitle

{\footnotesize Данное издание представляет собой все леммы, теоремы \итд из курса математического анализа
Т.\,П.\,Лукашенко во втором семестре. Основная цель --- помочь в запоминании основных теорем курса, для
улучшения понимания сути курса. Набор осуществлён Сергеем Шашковым. Отсутствие лажи не гарантируется.\par}

\begin{petit}
 {\bf \#!}Категорическая просьба взять исходники и привести в божеский
вид. Макросы кириллицей это странно.{$\sim\sim\sim\sim$}
\end{petit}

\dmvntrail

\bigskip

\билет 1

\begin{df} Неперекрывающиеся отрезки.
\end{df}

\begin{df} Разбиение отрезка.
\end{df}

\begin{df} Отмеченное разбиение.
\end{df}

\begin{df} Сумма $\sum\limits_{\mathbb{T}}f\Delta x= \sum\limits_{i=1}^n f(\xi)|\Delta_i|$ называется
интегральной суммой функции \ф, соответствующей отмеченному разбиению $\тт$ отрезка $[a,b]$. \end{df}

\begin{df} Масштаб.
\end{df}

\begin{df} Интегралы Римана, Мак-Шейна, Курцвейля-Хенстока.
\end{df}

\begin{df} Базы $\mathfrak{R},{ }\mathfrak{M},{ }\mathfrak{H}$, в частности $\mathcal
B_{\mathfrak{H}}=\{B_{\delta(x)}\усл B_{\delta(x)}=\{\тт\усл \xi_i\in\Delta_i,\Delta_i\влж
B_{\delta(\xi_i)}(\xi_i)\}\}$ \end{df}

\begin{lemma} Для любого масштаба $\delta(x)>0$ на $[a,b]$ существует согласованное разбиение \кх.
\end{lemma}

\begin{imp} Все элементы баз не пусты.
\end{imp}

Интегралы - пределы по соответствующим базам.

Свойства интегралов

\pt{1} Если $\ф$ интегрируема по \рим или по \мш и $\И$  -- её интеграл, то $\ф$ интегрируема по \кх и $\И$
её интеграл.

\pt{2} Если интеграл в любом смысле существует, то он единственен.

\pt{3} Линейность по функциям.

\pt{4} Переход к неравенству.

\begin{df} \выд{Критерий Коши}: функция $\ф$ имеет конечный предел по базе $\баз$, если $\ф$ определена на
некотором элементе базы и $\люб\эпс>0\  \сущ B\in\баз \усл \люб x,x'\in B\  |f(x)-f(x')|<\эпс$. \end{df}

\begin{df} Критерий Коши существования интегралов \рим, \мш, \кх.
\end{df}

\begin{theorem}[об интегрируемости на подотрезках]  Если $f$ интегрируема на $[a,b]$, то $f$ интегрируема на
любом подотрезке в том же смысле. \end{theorem}

\билет 2

\begin{theorem}[необходимое условие \рим] Если $f\in\рим[a,b]$, то $f$ ограничена на $[a,b]$.
\end{theorem}

\begin{theorem}[аддитивность интеграла по отрезкам] Если $f$ интегрируема на $[a,b]$ и на $[b,c]$ в одном смысле
то $f$ интегрируема на $[a,c]$ в том же смысле и $\оинт{a}{c}{f}=\оинт{a}{b}{f}+\оинт{b}{c}{f}$. \end{theorem}

\begin{theorem}[формула Ньютона-Лейбница] Если $f$ определена на $[a,b]$, $F \непр[a,b]$ и $F'(x)=f(x)$ всюду на $[a,b]$, кроме не более чем счетного множества точек (к.с.м.), то $f$ интегрируема на $[a,b]$ в смысле \кх, и $\кинт{a}{b}{f}=F(b)-F(a)$.
\end{theorem}

\begin{imp} Если $f$ определена на $[a,b]$, $F$ -- её обобщенная производная, то $f$ интегрируема на $[a,b]$ по
\кх и $\кинт{a}{b}{f}=F(b)-F(a)$. \end{imp}

\begin{imp} Если $F_1,\;F_2\непр[a,b]$ и $F'_1(x)=F'_2(x)\кон$ всюду к.с.м. на  $[a,b]$, то $F_1(x)-F_2(x)$
постоянна на $[a,b]$. \end{imp}


\билет 3


\begin{df} Для любого $E\влж\рр$ верхняя мера $\мю^*E=\inf\limits_{\bigcup_i l_i\сод E}\sum_i |l_i|$, где
$\{l_i\}$ - система интервалов, покрывающая $E$. Если брать вместо интервалов отрезки, то получится эквивалентное
определение. \end{df}

\pt{1} Если $E\влж D\влж \рр$, то $\мер E\le\мер D$.

\pt{2} Если $E\влж \bigcup_i E_i, i\in \нн$, то $\мер E\le\сум_i\мер E_i$.

\begin{lemma} Если дан ряд неотрицательных чисел $\sum_{k=1}^\infty a_k, a_k\ge0$, то при любой его перестановке
сумма его не меняется. \end{lemma}

\begin{df} Множество меры нуль по Лебегу:
\end{df}

\pt{1} Подмножество множества меры нуль по Лебегу -- множество меры ноль по Лебегу.

\pt{2} Не более, чем счетное объединение множеств меры нуль -- множество меры нуль.

\begin{df} Колебанием функции $f$ на множестве $E$ называется $\осц_E f=\sup_{x,y\in E}|f(x)-f(y)|$.
\end{df}

\begin{lemma} Если $f$ действительнозначна на $E$, то $\осц_E f=\sup_E f - inf_E f$.
\end{lemma}

\begin{lemma} Если $\{\Delta_i\}_{i=1}^n$ -- разбиение $[a,b]$, $\Delta=[c,d]\влж[a,b]$, то невырожденные отрезки
$\Delta\пер\Delta_i$ образуют разбиение $\Delta=[c,d]$ и $\sum_i|\Delta\пер\Delta_i|=|\Delta|$. \end{lemma}

\begin{df} Если какое-то свойство верно для всех точек, исключая точки некоторого множества меры нуль по Лебегу,
то говорят,  что свойство выполняется \выд{почти всюду}. \end{df}

\begin{theorem} Если функция $f$ на $[a,b]$ ограничена и непрерывна почти всюду, то $f\ри,\;f\ми$ . \end{theorem}

\begin{imp} Если $f\непр[a,b]$, то $f\ри[a,b]$ и $f\ми[a,b]$.
\end{imp}

\begin{imp} Если $f$ монотонна на $[a,b]$, то $f\ри[a,b]$ и $f\ми[a,b]$.
\end{imp}

\билет 4


\begin{lemma} Пусть $f$ огр. на $[a,b]$, тогда для любого разбиения $\тт=\сисин$ отрезка $[a,b]$ имеем:
$$\осц_{\xi_i\in\Delta_i}\sum_{i=1}^n f(\xi_i)|\Delta_i|=\sum_{i=1}^n \осц_{\Delta_i}f|\Delta_i|.$$
\end{lemma}

\begin{theorem} Если $f\ри[a,b]$, то $f\непр[a,b]$ почти всюду.
\end{theorem}

\begin{imp} (критерий интегрируемости Лебега) Функция $f\ри[a,b]$ тогда и только тогда, когда $f$ ограничена и
непрерывна почти всюду на $[a,b]$. \end{imp}

\begin{imp} Если $f\ри[a,b]$, то $f\ми[a,b]$.
\end{imp}

Дополнительные свойства интеграла Римана

\pt{1} Если $f,g\ри[a,b]$, то $fg\ри[a,b]$.

\pt{2} Если $f\ри[a,b]$, а $\фи\непр[c,d]$, $[c,d]\сод f([a,b])$, то $\фи(f)\ри[a,b]$.

\pt{3} Если значения интегрируемой по Риману функции изменить в конечном числе точек, что полученная функция
$\волн f\ри[a,b]$ и $\ринт{a}{b}{f}=\ринт{a}{b}{\волн f}$.

\pt{4} Если $f\ри[a,b]$, то $|f|\ри[a,b]$ и $\left|\оинт{a}{b}{f}\right|\le\оинт{a}{b}{|f|}$.

\pt{5} Если $f(x)\ge 0$ на $[a,b]$, $f\ри[a,b]$ и в некоторой точке $x_0$, в которой $f$ непрерывна на
$[a,b]$ $f(x_0)>0$, то $\ринт{a}{b}{f}>0$.


\билет   {5}


\begin{theorem} Если $f(x)=0$ почти всюду на $[a,b]$, то $f\ми[a,b]$ и $\минт{a}{b}{f}=0$.
\end{theorem}

\begin{imp} Если $f(x)=g(x)$ почти всюду на $[a,b]$, то $f$ и $g$ одновременно интегрируемы или неинтегрируемые
по \мш (\кх) и, в случае интегрируемости, их интегралы совпадают. \end{imp}

\begin{df} Если $f$ интегрируема на $[a,b]$ в каком-либо смысле, то для любого $x_0\in [a,b]$ функция
$F(x)=\оинт{x_0}{x}f$ называется \выд{интегралом с переменным верхним пределом}. \end{df}

\begin{df} Функция $F$ принадлежит классу Липшица на $E\влж\рр$, если $F$ определена на $E$ и существует $C\кон$,
что $\люб x_1,x_2\in E \усл |F(x_1)-F(x_2)|\le C|x_1-x_2|$. \end{df}

\begin{theorem} Если $f$ ограничена на $[a,b]$ и интегрируема на $[a,b]$ в каком-либо смысле, то её
неопределённый интеграл $F(x)$ принадлежит классу Липшица на $[a,b]$. \end{theorem}

\begin{imp} неопределённый интеграл \рим -- функция класса Липшица, а значит непрерывна.
\end{imp}

\begin{theorem} Пусть $f$ интегрируема на отрезке $[a,b]$ в каком-либо смысле и $F(x)=\оинт{x_0}{x}{f}$,
$x_0\in[a,b]$, -- неопределённый интеграл $f$. Если $f\непр(x\in[a,b])$, то $F\диф(x\in[a,b])$ и $F'(x)=f(x)$.
\end{theorem}

\begin{imp} Если $f\ри[a,b]$ и $F$ -- неопределённый интеграл $f$, то $F'(x)=f(x)$ почти всюду на $[a,b]$.
\end{imp}

\begin{imp} Если $f\непр[a,b]$, то её неопределённый интеграл -- точная первообразная $f$ на $[a,b]$. \end{imp}

\begin{imp} Если $f$ ограничена на $[a,b]$ и непрерывна всюду на $[a,b]$, кроме конечного числа точек, то её
неопределённый интеграл -- обобщенная первообразная $f$ на $[a,b]$. \end{imp}

\билет   {6}


\begin{df} Пусть $f(x)$ определена на $[a,b]$ почти всюду. Функция интегрируема по \мш (\кх) на $[a,b]$, если при
каком-то доопределении на весь отрезок $[a,b]$ получится интегрируемая по \мш (\кх) функция, при этом величина
интеграла не зависит от доопределения. \end{df}

\begin{df} Функция $f$ \выд{измерима} на $[a,b]$, если $f$ определена на $[a,b]$ почти всюду и $\люб\эпс>0\  \сущ
g\непр[a,b]$, что $\мер\{x\in[a,b] \усл f(x)\ne g(x)\}<\эпс$. \end{df}

\begin{df} Функция $f$ \выд{измерима} на $[a,b]$, если $f$ определена на $[a,b]$ почти всюду и $\люб\эпс>0\  \сущ
E\влж[a,b]$, $\мер [a,b]\backslash E<\эпс$, что $f\непр[a,b] \backslash E$. \end{df}

\begin{theorem} Определения равносильны.
\end{theorem}

\begin{lemma} Любое открытое множество $G\влж\рр$ является объединением не более, чем счетного семейства попарно
непересекающихся интервалов. \end{lemma}

\begin{df} Если $F$ -- замкнутое множество на прямой, то  составляющие интервалы $\рр\backslash F$ называют
\выд{смежными}. \end{df}

\begin{lemma} Если замкнутое множество $F\ne0$ и функция $f\непр F$, то доопределив $f$ на конечных смежных
интервалах $(\алф,\бет)$ как линейную функцию на $[\алф,\бет]$, а не бесконечных смежных интервалах постоянной,
совпадающей  $f(c)$, получим непрерывную функцию на \рр, причем $\sup_\рр|f|=\sup_F|f|$. \end{lemma}

\begin{theorem} Если $f$ -- измерима и ограничена на $[a,b]$, то $f$ интегрируема по \мш на $[a,b]$.
\end{theorem}

%\enlargethispage{1\baselineskip}
\билет   {7}


\begin{lemma}[Слабая Сакса-Хенстока] Пусть $f\ми[a,b]$ ($f\ки[a,b]$) и для $\эпс>0$ найден масштаб $\delta(x)>0$
такой, что $\люб \тт $ -- отмеченного разбиения \мш (\кх), согласованного с $\delta(x)$ выполнено неравенство:
$\left| \sum_{\тт}f\Delta x-\оинт{a}{b}{f} \right|<\эпс$. Тогда для любого отмеченного разбиения $\тт=\{ (
\Delta_i,\xi_i ) \}_{i=1}^n$ $\мш$ (\кх), согласованного с $\delta(x)$ и для любого $J\влж\{1,\ldots,n\}$
выполнено неравенство: $\left|\sum_{i\in J}\left(f(\xi_i)|\Delta_i|-\оинт{\Delta_i}{}{f})\right)\right|\le\эпс$.
\end{lemma}

\begin{lemma}[Сильная Сакса-Хенстока] Пусть $f\ми[a,b]$ ($f\ки[a,b]$) и для $\эпс>0$ найден масштаб $\delta(x)>0$
такой, что $\люб \тт $ -- отмеченного разбиения \мш (\кх), согласованного с $\delta(x)$ выполнено неравенство:
$\left| \sum_{\тт}f\Delta x-\оинт{a}{b}{f} \right|<\эпс$. Тогда для любого отмеченного разбиения $\тт=\{ (
\Delta_i,\xi_i ) \}_{i=1}^n$ $\мш$ (\кх), согласованного с $\delta(x)$ и для любого $J\влж\{1,\ldots,n\}$
выполнено неравенство: $\sum_{i\in J}\left|f(\xi_i)|\Delta_i|-\оинт{\Delta_i}{}{f})\right|\le2\эпс$. \end{lemma}

\begin{theorem} Если $f\ки[a,b]$, то её неопределённый интеграл $F\непр[a,b]$.
\end{theorem}

\begin{theorem} Если $f\ми[a,b]$, то $|f|\ми[a,b]$ и $\left|\минт{a}{b}{f}\right|\le\минт{a}{b}{|f|}$.
\end{theorem}

\begin{lemma} $\люб a,b\кон$ верно неравенство: $||a|-|b||\le|a-b|$.
\end{lemma}

\билет   {8}


\begin{df} Множество $E\влж\рр$ покрыто системой отрезков $\Omega$ в смысле Витали, если все отрезки системы
$\Omega$ невырождены и $\люб\эпс>0\ \люб x\in E \ \сущ\Delta\in\Omega \усл x\in\Delta$ и $|\Delta|<\эпс$.
\end{df}

\begin{theorem}[Витали] Если ограниченное множество $E\влж\рр$ покрыто системой отрезков $\Omega$ в смысле
Витали, то найдется такая конечная или бесконечная последовательность непересекающихся отрезков
$\Delta_i\in\Omega$, что: $\мер\left(E\backslash\bigcup_i \Delta_i\right)=0$ и $\sum_i|\Delta_i|<\беск$.
\end{theorem}

\begin{theorem}[Витали] Если ограниченное множество $E\влж\рр$ покрыто системой отрезков $\Omega$ в смысле
Витали, то для любого $\эпс>0$ найдется конечная последовательность отрезков $\Delta_i\in\Omega$, что:
$\мер\left(E\backslash\bigcup_{i=1}^I \Delta_i\right)<\эпс$. \end{theorem}

\begin{theorem} Если $f\ки[a,b]$, $F(x)$ -- неопределённый интеграл $f$, то почти всюду на $[a,b]$ $F'(x)=f(x)$.
\end{theorem}

\билет   {9}


\begin{df} Пусть функции $f$ и $g$ определены на $[a,b]$. Тогда для любого отмеченного разбиения $\тт$ отрезка
$[a,b]$ интегральной суммой $f$ по $g$ называется $\sum_{\тт}f\Delta g=\sum_{i=1}^n f(\xi_i)\Delta_i g$. \end{df}

\begin{df} Интегралы Римана-Стилтьеса, Мак-Шейна-Стилтьеса и Курцвейля-Хенстока-Стилтьеса.
$$\ксинт{a}{b}{f}{g}=\lim_{\баз^{\mathfrak{H}}}\sum_{\тт}f\Delta g.$$
\end{df}

Свойства интегралов \pt{1} Если $f$ интегрируема по $g$ на $[a,b]$ в смысле \римс или \мшс, то $f$
интегрируема по $g$ на $[a,b]$ в смысле \кхс и интегралы совпадают.

\pt{2} Единственность интеграла.

\pt{3} Линейность по $f$ и по $g$.

\pt{4} Если $f_1(x)\le f_2(x)$ на $[a,b]$, $g(x)$ неубывает на $[a,b]$, $f_1$ и $f_2$ интегрируемы по $g$, то
$\осинт{a}{b}{f_1}{g}\le\осинт{a}{b}{f_2}{g}$.

Критерий Коши существования интегралов Стилтьеса \begin{theorem} Если $f$ интегрируема по $g$ на $[a,b]$ в
каком-либо смысле, то $f$ интегрируема по $g$ на любом подотрезке в том же смысле. \end{theorem}

%\enlargethispage{3\baselineskip}

\билет  {10}


\begin{theorem} Если $f$ интегрируема по $g$ на $[a,b]$ и на $[b,c]$ в смысле \мшс или \кхс, то $f$ интегрируема
по $g$ на $[a,c]$ в том же смысле и $\осинт a c f g=\осинт a b f g + \осинт b c f g$. \end{theorem}

\begin{imp} Если $f$ интегрируема по $g$ на $[a,b]$, $[b,c]$ и $[a,c]$ в смысле \римс, то $\осинт a c f g=\осинт
a b f g + \осинт b c f g$. \end{imp}

\begin{df} Пусть функция $f$ определена на множестве $E\влж\рр$. Тогда $\вар_E
f=\supl{a_0<\ldots<a_n}\sum_{i=1}^n |f(a_i)-f(a_{i-1})|$. \end{df}

\begin{df} Пусть функция $f$ определена на множестве $E\влж\рр$. Тогда $\вар_E f=\supl{\{\Delta_i\}}\sum_i
|\Delta_i f|$, где $\{\Delta_i\}$ -- не более, чем счетная система попарно неперекрывающихся отрезков с концами
из $E$. \end{df}

\begin{theorem} Определения эквивалентны.
\end{theorem}

\begin{df} Функция $f$ -- ограниченной вариации на $E\влж\рр$, если $\вар_E f<+\беск$.
\end{df}

\pt{1} Если $f$ определена на $E\влж\рр$, то для любого \алф\вв\рр $\ \вар_E \alpha f=|\alpha|\вар_E f$.

\pt{2} Если $f$ и $g$ определены на $E\влж\рр$, то $\var_E(f\pm g)\le \var_E f + \var_E g$.

\pt{3} Если $\фи$ -- $VB$-функция на $[a,b]$ и на $[b,c]$, то $\фи$ -- $VB$-функция на $[a,с]$ и
$\var_{[a,c]}\фи=\var_{[a,b]}\фи+\var_{[b,c]}\фи$.

\pt{4} Если $\фи$ -- $VB$-функция на $E$, то \фи\ ограничена на $E$.

\pt{5} Если $\фи$ и $\пси$ -- $VB$-функции на $E$, то $\фи\пси$ -- $VB$-функция на $E$ и
$\var_E(\фи\пси)\le\sup_E|\фи|\var_E\пси +\sup_E|\пси|\var_E\фи$.

\pt{6} Если $\фи$ -- $VB$-функция на $E$ и $m=\inf_E|\фи|>0$, то $\frac{1}{\фи}$ -- $VB$-функция на $E$ и
$\var_E\frac{1}{\фи}\le\frac{1}{m^2}\var_E\фи$.

\pt{7} Если $\фи$ -- $VB$-функция, а $\пси$ из класса Липшица на $\фи(E)$, то $\пси(\фи)$ -- $VB$-функция на
$E$ и $\var_E\пси(\фи)\le C\var_E\фи$.

\begin{df} Вариация с переменным верхним пределом.
\end{df}

\begin{theorem} Если $\фи$ -- $VB$-функция на промежутке $I$, $x_0\in I$, то $\var_{x_0}^x\фи$ -- неубывающая
ограниченная функция от $x$ на $I$, причем если $\фи$ действительнозначна, то неубывающими ограниченными являются
также функции $\var_{x_0}^x\фи\pm\фи(x)$. \end{theorem}

\begin{lemma} Если $\фи$ -- монотонная функция на промежутке $I$, то $\var_I\фи=\sup_I\фи-\inf_I\фи=\осц_I\фи$.
\end{lemma}

\begin{theorem} Определенная на промежутке $I$ действительнозначная функция $\фи$ является $VB$-функцией на $I$,
тогда и только тогда, когда $\фи$ является разностью двух ограниченных неубывающих функций на I, причем их можно
взять такими, что сумма их вариаций будет равна $\var_I\фи$. \end{theorem}

\билет  {11}


\begin{lemma} Если $T=\{\Delta_i\}$ -- разбиение отрезка $\Delta=[a,b]$, а $g$ -- функция на $[a,b]$, то
$\sum_i\Delta_i g=g(b)-g(a)=\Delta g$. \end{lemma}

\begin{theorem} Если $f\ytgh[a,b]$, а $g\juhd[a,b]$, то $f$ интегрируема по $g$ на $[a,b]$ в смысле \римс  и в
смысле \мшс, при этом $\left|\jcbyn{a}{b}{f}{g}\right|\le\sup_{[a,b]}|f|\var_a^b g$. \end{theorem}

\begin{lemma} Если $g$ имеет разрыв первого рода в точке $x_0\in[a,b]$ или устранимый разрыв в точках $a$ или $b$
по отрезку $[a,b]$, то необходимым условием \римс является непрерывность $f$ в точке $x_0$. \end{lemma}

\begin{theorem}[об интегрируемости по частям для \римс интеграла] Если $f$ интегрируема по $g$ в смысле \римс на
$[a,b]$, то и $g$ \римс-интегрируема по $f$ на $[a,b]$ и
$\осинт{a}{b}{f}{g}=f(b)g(b)-f(a)g(a)-\осинт{a}{b}{g}{f}=(fg)|_a^b - \jcbyn{a}{b}{g}{f}$. \end{theorem}

\begin{imp} Если $f\огрв[a,b]$, $g\ytgh[a,b]$, то существует \римс-интеграл от \ф\ по $g$ на $[a,b]$.
\end{imp}

%\enlargethispage{-1\baselineskip}

\билет  {12}


\begin{theorem}[о сведении интеграла \римс к интегралу \рим] Пусть $g\hb[a,b]$ и $G$ -- её неопределённый
интеграл. Если $f$ ограничена на $[a,b]$, $fg\ри[a,b]$, то $f$ интегрируема по $G$ на $[a,b]$ в смысле \римс и
$\hcbyn{a}{b}{f}{G}=\hbyn{a}{b}{fg}$. \end{theorem}

\begin{theorem}[формула интегрирования по частям для интеграла \рим] Пусть $f\hb[a,b]$ и $F$ -- её неопределённый
интеграл, $g\hb[a,b]$ и $G$ -- её неопределённый интеграл. Тогда $fG,\ gF\hb[a,b]$ и
$\jbyn{a}{b}{fG}=FG|_a^b-\jbyn{a}{b}{gF}$. \end{theorem}

\begin{theorem}[Формула замены переменной для интеграла \рим] Пусть $f\ytgh[a,b]$ и $F$ -- её неопределённый
интеграл, $\фи$ непрерывно дифференцируема на $[\алф,\бет]$, $\фи($[\алф,\бет]$)\влж[a,b]$, $\фи(\алф)=a,\
\фи(\бет)=b$. Тогда $f(\фи(t))\фи'(t)\hb[a,b]$ и $\jcbyn{\алф}{\бет}{f(\фи)\фи'}{t}=\jbyn{a}{b}{f}=F(b)-F(a)$.
\end{theorem}

\begin{theorem}[формула Тейлора с остаточным членом в интегральной форме] Пусть $f\ n+1$ раз непрерывно
дифференцируема на отрезке с концами $x_0$ и $x$. Тогда
$f(x)=\sum_{k=0}^n\frac{f^{(k)}(x_0)}{k!}(x-x_0)^k+\frac{1}{n!}\jcbyn{x_0}{x}{(x-t)^nf^{(n+1)}(t)}{t}$.
\end{theorem}

\билет  {13}


\begin{theorem}[первая теорема о среднем] Пусть $f$ интегрируема по $G$ на $[a,b]$ (в любом смысле), $G$ --
монотонна на $[a,b]$, $m=\inf_{[a,b]}f,\ M=\sup_{[a,b]}f$. Тогда существует число $\mu\in[m,M]$, что
$\jcbyn{a}{b}{f}{g}=\mu\jcbyn{a}{b}{}{G}=\mu(G(b)-G(a))$. \end{theorem}

\begin{imp} Пусть $f\hb[a,b],\ g\hb[a,b]$ и $g(x)\ge0$ на $[a,b]$, $m=\inf_{[a,b]}f,\ M=\sup_{[a,b]}f$. Тогда
существует число $\mu\in[m,M]$, что $\hbyn{a}{b}{fg}=\mu\hbyn{a}{b}{g}$. \end{imp}

\begin{imp} Пусть $f\ytgh[a,b]$, $g\hb[a,b]$  и $g(x)\ge0$ на $[a,b]$. Тогда существует $\xi\in[a,b]$, что
$\hbyn{a}{b}{fg}=f(\xi)\jbyn{a}{b}{g}$. \end{imp}

\begin{theorem}[вторая теорема о среднем] Пусть $F\ytgh[a,b]$, $g$ монотонна на $[a,b]$. Тогда существует
$\xi\in[a,b]$, что $\jcbyn{a}{b}{g}{F}=g(a)(F(\xi)-F(a))+g(b)(F(b)-F(\xi))$; Если $g$ не возрастает и
неотрицательна на $[a,b]$, то существует $\zeta\in[a,b]$, что $\jcbyn{a}{b}{g}{F}=g(a)(f(\zeta)-F(a))$; Если $g$
не убывает и неотрицательна на $[a,b]$, то существует $\zeta\in[a,b]$, что
$\jcbyn{a}{b}{g}{F}=g(b)(f(b)-F(\zeta))$. \end{theorem}

\begin{imp} Если $f\hb[a,b]$, $g$ -- монотонна на $[a,b]$, то существует $\xi\in[a,b]$, что
$\jbyn{a}{b}{fg}=g(a)\jbyn{a}{\xi}{f}+g(b)\jbyn{\xi}{b}{f}$; Если $g$ не возрастает и неотрицательна на $[a,b]$,
то существует $\zeta\in[a,b]$, что $\jbyn{a}{b}{fg}=g(a)\jbyn{a}{\zeta}{f}$; Если $g$ не убывает и неотрицательна
на $[a,b]$, то существует $\zeta\in[a,b]$, что $\jbyn{a}{b}{fg}=g(a)\jbyn{a}{\zeta}{f}$. \end{imp}


\билет  {14}


\begin{df} Пусть $f$ определена на $[a,b)$ и интегрируема в каком-либо смысле на всех отрезках $[a,\djky b],\
a<\djky b<b$. Если существует предел $I=\lim_{\djky b\rightarrow b-0}\jbyn{a}{\djky b}{f}$, то говорят, что $f$
интегрируема на $[a,b)$ в несобственном смысле. \end{df}

\begin{df} Функция удовлетворяет условию Коши несобственного интегрируемости на $[a,b)$, если $f$ определена на
$[a,b)$ и интегрируема в каком-либо смысле на всех отрезках $[a,\djky b],\ a<\djky b<b$ и $\люб\эпс>0\ \сущ
B_{\delta}(b)\ \люб b',b''\in B_{\delta}(b)\gth[a,b) \eck |\jbyn{b'}{b''}{f}|<\эпс$. \end{df}

\begin{theorem}[критерий Коши несобственной интегрируемости] Функция $f$ интегрируема в несобственном смысле на
$[a,b)$ тогда и только тогда, когда $f$ удовлетворяет условию Коши на $[a,b)$. \end{theorem}

\begin{df} $f$ абсолютно интегрируема на $[a,b)$, если $f$ определена на $[a,b)$, $f,\ |f|$ интегрируемы в
некотором смысле на всех отрезках $[a,b'],\ a<b'<b$ и существует предел $I=\liml{b'\rightarrow
b-0}\jbyn{a}{b'}{f}$. \end{df}

\begin{theorem} Если $f$ абсолютно интегрируема на $[a,b)$ в некотором смысле, то $f$ интегрируема на $[a,b)$ в
том же смысле. \end{theorem}

Признаки сходимости. \призн (сравнения) Пусть $f,\ g$ определены на  $[a,b)$ и интегрируемы на всех отрезках
$[a,b'],\ a<b'<b$. Пусть $0\le f(x)\le g(x)$ на $[a,b)$. Если $g$ интегрируема на $[a,b)$ в несобственном
смысле, то и $f$ интегрируема на $[a,b)$ в несобственном смысле и $\jbyn{[a,b)}{}{f}\le\jbyn{[a,b)}{}{g}$.
Если $f$ не интегрируема, то и $g$ не интегрируема.

\призн (сравнения) Пусть неотрицательные $f$ и $g$ определены на $[a,b)$ и интегрируемы на всех отрезках
$[a,b'],\ a<b'<b$. Пусть $0<C_1\le \frac{f(x)}{g(x)}\le C_2<\беск$. Тогда $f$ и $g$ одновременно интегрируемы
или неинтегрируемы в несобственном смысле на $[a,b)$.

\призн (Абеля) Если $f$ интегрируема в несобственном смысле по \рим\ на $[a,b)$, а $\фи$ ограниченная
монотонная функция на $[a,b)$, то $f\фи$ интегрируема в несобственном смысле по \рим\ на $[a,b)$.

\призн (Дирихле) Если $f$ такова, что интегралы \ри $\jbyn{a}{b'}{f}, \ a<b'<b$ существуют и ограничены в
совокупности, а $\фи$ монотонная функция на $[a,b)$, стремящаяся к 0, то $f\фи$ интегрируема в несобственном
смысле по \рим\ на $[a,b)$.

\билет {15}


\begin{df} Метрическое пространство.
\end{df}

\pt{1} Любое подмножество метрического пространства -- метрическое пространство с той же метрикой.

\begin{df} Нормированное пространство.
\end{df}

\begin{df} $\рр^n$.
\end{df}

\begin{lemma}[неравенство Коши-Буняк.-Шварца] Для любых $a_k,\ b_k\in\hh,\ k=1,\bnl,n$ верно неравенство
$$|\sum_{k=1}^n a_k b_k|\le \sqrt{\summ a_k^2}\sqrt{\summ b_k^2}.$$ \end{lemma}

\begin{df} Открытый и замкнутые шары, $\эпс$-окрестность точки.
\end{df}

\begin{df} Внутренняя, внешняя, граничная точки подмножества.
\end{df}

\begin{df} Множество $E\влж M$ называется открытым, если все его точки внутренние.
\end{df}

\begin{df} Замкнутое множество.
\end{df}

\begin{lemma} Открытый шар -- открытое множество, замкнутый -- замкнутое.
\end{lemma}

\begin{theorem} Любое объединение и конечное пересечение открытых множеств -- открытое множество.
\end{theorem}

\begin{theorem} Любое пересечение и конечное объединение замкнутых множеств -- замкнутое множество.
\end{theorem}

\begin{df} Предельная, изолированная точки и точка соприкосновения.
\end{df}

\begin{lemma} Точка $x$ -- предельная, если в любой её окрестности бесконечно много точек множества.
\end{lemma}

\begin{theorem} Следующие утверждения эквивалентны: $E$ -- замкнутое; содержит все свои граничные, или предельные
точки или все свои точки соприкосновения. \end{theorem}


\билет  {16}


\begin{df} Система множеств покрывает $E$, если $E$ -- подмножество их объединения.
\end{df}

\begin{df} Непустое множество в метрическом пространстве называется ограниченным, если оно лежит в некотором открытом шаре.
\end{df}

\begin{lemma} Непустое множество $E$ ограничено тогда и только тогда, когда $\sup_{x,y\in E}\rho(x,y)<\беск$.
\end{lemma}

\begin{df} Множество $K$ называется компактным, если из любого его покрытия открытыми множествами можно выбрать
конечное подпокрытие. \end{df}

\begin{theorem} Любой компакт -- ограниченное, замкнутое множество.
\end{theorem}

\begin{theorem} Любое замкнутое подмножество компакта -- компакт.
\end{theorem}

\begin{df} Брус в $\hh^n$.
\end{df}

\begin{lemma} Замкнутый брус в $\hh^n$ -- компакт.
\end{lemma}

\begin{theorem} Множество в $\hh^n$ компакт тогда и только тогда, когда оно ограничено и замкнуто.
\end{theorem}

\begin{theorem}[Больцано-Вейерштрасса] Любое бесконечно подмножество компакта имеет предельную точку.
\end{theorem}

\begin{imp} Каждое бесконечное ограниченное множество в $\hh^n$ имеет предельную точку.
\end{imp}

%\enlargethispage{3\baselineskip}
\билет  {17}


\begin{df} Последовательностью точек метрического пространства $M$ называется отображение \нн\ в $M$.
\end{df}

\begin{df} Предел последовательности.
\end{df}

\pt{1} (о пределе подпоследовательности)

\pt{2} (о единственности предела)

\pt{3} (об ограниченности последовательности, имеющей предел)

\pt{4} (об отделимости)

\begin{df} Бесконечно малая последовательность нормированного пространства.
\end{df}

\begin{lemma}
  Если $a_n$ -- последовательность из нормированного пространства $N$,
  то $\lim_{n\rightarrow\беск}a_n=a$ тогда и только тогда, когда
  $a_n-a=o^.(1)$.
\end{lemma}

\begin{lemma} Сумма и разность двух бесконечно малых -- бесконечно малая. Произведение бесконечно малой и
числовой -- бесконечно малая. \end{lemma}

\begin{theorem} Предел суммы и разности последовательностей есть сумма и разность пределов соответственно.
\end{theorem}

\begin{df} Последовательность Коши.
\end{df}

\begin{theorem} Всякая сходящаяся последовательность является последовательностью Коши.
\end{theorem}

\begin{df} Метрическое пространство называется полным, если любая последовательность Коши сходится к элементу
этого пространства. \end{df}

\begin{df} Нормированное пространство называется \выд{банаховым}, если это полное относительно стандартной
метрики пространство. \end{df}

\begin{theorem} $\hh^n$ -- банахово пространство.
\end{theorem}

\begin{theorem} Метрическое пространство полно тогда и только тогда, когда в нем любая вложенная
последовательность замкнутых шаров с радиусами, стремящимися к нулю имеет общую точку. \end{theorem}

\begin{lemma} Если в невырожденном нормированном пространстве один замкнутый шар вложен в другой, то радиус
первого меньше радиуса второго, и центры не дальше разности радиусов друг от друга. \end{lemma}

\begin{theorem} В банаховом пространстве любая последовательность замкнутых вложенных шаров имеет общую точку.
\end{theorem}

\begin{imp} В $\hh^n$ любая последовательность вложенных шаров имеет общую точку.
\end{imp}

\begin{theorem} В $\hh^n$ последовательность $\vec x^k$ сходится к $\vec x_0$ тогда и только тогда, когда она
сходится к $\vec x_0$ покоординатно. \end{theorem}


\билет  {18}


\begin{df} Функция $f$ имеет в точке $a\in M_1$ предел по множеству $E\влж M_1$, равный $b\in M_2$, если $a$ --
предельная точка $E$, $f$ определена на некоторой $B_{\Delta}'(a)\пер E$ и ... определения по Коши и по Гейне.
\end{df}

\begin{theorem} Определения эквивалентны.
\end{theorem}

\pt{1} (о пределе по подмножеству)

\pt{2} (о единственности)

\pt{3} (об ограниченности последовательности, имеющей предел)

\pt{4} (об отделимости)

\begin{df} Бесконечно малая функция по множеству $E$ в точке $a$.
\end{df}

\begin{theorem} $\lim_{x\rightarrow a}f(x)=b$ тогда и только тогда, когда $f(x)-b=o^.(1)$.
\end{theorem}

\begin{lemma} Сумма и разность бесконечно малых -- бесконечно малая. Произведение бесконечно малой и ограниченной
-- бесконечно малая (одна -- численная, вторая из метрического пространства). \end{lemma}

\begin{theorem} Если $f$ и $g$ -- отображения из метрического пространства $M$ в нормированное пространство $N$.
Тогда предел суммы и разности равен сумме и разности пределов. \end{theorem}

\begin{df} Пусть $f$ отображение из метрического пространства $M_1$ в метрическое пространство $M_2$. Функция,
удовлетворяющая критерию Коши. \end{df}

\begin{theorem} Если отображение имеет предел, то оно удовлетворяет условию Коши. Если второе метрическое
пространство полно, то отображение Коши имеет предел. \end{theorem}

\билет {19}


\begin{df} (Коши) Пусть $f$ -- отображение из метрического пространства $M_1$ в метрическое пространство $M_2$.
Функция $f$ непрерывна в точке $a$, если $f$ определена в некоторой окрестности точки $a$ и $\люб \эпс>0 \
\exists\delta>0\ \forall x\in B_{\delta}(a) \eck f(x)\in B_{\эпс}(f(a))$. \end{df}

\begin{df} (Гейне) Пусть $f$ -- отображение из метрического пространств $M_1$ в метрическое пространство $M_2$.
Функция $f$ непрерывна в точке $a$, если $f$ определена в некоторой окрестности точки $a$ и $\люб x_n\in M_1,\
x_n\rightarrow a \eck f(x_n)\xrightarrow[n\cnh\беск]{}f(a)$. \end{df}

\begin{theorem} Определения эквивалентны.
\end{theorem}

\begin{theorem} $f\отб(M_1,\rho_1)\cnh(M_2,\rho_2)$, а $\фи \отб(M_2,\rho_2)\cnh(M_3,\rho_3)$. Если $f$
непрерывна в точке  $a\in M_1$, а $\фи$ непрерывна в точке $b=f(a)\in M_2$, то $\фи(а)$ непрерывна в точке $a$.
\end{theorem}

\begin{theorem} Пусть $f,g\jvdy$ Если $f$ и $g$ непрерывны в точке $a$, то $f\pm g$ непрерывна в точке $a$. Если
\алф(\х)\  -- действительнозначная функция, непрерывная в точке \а\, то $\алф(x)f(x)$ непрерывна в точке $a$, а
если $\алф(a)\ne0$, то и $\frac{f(x)}{\alpha{x}}$ непрерывна в точке $a$. \end{theorem}

\begin{df} Функция $f\jvdv$ непрерывна в точке $a\in M_1$ по множеству $E$, если $f$ непрерывна в точке $a$ как
функция из метрического пространства $(E\пер\{a\},\rho_1$ в $(M_2,\rho_2)$. \end{df}

\begin{df} Пусть $f\jvdv$. Функция $f\ytgh M_1$, если $f$ непрерывна в каждой точке $M_1$. \end{df}

\begin{theorem}[критерий непрерывности] Пусть $f\jvdv$. Функция $f$ непрерывна на $M_1$ тогда и только тогда,
когда для любого $G$ -- открытого подмножества $M_2$ множество $f^{-1}(G)$ открыто в $M_1$. \end{theorem}

\begin{theorem} Пусть $f\омвм$, $K$ -- компакт в $M_1$. Тогда $f(K)$ -- компакт в $M_2$.
\end{theorem}

\begin{imp} Пусть $f$ --  отображение компакта $K\влж M_1$ в $M_2$. Тогда $f(K)$ -- компакт в $M_2$.
\end{imp}

\begin{theorem}[Вейерштрасса] Пусть $f\окмвм$ -- непрерывное отображение. Тогда $f(K)$ -- ограниченное множество в $M_2$.
\end{theorem}

\begin{theorem}[Вейерштрасса] Пусть $f$ -- отображение из компакта $K$ в $\hh$. Тогда $f$ принимает на $K$
наибольшее и наименьшее значение. \end{theorem}

\begin{df} Равномерная непрерывность.
\end{df}

\begin{theorem}[Кантора] Непрерывное отображение компакта в метрическое пространство является равномерно
непрерывной функцией на компакте. \end{theorem}

\begin{theorem}[принцип сжимающих отображений] Пусть $f\отб(M,\rho)\cnh (M,\rho)$, $M$ -- полное метрическое
пространство, $\люб x,y\in M \eck \rho(f(x),f(y))\le q \rho(x,y),\ 0\le q<1$. Тогда существует и единственна
неподвижная точка. \end{theorem}

%
\билет  {20}


\begin{df} Метрическое пространство называется несвязным, если его можно разбить на два непустых непересекающихся
открытых (замкнутых) множества, или, эквивалентно, в нём есть нетривиальное подмножество, одновременно открытое и
замкнутое. \end{df}

\begin{theorem} Метрическое пространство несвязно тогда и только тогда, когда существует непрерывная
действительнозначная функция на нем, принимающая ровно два значения. \end{theorem}

\begin{theorem} Множество $E$ в метрическом пространстве $M$ несвязно если, и только если существуют такие
открытые непересекающиеся множества $G_1$ и $G_2$ в $M$, что $G_1\gth E\ne \gcn,\ g_2\gth E\ne\gcn,\ G_1\объ
G_2\cjl E$. \end{theorem}

\begin{theorem} Множество в \рр\ связно тогда и только тогда, когда это промежуток.
\end{theorem}

\begin{imp} В \рр\ одновременно открытым и замкнутыми множествами являются только $\gcn,\ \hh$.
\end{imp}

\begin{theorem} Если метрическое пространство $M$ связно, а $F$ непрерывное отображение $M$ в $M^*$, то $f(M)$
связно в $M^*$. \end{theorem}

\begin{theorem}[Больцано-Коши] Непрерывная действительнозначная функция на связном множестве из метрического
пространства принимает все свои промежуточные значения. \end{theorem}

\begin{df} Кривой будем называть непрерывное отображение промежутка в метрическое пространство. Образ промежутка
будем называть носителем кривой. \end{df}

\begin{theorem} Носитель кривой -- связное множество.
\end{theorem}

\begin{df} Множество $E$ в метрическом пространство называется линейно связным, если $\люб x,y\in E$ найдется
кривая, носитель которой содержит точки $x$ и $y$ и лежит в $E$. \end{df}

\begin{theorem} Линейно связное множество связно.
\end{theorem}

\begin{df} Отрезком в нормированном пространстве $N$ с концами $a$ и $b$ называется множество $[a,b]=\{t\in N\eck
\ceo\alpha\in[0,1]\ :\ t=\alpha a+(1-\alpha)b\}$. \end{df}

\begin{df} Ломанной с узлами в точках $x_0,\bnl,x_n$ называется $[x_0,x_1]\объ\bnl\объ[x_{n-1},x_n]$. \end{df}

\begin{theorem} Открытое множество в нормированном пространстве связно тогда и только тогда, когда любые две его
точки можно соединить ломанной, принадлежащей этому множеству. \end{theorem}

%\enlargethispage{2\baselineskip}
\билет  {21}


\begin{df} Кривая -- это непрерывное отображение промежутка $I\влж\hh$ в метрическое пространство $M$. \end{df}

\begin{df} Путь, соединяющий точки $A$ и $B$ в метрическом пространстве -- это непрерывное отображение отрезка
$[a,b]\влж\рр$ в метрическое пространство, такое, что образ $a$ -- $A$, образ $b$ -- $B$. \end{df}

\begin{df} Пусть $f$ -- непрерывное отображение промежутка \И\ в метрическое пространство. Длиной кривой \ф\
называется $\var_I f=\sup_{a_0<\bnl<a_n,\ a_i\in I} \sum_{k=1}^n \rho(f(a_{k-1}),f(a_k))$. \end{df}

\begin{theorem} Пусть \ф\ -- непрерывное отображение отрезка \отр\ в метрическое пространство \М. Тогда
$\люб\эпс>0\ \ceo \delta>0\ \люб a_k,\ k=0,\bnl,n,\ a=a_0<a_1<\bnl<a_n=b$ с параметром разбиения $\max_{1<i\le
n}(a_i-a_{i-1})<\delta$ имеем: $\sum_{k=1}^n\rho(f(a_{k-1}),f(a_k))\in B_{\эпс}(\var_{\jnh}f)$. \end{theorem}

\begin{df} Пусть \ф\ -- отображение из \рр\ в нормированное пространство \Н. Производной \ф\ в точке $x_0$  будем
называть $\lim_{x\cnh x_0} \frac{f(x)-f(x_0)}{x-x_0}$  и будем обозначать $f'(x_0)\in N$. \end{df}

\begin{theorem} Пусть $f$ непрерывное отображение отрезка \отр\ в нормированное пространство $N$, которое имеет
производную всюду на \отр, кроме не более, чем счетного множества точек. Тогда \ф\ имеет конечную длину если и
только если существует интеграл $\rbyn{a}{b}{\|f'(x)\|}$, при этом в случае существования интеграла, он равен
$\var_{\jnh}f$. \end{theorem}

\begin{imp} Если \ф\ -- действительнозначная функция действительного переменного на \отр\ и конечная производная
существует всюду на \отр, кроме не более, чем счетного множества точек, то $\var_{\jnh}f=\rbyn{a}{b}{|f'(x)|}$.
\end{imp}

\begin{theorem} Если $\overrightarrow{f(x)}$ -- непрерывное отображения \отр\ в \рн, то \ф\ имеет ограниченную
вариацию тогда и только тогда, когда $f_i,\ i=1,\bnl,n$ имеют ограниченную вариацию. Функция $\vec f$
дифференцируема в точке $x_0$ тогда и только тогда, когда $f_i,\ i=1,\bnl,n$ дифференцируемы в точке $x_0$. При
этом, если $\vec f$ имеет производную всюду на \отр, кроме не более, чем счетного множества, то $\vec f$
ограниченной вариации на \отр тогда и только тогда, когда существует интеграл
$\rbyn{a}{b}{\sqrt{\summ(f'_i(x))^2}}$. При этом в случае ограниченной вариации $\var_{\jnh}\vec f$ равен этому
интегралу. \end{theorem}

\билет  {22}


\begin{df} Пусть $l(x)$ -- отображение линейного пространства $\алф_1$ в линейное пространство $\алф_2$. Функция $l(x)$ линейна, если из существования $l(x)$ следует, что для любого числа \алф\ $l(\alpha x)=\alpha l(x)$, и из существования $l(x)$ и $l(y)$ следует, что существует $l(x+y)=l(x)+l(y)$.
\end{df}

\begin{theorem} Если $l(x)$ непрерывна в какой-то точке, то она непрерывна во всех точках.
\end{theorem}

\begin{df} Пусть $f\jydy$. Отображение \ф\ дифференцируемо в точке $x_0$, если \ф\ определена в некоторой окрестности точки $x_0$ и существует такое непрерывное линейное отображение $l(x)\jydy$, что $f(x_0+\Delta x)-f(x_0)=l(\Delta x)+o^.(\|\Delta x\|)$. Линейный оператор $l(x)$ называется полным дифференциалом $f$ в точке $x_0$. Обозначают $l(x)=df(x_0,\dx)$.
\end{df}

\begin{theorem} Пусть $f\jydy$. Если \ф\ дифференцируемо в точке \хх, то \ф\ непрерывно в точке \хх.
\end{theorem}

\begin{lemma} Любое линейное отображение из \рн\ в \рр\ имеет вид $l(\vec x)=\sum_{k=1}^n a_k x_k$, где $a_k\in \hh$ и является непрерывным на \рн.
\end{lemma}

\begin{df} Пусть \ф\ -- отображение из \рн\ в \рр. Частной производной функции \ф\ в точке \вхх\ по \к-ой переменной называют обычную производную \ф\ как функции одной \к-ой переменной в точке $x^0_k$ при фиксации остальных переменных.
\end{df}

\begin{theorem} Пусть \ф\ -- отображение из \рн\ в \рр. Если \ф\ дифференцируемо в точке \вхх, то \ф\ имеет в этой точке все частные производные и её дифференциал $df(\vxx,\vdx)=\sum_{k=1}^n \xgh f(\vxx)\Delta x_k$.
\end{theorem}

\begin{df} Функция $f$ имеет в точке $\вхх$ непрерывную частную производную $\frac{\pd}{\pd x_k} f(\вхх)$, если $\frac{\pd f}{\pd x_k}$ определена в некоторой окрестности точки $\вхх$ и непрерывна в точке $\вхх$.
\end{df}

\begin{theorem} Если \ф\ имеет в точке \вхх\ все непрерывные частные производные, что \ф\ дифференцируемо в точке \вхх.
\end{theorem}

\begin{lemma} Если $l(\vec x)=\sum_{k=1}^n\alpha_k x_k$ -- линейное отображение \рн\ в \рр, то верна оценка: $|l(\vec x)|\le c\|\vec x\|$, где $c=\sqrt{\summ \alpha_k^2}$.
\end{lemma}

\begin{df} Поверхность $S$, являющаяся графиком функции \ф, имеет в точке $(\vxx,f(\vxx))\in S$ касательную плоскость, задаваемую уравнением $x_{n+1}-x_{n+1}^0=\summ\alpha_k(x_k-x_k^0)$, где $x_{n+1}^0=f(\vxx)$, если угол между прямой, проходящей через точки $(\vxx,x_{n+1}^0)$ и $(\vec x, f(\vec x))$ и прямой, проходящей через точки $(\vxx,x_{n+1}^0)$ и $(\vec x, x_{n+1}^0+\summ\alpha_k(x_k-x_k^0))$ стремится к нулю при $\вдх=\vec x-\vxx\cnh 0$.
\end{df}

\begin{theorem} Функция $f(\vec x),\ \vec x\in\hy$ дифференцируема в точке \вхх\ тогда и только тогда, когда её графике имеет в точке $(\vxx,f(\vxx))$ касательную плоскость. При этом уравнение касательной плоскости имеет вид $x_{n+1}-x_{n+1}^0=\summ \чп{f(\vxx)}{x_k}(x_k-x_k^0)$..
\end{theorem}


\билет {23}
%\enlargethispage{2\baselineskip}


\begin{df} Направлением в \рн\ будем называть любой вектор $\vec w\in\hy,\ \|\vec w\|=1$.
\end{df}

\begin{df} Если $f(\vec x)$ -- отображение из \рн\ в \рр\, то $f(\vec x)$ имеет в точке \вхх\ производную по направлению $\vec w,\ \|\vec w\|=1$, если существует $\frac{\pd f(\vxx)}{\pd \vec w}=\lim_{t\cnh+0}\frac{f(\vxx+t\vec w)-f(\vxx)}{t}$.
\end{df}

\begin{theorem} Если $f(\vxx)$ из \рн\ в \рр\ дифференцируемо в точке \вхх, то $f(\vec x)$ имеет в точке \вхх\ производную по любому направлению $\vec w$, равную $\xg{f(\vxx)}{\vec w}=\sum_{k=1}^n\xg{f(\vxx)}{x_k}w_k$.
\end{theorem}

\begin{df} Вектор $\grad f(\vec x)=\vkt{(\xg{f(\vec x)}{x_1},\bnl,\xg{f(\vec x)}{x_n})}$
называется градиентом функции \ф\ в точке $\vec x$.
\end{df}

\begin{theorem} Если $f(\vec x)$ дифференцируема в точке \вхх, то $\grad f(\vxx)\ne0$ задаёт направление наибольшей производной по направлению, причем такое направление единственное, а $-\grad f(\vxx)\ne 0$ задаёт направление  с наименьшеё производной, которое тоже единственно.
\end{theorem}

\hspace{-5mm} (Правила дифференцирования) Если $f(\vec x)$ и $g(\vec x)$ дифференцируемы в точке \вхх, то:

\pt{1} $f(\vec x)\pm g(\vec x)$ дифференцируема в точке \вхх\ и $d(f\pm g)(\vxx)=df(\vxx)\pm dg(\vxx)$.

\pt{2} $f(\vec x) g(\vec x)$ дифференцируема в точке \вхх\ и $d(fg)(\vxx)=g(\xx)df(\vxx)+f(\vxx)dg(\vxx)$.

\pt{3} Если $g(\vxx)\ne0)$, то $\frac{f(\vec x)}{g(\vec x)}$ дифференцируема в точке \вхх\ и
$d(\frac{f}{g})(\vxx)=\frac{g(\vxx)df(\vxx)-f(\vxx)dg(\vxx)}{g^2(\vxx)}$.

\pt{4} Если $f(\vec x)$ дифференцируема в точке \вхх, $x_k(\vec t),\ k=1,\bnl,n,\ \vec t\in \hh^m,\ \vec
x\in\hy$, дифференцируема в точке $\vec t^0$ и $x_k(\vec t^0)=x_k^0$, то функция $f(\vec x(\vec t))$
дифференцируема в точке $\vec t^0$ и $df(\vec x(\vec t))|_{\vec t^0}=\sum_{i=1}^n\sum_{j=1}^m
\xg{f(\vxx)}{x_i}\xg{x_i(\vec t^0)}{t_j}\Delta t_j$.

\begin{imp}
Если $f(\drn x)$ из \рн\ в \рр\ дифференцируема в точке \вхх, а функции $x_k(\drn t)$ из $\hh^m$ в \рр,
$k=1,\bnl,n$ имеют в точке $\drn t^0$ частные производные  $\frac{\pd x_k(\drn t^0)}{\pd t_i},\ k=1,\bnl,n,\
x_k(\vec t^0)=x_k^0,\ k=1,\bnl,n$, то $f(\drn x(\drn t))$, где $\vec x (\vec t)=(x_1(\vec t),\bnl,x_n(\vec
t)),$ имеет частную производную $\frac{\pd f(\vec x(\vec t))}{\pd t_i}=\summ \frac{\pd f(\vxx)}{\pd
x_k}\frac{x_k(\vec t^0)}{\pd t_i}$.
\end{imp}

\begin{imp}
(инвариантность первого дифференциала) Если $f(\vec x)$ дифференцируема в точке \вхх, и функции $x_k(\vec t),\
k=1,\bnl,n,\ \vec t\in \hh^m,\ \vec x\in\hy$, дифференцируемы в точке $\vec t^0$, и $x_k(\vec t^0)=x_k^0$, то
функция $f(\vec x(\vec t))$ дифференцируема в точке $\vec t^0$ и прямое вычисление дифференциала и
последовательное вычисление дифференциала приводит к одному и тому же результату.
\end{imp}

\begin{df} Если частная производная $\xg{f(\vec x)}{x_i}$ имеет в точке \вхх\ частную производную  по $x_j$, то её  называют частной производной второго порядка $\xxg{f(\vxx)}{x_j}{x_i}=\xg{}{x_j}\left(\xg{}{x_i}f(\vec x)\right)\Bigl|_{\vxx}$.
\end{df}

\begin{df} Если определена частная производная \н-ого порядка $\frac{\pd^n f(\vec x)}{\pd x_{i_n}\bnl\pd x_{i_1}}$ в точке $\vxx+t\vkt{e_{i_{n+1}}}$ при достаточно малых $t$, и существует её частная производная по переменной $x_{i_{n+1}}$, то её называют частной производной $n+1$-го порядка функции $f$.
\end{df}

\begin{theorem} Пусть $f(x,y)$ -- функция из $\hh^2$ в \рр, смешанная производная $\xxg{f(x,y)}{y}{x}$ существует в окрестности точки $(x_0,y_0)$ и непрерывна в ней, частная производная $\xg{f(x,y)}{y}$ существует в окрестности точки $(x_0,y_0)$ и непрерывна в ней по $y$. Тогда существует частная производная $\xxg{f(x_0,y_0)}{x}{y}=\xxg{f(x_0,y_0)}{y}{x}$.
\end{theorem}

\begin{imp} (теорема Шварца) Пусть $f(x,y)$ -- функция из $\hh^2$ в \рр. Если производные $\xxg{f(x,y)}{y}{x}$ и $\xxg{f(x,y)}{x}{y}$ существуют в окрестности точки $(x_0,y_0)$ и непрерыны в ней, то они равны.
\end{imp}

\begin{df} Функция \ф\ из \рн\ в \рр\ \м\ раз дифференцируема в точке \вхх, если все её частные производные порядка \м-1 дифференцируемы в этой точке. Фунцкия \ф\ \м\ раз непрерывно дифференцируема в точке \вхх, если все её частные производные порядка \м\ существуют в некоторой окрестности этой точки и непрерывны в ней.
\end{df}

\begin{theorem}[Юнга] Если \ф\ дважды дифференцируема в точке \вхх, то $\xxg{}{y}{x}f(\vxx)=\xxg{}{x}{y}f(\vxx)$.
\end{theorem}

\begin{df} Если первый дифференциал $df(\vec x,\vkt{\dx})$ при фиксированном $\vkt{\dx}$ как функция от \х\ дифференцируема в точке \вхх, то дифференциалом при том же фиксированном $\vkt{\dx}$ называется вторым дифференциалом и обозначается $d^2f(\vec x,\vkt{\dx})$.
\end{df}

\begin{df} Если \н-ый дифференциал $d^nf(\vec x,\vkt{\dx})$ при фиксированном $\vkt{\dx}$ как функция от $\vec x$ дифференцируем в точке \вхх, то полученный дифференциал при том же фиксированном $\vkt{\dx}$ называют \м+1-м дифференциалом в точке \вхх.  $d^{m+1}f(\vec x,\vkt{\dx})|_{\vec x=\vkt{\dx}}=\sum_{i_{m+1}=1}^n\bnl\sum_{i_1=1}^n \frac{\pd^{m+1}}{\pd x_{i_{m+1}}\bnl\pd x_{i_1}}\Delta x_{i_1}\bnl\Delta x_{i_{m+1}}$.
\end{df}

\билет  {24}


\begin{df} Если \ф\ \м\ раз дифф-ма в точке \вхх, то формула $f(\vec x)=f(\vxx+\vkt{\dx})=\sum_{k=0}^m\frac{1}{k!}d^k f(\vec x,\vkt{\dx})+r_m(\vec x,\vkt{\dx})$ называется формулой Тейлора порядка \м.
\end{df}

\begin{theorem}[Лагранж] Если \ф\ \м+1 раз дифференцируема на отрезке $[\vxx,\vxx+\vkt{\dx}]$, то $f(\vec x)=f(\vxx+\vkt{\dx})=\sum_{k=0}^m\frac{1}{k!}d^k f(\vec x,\vkt{\dx})+\frac{1}{(m+1)!}d^{m+1}f(\vxx+\theta\vkt{\dx},\vkt{\dx})$, где $0<\theta<1$.
\end{theorem}

\begin{theorem}[интегральная] Если \ф\ \м+1 раз непрерывно дифференцируема на отрезке $[\vxx,\vxx+\vkt{\dx}]$, то $f(\vec x)=f(\vxx+\vkt{\dx})=\sum_{k=0}^m\frac{1}{k!}d^k f(\vec x,\vkt{\dx})+\frac{1}{m!}\int_0^1(1-t)^md^{m+1}f(\vxx+t\vkt{\dx},\vkt{\dx})\, dt$.
\end{theorem}

\begin{theorem}[Пеано] Если \ф\ \м\ раз непрерывно дифф-ма в точке \вхх, то
$$f(\vec x)=f(\vxx+\vkt{\dx})=\sum_{k=0}^m\frac{1}{k!}d^k f(\vec x,\vkt{\dx})+o^.(\|\vkt{\dx}\|^m).$$
\end{theorem}

\билет {25}


\begin{df} Функция \ф\ из \рн\ в \рр\ имеет в точке \вхх\ локальный максимум, если \ф\ определена в некоторой окрестности точки \вхх\ и существует $B'_{\delta}(\vxx)$, что $\люб\vec x\in B'_{\delta}(\vxx) \eck f(\vec x)\le f(\vxx)$.
\end{df}

\begin{theorem} Если \ф\ имеет в точке \вхх\ локальный экстремум и существует производная $\xg{f(\vxx)}{x_i}$, то она равно нулю.
\end{theorem}

\begin{theorem} Если \ф\ дифф-ма в точке \вхх\ и имеет в ней локальный экстремум, то $\grad f(\vxx)=\vec 0$ и $df(\vxx,\vkt{\dx})\equiv0$.
\end{theorem}

\begin{theorem} Если \ф\ \м\ раз непрерывно дифференцируема в точке \вхх, $d^mf(\vxx,\vkt{\dx})$ -- первый отличный от нуля дифференциал в точке \вхх\ и $d^mf(\vxx,\vkt{\dx})$ строго положителен $(\люб \vkt{\dx})$, то \ф\ имеет в точке \вхх\ строгий локальный максимум. Если \ф\ \м\ раз непрерывно дифференцируема в точке \вхх, $d^mf(\vxx,\vkt{\dx})$ -- первый отличный от нуля дифференциал в точке \вхх\ и \ф\ имеет в точке \вхх\ локальный максимум, то $d^mf(\vxx,\vkt{\dx})$ не положителен.
\end{theorem}

%\hspace{-9mm}(кр. Сильвестра) $d^2f(\vkt{\dx})$ строго положителен тогда и только тогда, когда для всех \к\ определители $\det\left(\xxg{f(\vxx)}{x_j}{x_i}\right)_{1\le i,j\le k}>0$.
\билет {26}


\begin{df} Если функция $y(\vec x)$ задается как решение уравнения $F(y,\vec x)=0$, то говорят, что $y(\vec x)$ задана неявно
\end{df}

\begin{theorem}[о неявной функции] Пусть $F(y,\vec x)$ -- действительная функция $n+1$-го переменного $y$ и $\vec
x=(x_1,\bnl,x_n)$. Если $F(y^0,\vxx)=0$, \Ф\ непрерывна в окрестности точки $(y^0,\vxx)$ и имеет в точке
$(y^0,\vxx)$ непрерывную частную производную $\xg{}{y}F(y,\vec x)\ne 0$, $\ceo\эпс_0>0\ \люб\эпс\in(0,\эпс_0)\
\ceo B_{\delta}(\vxx)\ \люб\vec x\in B_{\delta}(\vxx)\ \сущ! y\in B_{\эпс}(y^0) \eck F(y,\vec x)=0$ (то есть $y$
задается как функция от $\vec x$). При этом $y(x)$ будет непрерывна в точке \вхх, а если $\xg{}{y}F(y,\vec x)$
непрерывна в некоторой окрестности точки $(y^0,\vxx)$, то $y(\vec x)$ будет непрерывна в некоторой окрестности
точки \вхх. Если \Ф\ дифференцируема в точке (в окрестности) $(y^0,\vec x)$, то $y(\vec x)$ будет дифференцируема
в точке (в окрестности) \вхх\ и $dy(\vkt{\dx})|_{\vec x}=-\summ\left(\frac{\xg{}{x_k}F(y,\vec
x)}{\xg{}{y}F(y,\vec x)}\right)\dx_k,\ y=y(\vec x)$ в точке (в окрестности) \вхх \end{theorem}

\begin{theorem} Пусть $F_i(\vec y,\vec x),\ i=1,\bnl,m$ --  функции $m+n$ переменных, $y\in\hh^m,\ x\in\hy$,
которые непрерывно дифференцируемы в окрестности точки $(\vec y^0,\vec x^0),\ F_i(\vec y^0,\vec x^0)=0,\
i=1,\bnl,m$, и определитель Якоби $\det\left(\xg{F_i(\vec y^0,\vec x^0)}{y_j}\right)_{1\le i,j\le m} \ne 0$.
Тогда $\сущ \эпс^0_j>0,\ j=1,\bnl,m,\ \люб\эпс_j\in(0,\эпс^0_j),\ j=1,\bnl,m,\ \сущ B_{\delta}(\vxx)\ \люб\vec x
\in B_{\delta}(\vec x)\ \сущ!\vec y=(y_1,\bnl,y_m)\in \prod_{j=1}^m B_{\эпс_j}\усл F_i(\vec y,\vec x)=0,\
i=1,\bnl,m$ и $y_j,\ j=1,\bnl,m$ как неявные функции от $\vec x$ непрерывно дифференцируемы в некоторой
окрестности точки \вхх. $-\xg{F_i(\vec y^0,\vxx)}{x_k}=\sum_{j=1}^m \xg{F_i(\vkt
y^0,\vxx)}{y_j}\xg{y_j(\vxx)}{x_k},\ i=1,\bnl,m$. Из этой системы \м\ линейных уравнений с \м\ неизвестными все
производные находятся. Матрица невырожденна, значит решение единственно. \end{theorem}


\билет  {27}

\begin{df} Функция \ф\ имеет в точке \вхх\ локальный условный максимум при условиях связи $f_i(\vec x)=0,\ i=1,\bnl,m$, если $f_i(\vxx)=0,\ i=1,\bnl,m$ и $\сущ B_{\delta}(\vxx)\ \люб\vec x\in B_{\delta}(\vxx)\пер \{\vec t\in\hy \eck f_i(t)=0,\ i=1,\bnl,m\} \eck f(\vec x)\le f(\vxx)$.
\end{df}

\hspace{-5mm}(Метод множителей Лагранжа) $F(\vec\lambda,\vec x)=f(\vec x)+\summ\lambda_kf_k(\vec x)$ При
некоторых условиях точки условного локального экстремума \ф\ будут среди критических точек функции Лагранжа.

\begin{theorem} Пусть функции $f_i(\vec x),\ i=0,\bnl,m$ непрерывно дифференцируемы в окрестности точки \вхх -- точки условного локального экстремума функции $f_0$ при условиях связи $f_i(\vec x)=0,\ i=1,\bnl,m$. Тогда $\grad f_i(\vxx)$ линейно зависим.
\end{theorem}

\begin{theorem} Пусть при условии теоремы 27.2 $\grad f_i(\vxx),\ i=1,\bnl,m$ линейно независим. Тогда существуют $\lambda_i,\ i=1,\bnl,m$, не все равные нулю, что $\grad f_0(\vxx)+\sum_{i=1}^m\grad f_i(\vxx)=0$.
\end{theorem}

\end{document}


%% Local Variables:
%% eval: (setq compile-command (concat "xelatex  -halt-on-error -file-line-error " (buffer-name)))
%% End:
