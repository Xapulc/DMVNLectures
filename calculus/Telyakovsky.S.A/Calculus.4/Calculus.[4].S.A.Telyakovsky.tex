\documentclass[a4paper]{article}
\usepackage[utf]{dmvn}

\newcommand{\pv}[1]{{p_{[#1]}}}
\newcommand{\di}[3]{{d {#1}_{#2_{#3}}}}

\newcommand{\elx}[1]{#1!x^{#1}}
\newcommand{\ely}[1]{#1!y^{#1}}

\newcommand{\ppht}[2]{\pf{\ph_{{#1}}}{t_{{#2}}}}
\newcommand{\pphx}[2]{\pf{\ph_{{#1}}}{x_{{#2}}}}

\newcommand{\pxt}[2]{\pf{x_{{#1}}}{t_{{#2}}}}
\newcommand{\pyx}[2]{\pf{y_{{#1}}}{x_{{#2}}}}

\newcommand{\pphti}[3]{\br{\ppht{#1}{#2}}_{{#3}}}
\newcommand{\pdf}[3]{\frac{\pd^2 #1}{\pd #2 \pd #3}}
\newcommand{\sumfi}{\sums{\mathrm{I}}}
\newcommand{\sumse}{\sums{\mathrm{II}}}

\newcommand{\bigsym}[2]{\text{\raisebox{#1}{{\LARGE $#2$}}}}

\newcommand{\Rbt}{\wt{\Rb}}

\begin{document}
\dmvntitle{Курс лекций по}{математическому анализу}{Лектор Сергей Александрович Теляковский}
{II курс, 4 семестр, поток математиков}{Москва, 2004 г.}
\pagebreak

\pagestyle{plain}
\tableofcontents
\pagebreak

\section*{Введение}

\subsection*{Предисловие}

{\scriptsize
{\itshape
{\hfill Еще одно, последнее сказанье,\par}
{\hfill И летопись окончена моя.\par}
{\hfill Исполнен долг, завещанный от Бога,\par}
{\hfill Мне, \TeX ману... Недаром многих лекций\par}
{\hfill Свидетелем Господь меня поставил,\par}
{\hfill \TeX нарскому искусству вразумил...\par}
\vskip 5pt
{\hfill ...Когда-нибудь студент трудолюбивый\par}
{\hfill Зайдёт на сайт, чтоб выкачать матан,\par}
{\hfill Найдёт postscript огромный безымянный,\par}
{\hfill Часа за два скачает этот хлам,\par}
{\hfill Затем бумаги изведёт немало,\par}
{\hfill Чтобы матан предстал его очам,\par}
{\hfill И будет ботать... Мне ж теперь\par}
{\hfill Пора бы отдохнуть немного задолбало...\par}
}}
\vskip 5pt

Настоящее издание представляет собой полный вариант лекций С.\,А.\,Теляковского по математическому
анализу, читаемых в IV семестре. Материал слегка переработан автором данного документа Вельтищевым
Михаилом, и, стоит надеяться, что это пошло на пользу. Нельзя не отметить активное участие некоторых людей
в обнаружении и исправлении ошибок, однако здесь всегда может оставаться некоторое количество
опечаток и неточностей. Убедительная просьба ко всем читателям: в случае обнаружения ошибок
немедленно сообщайте автору на \dmvnmail{} или загляните
на \dmvnwebsite{} и посмотрите, где можно достать в настоящее время
самого автора. Все пожелания и предложения по поводу оформления
и содержания документа будут обязательно приняты к сведению. Последнее обновление: егодня~года.

\subsection*{Слова благодарности}
Автор благодарит лектора за прочитанный курс лекций. Кроме того, нельзя не сказать
спасибо тому, кто попросил набрать эти лекции. Здесь было бы больше ошибок,
если бы не было людей, которые эти ошибки замечают. За обнаружение опечаток и ценные замечания
хочется поблагодарить М.\,Берштейна, Г.\,Бровко, А.\,Воронцова, С.\,Захарова, М.\,Левина,
П.\,Наливайко, В.\,Осокина, Я.\,Смирнова, А.\,Соколовскую, Д.\,Филимонова,
а особенно И.\,Вегнера и Д.\,Котенко.

\subsection*{Принятые в тексте соглашения и используемые сокращения}

\begin{points}{-3}
\item Класс ограниченных функций обозначается так: $f \bw\in \Bb(A)$ функция $f$ ограничена на
      $A$. Обозначение происходит от слова \emph{bounded}. Точно так же будут иногда обозначаться
      ограниченные множества: $A \bw\in \Bb$, причём подразумеваются ограниченные подмножества в $\Ebb^m$.
\item Интегрируемые в несобственном смысле на $D$ функции обозначаются $\Rbt(D)$.
\item Следуя \cite{rokhlin}, топологические понятия обозначаются сокращениями соответствующих английских слов.
      Так, $\Int A$ множество внутренних точек множества $A$, $\Cl A$ замыкание множества $A$.
\item Класс измеримых по Жордану множеств обозначается через $\Jc$.
\item Множество натуральных чисел $\N \bw= \hc{1,2,3,\dots}$, а $\N_0 \bw= \hc{0} \bw\cup \N$.
\item Характеристическая функция (индикатор) множества $A$ обозначается $\chi_A$.
\item Используются аббревиатуры <<ФНЛ>>=<<Формула Ньютона Лейбница>>,
      <<ФГО>>=<<Формула Гаусса Остроградского>>, <<ФКПЛ>>=<<Формула конечных приращений Лагранжа>>.
\item При изложении материала из линейной алгебры мы будем неявно ссылаться на \cite{vinberg}.
\end{points}

\begin{thebibliography}{4}
\setlength\itemsep{-.5mm}
\bibitem{lectures}
М.\,Н.\,Вельтищев. \emph{Конспекты лекций С.\,А.\,Теляковского по математическому анализу.} 2004.
\bibitem{zorich}
В.\,А.\,Зорич. \emph{Математический анализ.} М.: МЦНМО, 2002.
\bibitem{rokhlin}
В.\,А.\,Рохлин, Д.\,Б.\,Фукс. \emph{Начальный курс топологии.} М.: Наука, 1977.
\bibitem{vinberg}
Э.\,Б.\,Винберг. \emph{Курс алгебры.} М.: Факториал Пресс, 2002.
\end{thebibliography}

\pagebreak
\pagestyle{headings}

\makeatletter
  \renewcommand{\headheight}{11mm}
  \renewcommand{\headsep}{2mm}
  \renewcommand{\sectionmark}[1]{}
  \renewcommand{\subsectionmark}[1]{}
  \renewcommand{\subsubsectionmark}[1]{\markright{\thesubsubsection. #1}}
  \renewcommand{\@oddhead}{\vbox{\hbox to \textwidth{\scriptsize\thepage\hfil\rightmark\strut}\hrule}}
  \renewcommand{\@oddfoot}{\hfil\thepage\hfil}
\makeatother

\section{Кратные интегралы}

\subsection{Мера Жордана}

\subsubsection{Элементарные фигуры и измеримость}

Будем считать, что действия разворачиваются в конечномерном евклидовом пространстве $\Ebb^m$. Мы
изберём терминологию, связанную с площадями, а не с объёмами, поскольку часто будем проводить
рассуждения на примере $m\bw=2$, не ограничивая, тем не менее, общности. Это связано хотя бы с тем,
что в случае $m\bw=2$ удобно рисовать картинки. Точки пространства $\Ebb$ будем обозначать так:
$x\bw=(x_1\sco x_m) \bw\in \Ebb^m$.

\begin{df}
Пусть $A\bw=(a_1\sco a_m)$, $B\bw=(b_1\sco b_m)$ и $a_i \bw\le b_i$. Тогда \emph{$m$-мерным
прямоугольником} будем называть множество точек, которое в некоторой ортогональной системе
координат может быть записано в виде $\si\bw= \hc{x \cln x_i \bw\in [a_i,b_i]}$. \emph{Площадью} $\si$
назовём выражение $|\si| \bw{:=} \prodl{i=1}{m}(b_i\bw-a_i)$.
\end{df}

\begin{df}
Пусть $A, B \subs \Ebb^m$. Если $\Int A \bw\cap \Int B \bw= \es$, то $A \bw\cup B$ будем записывать
в виде $A \bw\oplus B$.
\end{df}

\begin{df}
Пусть $\si_1\sco\si_n$ прямоугольники, а $\Om \bw= \si_1\sop\si_n$. Тогда $\Om$ называется
\emph{элементарной фигурой}, и положим $|\Om| \bw{:=} |\si_1|\spl|\si_n|$. Ясно, что необходима
проверка корректности определения, поскольку одну и ту же фигуру $\Om$ можно разбить на
прямоугольники несколькими способами. Геометрически этот факт очевиден, а строгое доказательство
неинтересно и потому здесь не приводится.
\end{df}

\begin{df}
\emph{Простейшей элементарной фигурой} называется элементарная фигура, у которой каждая из сторон
всех образующих её прямоугольников параллельна системе координат, е найдется такая плоскость
$e_iOe_j$, которой эта сторона параллельна. Такие фигуры будем обозначать $\si_K$.
\end{df}

Приведём без доказательства свойства элементарных фигур и их площадей.

\begin{itemize}
\setlength\itemsep{-1mm}
\item Если $\si_1 \subs \si_2$, то $|\si_1|\bw\le |\si_2|$.
\item Значение площади инвариантно относительно движений в $\Ebb^m$.
\item Полуаддитивность и аддитивность: $|\si_K' \bw\cup \si_K''| \bw\le |\si_K'| \bw+ |\si_K''|$,
причём равенство достигается тогда и только тогда, когда $\Int \si_K' \bw\cap \Int \si_K'' \bw= \es$.
\item $|\si_K \bw\wo \si_K'| \bw\ge |\si_K|\bw- |\si_K'|$, причём равенство достигается тогда и только тогда,
когда $\Int \si_K' \subs \Int \si_K$.
\item Если простейшую элементарную фигуру $\si_K$ распилить параллельно системе координат, то
получатся две простейшие элементарные фигуры $\si_K'$ и $\si_K''$, причём
$\si_K\bw=\si_K'\bw\oplus\si_K''$.
\item Можно слегка раздвинуть или сузить стенки простейшей элементарной фигуры $\si_K$ параллельно
системе координат и получить фигуру $\si_K'$ так, что площадь изменится на сколь угодно малую
величину. Важно, что при раздвигании стенок имеем $\si_K \subs \Int \si_K'$, а при сужении $\si_K'
\subs \Int \si_K$.
\end{itemize}

\begin{df}
Зафиксируем прямоугольную систему координат и рассмотрим в $\Ebb^m$ сетку с шагом $h$. Будем
считать, что $h\bw=\frac{1}{2^N}$, где $N \bw\in \N$. Пусть $A$ ограниченное подмножество в $\Ebb^m$.
Пусть $\ol{\om}_N(A)$ объединение всех квадратиков сетки, которые содержат точки из $A$, а
$\ul{\om}_N(A)$ объединение тех квадратиков сетки, которые целиком заняты точками множества $A$.
Когда ясно, о каком множестве идёт речь, не будем это указывать. Очевидно, $\ul{\om}_N \subs
\ol{\om}_N$, и, кроме того, $\ol{\om}_1 \sups \ol{\om}_2 \sups \dots$ и $\ul{\om}_1 \subs
\ul{\om}_2 \subs \dots$. Отсюда следует, что существуют пределы $\mu^* \bw{:=}
\liml{i}|\ol{\om}_i|$ и $\mu_* \bw{:=} \liml{i}|\ul{\om}_i|$, причём $\mu^* \bw\ge \mu_*$. Эти два
предела называются соответственно \emph{верхней и нижней мерами} Жордана ограниченного множества
$A$. Если $\mu^*\bw=\mu_*$, то их общее значение называется \emph{мерой} Жордана множества $A$,
обозначается $\mu(A)$, и говорят, что $A$ \emph{измеримо по Жордану}. Необходима проверка
корректности определения, состоящая в проверке его независимости от системы координат. Это будет
сделано позже.
\end{df}

\begin{note}
В дальнейшем мы будем просто говорить <<измеримо>> вместо <<измеримо по Жордану>>, поскольку
никаких других мер, кроме жордановой, мы рассматривать не будем.
\end{note}

Рассмотрим два поучительных примера неизмеримых по Жордану множеств.

\begin{ex}
Множество $A\bw{:=}[0,1] \bw\wo \Q$ неизмеримо, ибо его верхняя мера равна 1, а нижняя нулю.
Действительно, какую бы мелкую сетку мы не взяли, в любом квадратике на отрезке найдутся точки из
множества $A$ и точки не из множества $A$. Поэтому в качестве $\ol{\om}_N$ придётся всегда брать
весь отрезок, а в качестве $\ul{\om}_N$ пустое множество.
\end{ex}

\begin{ex}
Пусть $\hc{r_k}$ нумерация всех рациональных чисел отрезка $[0,1]$. Пусть $\sum \al_k \bw=
\frac{1}{2}$, причём $\al_k \bw> 0$. Рассмотрим $A\bw{:=}\bigcup U_k$, где $U_k$ интервал длиной
$\al_k$ с центром в $r_k$. Ясно, что $A$ открыто, но не является измеримым, поскольку $\mu^* \bw= 1$,
а $\mu_*\bw\le \frac{1}{2}$. Действительно, при подсчёте верхней меры мы должны включить в
$\ol{\om}_N$ все квадратики, поскольку в каждом квадратике есть рациональные точки. С другой
стороны, $\mu_*$ не может превосходить суммы длин окрестностей $U_k$.
\end{ex}

\begin{theorem}
Каждая элементарная фигура измерима, и её мера равна её площади.
\end{theorem}
\begin{proof}
Без ограничения общности можно считать, что $A$ есть некоторый прямоугольник, не обязательно
параллельный системе координат. Пусть задана сетка с шагом $h$. Раздуем немного множество $A$,
получим множество $A'$, затем сузим немного множество $A$, получим множество $A''$. При этом
раздутие и сжатие надо производить так, чтобы толщина «рамки» $B \bw{:=}A' \bw\wo A''$ была не меньше
$Ch$, где $C$ константа, зависящая только от размерности $m$. Нам это нужно для того, чтобы
квадратик сетки помещался внутри рамки целиком. При подходящем выборе $C$ имеем $M
\bw{:=}\ol{\om}_N \bw\wo \ul{\om}_N \subs B$, откуда $|M| \bw\le |B|$. Отсюда следует измеримость
множества $A$, поскольку $|B|$ можно сделать сколь угодно малым при достаточно большом измельчении
сетки. Отсюда следует, что $\mu(A) \bw= |A|$, поскольку множество $A$ можно сколь угодно точно
«зажать» между $\ol{\om}_N$ и $\ul{\om}_N$.
\end{proof}

\begin{theorem}[Эквивалентные определения мер]
$$\mu^*(A) \bw= \infl{N}|\ol{\om}_N(A)| \bw= \infl{\si_K \sups A}|\si_K| \bw= \infl{\si \sups A}|\si|,$$
$$\mu_*(A) \bw= \supl{N}|\ul{\om}_N(A)| \bw= \supl{\si_K \subs A}|\si_K| \bw= \supl{\si \subs A}|\si|.$$
\end{theorem}
\begin{proof}
Докажем утверждение про верхнюю меру, а про нижнюю доказательство аналогично. В силу свойств
$\ol{\om}$ первое равенство в цепочке очевидно, кроме того, очевидно, что $\infl{N}|\ol{\om}_N| \bw\ge
\infl{\si_K \sups A}|\si_K| \bw\ge \infl{\si \sups A}|\si|$, поскольку при расширении базы $\inf$ его
значение может только уменьшиться.

Рассмотрим $\si \sups A$, тогда $\mu(\si) \bw= \infl{N}|\ol{\om}_N(\si)|$, поэтому $\fa \ep \bw> 0$ имеем
$\mu(\si) \bw+ \ep \bw> |\ol{\om}_N(\si)|$. В этом неравенстве меру можно заменить на верхнюю меру (они
равны), а $\ol{\om}_N(\si)$ на $\ol{\om}_N(A)$, поскольку от такой замены неравенство только
усилится. Отсюда $\mu^*(\si) \bw+ \ep \bw> |\ol{\om}_N(A)| \bw\ge \mu^*(A)$, поэтому $|\si| \bw+ \ep \bw>
\mu^*(A)$, значит, по крайней мере, $|\si| \bw\ge \mu^*(A)$. Значит, $\infl{\si \sups A} |\si|$ не
может быть меньше $\mu^*(A)$. Но выше мы доказали, что он не может быть и больше, поэтому они
равны. Следовательно, зажатое между ними число $\infl{\si_K\sups A}|\si_K|$ также с ними совпадает.
\end{proof}

\begin{note}
Теперь можно считать, что произведена проверка корректности определения меры, поскольку доказано,
что верхняя и нижняя меры не зависят от системы координат.
\end{note}

\subsubsection{Свойства измеримых множеств}

\begin{theorem}[Критерий измеримости]
$A \bw\in \Jc \Lra \fa \ep \bw> 0 \exi \si_K', \si_K''\cln \si_K' \subs A \subs \si_K''$ и $\mu(\si_K'')
\bw- \mu(\si_K') \bw< \ep$.
\end{theorem}
\begin{proof}
Пусть $A\bw\in \Jc$. Тогда $\mu(A) \bw= \mu^*(A) \bw= \mu_*(A)$. В силу эквивалентного определения мер и
свойств точной верхней и точной нижней граней имеем $\fa \ep \bw> 0 \exi \si_K'' \sups A\cln \mu(A) \bw+
\ep \bw> \mu(\si_K'')$ и $\exi \si_K' \subs A\cln \mu(A) \bw- \ep \bw< \mu(\si_K')$. Отсюда следует, что
$\mu(\si_K')$ и $\mu(\si_K'')$ лежат в интервале $\br{\mu(A) \bw-\ep, \mu(A)\bw+\ep}$, поэтому
$\mu(\si_K'') \bw- \mu(\si_K') \bw< 2\ep$, что и требуется.

Обратно, пусть выполнено второе условие теоремы. В силу эквивалентного определения мер имеем
$\mu^*(A) \bw\le \mu^*(\si_K'')\bw=\mu(\si_K'')$, а $\mu_*(A) \bw\ge \mu_*(\si_K')\bw=\mu(\si_K')$. Но мы
знаем, что $\mu^*(A) \bw\ge \mu_*(A)$, поэтому $\mu^*(A) \bw-\mu_*(A) \bw\le \ep$, значит, они совпадают.
\end{proof}

\begin{imp}
$A\bw\in\Jc \Lra \fa \ep \bw> 0$ найдутся замкнутое множество $F \bw\in \Jc$ и открытое $U \bw\in \Jc$,
для которых $F \subs A \subs U$ и $\mu(U) \bw- \mu(F) \bw< \ep$.
\end{imp}
\begin{proof}
Возьмём множества $\si_K'$ и $\si_K''$ из предыдущей теоремы. Объемлющее множество $\si_K''$ слегка
раздуем и уберём его границу, получим открытое множество $U$. Мы уже замечали, что при правильном
раздувании сохранится свойство $A \subs U$. Далее, получим множество $F$ сужением $\si_K'$ и
добавлением его границы, тогда $F$ будет замкнуто. Но поскольку мы при этом изменили площадь не
сильно, рассуждения предыдущей теоремы остаются в силе.
\end{proof}

\begin{theorem}[Критерий измеримости в терминах границы]
Пусть $A\in\Bb$. Тогда $A \bw\in \Jc \Lra \mu(\pd A) \bw=0$.
\end{theorem}
\begin{proof}
Пусть $A$ измеримо. По предыдущей теореме $\fa \ep \bw> 0 \exi \si_K' \subs A \subs \si_K''$, причём
$\si_K'$ замкнуто, $\si_K''$ открыто, а $\mu(\si_K'') \bw- \mu(\si_K') \bw< \ep$. Ясно, что $\pd A \subs
\si_K'' \bw\wo \si_K'$, поэтому $\mu^*(\pd A) \bw\le \mu(\si_K'' \bw\wo \si_K') \bw= \lcomm$ поскольку мера
элементарных фигур совпадает с их площадью $\rcomm \bw= \mu(\si_K'') \bw- \mu(\si_K') \bw< \ep$. Поскольку
$\ep$ можно взять сколь угодно малым, получаем $\mu^*(\pd A) \bw= 0$, откуда $\mu(\pd A) \bw= 0$.

Обратно, если $\mu(\pd A) \bw= 0$, то $\fa \ep \bw> 0 \exi \si_K \sups \pd A\cln \mu(\si_K) \bw< \ep$. Тогда
рассмотрим $\si_K'' \bw{:=}A \bw\cup \si_K$ и $\si_K' \bw{:=}A \bw\wo \si_K$. Ясно, что это действительно
будут простейшие элементарные фигуры, поскольку добавляя или выкидывая $\si_K$, мы полностью меняем
границу множества $A$, делая её «хорошей». Далее, имеем $\mu(\si_K'') \bw- \mu(\si_K') \bw< \ep$, и
осталось сослаться на критерий измеримости.
\end{proof}

\begin{lemma}
Пусть $\mu(A) \bw= 0, \mu(B) \bw= 0$. Тогда $\mu(A \bw\cup B) \bw= 0$.
\end{lemma}
\begin{proof}
Пусть $\ep \bw> 0$. Рассмотрим $\si_K^A \sups A\cln \mu(\si_K^A) \bw< \ep$ и $\si_K^B \sups B\cln
\mu(\si_K^B) \bw< \ep$. Тогда $A \bw\cup B \subs \si_K^A \bw\cup \si_K^B$, откуда $\mu^*(A \bw\cup B) \bw\le \ep \bw+
\ep \bw= 2\ep$.
\end{proof}

\begin{theorem}
Пусть $A, B \bw\in \Jc$. Тогда $A \bw\cup B$, $A \bw\cap B$ и $A \bw\wo B$ измеримы.
\end{theorem}
\begin{proof}
Несложно видеть, что $\pd(A \bw\cup B) \subs \pd A \bw\cup \pd B$. Поскольку $\mu(\pd A \bw\cup \pd B) \bw= 0$, имеем
$\mu\br{\pd(A \bw\cup B)} \bw= 0$, и осталось сослаться на критерий измеримости в терминах границы.
Для пересечения и разности рассуждения аналогичны, поскольку $\pd(A \bw\cap B) \subs \pd A \bw\cup \pd B$ и
$\pd(A \bw\wo B) \subs \pd A \bw\cup \pd B$.
\end{proof}

\begin{theorem}
Пусть $A, B \bw\in \Jc$. Тогда $\mu(A \bw\cup B) \bw\le \mu(A) \bw+ \mu(B)$, а если
$\Int A \bw\cap \Int B \bw= \es$, то неравенство
обращается в равенство.
\end{theorem}
\begin{proof}
Пусть $\ep \bw> 0$, тогда $\exi \si_K^A \sups A, \si_K^B \sups B$, для которых $\mu(\si_K^A) \bw< \mu(A)
\bw+ \ep$ и $\mu(\si_K^B) \bw< \mu(B) \bw+ \ep$. Поскольку $A \bw\cup B \subs \si_K^A \bw\cup \si_K^B$, получаем
$\mu(A \bw\cup B) \bw\le \mu(\si_K^A \bw\cup \si_K^B) \bw\le \mu(\si_K^A) \bw+
\mu(\si_K^B) \bw< \mu(A) \bw+ \mu(B) \bw+ 2\ep$, поэтому $\mu(A \bw\cup B) \bw\le \mu(A) \bw+ \mu(B)$.

Пусть имеет место $A \bw\oplus B$. Тогда $\fa \ep \bw> 0 \exi \si_K^A \subs A, \si_K^B \subs B$, для
которых $\mu(\si_K^A) \bw> \mu(A) \bw- \ep$ и $\mu(\si_K^B) \bw> \mu(B) \bw- \ep$. Поскольку для $\si_K^A$ и
$\si_K^B$ тем более выполнено $\si_K^A \bw\oplus \si_K^B$, имеем $\mu(A \bw\cup B) \bw\ge \mu(\si_K^A \bw\cup
\si_K^B) \bw= \mu(\si_K^A) \bw+ \mu(\si_K^B) \bw> \mu(A) \bw+ \mu(B) \bw- 2\ep$. Итак,
$\mu(A \bw\cup B) \bw\ge \mu(A) \bw+ \mu(B)$, значит, верно и обратное неравенство.
\end{proof}

\begin{imp}[Аддитивность меры]
Если $A_i \bw\in \Jc$ и $A \bw= A_1 \sop A_n$, то $\mu(A) \bw= \mu(A_1)\spl\mu(A_n)$.
\end{imp}

\begin{note}
Сказанное выше, а также пример неизмеримого счётного множества $A \bw{:=}\Q \bw\cap [0,1]$ показывает,
что класс $\Jc$ является алгеброй, но не $\si$-алгеброй. Можно доказать, что мера Жордана является
счётно-аддитивной на алгебре элементарных фигур, но мы не будем этого делать.
\end{note}

Приведём некоторые \textbf{примеры измеримых множеств}.

\pt{1}. Пусть $f \bw\in \Rb[a,b]$ и $f \bw\ge 0$. Тогда её подграфик, е множество $\bc{(x,y) \vl x \bw\in
[a,b], \; y\bw\in \bs{0, f(x)}}$ измеримо по Жордану. Это следует из геометрической интерпретации
верхних и нижних сумм Дарбу.

\pt{2}. Из \pt{1} следует измеримость фигуры с границей, представимой в виде графиков непрерывных функций.

\pt{3}. Пусть $\Ga \bw{:=}\bc{r(t) \vl t \bw\in [a,b]}$ непрерывная плоская спрямляемая кривая
длины $S$. Покажем, что $\mu(\Ga) \bw= 0$. Разделим кривую на $n$ равных по длине частей, тогда длина
каждого куска будет $\frac{S}{n}$. Накроем каждый кусок кругом радиуса $\frac{S}{n}$, тогда
$\mu^*(\Ga) \bw\le n \cdot \pi\hr{\frac{S}{n}}^2 \bw= \frac{\pi S^2}{n} \ra 0$ при $n \ra \infty$.

\begin{df}
Пусть множество $A \subs \Ebb^m$ не является ограниченным. Рассмотрим концентрическую систему расширяющихся
квадратов $\si_1,\si_2,\si_3,\dots$ со сторонами $1, 2, 3, \dots$. Если существует предел $\liml{n
\ra \infty} \mu(\si_n \bw\cap A)$, то значение этого предела называют \emph{мерой} множества $A$ и
обозначают $\mu(A)$. Из определения ясно, что каждое из множеств $\si_n \bw\cap A$ должно быть
измеримым.
\end{df}

\begin{note}
Можно доказать, что это определение не зависит от системы координат. Однако в дальнейшем нам
не потребуется измерять неограниченные множества.
\end{note}

\subsection{Кратные интегралы}

\subsubsection{Определение кратного интеграла Римана}

\begin{df}
Пусть $A \bw\in \Jc$. Назовём набор $A_1\sco A_n$ \emph{разбиением} $A$, если $A \bw= A_1 \sop A_n$ и
$A_i \bw\in \Jc$. Разбиения обычно будем обозначать буквой $T$. Из определения следует, что $\mu(A) \bw=
\mu(A_1)\spl \mu(A_n)$. Будем иногда обозначать количество элементов разбиения $T$ символом $|T|$.
\end{df}

\begin{df}
\emph{Диаметром} непустого множества $A$ назовём число $\la(A) \bw{:=}\supl{x,y \bw\in A} \rho(x,y)$.
\emph{Диаметром} разбиения $T$ назовём число $\la_T \bw{:=}\max \hc{\la(A_1)\sco \la(A_n)}$.
\end{df}

\begin{df}
Пусть $T\cln A\bw=A_1\sop A_n$ разбиение $A\bw\in \Jc$. Пусть $f \cln A \ra K$, где $K$ поле $\R$ или
$\Cbb$. Пусть $P \bw{:=}\hc{p_k \vl p_k \bw\in A_k}$. Составим \emph{интегральную сумму Римана} $S_T(f, P)
\bw{:=}\sumkun f(p_k)\mu(A_k)$. Функция $f$ называется \emph{интегрируемой по измеримому множеству} $A$, если
$\exi I \bw\in K\cln \fa \ep \bw> 0 \exi \de\bw>0\cln \fa T(A)\cln \la_T \bw< \de, \fa P$ имеем $|S_T(f,P)
\bw- I| \bw< \ep$. Тогда $I$ называется \emph{интегралом} функции $f$ по множеству $A$ и обозначается
$\ints{A}f \,d x$. При этом пишут, что $f \bw\in \Rb(A)$.
\end{df}

Очевидно, если $f \equiv L$, то $\ints{A}f \,dx \bw= L \mu(A)$.

\begin{note}
Мы знаем, что для функций одной переменной из интегрируемости следует ограниченность. Новый
интеграл уже не обладает таким свойством: любая функция интегрируема по множеству меры нуль.
\end{note}

\begin{df}
Обозначим через $B(x,h)$ открытый шар радиуса $h$ с центром в точке $x \in \Ebb^m$.
Положим $A(h)\bw{:=} A \bw\cap \hr{\cups{x\bw\in\Int A} B(x,h)}$ и назовём $A(h)$ \emph{подстриженным
под} $h$ множеством $A$. Поясним, откуда берётся такое название. Ясно, что $\Int A \subs A(h)$,
кроме того, мы оставляем в $A(h)$ только те точки из $\pd A$, для которых существуют точки из $\Int
A$ ближе, чем на расстоянии $h$.
\end{df}

\begin{note}
Если $\Int A \bw= \es$, е $A \bw= \pd A$, то $A(h) \bw= \es$.
\end{note}

\begin{theorem}
Пусть $\mu(A) \bw> 0$ и $f \bw\in \Rb(A)$, тогда $\exi h \bw> 0\cln f \bw\in \Bb\br{A(h)}$.
\end{theorem}
\begin{proof}
В силу интегрируемости, $\exi \de\bw>0\cln \fa T(A)\cln \la_T \bw< \de, \fa P$ имеем $|I \bw- S_T(P)| \bw< 1$.
Рассмотрим $h \bw{:=}C\de$ и сетку с шагом $h$. Здесь $C$ зависит только от $m$ и таково, что
диаметр элемента сетки не превосходит $\frac{\de}{3}$. Такие тонкости нужны нам для того, чтобы
диаметр элемента сетки, взятого вместе со своими соседями, не превосходил $\de$. Под соседями
элемента мы подразумеваем те и только те элементы разбиения, которые могут иметь с данным элементом
общие точки.

Рассмотрим произвольный элемент сетки $\si$, который содержит точки из $\Int A$. Добавим к $\si$
всех его соседей, которые не содержат точек из $\Int A$, затем пересечём всё это с множеством $A$,
получим множество $A_1$. Теперь пройдёмся по всем таким $\si$ и построим множества $A_2,A_3\sco
A_n$. Заметим, что $\hc{A_i}$ образуют разбиение некоторого подмножества $A$, поскольку общих
внутренних точек они не имеют. Положим $A^* \bw{:=}A_1\sop A_n$. Легко видеть, что диаметр этого
разбиения меньше $\de$. Кроме того, $A(h) \subs A^*$, поскольку все элементы сетки $\si$,
содержащие внутренние точки, были окружены слоем соседей толщиной $h$. Заметим, что $\fa k$ имеем
$\mu(A_k) \bw> 0$, ибо каждое $A_k$ содержит точки из $\Int A$. Повторяя слово в слово рассуждения,
которые мы проводили для интегрирования на отрезке, получаем, что $f$ ограничена на $A^*$, значит,
она ограничена и на $A(h)$.
\end{proof}

\begin{note}
Теорема остаётся верной и в случае, когда $A(h)$ пусто, однако утверждение «функция ограничена на
пустом множестве» не очень содержательно, хотя и истинно.
\end{note}

\begin{imp}
Пусть $f \bw\in \Rb(A)$, причём $A$ прямоугольник. Тогда $f \bw\in \Bb(A)$.
\end{imp}

\begin{theorem}\label{ah.bounded.function}
Пусть $f \bw\in \Rb(A)$, и $\exi h \bw> 0\cln g\bw=f$ на $A(h)$. Тогда $g \bw\in\Rb(A)$ и
$\ints{A}g\,dx \bw= \ints{A} f\,d x$.
\end{theorem}
\begin{proof}
Рассмотрим разбиение $T(A)$, для которого $\la_T \bw< Ch$. Здесь $C$ некоторая константа, зависящая от $m$.
Рассмотрим все $A_k$ из $T$, для которых $A_k \bw\cap \br{A \bw\wo A(h)} \neq \es$, назовём их
\emph{плохими}. В силу условия на диаметр $T$, для плохих $A_k$ имеем $\mu(A_k) \bw= 0$. Для всех остальных
\emph{хороших} $A_k$ имеем $A_k \bw\cap\br{A \bw\wo A(h)} \bw= \es$, откуда $A_k \subs A(h)$. Разобьём
интегральную сумму на две суммы: к первой отнесём все хорошие $A_k$, а ко второй все плохие. Вторая сумма,
очевидно, равна нулю, поскольку в ней все меры нулевые. В первой сумме можно $f$ заменить на $g$, поэтому
если $S_T(f, P)$ стремится к $\ints{A}f \,dx$, то и $S_T(g,P)$ стремится к $\ints{A}f \,dx$, что и
требовалось доказать.
\end{proof}

\subsubsection{Суммы Дарбу}

В этом параграфе будем считать, что $A \bw\in \Jc$, а $f\cln A \ra \R$ и $f\bw\in \Bb(A)$.

\begin{df}
Рассмотрим $T\cln A \bw= A_1\sop A_n$. Положим
$$m_k(f) \bw{:=}\infl{x\bw\in A_k} f(x),\quad M_k(f) \bw{:=} \supl{x\bw\in A_k} f(x), \quad
\usd{T}(f) \bw{:=}\sumkun M_k\mu(A_k), \quad \lsd{T}(f) \bw{:=} \sumkun m_k \mu(A_k).$$
Величины $\usd{T}$ и $\lsd{T}$ назовём \emph{верхней и нижней суммами Дарбу} для функции $f$ по разбиению $T$.
\end{df}

\begin{lemma}
Для сумм Дарбу выполнено: $\fa P$ имеем $\lsd{T} \bw\le S_T(P) \bw\le \usd{T}$,
$\fa T', T''$ имеем $\lsd{T'} \bw\le \usd{T''}$.
\end{lemma}
\begin{proof}
Первое свойство очевидно из определения. Ясно также, что при измельчении разбиения верхняя сумма
Дарбу может только уменьшиться, а нижняя сумма Дарбу только увеличиться. Чтобы доказать второе,
рассмотрим измельчение $T \bw{:=} T' \bw\cup T''$, которое состоит не более чем из $|T'| \cdot |T''|$
элементов вида $A'_i \bw\cap A''_j$. Ясно, что $\usd{T} \bw\le \usd{T''}$, а $\lsd{T'}\bw\le \lsd{T}$. Кроме
того $\lsd{T} \bw\le \usd{T}$. Объединяя эти неравенства, получаем $\lsd{T'} \bw\le \lsd{T}\bw\le \usd{T}
\bw\le \usd{T''}$.
\end{proof}

\begin{df}
Числа $I^*(f) \bw{:=}\infl{T} \usd{T}(f)$ и $I_*(f) \bw{:=}\supl{T} \lsd{T}(f)$ называются соответственно
\emph{верхним и нижним интегралом Дарбу} функции $f$.
\end{df}

\begin{note}
В случае, когда ясно, о какой функции идёт речь, мы будем опускать значки $(f)$ у символов верхних
и нижних сумм Дарбу, а также у интегралов Дарбу.
\end{note}

\begin{theorem}[I-й критерий Дарбу]
$f \bw\in \Rb(A) \Lra \fa \ep \bw> 0 \exi \de \bw> 0\cln \fa T(A)\cln \la_T \bw< \de$ имеем
$\usd{T} \bw- \lsd{T} \bw< \ep$.
\end{theorem}
\begin{proof}
Пусть функция интегрируема, обозначим через $I$ её интеграл. Тогда $\fa \ep \bw> 0 \exi \de > 0\cln
\fa T(A)\cln \la_T \bw< \de, \fa P$ имеем $I \bw- \ep \bw< S_T(P) \bw< I \bw+ \ep$. Переходя к $\sup$ и $\inf$,
получаем $I \bw- \ep \bw\le \lsd{T} \bw\le \usd{T} \bw\le I \bw+ \ep$. Отсюда
$\usd{T} \bw- \lsd{T} \bw\le 2\ep$, что и требовалось доказать.

Обратно, пусть выполнено условие $\fa \ep \bw> 0 \exi \de \bw> 0\cln \fa T(A)\cln \la_T \bw< \de$ имеем
$\usd{T} \bw- \lsd{T} \bw< \ep$. Из определения верхнего интеграла и свойств сумм Дарбу вытекает, что
$\lsd{T} \bw\le I^* \bw\le \usd{T}$. С другой стороны, $\lsd{T} \bw\le S_T(P) \bw\le \usd{T}$, поэтому для
таких разбиений имеем $|S_T(P) \bw- I^*| \bw< \ep$. Но по определению интеграла это означает, что $f$
интегрируема, и $I^*$ является значением этого интеграла. Но вместо $I^*$ можно было бы подставить
и $I_*$, тогда получается, что $\ints{A}f \bw= I^* \bw= I_*$. Таким образом, попутно доказано ещё и то,
что если функция интегрируема, то значения верхнего и нижнего интегралов совпадают и равны
интегралу функции $f$.
\end{proof}

\begin{theorem}[II-й критерий Дарбу]
$f\bw\in \Rb(A) \Lra \fa \ep \bw> 0 \exi T(A)\cln \usd{T} \bw- \lsd{T} \bw< \ep$.
\end{theorem}
\begin{proof}
В одну сторону теорема уже доказана. Обратно, пусть разбиение $T^*\cln A \bw= A^*_1\sop A^*_n$
удовлетворяет условию теоремы. Надо показать, что тогда для достаточно мелких разбиений всё будет
хорошо. Проблемы могут возникнуть на стыке элементов разбиения $T^*$, поэтому потребуется окружить
стыки множеством малой меры. По соглашению имеем $|f| \bw\le B$. Положим $\Ga \bw{:=} \cups{k} \pd
A^*_k$. Поскольку $\mu(\Ga) \bw= 0$, $\exi \si_K \sups \Ga\cln \mu(\si_K) \bw< \ep$. Теперь раздуем
стенки $\si_K$ на такое малое $\de$, чтобы площадь его выросла не более, чем на $\ep$. Теперь можно
считать, что $\Ga \subs \Int \si_K$, а $\mu(\si_K) \bw<2\ep$. Число $\de$ мы и возьмём в качестве
ограничителя диаметров разбиений.

Рассмотрим произвольное $T\cln A \bw= A_1\sop A_N$, для которого $\la_T \bw< \de$. Назовём элемент
разбиения $T$ \emph{плохим}, если он зацепляет несколько элементов из $T^*$, и \emph{хорошим} в
противном случае. Разделим разность сумм Дарбу на хорошую $\chi$-сумму и плохую $\pi$-сумму:
$$\usd{T} \bw- \lsd{T} \bw= \suml{1}{N}(M_k \bw- m_k)\mu(A_k) \bw=
\sums{\chi} (M_k \bw- m_k)\mu(A_k) \bw+ \sums{\pi}(M_k \bw- m_k)\mu(A_k).$$

Поскольку мы раздвинули $\si_K$, имеем $\cups{\pi}A_k \subs \si_K$. Тогда
$\sums{\pi}(M_k\bw-m_k)\mu(A_k) \bw\le 2B\sums{\pi} \mu(A_k) \bw\le 2B \mu(\si_K) \bw\le 4B\ep$.
Осталось оценить $\chi$-сумму, вспомнив её определение:
\begin{multline*}
\sums{\chi}(M_k \bw- m_k)\mu(A_k) \bw= \sums{i}\sums{A_k \subs A_i^*} (M_k\bw-m_k)\mu(A_k) \bw\le
\sums{i}\sums{A_k \subs A_i^*} (M^*_i\bw-m^*_i)\mu(A_k) =\\=
\sums{i} (M^*_i \bw- m^*_i) \sums{A_k \subs A_i^*}\mu(A_k) \bw\le \sums{i} (M^*_i \bw- m^*_i) \mu(A^*_i) \bw< \ep.
\end{multline*}
Итак, обе суммы оценены числом $\ep$. Осталось применить первый критерий Дарбу.
\end{proof}

\begin{imp}
«Старый» интеграл по отрезку и «новый» интегралы совпадают: $\intl{a}{b}f\,d x \bw= \ints{[a,b]}f\,d x$.
\end{imp}

\begin{imp}[Критерий интегрируемости в терминах интегралов Дарбу]
$f \bw\in \Rb(A) \Lra I^* \bw= I_*$.
\end{imp}
\begin{proof}
В одну сторону это уже доказано. Обратно, пусть верхний и нижний интегралы совпадают, обозначим их
общее значение через $I$. Тогда в силу их определений $\fa \ep \bw> 0 \exi T'(A)\cln \lsd{T'} \bw> I \bw-
\ep$ и $\exi T''(A)\cln \usd{T''} \bw< I \bw+ \ep$. Рассмотрим измельчение $T \bw{:=} T' \bw\cup T''$, тогда
$\usd{T} \bw- \lsd{T} \bw< 2\ep$, и осталось сослаться на второй критерий Дарбу.
\end{proof}

\begin{theorem}
Пусть $A$ замкнуто, и $f \bw\in \Cb(A)$. Тогда $f \bw\in \Rb(A)$.
\end{theorem}
\begin{proof}
Имеем $A \bw\in \Jc$, поэтому $A \bw\in \Bb$, значит, $A$ компактно. Поэтому $f$ равномерно непрерывна на
$A$. Тогда $\fa \ep \bw> 0 \exi \de \bw> 0\cln \fa x, y \bw\in A\cln \rho(x,y) \bw< \de$ имеем $|f(x) \bw- f(y)| \bw<
\ep$. Рассмотрим $T(A)\cln \la_T \bw< \de$, тогда $\usd{T} \bw- \lsd{T} \bw= \sum (M_k\bw-m_k) \mu(A_k) \bw\le
\ep \sum \mu(A_k) \bw= \ep\mu(A)$. Осталось сослаться на первый критерий Дарбу.
\end{proof}

\begin{theorem}
Пусть $A$ замкнуто, а $B \subs A$ и $ \mu(B) \bw= 0$. Пусть $f \bw\in \Bb(A)$ и $f \bw\in \Cb(A\bw\wo B)$. Тогда
$f \bw\in \Rb(A)$.
\end{theorem}
\begin{proof}
Имеем $\fa \ep \bw> 0 \exi \si_K \sups B\cln \mu(\si_K) \bw< \ep$. Пусть $|f| \bw\le K$ на $A$.
Без ограничения общности можно считать, что $\si_K$ открыто, тогда $C \bw{:=} A \bw\wo \si_K$ замкнуто.
По предыдущей теореме $f \bw\in \Rb(C)$, откуда $\exi T(C)\cln \usd{T} \bw- \lsd{T} \bw< \ep$.
Рассмотрим $A_0 \bw{:=} \si_K \bw\cap A$, тогда $A \bw= A_0 \bw\oplus T$.
Сделаем из разбиения $T$ разбиение $T_0$, добавив к нему элемент $A_0$. Тогда
$\usd{T_0} \bw- \lsd{T_0} \bw= \br{\usd{T}\bw-\lsd{T}} \bw+ (M_0 \bw- m_0) \mu(A_0) \bw\le \ep \bw+ 2K\ep$.
\end{proof}

\subsubsection{Свойства кратных интегралов}

В этом разделе будем предполагать множества $A, B, C$ измеримыми.

\begin{theorem}[Аддитивность интеграла]
Пусть $A\bw=B\bw\oplus C$, а $f \bw\in \Bb(A)$. Тогда $f \bw\in \Rb(A)$ тогда и только тогда, когда $f \bw\in
\Rb(B)$ и $f \bw\in \Rb(C)$, и в случае интегрируемости имеет место равенство $\ints{A} f \bw= \ints{B} f
\bw+ \ints{C} f$.
\end{theorem}
\begin{proof}
Пусть $f \bw\in \Rb(A)$. Тогда $\fa \ep \bw> 0 \exi T(A)\cln \usd{T(A)} \bw- \lsd{T(A)} \bw< \ep$. Рассмотрим
разбиение $T(B)$, индуцированное на $B$ разбиением $T(A)$, для которого имеем $B \bw= \oplusl{1}{n}(B \bw\cap A_i)$.
Очевидно, что $\usd{T(B)}\bw-\lsd{T(B)} \bw\le \usd{T(A)} \bw- \lsd{T(A)} \bw< \ep$,
поэтому $f \bw\in \Rb(B)$. Точно так же доказывается, что $f \bw\in \Rb(C)$.

Обратно, пусть $f \bw\in \Rb(B)$ и $f \bw\in \Rb(C)$. Тогда $\fa \ep \bw> 0 \exi T(B)\cln \usd{T(B)} \bw-
\lsd{T(B)} \bw< \ep$ и $\exi T(C)\cln \usd{T(C)} \bw- \lsd{T(C)} \bw< \ep$, поэтому для $T(A) \bw{:=} T(B)
\bw\oplus T(C)$ имеем $\usd{T(A)} \bw- \lsd{T(A)} \bw< 2\ep$, поэтому $f \bw\in \Rb(A)$.

Докажем равенство интегралов. По доказанному выше можно считать, что $f \bw\in \Rb(A), \Rb(B),
\Rb(C)$. Имеем $\ints{A} f \bw= I^*(A) \bw= \infl{T(A)} \usd{T(A)}$, причём в силу того, что от
измельчения разбиения верхняя сумма может только уменьшиться, можно рассматривать только разбиения
вида $T(A) \bw= T(B) \bw\oplus T(C)$. Поэтому $\usd{T(A)} \bw= \usd{T(B)}\bw+ \usd{T(C)}$. Если перейти к
точной нижней грани слева, то мы получим $\infl{T(A)}\usd{T(A)} \bw\le \usd{T(B)}\bw+ \usd{T(C)}$. Теперь
перейдём к $\inf$ справа, тогда получим $\infl{T(A)}\usd{T(A)} \bw\le \infl{T(B)}\usd{T(B)} \bw+
\infl{T(C)}\usd{T(C)}$. Если же сперва перейти к $\inf$ справа, а потом слева, то получится
обратное неравенство $\infl{T(A)}\usd{T(A)} \bw\ge \infl{T(B)}\usd{T(B)} \bw+ \infl{T(C)}\usd{T(C)}$.
Но это означает, что $I^*(A) \bw= I^*(B) \bw+ I^*(C)$, но так как функция интегрируема, получаем, что
$\ints{A}f \bw= \ints{B} f \bw+ \ints{C} f$.
\end{proof}

\begin{note}
Для неограниченных функций эта теорема не имеет места. Пусть $f \bw= 0$ на множестве $B \bw{:=}
[0,1]\times(0,1]$, и $f$ неограничена на множестве $C \bw{:=} [0,1] \times \hc{0}$. Ясно, что $f
\bw\in \Rb(B)$, поскольку $f \bw= 0$ на $B$, и $f \bw\in \Rb(C)$, поскольку $\mu(C) \bw= 0$, но не менее
очевидно, что $f \notin \Rb(B \bw\oplus C)$.
\end{note}

\begin{imp}
Если $f \bw\in \Rb(A)$ и $f \bw\in \Bb(A)$, а $B \subs A$, то $f \bw\in \Rb(B)$.
\end{imp}

\begin{imp}
Пусть $f \bw\in \Rb(A)$ и $f \bw\in \Bb(A)$, а $B \subs A$ и $\mu(B) \bw= 0$. Пусть $g \bw\in \Bb(A)$, и $g \bw=
f$ на $A \bw\wo B$. Тогда $g \bw\in \Rb(A)$ и $\ints{A} f\bw= \ints{A} g$.
\end{imp}
\begin{proof}
В самом деле, имеем $\ints{A} f\bw=\ints{A\bw\wo B} f \bw+ \ints{B} f \bw= \ints{A\bw\wo B} f \bw=
\ints{A \bw\wo B} g \bw= \ints{A \bw\wo B} g \bw+ \ints{B} g\bw= \ints{A} g$.
\end{proof}

\begin{theorem}[Линейность интеграла]
Пусть $f, g\bw\in \Rb(A)$, тогда $\fa \al, \be$ имеем $\ints{A} (\al f \bw+ \be g) \bw=
\al\ints{A} f \bw+ \be \ints{A} g$.
\end{theorem}
\begin{proof}
Очевидно, что $S_T(\al f \bw+ \be g, P) \bw= \al S_T(f, P) \bw+ \be S_T(g,P)$,
откуда и следует утверждение теоремы.
\end{proof}

\begin{theorem}
Пусть $f \bw\in \Rb(A)$ и $f \bw\ge 0$. Тогда $\ints{A} f \bw\ge 0$.
\end{theorem}

\begin{theorem}
Пусть $f \bw\le g$, и $f, g\bw\in \Rb(A)$. Тогда $\ints{A} f \bw\le \ints{A} g$.
\end{theorem}
\begin{proof}
В самом деле, имеем $g \bw- f \bw\ge 0$, откуда $\ints{A}(g \bw- f) \bw\ge 0$, а
потому $\ints{A} g \bw- \ints{A} f \bw\ge 0$.
\end{proof}

\begin{theorem}
Пусть $f \bw\in \Rb(A)$. Тогда $|f| \bw\in \Rb(A)$ и $\bm{\ints{A} f} \bw\le \ints{A} |f|$.
\end{theorem}
\begin{proof}
Заметим, что для ограниченных функций можно слово в слово повторить доказательство
второго семестра, поскольку можно воспользоваться суммами Дарбу. Сведём рассуждения от
неограниченных функций к ограниченным. Мы знаем, что $\exi h \bw> 0$, для которого $f \bw\in
\Bb\br{A(h)}$. Рассмотрим на $A$ функцию $g \bw= f\cdot \chi_{A(h)}$, тогда по теореме
\ref{ah.bounded.function} получаем $g \bw\in \Rb(A)$ и $\ints{A} g \bw= \ints{A} f$. Поскольку $g$
ограничена, $\bm{\ints{A} g} \bw\le \ints{A} |g|$. Далее, имеем $|f| \bw= |g|$ на $A(h)$, поэтому по той
же теореме $|f| \bw\in \Rb(A)$ и $\ints{A}|f| \bw= \ints{A} |g|$.
\end{proof}

Следующее утверждение не было доказано лектором для случая, когда $f$ не является ограниченной на
множестве $A$. Мы сейчас полностью докажем это утверждение без предположения ограниченности
функции, а затем укажем, где в лекциях оно неявно используется. При доказательстве мы, естественно,
будем пользоваться тем, что мы знаем для ограниченных функций.

\begin{theorem}
\label{integration.on.subset}
Пусть $A, B \in \Jc$, причём $B \subs A$. Пусть $f \in \Rb(A)$. Тогда $f \in \Rb(B)$ и
$\ints{A} f = \ints{B} f + \ints{A \wo B} f$.
\end{theorem}
\begin{proof}
Поскольку $f \in \Rb(A)$, найдётся такое $h\bw>0$, что $f \in \Bb\br{A(h)}$. Кроме того, очевидно,
что $B(h) \subs A(h)$, а из построения подстриженных множества следует их измеримость по Жордану.
Обозначим $C := A(h)$, а $D := B(h)$. По известной теореме для ограниченных функций получаем
$\ints{C} f = \ints{D} f + \ints{C\wo D}f$. По той же теореме выводим, что
$f \cdot \chi_C \in \Rb(A)$ и $f \cdot \chi_D \in \Rb(B)$, поскольку эти две функции
ограничены. Кроме того, $\ints{A} f \cdot \chi_C = \ints{C} f$ и $\ints{B} f \cdot \chi_D = \ints{D} f$.
Применим теорему \ref{ah.bounded.function} к функциям $f$ и $f \cdot \chi_C$, получим
равенство $\ints{A} f = \ints{A} f \cdot \chi_C$. Аналогично, $\ints{B} f = \ints{B} f \cdot \chi_D$.
Теперь остаётся только провести аналогичные рассуждения для $C\wo D$, после чего круг замкнётся.
\end{proof}


\subsubsection{Кратные и повторные интегралы}

\begin{theorem}
Пусть $A \bw= [a,b]\times[c,d]$. Пусть $f \bw\in \Rb(A)$. Пусть для $\fa x \bw\in [a,b]$ имеет смысл функция
$\Phi(x) \bw{:=} \intl{c}{d} f(x,y) \,d y$. Тогда $\intl{a}{b}\Phi\,d x \bw= \ints{A} f \,d x \,d y$.
\end{theorem}
\begin{proof}
Пусть $T_x\bw=\hc{x_0\sco x_N}$ равномерное разбиение отрезка $[a,b]$, а $T_y \bw= \hc{y_0\sco
y_N}$ равномерное разбиение отрезка $[c,d]$. Тогда шаги разбиений будут, соответственно, $h_x \bw=
\frac{b-a}{N}$ и $h_y \bw= \frac{c-d}{N}$. Введём обозначения для отрезков разбиения: $[x]_i \bw{:=}
[x_{i-1}, x_i]$, аналогично для $T_y$. Так как $A$ прямоугольник, то $f$ ограничена на нём,
поэтому $\Phi$ тоже ограничена. Имеем
$$\usd{T_x}(\Phi) \bw= \suml{i=1}{N} \supl{[x]_i}\Phi(x)h_x\bw\le
\suml{i=1}{N}\supl{[x]_i}\bbr{\suml{j=1}{N}\supl{[y]_j} f(x,y) h_y}h_x\bw=
\suml{i=1}{N}\suml{j=1}{N}\supl{[x]_i\times[y]_j} f(x,y) h_y h_x \bw= \usd{T_x\times T_y}(f).$$
Совершенно аналогично выводится оценка $\lsd{T_x \times T_y}(f) \bw\le \lsd{T_x}(\Phi)$. В силу
критерия интегрируемости для $f$ можно получить, что $\fa \ep \bw> 0$ имеем $\usd{T_x}(\Phi) \bw-
\lsd{T_x}(\Phi) \bw< \ep$, откуда и следует, что $\Phi \bw\in \Rb[a,b]$. Утверждение о равенстве
интегралов следует из того, что $I_f$ и $I_\Phi$ оба зажаты по крайней мере между числами
$\lsd{T_x\times T_y}(f)$ и $\usd{T_x \times T_y}(f)$.
\end{proof}

\begin{note}
В этой теореме требование определённости для $\Phi$ на всём отрезке не очень существенно. Если
$\Phi$ не определена в точке $x$, то можно положить $\Phi(x)$ по определению равным любому числу из
отрезка $\bs{I_*\br{f(x,y)}, I^*\br{f(x,y)}}$.
\end{note}

\begin{df}
Пусть $M \subs \Ebb^2(x_1,x_2)$. Пусть $\Pc$ проецирование на ось $OX_1$. Положим $M_1 \bw{:=} \Pc(M)$,
а $M(x_1) \bw{:=} \Pc^{-1}(x_1)$, где $x_1 \bw\in M_1$.
\end{df}

\begin{theorem}
Пусть $M, M_1 \bw\in \Jc$. Пусть $\fa x_1 \bw\in M_1$ имеем $M(x_1) \bw\in \Jc$. Пусть $f \bw\in \Rb(M)$, $f
\bw\in \Bb(M)$, и имеет смысл функция $\Phi(x_1) \bw{:=} \ints{M(x_1)} f\,d x_2$. Тогда $\Phi \bw\in
\Rb(M_1)$ и $\ints{M} f \bw= \ints{M_1} \Phi(x_1)\,d x_1$.
\end{theorem}
\begin{proof}
Поскольку $M \bw\in \Jc$, имеем $M \bw\in \Bb$, поэтому $\exi$ прямоугольник $A \sups M$. Рассмотрим $g
\bw{:=} \chi_M\cdot f$. Будем считать, что $\ints{M(x_1)} g \,dx_2 \bw= 0$ там, где $M(x_1)$ не определено.
Используя предыдущую теорему, получаем $$\ints{M} f \bw= \ints{M} g \bw= \ints{M} g \bw+ \ints{A \bw\wo M} g
\bw= \ints{A} g \bw= \intl{a}{b}\intl{c}{d} g \,dx_2 \,d x_1 \bw= \intl{a}{b} \ints{M(x_1)} g\,dx_2 \,dx_1
\bw= \ints{M_1} \ints{M(x_1)} g \,dx_2 \,dx_1 \bw= \ints{M_1} \ints{M(x_1)} f \,dx_2 \,dx_1,$$
что и завершает доказательство.
\end{proof}

Сформулируем эту теорему для произвольной размерности $m$. Пусть $M \subs \Ebb^m$. Пусть $\Pc$
отображение проецирования на подпространство $\Ebb^k(x_1\sco x_k)$. Положим $M_k \bw{:=} \Pc(M)$,
а $M(x_1\sco x_k) \bw{:=} \Pc^{-1}(M_k)$.

\begin{theorem}
Пусть $M, M_k \bw\in \Jc$ и $\fa (x_1\sco x_k) \bw\in M_k$ имеем $M(x_1\sco x_k) \bw\in \Jc$.
Пусть $f \bw\in\Rb(M)$ и $f \bw\in \Bb(M)$. Пусть имеет смысл функция
$\Phi(x_1\sco x_k) \bw{:=} \ints{M(x_1\sco x_k)} f\,dx_{k+1}\dots d x_m$. Тогда
$\Phi \bw\in \Rb(M_k)$ и $\ints{M} f \bw= \ints{M_k}\Phi(x_1\sco x_k)\,d x_1\dots dx_k$.
\end{theorem}

\subsubsection{Геометрический смысл якобиана отображения}

Пусть $G$ область в $\Ebb^m(t_1\sco t_m)$. Пусть задано отображение $\Phi\cln \Ebb^m(t) \ra
\Ebb^m(x)$ по правилу $x_i \bw= \ph_i(t)$, где $i \bw= 1\sco m$. Будем предполагать, что $\Phi \bw\in
\Cb^1(G)$, е $\ph_i \bw\in \Cb(G)$ и $\ppht ij \bw\in \Cb(G)$. Положим $H \bw{:=} \Phi(G)$. Как мы
знаем, \emph{якобианом} отображения $\Phi$ называется определитель матрицы Якоби
$$D(t) \bw{:=} |d\Phi| \bw= \pf \ph t \bw= \mbmat{
\ppht11 & \dots & \ppht1m\\
\vdots & & \vdots \\
\ppht m1 & \dots & \ppht mm\\}$$

\begin{theorem}
Если $|d \Phi| \neq 0$ на $G$, то $H$ область.
\end{theorem}
\begin{proof}
Покажем, что $H$ открыто. Действительно, пусть $x \bw\in H$. Рассмотрим $t \bw\in \Phi^{-1}(x)$.
Поскольку, в частности, $|d\Phi|(t) \neq 0$, по теореме о неявных функциях найдутся окрестности
$U(t) \subs G$ и $V(x) \subs H$, между которыми $\Phi$ устанавливает взаимно однозначное
соответствие. Отсюда следует, что каждая точка в $H$ содержится в нём с некоторой своей
окрестностью. Значит, $H$ открыто.

Покажем, что $H$ связно. Действительно, пусть $x, y \bw\in H$. Покажем, что их можно соединить гладкой
кривой, целиком лежащей в $H$. Рассмотрим $u \bw\in \Phi^{-1}(x)$ и $w \bw\in \Phi^{-1}(y)$, тогда точки
$u$ и $w$ можно соединить гладкой кривой в силу связности $G$. Пусть $\Ga\cln [0,1] \ra G$ наша
кривая, причём $\Ga(0) \bw= u$, а $\Ga(1) \bw= w$. Рассмотрим отображение $\ga \bw{:=} \Phi \circ \Ga\cln
[0,1] \ra H$. Поскольку $\Phi$ хорошее отображение, $\ga$ будет гладкой кривой по теореме о
производной композиции функций. Осталось заметить, что $\ga(0) \bw= x$, а $\ga(1) \bw= y$.
\end{proof}

\begin{note}
Отображение $\Phi$, как несложно видеть, может отображать область $G$ в $H$ не биективно. В этом
легко убедиться на следующем примере. Пусть $\Phi\cln \Ebb^2(r,\ph) \to \Ebb^2(x,y)$. В качестве
$G$ возьмём открытый прямоугольник $(0,5)\times(-5\pi,5\pi)$. Ясно, что $G$ область. Устроим
отображение $\Phi$ по следующему правилу: $x \bw= r \cos \ph$, $y \bw= r \sin \ph$. Имеем $|d\Phi| \bw= r
\neq 0$ на $G$. Легко видеть, что $\ph(G)$ есть открытый диск радиуса $5$ с выколотым центром $(0,
0)$, однако $\Phi$ не биективно, поскольку, например, $\Phi(2, \pi) \bw= \Phi(2,3\pi)$.
\end{note}

Чтобы избежать подобных неприятностей, надо потребовать от $\Phi$ биективности, тогда проблем будет
меньше. Настало время ввести важное понятие, а кроме того, мы будем избавлены от необходимости
каждый раз перечислять все необходимые ограничения.

\begin{df}
Пусть $G, H \subs \Ebb^m$ области. Отображение $\Phi\cln G \ra H$ называется
\emph{диффеоморфизмом} областей $G$ и $H$ класса
гладкости $\Cb^p$, если:\\
\pt{1}. $\Phi$ биекция $G \lra H$,\\
\pt{2}. $|d\Phi| \neq 0$ на $G$,\\
\pt{3}. $\Phi, \Phi^{-1} \bw\in \Cb^p$.

Здесь $\Cb^p$ понимается в том смысле, что отображение непрерывно и существуют непрерывные частные
производные всех порядков до $p$ включительно. Нас особенно будут интересовать диффеоморфизмы
класса $\Cb^1$, поэтому если мы говорим о диффеоморфизме без указания класса гладкости, будем
считать, что он имеет класс гладкости $\Cb^1$.
\end{df}

\begin{theorem}
Пусть задана композиция дифференцируемых отображений $\Psi \circ \Phi$. Тогда производная
композиции отображения равна композиции производных: $d(\Psi \circ \Phi) \bw= d\Psi \circ d\Phi$.
\end{theorem}
\begin{proof}
Мы это доказали во втором семестре, когда доказывали теорему о дифференцируемости композиции. Легко
видеть, что в матричном виде формулы для производной сложной функции выглядят так: если $\Phi\cln
\Ebb^m(t) \ra \Ebb^n(x)$, а $\Psi\cln \Ebb^n(x)\ra\Ebb^k(y)$, и имеет смысл композиция
$\Psi\circ\Phi$, то
$$d(\Psi\circ\Phi) \bw= \rbmat{\pyx11 & \dots & \pyx1n\\\vdots & & \vdots\\\pyx k1 & \dots & \pyx kn}
\rbmat{\pxt11 & \dots & \pxt1m\\\vdots & & \vdots\\\pxt n1 & \dots & \pxt nm}.$$ Если перемножить
матрицы, возникнет матрица производной $\Psi\circ\Phi$ размера $k\times m$, а элементами в ней
будут частные производные $\pf{(\Psi\circ\Phi)_i}{t_j}$.
\end{proof}

\begin{imp}
Якобиан композиции равен произведению якобианов: $|d(\Psi\circ\Phi)| \bw= |d\Psi|\cdot |d\Phi|$.
\end{imp}

\begin{lemma}
Пусть $\Phi\cln G \ra H$ диффеоморфизм областей $G$ и $H$. Пусть $\si$ множество,
для которого $\Cl \si \subs G$. Пусть $\tau \bw= \Phi(\si)$. Тогда $\pd \tau \bw= \Phi(\pd \si)$.
Иначе говоря, диффеоморфизм сохраняет границу.
\end{lemma}
\begin{proof}
Мы уже доказывали, что $\Phi(\Int \si) \subs \Int \tau$. Поскольку обратное отображение устроено
так же, имеем $\Phi^{-1}(\Int \tau) \subs \Int \si$. Значит, $\Int \tau \bw= \Phi(\Int \si)$. Отсюда
$\pd \tau \bw= \Phi(\pd \si)$, поскольку $\Phi$ биективно.
\end{proof}

Пусть $G \bw\in \Bb$ область. Пусть $\Phi$ и все его частные производные допускают непрерывное продолжение
на $\Cl G$. Поскольку $\Cl G$ компакт, непрерывные на нём функции будут равномерно непрерывны.
Следовательно, $\om\hr{\ppht ij, \de}\ra 0$ при $\de \ra 0$. Тогда $\om(\de) \bw{:=}
\maxl{i,j}\om\hr{\ppht ij, \de} \ra 0$. В силу непрерывности имеем $|\ph_i| \bw\le L$ и $\bm{\ppht ij}
\bw\le L$. Распишем всё в случае $m\bw=2$ для сокращения выкладок. Пусть $t\bw\in G$, тогда $D(t) \bw=
\ppht11(t)\ppht22(t)\bw-\ppht12(t)\ppht21(t)$. Пусть $t^0 \bw\in G$, тогда каждое из слагаемых
в разности $D(t)\bw-D(t^0)$ можно оценить следующим образом:
\begin{multline*}
\bbm{\ppht11(t)\ppht22(t)\bw-\ppht11(t^0)\ppht22(t^0)} \bw\le
\bbm{\ppht11(t)\ppht22(t)\bw-\ppht11(t)\ppht22(t^0)}\bw+
\bbm{\ppht11(t)\ppht22(t^0)\bw-\ppht11(t^0)\ppht22(t^0)}\bw=\\=
\bbm{\ppht11(t)}\bbm{\ppht22(t)\bw-\ppht22(t^0)}\bw+
\bbm{\ppht22(t^0)}\bbm{\ppht11(t)\bw-\ppht11(t^0)}\bw\le 2L\cdot\om\br{|t\bw-t^0|}.
\end{multline*}
Отсюда следует справедливость оценки
$|D(t) \bw- D(t^0)| \bw\le C(L, m) \cdot \om\br{|t\bw-t^0|}$, где $C$ некоторая
константа.

\begin{theorem}
Пусть $\Phi$ осуществляет диффеоморфизм областей $G$ и $H$, причём $G \bw\in \Bb$. Пусть $\si$
квадрат со стороной $h$, причём $\Cl \si \subs G$, а его стороны параллельны системе координат.
Пусть $\tau \bw= \Phi(\si)$. Тогда множество $\tau$ измеримо, и для его меры справедлива оценка
$\mu(\tau) \bw= \mu(\si)\bs{|D(t)| \bw+ O\br{\om(\diam \si)}}$, причём эта оценка равномерна по $t \bw\in \si$.
\end{theorem}

\vskip-15pt
\rightpicture{pictures.10}
\begin{proof}
\hangindent=-40mm
\hangafter=-6
Без ограничения общности будем вести доказательство в случае, когда
$m\bw=2$. Обозначим вершины $\si$ буквами $A_{00}, A_{01}, A_{10}$ и $A_{11}$ соответственно, как
показано на рисунке. Мы знаем, что при диффеоморфизме внутренние точки переходят во внутренние, а
граница в границу. Поскольку $\pd\si$ есть объединение гладких кривых, $\pd\tau$ есть
объединение их образов, а они также будут гладкими кривыми. Значит, $\mu(\pd \tau) \bw= 0$,
следовательно, $\tau$ измеримо. Пусть $t^0$ соответствует вершине $A_{00}$.
Для краткости обозначим $\pphti ijk \bw{:=} \ppht ij \br{t^0 \bw+
\ta_k(t\bw-t^0)}$. В силу выведенной выше оценки, можно рассматривать значения якобиана именно в
точке $t^0$, поскольку ошибка, которую мы при этом получаем, будет съедена слагаемым $O(\cdot)$.
Применим ФКПЛ, получим
$$x\bw= x^0 \bw+ \rbmat{\pphti111 & \pphti121\\\pphti212 & \pphti222}(t \bw- t^0).$$
Введём линейное отображение $\Phi^*$, задаваемое формулами
$$x^* \bw= x^0 \bw+ \rbmat{\ppht11(t^0) & \ppht12(t^0)\\\ppht21(t^0) & \ppht22(t^0)}(t \bw- t^0).$$
Введём обозначения для вершин криволинейного четырёхугольника $\tau$: $B_{ij} \bw{:=}
\Phi(A_{ij})$. Кроме того, обозначим $C_{ij} \bw{:=} \Phi^*(A_{ij})$. Пусть $\tau^* \bw{:=}
\Phi^*(\si)$. Поскольку $\Phi^*$ линейно, $\tau^*$ есть параллелограмм. Вспоминая о том, что
ориентированный объём параллелепипеда, натянутого на вектора $a_1\sco a_n$ есть определитель,
составленный из координат этих векторов, получаем $\mu(\tau^*) \bw= |D(t^0)|h^2\bw= |D(t^0)|\mu(\si)$.
Посмотрим, насколько далеко могут оказаться точки $x$ и $x^*$. Имеем:
$$x \bw- x^* \bw= \hs{
\rbmat{\pphti111 & \pphti121\\\pphti212 & \pphti222}\bw- \rbmat{\ppht11(t^0) & \ppht12(t^0)\\
\ppht21(t^0) & \ppht22(t^0)}}(t \bw- t^0)\bw=
\rbmat{\pphti111 \bw- \ppht11(t^0) & \pphti121 \bw- \ppht12(t^0) \\
\pphti212 \bw- \ppht21(t^0) & \pphti222 \bw- \ppht22(t^0)}(t \bw- t^0).$$
Осталось заметить, что $|t_i \bw- t_i^0| \bw\le h$, поэтому разность между аргументами частных производных в каждой
компоненте матрицы не превосходит $h$. Отсюда следует, что $|x \bw- x^*| \bw\le 2\om\br{h\sqrt2}h \bw{=:}
\la$. Если бы мы доказывали теорему для произвольного $m$, мы получили бы оценку вида
$m\om\br{h\sqrt m}h$. Построим вокруг границы $\tau^*$ рамку $F$ толщиной $\la$, тогда $\pd \tau
\subs F$, а $|\mu(\tau^*) \bw- \mu(\tau)| \bw\le \mu(F)$. Ясно, что длины сторон параллелограмма линейно
зависят от $h$, поэтому с помощью простого подсчёта типа «площадь на длину» получаем $\mu(F) \bw\le K
h^2 \om\br{h\sqrt2} \bw= K \mu(\si)\om\br{h\sqrt 2}$. Очевидно, что ситуация только улучшится, если
рамка станет вырожденной, е её внутренняя часть исчезнет.
\end{proof}

\begin{imp}
Коэффициент искажения площади равен модулю якобиана: $\liml{h\ra 0} \frac{\mu(\tau)}{\mu(\si)} \bw= |D(t^0)|$.
\end{imp}

\subsubsection{Замена переменных в кратном интеграле}

Мы сохраняем обозначения, введённые в предыдущем параграфе, однако теперь будем дополнительно
предполагать, что область $G$ выпукла. Это ограничение несущественно, однако доказательство теоремы
будет выглядеть попроще.
\begin{theorem}[О замене переменных]
Пусть $f \bw\in \Rb(H)$, $f \bw\in \Bb(H)$. Рассмотрим $F \bw{:=} f \circ \Phi\cln G \ra \R$. Тогда $F \bw\in
\Rb(G)$ и $\ints{G}F|D|\,dt \bw= \ints{H} f\,dx$.
\end{theorem}
\begin{proof}
Из условий теоремы следует, что $|f| \bw\le B$ и $|F| \bw\le B$. Рассмотрим $T\cln G \bw= G_1\sop G_N$,
индуцируемое на нём сеткой с шагом $h$. Положим $H_i \bw{:=} \Phi(G_i)$, тогда в силу биективности
$\Phi$ набор $\hc{H_i}$ будет образовывать разбиение $T(H)$. Разделим элементы разбиения $T(G)$ на
две группы: к группе I отнесём все полные квадратики, а к группе II неполные. Тогда верхние
суммы Дарбу для функций $F|D|$ и $f$ естественным образом распадаются на две суммы по этим группам:
$\usd{T(G)} \bw= \sumfi \bw+ \sumse$ и $\usd{T(H)} \bw= \sumfi \bw+ \sumse$. В предположении о выпуклости
множества $G$ можно утверждать, что $\om(\la t) \bw\le (\la \bw+ 1)\om(t)$. Поскольку $O(\cdot)$ съедают
любые константные множители, не будем их писать. Разберёмся с суммами первой группы. По результатам
предыдущего параграфа, имеем $M_i(f)|D(t^0)| \bw= M_i(F)|D(t^0)| \bw= M_i\br{F|D(t^0)|} \bw= M_i\br{F|D|}
\bw+ O\br{\om(h)}$, поэтому
$$\sumfi M_i(f) \mu(H_i) \bw= \sumfi M_i(f)\bs{ |D(t^0)|\mu(G_i) \bw+ O\br{\mu(G_i)\om(h)} } \bw=
\sumfi M_i\br{F|D|}\mu(G_i) \bw+ O\br{\om(h)}.$$

Оценим теперь суммы второй группы. В силу измеримости $G$, можно сделать шаг сетки настолько
мелким, что $\sumse \mu(G_i) \bw< \ep$. В силу ограниченности функций имеем $\sumse
M_i\br{F|D|}\mu(G_i) \bw\le C\sumse \mu(G_i) \bw< C\ep$. С суммой второй группы для $f$ поступим иначе:
пусть $x', x'' \bw\in H_i$, тогда $\exi t',t'' \bw\in G_i$, для которых $x' \bw= \Phi(t')$ и $x'' \bw=
\Phi(t'')$. По ФКПЛ имеем
$$\ph_j(t') \bw- \ph_j(t'') \bw= \hr{\ppht j1\br{t' \bw+ \ta(t''\bw-t')} \sco \ppht jm
\br{t' \bw+ \ta(t''\bw-t')}}(t' \bw- t'') \bw= O(h).$$
Это означает, что $\mu(H_i) \bw\le K\mu(G_i)$, где
$K$ некоторая константа. Действительно, если $t'$ далеко не уйдёт от $t''$, то и образы их
далеко не разбегутся. Теперь очевидно, что $\sumse M_i(f)\mu(H_i) \bw\le P\ep$.

Осталось рассмотреть разность верхних сумм Дарбу и убедиться в том, что они близки:
$$\hm{\usd{T(G)}(F|D|)\bw- \usd{T(H)}(f)} \bw\le O\br{\om(h)} \bw+ C\ep \bw+ P\ep \ra 0, \; \ep \ra 0,\; h \ra 0.$$

Ясно, что для нижних сумм Дарбу можно вывести аналогичную оценку. Отсюда следует утверждение
теоремы, поскольку $f$ интегрируема по условию, а значения верхних сумм Дарбу для $F|D|$ оказались
близки к суммам Дарбу для $f$. Совпадение интегралов вытекает из того, что $\int f$ и $\int F|D|$
оба зажаты между стягивающимися суммами Дарбу.
\end{proof}

\subsection{Несобственные кратные интегралы}

\subsubsection{Исчерпывающие последовательности множеств}

\begin{df}
Пусть $D \subs \Ebb^m$, и пусть $\hc{D_k}$ последовательность подмножеств $\Ebb^m$. Скажем, что $\hc{D_k}$
\emph{исчерпывает} $D$, если $D_k \subs D_{k+1}$ и $D \bw= \cupkui D_k$.
\end{df}

Обоснование корректности формулировки следующей теоремы лежит на плечах теоремы \ref{integration.on.subset},
не доказанной на лекциях.
\begin{theorem}
Пусть $f \bw\in \Rb(D)$. Тогда для любой последовательности $\hc{D_k} \bw\subs \Jc$, исчерпывающих $D$, имеем
$\liml{k} \mu(D_k) \bw= \mu(D)$ и $\liml{k} \ints{D_k} f \bw= \ints{D} f$.
\end{theorem}
\begin{proof}
Поскольку $A \bw\in \Jc$, найдутся замкнутое $F \bw\in \Jc$ и открытое $G \bw\in \Jc$, такие что $F \subs A
\bw\subs G$, а $\mu(G) \bw- \mu(F) \bw< \ep$. Отсюда $\mu(G) \bw- \mu(A) \bw< \ep$ и $\mu(A) \bw-
\mu(F) \bw< \ep$. Построим таким образом множества $F_k \bw\subs D_k \bw\subs G_k$. Покажем, что
$\hc{G_k}$ можно взять монотонной. Зафиксируем $n$, тогда $\exi \ep'\cln \mu(G_n) \bw+ \ep' \bw<
\mu(D_n) \bw+ \ep$. Рассмотрим открытое $G'$, для которого $D_{n+1}\bw\wo D_n \subs G'$, но
$\mu(G') \bw- \mu(D_{n+1} \bw\wo D_n) \bw< \ep'$. Тогда рассмотрим $G_{n+1} \bw{:=} G_n \bw\cup
G'$. Оно будет открыто и измеримо. Очевидно, что $D_{n+1} \bw\subs G_{n+1}$, и осталось только
оценить его меру: $\mu(G_{n+1}) \bw\le \mu(G_n) \bw+ \mu(G') \bw< \mu(G_n) \bw+ \mu(D_{n+1}) \bw-
\mu(D_n) \bw+ \ep' \bw< \mu(D_{n+1}) \bw+ \ep$. Тем самым показана возможность выбрать монотонную
последовательность $\hc{G_k}$.

Мы знаем, что $\exi F \subs D$ замкнутое множество, для которого $\mu(D) \bw- \mu(F) \bw< \ep$.
Очевидно, $\cupkui G_k$ покрывают $D$, поэтому они покрывают и $F$. Поскольку $F$ измеримо, оно
ограничено и вследствие замкнутости компактно. Следовательно, из его открытого покрытия $\hc{G_k}$
можно выделить конечное подпокрытие, но в силу монотонности $\hc{G_k}$ найдётся $N \bw\in \N$, для
которого $F \subs G_N$. Тогда подавно $\fa n \bw\ge N$ имеем $\mu(F) \bw\le \mu(G_n)$. Следовательно,
$\mu(D) \bw< \mu(F) \bw+ \ep \bw\le \mu(G_n) \bw+ \ep \bw\le \mu(D_n) \bw+ 2\ep$. С другой стороны, поскольку $D_n
\subs D$, получаем $\mu(D_n) \bw\le \mu(D)$. Следовательно, $\mu(D_n) \ra \mu(D)$.

Осталось показать, что существует указанный предел интегралов. Если $f$ ограничена числом $B$, то
$\bm{\ints{D} \bw- \ints{D_k}} \bw\le B\mu(D\bw\wo D_k) \ra 0$. Если функция неограничена, то $\exi h \bw>0\cln
f \bw\in \Bb\br{D(h)}$, причём $\mu\br{D \bw\wo D(h)} \bw= 0$. Тогда $\ints{D} f \bw= \ints{D(h)}f$, а дальше
можно применить предыдущие рассуждения.
\end{proof}

\begin{df}
Пусть $f\cln D\ra \R$. Если $\fa \hc{D_k}$, исчерпывающей $D$, имеем $f \bw\in \Rb(D_k)$, и существует
предел $\liml{k} \ints{D_k} f \bw{=:} I$, не зависящий от выбора исчерпания, то функция $f$
называется \emph{интегрируемой в несобственном смысле} на множестве $D$. Мы будем обозначать этот
факт символом $f \bw\in \Rbt(D)$. Несложно видеть, что несобственный интеграл обладает свойством
линейности.
\end{df}

\subsubsection{Признаки сходимости}

В этом параграфе будем предполагать, что $D_k \bw\in \Jc$. Измеримость множества $D$ предполагается
только в том случае, когда этого требуют используемые в теоремах объекты.

\begin{theorem}
Пусть $f \bw\ge 0$ на $D$ и $\fa \hc{D_k}$, исчерпывающей $D$, существует $\ints{D_k} f$. Если
$\exi \hc{D_k^*}$, для которой существует конечный предел $I \bw{:=} \liml{k} \ints{D_k^*} f$, то
$f \bw\in \Rbt(D)$ и $\ints{D}f \bw= I$.
\end{theorem}
\begin{proof}
Пусть $\hc{D_j}$ произвольная последовательность, исчерпывающая $D$. Обозначим $I_j \bw{:=}
\ints{D_j} f$, а $I_k^* \bw{:=} \ints{D_k^*}f$. В силу неотрицательности $f$ имеем $I_j \up$ и
$I_k^* \up I$. Поскольку $\hc{D_j}$ и $\hc{D_k^*}$ возрастающие последовательности
множеств, исчерпывающие $D$, то $\fa j \exi k\cln I_j \bw\le I_k^*$. Отсюда $\hc{I_j}$ имеет предел, не
превосходящий $I$. Но по симметричным рассуждениям верно и обратное, поэтому $\liml{j} I_j \bw= I$.
\end{proof}

\begin{theorem}[Мажорантный признак сходимости]
Пусть $|f|\bw\le g$ на $D$. Пусть $g \bw\in \Rbt(D)$, а $\fa \hc{D_k}$, исчерпывающей $D$, имеем $f \bw\in
\Rb(D_k)$. Тогда $f \bw\in \Rbt(D)$ и $\bm{\ints{D}f} \bw\le \ints{D} |f|\bw\le \ints{D} g$.
\end{theorem}
\begin{proof}
Пусть $\hc{D_k}$ исчерпывает $D$. Для собственных интегралов мы знаем, что $\bm{\ints{D_k}f} \bw\le
\ints{D_k} |f| \bw\le \ints{D_k} g \bw\le \ints{D} g$. Тем самым доказана ограниченность любой
последовательности из интегралов $\ints{D_k} f$. Рассмотрим $f^+ \bw{:=} \frac{|f| + f}{2}$ и $f^-
\bw{:=} \frac{|f| \bw- f}{2}$, тогда $f \bw= f^+ \bw- f^-$. Легко видеть, что $f^+ \bw= \max\hc{f,
0}$, а $f^- \bw= -\min\hc{f, 0}$, и $f^+, f^- \bw\le |f|$. По предыдущей теореме $f^+$ и $f^-$
интегрируемы, и $\fa \hc{D_k}$, исчерпывающих $D$, имеем $\liml{k} \ints{D_k} f^+ \bw= \ints{D}
f^+$ и $\liml{k} \ints{D_k} f^- \bw= \ints{D} f^-$. Кроме того, для собственных интегралов
справедливо равенство $\ints{D_k} f \bw= \ints{D_k} f^+ \bw- \ints{D_k} f^-$. Поскольку в правой
части пределы существуют, можно перейти к пределу, откуда следует, что $f \bw\in \Rbt(D)$.

Остаётся показать справедливость неравенства. Имеем $|f| \bw= f^+ \bw+ f^-$, и осталось заметить, что
$\bm{\ints{D_k} f} \bw\le \ints{D_k}|f| \bw= \ints{D_k} f^+ \bw+ \ints{D_k} f^- \ra \ints{D} f^+ \bw+ \ints{D}
f^- \bw= \ints{D} |f|$. Но если нестрогое неравенство было верно для выражений под знаком предела, то
для пределов оно останется справедливым.
\end{proof}

\begin{theorem} \label{abs.int}
Если $f \bw\in \Rbt(D)$, то и $|f| \bw\in \Rbt(D)$.
\end{theorem}
\begin{proof}
\pt{1}. Докажем теорему от противного: предположим, что найдётся последовательность $\hc{D_k}$,
исчерпывающая $D$, для которой $\liml{k} \ints{D_k} |f| \bw= +\infty$. В силу расходимости можно
сделать так, что $\ints{D_{k+1}} |f| \bw> 3\ints{D_k} |f| \bw+ 2k$. Всегда будем выкидывать лишние члены
последовательности и проводить перенумерацию, чтобы не менять обозначений.

\pt{2}. Рассмотрим $A_k \bw{:=} D_{k+1} \bw\wo D_k$, тогда $D_{k+1} \bw= A_k \sqcup D_k$. Отсюда в силу
теоремы \ref{integration.on.subset} получаем
$\ints{D_{k+1}}|f| \bw=\ints{D_k}|f| \bw+ \ints{A_k}|f|$, поэтому $\ints{A_k}|f| \bw> 2 \ints{D_k}|f| \bw+
2k$, откуда $\ints{A_k}(f^+ \bw+ f^-) \bw> 2\ints{D_k}|f| \bw+ 2k$. Заметим, что $\fa k$ выполнено либо
неравенство $\ints{A_k}f^+ \bw\ge \ints{A_k}f^-$, либо неравенство $\ints{A_k} f^+ \bw< \ints{A_k}f^-$, ибо
третьего не дано. Неравенств хотя бы одного сорта будет бесконечно много, причём без ограничения
общности можно считать, что был выбран первый сорт. Оставим все члены последовательности, для которых
верно неравенство первого сорта, остальное выкинем и опять перенумеруем. Для того, что осталось, тем
более будет выполнено $2 \ints{A_k}f^+ \bw>2 \ints{D_k}|f| \bw+ 2k$. Сокращая на константу,
получаем $\ints{A_k}f^+ \bw> \ints{D_k}|f| \bw+ k$.

\pt{3}. Рассмотрим разбиение $T\cln A_k \bw= A_{k1}\sop A_{kn}$, для которого $\sumiun
f^+(p_i)\mu(A_{ki}) \bw> \ints{D_k}|f| \bw+ k$, причём это верно $\fa p_i \bw\in A_{ki}$. Отметим $*$ те
элементы $T$, на которых $f^+ \bw> 0$. Образуем множество $B_k \bw{:=} \oplusl{*}{}A_{ki} \subs
A_k$, тогда $f^+ \bw> 0$ на $B_k$. Несложно видеть, что $\sums{*} f^+(p_i)\mu(A_{ki}) \bw>
\ints{D_k}|f| \bw+ k$. В самом деле, допустим, что неравенство нарушилось. Тогда добавим к этой сумме
элементы, не попавшие в $B_k$, и выберем для них $p_i$ так, чтобы $f(p_i) \bw= 0$. Это не изменит
сумму. Но мы ведь предположили, что исходное неравенство верно для всех наборов $P$, значит, оно не
может нарушиться. Суммирование по $*$ есть просто интегральная сумма для $f^+$ на множестве $B_k$,
но в ней $f^+$ можно заменить на $f$, поскольку на $B_k$ имеем $f^+\bw=f$. Отсюда следует, что и
$\ints{B_k} f \bw\ge \ints{D_k}|f| \bw+ k$. В это несложно поверить, особенно если нарисовать картинку
и вспомнить, что это верно для всех наборов $P$.

\pt{4}. Имеем $\ints{B_k}f \bw\ge \ints{D_k}|f| \bw+ k \bw\ge -\ints{D_k} f \bw+ k$, откуда $\ints{B_k}f
\bw+ \ints{D_k}f \bw\ge k$. Но $B_k \bw\cap D_k \bw= \es$, ибо $B_k \subs A_k$. Тогда для $C_k \bw{:=} B_k \bw\cup
D_k$ имеем $\ints{C_k} f \bw\ge k$. Ясно, что $\hc{C_k}$ исчерпывает $D$, поскольку $D_k \subs C_k
\subs D_{k+1}$. Но для неё $\liml{k} \ints{C_k}f \bw= +\infty$, а это противоречит сходимости
$\ints{D} f$.
\end{proof}

\begin{note}
Таким образом, для кратного интеграла не существует понятия условной сходимости. Доказательство
теоремы напоминает теорему Римана о перестановке членов условно сходящегося ряда.
\end{note}

\section{Криволинейные и поверхностные интегралы}

\subsection{Криволинейные интегралы}

\subsubsection{Криволинейные интегралы первого рода}
Будем считать, что мы находимся в пространстве $\Ebb^3(x,y,z)$. Пусть задана спрямляемая гладкая
регулярная кривая $\Ga$ длины $S$ и для неё выбрана натуральная параметризация $\Ga\cln [0,S] \ra
\Ebb^3$, е $l\br{\Ga[0,s]} \bw= s$. Пусть компоненты её радиуса-вектора $r(s)$ суть функции $x \bw=
\ph(s)$, $y \bw= \psi(s)$ и $z \bw= \chi(s)$. Несложно видеть, что при натуральной параметризации
$|\dot{r}| \bw= 1$.

\begin{df}
Пусть $F\cln \Ga \ra \R$. Если существует интеграл $\intl{0}{S} F\br{\ph(s), \psi(s), \chi(s)}\,ds \bw{=:}
\ints{\Ga} F\,ds$, то такой интеграл называется \emph{криволинейным интегралом первого рода по кривой}
$\Ga$.
\end{df}

Формально, наше определение зависит от того, в какую сторону мы бежим по кривой. Покажем, что на самом деле
криволинейный интеграл первого рода не зависит от этого. Действительно, движение по кривой в обратную сторону
соответствует замене параметра $s$ на параметр $t \bw= S\bw-s$. Обозначим кривую с обратной параметризацией
через $\Ga^-$. Имеем
$$
  \intl{0}{S}F\br{\ph(s), \psi(s), \chi(s)}\,ds \bw= \intl{S}{0}F\br{\ph(S\bw-t),
  \psi(S\bw-t), \chi(S\bw-t)}\,d(S\bw-t) \bw=\intl{0}{S}F\br{\ph(S\bw-t), \psi(S\bw-t), \chi(S\bw-t)}\,dt
  \bw= \ints{\Ga^-} F\,dt.
$$

Нам хотелось бы уметь считать криволинейные интегралы, не прибегая к натуральной параметризации. Пусть $\Ga
\bw= \bc{\br{u(t),v(t),w(t)} \bvl t \bw\in [a,b]}$. Тогда $s'(t) \bw= \sqrt{\br{u'}^2 \bw+ \br{v'}^2 \bw+
\br{w'}^2}$, откуда, делая замену, получаем $\ints{\Ga} F\,ds  \bw= \intl{a}{b}F(u,v,w)\sqrt{\br{u'}^2 \bw+
\br{v'}^2 \bw+ \br{w'}^2}\,dt$.

Поскольку криволинейный интеграл первого рода определён через обыкновенный интеграл, многие
свойства, как то линейность и аддитивность, переносятся и на него без всяких ограничений. Поэтому
мы не будем задерживаться на их рассмотрении.

\begin{note}
Физический смысл этого интеграла масса тонкой нити с плотностью $F$.
\end{note}

\subsubsection{Криволинейный интеграл второго рода}
Мы сохраняем обозначения, введённые в предыдущем параграфе.

\begin{df}
Говорят, что в $\Ebb^3$ задано \emph{векторное поле}, если задана вектор-функция $a\cln \Ebb^3 \ra \Ebb^3$.
Компоненты вектора $a(x,y,z)$ обычно обозначаются $P$, $Q$ и $R$ соответственно и называются
\emph{компонентами векторного поля}. Будем обозначать жирными буквами $\ib, \jb, \kb$ три стандартных
базисных вектора в $\Ebb^3$. Тогда $a(x,y,z) \bw= P(x,y,z)\ib \bw+ Q(x,y,z)\jb \bw+ R(x,y,z)\kb$. Иногда нам
будет достаточно, чтобы векторное поле было задано не на всём пространстве, а лишь на его подмножестве.
\end{df}

\begin{df}
Пусть на кривой $\Ga$, отнесённой к натуральному параметру, задано векторное поле $a$. Обозначим через $\tau$
касательный вектор $\dot r$. Тогда $|\tau| \bw= 1$. Напомним, что через $(a,\tau)$ мы обозначаем скалярное
произведение векторов. Если существует криволинейный интеграл $\ints{\Ga}(a,\tau)\,ds$, то говорят, что на
ориентированной кривой $\Ga$ задан \emph{криволинейный интеграл второго рода}. Он обозначается
$\ints{\Ga}(a,ds)$.
\end{df}

Покажем, что криволинейный интеграл зависит от направления пробегания кривой. Пусть $\Ga^+$ кривая,
параметризованная в одну сторону, а $\Ga^-$ та же кривая, параметризованная в обратную сторону. Поскольку
интеграл первого рода не зависит от направления, $\ints{\Ga^+} (a,ds) \bw= \ints{\Ga}(a,\tau)\,ds \bw=
-\ints{\Ga}(a,-\tau)\,ds \bw= -\ints{\Ga^-}(a,ds)$. Итак, смена направления параметризации меняет знак
криволинейного интеграла второго рода.

Иногда интеграл второго рода обозначается так: $\ints{\Ga}(P\,dx \bw+ Q\,dy \bw+ R\,dz)$, причём часто
скобки не пишут. Такая запись удобна при вычислении интеграла: выражая $dx, dy, dz$ через $dt$,
получаем обычный интеграл по отрезку.

По определению касательного вектора, $\tau \bw= \dot{\ph}\ib \bw+ \dot{\psi}\jb \bw+ \dot{\chi}\kb$, поэтому
наш интеграл можно переписать в таком виде: $\ints{\Ga}(a,ds) \bw= \ints{\Ga}\bs{P(\ph,\psi,\chi)\dot{\ph}
\bw+ Q(\ph,\psi,\chi)\dot{\psi} \bw+ R(\ph,\psi,\chi)\dot{\chi}}\,ds$.

\begin{note}
Подинтегральное выражение по виду напоминает производную некоторой сложной функции $V\br{\ph(s),
\psi(s),\chi(s)}$, где функции $P,Q,R$ обозначают её частные производные $V_\ph, V_\psi, V_\chi$.
\end{note}

Посмотрим, как можно обойтись без натурального параметра в интеграле второго рода. Пусть $s \bw= s(t)$,
причём для определённости $s'(t) \bw> 0$ на отрезке параметризации. Тогда существует обратная функция $t \bw=
t(s)$, причём $t'(s) \bw> 0$. Тогда $x \bw= u(t) \bw= u\br{t(s)} \bw= \ph(s)$, откуда $\dot{\ph}(s) \bw=
u'(t)\frac{dt}{ds}$, поэтому $\dot{\ph}(s)\,ds \bw= u'(t)\,dt$. Аналогичные формулы можно записать
для функций $\dot{\psi}$ и $\dot{\chi}$. Делая замену переменной в нашем интеграле, получаем
$$
  \ints{\Ga}(a,ds) \bw= \intl{a}{b} P(u,v,w)u' \bw+ Q(u,v,w)v' \bw+ R(u,v,w)w' \,dt.
$$

\begin{df}
Если $\Ga$ замкнута, то $\ints{\Ga}(a,ds)$ называют \emph{циркуляцией векторного поля}.
\end{df}

\begin{note}
Физический смысл криволинейного интеграла второго рода работа по движению материальной точки
вдоль кривой $\Ga$ в поле силы, заданной векторным полем $a$.
\end{note}

\subsubsection{Формула Грина}

Пусть $G$ измеримая область в $\Ebb^2$, причём $\pd G$ есть кривая, обходящая область в
положительном направлении (это означает, что когда мы идём по границе, мы идём левой ногой по
области, а правой по её дополнению). Пусть на $G$ задано векторное поле $a \bw= (P, Q)$ класса
гладкости $\Cb^1(\Cl G)$. \emph{Формула Грина} гласит:
\rightpicture{pictures.20}
$$\ints{\Ga}(P\,d x \bw+ Q\,d y) \bw= \ints{G}(Q_x \bw- P_y)\,dx\,dy.$$

\hangindent=-38mm
\hangafter=-2
Сначала докажем формулу Грина для случая,
когда область $G$ представляет собой треугольник $ABC$, как показано на рисунке, причём сторона $AC$ есть
график монотонной функции $f \bw\in \Cb[a,b]$. Рассмотрим
\begin{multline*}
-\ints{G}P_y\,dx\,dy \bw=
-\intl{a}{b}\intl{c}{f(x)}P_y\,dy\,dx \bw= -\intl{a}{b}P(x,y)\Bigl\rvert_c^{f(x)}\,dx \bw=
-\intl{a}{b}P\br{x,f(x)}\,dx \bw+\intl{a}{b}P(x,c)\,dx =\\
= -\ints{AC}P\,dx \bw+ \ints{A B}P\,d x \bw= \ints{CA}P\,d x \bw+ \ints{A B}P\,d x \bw= \ints{\Ga}P \,dx,
\end{multline*}
поскольку можно замкнуть путь, добавив третье нулевое слагаемое $\ints{BC}P\,d x$.

Аналогично можно поступить с производной $Q_x$: мы знаем, что в силу монотонности $f$ существует монотонная
обратная функция $g \bw{:=} f^{-1} \bw\in \Cb[c,d]$. Тогда
\begin{multline*}
\ints{G}Q_x\,dx\,dy \bw=
\intl{c}{d}\intl{g(y)}{b}Q_x\,dx \,dy\bw= \intl{c}{d}Q(x,y)\Bigl\rvert_{g(y)}^b\,d y \bw=
\intl{c}{d}Q(b,y)\,d y \bw- \intl{c}{d}Q\br{g(y),y}\,d y =\\=
\ints{BC}Q\,d y \bw-\ints{AC}Q\,d y\bw=\ints{BC}Q\,d y \bw+ \ints{CA}Q\,d y \bw+
\ints{A B}Q\,dy \bw= \ints{\Ga}Q\,dy.
\end{multline*}

Имеет смысл подумать, в каких ещё случаях мы умеем доказывать формулу Грина. Ясно, что если область
удаётся порезать на куски, каждый из которых есть некоторый криволинейный треугольник того вида,
который мы рассматривали при доказательстве, то для такой области формула Грина тоже доказана: на
стыке треугольников мы обязательно один раз пройдём в одну сторону и один раз в другую, поэтому
останется только то, что было на границе области, а это и есть наша кривая.

\begin{imp}
Криволинейные интегралы можно использовать для вычисления площадей.
\end{imp}
\begin{proof}
Если $Q_x\bw-P_y\bw=1$, то $\mu(G) \bw= \ints{G} \,dx \,dy \bw= \ints{\Ga}P\,dx \bw+ Q \,dy$. Возьмём
поле с компонентами $P \bw= 0$ и $Q \bw= x$, тогда $Q_x \bw- P_y\bw=1$. Отсюда
$\mu(G) \bw= \ints{\Ga}x \,dy$.
\end{proof}

\subsection{Поверхностные интегралы}

\subsubsection{Геометрический смысл знака якобиана плоского отображения}

Пусть $\Phi$ осуществляет диффеоморфизм областей $G(x,y)$ и $H(u,v)$. Пусть частные производные
$v_{xy}, v_{yx} \bw\in \Cb$, тогда они равны. Пусть $\Om \subs G$ область, для которой справедлива
формула Грина, тогда $\Ga(t) \bw= \pd \Om$ гладкая граница этой области. Положим $\Ga^* \bw{:=}
\Phi(\Ga)$ и $\Om^* \bw{:=} \Phi(\Om)$. Назовём \emph{положительным направлением обхода} на $\Ga^*$
то направление, которое индуцируется положительным направлением обхода кривой $\Ga$. Очевидно,
$\Ga^*$ тоже будет гладкой кривой. Действительно, дифференцируемость следует из теоремы о
производной сложной функции, а регулярность из того, что
$$\rbmat{\dot{u}\\ \dot{v}}\bw= \rbmat{u_x & u_y\\ v_x & v_y}\rbmat{\dot{x} \\ \dot{y}}.$$
В самом деле, матрица линейного оператора в этой формуле в точности матрица Якоби, а она
невырождена. Поэтому если вектор $(\dot x,\dot y)$ не обращается в нуль, то и $(\dot u, \dot v)$ в
нуль не обращается.

Пусть для $\Om^*$ также справедлива формула Грина, тогда $\mu(\Om^*) \bw= \ints{\Om^*}\,du\,dv \bw= \ep
\ints{\Ga^*}u\,d v$, где $\ep \bw= \pm1$ и $\ep \bw=1$ соответствует положительному обходу кривой. Тогда
\begin{multline*}
\ints{\Ga^*}u \,d v \bw= \intl{a}{b}u \dot{v} \,d t \bw= \intl{a}{b} u \hr{v_x \dot{x} \bw+ v_y \dot{y}}\,d t \bw=
\ints{\Ga}\hr{u v_x \,d x \bw+ u v_y\,d y} \stackrel{!}{=} \ints{\Om} \hr{\frac{\pd}{\pd x}\hr{u v_y} \bw-
\frac{\pd}{\pd y}\hr{u v_x}}\,d x \,d y\bw=\\=
\ints{\Om} \hr{u_x v_y \bw+ u\frac{\pd}{\pd x}\hr{v_y} \bw- u_y v_x \bw- u\frac{\pd}{\pd y}\hr{v_x}}\,d x \,d y
\stackrel{!!}{=} \ints{\Om} \hr{u_x v_y \bw- u_y v_x }\,d x \,d y \bw= \ints{\Om} |d\Phi| \,d x\,d y.
\end{multline*}
Здесь переход, отмеченный знаком <<!>>, обусловлен формулой Грина, а отмеченный знаком <<!!>> фактом равенства вторых
смешанных частных производных. Таким образом, можно установить, что $\sgn |d\Phi| \bw= \ep$. Действительно,
по теореме о замене переменных, $\mu(\Om^*) \bw= \ints{\Om^*}d u\,d v \bw= \ints{\Om}|\det d\Phi|\,d x \,dy$.
Таким образом, ориентация кривой сохраняется тогда и только тогда, когда якобиан перехода положителен.

\subsubsection{Потенциальные векторные поля}

Пусть задано векторное поле $a\cln G\ra \Ebb^3$ с компонентами $P, Q, R$.

\begin{df}
Поле $a$ называется \emph{потенциальным} в области $G$, если $\exi V\cln G \ra \R$, для которой $(P,Q,R) \bw=
\hr{V_x,V_y,V_z}$.
В этом случае функция $V$ называется \emph{потенциалом} векторного поля $a$. Используя оператор Гамильтона
$\nabla \bw= \hr{\frac{\pd}{\pd x}, \frac{\pd}{\pd y}, \frac{\pd}{\pd z}}$, можно написать, что
$a \bw= \nabla V \bw= \grad V$.
\end{df}

\begin{theorem}[Критерий потенциальности поля]
Пусть задано векторное поле $a\cln G\ra \Ebb^3$. Тогда следующие условия эквивалентны:

\pt{1}. Поле $a$ потенциально.

\pt{2}. Для любой замкнутой кривой $\ints{\Ga}(a,ds) \bw= 0$.

\pt{3}. $\fa A, B \bw\in G$ интеграл $\ints{\Ga}(a,ds)$ не зависит от выбора $\Ga$: иными словами,
значение криволинейного интеграла по любой кривой, соединяющей точки $A$ и $B$, не зависит от того, по какому
пути мы пойдём.
\end{theorem}
\begin{proof}
Эквивалентность \pt{2} и \pt{3} очевидна, если вспомнить о том, что криволинейный интеграл второго
рода меняет знак при смене параметризации на противоположную.

\pt{1} $\Ra$ \pt{3}. Пусть существует потенциальная функция $V$ для поля $a$. Тогда
$(P, Q, R) \bw= \hr{V_x, V_y, V_z}$. Рассмотрим $A, B \bw\in G$ и кривую
$\Ga_{AB} \cln [0,1] \ra \Ebb^3$ с компонентами радиуса-вектора $\br{\ph(t), \psi(t),\chi(t)}$,
причём $\Ga(0) \bw= A$, а $\Ga(1) \bw= B$. Имеем
\begin{multline*}
\ints{\Ga_{AB}}(a, ds) \bw=
\intl{0}{1}\br{P(\ph,\psi,\chi)\dot{\ph} \bw+ Q(\ph,\psi,\chi)\dot{\psi} \bw+
               R(\ph,\psi,\chi)\dot{\chi}}\,d t =\\=
\intl{0}{1}\hr{V_x(\ph,\psi,\chi)\dot{\ph} \bw+ V_y(\ph,\psi,\chi)\dot{\psi} \bw+
               V_z(\ph,\psi,\chi)\dot{\chi}}\,d t\bw=
\intl{0}{1}(V\circ\Ga)'\,d t \bw= (V\circ\Ga)(1) \bw- (V\circ\Ga)(0) \bw= V(B)\bw-V(A).
\end{multline*}
Тем самым доказано,
что интеграл по $\Ga$ не зависит от самой кривой, поскольку он оказался равен разности значений функции $V$,
а она никак не зависит от кривой.

\pt{3} $\Ra$ \pt{1}. Пусть от кривой ничего не зависит. Зафиксируем точку $A \bw\in G$ и пусть $B \bw\in G$.
Рассмотрим функцию $V(B) \bw{:=} \ints{\Ga_{AB}}(P\,d x \bw+ Q\,d y \bw+ R\,d z)$. Такое определение
корректно, поскольку значение этого интеграла не зависит от $\Ga$. Построим определённый путь в $G$,
соединяющий точки $A$ и $B$. Поскольку $G$ область, точка $B$ обладает некоторой окрестностью $U(B) \subs
G$. Проведём из $B$ отрезок $I$, целиком лежащий в $U$, параллельный вектору $\ib$, и пусть конец этого
отрезка точка $C$. В силу связности $G$, точки $A$ и $C$ можно соединить кривой. Объединение кривой и
отрезка $I$ составят наш путь $\Ga$. Тогда $\ints{\Ga_{AC}} \bw+ \ints{\Ga_{CB}} \bw= \ints{\Ga_{AB}}$.
Разрешим точке $B$ бегать вдоль $I$, тогда $V(B) \bw= \ints{\Ga_{AC}}(a,d s) \bw+ \ints{\Ga_{CB}}P\,dx
\bw= \ints{\Ga_{AC}}(a,ds) \bw+ \intl{x_C}{x_B}P(t,y,z)\,dt$. Продифференцируем это равенство по $x$.
Поскольку первое слагаемое есть константа, от $x_B$ есть только в верхнем пределе интеграла Римана, получаем
$V_x(B) \bw= P(x,y,z)$. Аналогично можно получить, что $V_y(B) \bw= Q(x,y,z)$ и $V_z(B) \bw= R(x,y,z)$.
\end{proof}

\begin{df}
\emph{Ротором} или \emph{вихрем} векторного поля $a$ называется векторное поле $\rot a \cln \Ebb^3 \ra \Ebb^3$,
определённое по правилу
$$
  \rot a \bw{:=} [\nabla, a] \bw= \mbmat{\ib & \jb & \kb\\
  \frac{\pd}{\pd x} & \frac{\pd}{\pd y} & \frac{\pd}{\pd z}\\P & Q & R}\bw=
  \hr{R_y \bw- Q_z, P_z\bw-R_x, Q_x \bw- P_y}.
$$
Если $\rot a \equiv 0$, то поле называется \emph{безвихревым}.
\end{df}

\begin{note}
Формально ротор зависит от системы координат. На самом деле такой зависимости нет, и мы это докажем позднее, когда
выясним геометрический смысл ротора.
\end{note}

\begin{theorem}
Достаточно гладкие потенциальные векторные поля обладают нулевым ротором.
\end{theorem}
\begin{proof}
Если поле имеет потенциал $V$, то $Q_x \bw- P_y \bw=V_{xy} \bw- V_{yx} \bw= 0$, поскольку в условиях
гладкости смешанные частные производные равны между собой. Аналогично доказывается и про остальные компоненты.
\end{proof}

\begin{ex}
Покажем, что условие равенства ротора нулю не является достаточным. Иными словами, существуют непотенциальные
поля с нулевым ротором. Рассмотрим $P(x,y,z) \bw= -\frac{y}{x^2+y^2}$, $Q(x,y,z) \bw= \frac{x}{x^2+y^2}$, а
$R \bw= 0$. Пусть $G \bw= \Ebb^2\bw\wo \hc{0}$. Тогда $P_y\bw=\frac{y^2-x^2}{(x^2+y^2)^2}$,
$Q_x\bw=\frac{y^2-x^2}{(x^2+y^2)^2}$, поэтому $Q_x\bw-P_y \bw= 0$, а остальные компоненты автоматически равны
$0$, поэтому $\rot a \bw= 0$. Рассмотрим верхнюю половинку $S_+$ окружности $x \bw= \cos \ph$, $y \bw= \sin
\ph$, $\ph \bw\in [0,\pi]$. Тогда $\ints{S_+}(P\,d x \bw+ Q\,d y) \bw= \intl{0}{\pi}\br{(-\sin\ph)(-\sin\ph)
\bw+ (\cos\ph)(\cos\ph)}\,d \ph \bw= \pi$. Если теперь посчитать интеграл по нижней половинке окружности
$S_-$, получится $-\pi$, поскольку эта дуга параметризуется отрезком $\ph \bw\in [0,-\pi]$, причём порядок
точек именно такой! Таким образом, поле $a$ не потенциально.

Изучим поведение интегралов второго рода по замкнутым кривым в этом поле. Пусть кривая $\Ga$ ограничивает область $G$,
тогда рассмотрим несколько случаев:

\pt{1}. Пусть $(0,0)\notin G$, тогда применима формула Грина, и $\ints{\Ga}(P\,d x \bw+ Q\,d y) \bw=
\ints{G}(Q_x\bw-P_y)\,dx \,dy \bw= 0$.

\pt{2}. Пусть $G$ окружность c центром $(0,0)$, а кривая $\Ga$ пробегает по этой окружности $n$ раз в
одном и том же направлении. Тривиальный подсчёт, подобный тому, что мы проводили, когда доказывали
непотенциальность, приводит к результату $\ints{\Ga} \bw= 2\pi n$.

\vskip-5pt
\rightpicture{pictures.30}
\hangindent=-57mm
\hangafter=-8
\pt{3}. Пусть $G$ произвольная область, содержащая $(0,0)$. Опишем около нуля окружность, целиком
лежащую в $G$, и разделим область на фрагменты $G_1$, $G_2$ и $D$, как показано на рисунке.
Легко видеть, что интеграл по кривой $\Ga$ можно будет представить в виде суммы трёх интегралов:
по $\Ga_1 \bw{:=} \pd G_1$, по $\Ga_2 \bw{:=} \pd G_2$ и по $S \bw= \pd D$. Поскольку $(0,0) \notin G_1, G_2$,
имеем $\ints{\Ga_1} \bw= 0$, $\ints{\Ga_2} \bw= 0$, а $\ints{S} \bw= 2\pi$. Можно рассмотреть
несколько оборотов вокруг особой точки, тогда получим $\ints{S} \bw= 2\pi n$.
\end{ex}

\begin{note}
Функция $V(x,y) \bw= \arctg \frac yx$ годится для областей $x \bw> 0$ и $x \bw< 0$, поскольку если
её продифференцировать, то получатся как раз компоненты $P$ и $Q$ векторного поля, рассмотренного в
примере. Но поскольку эта функция вообще не определена при $x \bw= 0$, потенциалом поля
на $\Ebb^2 \bw\wo \hc{0}$ она не является.
\end{note}

\begin{theorem}
Пусть $G$ открытый прямоугольник (возможно, неограниченный) со сторонами, параллельными системе координат.
Тогда любое гладкое векторное поле~$a$ в $G$ с нулевым ротором потенциально.
\end{theorem}
\begin{proof}
Рассмотрим $A(x_0,y_0,z_0), B(x,y,z) \bw\in G$ и соединим их кусочно гладким путём, состоящим из трёх отрезков,
параллельных системе координат: $\Ga_{AB} =(x_0,y_0,z_0)\dots(x,y_0,z_0)\dots(x,y,z_0)\dots(x,y,z)$. Несложно видеть,
что указанный путь целиком лежит в области $G$. Покажем, что искомым потенциалом будет функция
$$V(B) \bw{:=} \ints{\Ga_{AB}}(a,ds) \bw= \intl{x_0}{x}P(u,y_0,z_0)\,du
\bw+ \intl{y_0}{y}Q(x,v,z_0)\,d v \bw+ \intl{z_0}{z}R(x,y,w)\,dw.$$

В силу непрерывности поля, можно дифференцировать под знаком интеграла, тогда $V_x \bw= P(x,y_0,z_0) \bw+
\intl{y_0}{y}Q_x(x,v,z_0)\,d v \bw+ \intl{z_0}{z}R_x(x,y,w)\,d w$. Если ротор нулевой, то $R_y\bw=Q_z$,
$P_z\bw=R_x$ и $Q_x \bw= P_y$. Тогда $V_x \bw= P(x,y_0,z_0) \bw+ \intl{y_0}{y}P_y(x,v,z_0)\,d v \bw+
\intl{z_0}{z}P_z(x,y,w) \,d w \bw= \lcomm$ ФНЛ $\rcomm \bw= P(x,y_0,z_0) \bw+ P(x,y,z_0)\bw-P(x,y_0,z_0) \bw+
P(x,y,z)\bw-P(x,y,z_0)$. Аналогично проверяется, что $V_y \bw= Q(x,y,z)$ и $V_z \bw= R(x,y,z)$.
Тем самым, построен потенциал для поля $a$.
\end{proof}

\subsubsection{Гладкие поверхности и поверхностные интегралы}

\begin{df}
Скажем, что в $\Ebb^3$ задана \emph{гладкая поверхность}, если задана функция
$r\cln \Om(u,v) \ra \Ebb^3(x,y,z)$, причём $r \bw\in \Cb^1(\Om)$, и
векторы $r_u$ и $r_v$ не коллинеарны. Последнее условие можно записать так: $\hs{r_u, r_v} \neq 0$.
Действительно, модуль векторного произведения $[a,b]$ это площадь параллелограмма, натянутого
на вектора $a$ и $b$, поэтому если они коллинеарны, то оно нулевое.
\end{df}

Поверхности можно задавать и другими способами. Например, если $z \bw= f(x,y)$ гладкая функция,
то её график будет гладкой поверхностью в $\Ebb^3$. Действительно, имеем
$r_x \bw= \br{1, 0, f_x}$ и $r_y \bw= \br{0, 1, f_y}$, поэтому эти вектора не могут быть коллинеарными.
Кроме того, если функция задана неявно гладким уравнением вида $F(x,y,z)\bw= 0$, но его можно
разрешить относительно одной из переменных,
то локально это уравнение также будет задавать гладкую поверхность.

Вернёмся к общему определению и посмотрим, что даёт условие неколлинеарности. Распишем векторное
произведение: \eqn{\label{tan.vec.coord.form} \mbmat{\ib & \jb & \kb\\ x_u & y_u & z_u \\ x_v & y_v & z_v}
\bw= \hr{\De_{yz}, \De_{zx}, \De_{xy}}, \quad \De_{yz} \bw{:=} \pf{(y,z)}{(u,v)},\; \De_{zx} \bw{:=}
\pf{(z,x)}{(u,v)},\; \De_{xy} \bw{:=} \pf{(x,y)}{(u,v)}.} Если этот вектор ненулевой, то хотя бы одна из его
компонент ненулевая. Пусть, для определённости, $\De_{xy} \neq 0$, тогда по теореме о системе неявных
функций, можно выразить $u,v$ через $x,y$: $u \bw= u(x,y)$, $v \bw= v(x,y)$. Следовательно, $z \bw= z(u,v)
\bw= z\br{u(x,y), v(x,y)}\bw=f(x,y)$. Таким образом, все три способа локально эквивалентны.

Рассмотрим замену переменных для параметров, задающих поверхность. Пусть
$$
  \case{x \bw= x(u,v),\\y \bw= y(u,v),\\ z \bw= z(u,v),} \quad \case{u \bw= u(s,t),\\v \bw= v(s,t).}
$$
Дифференцируя сложную функцию, получаем $\pf{(x,y)}{(s,t)} \bw= \De_{xy}\cdot \pf{(u,v)}{(s,t)}$.
Аналогично получаем $\pf{(y,z)}{(s,t)} \bw= \De_{yz}\cdot \pf{(u,v)}{(s,t)}$ и $\pf{(z,x)}{(s,t)} \bw=
\De_{zx}\cdot\pf{(u,v)}{(s,t)}$. Используя соотношение (\ref{tan.vec.coord.form}), получаем
$\hs{r_s, r_t} \bw= \pf{(u,v)}{(s,t)} \hs{r_u, r_v}$. Таким образом, если мы хотим, чтобы замена
сохраняла регулярность поверхности, необходимо требовать невырожденности матрицы Якоби замены координат,
что вполне естественно.

Зафиксируем один из параметров $v \bw= v_0$, а второй параметр $u$ будем менять. Тогда на поверхности
получится гладкая кривая с касательным вектором $r_u(u,v_0)$. Эта кривая называется \emph{$u$-линией}.
Аналогично определяются \emph{$v$-линии} на поверхности. Рассмотрим касательные вектора $r_u$ и $r_v$.
Они, очевидно, лежат в касательной к поверхности плоскости, поскольку иначе мы получили бы противоречие с
близостью точек касательной плоскости и поверхности. Поскольку они не коллинеарны, они образуют базис
касательной плоскости.

Будем считать, что функции, задающие поверхность, можно непрерывно продолжить на $\Cl \Om$.

\begin{df}
Если $r(\pd \Om)$ является кусочно-гладкой кривой, то мы будем называть такую поверхность \emph{гладким куском}.
Если поверхность может быть разбита на конечное число гладких кусков, то мы будем называть её \emph{кусочно-гладкой}.
\end{df}

\subsubsection{Площади поверхностей}
Пусть поверхность $S$ задана функцией $z \bw= f(x,y)$ в измеримой области $\Om$. Рассмотрим разбиение
$\Om \bw= \Om_1\sop \Om_N$. Пусть $(x_i,y_i) \bw\in \Om_i$, тогда
$Q_i \bw{:=} \br{x_i,y_i, f(x_i,y_i)} \bw\in S$. Пусть $L_i \bw{:=} T_{Q_i}S$ касательная плоскость
в точке $Q_i$ к поверхности $S$, а $l_i$ тот её кусок, который проецируется на $\Om_i$. Будем называть
эти куски чешуйками. Поскольку <<в малом>> касательная плоскость совпадает с поверхностью, а разбиение
можно считать достаточно мелким, естественно было бы назвать площадью поверхности предел сумм площадей
этих чешуек, если он существует. Формализуем это понятие.

Обозначим $\la \bw{:=} \cos\angle(\nb, \kb)$, где $\nb$ вектор нормали к поверхности в точке $Q_i$. Из
геометрических соображений ясно, что если $\Om_i$ прямоугольник, то $\mu(\Om_i) \bw= \la\cdot\mu(l_i)$. Если же
$\Om_i$ не является прямоугольником, то, в силу измеримости, его всегда можно приблизить сколь угодно точно
наборами прямоугольников. Дефект может внести только граница, но $\mu(\pd \Om_i) \bw= 0$, поэтому она не
навредит. Итак, соотношение $\mu(\Om_i) \bw= \la\cdot \mu(l_i)$ верно для произвольных измеримых множеств.

Найдём коэффициент $\la$. Пусть $F(x,y,z) \bw= f(x,y) \bw- z$, тогда нормаль может быть записана как $\nb
\bw= \grad F \bw= (f_x,f_y,-1)$. Поскольку нам нужны значения $\la$ в точках $(x_i,y_i)$, не будем для
краткости записи писать аргументы у частных производных. Имеем $\la \bw= \frac{|(\nb,\kb)|}{|\nb|\cdot|\kb|}
\bw= \frac{1}{\sqrt{f_x^2 \bw+ f_y^2 \bw+ 1}}$. Таким образом, получаем формулу $\mu(l_i) \bw= \sqrt{f_x^2
\bw+ f_y^2 \bw+ 1}\cdot\mu(\Om_i)$. Из этих слагаемых естественным образом составляется интегральная сумма Римана,
и мы приходим к формуле площади гладкой поверхности:
$$\area(S) \bw= \ints{\Om}\sqrt{f_x^2 \bw+ f_y^2 \bw+ 1} \,d x\,d y.$$

Рассмотрим теперь общий случай, когда поверхность задана параметрически. Как уже было выяснено в предыдущем разделе,
без ограничения общности можно считать, что
$$x \bw= x(u,v),\; y \bw= y(u,v),\; z \bw= z\br{x(u,v),y(u,v)}.$$
Чтобы выразить из этих соотношений $z_x, z_y$, продифференцируем $z$ как сложную функцию по $u$ и $v$.
Получим систему линейных уравнений
$$\rbmat{z_u\\z_v} \bw= \rbmat{x_u & y_u\\x_v & y_v} \rbmat{z_x\\z_y}.$$
Матрица этой системы в точности матрица Якоби, а она в данном случае невырождена. Решая систему по правилу
Крамера, находим $$z_x \bw= \frac{\De_{zy}}{\De_{xy}},\; z_y \bw=\frac{\De_{xz}}{\De_{xy}}.$$
Пусть $\Om(x,y) \lra G(u,v)$, тогда остаётся подставить выражения для производных и сделать замену
переменных в кратном интеграле:
$$
  \ints{\Om}\sqrt{z_x^2 \bw+ z_y^2 \bw+ 1}\,d x \,d y \bw= \ints{G}\sqrt{\frac{\De_{zy}^2}{\De_{xy}^2}
  \bw+ \frac{\De_{xz}^2}{\De_{xy}^2} \bw+ 1}
  \hm{\De_{xy}}\,d u\,d v \bw= \ints{G}\sqrt{\De_{zy}^2\bw+\De_{xz}^2 \bw+ \De_{xy}^2 }\,d u\,d v \bw=
  \ints{G}\hm{\hs{r_u, r_v}}\,d u\,d v.
$$

Осталось проверить, что данное определение не зависит от выбора параметризации. Пусть задан диффеоморфизм
$G(u,v) \lra H(s,t)$. Тогда $\pf{(u,v)}{(s,t)} \neq 0$, а векторные произведения связаны формулой $\hs{r_u,
r_v} \bw= \hs{R_s, R_t} \pf{(s,t)}{(u,v)}$. Тогда $\ints{G}\bm{\hs{r_u, r_v}}\,d u\,d v \bw=
\ints{H}\bm{\hs{R_s, R_t}}\cdot\hm{\pf{(s,t)}{(u,v)}}\cdot\hm{\pf{(u,v)}{(s,t)}}\,d s\,d t\bw=
\ints{H}\bm{\hs{R_s, R_t}}\,d s\,d t$.

\subsubsection{Поверхностный интеграл первого рода}

\begin{df}
Пусть в области $\Om(u,v)$ параметрически задана хорошая поверхность $S$ своим радиус вектором. Пусть
$\Phi\cln S\ra \R$. Если существует интеграл $\ints{\Om}\Phi(r)\bm{\hs{r_u, r_v}}\,d u \,d v$, то говорят,
что на $S$ задан \emph{поверхностный интеграл первого рода}. Обозначение: $\ints{S}\Phi\,d S$.
\end{df}

Многие свойства этого интеграла, например, линейность, очевидны из свойств кратного интеграла. Можно
показать, что его значение не зависит от параметризации поверхности, но мы не будем этого проделывать,
поскольку рассуждения аналогичны тем, что были приведены в конце предыдущего параграфа.

Когда поверхность задана функцией $z \bw= f(x,y)$, получаем $\ints{S}\Phi\,d S \bw=
\ints{\Om}\Phi\br{x,y,f(x,y)}\sqrt{f_x^2 \bw+ f_y^2 \bw+ 1}\,d x \,d y$.

\subsubsection{Ориентированные поверхности. Поверхностный интеграл второго рода}

Пусть $S$ хорошая поверхность, заданная радиус-вектором $r(u,v)$. Изготовим нормаль к поверхности:
$$\nb \bw= \pm \frac{\hs{r_u, r_v}}{\bm{\hs{r_u,r_v}}}.$$
Знак <<$+$>> или <<$-$>> мы можем выбирать произвольно, что соответствует выбору <<положительной>>
или <<отрицательной>> стороны поверхности, или \emph{ориентации}.

\begin{df}
\emph{Полем нормалей} к поверхности $S$ называется отображение $N\cln S \ra \Ebb^3$, заданное по правилу
$N\cln P \mapsto \nb(P)$, при условии, что мы зафиксировали один и тот же знак для всех нормалей. Если $N$
гладкое отображение, то $S$ называется \emph{ориентируемой}, или \emph{двухсторонней}, поверхностью.
\end{df}

\begin{ex}
Лист Мёбиуса не обладает подобным свойством: непрерывно протащив вектор нормали по его оси, мы не
сможем состыковать его с начальным положением, ибо они будут направлены в противоположные стороны.
Такие поверхности называются \emph{неориентируемыми}.
\end{ex}

\begin{df}
Если на ориентируемой поверхности зафиксирована какая-либо ориентация, то такую поверхность
называют \emph{ориентированной}. Мы будем обозначать ориентированные поверхности символом $S^*$.
\end{df}

Обычно мы будем выбирать нормаль со знаком $+$. Это не ограничивает общности, поскольку при перестановке
переменных параметризации $u$ и $v$ вектор нормали изменит знак на противоположный.

\begin{ex}
Пусть поверхность $S$ задана функцией $z \bw= f(x,y)$. Тогда вполне естественно называть положительной её
верхнюю сторону, на которой $(\nb, \kb) \bw> 0$.
\end{ex}

Направление обхода границы гладкого куска поверхности должно быть согласовано с выбранной ориентацией:
если вектор нормали смотрит на нас, то обход контура должен быть задан против часовой стрелки. При
выборе ориентации для кусочно гладких поверхностей нужно следить за тем, чтобы направления обходов границ
на стыке были противоположными. Мы \emph{не} будем исследовать вопрос о том, когда это можно сделать.

Выясним, какими свойствами должна обладать замена параметров, чтобы знак нормали сохранялся. При
замене координат $r(u,v) \lra R(s,t)$ имеем
$$
  \nb \bw= \frac{\hs{r_u, r_v}}{\bm{\hs{r_u, r_v}}} \bw=
  \frac{\hs{R_s, R_t}\cdot \pf{(u,v)}{(s,t)}}{\bm{\hs{R_s, R_t}}\cdot \hm{\pf{(u,v)}{(s,t)}}}.
$$
Отсюда видно, что для сохранения знака нормали необходимо требовать $\pf{(u,v)}{(s,t)} \bw> 0$, что естественно.

\begin{df}
Пусть $S^*$ ориентированная поверхность, на которой задано непрерывное векторное поле $a \bw= (P, Q, R)$.
Тогда $(a, \nb)$ тоже будет непрерывной функцией. В этом случае имеет смысл говорить об интеграле
$\ints{S}(a, \nb)\,d S$, который обозначается $\ints{S^*}(a, d S^*)$ и называется \emph{поверхностным
интегралом второго рода}.
\end{df}

Получим явное представление для этого интеграла:
$$\ints{S^*} (a,d S^*) \bw= \ints{S}\frac{P \De_{yz} \bw+ Q\De_{zx} \bw+ R\De_{xy}}{|\nb|}\,d S \bw=
\ints{\Om} \frac{P \De_{yz} \bw+ Q\De_{zx} \bw+ R\De_{xy}}{|\nb|}|\nb| \,d u \,d v \bw=\ints{\Om} P \De_{yz}
\bw+ Q\De_{zx} \bw+ R\De_{xy} \,d u \,d v.$$

Пусть $A \subs \Ebb^3$. Через $A_x, A_y, A_z$ мы будем обозначать проекции $A$ на координатные плоскости
параллельно осям $x, y, z$ соответственно.

Предположим, что векторное поле имеет вид $(0, 0, R)$, а поверхность $S$ такова, что однозначно проецируется
на плоскость $(x,y)$. Это означает, что её можно представить в виде графика функции $z \bw= f(x,y)$. В этом
случае $\De_{xy} \bw= 1$, поэтому получаем
$\ints{S^*}(a, d S^*) \bw= \pm \ints{S_z}R\br{x,y,f(x,y)}\,d x \,d y \bw{=:}
\ints{S^*}R(x,y,z)\,d x \,d y$. Следует понимать, что последняя запись всего лишь условное
обозначение. Все наши рассуждения можно слово в слово повторить для полей вида $(0, Q, 0)$ и $(P,0,0)$, если
дополнительно предположить, что поверхность допускает проецирование на плоскости $(x,z)$ и $(y,z)$. Используя
похожие обозначения, приходим к символической записи
$$\ints{S^*}(a, d S^*) \bw= \ints{S^*}P(x,y,z)\,d y \,d z \bw+ Q(x,y,z)\,d x\,d z \bw+ R(x,y,z)\,d x \,d y.$$

\subsubsection{Формула Гаусса Остроградского}

Пусть $G \subs \Ebb^3$ измеримая область, причём её граница $S \bw{:=} \pd G$ является кусочно гладкой поверхностью.
Пусть в области $G$ задано векторное поле $a \bw\in \Cb^1$.

\begin{df}
\emph{Дивергенцией} векторного поля $a$ называется функция $\Div a \bw{:=} (\nabla, a) \bw= P_x \bw+ Q_y \bw+ R_z$.
\end{df}

\emph{Формула Гаусса Остроградского} утверждает, что
$$
  \ints{G}\Div a \,d x \,d y \,d z \bw= \ints{S^*}(a,d S^*) \bw= \ints{S}(a,\nb)\,d S,
$$
где $S^*$ отвечает внешней стороне поверхности.

\begin{df}
Рассмотрим односвязную измеримую область $V$ в плоскости $(x,y)$, ограниченную гладкой кривой, и
цилиндр над ней. Кроме того, рассмотрим гладкие функции $U(x,y) \bw\ge L(x,y)$, которые назовём
\emph{верхней} и \emph{нижней крышками}. \emph{Цилиндроидом} над $V$ называется область
$G \bw= V \times \bs{L(x,y), U(x,y)} \subs \Ebb^3$. Обозначения $L$ и $U$ для крышек происходят,
соответственно, от английских слов \emph{lower} <<нижний>> и \emph{upper} <<верхний>>.
\end{df}

Докажем нашу формулу в частном случае, когда $a \bw= (0,0,R)$. Пусть $G$ представляет собой
некоторый цилиндроид над $G_z$, причём его цилиндрическая часть может и отсутствовать. Тогда
\begin{multline*}
\ints{G} \Div a \,d x \,d y \,d z \bw= \ints{G} R_z\,d x \,d y \,d z \bw=
\ints{G_z} \intl{L}{U}R_z \,dz \,dx \,dy \stackrel{!}{=}
\ints{G_z} \hs{R\br{x,y,U(x,y)} \bw- R\br{x,y, L(x,y)}} \,d x \,d y \stackrel{!!}{=}\\ \stackrel{!!}{=}
\ints{S^*_z}R\br{x,y,U(x,y)}\,d x\,d y \bw+ \ints{S^*_z}R\br{x,y,L(x,y)}\,d x\,d y \stackrel{!!!}{=}
\ints{S}(a,\nb)\,d S.
\end{multline*}
Переход <<!>> обоснован с помощью ФНЛ. Далее, <<!!>> следует из того, что мы употребили знак <<$-$>> на то,
чтобы установить ориентацию в нижней крышке. Действительно, поскольку мы договорились брать внешнюю нормаль,
на верхней крышке она и так такая, какая нам нужна, а на нижней крышке она неправильная, поэтому её надо
исправить. Что касается третьего перехода, отмеченного <<!!!>>, то он обоснован тем, что поверхностный
интеграл по цилиндрической части равен 0. Действительно, на этой части поверхности нормаль параллельна
плоскости $(x,y)$, поэтому $a \bot \nb$, значит, $(a,\nb) \bw= 0$. Значит, эту компоненту интеграла
действительно можно добавить к двум другим слагаемым. Таким образом, в этом частном случае формула доказана.

Теперь перейдём к общему случаю. Здесь уже необходимо предположить, что область $G$ представляется в
виде цилиндроидов по всем трём осям. Осталось воспользоваться аддитивностью.

\begin{imp}
Если $G$ хорошая выпуклая область, то условия теоремы заведомо выполнены.
\end{imp}

\begin{imp}
Если область можно разбить на хорошие выпуклые куски, то формула тоже верна, поскольку на общей границе кусков
нормали будут выбраны противоположные, и интегралы по этим кускам уничтожатся.
\end{imp}

\begin{imp}\label{div.indep}
Дивергенция поля не зависит от системы координат.
\end{imp}
\begin{proof}
Пусть $\Div a \bw\in \Cb(G)$. Рассмотрим $A \bw\in G$, тогда $\exi$ шарик $U_\ep(A) \subs G$. Пусть
$S^*_\ep \bw{:=} \pd U_\ep$. Применяя ФГО, получаем $\ints{U_\ep} \Div a \,d x \,d y \,d z \bw=
\ints{S^*_\ep}(a, d S^*)$. Применим к интегралу в левой части теорему о среднем\footnote{Речь идёт об
обобщении одномерной первой теоремы о среднем на случай кратных интегралов. Её доказательство
ничем не отличается от доказательства одномерной теоремы. \emph{Прим. наб.}}, получим $(\Div a)(\xi)\cdot \mu(U_\ep)
\bw= \ints{S^*_\ep}(a, d S^*)$, где $\xi \bw\in U_\ep$. Будем уменьшать радиус $\ep$ шарика, тогда в
пределе шарик стянется в точку $A$. В силу непрерывности, $(\Div a)(A) \bw= \liml{\ep \ra 0}
\frac{1}{\mu(U_\ep)} \ints{S^*_\ep}(a, d S^*)$. Выражение в правой части не зависит от системы координат,
поэтому и дивергенция от неё не зависит. Эту формулу можно считать \emph{геометрическим определением
дивергенции}.
\end{proof}

\begin{df}
Поле $a$ называется \emph{соленоидальным} в области $G$, если для любой кусочно гладкой, замкнутой
поверхности $S \subs G$ имеем $\ints{S}(a,\nb) \,d S \bw= 0$.
\end{df}

\begin{df}
Область называется \emph{поверхностно-односвязной}, если любую замкнутую поверхность в области
можно непрерывно стянуть в точку.
\end{df}
\begin{ex}
Открытый шар является поверхностно-односвязной областью. Если $U$ открытый шар радиуса $R$, а
$B$ замкнутый шар с центром в той же точке и радиуса $r \bw< R$, то $U \bw\wo B$ <<вишня
без косточки>> доставляет пример односвязной области, не являющейся поверхностно односвязной.
\end{ex}

Из следствия \ref{div.indep} следует, что $\Div a \bw= 0$ в соленоидальном поле. Таким
образом, $\Div a \bw= 0$ необходимое условие для соленоидальности. Как следует из ФГО,
для поверхностно односвязных областей это также и достаточное условие.

\subsubsection{Формула Стокса}
Пусть $a$ векторное поле с непрерывным ротором. \emph{Формула Стокса} гласит:
$$
  \ints{S^*}(\rot a, d S^*) \bw= \ints{\Ga^*}(a, d s).
$$
Здесь $S^*$ ориентированная гладкая поверхность, а $\Ga^*$ край этой поверхности с
согласованной ориентацией.

Сначала мы докажем формулу в частном случае, когда $a \bw= (P,0,0)$, а поверхность задана
функцией $z \bw= f(x,y)$. В этом случае имеем $\rot a \bw= (0,P_z,-P_y)$. Распишем интеграл
в левой части формулы Стокса:
$$
  \ints{S^*}(\rot a, d S^*) \bw= \ints{S}(\rot a,\nb)\,d S \bw=
  -\ints{S} \br{P_y\cos(\kb, \nb) \bw- P_z\cos(\jb, \nb)}\,d S \bw{=:} I.
$$

Рассмотрим уравнение касательной плоскости: $f_x(x\bw-x_0) \bw+ f_y(y\bw-y_0) \bw- (z\bw-z_0) \bw= 0$.
Нормированные коэффициенты в этом уравнении и есть направляющие косинусы нормали. Обозначим $K \bw{:=}
\sqrt{f_x^2 \bw+ f_y^2 \bw+ 1}$ нормирующий множитель. Тогда $\cos(\jb, \nb) \bw= \ep\frac{f_y}{K}$, а
$\cos (\kb, \nb) \bw= \ep\frac{-1}{K}$, откуда $\cos(\jb, \nb) \bw= -f_y\cos (\kb, \nb)$. Подставляя эти
косинусы в интеграл $I$, получаем
\begin{multline*}
I \bw= \bw-\ints{S}(P_y \bw+ P_zf_y)\cos (\kb, \nb)\,d S \bw=
-\ints{S}\pf{}{y}P\br{x,y,f(x,y)} \cos(\kb, \nb)\,d S =\\=
-\ints{S_z^*} \pf{}{y} P(x,y,z) \ep\frac{-1}{K} K\,d x \,d y \bw=
-\ints{S_z^*} \pf{}{y} P(x,y,z)\,d x \,d y \bw{=:} J.
\end{multline*}
В этой формуле знак <<$-$>> выбран за счёт ориентации проекции.

Применим к этому интегралу формулу Грина, не особо задумываясь о том, можно ли это делать. Поскольку
$Q \bw= 0$ и $R \bw= 0$, получаем $J \bw= \ints{\Ga_z^*}P\br{x,y,f(x,y)}\,d x \bw= \ints{\Ga^*}(a, d s)$,
что и требовалось доказать.

В этом доказательстве по существу было только то, что $R \bw= 0$, а для полей вида $(0,Q, 0)$ оно вновь
проходит слово в слово. Если функция допускает проецирование на плоскость $(x,z)$, то проблем тоже не
возникает. Такой трюк заведомо не пройдёт, если всё расположено в плоскости, параллельной $(x, y)$. Но тогда
$\nb \bw= \const$, и несложно показать, что формула Стокса утверждает равенство типа $0\bw=0$.

\begin{imp}
Ротор векторного поля не зависит от системы координат.
\end{imp}
\begin{proof}
Рассмотрим нормаль $\nb$ в какой-либо точке области $G$ и рассмотрим перпендикулярную ей плоскость,
проходящую через её начало. Рассмотрим в этой плоскости диск $\si_\ep$ радиуса $\ep$. Здесь мы используем
рассуждения, похожие на те, которые были при доказательстве независимости дивергенции от системы координат.
Из формулы Стокса получаем формулу $(\rot a, \nb)(A) \bw= \liml{\ep \ra 0}
\frac{1}{\mu(\si_\ep)}\ints{\Ga_\ep}(a, d s)$, где $\Ga_\ep \bw= \pd \si_\ep$. Это выражение для ротора
может служить его геометрическим определением.
\end{proof}

Рассмотрим вопрос о потенциальности полей. Мы доказывали, что из потенциальности поля следует равенство
$\rot a \bw= 0$. Обратное, вообще говоря, неверно, но если нам удастся в некоторой области $G$ для
любого контура выбирать поверхность, затягивающую этот контур, то из формулы Стокса и критерия
потенциальности следует потенциальность поля:
$\ints{\Ga}(a, d s) \bw= \ints{S}(\rot a, \nb)\,d S  \bw= 0$.

\section{Начальные сведения о дифференциальных формах}

\subsection{Дифференциальные формы}

\subsubsection{Гладкие многомерные поверхности}
Пусть задано пространство $\Ebb^m$, и $1 \bw\le k \bw\le m$. Сейчас мы заново определим $k$-мерные поверхности,
несмотря на то, что в трёхмерном пространстве это уже было сделано. Пусть сначала $k \bw= 1$.

\begin{df}
Будем говорить, что задана \emph{гладкая одномерная поверхность}, если
$\Phi\cln G \ra \Ebb^m$ биективно и имеет класс гладкости $\Cb^1$ в области $G \subs \Ebb^1$.
В координатной форме $\Phi(t) \bw= \br{\ph_1(t)\sco \ph_m(t)}$.
Потребуем также, чтобы ранг дифференциала отображения $\Phi$ был полным, иначе говоря,
$\rk d\Phi \bw= \rk (\ph_1'\sco \ph_m') \bw= 1$. Направление обхода такой гладкой поверхности,
которую можно называть гладкой кривой, задаётся вдоль возрастания параметра $t$.
\end{df}

\begin{note}
Мы будем называть матрицей Якоби матрицу \emph{дифференциала отображения}, даже если она не квадратная.
По этой причине приходится говорить о рангах, а не об определителях.
\end{note}

Теперь рассмотрим случай $k \bw= 2$.

\begin{df}
Будем говорить, что задана \emph{гладкая двумерная поверхность}, если
$\Phi\cln G \ra \Ebb^m$ биективно и имеет класс гладкости $\Cb^1$, причём $\rk d\Phi \bw= 2$, а
$G$ удовлетворяет следующим свойствам:

\pt{1}. $G \subs \Ebb^2$ выпуклая и ограниченная область.

\pt{2}. Отображение $\Phi$ и его частные производные могут быть продолжены по непрерывности на $\Cl G$.

\pt{3}. $\pd G$ одномерная поверхность.
\end{df}

\emph{Ориентация} на поверхности задаётся так. Пусть на $\pd G$ задана ориентация, тогда
рассмотрим касательный вектор $v \bw= (\ph'_1, \ph'_2)$ к этой кривой. Рассмотрим также
внешнюю нормаль $n \bw= (n_1, n_2)$ к области $G$. Будем говорить,
что параметризация $\pd G$ \emph{согласована с внешней нормалью}, если
\vskip-10pt
\rightpicture{pictures.40}
$$\mbmat{n_1 & \ph'_1\\n_2 & \ph'_2} \bw> 0.$$

\hangindent=-30mm
\hangafter=-2
\noindent
Раскрывая определитель, получаем
$n_1\ph'_2 \bw- n_2\ph'_1 \bw> 0$, но это условие означает, что угол между векторами
$n$ и $(\ph'_2, \bw-\ph'_1) \bw{=:} \wt{v}$ положителен. Заметим теперь, что~$v$
получается из $\wt{v}$ поворотом на угол $+\frac\pi2$, поэтому $\wt{v}$ и $n$ сонаправлены.

Переходим к общему случаю, и будем считать, что уже определены поверхности размерности $k\bw-1$.

\begin{df}
Будем говорить, что задана \emph{гладкая $k$-мерная поверхность}, если задано биективное
отображение $\Phi\cln G \ra \Ebb^m$ класса гладкости $\Cb^1$, причём $\rk d\Phi \bw= k$, а
$G$ удовлетворяет следующим свойствам:

\pt{1}. $G \subs \Ebb^k$ выпуклая и ограниченная область.

\pt{2}. Отображение $\Phi$ и его частные производные могут быть продолжены по непрерывности на $\Cl G$.

\pt{3}. $\pd G$ является поверхностью размерности $k\bw-1$.

\end{df}

\begin{note}
Часто поверхностью называют не отображение $\Phi$, а его образ $S \bw= \Phi(G) \subs \Ebb^m$.
\end{note}

\begin{df}
\emph{Границей} поверхности $S$ называется множество $\Phi(\pd G)$ и обозначается $\pd S$.
\end{df}

Из определения следует, что граница $k$-мерной поверхности является гладкой поверхностью размерности $k\bw-1$.
Действительно, пусть $\chi$ задаёт границу $\pd G$, тогда $d(\Phi\circ\chi) \bw= d\Phi\circ d\chi$,
но это композиция невырожденных линейных отображений, поэтому $\rk d(\Phi\circ\chi) \bw= \max$.
Гладкость следует из теоремы о дифференцируемости композиции.

Остаётся разобраться с \emph{ориентацией}. Если $\chi(\tau)$ задаёт границу $\pd G$,
то, по аналогии с двумерным случаем, будем говорить, что ориентации согласованы, если
$$
\mbmat{
n_1 & \pf{\chi_1}{\tau_1} & \dots & \pf{\chi_1}{\tau_{k-1}}\\
\vdots & \vdots & \ddots & \vdots \\
n_k & \pf{\chi_k}{\tau_1} & \dots & \pf{\chi_k}{\tau_{k-1}}\\
} \bw> 0,
$$
где $n \bw= (n_1\sco n_k)$ есть вектор внешней нормали к области $G$. Напомним, что для упрощения
жизни мы рассматриваем выпуклые области.

Посмотрим, что нужно потребовать, если мы хотим сохранить ориентацию при переходе к другой системе
координат. Пусть сделана замена $\tau \lra \nu$, тогда, дифференцируя сложные функции, получаем
$$
  \mbmat{
    n_1 & \pf{\chi_1}{\nu_1} & \dots & \pf{\chi_1}{\nu_{k-1}}\\
    \vdots & \vdots & \ddots & \vdots \\
    n_k & \pf{\chi_k}{\nu_1} & \dots & \pf{\chi_k}{\nu_{k-1}}\\
  } \bw=
  \mbmat{
    n_1 & \pf{\chi_1}{\tau_1} & \dots & \pf{\chi_1}{\tau_{k-1}}\\
    \vdots & \vdots & \ddots & \vdots \\
    n_k & \pf{\chi_k}{\tau_1} & \dots & \pf{\chi_k}{\tau_{k-1}}\\
  } \cdot
  \mbmat{
    1 & 0 & \dots & 0\\
    0 & \pf{\tau_1}{\nu_1} & \dots & \pf{\tau_1}{\nu_{k-1}}\\
    \vdots & \vdots & \ddots & \vdots \\
    0 & \pf{\tau_{k-1}}{\nu_1} & \dots & \pf{\tau_{k-1}}{\nu_{k-1}}\\
  }.
$$
Из этого соотношения сразу следует, что нужно условие $\hm{\pf{\tau}{\nu}} \bw> 0$. Действительно,
мы хотим, чтобы определитель слева был положителен, а первый сомножитель справа положителен по условию.
Таким образом, выбранная ориентация на $\pd G$ индуцирует некоторую ориентацию на $\pd S$.

Рассмотрим частный случай, когда поверхность задана графиком функции. Для удобства положим $p \bw{:=} k\bw-1$.
В нашем случае имеем
$$
  \case{t_1 \bw= t_1,\\\dots\\t_p \bw= t_p,\\t_k \bw= f(t_1\sco t_p),} \quad \pf{t}{t} \bw=
  \id \quad \Ra \quad \rk \bw= p.
$$
Рассмотрим уравнение касательной плоскости:
$$
  t_k\bw-t_k^0 \bw- \hr{f_{t_1}(t_1\bw-t_1^0)\spl f_{t_p}(t_p \bw- t_p^0)} \bw= 0.
$$
Следовательно, вектор нормали имеет координаты $\hr{\bw-f_{t_1}\sco \bw-f_{t_p}, 1}$. Чтобы проверить,
согласован ли вектор нормали с ориентацией, необходимо вычислить знак определителя
$$
  D \bw{:=} \mbmat{
  -f_{t_1} & 1 &  & \bigsym{-4pt}{0}\\
  \vdots   &   & \ddots & \\
  -f_{t_p} & \bigsym{1pt}{0} & & 1\\
  1 & f_{t_1} & \dots & f_{t_p}\\}.
$$

Этот определитель легко вычислить,
транспонировав его матрицу и разложив по первому столбцу.
Получаем, что при нечётных $k$ он равен $1 \bw+ f_{t_1}^2\spl f_{t_p}^2$, а при чётных
тому же числу, но со знаком <<$-$>>. Таким образом, для нечётных $k$ нормаль согласована с
ориентацией, а для чётных необходимо брать нормаль с другим знаком.

\begin{note}
\label{orientation.note}
Выведенное соотношение достаточно важно, поскольку оно будет использовано при доказательстве
\emph{общей формулы Стокса}.
\end{note}

\subsubsection{Определение дифференциальной формы}

\begin{df}
Пусть $V$ векторное пространство над полем $K$. Отображение
$$f\cln \ub{V \st V}_k \ra K$$
называется \emph{полилинейным}, или, точнее, $k$-линейным, если оно линейно по каждому
своему аргументу при фиксированных остальных:
$f(\dots,\al x \bw+ \be y,\dots) \bw= \al f(\dots,x,\dots) \bw+ \be f(\dots,y,\dots)$, для любых
$x,y \bw\in V$ и $\al,\be \bw\in K$.
\end{df}

\begin{df}
Пусть задано пространство $\Ebb^m$, а в нём область $G$. \emph{Дифференциальной формой} порядка
$k$ на $G$ называется формальное выражение
$$
  \om(x,d x) \bw{:=} \sums{p_1 \bw< \dots \bw< p_k} F_{p_1\sco p_k}(x)\,\di xp1 \sw \di xpk,
$$
где $x \bw\in G$, а $F_{p_1\sco p_k} \bw\in \Cb^1(G)$ для любого набора индексов $\hc{p_i}$,
где $p_i \bw\in \hc{1\sco m}$.
При этом \emph{внешнее произведение дифференциалов} обладает следующими свойствами:

\pt{1}. Полилинейность.

\pt{2}. Кососимметричность: $d x \wg d y \bw= -d y \wg d x$.

\pt{3}. Ассоциативность: $(d x \wg d y)\wg d z \bw= d x \wg (d y \wg d z)$.
\end{df}

\begin{imp}
$d x \wg d x \bw= 0$.
\end{imp}

\begin{note}
Символом $p\up k$ будем обозначать запись $p_1 \bw< \dots \bw< p_k$,
а вместо $p_1\sco p_k$ будем писать $\pv k$.
\end{note}

Хотя мы не определили, где действуют значки, участвующие в записи дифференциальной формы, всё это очень
похоже на внешнюю алгебру. Поливекторы $\di xp1\sw \di xpk$ являются базисом этого пространства, а
дифференциальная форма их линейной комбинацией с коэффициентами из пространства $\Cb^1$.

Дифференциальные формы можно перемножать внешним образом: произведение $\om^k \wg \om^l$ форм порядка $k$ и $l$
есть дифференциальная форма порядка $k \bw+ l$.

\subsubsection{Замена переменных в дифференциальной форме}

Пусть $x \bw= \Phi(t)$, где $\Phi\cln \Ebb^m \ra \Ebb^m$ класса гладкости $\Cb^1$. Ввиду линейности, многие
свойства форм достаточно изучить на базисных векторах. Рассмотрим базисный элемент
$\om_0 \bw= \di xp1 \sw \di xpk$. Тогда, подставляя дифференциалы для координат $x$, получаем
$$
  \om_0 \bw= \hr{\suml{s_1 = 1}{m}\ppht{p_1}{s_1}\,\di ts1} \sw \hr{\suml{s_k = 1}{m}\ppht{p_k}{s_k}\,\di tsk}\bw=
  \sums{s\up k} \sgn \si \cdot \ppht{p_1}{s_1}\sd\ppht{p_k}{s_k} \cdot \,d t_{\si(s_1)}\sw \,d t_{\si(s_k)}.
$$
Поясним, откуда берётся такое выражение. Сначала мы расписали все слагаемые по линейности, затем вынесли в
начало все скалярные множители (в нашем случае все частные производные). Заметим, что можно не писать те
слагаемые, где есть повторяющиеся индексы $s_i$, поскольку при внешнем перемножении они дадут нуль. Наконец,
можно упорядочить все индексы по возрастанию, а от этого в качестве коэффициента при каждом слагаемом
появится знак чётности подстановки индексов.

Чтобы понять, что представляет собой это выражение, рассмотрим пример, в котором $k\bw=2$, а $m\bw=3$.
Обозначим для краткости $c_i^j \bw{:=} \ppht ji$, а $e^i \bw{:=} d t_i$. Тогда
$$
  \om_0 \bw= \di xp1 \wg \di xp2 \bw= \hr{c_1^{p_1}e^1 \bw+ c_2^{p_1}e^2 \bw+ c_3^{p_1}e^3}\wg
  \hr{c_1^{p_2}e^1 \bw+ c_2^{p_2}e^2 \bw+ c_3^{p_2}e^3}\bw=
$$

$$
  =
  c_1^{p_1}c_1^{p_2}e^1\wg e^1 \bw+
  c_2^{p_1}c_1^{p_2}e^2\wg e^1 \bw+
  c_3^{p_1}c_1^{p_2}e^3\wg e^1 \bw+
$$
$$
  +
  c_1^{p_1}c_2^{p_2}e^1\wg e^2 \bw+
  c_2^{p_1}c_2^{p_2}e^2\wg e^2 \bw+
  c_3^{p_1}c_2^{p_2}e^3\wg e^2 \bw+
$$
$$
  +
  c_1^{p_1}c_3^{p_2}e^1\wg e^3 \bw+
  c_2^{p_1}c_3^{p_2}e^2\wg e^3 \bw+
  c_3^{p_1}c_3^{p_2}e^3\wg e^3 \bw=
$$

$$
  \bw=
  \hr{c_1^{p_1}c_2^{p_2} \bw-
  c_2^{p_1}c_1^{p_2}}e^1\wg e^2 \bw+
  \hr{c_1^{p_1}c_3^{p_2} \bw-
  c_3^{p_1}c_1^{p_2}}e^1\wg e^3 \bw+
  \hr{c_2^{p_1}c_3^{p_2} \bw-
  c_3^{p_1}c_2^{p_2}}e^2\wg e^3 \bw=
$$

$$
  \bw=
  \mbmat{
  c_1^{p_1} & c_1^{p_2}\\
  c_2^{p_1} & c_2^{p_2}\\
  }e^1\wg e^2 \bw+
  \mbmat{
  c_1^{p_1} & c_1^{p_2}\\
  c_3^{p_1} & c_3^{p_2}\\
  }e^1\wg e^3 \bw+
  \mbmat{
  c_2^{p_1} & c_2^{p_2}\\
  c_3^{p_1} & c_3^{p_2}\\
  }e^2\wg e^3.
$$

В общем случае можно провести те же выкладки, и мы получим следующее:
$$
  \om_0 \bw= \sums{s\up k} \pf{\hr{\ph_{p_1}\sco \ph_{p_k}}}{\hr{t_{s_1}\sco t_{s_k}}}\,\di ts1\sw\di tsk.
$$
Теперь остаётся вспомнить, что мы преобразовывали только одно слагаемое. В общем случае получаем
$$
  \om \bw=\sums{p \up k} \sums{s\up k} F_{p_{[k]}}\circ\Phi
  \pf{\hr{\ph_{p_1}\sco \ph_{p_k}}}{\hr{t_{s_1}\sco t_{s_k}}} \,\di ts1\sw \di tsk.
$$

\begin{df}
Полученное выражение называется \emph{прообразом} дифференциальной формы при
отображении $\Phi$ и обозначается $\Phi^*\om$.
\end{df}

Рассмотрим ещё один важный случай, когда $\Phi\cln \Ebb^k \ra \Ebb^m$. Можно слово в слово повторить
предыдущие рассуждения, но теперь от суммы по наборам $s \up k$ останется только одно слагаемое,
поскольку теперь индексы могут меняться в пределах $1\sco k$.
Тогда получим следующее выражение:
$$
  \om \bw=\sums{p \up k} F_{\pv k}\circ\Phi \pf{\hr{\ph_{p_1}\sco \ph_{p_k}}}{\hr{t_1\sco t_k}} \,d t_1\sw d t_k.
$$

\begin{note}
По определению, \emph{дифференциальная форма порядка} 0 это функция $F\cln \Ebb^m \ra \R$.
\end{note}

\subsubsection{Внешнее дифференцирование дифференциальных форм}

\begin{df}
Пусть задана дифференциальная форма порядка $k$, причём $k \bw< m$. \emph{Внешним дифференциалом} формы
называется дифференциальная форма порядка $k \bw+ 1$ вида
$$d \om \bw{:=} \sums{p\up k} d F_{\pv k} \wg \di xp1 \sw \di xpk.$$
Отметим, что такая запись не является канонической для дифференциальной формы.
\end{df}

\begin{theorem}
Пусть $\om_1$ и $\om_2$ дифференциальные формы, причём $\om_1$ имеет порядок $k$. Тогда $d(\om_1\wg
\om_2) \bw= d \om_1 \wg \om_2 \bw+ (-1)^k \om_1\wg d \om_2$.
\end{theorem}
\begin{proof}
В силу линейности, утверждение теоремы можно проверять только на базисных векторах. Пусть
$\om_1 \bw= F\di xp1 \sw \di xpk$, $\om_2 \bw= G\di xq1 \sw \di xqr$, тогда $\om_1\wg \om_2 \bw=
FG\di xp1\sw \di xpk\wg \di xq1\sw \di xqr$.
Имеем
\begin{multline*}
d(\om_1\wg \om_2) \bw= d(F G)\wg\di xp1 \sw \di xpk\wg \di xq1\sw \di xqr =\\=
\br{\hr{F_{x_1}\,d x_1 \spl F_{x_m}\,d x_m} G \bw+ \hr{G_{x_1}\,d x_1 \spl G_{x_m}\,d x_m}F}
\wg\di xp1\sw \di xpk\wg \di xq1\sw \di xqr=\\=
\hr{F_{x_1}\,d x_1 \spl F_{x_m}\,d x_m} \wg\di xp1\sw \di xpk\wg G \di xq1\sw \di xqr +\\+
\hr{G_{x_1}\,d x_1 \spl G_{x_m}\,d x_m}F \wg\di xp1\sw \di xpk\wg \di xq1\sw \di xqr =\\=
d \om_1 \wg \om_2 \bw+ (-1)^k F \di xp1\sw \di xpk \wg \hr{G_{x_1}\,d x_1 \spl G_{x_m}\,d x_m} \wg \di xq1\sw \di xqr =\\=
d \om_1 \wg \om_2 \bw+ (-1)^k \om_1 \wg d \om_2.
\end{multline*}
В этих формулах появление $(-1)^k$ обусловлено протаскиванием $d x_i$ через $k$ сомножителей.
\end{proof}

\begin{theorem}
Пусть $F_{\pv k} \bw\in \Cb^2$. Тогда $d(d \om) \bw= 0$.
\end{theorem}
\begin{proof}
По определению дифференциала, имеем
$$d \om \bw= \sums{p\up k} \bbr{\sums{i}\pf{F_{\pv k}}{x_i}\,d x_i }\wg \di xp1 \sw \di xpk,$$
откуда, дифференцируя ещё раз, получаем
$$d(d \om) \bw= \sums{p\up k}\bbr{\sums{i}\sums{j}\pdf{F_{\pv k}}{x_i}{x_j}\,d x_i\wg d x_j}\wg \di xp1 \sw \di xpk.$$
Осталось заметить, что внутри скобок в выражении для второго дифференциала стоит тождественный нуль. Действительно,
в силу непрерывности частных производных имеет место равенство $\pdf{F_{\pv k}}{x_i}{x_j} \bw= \pdf{F_{\pv k}}{x_j}{x_i}$,
а тогда в силу кососимметричности внешнего произведения, каждое слагаемое в скобке встретится один раз со знаком <<$+$>>,
а второй раз со знаком <<$-$>>. В итоге всё сократится.
\end{proof}

\begin{theorem}
Пусть $\Phi\cln \Ebb^m(t) \ra \Ebb^m(x)$. Пусть $F \bw\in \Cb^2$. Тогда $d(\Phi^*\om) \bw= \Phi^*d \om$.
\end{theorem}
\begin{proof}
Докажем утверждение для базисных векторов. Имеем $\om \bw= F\di xp1 \sw \di xpk$, тогда, используя правило дифференцирования
произведения форм, получаем
$$
  d(\Phi^*\om) \bw= d\hr{F \circ \Phi \di\ph p1\sw \di\ph pk} \bw=
  d\hr{F\circ \Phi} \wg \di\ph p1\sw \di\ph pk \bw+
  F\circ\Phi\,d\hr{\di\ph p1\sw \di\ph pk}.
$$
Второе слагаемое равно нулю, поскольку это будет линейная комбинация нулевых вторых
дифференциалов. Что касается первого слагаемого, то это и есть $\Phi^*d\om$ по определению дифференциала и
$\Phi^*$.
\end{proof}

\subsection{Формула Стокса}

\subsubsection{Интеграл от дифференциальной формы}

\begin{df}
\emph{Интеграл от дифференциальной формы.} Пусть в $\Ebb^m$ задана гладкая поверхность $S$ размерности $k$.
Как мы знаем, это означает, что есть отображение $\Phi\cln A \ra \Ebb^m$, где $A \subs \Ebb^k$, а $S \bw=
\Phi(A)$. Пусть $\om$ дифференциальная форма порядка $k$ в $\Ebb^k$. Рассмотрим $\Phi^*\om \bw= F\,d t_1\sw
d t_k$. Положим $\ints{A}\Phi^*\om \bw{:=} \ints{A}F\,d t$. Пусть ориентации $S$ и $\pd S$ согласованы,
тогда
$$\ints{S} \om \bw{:=} \ints{A} \Phi^*\om.$$
\end{df}

\begin{theorem}[Корректность определения интеграла]
Пусть сделана замена \ $t \bw= t(u)$, где $t\cln \Ebb^k \ra \Ebb^k$, а $\pf tu \bw> 0$ для
сохранения ориентации. Пусть $x \bw= \Phi\circ t(u) \bw{=:} \Psi(u)$, где $u \bw\in B \subs \Ebb^k$.
Тогда $\ints{B} \Psi^* \om \bw= \ints{A} \Phi^*\om$.
\end{theorem}
\begin{proof}
Действительно, имеем
$$\Psi^*\om \bw= F\circ t(u) \,d t_1\sw d t_k \bw= F\circ t(u) \cdot \pf tu \,d u_1\sw d u_k.$$
Отсюда
$$\ints{B}\Psi^*\om \bw= \ints{B}F\circ t (u) \cdot \pf tu \,d u_1\sw d u_k \bw=
\ints{B}F\circ t(u) \cdot \pf tu \,d u \stackrel{!}{=} \ints{A}F\,d t\bw= \ints{A} \Phi^*\om.$$ Равенство,
отмеченное восклицательным знаком, следует из теоремы о замене переменных.
\end{proof}

\subsubsection{Общая формула Стокса}

\begin{theorem}[Формула Стокса]
Пусть $S$ кусочно гладкая ориентированная поверхность размерности~$k$ в $\Ebb^m$. Пусть $\om$ дифференциальная форма
порядка $k\bw-1$. Тогда
$$\ints{S}\,d \om \bw= \ints{\pd S} \om.$$
\end{theorem}
\begin{proof}
Пусть отображение $\Phi\cln \Ebb^k(t) \ra \Ebb^m(x)$ задаёт нашу поверхность $S$, тогда $S \bw= \Phi(A)$, где
$A \subs \Ebb^k$. В силу определения интеграла и свойств дифференциальных форм имеем $\ints{S}\,d \om \bw=
\ints{A}\Phi^*d \om \bw= \ints{A}\,d(\Phi^*\om)$. С другой стороны, $\ints{\pd S}\om \bw= \ints{\pd A}
\Phi^*\om$. Теорема будет доказана, если мы докажем, что
$$\ints{\pd A} \Phi^*\om \bw= \ints{A}\,d(\Phi^*\om).$$

Имеем
$$\Phi^*\om \bw= \sums{p\up k-1} F_{\pv{k-1}} \,\di tp1\sw \di tp{k-1}.$$
Распишем дифференциал от этой формы:
$$
d(\Phi^*\om) \bw= \sums{p\up k-1} \,d F_{\pv{k-1}} \wg \di tp1\sw \di tp{k-1}\bw= \sums{p\up k-1}
\hr{\sums{s}\pf{F_{\pv{k-1}}}{t_s}\,d t_s} \wg \di tp1\sw \di tp{k-1}.
$$
Зафиксируем набор $\pv{k-1}$ и посмотрим, как будет выглядеть каждое слагаемое внешней суммы.
Оно будет состоять из $k$ слагаемых $k$-ого порядка, но поскольку числа $p_i$ могут меняться от 1 до $k$,
а в каждом наборе нет одинаковых, то во всех внутренних $k$ слагаемых, кроме одного, будут множители с
повторяющимися индексами. Следовательно, они обнулятся,
и останется слагаемое с таким $s$, которого не было в наборе $\pv{k-1}$.

В силу линейности можно, не ограничивая, однако, общности, доказывать утверждение для $s \bw= k$, е для
формы, имеющей вид $\pf{F}{t_k}\,d t_k \wg dt_1 \sw dt_{k-1}$. Пропихивая первый множитель в хвост
произведения, получаем
$$
  \ints{A}\,d(\Phi^*\om) \bw= (-1)^{k-1}\ints{A}\pf{F}{t_k} \,d t_1 \sw d t_k \bw=
  (-1)^{k-1}\ints{A} \pf{F}{t_k} \,d t.
$$

Теперь мы временно забудем про этот интеграл и будем разбираться с другой половиной формулы. Имеем
$$\ints{\pd A}\Phi^*\om \bw= \ints{\pd A}F\,d t_1\sw d t_{k-1}.$$

Спроецируем $A$ на гиперплоскость $\pi \bw{:=} \hc{t_k \bw= 0}$, получим множество $D$. Будем считать,
что множество $A$ выпукло,
значит, у него есть <<верхняя>> и <<нижняя>> крышки $U$ и $L$. Тогда
$$A \bw= \hc{t\cln (t_1\sco t_{k-1}) \bw\in D,\; t_k \bw\in \bs{L(t_1\sco t_{k-1}), U(t_1\sco t_{k-1})}}.$$
Пусть $\Pi_L$ и $\Pi_U$ образы нижней и верхней крышек соответственно, а $\Pi_S$ боковая часть. Будем
предполагать, что проекция $\Pi_S$ на $\pi$ есть $\pd D$ размерности $k\bw-2$, а $\pd A \bw= \Pi_L \bw\cup
\Pi_U \bw\cup \Pi_S$. Тогда $\pd D$ как-то параметризуется: $(t_1\sco t_{k-1}) = f(\tau_1\sco \tau_{k-2})$.
Рассмотрим $d t_1\sw d t_{k-1}$. Имеем $\ints{\Pi_S} \bw= 0$, поскольку $d \tau_j$ уже только $k\bw-2$
штук, и когда мы подставим выражения для $t_i$ через $\tau_j$ в дифференциалы, обязательно появятся
повторяющиеся индексы в каждом слагаемом. Теперь посмотрим на верхнюю крышку: ориентация на ней правильная,
поэтому
$$\ints{\Pi_U}F\hr{t_1\sco t_{k-1}, t_k}\,d t_1\sw d t_{k-1} \bw=
(-1)^{k-1}\ints{D}F\br{t_1\sco t_{k-1}, U(t_1\sco t_{k-1})}\,d t_1\dots d t_{k-1}.$$ На нижней крышке
ориентация неправильная, поэтому
$$\ints{\Pi_L}F\hr{t_1\sco t_{k-1}, t_k}\,d t_1\sw d t_{k-1} \bw=
(-1)^k\ints{D}F\br{t_1\sco t_{k-1}, L(t_1\sco t_{k-1})}\,d t_1\dots d t_{k-1}.$$ Отсюда
\begin{multline*}
\ints{\pd A}\Phi^*\om \bw= \ints{\Pi_U} \bw+ \ints{\Pi_L} \bw=
(-1)^{k-1}\ints{D}F\br{t_1\sco t_{k-1}, U(t_1\sco t_{k-1})}\,d t_1\dots d t_{k-1} -\\
-(-1)^{k-1}\ints{D}F\br{t_1\sco t_{k-1}, L(t_1\sco t_{k-1})}\,d t_1\dots d t_{k-1}\stackrel{!}{=}\\
\stackrel{!}{=}(-1)^{k-1}\ints{D}\intl{L}{U} \pf{F}{t_k} \,d t_k \,d t_1\dots d t_{k-1} \stackrel{!!}{=}
(-1)^{k-1}\ints{A} \pf{F}{t_k} \,d t.
\end{multline*}
В этих выкладках <<!>> следует из ФНЛ, а <<!!>> из теоремы о сведении кратного интеграла к повторному.
\end{proof}

\begin{problem}
Выведите из общей формулы Стокса трёхмерную формулу.
\end{problem}
\begin{hint}
Возьмите поле $(P, Q, R)$,
рассмотрите дифференциальную форму $\om \bw= P\,d x \bw+ Q\,d y \bw+ R\,d z$ и продифференцируйте,
а потом напишите общую формулу Стокса для этой формы.
\end{hint}

\section{Кратные ряды}

Хотя в названии главы этого не указано, мы будем для упрощения жизни изучать только двойные ряды.

\subsection{Виды сходимости двойных рядов}

\subsubsection{Понятие двойного ряда. Методы суммирования}

\begin{df}
\emph{Двойной ряд} формальная запись $\suml{i,j=1}{\infty}a_{ij}$, где $a_{ij} \bw\in K$, а $K$
основное поле $\R$ или $\Cbb$.
\end{df}
\begin{df}
Будем говорить, что задан \emph{метод $\Mc$ суммирования} двойного ряда $\sum a_{ij}$, если задано
правило выбора последовательности $\hc{\Sc_k}$, удовлетворяющей следующему свойству: $\Sc_k$
произвольные конечные подмножества членов ряда, такие, что любой член ряда $a_{ij}$
содержится во всех $\Sc_k$, начиная с некоторого.
\end{df}

\begin{note}
Иногда в определении метода суммирования требуют, чтобы $\Sc_k \subs \Sc_{k+1}$.
\end{note}

\begin{note}
Задание одного только метода суммирования ещё не позволяет сказать, сходится данный ряд при
суммировании этим методом, или нет. Для этого необходимо ввести \emph{сходимость ряда по
методу суммирования}, что сейчас и будет сделано.
\end{note}

\begin{df}
Двойная последовательность $\hc{S_{kl}}$ \emph{сходится} к числу $S$, если
$\fa \ep \bw> 0 \exi N\cln \fa k, l \bw> N$ имеем $|S_{kl} \bw- S| \bw< \ep$.
\end{df}

Введём несколько наиболее употребительных методов суммирования двойных рядов и определим
сходимость по этим методам.

\begin{df}
Рассмотрим $S_{kl} \bw{:=} \suml{i=1}{k}\suml{j=1}{l} a_{ij}$, получим двойную последовательность $\hc{S_{kl}}$.
Ряд $\sum a_{ij}$ сходится \emph{по прямоугольникам} к числу $S$, если $S_{kl} \ra S$ как двойная последовательность.
Такая сходимость ещё называется сходимостью по Принсгейму, поэтому мы будем обозначать эту сходимость символом $P$.
\end{df}

Разновидностью $P$-сходимости является сходимость \emph{по прямоугольникам с ограниченным
отношением}. Её определение таково: пусть $0 \bw< c_1 \bw< c_2 \bw< \infty$, тогда ряд сходится к
$S$, если $\fa \ep \bw> 0 \exi N\cln \fa k, l\bw> N\cln 0 \bw< c_1 \bw\le \frac{l}{k} \bw\le c_2$
имеем $|S_{kl} \bw- S| \bw< \ep$. Графически это означает, что мы запрещаем рассматриваемым
прямоугольникам быть слишком плоскими, если рассматривать элементы ряда как точки на плоскости.
Такую сходимость можно обозначать символом $(P, c_1,c_2)$. Ещё одна разновидность $P$-сходимости
сходимость \emph{по кубам}. Её можно получить из предыдущей, если положить $c_1 \bw= c_2 \bw= 1$.

Принципиально другим видом сходимости является $T$-\emph{сходимость}, где строится
последовательность сумм $S_n \bw{:=} \sums{i+j \bw\le n} a_{ij}$, а затем исследуется на сходимость
как обычная последовательность.

Следующие два вида сходимости похожи на $T$-сходимость и различаются только выбором $S_n$. Для
\emph{сферической $S$-сходимости} имеем $S_n \bw{:=} \sums{i^2+j^2 \bw\le n^2} a_{ij}$, а для
\emph{$H$-сходимости по гиперболическим крестам} $S_n \bw{:=} \sums{|ij| \bw\le n} a_{ij}$.
Название последней объясняется тем, что можно рассматривать и отрицательные индексы, тогда
множества $\Sc_n$ будут образовывать крестовидные фигуры на плоскости.

\begin{note}
С $T$-сходимостью мы уже сталкивались, когда доказывали теорему Мертенса: речь шла о двойном ряде,
в котором $a_{ij} \bw{:=} u_iv_j$.
\end{note}

\begin{theorem}[Линейность сходимости]
Пусть ряды $\sum a_{ij}$ и $\sum b_{ij}$ сходятся при одинаковом методе суммирования. Тогда при том же методе
сходится ряд $\sum (\la a_{ij} \bw+ \mu b_{ij})$ для любых $\la, \mu \bw\in \R$.
\end{theorem}

\begin{ex}
Если для обыкновенных рядов сходимость влечёт стремление к нулю членов ряда, то для двойных рядов
из сходимости не вытекает даже их ограниченность. Рассмотрим ряд $a_{1j} \bw{:=} j$, $a_{2j}
\bw{:=} \bw-j$, остальные положим равными 0, и просуммируем его по методу прямоугольников. Очевидно,
при $k \bw\ge 2$ имеем $S_{kl} \bw= 0$, поэтому $\sum a_{ij} \bw= 0$.
\end{ex}

\begin{note}
Сходимость ряда по одному методу почти никогда не влечёт сходимость по другому методу, поскольку
можно привести соответствующие примеры. Исключение составляет серия $P$-методов: из $P$-сходимости,
очевидно, следует $(P, c_1,c_2)$-сходимость и тем более $(P,1,1)$-сходимость. Как несложно увидеть,
ряд в примере, приведённом выше, не является ни $T$-сходящимся, ни $S$-сходящимся.
\end{note}

\subsubsection{Повторные ряды}

Наряду с двойными рядами, имеет смысл рассматривать \emph{повторные ряды}. Пусть дан ряд $\sum
a_{ij}$, тогда можно рассмотреть два повторных ряда $\sumiui\suml{j=1}{\infty} a_{ij}$ и
$\suml{j=1}{\infty}\sumiui a_{ij}$. Несложно видеть, что фактически речь идёт о двойных пределах
$\liml{k}\liml{l} S_{kl}$ и $\liml{l}\liml{k} S_{kl}$. Аналогично тому, как из существования
повторных пределов не следует существования общего предела функции многих переменных, из
существования повторных пределов не следует суммируемость ряда. Кроме того, различные повторные
пределы могут быть различными по значению, даже если они существуют.

\begin{ex}
Рассмотрим ряд $a_{ij} \bw{:=} \frac{1}{i+1}\hr{\frac{i}{i+1}}^j \bw-
\frac{1}{i+2}\hr{\frac{i+1}{i+2}}^j$. В качестве упражнения предлагается показать, что суммирование
в одном порядке даёт $\frac12$, а в другом $\bw-\frac12$.
\end{ex}

\begin{theorem}
Пусть ряд $\sum a_{ij}$ является $P$-суммируемым. Пусть $\fa i$ сходится $\sums{j} a_{ij}$. Тогда
$\sums{i} \sums{j} a_{ij} \bw= \sum a_{ij}$, причём утверждается и существование этого повторного
ряда, и равенство.
\end{theorem}
\begin{proof}
Имеем $\fa \ep \bw> 0 \exi N\cln \fa k, l \bw> N$ имеем $|S_{kl} \bw- S| \bw<\ep$,
где $S \bw{:=} \sum a_{ij}$. Кроме того, поскольку конечная сумма пределов равна пределу
суммы, имеем $\liml{l} \suml{i=1}{k}S_{il} \bw= \suml{i=1}{k}\suml{j=1}{\infty} a_{ij}$.
Перейдём в неравенстве $|S_{kl} \bw- S| \bw< \ep$ к пределу при $l \ra \infty$, получим
$$\bbm{\suml{i=1}{k}\suml{j=1}{\infty} a_{ij} \bw- S} \bw\le \ep,$$ откуда следует требуемое равенство.
\end{proof}

\begin{theorem}[Маркова]
Пусть сходится повторный ряд $\sums{i}\sums{j} a_{ij}$. Пусть $\fa j$ сходится $\sums{i} a_{ij}$.
Для $l \bw\in \N_0$ положим $R_{il} \bw{:=} \suml{j=l+1}{\infty} a_{ij}$.
Тогда:

\pt{1}. $\sums{i} R_{il}$ сходится $\fa l \bw\in \N_0$.

\pt{2}. Положим $R_l \bw{:=} \sums{i} R_{il}$. Тогда сходимость $\sums{j}\sums{i} a_{ij}$
равносильна существованию предела $\liml{l} R_l$.

\pt{3}. Пусть $R \bw= \liml{l} R_l$, тогда $\sums{i}\sums{j}a_{ij} \bw= \sums{j}\sums{i} a_{ij}$
тогда и только тогда, когда $R \bw= 0$.
\end{theorem}
\begin{proof}
Сам факт существования величин $R_{il}$ следует из условия теоремы. Имеем
$$R_{il} \bw= \sums{j} a_{ij} \bw- (a_{i1}\spl a_{il}).$$
Эти равенства можно просуммировать по $i$ от 1 до $\infty$, поскольку в правой части всё сходится по условию:
$$\sums{i} R_{il} \bw= \sums{i} \sums{j} a_{ij} \bw- \hr{\sums{i}a_{i1}\spl \sums{i}a_{il}}.$$
Таким образом, свойство \pt{1} доказано. Предыдущее равенство в точности означает, что
$R_l \bw= \sums{i}\sums{j}a_{ij} \bw- \suml{j=1}{l}\sums{i} a_{ij}$.
Рассмотрим $R_0 \bw= \sums{i}\sums{j} a_{ij}$, тогда получаем
$R_0 \bw- R_l \bw= \suml{j=1}{l}\sums{i}a_{ij}$, откуда следует
свойство \pt{2}. Но, как несложно видеть, из этого равенства следует и свойство \pt{3}, что
проверяется возможностью устремить $l \ra \infty$.
\end{proof}

\begin{df}
Будем говорить, что $(k, l) \bw> n$, если $k \bw> n$ и $l \bw> n$. Будем также говорить,
что $(x,y) \neq 0$, если $x \neq 0, y \neq 0$.
\end{df}

\begin{theorem}[Критерий $P$-сходимости Коши]
Для сходимости $\sum a_{ij}$ необходимо и достаточно условия Коши:
$\fa \ep \bw> 0 \exi N\cln \fa (k, l), (p,q) \bw> N$ имеем $|S_{kl} \bw- S_{pq}| \bw< \ep$.
\end{theorem}
\begin{proof}
\emph{Необходимость}. Пусть ряд сходится, тогда $\fa \ep \bw> 0 \exi N\cln \fa (k,l) \bw> N, \fa (p,q) \bw> N$
имеем $|S_{kl} \bw- S| \bw< \ep$ и $|S_{pq} \bw- S| \bw< \ep$. Тогда
$|S_{kl} \bw- S_{pq}| \bw= |S_{kl} \bw- S \bw+ S \bw- S_{pq}| \bw< \ep \bw+ \ep$.

\emph{Достаточность}. Заметим, что $\hc{S_{ii}}$ является обычной числовой последовательностью,
для которой критерий Коши доказан давным давно. Из условия Коши для двойных последовательностей, в частности,
следует, что $\fa \ep \bw> 0 \exi N\cln \fa (k,k), (p,p) \bw> N$ имеем $|S_{kk} \bw- S_{pp}| \bw< \ep$, поэтому
для $\hc{S_{ii}}$ выполнен обычный критерий Коши. Значит, она сходится к некоторому числу $S$.
Потребуем, чтобы $\fa \ep \bw> 0 \exi N\cln \fa (k,l), (p,q) \bw> N$ было выполнено $|S_{kl} \bw- S_{pq}| \bw< \ep$ и
$\fa i \bw> N$ было выполнено $|S_{ii} \bw- S| \bw< \ep$. Тогда $|S \bw- S_{kl}| \bw=
|S \bw- S_{kk} \bw+ S_{kk} \bw- S_{kl}| \bw\le |S \bw- S_{kk}| \bw+ |S_{kk} \bw- S_{kl}| \bw\le \ep \bw+ \ep$.
Отсюда следует сходимость.
\end{proof}

\subsubsection{Абсолютно сходящиеся двойные ряды}
\begin{df}
Пусть $\Sigma$ множество всех частных сумм ряда $\sum |a_{ij}|$. Если $\sup \Sig \bw< \infty$,
то двойной ряд $\sum a_{ij}$ называется \emph{абсолютно сходящимся}.
\end{df}

Покажем, что в этом случае любой метод суммирования приведёт к тому же $S$. Положим
$b_{ij} \bw{:=} |a_{ij}|$ и $\sum b_{ij}$ сходится
абсолютно. Если $S \bw= \sup \Sigma$, то $\fa \ep \bw> 0 \exi \si \bw\in \Sig$, для
которой $S \bw- \si \bw< \ep$. Тогда, по определению метода суммирования, в последовательности
$\Sc_n$ рано или поздно окажутся все слагаемые $\si$, значит, подавно, будем иметь $S \bw- \Sc_n \bw< \ep$.

\begin{ex}
Исследуйте на сходимость ряды: $\sums{i,j\bw\ge 2} i^j,
\quad \sum \frac1{(i + j)^\al}, \quad \sum \frac1{(i^2 + j^2)^\al}$.
\end{ex}

\subsection{Двойные степенные ряды}

\subsubsection{Понятие двойного степенного ряда}

\begin{df}
\emph{Двойным степенным рядом} называется выражение вида
$$\sums{\substack{i \bw= 0\\j\bw=0}} a_{ij}(x \bw- x_0)^i(y \bw- y_0)^j,$$
где $a_{ij} \bw\in \R$ и $x, y, x_0, y_0 \bw\in \R$. Точка $(x_0, y_0)$ называется \emph{центром} степенного ряда.
\end{df}

Очевидно, что линейным сдвигом всё сводится к случаю, когда центр ряда находится в точке $(0,0)$. По этой причине
будем рассматривать только такие ряды.

\begin{theorem}
Пусть члены ряда ограничены в точке $(x_*,y_*)$. Тогда ряд сходится абсолютно при $|x| \bw< |x_*|$
и $|y| \bw< |y_*|$.
\end{theorem}
\begin{proof}
В силу ограниченности, найдётся $M$, для которого $\hm{a_{ij}x_*^iy_*^j} \bw\le M$. Тогда
$$\hm{a_{ij} x^iy^j} \bw= \hm{a_{ij}x_*^iy_*^j}\cdot
\hm{\frac{x}{x_*}}^i\cdot\hm{\frac{y}{y_*}}^j \bw\le M\cdot\hm{\frac{x}{x_*}}^i\cdot\hm{\frac{y}{y_*}}^j.$$
Всё сведено к доказательству сходимости двойной геометрической прогрессии $\sum \al^i\be^j$, где
$\al, \be \bw< 1$, но она очевидна, поскольку произвольная её частная сумма ограничена числом
$\sums{i} \al^i\cdot \sums{j} \be^j$.
\end{proof}

\begin{note}
В этой теореме используется такая мелочь, как теорема сравнения для двойных рядов, но ввиду
очевидности мы её здесь не приводим.
\end{note}

Покажем, что бывают двойные ряды, которые сходится по кубам только в двух точках. Рассмотрим ряд
$\sum a_{ij}x^iy^j$, где $a_{i0} \bw= a_{0i} \bw= i!$ при $i \bw\ge 2$, а $a_{i1} \bw= a_{1i} \bw=
\bw-i!$ при $i \bw\ge 2$. Остальные члены ряда положим равными нулю. Имеем
$$\rbmat{
       &         & \phantom{-}\ely2\phantom{x} &  \phantom{-}\ely3\phantom{x} &  \phantom{-}\ely4\phantom{x} & \dots\\
       &         & -\ely2x & -\ely3x & -\ely4x & \dots\\
\elx2 & -\elx2y &         &         &         &      \\
\elx3 & -\elx3y &         &         &         &      \\
\elx4 & -\elx4y &         &         &         &      \\
\dots  & \dots  &         &         &         &      \\
}
$$

Введём $X_n \bw{:=} \elx2 \bw+ \elx3\spl \elx n$ и $Y_n \bw{:=} \ely2 \bw+ \ely3\spl \ely n$. Пусть
$S_n$ частичные суммы нашего ряда при суммировании по кубам. Легко видеть, что $S_n \bw= X_n \bw-
yX_n \bw+ Y_n \bw- xY_n \bw= X_n(1\bw-y) \bw+ Y_n(1\bw-x)$. Отсюда ясно, что ряд сходится в $(0,0)$ и $(1,1)$.
Покажем, что других точек сходимости у него нет. Очевидно, что случаи $x\bw>y$ и $y\bw>x$
симметричны, поэтому достаточно рассмотреть один из них. При $x \bw= y \neq 1$ имеем $X_n \bw= Y_n$,
откуда $S_n \bw= 2X_n(1\bw-x) \ra \infty$ при $n \ra \infty$. Если $x \bw> 1$, то доказывать нечего. Если
$x\bw= 0$, то доказывать опять нечего, поскольку факториалы забивают показательную функцию. Если $x \bw=
1$, то всё опять очевидно. Самый сложный случай, когда $0 \bw< y \bw< x \bw< 1$, предоставляется читателю в
качестве упражнения. В качестве указания приведём полезное представление для $S_n$:
$$
  S_n \bw= 2!\br{x^2(1\bw-y) \bw+ y^2(1\bw-x)} \bw+ 3!\br{x^3(1\bw-y) \bw+ y^3(1\bw-x)}
  \spl n!\br{x^n(1\bw-y) \bw+ y^n(1\bw-x)}.
$$

\subsubsection{Абсолютная сходимость степенных рядов}

\vskip-10pt
\rightpicture{pictures.50}
\hangindent=-40mm
\hangafter=-8
Исследуем ряд $\sum a_{ij}x^iy^j$ на абсолютную сходимость в точках луча
$L_\al \bw= \hc{y \bw= \al x}$, где $\al \bw\in (0,+\infty)$.
По теореме о сходимости из предыдущего раздела $\exi! \; M_\al \bw\in L_\al$, для которой ряд сходится
во всех точках луча, которые расположены ближе, чем $M_\al$, к началу координат, и расходится во
всех точках, расположенных дальше. При этом мы не исключаем возможности $M_\al \bw= +\infty$. Покажем,
что отображение $\al \mapsto M_\al$ непрерывно. Действительно, рассмотрим другой луч $y \bw= \be x$.
По той же теореме точка $M_\be$ должна находиться на отрезке $[X_\be, Y_\be]$, в противном случае
точка $M_\al$ не была бы границей сходимости: она лежала бы либо внутри области сходимости, либо
внутри области расходимости. Но ясно, что при небольших отклонениях угла $\be$ от $\al$ точка
$M_\be$ далеко не уедет.

Таким образом, на плоскости можно прочертить непрерывную кривую граничных точек сходимости ряда.
Она, очевидно, не имеет самопересечений и, как несложно видеть, является графиком некоторой
монотонно убывающей функции. Вблизи осей координат её поведение может быть различным:
она может <<втыкаться>> в ось, а может асимптотически к ней приближаться.

\end{document}
