\documentclass[a4paper]{article}
\usepackage[utf,simple]{dmvn}

\title{Программа экзамена по математическому анализу}
\author{Лектор С.\,А.\,Теляковский}
\date{I--IV семестры, 2002--2004 г.}

\begin{document}
\maketitle

\section*{I семестр}
\begin{nums}{-3}
\item Непрерывность множества действительных чисел в терминах теорем
      о точных гранях, вложенных отрезках и дедекиндовых сечениях. Равносильность
      этих теорем.
\item Счётность множества рациональных чисел и несчётность множества действительных чисел.
\item Предел последовательности. Свойства пределов, связанные с неравенствами
      и арифметическими действиями. Определение числового ряда и его суммы.
\item Подпоследовательности, теорема Больцано Вейерштрасса. Верхний и нижний пределы последовательности,
их свойства.
\item Предел монотонной последовательности. Число $e$. Критерий Коши сходимости последовательности.
\item Определения предела функции по Коши и по Гейне, их эквивалентность. Свойства пределов функции.
\item Критерий Коши существования предела функции. Односторонние пределы. Пределы монотонных функций.
 Классификация точек разрыва.
\item Непрерывность элементарных функций.
\item Предел и непрерывность сложной функции. Непрерывность обратной функции.
\item Свойства функций, непрерывных на отрезке: ограниченность, достижимость точной верхней и точной
нижней граней значений.
\item Теорема Коши о промежуточных значениях для функций, непрерывных на отрезке и на интервале.
\item Равномерная непрерывность функции, непрерывной на отрезке. Модуль непрерывности, критерий равномерной
непрерывности функции.
\item Производная и дифференциал функции. Касательная к графику функции.
\item Производная обратной функции, производная сложной функции.
\item Правила вычисления производных. Производные элементарных функций.
\item Производные и дифференциалы высших порядков. Формула Лейбница для производной произведения.
Вопрос об инвариантности формы дифференциалов.
\item Возрастание и убывание функции в точке. Теорема Ферма. Теорема Дарбу о промежуточных
значениях.
\item Теоремы Ролля, Лагранжа и Коши о среднем.
\item Следствия из формулы конечных приращений Лагранжа. Возрастание и убывание функции на отрезке.
\item Раскрытие неопределённостей. Правила Лопиталя.
\item Формула Тейлора с остаточным членом в форме Лагранжа и в форме Пеано.
\item Поведение остаточного члена формулы Тейлора функций $e^x$, $\sin x$, $\cos x$, $\ln(1+x)$. Бином Ньютона.
\item Достаточные условия и необходимые условия локального экстремума в терминах старших производных.
\item Выпуклость и вогнутость функции в точке. Точки перегиба.
\item Необходимое и достаточное условие выпуклости функции на промежутке.
\item Свойства производных функций, выпуклых на промежутке.
\item Неравенства Йенсена, Гёльдера, Минковского и Чебышева.
\item Непрерывность и дифференцируемость векторнозначных функций. Свойства производных.
\item Длина дуги кривой. Гладкие кривые.
\end{nums}

\pagebreak

\section*{II семестр}

\begin{nums}{-3}
\item Первообразная и ее свойства. Неопределенный интеграл, интегрирование по частям и замена переменной.
Интегрирование рациональных дробей.
\item Определение интеграла Римана. Ограниченность интегрируемой функции. Верхний и нижний интегралы Дарбу.
\item Критерии интегрируемости функции в терминах сумм Дарбу.
\item Интегрируемость непрерывных и кусочно-непрерывных функций. Площадь криволинейной трапеции.
Интегрируемость монотонных функций.
\item Аддитивность интеграла относительно промежутка интегрирования. Интегрируемость
суммы, произведения и частного интегрируемых функций. Интегрирование неравенств. Первая теорема о среднем.
\item Приближение интегрируемых функций ступенчатыми и непрерывными функциями.
\item Интегрируемость в степени $p>0$ модуля интегрируемой функции.
\item Свойства интеграла с переменным верхним пределом: выполнение условия Липшица, дифференцируемость.
Формула Ньютона Лейбница.
\item Интегрирование по частям и замена переменной в определенном интеграле. Формула Тейлора с остаточным
членом в интегральной форме.
\item Вторая теорема о среднем.
\item Несобственные интегралы. Критерий Коши, признаки сравнения. Признаки Дирихле и Абеля
сходимости несобственных интегралов. Формула Ньютона Лейбница и интегрирование по частям для несобственных интегралов.
\item Неравенства Йенсена, Гёльдера, Минковского и Чебышева для интегралов.
\item Функции ограниченной вариации и их свойства.
\item Интеграл Римана Стилтьеса, критерий Коши интегрируемости. Интегрируемость непрерывной функции
по функции ограниченной вариации.
\item Интегрирование по частям для интеграла Римана Стилтьеса. Связь интеграла Римана Стилтьеса с интегралом Римана.
\item Открытые и замкнутые множества в многомерном евклидовом пространстве.
\item Предел функции многих переменных, критерий Коши. Повторные пределы.
\item Непрерывность функции многих переменных. Непрерывность сложной функции. Свойства функций, не-\break прерывных на компакте.
\item Частные производные и дифференцируемость функций многих переменных. Дифференцируемость сложной функции.
Инвариантность формы первого дифференциала. Градиент функции и его инвариантность относительно системы координат.
\item Касательная плоскость к графику функции многих переменных.
\item Частные производные и дифференциалы высших порядков функций многих переменных. Достаточные условия
равенства смешанных производных.
\item Формула Тейлора для функций многих переменных с остаточным членом в форме Лагранжа и в форме Пеано.
\item Выпуклые функции многих переменных.
\item Теорема о неявной функции.
\item Теорема о системе неявных функций.
\item Безусловный локальный экстремум функций многих переменных.
\item Локальный относительный экстремум функций многих переменных.
\item Метод неопределенных множителей Лагранжа для нахождения локального относительного экстремума функций
      многих переменных.
\end{nums}

\pagebreak

\subsection*{III семестр}

\begin{nums}{-3}
\item Числовые ряды, критерий Коши. Ряды с неотрицательными членами, критерий сходимости. Признаки
сравнения. Признаки Даламбера и Коши сходимости рядов, сравнение этих признаков.
\item Ряды с монотонными членами. Интегральный признак сходимости. Постоянная Эйлера. Теорема Коши.
Необходимое условие сходимости рядов с монотонными членами. Признак Лейбница сходимости знакочередующихся рядов.
\item Признаки Раабе, Гаусса и Куммера сходимости рядов. Неравенства Гёльдера и Минковского для рядов.
\item Преобразование Абеля. Признаки Дирихле и Абеля сходимости рядов. Бесконечные произведения. Связь
сходимости бесконечных произведений и рядов.
\item Перестановка членов абсолютно сходящихся рядов. Теорема Римана о перестановке членов условно сходящихся рядов.
\item Почленное умножение абсолютно сходящихся рядов. Теорема Мертенса об умножении рядов.
\item Равномерная сходимость функциональных последовательностей и рядов. Критерий Коши, признаки сравнения.
Признаки Дирихле, Абеля и Дини равномерной сходимости рядов.
\item Теоремы о равномерно сходящихся последовательностях и рядах: предельный переход, непрерывность,
почленное интегрирование.
\item Теорема о дифференцировании равномерно сходящихся последовательностей и рядов. Пример непрерывной
нигде не дифференцируемой функции.
\item Первая теорема Абеля о степенных рядах. Радиус сходимости степенного  ряда, формула Коши Адамара.
Теорема единственности для степенных рядов. Представление функций рядами Тейлора.
\item Функции комплексной переменной $e^z$, $\sin z$, $\cos z$, $\sh z$, $\ch z$. Формула Эйлера.
Почленное дифференцирование степенных рядов по комплексному аргументу.
\item Вторая теорема Абеля о степенных рядах.  Представление функций $\ln(1 + х)$ и $\arctg x$ рядами Тейлора.
\item Представление степенной функции $(1 + х)^m$ рядом Тейлора, $|х| \le 1$.
\item Суммирование рядов методами средних арифметических и Абеля Пуассона.
\item Свойства собственных интегралов, зависящих от параметра (непрерывность, переход к пределу,
дифференцирование, интегрирование).
\item Равномерная сходимость несобственных интегралов, зависящих от параметра. Критерий Коши,
признаки равномерной сходимости.   Свойства равномерно сходящихся несобственных интегралов: переход к
пределу, непрерывность.
\item Свойства равномерно сходящихся несобственных интегралов: интегрирование, дифференцирование.
\item $\Ga$ функция. Формула Стирлинга.
\item $\Be$ функция, выражение через $\Ga$ функцию,
\item Разложение функции $\frac{1}{\sin x}$ по полюсам. Представление функции $\sin x$ в виде бесконечного
произведения многочленов. Формула дополнения для $\Ga$ функции.
\item Наилучшее приближение элементов полиномами в линейном нормированном пространстве. Ортонормированные
системы в пространствах со скалярным произведением. Минимальное свойство коэффициентов Фурье.
Неравенство Бесселя, равенство Парсеваля. Полнота и замкнутость систем.
\item Ряды Фурье по тригонометрической системе. Лемма Римана. Принцип локализации Римана. Почленное
дифференцирование и интегрирование рядов Фурье. Ряды Фурье в комплексной форме.
\item Сходимость тригонометрического ряда Фурье в точке. Пример непрерывной функции с рядом
Фурье, расходящимся в точке.
\item Обобщённая лемма Римана. Принцип локализации для равномерной сходимости рядов Фурье. Явление Гиббса.
\item Наилучшее приближение функций тригонометрическими полиномами. Теорема Лебега, константы Лебега.
Теорема Джексона. Условие Дини Липшица.
\item Суммирование тригонометрических рядов Фурье методом средних арифметических. Теоремы Вейерштрасса
о полноте тригонометрической системы и системы алгебраических многочленов.
\item Преобразование Фурье.
\end{nums}

\pagebreak

\section*{IV семестр}

\begin{nums}{-3}
\item Элементарные фигуры, внутренняя и внешняя меры множества. Мера Жордана.
\item Свойства измеримых множеств. Измеримость множества и мера его границы. Мера объединения множеств, аддитивность меры.
\item Кратный интеграл Римана. Вопрос об ограниченности интегрируемых функций.
\item Критерии интегрируемости в терминах сумм Дарбу.
\item Свойства кратного интеграла Римана.
\item Кратные и повторные интегралы.
\item Геометрический смысл модуля якобиана плоского отображения.
\item Замена переменных в кратном интеграле.
\item Достаточные условия сходимости несобственных кратных интегралов.
\item Абсолютная сходимость несобственных кратных интегралов.
\item Криволинейные интегралы первого и второго рода, их свойства.
\item Формула Грина.
\item Геометрический смысл знака якобиана плоского отображения.
\item Потенциальные векторные поля.
\item Площадь гладкой поверхности в $\Ebb^3$. Поверхностные интегралы первого рода.
\item Ориентируемые гладкие и кусочно гладкие поверхности в $\Ebb^3$. Поверхностные
      интегралы второго рода.
\item Формула Гаусса~--- Остроградского. Независимость дивергенции от выбора системы координат.
      Соленоидальные векторные поля.
\item Формула Стокса в $\Ebb^3$. Независимость ротора от выбора системы координат.
\item Многомерные ориентированные гладкие поверхности.
\item Дифференциальные формы. Внешнее дифференцирование. Интеграл от дифференциальной формы.
\item Общая формула Стокса.
\item Сходимость и абсолютная сходимость двойных рядов. Двойные степенные ряды.
\end{nums}

\medskip\dmvntrail
\end{document}
