\documentclass{article}
\usepackage[utf]{dmvn}
\newcommand{\mbb}{\mathbb}
\newcommand{\mrm}{\mathrm}
\newcommand{\mfr}{\mathfrak}
\newcommand{\sm}{\setminus}
\newcommand{\hookra}{\hookrightarrow}
\DeclareMathOperator{\Lie}{Lie}
\begin{document}
\title{}
\author{}
\date{\today}
\maketitle
\section{Обозримые подгруппы алгебраических групп}

\subsection{Постановка проблемы}

Основное поле будем обозначать через $k$ и считать алгебраически
замкнутым. Кроме того, пусть $\Char k=0$.

Пусть $G$~--- некоторая алгебраическая группа. Она действует на алгебре $k[G]$ регулярных функций правыми сдвигами
аргумента: \eqn{\label{act} (r_{g}f)(g'):=f(g'g),} все $r_g$~--- автоморфизмы алгебры $k[G]$. Для любой
алгебраической подгруппы $H\subs G$ имеется многообразие $G/H$ левых смежных классов $G$ по $H$ с алгеброй
регулярных функций $k[G/H]=k[G]^H$.

\begin{df} Алгебраическая подгруппа $H\subs G$ называется \textit{обозримой}, если многообразие $G/H$ квазиаффинно, то
есть изоморфно некоторому открытому подмножеству аффинного многообразия.
\end{df}
Предметом доклада является следующая теорема.
\begin{theorem}\label{main} Пусть $G$~--- алгебраическая группа, $H\subs G$~--- её алгебраическая подгруппа. Тогда
следующие условия эквивалентны:
\begin{nums}{-1}
\item подгруппа $H$ обозрима;
\item существует линейное представление $\rh\cln G\to\GL(V)$ группы $G$ в конечномерном пространстве $V$ и вектор $v\in
V$, стабилизатором которого является в точности подгруппа $H$: $H=G_v$;
\item существует конечное множество регулярных функций $f_1\sco f_n\in k[G/H]$, которые разделяют точки многообразия
$G/H$;
\item для любого одномерного $H$-модуля $U$, вкладываемого в конечномерный $G$-модуль как $H$-подмодуль,
двойственный ему $H$-модуль $U^*$ также вкладывается в конечномерный $G$-модуль как $H$-подмодуль;
\item любой конечномерный $H$-модуль вкладывается в конечномерный $G$-модуль как $H$-подмодуль.
\end{nums}
\end{theorem}

\subsection{Пример необозримой подгруппы}
Приведём пример группы $G$ и её подгруппы $H$, для которых условия 1-5
неверны (правда, мы пока проверим невыполнение лишь некоторых из них).

Пусть $G:=\SL_2$, $H\subs G$~--- борелевская подгруппа всех верхнетреугольных матриц из $G$. Тогда многообразие
$G/H$ изоморфно $\mbb{P}^1$ и потому проективно. Значит, все регулярные функции на нём постоянны:
\eqn{\label{const} k[G/H]=\Cbb,} и эти функции не разделяют точки в $G/H$. В частности, никакой конечный набор
этих функций не разделяет точки. Поэтому условие 3 не выполнено.

Из равенства \eqref{const} следует также, что любой морфизм из $G/H$ в аффинное многообразие постоянный.
Следовательно, любой морфизм из $G$ в аффинное многообразие, инвариантный относительно правых сдвигов аргумента на
элементы $H$, тоже постоянный. Если задано линейное представление группы $G$ в пространстве $V$, такое что
$H=G_v$, $v\in V$, то морфизм $\ph\cln G\to V$, $f(g):=gv$ инвариантен относительно правых сдвигов на $H$:
$(gh)v=gv$, поэтому он постоянный, откуда $G_v=G$, в то время как $H\ne G$. Поэтому условие 2 также не
выполняется.

Для доказательства невыполнения условия 5 опишем все одномерные $H$-модули, которые вкладываются в конечномерные
$G$-модули как $H$-подмодули. Что то же самое, опишем все характеры $\chi\cln H\to k^*$, которым соответствуют
одномерные $H$-подмодули $G$-модулей. Любой $G$-модуль $V$, содержащий одномерный $H$-модуль $U$, раскладывается в
прямую сумму неприводимых, и хотя бы на одно из прямых слагаемых $V'$ подпространство $U$ проецируется
нетривиально. Оператор проецирования на $V'$ изоморфно отображает $U$ на свой образ, поэтому $H$-модуль $U$
вкладывается в $G$-модуль $V'$ как $H$-подмодуль. Из теории $\SL_2$-представлений следует, что $V'$ как $G$-модуль
изоморфен пространству однородных многочленов степени $m\ge0$ от двух переменных $x_1$ и $x_2$, на котором $G$
действует левым сдвигом аргумента: $(gf)(x):=f\hr{g^{-1}x}$, $x=\rbmat{x_1\\x_2}\in k^2$. Единственное одномерное
$H$-инвариантное подпространство в $V'$~--- это $U:=\ha{x_2^m}$. Если $f:=x_2^m\in V'$, $h=\rbmat{a&b\\0&c}\in H$,
то \eqn{\label{hf} hf(x)=f\hr{h^{-1}x}=\hr{c^{-1}x_2}^m=c^{-m}f(x),} то есть $hf=c^{-m}f=a^mf$. Итак, элемент $h$
действует на $U$ умножением на $a^m$, стало быть, $H$-модулю $U$ соответствует характер \eqn{\label{chiU} \chi\cln
H\to k^*,\rbmat{a&b\\0&c}\to a^m.} Значит, все характеры $\chi\cln H\to k^*$, соответствующие $H$-подмодулям
$G$-модулей,~--- это в точности все характеры, задаваемые формулой \eqref{chiU}. В частности, характер
$\chi\cln\rbmat{a&b\\0&c}\to a$ соответствует $H$-модулю $U$, вкладываемому в $G$-модуль как $H$-подмодуль, в то
время как $H$-модулю $U^*$ соответствует характер $\chi^{-1}\cln\rbmat{a&b\\0&c}\to c$, а так как $c=a^{-1}$, то
$U^*$ не может быть реализован как $H$-подмодуль $G$-модуля. Следовательно, не выполняются условия 5 и 4.

\subsection{Доказательство основной теоремы}

Условие 5 предполагает, что любой конечномерный (в частности, любой одномерный) $H$-модуль вкладывается в
конечномерный $G$-модуль как $H$-подмодуль, откуда немедленно вытекает условие 4. Итак, ($5\Ra4$).

Докажем остальные импликации.

($1\Ra3$) Если подгруппа $H\subs G$ обозрима, то многообразие $G/H$ квазиаффинно, то есть изоморфно открытому
подмножеству $X'$ аффинного многообразия $X\subs\A^n$. Тогда координатные функции $x_1\sco x_n$ на аффинном
многообразии $X$, ограниченные на $X'$, будут регулярными функциями на многообразии $X'$, разделяющими его точки.

($2\Ra1$) Если $\rh\cln G\to\GL(V)$~--- линейное представление, $v\in V$ и $H=G_v$, то слои морфизма $\ph\cln G\to
V$, $\ph(g):=gv$ суть левые смежные классы $G$ по $H$. Значит, существует инъективный морфизм из $G/H$ в $V$,
образ которого является орбитой $Gv$ вектора $v$. Итак, многообразие $G/H$ изоморфно орбите $Gv$, которая открыта
в своём замыкании $X\subs V$. Следовательно, $G/H$ квазиаффинно, то есть $H$ обозрима.

($3\Ra2$) Пусть $f_1\sco f_n\in k[G/H]$~--- регулярные функции, разделяющие точки $G/H$. Функции $f_i$ можно
интерпретировать как регулярные функции на $G$, инвариантные относительно правых сдвигов аргумента на $H$: $f_i\in
k[G/H]=k[G]^H$. Для действия группы $G$ на алгебре $k[G]$ по формуле \eqref{act} любой вектор $f\in k[G]$
содержится в некотором конечномерном $G$-модуле. В частности, каждая функция $f_i$ содержится в конечномерном
$G$-модуле $V_i\subs k[G]$. Рассмотрим абстрактную прямую сумму $V$ пространств $V_i$, на которой также имеется
структура конечномерного $G$-модуля. Поскольку $f_i\in k[G]^H$, подгруппа $H$ оставляет на месте каждый вектор
$f_i\in V_i$, а значит, и вектор $f\in V$. Обратно, пусть $gf=f$ для некоторого $g\in G$. В таком случае
$r_gf_i=f_i$ для любого $i=1\sco n$, в частности, $f_i(g)=f_i(eg)=(r_gf_i)(e)=f_i(e)$, а если рассматривать $f_i$
как регулярную функцию на $G/H$, то $f_i(gH)=f_i(eH)$ ($gH,eH\in G/H$). Последнее верно для любого $i=1\sco n$, а
так как $f_1\sco f_n$ разделяют точки в $G/H$, имеем $gH=eH$, откуда $g\in H$. Тем самым показали, что в
конечномерном $G$-модуле $V$ стабилизатор вектора $f$ совпадает с $H$.

($4\Ra2$) По теореме Шевалле существует конечномерное линейное представление $\rh\cln G\to\GL(V)$, а также вектор
$v\in V$, такой что $H=G_{\ha{v}}$, то есть $H$~--- подгруппа всех элементов из $G$, переводящих одномерное
подпространство $\ha{v}$ в себя. Тогда $U:=\ha{v}$~--- одномерный $H$-подмодуль $G$-модуля $V$, поэтому $H$-модуль
$U^*$ вкладывается в некоторый конечномерный $G$-модуль $W$ как $H$-подмодуль, а тривиальный одномерный $H$-модуль
$U\otimes U^*$ вложен как $H$-подмодуль в $G$-модуль $V\otimes W$. Причём $U\otimes U^*\subs V\otimes W$~---
одномерное подпространство, порождённое разложимым тензором $v\otimes w$, где $w\in W$, а $\ha{w}\subs W$~---
одномерный $H$-подмодуль, изоморфный $U^*$. Если некоторый элемент $g\in G$ оставляет на месте вектор $v\otimes
w$, то $v\otimes w=g(v\otimes w)=(gv)\otimes(gw)$, откуда $gv\in\ha{v}=U$, $gw\in\ha{w}$. То есть
$g\ha{v}=\ha{v}$, $g\in G_{\ha{v}}=H$. Обратно, если $h\in H$, то $h(v\otimes w)=v\otimes w$, так как $U\otimes
U^*$~--- тривиальный $H$-модуль. Итак, мы нашли линейное представление группы $G$ в конечномерном пространстве
$V\otimes W$, в котором вектор $v\otimes w$ имеет стабилизатор $H$.

\begin{lemma}\label{dim1} Одномерный $H$-модуль $U$ вкладывается как $H$-подмодуль в $G$-модуль $k[G]$ с действием
\eqref{act} тогда и только тогда, когда он вкладывается как $H$-подмодуль в некоторый конечномерный $G$-модуль
$V$.
\end{lemma}
\begin{proof} Если $U$ вкладывается в $k[G]$ как $H$-подмодуль, то $U=\ha{f}$, $f\in k[G]$. Тогда существует
конечномерный $G$-подмодуль $V\subs k[G]$, содержащий $f$, и в него вкладывается $U$ как $H$-подмодуль.

Обратно, пусть одномерный $H$-модуль $U$, соответствующий характеру $\chi\cln H\to k^*$, вложен как $H$-подмодуль
в конечномерный $G$-модуль $V$: $U=\ha{v}\subs V$, $0\ne v\in V$, $hv=\chi(h)v$ ($h\in H$). В двойственном
пространстве $V^*$ существует функционал $\la$, такой что $\la(v)\ne0$. Введём регулярную функцию $f\in k[G]$,
$f(g):=\la(gv)$. Выясним, как действует элемент $h\in G$ на элемент $f$ алгебры $k[G]$, рассматриваемой как
$G$-модуль с действием \eqref{act}: $(r_hf)(g)=f(gh)=\la(ghv)=\la\br{\chi(h)gv}=\chi(h)\la(gv)=\chi(h)f(g)$, то
есть $r_hf=\chi(h)f$. Кроме того, $f(e)=\la(v)\ne0$, откуда $f\ne0$, и, значит, $\ha{f}\subs k[G]$~--- одномерный
$H$-подмодуль, отвечающий характеру $\chi$ и потому изоморфный $U$.
\end{proof}

Допустим, что $G=G^0H$. Тогда любая функция из $k[G]$, инвариантная относительно правых сдвигов на $H$ и равная
нулю на $G^0$, равна нулю на всей $G$, поэтому алгебра $k[G/H]$ вкладывается в целостную алгебру $k\hs{G^0}$
(ограничением на $G^0$) и потому является целостной; можно считать, что $k[G/H]\subs k\hs{G^0}$.

Действие группы $G^0$ на целостной алгебре $k\hs{G^0}$ правыми сдвигами аргумента можно продолжить до действия
$G^0$ на алгебре $k\hr{G^0}=Qk\hs{G^0}$ (являющейся полем); действие~--- вновь правыми сдвигами аргумента. Тогда
для алгебраической подгруппы $H\subs G$ имеется подполе $k(G/H):=k\hr{G^0}^{G^0\cap H}$, которое является полем
рациональных функций на многообразии $G/H$.

\begin{lemma}\label{quot} Пусть подгруппа $H\subs G$ обозрима, а также справедливо равенство $G=G^0H$. В таком
случае $k(G/H)=Qk[G/H]$.
\end{lemma}

\begin{proof} Поскольку $k[G/H]\subs k\hs{G^0}$, имеем $Qk[G/H]\subs Qk\hs{G^0}=k\hr{G^0}$, причём все элементы поля
$Qk[G/H]$, как и все регулярные на $G$ функции из $k[G/H]$, $G^0\cap H$-инвариантны, откуда $Qk[G/H]\subs
k\hr{G^0}^{G^0\cap H}=k(G/H)$.

Докажем обратное включение. Поскольку $H$ обозрима, многообразие $G/H$ изоморфно некоторому открытому подмножеству
$X'$ аффинного многообразия $X$, и $k[G/H]=k[X']$. Тождественное вложение $X'\inj X$~--- доминантный морфизм,
поэтому двойственные гомоморфизмы алгебр $k[X]\to k[X']$ и полей $Qk[X]\to Qk[X']$ являются вложениями; можно
считать, что $k[X]\subs k[X']$ и $Qk[X]\subs Qk[X']$. Кроме того, \eqn{\label{quotf} k(G/H)=k(X')=k(X)=Qk[X]\subs
Qk[X']=Qk[G/H].}
\end{proof}

На данный момент доказано, в частности, что условия 1, 2 и 3 эквивалентны.

\subsubsection{Доказательство импликации ($1\Ra4$)}

Как показывает лемма \ref{dim1}, условие 4 равносильно следующему: если в $G$-модуль $k[G]$ с действием
\eqref{act} вкладывается одномерный $H$-модуль $U$ как $H$-подмодуль, то в него вкладывается и $H$-модуль $U^*$
как $H$-подмодуль. Докажем это в предположении, что выполнено 1, то есть $H$ обозрима.

Вначале предположим, что $G=G^0H$.

Допустим, что одномерный $H$-модуль $U$, соответствующий характеру $\chi\cln H\to k^*$, вложен как $H$-подмодуль в
$G$-модуль $k[G]$ с действием \eqref{act}: $U=\ha{f'}\subs k[G]$, $0\ne f'\in k[G]$, $r_hf'=\chi(h)f'$. Другими
словами, $f'\in k[G]$~--- весовой вектор для подгруппы $H$, соответствующий характеру $\chi\cln H\to k^*$.

Для произвольного $x\in G$ рассмотрим автоморфизм $l_x$ алгебры $k[G]$, действующий левым сдвигом аргумента:
$(l_xf)(g):=f(xg)$. В пространстве $k[G]$ любой оператор $l_x$ левого сдвига аргумента коммутирует с любым
оператором $r_g$ правого сдвига аргумента. Поэтому под действием $l_x$ все весовые векторы для действия правыми
сдвигами на $H$, соответствующие характеру $\chi\cln H\to k^*$, переставляются (то есть переходят друг в друга).
Поскольку $f'\ne0$, существует точка $y\in G$, такая что $f'(y)\ne0$. Тогда если $f:=l_yf'$, то $f(e)=f'(y)\ne0$,
$f\ne0$~--- весовой вектор для действия правыми сдвигами на $H$, отвечающий характеру $\chi$. Итак,
$r_hf=\chi(h)f$ ($h\in H$).

Поскольку $f\ne0$, для всякого $x\in G$ имеем $f_x:=\frac{l_xf}{f}\in k(G/H)$, так как для любого $h\in H$ имеем
\eqn{\label{rhfrac}
r_hf_x=r_h\hr{\frac{l_xf}{f}}=\frac{r_hl_xf}{r_hf}=\frac{l_xr_hf}{r_hf}=\frac{\chi(h)l_xf}{\chi(h)f}=\frac{l_xf}{f}=f_x;}
а так как $H$ обозрима, то (лемма \ref{quot}) $f_x\in Qk[G/H]$.

К идеалу $I:=(l_xf\cln x\in G)\nl k[G]$ можно применить теорему Гильберта о базисе, согласно которой в его системе
порождающих $\{l_xf\cln x\in G\}$ найдётся конечное порождающее подмножество $\{l_{x(1)}f\sco l_{x(m)}f\}$,
$x_i\in G$: \eqn{\label{ideal}I=(l_{x(1)}f\sco l_{x(m)}f).} Идеал $I\nl k[G]$ не может иметь нулей на $G$: для
всякого $x\in G$ имеем $(l_{x^{-1}}f)(x)=f(e)\ne0$, в то время как $l_{x^{-1}}\in I$. Следовательно, в $G$ не
существует точки, аннулируемой всеми многочленами $l_{x(1)}f\sco l_{x(m)}f$, иначе она аннулировалась бы всем
идеалом $I$.

Для каждого $i=1\sco m$ функция $f_{x(i)}$ лежит в $Qk[G/H]$ и потому представима в виде
$f_{x(i)}=\frac{a_i}{b_i}$, где $a_i,b_i\in k[G/H]=k[G]^H$, $b_i\ne0$. Если точка $x\in G$ такова, что $f(x)=0$,
то для всех $i=1\sco m$ справедливо равенство $(l_{x(i)}f)(x)b_i(x)=f(x)a_i(x)=0$, при этом $x$ не может
аннулироваться всеми многочленами $l_{x(i)}f$ одновременно, значит, существует $i$, такое что $b_i(x)=0$. Отсюда
следует, что любая точка в $G$, аннулируемая функцией $f\in k[G]$, аннулируется и функцией $b:=b_1\dots b_m\in
k[G]^H$. В силу теоремы Гильберта о нулях, для некоторого натурального $p$ элемент $b^p\in k[G]^H$ лежит в идеале
$(f)\nl k[G]$: $b^p=cf$, $c\in k[G]$. Поскольку $b_i\ne0$, $b\ne0$, и $cf=b^p\ne0$. Как мы знаем, $cf=b^p\in
k[G]^H$, и если $h\in H$, то $cf=r_h(cf)=r_hc\cdot r_hf=\chi(h)(r_hc)f$, а так как $f\ne0$, имеем
$r_hc=\br{\chi(h)}^{-1}c$. Вектор $c\in k[G]$ ненулевой, так как $cf\ne0$, значит, $c\in k[G]$~--- весовой вектор
для действия правыми сдвигами аргумента на $H$, соответствующий характеру $\chi^{-1}$.

Одномерный $H$-модуль $U$ отвечает характеру $\chi$, а $H$-модуль $U^*$~--- характеру $\chi^{-1}$. Следовательно,
$H$-подмодуль $\ha{c}\in k[G]$ изоморфен $U^*$.

Итак, мы доказали, что условие 1 (а значит, 2 и 3) влечёт 4 при $G=G^0H$.

Пусть в общем случае подгруппа $H\subs G$ обозрима, тогда условие 2 тоже выполнено: существует линейное
представление группы $G$ в пространстве $V$ и вектор $v\in V$ с условием $H=G_v$. Ограничив представление на
подгруппу $G^0H\sups H$, получаем тот же стабилизатор вектора $v$, поэтому для группы $G^0H$ и её подгруппы $H$
выполнено 2, вместе с ним 1, а значит, и 4 (по уже доказанному). Докажем, что и для $G$ и $H$ выполнено 4. Пусть
$U$~--- одномерный $H$-подмодуль конечномерного $G$-модуля $V$, отвечающий характеру $\chi\cln H\to k^*$.
Ограничив линейное представление на подгруппу $G^0H\sups H$, получаем, что $U$~--- одномерный $H$-подмодуль
конечномерного $(G^0H)$-модуля $V$. Следовательно, найдётся регулярная функция $c\in k\hs{G^0H}$, такая что для
любого $h\in H$ имеем \eqn{\label{rhc} r_hc=\br{\chi(h)}^{-1}c.} Функцию $c$ можно продолжить до регулярной
функции на $G$, положив её нулём на $G\sm G^0H$. Для новой функции останется верной формула \eqref{rhc}, так как
множество $G\sm G^0H$ инвариантно относительно правых сдвигов на $H$.

Получив указанную функцию $c$, регулярную на всей группе $G$, мы доказали выполнение условия 4 для группы $G$ и
подгруппы $H$.

Теперь импликация ($1\Ra4$) полностью доказана.

\subsubsection{($4\Ra5$)}

Предположим, что для группы $G$ и её подгруппы $H$ выполнено условие 4. Чтобы доказать 5, рассмотрим некоторый
конечномерный $H$-модуль $V$, $n:=\dim V$. Требуется вложить его в конечномерный $G$-модуль как $H$-подмодуль.

Для произвольного функционала $\la\in V^*$ определим линейное отображение $\ph_{\la}\cln V\to k[H]$ по формуле
$\hr{\ph_{\la}(v)}(h):=\la(hv)$. Если рассматривать $k[H]$ как $H$-модуль с действием правыми сдвигами аргумента,
то $\ph_{\la}$~--- гомоморфизм $H$-модулей: \eqn{\label{eqivqr}
\br{r_h\ph_{\la}(v)}(h')=\hr{\ph_{\la}(v)}(h'h)=\la(h'hv)=\hr{\ph_{\la}(hv)}(h'),} то есть
$r_h\ph_{\la}(v)=\ph_{\la}(hv)$. Зафиксируем $n$ линейно независимых функционалов $\la_1\sco\la_n$ и построим
гомоморфизм $H$-модулей $\ph:=\ph_{\la_1}\sop\ph_{\la_n}\cln V\to B$, где $B:=\underbrace{k[H]\sop k[H]}_n$. Тогда
$\ph$~--- вложение: если $v\in V$ и $v\ne0$, то найдётся число $i=1\sco n$, такое что $\la_i(v)\ne0$, в таком
случае $\hr{\ph_{\la_i}(v)}(e)\ne0$, и $\ph_{\la_i}(v)\ne0$. Таким образом, $H$-модуль $V$ можно считать
$H$-подмодулем $H$-модуля $B$.

Далее, имеется сюръективный гомоморфизм алгебр $\pi_0\cln k[G]\to k[H]$, двойственный тождественному вложению
$H\hookra G$ (то есть гомоморфизм ограничения). Легко понять, что $\pi_0$~--- гомоморфизм $H$-модулей $k[G]$ и
$k[H]$ (с действием правыми сдвигами аргумента). Положим $A:=\underbrace{k[G]\sop k[G]}_n$ и построим сюръективный
гомоморфизм $H$-модулей $\pi\cln A\to B$, отображающий $i$-ю копию $k[G]$ в $i$-ю копию $k[H]$ гомоморфизмом
$\pi_0$. Зафиксируем базис $\{e_1\sco e_n\}$ в $V$, в частности, $e_i\in B$, затем для каждого $i$ выберем
некоторый прообраз $f_i\in A$ вектора $e_i$ при $\pi$. В $G$-модуле $k[G]$ (вновь с действием правыми сдвигами)
существует конечномерный $G$-подмодуль $M_0$, содержащий проекции всех $f_i$ на все копии $k[G]$. Тогда
$M:=\underbrace{M_0\sop M_0}_n\subs A$~--- конечномерный $G$-подмодуль, содержащий все $f_i$. Соответственно,
подпространство $\pi(M)\subs B$ содержит все $e_i$, откуда $V\subs\pi(M)$. Поскольку $\pi$~--- гомоморфизм
$H$-модулей, а $V\subs B$~--- $H$-подмодуль, $\pi^{-1}(V)$ и $\Ker\pi$~--- $H$-подмодули $G$-модуля $A$. В таком
случае $W:=M\cap\br{\pi^{-1}(V)}$ и $Z:=M\cap\br{\Ker\pi}$~--- $H$-подмодули $G$-модуля $M$, кроме того, $Z\subs
W$, и \eqn{\label{piWV} \pi(W)=\pi\Br{M\cap\br{\pi^{-1}(V)}}=\pi(M)\cap V=V.}

%%W/Z\us{H}\cong\pi(W)=\pi\Br{M\cap\br{\pi^{-1}(V)}}=\pi(M)\cap V=V.} Итак, $H$-модуль $V$ является
%%фактормодулем $W/Z$ конечномерного $H$-модуля $W$ по $H$-подмодулю $Z\subs W$, причём $W$ вкладывается как $H$-подмодуль
%%в конечномерный $G$-модуль $M$.

Рассмотрим внешнюю алгебру $\La(M)$, также имеющую структуру конечномерного $G$-модуля. Пространство $U:=\La^m(Z)$
($m:=\dim Z$) является, очевидно, одномерным $H$-модулем. В алгебре $\La(M)$ содержится $H$-подмодуль
$W\wg\La^m(Z)$. Поскольку $\pi(W)=V$ и $\Ker\pi=Z$, отображение \eqn{\label{isoH} W\wg\La^m(Z)\to
V\otimes\La^m(Z),\; w\wg\la\to\pi(w)\otimes\la,} где $\la\in\La^m(Z)$, $w\in W$, $\pi(w)\in V$, является
изоморфизмом $H$-модулей $W\wg\La^m(Z)$ и $V\otimes\La^m(Z)$.

Итак, $H$-модуль $W\wg\La^m(Z)$ изоморфен $H$-модулю $V\otimes\La^m(Z)$, а значит, и $V\otimes U$. Одномерный
$H$-модуль $U$ вкладывается как $H$-подмодуль в конечномерный $G$-модуль $\La(M)$, следовательно, $H$-модуль $U^*$
тоже вкладывается как $H$-подмодуль в некоторый конечномерный $G$-модуль $M_1$ (условие 4). Но тогда $H$-модуль
$V$, изоморфный $H$-модулю $(V\otimes U)\otimes U^*$, вкладывается как $H$-подмодуль в конечномерный $G$-модуль
$\La(M)\otimes M_1$, так как $H$-модуль $(V\otimes U)$ изоморфен $H$-модулю $W\wg\La^m(Z)\subs\La(M)$.

Мы доказали импликацию ($4\Ra5$), а вместе с ней и всю теорему.

\end{document}
