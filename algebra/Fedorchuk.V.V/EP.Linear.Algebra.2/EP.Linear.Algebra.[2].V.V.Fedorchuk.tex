\documentclass[a4paper]{article}
\usepackage[simple,utf]{dmvn}

\title{Программа экзамена по линейной алгебре}
\author{Лектор Виталий Витальевич Федорчук}
\date{II семестр, 2006 г.}

\begin{document}
\maketitle

\centerline{\scriptsize Программы курса была восстановлена исключительно по конспектам лекций.}
\centerline{\scriptsize Поэтому на абсолютную полноту и достоверность претендовать не может.}

\begin{nums}{-1}
\item Определение линейного пространства и примеры.
\item Линейная зависимость.
\item Базисы и размерность.
\item Координаты. Переход от одного базиса к другому.
\item Подпространства. Операции над ними.
\item Прямые суммы. (+ алг. дополнение)
\item Линейные отображения. Изоморфизмы.
\item Факторпространства.
\item Определение и простейшие свойства сопряженных пространств.
\item Второе сопряженное пространство.
\item Аннуляторы и нулевые пространства. ОСЛУ.
\item Матрица линейного оператора.
\item Алгебра линейных операторов и алгебра матриц.
\item Инвариантные подпространства. Приводимые операторы.
\item Собственные векторы. Спектр оператора. Диагонализуемые операторы.
\item Характеристический многочлен оператора. Алгебраическая и геометрическая кратность корней.
\item Нильпотентные операторы. Их характеристический многочлен.
\item Разложение вырожденного оператора в прямую сумму нильпотентного и невырожденного.
\item Теорема Гамильтона Кэли.
\item Существование жорданового базиса для нильпотентного оператора.
\item Единственность жордановой формы нильпотентного оператора.
\item Жорданова форма произвольного оператора.
\item Билинейные  полуторалинейные функционалы и их матрицы.
\item Ранг билинейного функционала. Левое и правое ядро.
\item Квадратичные функции и полярные к ним билинейные функционалы.
\item Приведение квадратичной формы к каноническому виду методом Лагранжа.
\item Нормальный вид квадратичной формы. Закон инерции.
\item Метод Якоби.
\item Положительно определенные квадратичные функции. Критерий Сильвестра. Определитель Грамма. Неравенство Коши Буняковского.
\item Нормальный вид эрмитового функционала. Индексы инерции. Положительные функции.
\item Евклидовы, унитарные и нормированные пространства.
\item Длины и углы. Ортогональные системы. Процесс ортогонализации.
\item Ортогональное дополнение. Общий вид линейного функционала в евклидовом и унитарном пространстве.
\item Изоморфизмы евклидовых и унитарных пространств. Сопряженные операторы.
\item Самосопряжённые операторы.
\item Ортогональные и унитарные операторы в форме характеристического многочлена.
\item Канонический вид ортогонального, унитарного и нормального оператора.
\item Неотрицательные операторы.
\item Разложение произвольного оператора в композицию неотрицательного  изометрического.
\item Квадратичные формы в евклидовых пространствах.
\item Аффинные и точечные евклидовы пространства.
\item Плоскости в евклидовых пространствах. Способы их задания.
\item Взаимное расположение плоскостей.
\item Выпуклые множества в аффинных пространствах.
\item Точки общего положения. Симплексы. Барицентрические координаты.
\item Аффинные отображения аффинных пространств. Разложение аффинных отображений.
\item Классификация движений пространства.
\item Поверхности второго порядка в $n$ мерном пространстве.
\item Тензоры. Запись в координатах.
\item Операции над тензорами. Базис в пространстве тензоров.
\item Тензорное произведения линейных пространств. Его размерность.
\item Тензоры типа $(p,q)$ как элементы тензорного произведения. Пространство $T^{(p,q)}(V)$ как тензорное произведение пространств.
\item (???)
\item Внешнее умножение. Базис в пространстве $T^{(p,q)}_{\text{кос}}(V)$.
\item Симметрирование. Базис в $T^{(p,q)}_{\text{сим}}(V)$.
\end{nums}

\medskip\dmvntrail

\end{document}
