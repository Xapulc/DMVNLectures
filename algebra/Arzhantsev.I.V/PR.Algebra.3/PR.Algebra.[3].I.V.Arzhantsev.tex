\documentclass[a4paper]{article}
\usepackage[utf]{dmvn}

\title{Материалы с контрольных по алгебре}
\author{Преподаватель Иван Владимирович Аржанцев}
\date{III семестр. 2005 г.}

\def\dx{\,dx}

\begin{document}
\maketitle
\centerline{\small Набрано П. Рахмоновым, отредактировано и свёрстано DMVN Corporation.}

\medskip
\dmvntrail

\section{Контрольная работа 17 декабря 2005~г.}

\subsection{Вариант 1}

1. В факторгруппе свободной абелевой группы $A$ с базисом $x_1$,
$x_2$, $x_3$ по подгруппе $B$, порожденной $3x_1+x_2+7x_3$,
$-4x_1-2x_2+5x_3$, $4x_1-6x_2+5x_3$ найти порядок элемента
$21x_1-13x_2+7x_3+B$

2. Найти централизатор подстановки $\sigma =(2\ 4\ 6\ \ldots
2n)\in \mathbf{S}_{2n}$

3. Доказать, что любая группа порядка 100 разрешима.

4. Описать все одномерные комплексные представления группы
$\mathbf{V}_4\times \Db_4\times \Ab_5$.

5. Доказать, что все для любого 10-мерного комплексного
представления группы $\Z_2\times \mathbf{S}_3$ существует
двумерное инвариантное подпространство.

6. Представить в виде многочлена степени $\leq 2$ от $\alpha $
выражение $\frac{\alpha +1}{\alpha^2 -\alpha +1}$, где $\alpha$
--- корень $3x^3-2x^2+x+\overline{2}\in \Z_3[x]$


\subsection{Вариант 2}

1. В факторгруппе свободной абелевой группы $A$ с базисом $x_1$,
$x_2$, $x_3$ по подгруппе $B$, порожденной $2x_1+2x_2-3x_3$,
$x_1+7x_2-2x_3$, $5x_1-6x_2+x_3$ найти порядок элемента
$30x_1-7x_2+20x_3+B$

2. Конечно или бесконечно число орбит для действия группы
$
G=\left\{
\begin{pmatrix}
* & 0 & 0 \\
* & * & 0 \\
0 & 0 & *
\end{pmatrix}
\right\} \subset \mathbf{GL}_3(\Cbb ) $ в $\Cbb^3$? Указать по
одному представителю с каждой орбиты.

3. Найти нормализатор какой-либо силовской 2-подгруппы в
$\Ab_5$.

4. Описать все вещественные одномерные представления группы
$\Ab_4\times \Db_5\times \Ab_5$.

5. Верно ли, что в любом 5-мерном комплексном представлении
$\Z_3\times \mathbf{S}_4$ существует двумерное
инвариантное подпространство?

6. Представить в виде многочлена степени $\leq 2$ от $\alpha $
выражение $\frac{\alpha^5 + \alpha +2}{\alpha^4 -\alpha +1}$, где
$\alpha$ --- корень $x^3+x^2+x+\overline{2}\in \Z_3[x]$
\end{document}
