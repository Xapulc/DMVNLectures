\documentclass[a4paper]{article}
\usepackage[simple,utf]{dmvn}

\title{Программа экзамена по высшей алгебре}
\author{Лектор Альфред Львович Шмелькин}
\date{1 семестр, 2005 г. (поток механиков)}
\begin{document}
\maketitle
\begin{enumerate}
\item Приведение матрицы к ступенчатому виду.
\item Метод Гаусса.
\item Перестановки и их количество.
\item Изменение четности при транспозиции.
\item Подстановки и их четность.
\item Определители и их свойства.
\item Линейная зависимость (два определения).
\item Критерий равенства нулю определителя.
\item Основная лемма <<о линейной зависимости>>.
\item Базы системы строк.
\item Базы системы всех строк.
\item  Алгоритм нахождения базы системы строк.
\item  Определитель матрицы с углом нулей.
\item  Теорема <<О ранге матрицы>>.
\item  Определитель Вандермонда.
\item  Фундаментальная система решений.
\item  Общее решение неоднородной системы.
\item  Теорема Кронекера Капелли.
\item  Действия с матрицами.
\item  Умножение матрицы на диагональную матрицу. Некоммутативность умножения.
\item  Ассоциативность умножения матриц.
\item  Существование и единственность единичной и обратной матриц.
\item  Определитель произведения.
\item  Фальшивое разложение.
\item  Формула обратной матрицы.
\item  Определенность квадратной системы с ненулевым определителем.
\item  Правило Крамера.
\item  Элементарные преобразования и элементарные матрицы.
\item  Вычисление обратной матрицы элементарными преобразованиями.
\item  Делители нуля среди матриц.
\item  Кольца и их свойства. Поля.
\item  Делители нуля, отсутствие их в поле.
\item  Поле комплексных чисел.
\item  Алгебраическая и тригонометрическая форма. Сопряжение.
\item  Корни из комплексных чисел.
\item  Первообразные корни.
\item  Определяемость многочленом своих коэффициентов над бесконечным полем.
\item  Алгоритм деления с остатком.
\item  Однозначность частного и остатка.
\item  Формула $d(x) = f(x)u(x) - g(x)v(x)$.
\item  Неприводимые многочлены.
\item  Свойства взаимно-простых многочленов.
\item  Возможность разложения многочлена на неприводимые множители.
\item  <<Единственность>> разложения на неприводимые множители.
\item  Основная теорема о комплексных числах (без доказательства). Неприводимые многочлены с комплексными коэффициентами.
\item  Неприводимые многочлены с действительными коэффициентами.
\item  Разложение правильной дроби в сумму простейших.
\item  Лексикографический порядок. Старший член многочлена.
\item  Старший член симметрического многочлена.
\item  Выражение симметрического многочлена через элементарные.
\item  Формула Виета.
\item  Дискриминант многочлена.
\item  Кратные корни. Понижение кратности дифференцированием. Отделение кратных корней.
\item  Теорема Декарта.
\item  Группы. Примеры групп.
\item  Подгруппы.
\item  Группа $\Sb_n$. Умножения подстановки на транспозицию.
\item  Циклические группы. Порядок элементов.
\item  Подгруппы у циклических групп.
\item  Разложение в левые сложные классы. Теорема Декарта.
\item  Изоморфизм группы, его свойства.
\item  Изоморфизм циклических групп одного порядка.
\end{enumerate}

\medskip\dmvntrail
\end{document}
