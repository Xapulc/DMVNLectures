\section{Определители}
\label{det}

\epigraph{We build too many walls and not enough bridges.}{Isaac
  Newton}

\subsection{Определение}

\label{matrixdet:def}

Путь задана некоторая квадратная матрица $A$, каждый элемент которой
есть элемент коммутативного кольца с единицей.\index{определитель}\index{матрица!квадратная}

$$ A =
\begin{pmatrix}
a_{11} & a_{12} & \cdots & a_{1n} \\
a_{21} & a_{22} & \cdots & a_{2n} \\
\vdots & \vdots & \ddots & \vdots \\
a_{n1} & a_{n2} & \cdots & a_{nn}
\end{pmatrix}
$$

Тогда определителем матрицы называют следующую сумму по всем
перестановкам $n$ символов:

$$
\det A = \mbmat { a_{11} & \cdots & a_{1n} \\ \vdots & \ddots &
\vdots \\ a_{n1} & \cdots & a_{nn} } =
\sum_\sigma(
\sgn\sigma\cdot
a_{1\sigma(1)}
a_{2\sigma(2)}
\cdots
a_{n\sigma(n)}),\cln\text{Где}\cln\sigma=\rbmat{1&2&\cdots&n\\i_1&i_2&\cdots&i_n}
$$

%Простейшие определители матриц:
%\begin{description}
%	\item[Порядок $2\times2$:]
%		\begin{math}
%			\mbmat{a_1&b_1\\a_2&b_2} = a_1b_2 - a_2b_1
%		\end{math}
%	\item[Порядок $3\times3$:]
%		\begin{math}
%			\mbmat{a_1&b_1&c_1\\a_2&b_2&c_2\\a_3&b_3&c_3} =
%			a_1b_2c_3+a_2b_3c_1+a_3b_1c_2-a_3b_2c_1-b_3c_2a_1-c_3a_2b_1
%		\end{math}
%\end{description}

\begin{df}
  Матрица называется \emph{вырожденной}\index{матрица!вырожденная}, если её определитель равен 0 и \emph{невырожденной}\index{матрица!невырожденная} в обратном случае.
\end{df}

\subsection{Свойства определителей}

\label{matrixdet:props}

\begin{itemize}
	\item $\det A^\Tb=\det A$. Доказательство
		заключается в том, чтобы выразить элемент $A^\Tb$
		через элементы $A$:
		$$
			\det A^\Tb\eqdef
			\sum_\sigma\sgn\sigma\cdot
			a^\Tb_{1\sigma(1)}
			a^\Tb_{2\sigma(2)}
			\cdots
			a^\Tb_{n\sigma(n)} =
			\sum_\sigma\sgn\sigma\cdot
			a_{\sigma(1)1}
			a_{\sigma(2)2}
			\cdots
			a_{\sigma(n)n} =
		$$ $$
			= \sum_\sigma\sgn\sigma\cdot
			a_{1\sigma^{-1}(1)}
			a_{2\sigma^{-1}(2)}
			\cdots
			a_{n\sigma^{-1}(n)} =
			\left\{\tau=\sigma^{-1};\;|\tau|=|\sigma|\right\} =
		$$ $$
			= \sum_\tau(-1)^{|\tau|}
			a_{1\tau(1)}
			a_{2\tau(2)}
			\cdots
			a_{n\tau(n)} = \det A
		$$
	\item При умножении какой-либо строки/столбца на $\lambda$,
		определитель матрицы тоже увеличивается в $\lambda$ раз.
		$$
		\mbmat {
		a_{11} & a_{22} & \cdots & a_{1n} \\
		\vdots & \vdots & \ddots & \vdots \\
		\lambda a_{i1} & \lambda a_{i2} & \cdots & \lambda a_{in} \\
		\vdots & \vdots & \ddots & \vdots \\
		a_{n1} & a_{n2} & \cdots & a_{nn} } =
		\lambda
		\mbmat{
		a_{11} & a_{22} & \cdots & a_{1n} \\
		\vdots & \vdots & \ddots & \vdots \\
		a_{i1} & a_{i2} & \cdots & a_{in} \\
		\vdots & \vdots & \ddots & \vdots \\
		a_{n1} & a_{n2} & \cdots & a_{nn} }
		$$
		Доказательство тривиально.
	\item Если каждый элемент одной строки/столбца разбит на сумму, то
		определитель матрицы разбивается на сумму определителей.
		$$
		\mbmat{
		a_{11} & \cdots & a_{1n} \\
		\vdots & \ddots & \vdots \\
		a'_{i1} + a''_{i1} & \cdots & a'_{in} + a''_{in} \\
		\vdots & \ddots & \vdots \\
		a_{n1} & \cdots & a_{nn}
		} = \mbmat{
		a_{11} & \cdots & a_{1n} \\
		\vdots & \ddots & \vdots \\
		a'_{i1} & \cdots & a'_{in} \\
		\vdots & \ddots & \vdots \\
		a_{n1} & \cdots & a_{nn}
		} + \mbmat{
		a_{11} & \cdots & a_{1n} \\
		\vdots & \ddots & \vdots \\
		a''_{i1} & \cdots & a''_{in} \\
		\vdots & \ddots & \vdots \\
		a_{n1} & \cdots & a_{nn}
		}
		$$
		Для доказательства рассмотрим левую часть:
		$$
		\sum_\sigma\sgn\sigma\cdot
		a_{1\sigma(1)}
		a_{2\sigma(2)}
		\cdots
		\left(a'_{i\sigma(i)} + a''_{i\sigma(i)}\right)
		\cdots
		a_{n\sigma(n)} =
		$$ $$
		= \sum_\sigma\sgn\sigma\cdot\left(
		a_{1\sigma(1)}
		a_{2\sigma(2)}
		\cdots
		a'_{i\sigma(i)}
		\cdots
		a_{n\sigma(n)}
		+
		a_{1\sigma(1)}
		a_{2\sigma(2)}
		\cdots
		a''_{i\sigma(i)}
		\cdots
		a_{n\sigma(n)}
		\right) =
		$$ $$
		\sum_\sigma\sgn\sigma\cdot
		a_{1\sigma(1)}
		a_{2\sigma(2)}
		\cdots
		a'_{i\sigma(i)}
		\cdots
		a_{n\sigma(n)}
		+
		\sum_\sigma\sgn\sigma\cdot
		a_{1\sigma(1)}
		a_{2\sigma(2)}
		\cdots
		a''_{i\sigma(i)}
		\cdots
		a_{n\sigma(n)} \eqdef
		$$ $$
		\eqdef \det A_1 + \det A_2
		$$
		($A_1$, $A_2$~--- матрицы правой части)
\end{itemize}

\subsection{Частные случаи при вычислении определителя}

\label{matrixdet:part}

\begin{theorem}
	\label{matrixdet:part:swaprows}
	Если в матрице поменять местами две строки, то определитель поменяет
	знак.
\end{theorem}
\begin{proof}
	Пусть начальная матрица~--- $A$, и в ней поменяли местами $i$-ую и
	$j$-ую строки и получили матрицу $\tilde A$. Тогда имеют место
	следующие соотношения;
	$$
	\begin{cases}
		\det A &\eqdef
			\sum\limits_\sigma\sgn\sigma\cdot
			a_{1\sigma(1)}
			\cdots
			a_{i\sigma(i)}
			\cdots
			a_{j\sigma(j)}
			\cdots
			a_{n\sigma(n)}
		\\
		\det\tilde A &\eqdef
			\sum\limits_\sigma\sgn\sigma\cdot
			a_{1\sigma(1)}
			\cdots
			a_{i\sigma(j)}
			\cdots
			a_{j\sigma(i)}
			\cdots
			a_{n\sigma(n)}
	\end{cases}
	$$
	Пусть подстановка $\sigma$ имеет такой вид:
\begin{center}
\scalebox{0.7}{
	$\sigma=\rbmat{
		1&\cdots&i&\cdots&j&\cdots&n \\
		p_1&\cdots&p_i&\cdots&p_j&\cdots&p_n
	}$
}
\end{center}
	Тогда введем подстановку $\tau$:
\begin{center}
\scalebox{0.7}{
	$\tau=\rbmat{
		1&\cdots&i&\cdots&j&\cdots&n \\
		p_1&\cdots&p_j&\cdots&p_i&\cdots&p_n
	}$
}
\end{center}

	То есть подстановку, отличающуюся от $\sigma$ транспозицией $(ij)$.
	Из этого следует, что $\sgn\sigma=-\sgn\tau$. Ясно, что количество
	всевозможных $\sigma$ и $\tau$ совпадает, так как существует
	взаимооднозначное соответствие между ними. То есть нет разницы,
	суммировать по всем $\sigma$ или по всем $\tau$. Заменим тогда
	в $\det\tilde A$ подстановку:

	$$
	\det\tilde A =
	\sum_\sigma \sgn\sigma\cdot
		a_{1\sigma(1)}
		\cdots
		a_{i\sigma(j)}
		\cdots
		a_{j\sigma(i)}
		\cdots
		a_{n\sigma(n)}
	=
	$$ $$
	= \sum_\tau (-1)\sgn\tau
		a_{1\tau(1)}
		\cdots
		a_{i\tau(i)}
		\cdots
		a_{j\tau(j)}
		\cdots
		a_{n\tau(n)}
	= -\det A
	$$
\end{proof}

\begin{theorem}
  Определитель с двумя одинаковыми строками равен нулю
  \label{matrixdet:part:samerows}
\end{theorem}
\begin{proof}
  Доказательство тривиально и основано на предыдущей теореме.
\end{proof}

\begin{theorem}
	Определитель матрицы не изменится, если к какой-либо ее строке
	добавить другую строку, умноженную на $\lambda$.
\end{theorem}
\begin{proof}
	$$
	\mbmat{
	a_{11} & a_{12} & a_{13} & \cdots & a_{1n} \\
	\vdots & \vdots & \vdots & \ddots & \vdots \\
	\left(a_{i1} + \lambda a_{j1}\right) & \left(a_{i2} + \lambda
	a_{j2}\right) & \left(a_{i3} + \lambda a_{j3}\right) & \cdots & \left(a_{in} + \lambda a_{jn}\right) \\
	\vdots & \vdots & \vdots & \ddots & \vdots \\
	a_{j1} & a_{j2} & a_{j3} & \cdots & a_{jn} \\
	\vdots & \vdots & \vdots & \ddots & \vdots \\
	a_{n1} & a_{n2} & a_{n3} & \cdots & a_{nn}
	} =
	$$ $$
	= \mbmat{
	a_{11} & a_{12} & \cdots & a_{1n} \\
	\vdots & \vdots & \ddots & \vdots \\
	a_{i1} & a_{i2} & \cdots & a_{in} \\
	\vdots & \vdots & \ddots & \vdots \\
	a_{j1} & a_{j2} & \cdots & a_{jn} \\
	\vdots & \vdots & \ddots & \vdots \\
	a_{n1} & a_{n2} & \cdots & a_{nn}
	} + \lambda\underbrace{\mbmat{
	a_{11} & a_{12} & \cdots & a_{1n} \\
	\vdots & \vdots & \ddots & \vdots \\
	a_{j1} & a_{j2} & \cdots & a_{jn} \\
	\hline
	\vdots & \vdots & \ddots & \vdots \\
	a_{j1} & a_{j2} & \cdots & a_{jn} \\
	\hline
	\vdots & \vdots & \ddots & \vdots \\
	a_{n1} & a_{n2} & \cdots & a_{nn}
	}}_{=\cln0\;\text{(\ref{matrixdet:part:samerows})}}
	= \mbmat{
	a_{11} & \cdots & a_{1n} \\
	\vdots & \ddots & \vdots \\
	a_{n1} & \cdots & a_{nn}
	}
	$$
\end{proof}

Из этих частных случаев следует, что матрица является невырожденной тогда и только тогда, когда система её строк линейно независима.

\subsection{Аксиоматический подход}

Стоит отметить, что определитель однозначно определяется своими свойствами, а именно можно ввести функцию определителя и аксиоматически:

\emph{Определителем}\index{определитель} называется функция $\det\colon M_{n\times n}(\R)\to\R$, удовлетворяющая следующим условиям:

\begin{enumerate}
	\item(линейность)\par
		К $i$-ой строке можно прибавить $j$-ю, домноженную на некоторое число
		$\lambda\in\R$. Также $i$-ю строку можно домножить на ненулевое число
		$\lambda\in\R$. При этом определитель не изменяется.
		$$
			\mbmat{\cdots\\ i\\\cdots\\ j\\\cdots}=
			\mbmat{\cdots\\ i+\lambda j\\\cdots\\ j\\\cdots}
		$$ $$
			\mbmat{\cdots\\ i\\\cdots}=
			\mbmat{\cdots\\\lambda i\\\cdots}
		$$
	\item(кососимметричность)\par
		Если $i$-ю и $j$-ю строки поменять местами, то определитель изменит знак.
		$$
			\mbmat{\cdots\\ i\\\cdots\\ j\\\cdots}=-
			\mbmat{\cdots\\ j\\\cdots\\ i\\\cdots}
		$$
	\item(равенство нулю)\par
		Если две строки матрицы совпадают, то ее определитель равен нулю.
		$$
			\mbmat{\cdots\\\alpha\\\cdots\\\alpha\\\cdots}=0
		$$
	\item(нормировка)\par
		Определитель единичной матрицы равен единице.
		$$ \det E=1 $$
\end{enumerate}

Покажем вывод развёрнутой формулы определителя по заданным условиям:

\begin{multline*}
	\det A=
	\mbmat{a_{11} & \cdots & a_{1n} \\ \vdots & \ddots & \vdots \\ a_{n1} & \cdots & a_{nn}}=
	\mbmat{
		a_{11}\varepsilon_1+a_{12}\varepsilon_2+\cdots+a_{1n}\varepsilon_n\\
		\cdots \\
		a_{n1}\varepsilon_1+a_{n2}\varepsilon_2+\cdots+a_{nn}\varepsilon_n
	}= \\
	=\{\text{раскрывая по линейности определителя}\}= \\
	=\sum_{i_1,\cdots,i_n\in\{1,\cdots,n\}}\mbmat{
		a_{1i_1}\varepsilon_{i_1} \\
		a_{2i_2}\varepsilon_{i_2} \\
		\cdots \\
		a_{ni_n}\varepsilon_{i_n}
	}=
	\sum_{i_1,\cdots,i_n\in\{1,\cdots,n\}}
	a_{1i_1}a_{2i_2}\cdots a_{ni_n}
	\mbmat{
		\varepsilon_{i_1} \\
		\varepsilon_{i_2} \\
		\cdots \\
		\varepsilon_{i_n}
	}= \\
	=\{\text{убираем нулевые слагаемые}\}=
	\sum_{\substack{
		i_1,\cdots,i_n\in\{1,\cdots,n\} \\
		\forall\alpha,\beta\in\{1,\cdots,n\}\colon i_\alpha\ne i_\beta
	}}
	a_{1i_1}a_{2i_2}\cdots a_{ni_n}
	\mbmat{
		\varepsilon_{i_1} \\
		\varepsilon_{i_2} \\
		\cdots \\
		\varepsilon_{i_n}
	}= \\
	=\{\text{введем подстановку $\sigma\in\Sb_n\colon\rbmat{1&\cdots&n\\i_1&\cdots&i_n}$}\}=
	\sum_{\sigma\in\Sb_n}
	a_{1\sigma(1)}a_{2\sigma(2)}\cdots a_{n\sigma(n)}
	\mbmat{
		\varepsilon_{\sigma(1)} \\
		\varepsilon_{\sigma(2)} \\
		\cdots \\
		\varepsilon_{\sigma(n)}
	}
\end{multline*}

Теперь остается понять, что матрица, составленная из строк $\varepsilon_{\sigma(1)},
\varepsilon_{\sigma(2)}, \cdots, \varepsilon_{\sigma(n)}$, есть единичная
матрица со строками, переставленными в соответствии с подстановкой $\sigma$. То есть
определитель этой переставленной матрицы равен $\sgn\sigma\cdot\det E$.

Тогда:

$$
	\det A
	=\sum_{\sigma\in\Sb_n}
	\sgn\sigma
	a_{1\sigma(1)}a_{2\sigma(2)}\cdots a_{n\sigma(n)}
	\underbrace{\det E}_{=\cln1}
$$

\subsection{Треугольная матрица}\index{матрица!треугольная}\index{матрица!ступенчатая}

\label{matrixdet:triangular}

Треугольной матрицей называется матрица, имеющая следующий вид:
$$ A = \rbmat{
a_{11} & a_{12} & a_{23} & \cdots & a_{1n} \\
0      & a_{22} & a_{23} & \cdots & a_{2n} \\
0      & 0      & a_{33} & \cdots & a_{3n} \\
\vdots & \vdots & \vdots & \ddots & \vdots \\
0      & 0      & 0      & 0      & a_{nn}
}
$$

Отсюда видно, что $\det A = a_{11} a_{22} a_{33} \cdots a_{nn}$, так как
существует единственная (диагональная) подстановка, которая не содержит
нулевых элементов.

Любую матрицу можно привести к треугольному виду ({\ref{mat:echelon}}).

\subsection{Разложение определителя по строке или столбцу}

\label{matrixdet:rowexpansion}

\begin{lemma}[Об определителе блочной матрицы]\index{лемма!об определителе блочной матрицы}
	\label{matrixdet:blocklemma}
	Пусть задана матрица размером $n\times n$ следующего вида:
	$$
	M =
	\left(
	\begin{array}{rcl|rrrclll}
		& &  &  &&& &&& \\
		&A&  &  &&&*&&& \\
		&_{k \times k}&  &  &&& &&& \\
		\hline
		& &  &  &&& &&& \\
		& &  &  &&& &&& \\
		&0&  &  &&&B&&& \\
		& &  &  &&& &&& \\
		& &  &  &&&_{(n-k)\times(n-k)}&&&
	\end{array}
	\right)
	$$
	$A$, $B$~--- некоторые подматрицы; $*$~--- произвольные элементы; $0$~--- нулевые элементы. Тогда выполняется следующее равенство:
	$$ \det M = \det A \cdot \det B $$
\end{lemma}
\begin{proof}
  Будем приводить матрицы $A$ и $B$ к треугольному виду. Очевидно, что эти же преобразования, применённые к исходной матрице приведут и её к треугольному виду. Тогда её определитель будет равен произведению элементов главной диагонали, то есть произведению диагональных элементов первой матрицы, умноженному на произведение диагональных элементов второй матрицы. Доказательство завершено.
\end{proof}

Теперь рассмотрим определитель матрицы размером $n\times n$.

$$
\mbmat{
a_{11} & a_{12} & \cdots & a_{1n} \\
\vdots & \vdots & \ddots & \vdots \\
a_{i1} & a_{i2} & \cdots & a_{in} \\
\vdots & \vdots & \ddots & \vdots \\
a_{n1} & a_{n2} & \cdots & a_{nn}
}
=
\mbmat{
a_{11} & a_{12} & \cdots & a_{1n} \\
\vdots & \vdots & \ddots & \vdots \\
a_{i1} &    0   & \cdots &    0   \\
\vdots & \vdots & \ddots & \vdots \\
a_{n1} & a_{n2} & \cdots & a_{nn}
}
+ \cdots +
\mbmat{
a_{11} & a_{12} & \cdots & a_{1n} \\
\vdots & \vdots & \ddots & \vdots \\
   0   &    0   & \cdots & a_{in} \\
\vdots & \vdots & \ddots & \vdots \\
a_{n1} & a_{n2} & \cdots & a_{nn}
}
$$

Такое разложение можно получить, представив $i$-ю строку в виде суммы:

$$
\left(a_{i1},a_{i2},a_{i3},\cdots,a_{in}\right) =
\left(a_{i1},   0  ,   0  ,\cdots,   0  \right) +
\left(   0  ,a_{i2},a_{i3},\cdots,a_{in}\right) =
$$ $$
=
\left(a_{i1},   0  ,   0  ,\cdots,   0  \right) +
\left(   0  ,a_{i2},   0  ,\cdots,   0  \right) +
\left(   0  ,   0  ,a_{i3},\cdots,a_{in}\right) =
$$ $$
=
\left(a_{i1},   0  ,   0  ,\cdots,   0  \right) +
\cdots                                          +
\left(   0  ,   0  ,   0  ,\cdots,a_{in}\right)
$$

Теперь в правой части из каждого определителя вынесем элемент $i$-ой
строки:

$$
\cdots =
a_{i1}\left|
\begin{array}{|c|ccc}
\cline{1-1}
a_{11} & a_{12} & \cdots & a_{1n} \\
\vdots & \vdots & \ddots & \vdots \\
\hline
\multicolumn{1}{|c|}{1} & 0 & \cdots &\multicolumn{1}{c|}{0}\\
\hline
\vdots & \vdots & \ddots & \vdots \\
a_{n1} & a_{n2} & \cdots & a_{nn} \\
\cline{1-1}
\end{array}\right|
+
a_{i2}\left|
\begin{array}{c|c|cc}
\cline{2-2}
a_{11} & a_{12} & \cdots & a_{1n} \\
\vdots & \vdots & \ddots & \vdots \\
\hline
\multicolumn{1}{|c}{0} & \multicolumn{1}{|c|}{1} & \cdots &\multicolumn{1}{c|}{0}\\
\hline
\vdots & \vdots & \ddots & \vdots \\
a_{n1} & a_{n2} & \cdots & a_{nn} \\
\cline{2-2}
\end{array}\right|
+
\cdots
+
a_{in}\left|
\begin{array}{ccc|c|}
\cline{4-4}
a_{11} & a_{12} & \cdots & a_{1n} \\
\vdots & \vdots & \ddots & \vdots \\
\hline
\multicolumn{1}{|c}{0} & 0 & \cdots &\multicolumn{1}{|c|}{1}\\
\hline
\vdots & \vdots & \ddots & \vdots \\
a_{n1} & a_{n2} & \cdots & a_{nn} \\
\cline{4-4}
\end{array}\right|
$$

Теперь будем менять местами столбцы, передвигая $j$-й столбец (с
единицей в $i$-ой строке) на первое место. При этом знак определителя
изменится $(i-1)+(j-1)$ раз. Таким образом:

\begin{multline*}
\cdots=
(-1)^{(i-1)+(1-1)}\cdot a_{i1}\left|
\begin{array}{cccc}
1 & 0 & \cdots & 0 \\
\cline{2-4}
a_{11} & \multicolumn{1}{|c}{a_{12}} & \cdots & \multicolumn{1}{c|}{a_{1n}} \\
\vdots & \multicolumn{1}{|c}{\vdots} & \ddots & \multicolumn{1}{c|}{\vdots} \\
\widehat{a_{i1}} & \multicolumn{1}{|c}{\widehat{a_{i2}}} & \cdots & \multicolumn{1}{c|}{\widehat{a_{in}}} \\
\vdots & \multicolumn{1}{|c}{\vdots} & \ddots & \multicolumn{1}{c|}{\vdots} \\
a_{n1} & \multicolumn{1}{|c}{a_{n2}} & \cdots & \multicolumn{1}{c|}{a_{nn}} \\
\cline{2-4}
\end{array}\right|
+ \\ +
(-1)^{(i-1)+(2-1)}\cdot a_{i2}\left|
\begin{array}{ccccc}
1 & 0 & 0 & \cdots & 0 \\
\cline{2-5}
a_{12} & \multicolumn{1}{|c}{a_{11}} & a_{13} & \cdots & \multicolumn{1}{c|}{a_{1n}} \\
\vdots & \multicolumn{1}{|c}{\vdots} & \vdots & \ddots & \multicolumn{1}{c|}{\vdots} \\
\widehat{a_{i2}} & \multicolumn{1}{|c}{\widehat{a_{i1}}} & \widehat{a_{i3}} & \cdots & \multicolumn{1}{c|}{\widehat{a_{in}}} \\
\vdots & \multicolumn{1}{|c}{\vdots} & \vdots & \ddots & \multicolumn{1}{c|}{\vdots} \\
a_{n2} & \multicolumn{1}{|c}{a_{n1}} & a_{n3} & \cdots & \multicolumn{1}{c|}{a_{nn}} \\
\cline{2-5}
\end{array}\right|
+\cdots \\ \cdots+
(-1)^{(i-1)+(n-1)}\cdot a_{in}\left|
\begin{array}{ccccc}
1 & 0 & 0 & \cdots & 0 \\
\cline{2-5}
a_{1n} & \multicolumn{1}{|c}{a_{11}} & a_{12} & \cdots & \multicolumn{1}{c|}{a_{1(n-1)}} \\
\vdots & \multicolumn{1}{|c}{\vdots} & \vdots & \ddots & \multicolumn{1}{c|}{\vdots} \\
\widehat{a_{in}} & \multicolumn{1}{|c}{\widehat{a_{i1}}} & \widehat{a_{i2}} & \cdots & \multicolumn{1}{c|}{\widehat{a_{i(n-1)}}} \\
\vdots & \multicolumn{1}{|c}{\vdots} & \vdots & \ddots & \multicolumn{1}{c|}{\vdots} \\
a_{nn} & \multicolumn{1}{|c}{a_{n1}} & a_{n2} & \cdots & \multicolumn{1}{c|}{a_{n(n-1)}} \\
\cline{2-5}
\end{array}\right|
=\cdots
\end{multline*}

Назовем \emph{минором}\index{минор} $i$-ой строки и $j$-ого столбца определитель
матрицы, полученный <<вычеркиванием>> $i$-ой строки и $j$-ого столбца
(обозначим за $M_{ij}$). По лемме о блочной матрице
(\ref{matrixdet:blocklemma}) разложим определители в полученном
выражении. Также заметим, что если к степени $(-1)$ прибавить двойку, то
ничего не изменится. Таким образом, заменяя соответствующие определители
минорами, получаем.

$$
\cdots=
a_{i1}\cdot(-1)^{i+1}\cdot M_{i1} +
a_{i2}\cdot(-1)^{i+2}\cdot M_{i2} +
\cdots                            +
a_{in}\cdot(-1)^{i+n}\cdot M_{in} +
=
\sum_{j=1}^n a_{ij}\cdot(-1)^{i+j}\cdot M_{ij}
=\cdots
$$

Назовем выражение $A_{ij}=(-1)^{i+j}\cdot M_{ij}$ \emph{алгебраическим
дополнением}\index{алгебраическое дополнение} к элементу $a_{ij}$. Тогда окончательно получаем формулу
разложения определителя матрицы по строке:

$$
\Delta = \cdots = \sum_{j=1}^n a_{ij}\cdot A_{ij}
$$

%Можно легко получить связь миноров и алгебраических дополнений: знаки
%$+$ и $-$, расставленные в матрице, будут образовывть шахматную сетку:

%$$
%\begin{array}{|c|c|c|c|c}
%	\hline
%	+ & - & + & - & \cdots \\
%	\hline
%	- & + & - & + & \cdots \\
%	\hline
%	+ & - & + & - & \cdots \\
%	\hline
%	- & + & - & + & \cdots \\
%	\hline
%	\vdots & \vdots & \vdots & \vdots & \ddots
%\end{array}
%$$

%\begin{sample}
%	Разложим определитель матрицы по второй строке:
%	$$\mbmat{1&-3&2&4\\a&b&c&d\\0&-1&1&2\\1&1&0&3}=
%	a\cdot A_{21}+b\cdot A_{22}+c\cdot A_{23}+d\cdot A_{24}$$
%	Вычислим $A_{2i}$:
%	\begin{description}
%		\item[$A_{21}=$]
%			\begin{math}
%				-\mbmat{-3&2&4\\-1&1&2\\\phm1&0&3}=-(-3)=3
%			\end{math}
%		\item[$A_{22}=$]
%			\begin{math}
%				\phm\mbmat{-1&2&4\\\phm0&1&2\\\phm1&0&3}=3
%			\end{math}
%		\item[$A_{23}=$]
%			\begin{math}
%				-\mbmat{1&-3&4\\0&-1&2\\1&\phm1&3}=-(-9)=9
%			\end{math}
%		\item[$A_{24}=$]
%			\begin{math}
%				\phm\mbmat{1&-3&2\\0&-1&1\\1&\phm1&0}=-2=2
%			\end{math}
%	\end{description}
%	Таком образом,
%	$$\Delta=3a+3b+9c+2d$$
%\end{sample}

С разложением по строке связана следующая лемма:

\begin{lemma}
	\label{matrixdet:rowexpansion:replace}
	Если в разложении матрицы по $i$-ой строке вместо элементов $i$-ой
	строки взять элементы $j$-ой строки, то получится нуль.
	$$
	a_{i1}\cdot A_{i1}+\cdots+a_{in}\cdot A_{in}=\Delta
	$$ $$
	a_{i1}\cdot A_{j1}+\cdots+a_{in}\cdot A_{jn}=0
	$$
\end{lemma}
\begin{proof}
	$$\Delta =a_{j1}\cdot A_{j1}+\cdots+a_{jn}\cdot A_{jn}$$
	Так как $A_{ji}$ не зависит от элементов $j$-ой строки, то можно
	<<подставить>> на $j$-ую строку элементы $i$-ой:
	$$\Delta'=a_{i1}\cdot A_{j1}+\cdots+a_{in}\cdot A_{jn}$$
	А $\Delta'=0$, так как в матрице получились две одинаковые строки
	(\ref{matrixdet:part:samerows}).
	Значит, равенство верно.
\end{proof}

\subsection{Определитель произведения матриц}
\label{matrixdet:mul}

\begin{theorem}
	Определитель\index{определитель!произведения матриц} произведения двух квадратных матриц размера $n\times n$
	равен произведению их определителей.
	$$C=AB \ra \det C = \det A \det B$$
\end{theorem}

\begin{lemma}
  Всякая невырожденная матрица представляется в виде произведения элементарных матриц. Всякая вырожденная матрица представляется в виде произведения элементарных матриц и матрицы, имеющей нулевую строку.
\end{lemma}

\begin{proof}
\begin{enumerate}
  \item Если $A$ -- элементарная матрица, то равенство очевидно.

  \item Если $A$ -- невырожденная матрица, она может быть представлена в виде произведения элементарных матриц на единичную:
  $$A=U_nU_{n-1}\dots U_1E$$

  Тогда применяя применяя пункт 1 получаем, что $\det(AB)=\det(U_n\dots U_1B)\bw=\det U_n\det(U_{n-1}\dots U_1B)=\dots =\det U_n \dots\det U_1 \det B=\det A\cdot\det B$

  \item Если $A$ -- вырожденная матрица, то используя пп. 1 и 2 получаем, что
  $$\det(AB)=\det U_n \dots\det U_1 \det(A'B),$$ где $A'$ имеет нулевую строку, а следовательно и $A'B$ имеет нулевую строку, $\det(AB)\bw=\det A\det B=0$.
\end{enumerate}
\end{proof}

%%\begin{proof}
%	Из определения операции умножения следует:

%	$$
%	c_{ij}=\sum_k^n a_{ik}b_{kj}
%	$$

%	Определитель же матрицы $C$, по определению, равен:

%	$$
%	\det C=
%	\sum_\sigma\sgn\sigma\cdot
%	c_{1\sigma(1)}
%	c_{2\sigma(2)}
%	\cdots
%	c_{n\sigma(n)}
%	$$

%	Пусть $\sigma=\rbmat{1&2&\cdots&n\\i_1&i_2&\cdots&i_n}$. Введем для
%	элеметнов матрицы $C$ индекс суммирования: $k$-ый элемент $i$-ой строки
%	матрицы $C$~--- это $c_{ik}=\sum_{j_k}^n a_{ij_k} b_{j_kk}$ (индекс
%	суммирования -- $j$). Тогда:

%	$$
%	\det C=
%	\sum_{i_1,\cdots,i_n}
%	\sgn\sigma\cdot
%	\left(\sum_{j_1} a_{1j_1}b_{j_1i_1}\right)
%	\left(\sum_{j_2} a_{2j_2}b_{j_2i_2}\right)
%	\cdots
%	\left(\sum_{j_n} a_{nj_n}b_{j_ni_n}\right)
%	=\cdots
%	$$

%	Все суммы по $j$ можно объединить в одну, последовательно объединяя
%	соседние скобки\linebreak($\left(\sum_a x_a\right)\left(\sum_b
%	y_b\right)=\sum_{a,b} x_ay_b$):

%	$$
%	\cdots=
%	\sum_{i_1,\cdots,i_n}
%	\sgn\sigma\cdot
%	\sum_{j_1,\cdots,j_n}
%	a_{1j_1}b_{j_1i_1}
%	\cdot
%	a_{2j_2}b_{j_2i_2}
%	\cdots
%	a_{nj_n}b_{j_ni_n}
%	=\cdots
%	$$

%	Внесем также $(-1)$ под знак суммы и поменяем порядок суммирования:
%	внешнее по $i$, а внутреннее~по~$j$.

%	$$
%	\cdots=
%	\sum_{j_1,\cdots,j_n}
%	\sum_{i_1,\cdots,i_n}
%	\sgn\sigma\cdot
%	a_{1j_1}b_{j_1i_1}
%	\cdot
%	a_{2j_2}b_{j_2i_2}
%	\cdots
%	a_{nj_n}b_{j_ni_n}
%	=\cdots
%	$$

%	Теперь вынесем из-под внутренней суммы $a_{kj_k}$, так как они не
%	зависят от $i_k$.

%	$$
%	\cdots=
%	\sum_{j_1,\cdots,j_n}
%	a_{1j_1} a_{2j_2} \cdots a_{nj_n}
%	\underbrace{
%		\sum_{i_1,\cdots,i_n}
%		\sgn\sigma\cdot
%		b_{j_1i_1} b_{j_2i_2} \cdots b_{j_ni_n}
%	}_{=\cln\det\tilde B}
%	=\cdots
%	$$

%	$\tilde B$~--- это матрица, полученная из матрицы $B$ путем
%	некоторых перестановок строк:

%	$$
%	B=\rbmat{
%	b_{11} & \cdots & b_{1n} \\
%	\vdots & \ddots & \vdots \\
%	b_{n1} & \cdots & b_{nn}
%	}
%	\ra
%	\tilde B=\rbmat{
%	b_{j_11} & \cdots & b_{j_1n} \\
%	\vdots   & \ddots & \vdots   \\
%	b_{j_n1} & \cdots & b_{j_nn}
%	}
%	$$

%	Очевидно, что если все $j$ различны (то есть не существует таких
%	$\alpha$ и $\beta$, что $j_\alpha=j_\beta$), то можно задать
%	перестановку $\tau$:

%	$$
%	\exists\tau=\rbmat{1&2&\cdots&n\\j_1&j_2&\cdots&j_n}
%	$$

%	Тогда $\det\tilde B=(-1)^{|\tau|}\det B$ (если четное число раз
%	поменяли строки, то знак определителя остался прежним, а значит
%	равен $\det B$; в противном случае~---$\,-\det B$).

%	Если есть такие $\alpha$ и $\beta$, что $j_\alpha=j_\beta$, то тогда
%	определитель матрицы $\tilde B$ равен нулю из-за одинаковых строк
%	(\ref{matrixdet:part:samerows}).

%	Таким образом, можно считать, что все $j$ различны, так как
%	одинаковые $j$ дают нулевое слагаемое, которое не влияет на
%	окончательную сумму.

%	Тогда окончательно получаем:
%	$$
%	\det C=
%	\sum_{j_1,\cdots,j_n}
%	a_{1j_1} a_{2j_2} \cdots a_{nj_n}
%	(-1)^{|\tau|}\cdot\det B\eqdef\det A\det B
%	$$
%%\end{proof}


\subsection{Определитель Вандермонда. Интерполяция}\index{определитель!Вандермонда}\index{интерполяция}
\label{sle:interpolation}

Определитель Вандермонда $V(a_1,\dots a_n)$ --- определитель следующей матрицы:

$$\mbmat{
1 &    1 & \cdots & 1 \\
a_{1} & a_{2} & \cdots & a_{n} \\
a_{1}^2 & a_{2}^2 & \cdots & a_{n}^2 \\
\vdots & \vdots & \ddots & \vdots \\
a_{1}^{n-1} & a_{2}^{n-1} & \cdots & a_{n}^{n-1}
}
$$
Сделаем все элементы первого столбца, кроме самого первого, нулями. Для этого вычтем из $n$-й строки $a_1$ раз $n-1$-ю, из $n-1$-й~--- $a_1$ раз ${n-2}$-ю и т.д. По схеме вычисления определителя блочной матрицы, определитель исходной матрицы равен определителю следующей матрицы
\begin{multline*}
\mbmat{
a_2-a_1 &    a_3-a_1 & \cdots & a_n-a_1 \\
a_2^2-a_2a_1 & a_3^2-a_3a_1 & \cdots & a_n^2-a_na_1 \\
%a_2^3-a_1a_2^2 & a_3^3-a_1a_3^2 & \cdots & a_n^2 \\
\vdots & \vdots & \ddots & \vdots \\
a_{2}^{n-1} - a_1a_2^{n-2} & a_{3}^{n-1} - a_1a_3^{n-2} & \cdots & a_{n}^{n-1}-a_1a_n^{n-2}
}=\\=(a_2-a_1)(a_3-a_1)\dots(a_n-a_1)\mbmat{
1 &    1 & \cdots & 1 \\
a_{2} & a_{3} & \cdots & a_{n} \\
a_{2}^2 & a_{3}^2 & \cdots & a_{n}^2 \\
\vdots & \vdots & \ddots & \vdots \\
a_{2}^{n-2} & a_{3}^{n-2} & \cdots & a_{n}^{n-2}
}=\dots=\prod_{i>j}(a_i-a_j)
\end{multline*}

Этот определитель тесно связан с интерполяцией функций. Действительно, через любые $n+1$ точек с разными абсциссами обязательно проходит многочлен степени не больше $n$, причём только один (по двум точкам можно построить единственную прямую, проходящую через них, по трём~--- единственную параболу и т.п.)

Пусть у нас имеется набор из $n$ точек $(x_i,y_i)$. Пусть искомый многочлен имеет вид $$P(x)=a_0+a_1x+\dots+a_nx^n.$$ Нам известно, что $\forall i\cln P(x_i)=y_i$. Запишем эти условия в виде системы ($n$ уравнений, $a_0,\dots a_n$~--- неизвестные):
$$
\left\{
 \begin{array}{lllllllll}
  a_0&+&a_1x_1&+&\dots&+&a_nx_1^n&=&y_1\\
%  a_0&+&a_1x_2&+&\dots&+&a_nx_2^n&=&y_2\\
  \hdotsfor[2]{9}\\
  a_0&+&a_1x_{n+1}&+&\dots&+&a_nx_{n+1}^n&=&y_{n+1}\\
 \end{array}
\right.
$$
Эта система имеет единственное решение (см. выше).

Теперь найдём саму интерполированную функцию: введём многочлен

  \begin{center}
 \scalebox{0.8}{ $L_i(x)=\frac{ (x-x_1) \dots (x-x_{i-1}) (x-x_{i+1}) \dots (x-x_{n+1}) }{(x_i-x_1) \dots (x_i-x_{i-1}) (x_i-x_{i+1}) \dots (x_i-x_{n+1})}$}
  \end{center}

 Этот многочлен называется интерполяционным многочленом Лагранжа. $L_i(x_j)\bw=0$, $L_i(x_i)=1$. Тогда искомая функция будет иметь вид
 $$P(x)=\sum_{i=1}^{n+1}y_iL_i(x)$$

\subsection{Обратная матрица}
\label{matrixdet:reverse}

\begin{df}
	\emph{Обратной матрицей} $A^{-1}$ к матрице $A$ называется такая\index{матрица!обратная}
	матрица, что выполнено следующее равенство:
	$$ A^{-1}A=AA^{-1}=E$$%\quad\dim A=\dim A^{-1} $$
	Где $E$~--- единичная матрица.
\end{df}

\begin{note}
	Понятие обратной матрицы вводится в том случае, если элементы
	$A$~--- элементы некоторого поля.
\end{note}

%\begin{df}
%	Матрица называется \emph{вырожденной}, если ее определитель равен
%	нулю и называется \emph{невырожденной} в обратном случае.
%\end{df}

\begin{theorem}
	Если существует обратная матрица, то она единственна.
\end{theorem}
\begin{proof}
	Пусть существуют различные две матрицы $B_1$ и $B_2$, обратные к $A$.
	Рассмотрим произведение $B_1AB_2$. В силу ассоциативности умножения
	матриц, имеем:
	$$(B_1A)B_2=B_1(AB_2) \ra EB_2=B_1E \ra B_2=B_1$$
	Значит, они совпадают. Противоречие с тем, что $B_1\ne B_2$.
\end{proof}

\begin{theorem}
	Невырожденность матрицы является необходимым и достаточным условием\index{матрица!невырожденная}
	для существования обратной матрицы.
\end{theorem}
\begin{proof}[Необходимость]
	Доказательство проведем от обратного. Пусть $B$~--- обратная
	матрица к $A$ и $\det A=0$. Тогда, по определению, $AB=BA=E$. По теореме об
	определителе произведение (\ref{matrixdet:mul}), имеем равенство:

	$$\det AB=\det E \ra \det A\det B=\det E$$
	$$0\cdot\det B=1 \ra \det B=\frac10$$

	Такого в поле быть не может, поэтому предположение неверно.
\end{proof}
\begin{proof}[Достаточность]
	Для доказательства достаточности воспользуемся формулой обратной
	матрицы:

	$$
	A^{-1}=
	\frac1{\det A}\rbmat{
	A_{11} & \cdots & A_{1n} \\
	\vdots & \ddots & \vdots \\
	A_{n1} & \cdots & A_{nn}}^T=
	\frac1{\det A}\rbmat{
	A_{11} & \cdots & A_{n1} \\
	\vdots & \ddots & \vdots \\
	A_{1n} & \cdots & A_{nn}}
	$$

	Где $A_{ij}$~--- алгебраическое дополнение к элементу $a_{ij}$ (см.
	\ref{matrixdet:rowexpansion}).\index{алгебраическое дополнение}

	Остается проверить, что эта матрица действительно является обратной
	к $A$. Обозначим произведение $AA^{-1}=X$. Тогда:

	$$
	X=\frac1{\det A}
	\rbmat{
	a_{11} & \cdots & a_{1n} \\
	\vdots & \ddots & \vdots \\
	a_{n1} & \cdots & a_{nn}
	}
	\rbmat{
	A_{11} & \cdots & A_{n1} \\
	\vdots & \ddots & \vdots \\
	A_{1n} & \cdots & A_{nn}
	}
	=
	\frac1{\det A}
	\rbmat{
	x_{11} & \cdots & x_{1n} \\
	\vdots & \ddots & \vdots \\
	x_{n1} & \cdots & x_{nn}
	}
	$$

	Найдем элементы $x_{ij}$:

	$$
	\begin{array}{rclcl}
		x_{11}&=&a_{11}A_{11}+a_{12}A_{12}+\cdots+a_{1n}A_{1n}&=&\det A \\
		x_{12}&=&a_{11}A_{21}+a_{12}A_{22}+\cdots+a_{1n}A_{2n}&=&0 \\
		&\cdots&&\cdots& \\
		x_{21}&=&a_{21}A_{11}+a_{22}A_{12}+\cdots+a_{2n}A_{1n}&=&0 \\
		x_{22}&=&a_{21}A_{21}+a_{22}A_{22}+\cdots+a_{2n}A_{2n}&=&\det A \\
		&\cdots&&\cdots& \\
		x_{n(n-1)}&=&a_{n1}A_{(n-1)1}+a_{n2}A_{(n-1)2}+\cdots+a_{nn}A_{(n-1)n}&=&0 \\
		x_{nn}    &=&a_{n1}A_{n    1}+a_{n2}A_{n    2}+\cdots+a_{nn}A_{n    n}&=&\det A \\
	\end{array}
	$$

	Таким образом:

	$$
	X=\frac1{\det A}
	\rbmat{
	\det A & 0      & 0      & \cdots & 0      \\
	0      & \det A & 0      & \cdots & 0      \\
	0      & 0      & \det A & \cdots & 0      \\
	\vdots & \vdots & \vdots & \ddots & \vdots \\
	0      & 0      & 0      & \cdots & \det A
	}=
	\rbmat{
	1      & 0      & 0      & \cdots & 0      \\
	0      & 1      & 0      & \cdots & 0      \\
	0      & 0      & 1      & \cdots & 0      \\
	\vdots & \vdots & \vdots & \ddots & \vdots \\
	0      & 0      & 0      & \cdots & 1
	}=
	E
	$$

	Аналогично проверяется и произведение $A^{-1}A$.
\end{proof}

Обратная матрица легко вычисляется с помощью элементарных преобразований: будем приводить исходную матрицу к единичной, применяя одновременно те же преобразования к единичной матрице. В тот момент, когда из исходной мы получим единичную, из единичной мы получим обратную.\index{матрица!единичная}
 $$\left(A\mid E \right) \longrightarrow \left(E\mid A^{-1} \right)$$

\newpage
\subsection{Характеризация ранга матрицы в терминах миноров}\index{ранг!матрицы}\index{минор}

\begin{theorem}
  Ранг матрицы равен наибольшему порядку её отличных от нуля миноров
\end{theorem}
\begin{proof}
  Пусть ранг матрицы равен $r$. Тогда всякая система столбцов или строк, содержащая в себе больше, чем $r$ векторов, будет линейно зависима. Тогда и укороченные вектора будут линейно зависимы, то есть миноры порядков $r+1$ и больших будут равны нулю.

  Теперь покажем, что существует минор порядка $r$, отличный от нуля. Выберем в исходной матрице $r$ линейно независимых строки, ранг этой матрицы по строкам, а следовательно и по столбцам равен $r$. То есть мы можем выбрать в ней $r$ линейно независимых столбцов, тем самым получаем квадратную подматрицу порядка $r$, ранг которой равен $r$, то есть определитель которой не равен нулю.

  Доказательство завершено.
\end{proof}
%  Пусть $s$ -- максимальный порядок отличных от нуля миноров, а ранг матрицы равен $r$. Выберем в матрице минор $M\ne 0$ с порядком $s$. Тогда $s$ столбцов (строк) матрицы, проходящих через $M$ образуют линейно независимую систему (если укороченные векторы линейно независимы, то и полные векторы также линейно независимы). Таким образом мы показали, что $r \ge s$.
