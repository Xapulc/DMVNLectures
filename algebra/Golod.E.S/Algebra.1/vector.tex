\section{Векторная алгебра}
\label{vector}

\epigraph{He who seeks for methods without having a definite problem 
  in mind seeks in the most part in vain.}{David Hilbert}

\subsection{Основные понятия. Линейная комбинация векторов}

\begin{df}
  \emph{Вектор}\index{вектор} -- упорядоченный набор из чисел, называемых координатами вектора. Вектор -- элемент арифметического n-мерного \index{пространство}пространства $\R^n$. Над векторами определены следующие операции:
  \begin{enumerate}
    \item Сложение (означает почленное сложение его координат)
    \item Умножение на скаляр
  \end{enumerate}
\end{df}

Основные свойства операций:

\begin{enumerate}
  \item $\forall \ab,\bb\in\R^n\cln \ab+\bb=\bb+\ab$
  \item $\forall \ab,\bb,\cb \cln (\ab+\bb)+\cb=\ab+(\bb+\cb)$
  \item $\exists \mathbf 0\colon \ab+\mathbf 0=\ab;$
  \item $\forall \ab\cln\exists(-\ab)\colon \ab+(-\ab)=0$
  \item $\forall\al,\beta\in\R, \ab\colon (\al\beta)\ab=\al(\beta \ab)$
  \item $(\al+\beta)\ab=\al \ab+\beta \ab$
  \item $\al(\ab+\bb)=\al \ab+\al \bb$
  \item $\forall\ab\in\R^n\cln1\cdot\ab=\ab,\cln 0\cdot\ab=\mathbf 0$
\end{enumerate}

\begin{df}
  Пусть даны $\ab_1, \ab_2,\ab_3,\dots, \ab_m\in\R^n$ и $\al_1,\al_2,\dots\al_m\in\R$. Тогда выражение $\al_1\ab_1+\al_2\ab_2+\dots+\al_m\ab_m\in\R^n$ назовём \emph{линейной комбинацией}\index{линейная!комбинация} векторов $\ab_1, \ab_2,\dots, \ab_m$. Если $\al_1\ab_1+\al_2\ab_2+\dots+\al_m\ab_m=0$, то говорят, что задано \emph{линейное соотношение} этих векторов. Соотношение $0\cdot\ab_1+\dots+0\cdot\ab_m=0$ называется \index{соотношение!тривиальное}\emph{тривиальным}. \index{соотношение!нетривиальное}\emph{Нетривиальным} называется соотношение, в котором хотя бы один из коэффициентов не равен 0.
\end{df}
\begin{df}
  Под \emph{системой векторов} понимается индексированная совокупность векторов, т.е. в системе могут содержаться равные между собой вектора, но наделённые разными индексами.
\end{df}
\begin{df}
  Конечная система векторов называется \emph{линейно зависимой}\index{линейная!зависимость}, если для неё существует нетривиальное линейное соотношение. Если же существует только тривиальное линейное соотношение, то такая система называется \emph{линейно независимой}. Пустая система векторов линейно независима.
\end{df}
\begin{stm}
  Любая подсистема в линейно независимой системе линейно независима.
\end{stm}
\begin{stm}
  Система линейно зависима тогда и только тогда, когда хотя бы один вектор этой системы выражается через остальные.
\end{stm}
\begin{stm}
  Если какая-то подсистема системы зависима, то и вся система линейно зависима. \emph{Другими словами, любую линейно зависимую систему можно расширить}.
\end{stm}
\begin{stm}
  Если система векторов линейно зависима, то и система укороченных
  векторов линейно зависима.
\end{stm}
\begin{stm}
  В пространстве $\R^n$ всякая система, содержащая больше, чем $n$ векторов, линейно зависима.
\end{stm}
\begin{df}
  Говорят, что вектор $\bb$ линейно выражается через систему векторов $\{\ab_i\}$, если $\exists\al_1,\dots\al_n\colon \bb=\sum\limits_i \al_i\ab_i$.

  Говорят, что система $\Sb\subset\R^n$ линейно выражается через систему $\Tb\subset\R^n$, если каждый вектор из $\Sb$ линейно выражается через конечную подсистему в $\Tb$. Две системы векторов называются 
\emph{эквивалентными}, если каждая из них линейно выражается через другую.
\end{df}

\begin{theorem}[Основная лемма о линейной зависимости]
  \label{mainlemma}\index{лемма!о линейной зависимости}
  Пусть даны системы векторов $\Ab=\{\ab_1,\dots,\ab_m\},\Bb=\{\bb_1,\dots,\bb_k\}\in\R^n$ и система $\Bb$ линейно выражается через $\Ab$. Тогда если $k>m$, то $\Bb$ линейно зависима.
\end{theorem}
\begin{proof}
  Имеем систему уравнений:
  $$
   \left\{
    \begin{array}{rcl}
      \bb_1 &   =   & \sum\limits_i^m\al_{1i}\ab_i\\
          & \dots & \\
      \bb_k &   =   & \sum\limits_i^m\al_{ki}\ab_i\\
    \end{array}
   \right.
  $$
  
  Рассмотрим систему векторов $\{\mathbf\lambda_j\}$, в которой i-й вектор будет иметь координаты $(\al_{i_1},\dots\al_{i_m})$. Система этих векторов обязательно линейно зависима, потому что количество векторов в ней больше размерности пространства, которому они принадлежат. Таким образом мы всегда можем выбрать коэффициенты $\{ \mu_j \}$, так, чтобы $\sum_j^k \mu_j\mathbf\lambda_j = 0$. Понятно, что если мы возьмём линейную комбинацию $\sum_j^k \mu_j \bb_j$, то она тоже окажется равной 0. Теорема доказана.
\end{proof}
\begin{note}
  \emph{Другая формулировка этой леммы такова: <<Линейно независимую систему систему нельзя выразить через меньшее количество векторов, чем она содержит>>}.
\end{note}

\subsection{Базис системы векторов}\index{базис!системы векторов}

\begin{df}
  Пусть $\Sb\subset\R^n$ -- любая система векторов. Набор векторов $\ab_1,\dots \ab_r$ называется \emph{базисом} системы $\Sb$, если
  \begin{enumerate}
    \item $\ab_1,\dots \ab_r$ линейно независимы.
    \item $\forall \bb\in\Sb$ линейно выражается через $\ab_1,\dots \ab_r$.
  \end{enumerate}
\end{df}

\begin{stm}
  Базис -- максимальная линейно независимая подсистема. Всякую линейно
  независимую систему в $\Sb$ можно дополнить до базиса $\Sb$.
\end{stm}

\begin{stm}
  Любая система $\Sb\subset\R^n$ имеет базис. Стандартным базисом для $\R^n$ называется система векторов вида $\eb_i = (\underbrace{0,\dots,0}_{i-1},1,\underbrace{0,\dots,0}_{n-i})$\index{базис!стандартный}
\end{stm}

\begin{theorem}
  Любые 2 базиса $\Ab$ и $\Bb$ системы содержат равное количество векторов.
\end{theorem}
\begin{proof}
  Пусть подсистемы $\Ab = \{a_1\dots,a_n\}$ и $\Bb=\{b_1,\dots,b_m\}$
  являются базисами, тогда каждая из них линейно выражается через
  другую. По основной лемме о линейной зависимости (\ref{mainlemma})
  имеем, что $m\le n $ и $m\ge n$, т.е. $m=n$.
\end{proof}

\begin{df}
  \emph{Рангом}\index{ранг!системы векторов} системы векторов называется число векторов любом её базисе. Обозначается $\rk\Sb = r$. Например, $\rk\R^n=n$.
\end{df}

\begin{theorem}
  Конечная подсистема $(a_1,\dots, a_r)$ системы векторов $\ab$ является базисом её базисом в том и только том случае, если всякий вектор из $\ab$ выражается через $(a_1,\dots, a_r)$ единственным образом.
\end{theorem}
\begin{proof}
  Докажем необходимость и достаточность:
  \begin{itemize}
    \item[($\Leftarrow$)]  Пусть $(a_1,\dots, a_r)$ -- базис и допустим, что имеются два представления какого-то вектора $b\in\ab$:
    \begin{gather*}
     b=\lambda_1a_1+\lambda_ra_r\\
     b=\lambda_1'a_1+\lambda_r'a_r
    \end{gather*}
    Тогда $0=b-b=(\lambda_1-\lambda_1')a_1+(\lambda_r-\lambda_r')a_r\ra\forall i\colon\lambda_i=\lambda_i'$ так как $(a_1,\dots, a_r)$ линейно независима и для неё может существовать только тривиальная линейная комбинация.
    
    \item[($\Rightarrow$)] Предположим, что векторы из $\ab$ единственным образом выражаются через систему $(a_1,\dots, a_r)$. Чтобы установить, что подсистема $(a_1,\dots, a_r)$ является базисом, нужно показать, что $(a_1,\dots, a_r)$ -- линейно независима. Допустим обратное, т.е. имеется линейное соотношение
    $$\alpha_1a_1+\dots+\al_ra_r=0.$$
    Рассмотрим представление какого-то вектора $b\in\ab$:
    $$b=\lambda_1a_1+\dots+\lambda_ra_r$$
    В силу единственности представления вектора $b$ в виде линейной комбинации векторов $a_1,\dots, a_r$ получаем, что $\forall i\colon\lambda_i=\lambda_i+\al_i\ra \al_i=0$, т.е. $(a_1,\dots, a_r)$ линейно независима.
  \end{itemize}
\end{proof}

\begin{stm}
  Если $\Bb$ линейно выражается через $\Ab$, то $\rk \Bb\le\rk\Ab$. \emph{(Доказательство проводится по основной лемме о линейной зависимости (\ref{mainlemma}))}
\end{stm}

\subsection{Подпространства в $\R^n$}\index{пространство!подпространства}

\begin{df}
  Подмножество $\Lb\subset \R^n$ называется подпространством, если
  \begin{enumerate}
    \item $\Lb\ne\emptyset$
    \item $a,b\in\Lb\ra a+b\in\Lb$
    \item $a\in\Lb\ra \forall\lambda\in\R\cln\lambda a\in\Lb$
  \end{enumerate}
\end{df}

\begin{df}
  \index{плоскость}Плоскость в n-мерном пространстве есть множество векторов, полученное сдвигом какого-то подпространства на несущий вектор.
\end{df}

  \begin{df}
  Пусть имеется какая-то система векторов $\mathbf S=(a_1\dots a_s)\in\R^n$. Тогда множество всех линейных комбинаций векторов из $\mathbf S$ называется линейной оболочкой множества $\mathbf S$ и обозначается $<\!\mathbf S\!>$.
  $$<\!\mathbf S\!>=<a_1,\dots, a_r>=\{ \lambda_1a_1+\lambda_2a_2+\dots+\lambda_sa_s\mid\forall \lambda_i\in\R\}$$
  Линейная оболочка пустого множества $<\!\emptyset\!>$ равна нулевому вектору.\index{линейная!оболочка}
  \end{df}
  
  Сформулируем несколько следствий из определения:
  \begin{itemize}
    \item Линейная оболочка любого множества является подпространством
    \item Всякое подпространство является линейной оболочкой своего базиса.
    \item Условие, что система векторов $\bb$ линейно выражается через систему векторов $\ab$, равносильно тому, что $<\!\bb\!>\,\subseteq\,<\!\ab\!>$.
    \item Две системы векторов эквивалентны тогда и только тогда, когда из линейные оболочки совпадают.
  \end{itemize}
  
  Задание подпространства линейной оболочкой какой-либо системы векторов является только одним из возможных вариантов. Вторым способом является задание подпространства множеством решений ОСЛУ (см. \ref{oslu}). В данном случае можно, конечно, считать, что подпространство задаётся линейной оболочкой её фундаментальной системы решений\index{фундаментальная система решений} (далее: ФСР).
  
  \begin{df}
    В случае разговора о подпространствах, вместо термина <<ранг>>\index{ранг!подпространства} подпространства употребляют термин \index{размерность}<<размерность>> подпространства. Обозначается $\dim \Lb$. Размерность подпространства равна его рангу как системы векторов.
  \end{df}
