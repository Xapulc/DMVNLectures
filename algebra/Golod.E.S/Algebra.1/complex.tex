\newpage
\section{Комплексные числа}
\label{complex}
%
%\epigraph{Honesta mors turpi vita potior\footnotemark}{Tacit}
%\footnotetext{Честная смерть лучше позорной жизни}

\subsection{Построение поля комплексных чисел}

Полем комплексных чисел $\Cbb$ будем называть поле, которое удовлетворяет следующим свойствам:\index{поле!комплексных чисел}
\begin{enumerate}
  \item $\R\subset\Cbb$
  \item $\exists i\in\Cbb\colon i^2=-1$
  \item $\R\subset\Lb\subset\Cbb\ra\Lb=\R$ или $\Lb=\Cbb$
\end{enumerate}

Существует несколько различных моделей (реализаций) поля комплексных чисел:
\begin{enumerate}
  \item Евклидова плоскость. Элементы поля -- векторы на этой плоскости с началом в точке $(0;0)$.
  \item $\R^2$ -- множество пар двух действительных чисел, записывается как $(a;b)$ или $a+bi$.
  \item $\left\{\left(\begin{matrix}a&-b\\ b&a\end{matrix}\right)\in\Mb_2(\R) \right\}$ -- множество матриц такого вида.
\end{enumerate}

Во всех трёх случаях операция сложения над элементами поля нам уже понятна, операция умножения для случая 3 тоже определена нами, откуда следует её определение и для случая 2. Умножение в случае 1 будет определено позже.

Необходимо проверить, что построенное нами множество с двумя операциями является полем. Удобнее всего это делать в случае реализации поля комплексных чисел как множества квадратных матриц специального вида.

Также необходимо проверить выполнение дополнительных условий на поле $\Cbb$, наложенных нами. Покажем только выполнение условия 3 в этом случае: пусть $\Lb\supset\R\ra \exists z\in\Lb\setminus\R$, но так как $\Lb\subset\Cbb$, то $z$ представимо в виде $z=a+bi$. Так как $z\not\in\R\ra b\ne0$, то мы имеем право переписать в виде $i=(z-a)b^{-1}\in\Lb$. Имеем, что $\forall x,y\cln z=x+iy$ является элементом $\Lb\ra\Lb=\Cbb$.

\subsection{Тригонометрическая форма. Формула Муавра}

Рассматривая комплексное число как вектор с координатами $(a,b)$, можно сказать, что образует некоторый угол с положительным направлением оси $Ox$.

\begin{df}
\emph{Аргументом} комплексного числа $z$, $\arg z$, называется число $\varphi\bw\in[0,2\pi)$, что $z = x+iy = |z|\cos\varphi+i|z|\sin\varphi$.
\end{df}

Обычно $|z|$ обозначается как $\rho$, и комплексное число записывается в форме $z =\bw \rho(\cos\varphi+i\sin\varphi)$

\begin{ex}
$z = 2+2i = \sqrt{2}( \cos{\frac{\pi}{4}} + i\sin{\frac{\pi}{4}} )$
\end{ex}

С тригонометрической формой записи связана и показательная (экспоненциальная) форма записи комплексного числа: $z = \rho(cos\varphi+i\sin\varphi) = \rho e^{i\varphi}$.

\begin{ex}$e^{i\pi} = -1$\end{ex}

Рассмотрим произведение двух комплексных чисел, записанных в тригонометрической форме. Пусть 

$$z_1 = \rho_1(\cos{\varphi_1}+i\sin{\varphi_1})$$
$$z_2 = \rho_2(\cos{\varphi_2}+i\sin{\varphi_2})$$

Тогда
\begin{multline*}z_1z_2 = \rho_1\rho_2(\cos{\varphi_1}+i\sin{\varphi_1})(\cos{\varphi_2}+i\sin{\varphi_2}) = \dots=\\ =\dots= \rho_1\rho_2\left(\cos(\varphi_1+\varphi_2)+i\sin(\varphi_1+\varphi_2)\right)\end{multline*}

Из этого соотношения вытекает, что 
$$|z_1z_2|=|z_1||z_2|,\qquad\arg(z_1z_2)=\arg z_1+\arg z_2$$

Аналогично можно получить, что 
$$ \frac{z_1}{z_2} = \frac{\rho_1}{\rho_2}\left(\cos(\varphi_1-\varphi_2)+i\sin(\varphi_1-\varphi_2)\right)$$

\begin{stm}
Из формулы для умножения двух комплексных чисел в тригонометрической форме следует, что
$$z^n=\rho^n(\cos n\varphi+i\sin n\varphi)$$
Эта формула называется формулой Муавра.\index{формулы!Муавра}
\end{stm}

Из формулы Муавра можно вывести формулу для корня n-ной степени из комплексного числа, также называемой формулой Муавра.

Пусть 
$\sqrt[n]{z} = \omega \lra \omega^n\bw=z$,\  $z=\rho(\cos\varphi+i\sin\varphi)$,\ $\omega=\sigma(\cos\alpha+i\sin\alpha)$.

Тогда
$$\omega^n=\sigma^n(\cos n\alpha+i\sin n\alpha)=\rho(\cos\varphi+i\sin\varphi)=z$$
$$\Updownarrow$$
$$
 \left\{
  \begin{array}{lcl}
   \rho&=&\sigma^n\\
   \cos n\alpha&=&\cos\varphi\\
   \sin n\alpha&=&\sin\varphi
  \end{array}
 \right.
$$ 
$$\Downarrow$$
$$
 \left\{
  \begin{array}{lcl}
   \sigma&=&\sqrt[n]{\rho}\\
   n\alpha&=&\varphi+2\pi k,\cln k\in\mathbb Z
  \end{array}
 \right.
$$
То есть существует $n-1$ различных корней n-ной степени из $z$.
$$\sqrt[n]{z}=\sqrt[n]{\rho}(\cos \frac{\varphi+2\pi k}{n}+i\sin \frac{\varphi+2\pi k}{n}),\cln k\in\{0,1,\dots ,n-1\} $$

\subsection{Корни из единицы в поле комплексных чисел}

Множество корней $n$-ной степени из 1 в поле комплексных чисел $\mu_n$ согласно формуле Муавра будет иметь вид $\{\cos(2\pi k/n)+i\sin(2\pi k/n)\mid k=0,1,\dots,n-1\}$. Устройство множества корней $n$-ной степени из 1 в поле комплексных чисел тесно связано со многими другими алгебраическими понятиями.

\begin{df}
  \emph{Первообразным} корнем $n$-ной степени из 1\index{первообразный из единицы} называется такое число, которое не является корнем из 1 никакой меньшей степени. Можно показать, что $\ep_k$ является первообразным корнем $n$-ной степени из 1 тогда и только тогда тогда, когда $(k,n)=1$.
  
  Число первообразных корней определяется функцией Эйлера $$\varphi(n)=|\{k\mid 1\le k<n, (k,n)=1\}|.$$ Если $n=p_1^{k_1}\cdot\dots\cdot p_s^{k_s}$, где $p_i$ -- различные простые числа, то $$\varphi(n)=(p_1-1)p_1^{k_1-1}\cdot\dots\cdot (p_s-1)p_s^{k_s-1}.$$\index{функция!Эйлера}
\end{df}

\subsection{Единственность поля $\Cbb$}

\begin{theorem}
  Пусть $\Pb$ -- поле, содержащее $\R$ и такой элемент $j$, что $j^2=-1$. Тогда отображение $f\colon \Cbb\to\Pb$, при котором $f(a+bi)=a+bj$ является гомоморфизмом, и это единственный гомоморфизм, при котором поле $\R$ отображается тождественно, а $i$ переходит в $j$.
\end{theorem}
\begin{proof}
  Очевидно, что $f(z_1+z_2)=f(z_1)+f(z_2)$, так же легко проверить, что $f(z_1z_2)=f(z_1)f(z_2)$. Единственность очевидна.
\end{proof}

То есть из этого следует, что всякое поле, удовлетворяющее исходным условиям, изоморфно полю комплексных чисел, причём существует изоморфизм, тождественный на поле действительных чисел. Поле $\Cbb$ имеет ровно два автоморфизма, тождественных на $\R$: один тождественный, а второй переводит $i$ в $-i$ (комплексное сопряжение).
