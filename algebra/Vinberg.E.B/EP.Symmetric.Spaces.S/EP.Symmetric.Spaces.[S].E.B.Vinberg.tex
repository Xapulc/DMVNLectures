\documentclass[a4paper]{article}
\usepackage[simple,utf]{dmvn}

\title{Программа спецкурса <<Симметрические пространства>>\\
(Для желающих сдавать 1/2 года)}
\author{Лектоp Э.\,Б.\,Винберг}
\date{2006 г.}

\begin{document}
\maketitle

\begin{nums}{-2}
\item Критерии существования и единственности инвариантной римановой метрики на однородном пространстве группы
Ли. Определение симметрического пространства. Критерий симметричности однородного риманова многообразия в терминах
группы и подгруппы.
\item Примеры симметрических пространств: <<элементарные>> пространства, грассмановы многообразия, многообразия
максимальных евклидовых подпространств псевдоевклидова пространства, пространства положительно определённых
симметричных матриц, групповые пространства.
\item Накрытия симметрических пространств. Существование и единственность односвязного накрытия. Симметрические
алгебры Ли.
\item Разложение односвязного симметрического пространства в прямое произведение евклидова пространства и
неприводимых полупростых симметрических пространств. Неприводимые симметрические пространства компактного и
некомпактного типов, двойственность между ними.
\item Эрмитовы симметрические алгебры Ли и эрмитовы симметрические пространства. Вложение эрмитова симметрического
пространства некомпактного типа в двойственное ему пространство. Классификация эрмитовых симметрических
пространств.
\item Первая вариация длины кривой и геодезические на римановом многообразии. Полные римановы многообразия.
Геодезические симметрического пространства.
\item Параллельный перенос векторов на римановом многообразии. Гауссова кривизна двумерного риманова многообразия.
Параллельный перенос векторов в симметрическом пространстве.
\item Коммутатор векторных полей на многообразии и производная Ли. Ковариантное дифференцирование на римановом
многообразии, его свойства. Формула ковариантного дифференцирования в симметрическом пространстве.
\item Тензор кривизны риманова многообразия, свойства его симметрии. Вычисление тензора кривизны симметрического
пространства.
\item Уравнение Якоби для деформации геодезической на римановом многообразии. Связь тензора кривизны двумерного
риманова многообразия с гауссовой кривизной.
\item Геодезические в точке и вполне геодезические подмногообразия риманова многообразия. Кривизна по двумерному
направлению. Вполне геодезические подмногообразия симметрического пространства.
\end{nums}

\medskip\dmvntrail
\end{document}
