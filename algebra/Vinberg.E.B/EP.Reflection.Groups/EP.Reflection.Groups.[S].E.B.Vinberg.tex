\documentclass[a4paper]{article}
\usepackage[simple,utf]{dmvn}

\title{Программа спецкурса <<Дискретные группы отражений>>}
\author{Лектоp Э.\,Б.\,Винберг}
\date{2005--2006 г.}

\begin{document}
\maketitle

\begin{nums}{0}
\item Векторная модель пространства Лобачевского. Гиперплоскости,
      полупространства и отражения в пространствах постоянной кривизны.
      Взаимное расположение двух гиперплоскостей.
\item Общая теория дискретных групп отражений: их камеры, порождающие
      элементы и определяющие соотношения, стабилизаторы точек. Разложение
      произвольной дискретной группы движений в полупрямое произведение
      подгруппы отражений и некоторой группы симметрий ее камеры.
\item Абстрактные группы Кокстера. Схемы Кокстера. Каноническое линейное
      представление группы Кокстера.
\item Решение проблемы равенства слов в группе Кокстера. Стандартные
      подгруппы группы Кокстера.
\item Классы сопряженных отражений в группе Кокстера.
\item Построение дискретной группы отражений по любому многограннику Кокстера.
\item Критерий конечности абстрактной группы Кокстера.
\item Тупоугольные системы векторов и строение остроугольных многогранников
      в евклидовом пространстве.
\item Эллиптические и параболические схемы Кокстера, их классификация.
\item Системы корней, их группы Вейля, камеры Вейля, системы простых корней,
      матрицы Картана и схемы Дынкина. Дуальная система корней.
      Классификация систем корней и их явная реализация в классических случаях.
\item Расширенные группы Вейля, ограниченные камеры Вейля, расширенные
      системы простых корней, расширенные схемы Дынкина.
\item Инварианты конечных линейных групп. Теорема Гильберта об инвариантах.
      Топологический и алгебраический факторы, их сравнение (в комплексном случае).
\item Теорема Шепарда Тодда Шевалле. (\emph{Доказательство в любую сторону по желанию.})
\item Остроугольные многогранники и остроугольные семейства полупространств.
\item Теорема Перрона Фробениуса (для симметричных матриц). Существование
      и единственность неразложимого невырожденного остроугольного
      многогранника с заданной матрицей Грама.
\item Простота остроугольного многогранника и описание его комбинаторного
      строения в терминах матрицы Грама.
\item Квазиограниченные выпуклые многогранники в пространстве Лобачевского,
      их невырожденность и неразложимость. Характеризация
      квазиограниченных многогранников как многогранников конечного объема.
\item Модулярная группа Клейна $\PSL(2,\Z)$ как подгруппа индекса~2 в дискретной
      группе отражений на плоскости Лобачевского.
\item Дискретные группы движений пространства постоянной кривизны как
      дискретные подгруппы группы всех движений. Фундаментальная область
      дискретной группы движений, ее существование и инвариантность ее
      объема и свойства ограниченности.
\item Алгоритм для нахождения камеры дискретной группы отражений.
\item Арифметические дискретные группы (простейшего типа) в пространстве
      Лобачевского. Нахождение фундаментального многогранника для группы
      $\Ob(n,1,\Z)$ целочисленных преобразований Лоренца при $n\le9$.
\end{nums}

\medskip\dmvntrail
\end{document}
