\smallskip
\hrule
\rule{0pt}{20pt}
\marginpar{15.09.03}

Если на группе $(M, *)$ введена конгруэнция $"\sim"$, то существует естественный сюръективный
гомоморфизм $\pi : (M,*)\rightarrow (M/\sim, *),$
при котором $a\mapsto [a].$ При этом $\pi(a*b)=[a*b]=[a]*[b]=\\=\pi(a)*\pi(b),$ то есть это действительно гомоморфизм.
\par\de Сюръективный \gmm называется эпиморфизмом.
\par Рассмотрим левое разложение группы \rr{G} по подгруппе $\rr{H}\subseteq\rr{G}$ \ \ ---\ \ $\rr{G}=\bigcup\limits_{a\in\rr{G}}a\rr{H}$. Этому
разложению соответствует отношение эквивалентности $a\sim b\lra a^{-1}b\in \rr{H}$ (аналогичное правое разложение отвечает отношению $a\sim b\lra ba^{-1}\in\rr{H}$).
\par\de Подгруппа $\rr{H\subseteq G}$ называется нормальной (инвариантной, нормальным делителем), если для $\forall a\in\rr{G}$ выполнено $a\rr{H}=\rr{H}a$ (это равенство понимается в смысле множеств, то есть оно
равносильно условию $\forall a\in\rr{G},\  \forall h_1\in\rr{H}\ \ \exists h_2\in\rr{H} : ah_1=h_2a$). Обозначение того, что подгруппа \rr{H} нормальна в \rr{G}:\ \ $\rr{H}\lhd\rr{G}.$
\par Если группа абелева, то любая подгруппа в ней нормальна. Также в любой группе существуют тривиальные нормальные подгруппы: $\{e\}\lhd\rr{G},\ \ \rr{G}\lhd\rr{G}.$
Если в группе нет иных нормальных подгрупп, кроме тривиальных (например, в группе простого порядка), то такая группа называется простой.
\par\de Два элемента группы $x,y\in\rr{G}$ называются сопряженными при помощи элемента $g\in\rr{G},$ если $y=gxg^{-1}\ (=x^g$ --- другое обозначение).
Если $g$ умножить на любой элемент из центра группы (то есть на элемент, коммутирующий со всеми элементами группы), то полученный элемент тоже будет сопрягать $x$ и $y$.
\par Заметим, что отношение сопряженности является отношением эквивалентности:
\par $x\sim y \lra \ \exists g\in\rr{G} : y=gxg^{-1}.$ Действительно,
\begin{enumerate}
    \item $x\sim x : x=exe^{-1};$
    \item $x\sim y\ra y\sim x : y=gxg^{-1}\ra x=g^{-1}yg=g^{-1}y(g^{-1})^{-1};$
    \item $x\sim y,\ y\sim z\ra x\sim z : \ \exists g_1\in \rr{G} : y=g_1xg_1^{-1},\ \exists g_2\in\rr{G} : z=g_2yg_2^{-1}\ra z=(g_2g_1)x(g_2g_1)^{-1}.$
\end{enumerate}
\par Данное отношение эквивалентности определяет разбиение всей группы \rr{G} на непересекающиеся классы эквивалентности сопряженных между собой элементов.
Эти классы называются классами сопряженных элементов. Это разбиение важно для изучения структуры любой группы.
\par Рассмотрим $\rr{Aut G}$ --- автоморфизмы группы \rr{G}. Они образуют подгруппу в группе подстановок $\rr{S_G}$ (произведение автоморфизмов есть автоморфизм,
обратное отображение к автоморфизму снова есть автоморфизм, тождественное отображение также является автоморфизмом). Таким образом,
$\rr{Aut G}\subset\rr{S_G}$ --- группа автоморфизмов группы \rr{G}.
\par Теперь фиксируем произвольный элемент $g\in\rr{G}$ и рассмотрим отображение $I_g : \rr G\rightarrow \rr G,\\ I_g : x\mapsto gxg^{-1}.$
Покажем, что это отображение является автоморфизмом:
\begin{enumerate}
    \item $I_g$ --- биекция.
    Действительно, это сюръекция: $I_g(g^{-1}xg)=x\ \forall x\in\rr{G};$ это также инъекция: $I_g(x)=I_g(y)\ra gxg^{-1}=gyg^{-1}\ra x=y$ (умножим слева на $g^{-1}$, справа на $g$);
    \item Отображение $I_g$ сохраняет операции : $I_g(xy)=g(xy)g^{-1}=(gxg^{-1})(gyg^{-1})=I_g(x)I_g(y).$
\end{enumerate}
\par Таким образом, $I_g\in \rr{Aut G}.$ Такие автоморфизмы (то есть автоморфизмы вида $I_g$) называются внутренними. Для абелевых групп понятие внутреннего автоморфизма несодержательно
(любой внутренний автоморфизм тождествен). Если $\rr{H\subseteq G}$ --- подгруппа, то $I_g(\rr H)=g\rr Hg^{-1},$ но образ подгруппы при гомоморфизме является подгруппой,
поэтому $g\rr Hg^{-1}=\{ghg^{-1}\,\vrule\ h\in \rr H\}\subseteq\rr G$ --- подгруппа, она называется подгруппой, сопряженной с \rr H при помощи элемента $g\in\rr G.$
Множество внутренних автоморфизмов группы \rr G обозначается $\rr{Int G}\subseteq\rr{Aut G}.$ Покажем, что это множество является подгруппой в $\rr{Aut G}:$
\begin{enumerate}
    \item $\mathcal{E}=I_e\in\rr{Int G};$
    \item $\left(I_{g_1}I_{g_2}\right)(x)=(g_1g_2)x(g_1g_2)^{-1}=I_{g_1g_2}(x)\ \ \forall x\in\rr G\ra I_{g_1}I_{g_2}=I_{g_1g_2}\in\rr{Int G};$
    \item $I_gI_{g^{-1}}=I_{gg^{-1}}=I_e=\mathcal{E}\ra \left(I_g\right)^{-1}=I_{g^{-1}}\in \rr{Int G}.$
\end{enumerate}
\par Таким образом, мы доказали, что $\rr{Int G}\subseteq\rr{Aut G}$ --- подгруппа. Рассмотрим нормальные подгруппы.
$\rr H\lhd\rr G\lra\forall a\in \rr G\ a\rr H=\rr Ha$ --- это исходное определение. Другие характеризации нормальной подгруппы:
$\rr H\lhd \rr G\lra\forall a\in\rr G\ \ a\rr Ha^{-1}=\rr H\lra$\rr H совпадает с любой своей сопряженной подгруппой$\lra$\rr H инвариантна относительно всех
внутренних автоморфизмов группы $\rr G$, то есть $$\forall h\in \rr H,\ \forall a\in\rr G\ : aha^{-1}\in\rr H.$$
\par Также можно выписать одно свойство нормальной подгруппы: $\rr H\lhd \rr G\ra\rr H$ есть объединение некоторого количества классов сопряженных элементов (потому что в силу нормальности \rr H
не может быть, чтобы одна часть класса сопряженности принадлежала \rr H, а другая --- нет).
\te{Предложение}{Все конгруэнции на группе являются эквивалентностями, связанными с разложениями группы по нормальной подгруппе}.
\par\dok Надо доказать в обе стороны.
\begin{enumerate}
    \item Пусть $\rr H\lhd\rr G$, положим $a\sim b\lra a^{-1}b\in\rr H.$ Докажем, что ''$\sim$'' --- конгруэнция, то есть $a\sim a',\ b\sim b'\ra ab\sim a'b'.$
    Нам дано, что $a^{-1}a'\in\rr H,\ b^{-1}b'\in\rr H.$ Рассмотрим $(ab)^{-1}(a'b')=b^{-1}\underbrace{(a^{-1}a')}_{\in \rr H}b'=\underbrace{(b^{-1}(a^{-1}a')b)}_{\in\rr H}\underbrace{(b^{-1}b')}_{\in \rr H}\in\rr H,$ ведь
    $a^{-1}a',\ b^{-1}b'\in\rr H$ по условию, а $b^{-1}(a^{-1}a')b\in\rr H$ в силу нормальности подгруппы $\rr H.$\\
    Таким образом, отношение эквивалентности, связанное с разложением группы \rr G по подгруппе $\rr H\lhd\rr G$, является конгруэнцией. Классы эквивалентности
    при этой конгруэнции --- это смежные классы (в силу нормальности \rr H левые и правые смежные классы одного элемента совпадают) по нормальной подгруппе $\rr H.$
    Они образуют факторгруппу относительно естественной операции: $G/{}_\sim\,=\{a\rr H\,\vrule\,a\in\rr G\},$ операция: $(a\rr H)(b\rr H)=(ab)\rr H.$
    Единица факторгруппы --- это смежный класс единицы исходной группы, то есть сама подгруппа $\rr H=e\rr H.$ Данная факторгруппа обозначается $\rr G/\rr H$ и
    называется факторгруппой группы $\rr G$ по подгруппе $\rr H\lhd\rr G.$
    \item Обратно: Пусть ''$\sim$'' --- конгруэнция, тогда должна существовать нормальная подгруппа $\rr H\lhd \rr G$, такая, что $a\sim b\lra a^{-1}b\in\rr H.$
    Обозначим $\rr H=\{a\in\rr G\,\vrule\ a\sim e\}.$ Докажем, что это и есть искомая подгруппа. Это действительно подгруппа:
    \begin{enumerate}
        \item $e\in \rr H : e\sim e;$
        \item $a, b\in \rr H\lra a\sim e,\ b\sim e\ra ab\sim e^2=e\ra ab\in \rr H;$
        \item $a\in\rr H\lra a\sim e,\mbox{ также }a^{-1}\sim a^{-1}\ra e=aa^{-1}\sim ea^{-1}\ra a^{-1}\in\rr H.$
    \end{enumerate}
    Теперь докажем, что $\rr H$ нормальна в \rr G. Рассмотрим $\ \forall h\in\rr H\lra h\sim e.$ Рассмотрим также $\ \forall g\in\rr G, \ g\sim g,\ g^{-1}\sim g^{-1}\ra
    ghg^{-1}\sim geg^{-1}=e\ra ghg^{-1}\in\rr H\ra \rr H\lhd G.$ Теперь осталось только показать, что отношение ''$\sim$'' действительно совпадает с
    отношением эквивалентности, соответствующем разложению группы $\rr G$ по нормальной подгруппе \rr H.\\ В самом деле,
    $a\sim b,\ a^{-1}\sim a^{-1}\lra a^{-1}a\sim a^{-1}b\lra a^{-1}b\sim e\lra a^{-1}b\in\rr H.$\qquad\qed
\end{enumerate}

Таким образом, мы доказали, что все конгруэнции на группе исчерпываются отношениями эквивалентности, связанными с разложениями группы по нормальным подгруппам. Теперь
докажем еще одно предложение:
\te{Предложение}{Ядра \gmm{ов} и только они являются нормальными подгруппами}.
\par\dok \begin{enumerate}
    \item Пусть $\phi : \rr G\rightarrow\rr L$ --- \gmm групп, и $\rr H=\rr{Ker}\phi.$ Уже доказано, что $\rr H\subseteq\rr G$ --- подгруппа. Теперь проверим, что
    $\rr H\lhd\rr G$. Фиксируем $h\in\rr H$ и $g\in\rr G$, и рассмотрим $\phi(ghg^{-1})=\phi(g)\phi(h)\phi(g)^{-1}=\phi(g)e'\phi(g)^{-1}=e'\in\rr L$ --- единица группы \rr L, поэтому
    $ghg^{-1}\in\rr H.$ Значит, $\rr H=\rr{Ker}\phi\lhd\rr G.$
    \item Обратно, пусть $\rr H\lhd\rr G.$ Рассмотрим естественный эпиморфизм $\pi : \rr G\rightarrow \rr G/\rr H, $ при котором $a\mapsto a\rr H.$
    Тогда имеем $a\in\rr{Ker}\pi\lra\pi(a)=a\rr H=\pi(e)=\rr H\lra a\in\rr{H}\ra\rr{Ker}\pi=\rr H.$\qquad\qed
\end{enumerate}
\ste{Теорема}{о \gmm{ах} групп}{Пусть $\phi : \rr G\rightarrow \rr L$ --- эпиморфизм групп, $\rr H\lhd \rr G$ --- его ядро, $\pi : \rr G\rightarrow \rr G/\rr H$ --- естественный эпиморфизм.
Тогда существует изоморфизм $\psi : \rr L\rightarrow \rr G/\rr H,$ такой, что следующая диаграмма отображений коммутативна:
$$
\begin{array}{ccc}
   &  & \rr G \\
   & \swarrow \lefteqn{\pi}& \downarrow \lefteqn{\phi} \\
  \rr G/\rr H & \stackrel{\psi}\longleftarrow  & \rr L, \\
\end{array}
$$
то есть $\pi=\psi\phi.$
}

\dok Рассмотрим $x\in\rr L,$ для него существует такой $a\in\rr G$, что $x=\phi(a).$ Определим $\phi : \rr L\rightarrow \rr G/\rr H,$ при котором $\psi : x=\phi(a)\mapsto a\rr H.$
При таком определении сначала надо доказать корректность данного отображения. Действительно, пусть $\phi(a)=\phi(b)=x$ для некоторых $a,b\in\rr G.$
Но $\phi(a)=\phi(b)\lra\phi(a^{-1}b)=e'$ --- единице группы $\rr L,\lra a^{-1}b\in\rr H\lra a\rr H=b\rr H,$ что и означает корректность.

Далее, $\psi$ --- биекция. Действительно, сюръективность очевидна, ведь на всякий смежный класс $a\rr H$ что-то отобразилось. Докажем инъективность $\phi$.
Рассмотрим: $$\psi(\phi(a))=\psi(\phi(b))\lra a\rr H=b\rr H\lra\\\lra a^{-1}b\in\rr H\ra\phi(a^{-1}b)=e'=\phi(a)^{-1}\phi(b)\ra \phi(a)=\phi(b).$$
Таким образом заключаем, что $\psi$ --- биекция. Осталось теперь доказать сохранение операций. Действительно, $$\psi(\phi(a)\phi(b))=\psi(\phi(ab))=(ab)\rr H=(a\rr H)(b\rr H)=\psi(\phi(a))\psi(\phi(b)),$$
что означает, что $\psi$ --- изоморфизм. Что касается коммутативности диаграммы, то рассмотрим $\forall a\in\rr G,\ \pi(a)=a\rr H,\ \psi(\phi(a))=(\psi\phi)(a)=a\rr H=\pi(a).$\qquad\qed

Из доказанной выше теоремы вытекает, что гомоморфные образы группы $\rr G$ с точностью до изоморфизма совпадают с факторгруппами группы \rr G.

Данная теорема очень помогает в решении задач. Рассмотрим простой пример. Доказать, что $\rr{SL}(n, k)\lhd\rr{GL}(n, k)$ и вычислить $\rr{SL}(n, k)/\rr{GL}(n, k).$
Решение выглядит так:
\\рассмотрим $\phi : \rr{GL}(n, k)\rightarrow k^*,$ при котором $A\mapsto \det A.$ Очевидно, что $\phi$ --- эпиморфизм. $A\in\rr{Ker}\phi\lra\det A=\phi(A)=1\lra A\in\rr{SL}(n, k).$
Таким образом, мы видим, что $\rr{SL}(n, k)\lhd\rr{GL}(n, k)$ и  $\rr{SL}(n, k)/\rr{GL}(n, k)\cong k^*.$

Пусть $\phi : \rr G\rightarrow \rr L$ --- эпиморфизм. Оказывается, существует жесткое соответствие подгруппами \rr L и подгруппами \rr G. Если $\rr U\subseteq \rr L$ --- подгруппа, то
полный прообраз $\rr{Coim U}=\phi^{-1}(\rr U)\subseteq G$ --- тоже подгруппа.
Действительно, возьмем $x, y\in \rr{Coim U}\lra \phi(x), \phi(y)\in\rr U\ra \phi(x^{-1}y)=\phi(x)^{-1}\phi(y)\in\rr U,$ так как \rr U --- подгруппа.
Но это значит, что $x^{-1}y\in \rr{Coim U},$ то есть что $\rr{Coim U}\subseteq G$ --- действительно подгруппа.
Также для любой подгруппы $\rr U\subseteq\rr L$ выполнено $\rr{Ker}\phi\subseteq\rr{Coim U},$ поскольку $e\in \rr U,$ а $\rr{Ker}\phi=\rr{Coim}\{e\}.$
\smallskip
\hrule
\rule{0pt}{10pt}
\marginpar{22.09.03}

Пусть $\phi : \rr G\rightarrow \rr L$ --- эпиморфизм групп. Пусть $\rr{Ker}\phi=\rr H\lhd \rr G.$ Подгруппу $A\subseteq\rr G$ будем называть выделенной
данным эпиморфизмом, если она содержит его ядро, то есть $\rr H\subseteq A$ (это не общепринятое обозначение, но оно удобно для краткости изложения). Тогда справедлива следующая теорема:
\ste{Теорема}{о соответствии подгрупп при эпиморфизме}{Сопоставление $\theta$ выделенной подгруппе $A\subseteq\rr G$ ее образа $\rr{Im}\phi=\phi(A)$ в $\rr L$
определяет биекцию между выделенными подгруппами в $\rr G$ и всеми подгруппами в \rr H. При этом соответствующие подгруппы одновременно нормальны, а
факторгруппы по ним изоморфны.}

\dok \begin{enumerate}
    \item $\theta$ --- биекция. Во-первых, это очевидно сюръекция, так как прообраз $\rr{Coim}\rr U$, где $\rr U\subseteq\rr L$ --- подгруппа, является
    выделенной подгруппой в \rr G (по прошлой лекции). Докажем инъективность этого соответствия. Рассмотрим $A, B\subseteq\rr G$ --- выделенные подгруппы.
    Пусть $\phi(A)=\phi(B).$ Рассмотрим $\forall a\in A$, и поскольку образы совпадают, $\exists b\in B : \phi(a)=\phi(b).$ Но это означает, что $\phi(b^{-1}a)=e'$ --- единице группы \rr L.
    Поэтому $b^{-1}a\in\rr{Ker}\phi=\rr H\ra a=bh,$ где $b\in B, h\in\rr H\subseteq B$ $\ra a\in B.$ Это значит, что $A\subseteq B.$ Из симметричных соображений следует, что
    $B\subseteq A,$ поэтому $A=B.$ Из этого следует, что $\theta$ --- биекция.
    \item Одновременная нормальность соответствующих подгрупп. Пусть $A\, \lhd\, \rr G$ --- выделенная подгруппа. Значит, $\forall g\in\rr G\  \ gAg^{-1}=A\lra
    \phi(g)\phi(A)\phi(g)^{-1}=\phi(g)\rr U\phi(g)^{-1}=\rr U,$ где $\rr U=\phi(A).$ Но в силу того, что $\phi$ --- эпиморфизм, когда $g$ пробегает всю группу \rr G,
    $\phi(g)$ пробегает всю группу \rr L, поэтому одновременная нормальность соответствующих подгрупп доказана.
    \item Изоморфизм факторгрупп по соответствующим нормальным подгруппам. Рассмотрим отображение $\psi : \rr G\rightarrow \rr L/\rr U,$ где $A\lhd G,\ \rr U=\phi(A),$ при котором
    $\phi : g\mapsto\phi(g)\rr U.$ Сюръективность отображения очевидна в силу эпиморфности $\phi$: когда $g$ пробегает всю группу \rr G,
    $\phi(g)$ пробегает всю группу \rr L. Докажем сохранение операции. Рассмотрим
    $$
    \psi(g_1g_2)=\phi(g_1g_2)\rr U=\left(\phi(g_1)\phi(g_2)\right)\rr U=\left(\phi(g_1)\rr U\right)\left(\phi(g_2)\rr U\right)=\psi(g_1)\psi(g_2).
    $$
    Таким образом мы видим, что $\psi$ --- эпиморфизм групп. $g\in\rr{Ker}\psi\lra\psi(g)=\phi(g)\rr U=\rr U$ --- единица факторгруппы $\rr L/\rr U.$ Но
    $\phi(g)\in \rr U\lra g\in A,$ так как $\rr U=\phi(A).$ По теореме о гомоморфизмах групп, $\rr G/A\cong\rr L/\rr U.$\quad\qed
\end{enumerate}

\te{Следствие}{Пусть $\rr L\lhd\rr K\lhd\rr G,\ \rr L,\rr K\lhd\rr G$ --- короткая башня нормальных подгрупп. Очевидно, что $\rr G/\rr L\rhd \rr K/\rr L.$
Тогда $\frac{\displaystyle \rr G/\rr L}{\displaystyle \rr K/\rr L}\cong \rr G/\rr K$ (мнемоническое правило о понижении этажности дроби, состоящей из факторгрупп).}

\dok Рассмотрим естественный эпиморфизм $\pi : \rr G\rightarrow \rr G/\rr L,$ при котором $\pi : a\mapsto a\rr L.$ $\rr{Ker}\,\pi=\rr{L}, \ \ \pi(\rr K)=\rr K/\rr L$
\ ($\pi(\rr K)$ --- смежные классы, порожденные элементами из \rr K). Поэтому $\rr K$ и $\rr K/\rr L$ --- соответствующие друг другу при $\theta$ подгруппы.
По пункту 3 теоремы они одновременно нормальны, и факторгруппы по ним изоморфны. \quad\qed
\smallskip

\centerline{\bf Факторкольца.}
\smallskip

Кольцо --- это множество $(A, +, \cdot)$, при этом обе операции связаны дистрибутивностью (левой и правой), по сложению $(A, +)$ --- абелева группа, а
на умножение не накладывается никаких условий.

Конгруэнция кольца. Рассмотрим ''$\sim$'' --- отношение эквивалентности. Оно называется конгруэнцией, если согласовано с обеими операциями кольца:
$$
a\sim a',\ b\sim b'\ra a+b\sim a'+b',\ ab\sim a'b'.
$$
Если на кольце $A$ задана конгруэнция, то можно определить две естественные операции на множестве классов: $[a]+[b]=[a+b]$ и $[a][b]=[a+b].$
Естественная операция сложения будет удовлетворять закону дистрибутивности относительно естественного умножения:
$$[c]\left([a]+[b]\right)=[c][a+b]=[c(a+b)]=[ca+cb]=[ca]+[cb]=[c][a]+[c][b],$$
и аналогично с правой дистрибутивностью. Поэтому $A/{}_{\sim}$ относительно естественных операций образует кольцо, оно называется факторкольцом по
конгруэнции ''$\sim$''. Исходное кольцо и его фактор по конгруэнции одновременно ассоциативны, коммутативны, etc.

\de Идеал кольца --- это абелева подгруппа $(I, +)$ в его аддитивной группе, для которой верно, что $$\forall a\in A:\
aI, Ia\subseteq I,\mbox{ то есть } \forall a\in A,\ \forall x\in I\ :\ ax\in I,\ xa\in I.
$$

Идеал является подкольцом --- действительно, он замкнут относительно умножения. Также наряду с двусторонним идеалом можно говорить о
левом или правом идеале, а для коммутативного кольца односторонние идеалы являются двусторонними.

На самом деле, все конгруэнции на кольце --- это отношения эквивалентности, связанные с разложением аддитивной группы кольца по идеалу, это мы докажем чуть позже.
Определим отношение: $a\sim b\lra a-b\in I$ (где $I\subseteq A$ --- подгруппа в аддитивной группе кольца). Оно является конгруэнцией относительно сложения, это вытекает из того, что уже доказано про группы.
Но если $I$ --- идеал кольца, то ''$\sim$'' --- конгруэнция относительно умножения, действительно, $a\sim a',\ b\sim b'\ra a-a'\in I,\ b-b'\in I\ra ab-a'b'=ab-ab'+ab'-a'b'=a(b-b')+(a-a')b'\in I,$
поскольку $b-b'\in I\ra a(b-b')\in I,$ так как $I$ --- идеал кольца, также $a-a'\in I\ra (a-a')b'\in I$, поэтому $ab\sim a'b',$ а значит, ''$\sim$'' --- конгруэнция.
Таким образом, отношения, связанные с разложением аддитивной группы кольца по идеалу, являются конгруэнциями. Обратное доказательство абсолютно аналогично доказательству
про группы, то есть все конгруэнции на кольце соответствуют отношениям эквивалентности, связанными с разложением $(A, +)$ по идеалу $I$. Рассмотрим $(A, +)/I$, это
факторгруппа, но классы эквивалентности в ней можно естественным образом перемножать, поэтому относительно
двух операций $A/I$ --- факторкольцо кольца $A$ по идеалу $I$.

\te{Пример}{}
Пусть $I\lhd\mathbb{Z}$ --- идеал кольца целых чисел. Тогда $(I, +)\lhd(\mathbb{Z}, +),$ а $(\mathbb{Z}, +),$ в свою очередь, является циклической группой, поэтому $(I, +)$ --- тоже циклическая группа.
Значит, $\exists n\in\mathbb{N} : I=n\mathbb Z.$ Обратно, подкольца вида $n\mathbb Z$ все являются идеалами, очевидно. Факторкольцо $\mathbb Z/n\mathbb{Z}\cong\mathbb Z_n,$ действительно,
$m_1\sim m_2\lra m_1-m_2\in n\mathbb Z\lra m_1\equiv m_2 (\rr{mod}\ m).$ Каждый класс эквивалентности имеет вид $i+ n\mathbb Z,$ где $i=0,1,\dots, n-1,$ поэтому
$\mathbb Z/n\mathbb{Z}\cong\mathbb Z_n.$ Естественные операции в факторкольце выглядят следующим образом: $(i+n\mathbb Z)(j+n\mathbb Z)=(ij)+n\mathbb Z;\
(i+n\mathbb Z)+(j+n\mathbb Z)=(i+j)+n\mathbb Z.$

\de Отображение $\phi : A\rightarrow B,$ где $A, B$ --- кольца, называется \gmm{ом} колец, если оно сохраняет обе операции кольца, то есть:
\begin{enumerate}
    \item $\phi(a+b)=\phi(a)+\phi(b);$
    \item $\phi(ab)=\phi(a)\phi(b).$
\end{enumerate}
Гомоморфизм колец одновременно является гомоморфизмом их аддитивных групп, очевидно, и поэтому простейшие свойства \gmm{а} выполнены:
$\phi(0)=0;\ \phi(-a)=-\phi(a).$
Ядро гомоморфизма колец --- это его ядро как \gmm{а} аддитивных групп: $\phi : (A,+)\rightarrow (B,+),\ \rr{Ker}\phi=\{a\in A\,\vrule\,\phi(a)=0\}.$
\te{Предложение}{Ядра \gmm{ов} и только они являются идеалами кольца.}

\dok
\begin{enumerate}
    \item Пусть $\phi$ --- гомоморфизм колец $A$ и $B$, обозначим $I=\rr{Ker}\phi.$ Для $\forall x\in I,\ \forall a\in A$ из теорем о группах следует, что
    по сложению $(I,+)\subseteq(A,+)$ --- подгруппа. Проверим, что $I$ ''выдерживает'' умножение: $\phi(ax)=\phi(a)\phi(x)=\phi(a)\cdot 0=0\ra ax\in I.$
    Аналогично доказывается, что $xa\in I\ra I$ --- идеал кольца.
    \item Пусть $I$ --- идеал в кольце $A$, $I\lhd A.$ Рассмотрим естественный \gmm $$\pi : \rightarrow A/I,\ \ \pi : a\mapsto a+I,$$ $\ra\rr{Ker}\phi=I$, что следует из
    доказанного про группы.\qquad\qed
\end{enumerate}

\ste{Теорема}{о гомоморфизмах колец}{Пусть $\phi : A\rightarrow B$ --- эпиморфизм колец, $I=\rr{Ker}\phi,$ $\pi : A\rightarrow A/I$ --- естественный гомоморфизм.
Тогда существует изоморфизм $\phi : B\rightarrow A/I,$ такой, что следующая диаграмма отображений коммутативна:
$$
\begin{array}{ccc}
   &  & \rr G \\
   & \swarrow \lefteqn{\pi}& \downarrow \lefteqn{\phi} \\
  \rr G/\rr H & \stackrel{\psi}\longleftarrow  & \rr L, \\
\end{array}
$$
то есть $\pi=\psi\phi$.}

\dok Поскольку все участвующие в теореме кольца являются абелевыми группами, а нами доказано, что для аддитивных групп колец изоморфизм $\psi$ существует,
и диаграмма отображений коммутативна. Заметим, что этот изоморфизм групп $\psi$ одновременно является изоморфизмом колец. Надо проверить, что $\psi$ сохраняет умножение.
Вспомним, какой изоморфизм $\psi$ мы рассматривали в теореме о гомоморфизмах групп. Мы определяли $\psi : B\rightarrow A/I,\ \psi : \phi(a)\mapsto a+I\ \forall a\in A,$ потому что
$\forall x\in B\ \ x=\phi(a)$ для некоторого $a\in A$ (как в теореме о группах). Мы доказали корректность $\psi$ и изоморфность его, проверим сохранение умножения.
Возьмем $a_1, a_2\in A\ra \phi(a_1), \phi(a_2)\in B.$ Применим $\psi$:
$$
\psi(\phi(a_1)\phi(a_2))=\psi(\phi(a_1a_2))=(a_1a_2)+I=(a_1+I)(a_2+I)=\psi(\phi(a_1))\psi(\phi(a_2))
$$
$\ra \psi$ --- действительно изоморфизм колец, а коммутативность диаграммы проверять не надо, так как она выполнена уже для аддитивных групп данных колец.\qquad\qed

Таким образом, мы получили внутреннее описание всех гомоморфных образов кольца --- это только его факторкольца. В любом кольце есть два тривиальных идеала:
$\{0\}$ и все кольцо. Если $P$ --- поле, то в нем нет нетривиальных идеалов: пусть $\{0\}\ne I\lhd P,$ рассмотрим $0\ne x\in I\ra x^{-1}x=e\in I\ra\forall a\in P,$ так как
$ae=a\in I\ra I=P.$ Кольцо, в котором нет нетривиальных идеалов, называется простым. Например, кольцо квадратных матриц над полем является простым (однако в нем
есть нетривиальные левые или правые идеалы). Рассмотрим идеалы кольца многочленов над полем $k[x]$.
Пусть пока $A$ --- коммутативное кольцо, фиксируем $x_1, x_2, \dots, x_r\in A$. Обозначим $I=\{a_1x_1+\dots+a_rx_r\,\vrule\, a_i\in A\}.$
На самом деле, $I\lhd A.$ Действительно, сумма $(a_1x_1+\dots+a_rx_r)+(b_1x_1+\dots+b_rx_r)=(a_1+b_1)x_1+\dots+(a_r+b_r)x_r\in I,$ произведение
$b(a_1x_1+\dots+a_rx_r)=(ba_1)x_1+\dots+(ba_r)x_r\in I\ra I\lhd A,$ так как $A$ --- коммутативное кольцо. Этот идеал называется идеалом, порожденным
элементами $x_1, \dots, x_r,$ и обозначается $I=(x_1, \dots, x_r).$

Если в предыдущем примере $r=1$, то $I=(x).$ Идеал, порожденный одним элементом, называется главным. Кольцо, в котором всякий идеал главный,
называется кольцом главных идеалов (КГИ). Например, $\mathbb Z$ --- это КГИ.
\te{Предложение}{$k[x],$ где $k$ --- поле, является кольцом главных идеалов.}

\dok Пусть $I\lhd k[x].$ Либо $I=\{0\}$, и это главный идеал, либо $I\ne\{0\}.$ Рассмотрим степени многочленов, входящих в $I$.
Существует $0\ne d(x)\in I$ минимально возможной степени. Докажем, что $I=(d(x))$. Рассмотрим $f(x)\in I,$ поделим: $f(x)=d(x)q(x)+r(x),$ где
либо $r(x)=0,$ либо $\rr{deg}\,r(x)<\rr{deg}\,d(x).$ Но тогда так как $r(x)=f(x)-d(x)q(x),$ где $f(x)\in I,\ d(x)\in I\ra r(x)\in I.$ Поэтому $r(x)=0,$
так как $d(x)$ был выбран минимально возможной степени, поэтому $f(x)\in (d(x)).$\quad\qed

Если $I=(d_1),\ I=(d_2)\ra d_1\,\vrule\,d_2$ и $d_2\,\vrule\,d_1\ra d_1\sim d_2$ (это обозначение для ассоциированных элементов кольца).
Таким образом, все порождающие главного идеала ассоциированы между собой, значит, в главном идеале кольца многочленов есть единственный порождающий его
многочлен со старшим коэффициентом, равным единице.

Пусть $I=(f, g)=(d(x))\ra d(x)=f(x)u(x)+g(x)v(x),$ таким образом $d(x)=\mbox{НОД}(f, g),$ и мы получили формулу ''$fu+gv$'' из первого семестра.
\smallskip
\hrule
\rule{0pt}{10pt}
\marginpar{29.09.03}

Вначале --- два замечания по прошлой лекции:
\begin{enumerate}
    \item Доказательство того, что $k[x]$ --- кольцо главных идеалов, годится и для $\mathbb Z$, поэтому $\mathbb Z$ тоже является кольцом главных идеалов.
    \item Если $\phi : F\rightarrow L$ --- гомоморфизм полей, то $\phi=0$ либо $\phi$ --- инъекция (изоморфное вложение). Действительно, обозначим $\rr{Ker}\phi=I\lhd F,$ но
    так как в поле нет нетривиальных идеалов, то либо $I=F\ra \phi=0,$ либо $I=\{0\},$ тогда $\phi$ --- инъекция.
\end{enumerate}

Рассмотрим факторкольца колец $\mathbb Z$ и $k[x].$ В $\mathbb Z$ всякий идеал главный, поэтому всякое факторкольцо имеет вид $\mathbb{Z}/n\mathbb Z\cong \mathbb Z_n.$ Факторкольцо
$\mathbb Z_n$ является полем тогда и только тогда, когда $n=p$ --- простое число.

$k[x]$ также является кольцом главных идеалов. Рассмотрим
факторкольцо $k[x]/(d(x)).$ Пусть
$d(x)=a_nx^n+a_{n-1}x^{n-1}+\dots+a_1x+a_0,\ a_n\ne 0,$ то есть
идеал $(d(x))$ нетривиален. Обозначим $I=(d(x)).$ Существует
инъекция $\phi : k\rightarrow A=k[x]/I,$ при котором $\phi :
\alpha\mapsto \alpha + I.$ Действительно, это отображение
ненулевое, так как $d(x)$ имеет положительную степень, поэтому
если $\alpha\ne 0,$ то $\alpha+ I$ --- ненулевой смежный класс.
Обозначим $\overline{f(x)}=f(x)+I.$ Докажем, что отображение
сохраняет операции. Рассмотрим:
$$\phi(\alpha+\beta)=(\alpha+\beta)+I=(\alpha +
I)+(\beta+I)=\phi(\alpha)+\phi(\beta);$$
$$
\phi(\alpha\beta)=\alpha\beta+I=(\alpha+I)(\beta+I)=\phi(\alpha)\phi(\beta).
$$
Таким образом, $\phi$ --- инъекция (по замечанию в начале лекции).
Обозначим $\ov k=\{\ov\alpha=\alpha+I\,\vrule\,\alpha\in k[x]\}.$
Это подполе, оно изоморфно полю $k$ как его гомоморфный образ при
ненулевом гомоморфизме. Мысленно отождествляя элементы $k$ и $\ov
k$ согласно гомоморфизму $\phi,$ мы получим, что $k\hookrightarrow
k[x]/(d(x)).$ Поэтому $k[x]/(d(x))$ может рассматриваться как
линейное пространство над $\ov k$, то есть над $k$.

Разделим с остатком: $f(x)=d(x)q(x)+r(x),$ где $r(x)=0$ или
$\rr{deg}\, r(x)<\rr{deg}\, d(x).$ Значит,
$$r(x)+(d(x)q(x)+I)=f(x)+I=\ov{f(x)}=r(x)+I=\ov{r(x)},$$ поскольку
$d(x)q(x)\in I.$ Значит, $\ov{f(x)}=\ov{r(x)},$ где либо $r(x)=0$,
либо $\rr{deg}\, r(x)<\rr{deg}\, d(x).$ Таким образом, всякий
смежный класс порождается многочленом степени строго меньше $n$,
поэтому $$\rr{dim}_k k[x]/(d(x))=\rr{deg}\, d(x)=n,$$ так как
можно явно указать базис: $(\ov 1, \ov x, \dots, \ov{x^{n-1}}).$
Это действительно базис $A=k[x]/(d(x))$ над $k$, поскольку каждый
смежный класс $\ov{f(x)}$ имеет степень порождающего $\rr{deg}\,
f(x)\le n-1,$ то есть
$\ov{f(x)}=\ov{\alpha_{n-1}x^{n-1}+\dots+\alpha_0},\ \alpha_i\in
k,$ что в свою очередь равно $\alpha_{n-1}\ov
x^{n-1}+\dots+\alpha_0\ov 1,$ ибо взятие смежного класса ---
естественный гомоморфизм. Значит, любой смежный класс выражается
через базис $(\ov 1, \ov x, \dots, \ov{x}^{n-1}).$ Еще надо
доказать линейную независимость степеней $\ov x$. Пусть
$$\alpha_0\ov 1+\dots+\alpha_{n-1}\ov x^{n-1}=\ov 0,$$ но тогда
$\ov{\alpha_{n-1}x^{n-1}+\dots+\alpha_01}=\ov 0\lra
\alpha_{n-1}x^{n-1}+\dots+\alpha_0\in I=(d(x))\lra
\alpha_{n-1}=\dots=\alpha_0,$ так как в $I$ либо многочлены
степени $\ge n$, либо нулевой многочлен. Поэтому $\rr{dim}_k
k[x]/(d(x))=\rr{deg}\, d(x)=n.$ Таким образом, факторкольцо
$k[x]/(d(x))$ всегда конечномерно. Когда оно является полем?
\te{Предложение}{Факторкольцо $A=k[x]/(d(x))$ является полем тогда
и только тогда, когда $d(x)$ --- неприводимый над $k$ многочлен.}

\dok \begin{enumerate} \item Пусть $d(x)$ --- приводим, докажем,
что факторкольцо $A$ не является полем.\\ Пусть
$d(x)=d_1(x)d_2(x),\ \rr{deg}\,d_1,\ \rr{deg}\,d_2<n=\deg d.$
 Тогда $$\ov{d_1(x)}=d_1(x)+I\ne\ov 0=I,\
\ov{d_2(x)}=d_2(x)+I\ne\ov 0=I.$$ Степени $d_1, d_2$ меньше $n$,
поэтому $d_1,d_2\notin I.$ Но
$\ov{d_1(x)}\cdot\ov{d_2(x)}=\ov{d_1(x)\cdot d_2(x)}=\ov{d(x)}=\ov
0=I$ (поскольку взятие смежного класса --- это гомоморфизм и так
как $d(x)\in I$).
 Таким образом, $A=k[x]/(d(x))$ не является полем.
 \item Пусть теперь $d(x)$ --- неприводим. Докажем, что $A$ --- поле.\\
Рассмотрим $\ov{f(x)}\ne\ov 0\lra f(x)$ не делится на $d(x)\lra
(d, f)=1$ так как $d(x)$ --- неприводим. Значит,
$1=f(x)u(x)+d(x)v(x)$ для некоторых $u, v.$
Возьмем смежные классы
левой и правой частей: $\ov f\ov u+\ov d\ov v=\ov 1\ra \ov f\ov
u=1\lra \ov u=(\ov f)^{-1}\ra $ все ненулевые элементы в $A$
обратимы.
    Поэтому $A$ является полем.\qquad\qed
\end{enumerate}

Надо заметить, что доказанная теорема переносится на $\mathbb Z$,
поэтому $$\mathbb Z_n\cong \mathbb Z/n\mathbb Z\mbox{ ---
поле}\lra n=p\mbox{ --- простое.}$$

Если $F\supset k$, где $F, k$ --- поля, то говорят, что $F$ ---
расширение поля $k$, или ''надполе''.\\ Пусть $p(x)$ ---
неприводим в $k[x],$ тогда $k\hookrightarrow {k[x]}/{(p(x))}$ ---
поле. Таким образом, факторкольцо $k[x]/(p(x))$ является
расширением поля $k$. Оно называется простым алгебраическим
расширением поля $k$.

Есть еще одно расширение поля $k$, которое мы строили в первом
семестре: $k\hookrightarrow k(x)$ --- поле рациональных дробей
поля $k$. Оно называется простым трансцендентным расширением поля
$k$, но его мы рассматривать не будем.

Возникает вопрос: выполняет ли простое алгебраическое расширение
поля $k$ те же задачи, что и поле $\mathbb C$ как расширение
$\mathbb R?$

Имеет ли $p(x)$ корни в $k[x]/(p(x))$? Этот многочлен в $k$ корней
не имеет, ведь над $k$ он неприводим. Но $p(\ov x)=\alpha_n\ov
x^n+\dots+\alpha_0\ov
1=\ov{\alpha_nx^n+\dots+\alpha_0}=\ov{p(x)}=0$ в $k[x]/(p(x)).$
Значит, $\ov x$ --- корень многочлена $p(x)$ в простом
алгебраическом расширении $k[x]/(p(x)).$ Однако это не дает нам
права утверждать, что в $k[x]/(p(x))$ содержатся все корни $p(x).$

\te{Предложение}{Пусть $F\hookleftarrow k$ --- расширение поля
$k$, содержащее корень $\theta\in F$ неприводимого над $k$
многочлена $p(x)\in k[x].$ Тогда существует инъекция
$k[x]/(p(x))\hookrightarrow F$. Говорят, что простое
алгебраическое расширение --- это поле, полученное из $k$
присоединением корня $\theta$ неприводимого многочлена $p(x)$.}

\dok Рассмотрим $\phi : k[x]\rightarrow F,$ при котором $\phi :
f(x)\mapsto f(\theta).$ Очевидно, что $\phi$ --- гомоморфизм (ведь
он сохраняет обе операции). Посмотрим, каково его ядро.
$f(x)\in\rr{Ker}\phi\lra f(\theta)=0\lra (f(x), p(x))\ne 1\ra
p(x)\,\vrule\,f(x),$ так как $p(x)$ неприводим. Значит,
$\rr{Ker}\phi=I=(p(x)).$ По теореме о гомоморфизмах колец,
существует инъекция $k[x]/(p(x))\hookrightarrow F.$\qquad\qed

Существует обозначение: образ $\rr{Im}\phi$, то есть подполе, на
которое отображается простое алгебраическое расширение
поля $k$, обозначается $k(\theta):$
$$
k[x]/(p(x))\cong k(\theta)\subseteq F.
$$

Из соображений размерности вытекает, что простое алгебраическое расширение не содержит собственных подполей в которых есть корень $\theta$ неприводимого
многочлена $p(x)$, ведь не бывает собственного подполя равной размерности с надполем.

Построим по этой теории поле $\mathbb C$ из поля $\mathbb R$. Рассмотрим $p(x)=ax^2+bx+c\in \mathbb R[x], \Delta=b^2-4ac<0.$
Пусть $\theta$ --- корень $p(x):\ \ p(\theta)=0,\ \theta\in\mathbb C.$ Рассмотрим отображение $\phi : \mathbb R[x]\rightarrow \mathbb C,$ при котором $\phi : f(x)\mapsto f(\theta).$
По доказанному предложению, получим инъекцию $\psi : \mathbb R[x]/(p(x))\hookrightarrow \mathbb C.$ Но $\phi(1)=1,\ \phi(x)=\theta$, 1 и $\theta$ линейно
независимы над $\mathbb R,\ \dim_\mathbb R\mathbb C=2\ra \rr{Im}\phi=\mathbb C.$ Поэтому $\psi$ --- изоморфизм, поэтому $\mathbb C$ --- это простое
алгебраическое расширение поля $\mathbb R:$
$$
\mathbb R[x]/(p(x))\cong\mathbb {C}.
$$

Пусть поле $k$ имеет два расширения: $F\supset k$ и $L\supset k.$
Пусть существует изоморфизм $\phi : F\rightarrow L.$ Будем
говорить, что $\phi$ --- изоморфизм над $k$, если и только если
$\forall \alpha\in k : \phi(\alpha)=\alpha.$ Например, изоморфизм
$\mathbb R[x]/(p(x))\rightarrow\mathbb {C}$ --- над $\mathbb R.$

Теперь фиксируем поле $k$ и многочлен $f(x)\in k[x],\ \deg
f(x)=n\ge 1.$ Над полем $k$ многочлен $f(x)$ может быть
неприводим, но рассмотрим простое алгебраическое расширение
$k\subset L_1$, полученное с помощью неприводимого множителя
многочлена $f(x).$ Пусть над полем $L_1$ $f(x)$ имеет корень
$\theta_1$: $f(x)=(x-\theta_1)g_1(x)$ --- это разложение над
$L_1.$ Если $g_1(x)$ раскладывается на неприводимые множители, то
разложим над $L_2$, которое является простым алгебраическим
расширением поля $L_1,$ получим
$f(x)=(x-\theta_1)(x-\theta_2)g_2(x).$ Продолжая процесс
расширения полей, получим, что $k\subset L_1\subset
L_2\subset\dots\subset L_r,$ и над последним полем будем иметь:
$f(x)=(x-\theta_1)(x-\theta_2)\dots(x-\theta_n).$ Поле $L_r$
получено из поля $k$ с помощью последовательности простых
алгебраических расширений. Однако поле $L_r$ однозначно
многочленом $f(x)$ не определяется. $\theta_1,\dots,\theta_n\in
L_r$ --- корни многочлена $f$: $f(\theta_i)=0\ \ \forall
i=1,2,\dots,n.$ В $L_r$ можно взять пересечение всех подполей,
содержащих все корни $\theta_i$ многочлена $f(x),$ это будет
подполе $E\subseteq L_r.$ Рассмотрим свойства поля $E$:
\begin{enumerate}
    \item $f(x)$ над $E$ разлагается на линейные множители;
    \item Не существует промежуточного подполя $F$, такого, что $k\subset F\subsetneq E,$ над которым $f(x)$ разлагается на линейные множители.
\end{enumerate}

\de Поле $E\supset k$ называется полем разложения многочлена $f(x),$ если для него выполнены условия 1 и 2, написанные выше
(такие подполя существуют, одно из них было только что построено).

\te{Теорема}{Поля разложения многочлена $f(x)\in k[x]$ изоморфны между собой над $k$.}

\dok Докажем теорему индукцией по степени полинома $f(x)$, $n=\deg f(x)\ge 1$.

Основание: $n=1.$ Тогда корень многочлена $f(x)$ есть уже в основном поле $k$, поэтому любое поле разложения $f(x)$ совпадает с $k$: $E=k$.

Индуктивный переход: пусть утверждение теоремы верно для полиномов степени $\le n-1,$ над любым полем. Пусть $E\supset k$ и $E'\supset k$ --- поля
разложения многочлена $f(x)\in k[x].$ Пусть $\theta\in E : p\,\vrule\,f,\ p(x)$ --- неприводим, $p(\theta)=0.$ Пусть также $\theta'\in E' : p\,\vrule\,f,\ p(x)$ --- неприводим, $p(\theta')=0.$
В поле $E$ есть подполе $L=k(\theta)\hookrightarrow E,\ k(\theta)\cong k[x]/(p(x))$ над $k$. Также $L'=k(\theta')\hookrightarrow E',\ k(\theta')\cong k[x]/(p(x))$ над $k$.
Поэтому $L\cong L'$ над $k$. Мысленно отождествим $L\equiv L',$ в силу этого изоморфизма над $k$ имеем $\theta=\theta'.$ Поэтому $E\supset L,\mbox{ и }E'\supset L$ после отождествления.
Многочлен $f(x)$ над $L$ разлагается на множители следующим образом: $f(x)=(x-\theta)g(x);\ x-\theta,\ g(x)\in L[x];\ \deg g(x)=n-1.$
Также заметим, что $E$ и $E'$ являются полями разложения многочлена $g(x)\in L[x],$ докажем это.
Рассмотрим $E$: $g$ разлагается над $E$ на линейные множители, так как $f$ разлагается. Докажем условие 2. Если бы существовало промежуточное поле $L\subset F\subsetneq E$, то
над ним многочлен $f(x)$ разлагался бы на линейные множители, так как $\theta\in L\subseteq F.$ Значит, $E$ --- не является полем разложения для $f$. Противоречие, которое показывает, что
$E$ --- поле разложения для многочлена $g$, аналогично $E'$ --- тоже поле разложения для $g$. Поэтому по индуктивному предположению, существует изоморфизм $E$ и $E'$ над
подполем $L$, но так как $k\subset L,$ то и над $k$ $E$ и $E'$ тоже изоморфны.\quad\qed

Вернемся теперь к полю $\mathbb C\supset \mathbb R.$ Если
$F\supseteq k$ --- алгебраически замкнутое расширение, то возьмем
$f(x)\in k[x],$ и пусть $E$ --- поле разложения для полинома
$f(x).$ Тогда существует инъекция $E\hookrightarrow F$ (так как
$F$ алгебраически замкнуто, то в $F$ лежат все корни $f(x)$.
Возьмем пересечение всех подполей в $F$, которые содержат корни
$f(x)$. Получим поле разложения (то есть подполе, над которым $f$
раскладывается на линейные множители, которое не содержит
собственных подполей с этим свойством)). Возьмем
$p(x)=ax^2+bx+c\in\mathbb R[x],\ \Delta=b^2-4ac<0.$ Рассмотрим
поле разложения этого многочлена $p(x)$. Оно по предыдущему
замечанию вкладывается в $\mathbb C:$ $E\hookrightarrow\mathbb C.$
В силу двумерности над $\mathbb R,$ $E=\mathbb C.$ Таким образом,
$\mathbb C$ --- поле разложения для всех квадратных трехчленов с
отрицательным дискриминантом.
\smallskip
\hrule
\rule{0pt}{10pt}
\marginpar{06.10.03}

Минимальные подполя --- это $\q$ и $\z_p=F_p$, где $p$ --- простое, так как в них нет собственных подполей.
Рассмотрим произвольное поле $k$. возможны два случая:
\begin{enumerate}
    \item $\rr{char}\,k=0,$ тогда существует инъекция $\q\hookrightarrow k$. Действительно, рассмотрим $\phi : \q\rightarrow k$, при котором $\phi : \frac mn\mapsto (me)(ne)^{-1}.$
    $\phi$, очевидно, сохраняет операции, значит, это инъекция (ведь у полей нет нетривиальных гомоморфизмов, кроме инъекций).
    \item $\rr{char\,}k=p$, а из первого семестра знаем, что $p$ --- всегда простое. Тогда существует инъекция $\z_p\hookrightarrow k.$ Действительно, рассмотрим $\phi : \z_p\rightarrow k,$ при котором $\phi : m\mapsto me.$
    $\phi,$ очевидно, сохраняет операции, то есть, это гомоморфизм. Рассмотрим его ядро. $$m\in\rr{Ker}\phi\lra\phi(m)=me=0\lra p\,\vrule\, m,$$ так как $p=O(e)$ в аддитивной группе поля $k$.
    Значит, $\rr{Ker}\phi=(p)\subset \z\ra$ по теореме о гомоморфизмах колец, существует инъекция $\z/(p)\hookrightarrow k,$ но $\z/(p)=F_p,$ поэтому все доказано.
\end{enumerate}

Рассмотрим расширение $F\supset k,$ пусть $\dim_kF<\infty$ (например, в качестве $F$ можно взять простое алгебраическое расширение поля $k$; напомним, что $\dim_kF$ в этом случае есть
степень неприводимого многочлена, порождающего идеал, фактор по которому --- данное расширение).
Обозначается $\dim_kF=[F:k].$

\de Элемент $\alpha\in F$ называется алгебраическим, если $\exists\  0\ne f(x)\in k[x],$ такой, что $f(\alpha)=0.$

Если расширение конечно (то есть конечномерно), то все его элементы являются алгебраическими.
Действительно, обозначим $m=[F:k],$ рассмотрим $\alpha\in F,$ и возьмем $m>n.$ Степени $\alpha^0=e,\alpha,\alpha^2,\dots,\alpha^n$ линейно зависимы над $k$.
Поэтому $\exists c_i\in k,\ i=0,\dots, n$ не все равные нулю, такие, что $c_0e+c_1\alpha+\dots+c_n\alpha^n=0,$ обозначая $f(x)=c_0+c_1x+\dots+c_nx^n\ne 0,$ имеем, что $f(\alpha)=0.$

\ste{Предложение}{о размерности короткой башни полей}{Пусть $F\supset L\supseteq k$ --- короткая башня полей, $[F:L]<\infty,\ [L:k]<\infty$. Тогда $[F:k]=[F:L][F:k]<\infty$.}

\dok Обозначим $x_1,\dots x_n$ --- базис $L$ над $k$, $y_1,\dots, y_m$ --- базис $F$ над $L$. Докажем, что $\{x_iy_j\}_{i=1,\dots, n;\ j=1,\dots, m}$ --- базис $F$ над $k$.
\begin{enumerate}
    \item Возьмем $z\in F.$ Тогда $z=\sum\limits_{j=1}^m\beta_jy_j,\ \beta_j\in L.$ Но так как $\beta_j=\sum\limits_{i=1}^n\alpha_{ij}x_i,\ j=1,\dots, m\ra$
    $$z=\sum\limits_{i=1}^n\sum\limits_{j=1}^m\alpha_{ij}x_iy_j,$$ таким образом, всякий элемент из $F$ линейно выражается через предполагаемый базис.
    \item Докажем линейную независимость. Пусть $\sum\limits_{i=1}^n\sum\limits_{j=1}^mc_{ij}x_iy_j=0,\ c_{ij}\in k.$ Рассмотрим $$\sum_{j=1}^m\left(\sum\limits_{i=1}^nc_{ijx_i}\right)y_i=0,$$ но
    так как $\sum\limits_{i=1}^nc_{ij}x_i\in L,$ а $y_1,\dots, y_m$ --- базис $F$ над $L$, то $$\sum\limits_{i=1}^nc_{ij}x_i=0\ \forall\,j=q,\dots,m.$$
    Но $x_1,\dots,x_n$ --- базис $L$ над $k,$ поэтому $c_ij=0\ \forall\,i,j.$ Поэтому элементы $\{x_iy_j\}$ линейно независимы, и это действительно базис.\qquad\qed
\end{enumerate}

Замечание из предложения: Если есть $f(x)\in k[x],$ можно взять $E\supset k$ --- его поле разложения, и из предложения следует, что $E$ --- конечное
расширение $k$, так как это последовательность простых алгебраических расширений $E\supset L_m\supset \dots\supset L_1\supseteq k,$ но $[L_i:L_{i-1}]<\infty\ra[E:k]<\infty.$

Пусть расширение $F\supset k$ алгебраически замкнуто. Обозначим $\ov k$ --- множество корней многочленов над $k$, лежащих в $F$, оно называется алгебраическим замыканием поля $k$.
\te{Предложение}{$\ov k$ --- алгебраически замкнутое подполе в $F$.}

\dok \begin{enumerate}
    \item $k$ --- подполе. Рассмотрим $\alpha_1,\dots, \alpha_s$ --- конечное число корней многочленов: $$f_1(\alpha_1)=0, \dots f_s(\alpha_s)=0,$$ где все $f_i(x)\in k[x].$
    Рассмотрим многочлен $f(x)=f_1(x)f_2(x)\dots f_s(x)\in k[x],$ и его поле разложения $F\supset E.$ $f(x)$ имеет $\alpha_1,\dots,\alpha_s$ корнями, значит,
    $\forall\,i,j : \alpha_i+\alpha_j;\ \alpha_i\alpha_j;\ \alpha^{-1}_i$ суть алгебраические элементы, то есть корни некоторых многочленов над $k$ (так как $[E:k]<\infty,$ и в силу доказанного ранее, это расширение состоит из алгебраических элементов).
    \\Поэтому $\ov k$ --- поле.
    \item Для доказательства алгебраической замкнутости надо доказать, что многочлен над $\ov k$ имеет корнем элемент из $\ov k$.
    Рассмотрим $g(x)=c_nx^n+\dots+c_1x+c_0\in\ov k[x].$ Так как $F$ алгебраически замкнуто, то $g$ в этом поле имеет корень, возьмем $\theta\in F : g(\theta)=0.$
    Так как $c_0,c_1,\dots c_n\in\ov k$, то они являются корнями полиномов из $k[x],$ и без ограничения общности можно считать, что $c_0,c_1,\dots c_n$ суть корни одного многочлена из $k[x].$
    Рассмотрим поле разложения этого многочлена $E\subset F,\ [E:k]<\infty.$ Можно считать, что $g(x)\in E[x].$ Существует инъекция простого алгебраического расширения поля $E$ ---
    $E(\theta)$ в $F$, и $[E(\theta):E]<\infty.$ Таким образом, $E(\theta)\supset E\supset k,\ [E:k]<\infty,\ [E(\theta):E]<\infty\ra[E(\theta):k]<\infty.$ Значит, любой элемент из $E(\theta)$ является
    алгебраическим, поэтому $\theta$ --- корень некоторого полинома из $k[x],$ то есть он лежит в $\ov k[x].$\qquad\qed
\end{enumerate}

Возьмем $\rrr\supset\q;$ $\ov\q$ --- алгебраические числа (корни
полиномов с целыми ($\lra$рациональными)
коэффициентами), по доказанному они образуют алгебраически
замкнутое подполе в $\rrr$. Но в $\rrr$ кроме алгебраических чисел
есть трансцендентные числа, это вытекает из соображений мощности.
\medskip

\centerline{\bf Конечные поля.}

В этом ''разделе'' будет дано полное описание конечных полей.

Рассмотрим $F_q$, по определению это конечное поле, содержащее $q$ элементов: $|F_q|=q.$
Всякое конечное поле имеет положительную характеристику, так как его аддитивная группа конечна. Обозначим $\rr{char\,}F_q=p.$
По тому, что было сказано в начале лекции, существует инъекция $F_p\hookrightarrow F_q.$ Значит, $[F_q:F_p]=n<\infty,$ так как оба поля конечны.
Пусть $x_1,\dots,x_n$ --- базис $F_q$ над $F_p.$ Тогда $\forall y\in F_q \ \ \exists !\mbox{ выражение } y=c_1x_1+\dots+c_nx_n,\ c_i\in F_p,\ i=1,\dots,n.$
Таких элементов $y\in F_q$ столько, сколько строчек $(c_1,\dots,c_n)\in F_p^n,$ то есть $p^n.$ Поэтому $|F_q|=p^n,$ то есть $q=p^n.$

Пусть $\rr{char\,}F_q=p,$ рассмотрим $x, y\in F_q.$ Тогда $(x+y)^p=x^p+y^p$ в $F_q$ (это понятно по биному Ньютона). Ясно также, что $(xy)^p=x^py^p,$ поэтому
возведение в степень $p$ является автоморфизмом $F_q.$ Значит, $(x+y)^{p^k}=x^{p^k}+y^{p^k}$ --- это $k-$кратное применение автоморфизма.
$|F_q^*|=p^n-1,$ значит, по теореме Лагранжа о группах, порядок любого элемента $F_q^*$ делит $p^n-1,$ то есть $\forall a\ne 0,\ a\in F_q : a^{p^n-1}=1.$ Это значит, что
$a^{p^n}=a.$ Рассмотрим многочлен $f(x)=x^{p^n}-x\in\z_p[x].$ Все элементы поля $F_q$ и только они являются корнями этого полинома, так как у него $\le p^n$ корней, а в поле $p^n$ элементов.
Значит, $F_q$ --- поле разложения полинома $f(x)$, ведь оно содержит все его корни, а меньших подполей с таким свойством нет, так как $f(x)$ имеет ровно $|F_q|$ корней.

Таким образом, все конечные поля, содержащие одинаковое количество элементов, изоморфны между собой (как поля разложения одного и того же многочлена над одним и тем же полем).
Теперь осталось предъявить $F_{p^n}$ для всякого простого $p$ и натурального $n$.

Возьмем простое $p$ и натуральное $n$, и рассмотрим многочлен $f(x)=x^{p^n}-x\in F_p[x].$ Формальная производная
$$
f'(x)=p^nx^{p^n-1}-1=-1\mbox{\ \ в\ \ }F_p[x].
$$
Таким образом, $f(x)$ не имеет кратных корней ни в каком расширении поля $F_p,$ так как $(f, f')=1.$
Пусть $E$ --- поле разложения многочлена $f(x),$ оно существует по доказанному ранее. Это поле определено однозначно с точностью до изоморфизма, в нем лежат все корни $f(x),$ и нет промежуточных подполей с этим свойством.
Докажем, что все корни $f(x)$ образуют подполе в $E$. Действительно, пусть $\alpha$ и $\beta$ --- корни $f(x),$ то есть $\alpha^{p^n}=\alpha,\ \beta^{p^n}=\beta.$ Тогда $(\alpha+\beta)^{p^n}=\alpha^{p^n}+\beta^{p^n}=\alpha+\beta,$ так как
поле имеет характеристику $p$. Также $(\alpha\beta)^{p^n}=\alpha^{p^n}\beta^{p^n}=\alpha\beta$ и $(\alpha^-1)^{p^n}=(\alpha^{p^n})^{-1}=\alpha^{-1}.$ Значит,
$f(\alpha+\beta)=0;\ f(\alpha\beta)=0;\ f(\alpha^{-1})=0,$ то есть все корни $f$ образуют подполе в $E,$ в котором $f$ имеет все корни, поэтому, в силу минимальности поля
разложения, $E=\{\mbox{все корни }f(x)\}$. Но так как $|E|=p^n,$ то $E=F_{p^n},$ поэтому поле $F_{p^n}$ существует.

Мы доказали теорему:
\te{Теорема}{Для всякого простого $p$ и всякого натурального $n$ существует единственное с точностью до изоморфизма поле, содержащее $p^n$ элементов.}\qquad\qed

\te{Предложение}{Существует вложение полей $F_{p^d}\hookrightarrow F_{p^n}\lra d\,\vrule\, n.$}

\dok \begin{enumerate}
    \item Пусть $F_{p^d}\hookrightarrow F_{p^n}.$ Обозначим $r=[F_{p^n}:F_{p^d}]$. Имеем короткую башню полей $F_{p^n}\supset F_{p^d}\supset F_{p},$ причем
    $[F_{p^n}:F_{p^d}]=r,\ [F_{p^d}:F_{p}]=d,\ [F_{p^n}:F_{p}]=n\ra n=dr\ra d\,\vrule\, n.$
    \item Пусть $d\,\vrule\,n\ra n=dr$ для некоторого $r$. Рассмотрим $p^n-1=p^{dr}-1=(p^d)^r-1=(p^d-1)m$ для некоторого $m$ (по известной школьной формуле).
    Теперь рассмотрим
    $$
    x^{p^n}-x=x\left(x^{p^n-1}-1\right)=x\left(x^{(p^d-1)m}-1\right)=xg(x)\left(x^{p^d-1}-1\right)=g(x)\left(x^{p^d}-x\right)
    $$
    для некоторого многочлена $g(x).$ Таким образом, из того, что $d\,\vrule\,n$, следует, что среди корней полинома $x^{p^n}-x$ содержатся корни $x^{p^d}-x,$ то есть, что
    поле $F_{p^n}$ содержит $F_{p^d}$ в качестве подполя.\qquad\qed
\end{enumerate}

В следующий раз мы докажем, что мультипликативная группа конечного поля всегда является циклической, а сейчас с помощью этого факта  установим

\te{Предложение}{Над конечным полем существуют неприводимые многочлены любой степени, большей или равной единице}.

\dok Пусть $\rr{char\,}F_q=\rr{char\,}F_{p^n}=p.$ Тогда из предыдущей леммы (о цикличности мультипликативной группы поля $F_{p^n}$) следует, что $F_{p^n}^*=\lob{\theta}_{p^n-1}.$
Рассмотрим $\phi : F_p[x]\rightarrow F_{p^n},$ которое переводит $\phi : f(x)\mapsto f(\theta).$ Заметим, что $\phi$ --- сюръективный гомоморфизм (сюръективный, так как все ненулевые элементы $F_{p^n}$ --- это
степени $\theta$). Пусть $\rr{Ker}\phi=I=(d(x))\lhd F_p[x]$ (это главный идеал), а по теореме о гомоморфизмах колец $F_{p}[x]/(d(x))\cong F_{p^n}.$
Так как образ $\phi$ --- поле, то $d(x)$ неприводим над $F_p$. Так как $\dim_{F_p}F_{p^n}=n,$ то $\deg f(x)=n,$ это было установлено на прошлой лекции.
Значит, для всякого $n\ge 1$ над $F_p$ существуют неприводимые многочлены степени $n$.\qquad\qed
