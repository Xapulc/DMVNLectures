\documentclass[a4paper]{article}
\usepackage[simple,utf]{dmvn}

\title{Программа экзамена по высшей алгебре}
\author{Лектор Виктор Николаевич Латышев}
\date{1, 3 семестры, 2002--2003 г.}

\begin{document}
\maketitle

\section*{1 семестр}

\begin{nums}{-2}
\item Решение СЛУ методом исключения неизвестных, общее решение,
исследование различных случаев.
\item Связь между решениями СЛУ и решениями ее присоединённой однородной СЛУ.
\item Линейная зависимость строк, различные определения, основная лемма о линейной зависимости.
\item База и ранг системы строк.
\item Совпадение рангов системы строк и системы столбцов в любой матрице.
\item Критерий совместности (Теорема Кронекера Капелли) и критерий определенности в терминах
рангов матриц.
\item Алгоритм вычисления ранга и отыскания базы в конечной системе строк, использующий
элементарные преобразования строк матрицы.
\item Фундаментальная система решений однородной СЛУ.
\item Плоскости арифметического пространства как решения СЛУ.
\item Группа подстановок конечного множества, знак подстановки, разложение в произведение
транспозиций.
\item Определитель квадратной матрицы и его основные свойства: линейность по строкам, косая
симметрия по строкам, условие нормировки. Вычисление определителя приведением к треугольному виду.
\item Неизменность определителя при транспонировании его матрицы.
\item Критерий равенства определителя нулю.
\item Определитель матрицы с углом нулей.
\item Формула разложения определителя по строке (столбцу). Теорема о фальшивом разложении.
\item Теорема Крамера о системах линейных уравнений с квадратной матрицей.
\item Определитель Вандермонда. Интерполяционная теорема Лагранжа.
\item Теорема о ранге матрицы.
\item Действия над матрицами и их свойства.
\item Теорема о ранге произведения двух матриц.
\item Определитель произведения квадратных матриц.
\item Обратная матрица, её явный вид (формула обратной матрицы через алгебраические дополнения),
способ вычисления с помощью элементарных преобразований строк.
\item Основные алгебраические структуры группы, кольца, поля. Определения, их простейшие
следствия, понятие подструктуры, изоморфизмы. Примеры.
\item Кольцо вычетов, случай простого модуля. Характеристика поля.
\item Поле комплексных чисел, геометрическое изображение, алгебраическая и тригонометрическая
формы записи, извлечение корней, корни из единицы.
\item Теорема Гаусса об алгебраической замкнутости поля комплексных чисел.
\item Кольцо многочленов от одной переменной, формальное и функциональное равенство
многочленов.
\item Возможность и единственность деления на ненулевой многочлен с остатком. Теорема Безу и её
следствия. Схема Горнера.
\item НОД двух многочленов, его выражение через сами многочлены. Алгоритм
Евклида.
\item Факториальность кольца многочленов и кольца целых чисел.
\item Неприводимые многочлены над полем комплексных и действительных чисел.
\item Формальная производная многочлена, ее свойства. Понижение кратности неприводимого
множителя (корня) при дифференцировании, освобождение от кратных корней.
\item Формула Лагранжа и метод Ньютона для интерполяционного многочлена.
\item Формулы Виета для корней многочлена.
\item Границы корней многочлена, грубая локализация, формула Тейлора для многочлена, признак
Ньютона верхней границы положительных корней.
\item Метод Штурма для отделения вещественных корней многочлена.
\item Представление правильной рациональной дроби в виде суммы простейших дробей. Случаи~$\R$
и~$\Cbb$.
\item Существование и единственность представления симметрического многочлена в виде многочлена
от элементарных многочленов Виета.
\item Результант двух многочленов, его выражение через многочлены и их корни. Дискриминант
многочлена.
\item Теорема Декарта.
\end{nums}
\pagebreak

\section*{3 семестр}

\begin{nums}{-3}
\item Теорема Кэли о представлении группы подстановками.
\item Гомоморфизм групп (колец), его простейшие свойства, ядро и образ
гомоморфизма, образ и прообраз подгруппы (подкольца), примеры.
\item Системы порождающих группы, циклические группы, теорема о
подгруппах циклической группы.
\item Теорема об изоморфизме циклических групп одинакового порядка.
\item Разложение группы по подгруппе, теорема Лагранжа и её следствия.
\item Нормальные подгруппы и конгруэнции группы, факторгруппа.
\item Теорема о гомоморфизмах групп.
\item Теорема о соответствии подгрупп при эпиморфизме и её следствие
о понижении этажности <<дроби>>, составленной из факторгрупп.
\item Идеалы и конгруэнции кольца, факторкольцо. Теорема о гомоморфизмах колец. Теорема
о том, что $\Z$ и $K[x]$ являются КГИ.
\item Простое алгебраическое расширение поля, его размерность над исходным полем и свойство
минимальности.
\item Поле разложение многочлена, его единственность с точностью до изоморфизма и свойство
минимальности.
\item Формула размерности короткой башни полей. Поле алгебраических чисел, его алгебраическая
замкнутость.
\item Условия вложимости конечных полей.
\item Цикличность мультипликативной группы конечного поля. Существование
над конечным полем неприводимого многочлена наперёд заданной
степени.
\item Прямые произведения групп и подгрупп, условие, при котором произведение нормальных
подгрупп является прямым. Теорема о факторизации по прямым множителям.
\item Свободные абелевы группы конечного ранга: эквивалентность
различных определений, инвариантность ранга.
\item Теорема о подгруппах свободной абелевой группы, согласованные базы.
\item Существование примарного разложения конечно порождённой абелевой группы.
\item Единственность  примарного разложения конечно порождённой абелевой группы с
 точностью до изоморфизма.
\item Нетривиальность центра конечной $p$-группы, её разрешимость.
\item Теорема существования силовской $p$-подгруппы.
\item Теорема о том, что всякая $p$-подгруппа содержится в некоторой силовской
 $p$-подгруппе и что все силовские $p$-подгруппы сопряжены.
\item Теорема о том, что количество силовских  $p$-подгрупп сравнимо с 1 по модулю $p$.
\item Линейные представления групп: приводимость, полная приводимость, гомоморфизмы
и изоморфизмы, матричные представления и их эквивалентность, лемма Шура.
\item Теорема Машке о полной приводимости представления конечной группы.
\item Описание неприводимых комплексных представлений конечно
порождённых абелевых групп.
\item Вложение неприводимого представления в регулярное. Конечность числа неприводимых
представлений конечной группы.
\item Разрешимость группы верхнетреугольных матриц.
\item Простота знакопеременной группы $\Ab_5$.
\item Малая теорема Ферма. Явное выражение для функции Эйлера.
\item Китайская теорема об остатках.
\item Теорема Вильсона.
\item Кратность неприводимого представления группы.
\item Совпадение кратности и размерности неприводимого подпредставления
группы в её регулярном представлении. Следствие для оценки числа
неприводимых представлений группы.
\end{nums}

\medskip\dmvntrail
\end{document}
