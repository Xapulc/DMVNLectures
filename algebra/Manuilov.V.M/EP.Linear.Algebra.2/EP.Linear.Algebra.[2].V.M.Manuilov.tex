\documentclass[a4paper]{article}
\usepackage[simple,utf]{dmvn}

\title{Программа экзамена по линейной алгебре}
\author{Лектор Владимир Маркович Мануйлов}
\date{II семестр, 2005--2006~г.}

\begin{document}
\maketitle

\centerline{\scriptsize Различия программ 2006 и 2005 года минимальны, хотя и имеются. Для полноты картины старая программа оставлена.}

\section*{Программа 2006 года}

\begin{nums}{-2}
\item Линейное пространство. Определение, примеры. Линейная оболочка. Аффинное пространство.
\item Линейная (не)зависимость системы векторов. Ранг системы векторов. Размерность. Базис. Координаты.
Зависимость координат от базиса.
\item Подпространство. Факторпространство. Теорема о сумме размерностей подпространства и факторпространства.
\item Пересечение и сумма подпространств. Теорема об их размерностях. Прямая сумма двух и более подпространств.
Внешняя прямая сумма.
\item Двойственное пространство. Двойственный базис. Зависимость
координат в двойственном пространстве от базиса.
\item Изоморфизм линейных пространств. Изоморфизм линейных пространств одинаковой размерности.
\item Линейные отображения. Матрица линейного отображения. Зависимость её от базиса.
\item Линейные операторы. Ядро и образ оператора. Инвариантное
подпространство. Ограничение оператора и фактор-оператор. Вид
матрицы оператора, обладающего инвариантным подпространством.
\item Собственные значения и собственные векторы. Существование нетривиальных инвариантных
подпрос\-транств в случае алгебраически замкнутого поля.
\item Операторы проектирования. Их алгебраическая и геометрическая характеризация.
\item Нильпотентные операторы. Теорема о нормальной форме для нильпотентного оператора.
\item Корневые подпространства. Аннулирующий многочлен. Минимальный многочлен.
\item Теорема Гамильтона Кэли (доказательство для случая алгебраически замкнутого поля).
\item Теорема о разложении в прямую сумму корневых подпространств (для случая алгебраически замкнутого поля).
\item Теорема Жордана о приведении к нормальной форме.
\item Овеществление и комплексификация линейных пространств и операторов.
\item Существование одномерных или двумерных инвариантных подпространств у операторов над $\R$.
\item Евклидовы и эрмитовы пространства. Неравенство Коши Буняковского и неравенство треугольника.
\item Процесс ортогонализации. Ортогональное дополнение. Проекция и ортогональная составляющая.
\item Расстояние от вектора до подпространства, угол между вектором и подпространством.
\item Метод наименьших квадратов.
\item Определитель Грама $G(a_1,\ldots,a_n)$. Объём $n$ мерного параллелепипеда. Критерий невырожденности
матрицы Грама. Связь между
$G(f(a_1),\ldots,f(a_n))$ и $G(a_1,\ldots,a_n)$, где $f$ оператор. Критерий для матрицы быть матрицей Грама.
\item Изометрические операторы как операторы, сохраняющие скалярное произведение. Свойство матриц
изометрических операторов в ортонормированном базисе.
\item Канонический вид унитарного оператора.
\item Канонический вид ортогонального оператора.
\item Сопряжённый оператор. Самосопряжённые и кососимметрические операторы, их канонический вид.
\item Нормальные операторы, связь нормальности с диагонализируемостью.
\item Неотрицательные операторы. Существование и единственность неотрицательного квадратного корня из неотрицательного оператора.
\item Полярное разложение операторов (доказательство существования только для обратимых операторов).
\item Билинейные и полуторалинейные функции. Правое и левое ядро. Невырожденность.
\item Матрица билинейной (полуторалинейной) функции, её изменение при заменах базиса. (Анти)симметрич\-ные и эрмитовы функции.
\item Ортогональное дополнение относительно (анти)симметричной билинейной (эрмитовой полуторалинейной) функции. Его размерность.
Сумма подпространства и его ортогонального дополнения. Второе ортогональное дополнение.
\item Нормальный вид (анти)симметричных билинейных функций над полями $\R$ и $\Cbb$, эрмитовых полуторалинейных функций.
\item Теорема инерции. Теорема Якоби. Критерий Сильвестра.
\item Квадратичные функции в линейных и аффинных пространствах.
\item Группы $\Ob(p,q)$, $\Ub(p,q)$, $\Sp(2m)$. Частные случаи.
\item Естественный изоморфизм между $L(V)$ и $B(V)$. Приведение симметрической билинейной функции к каноническому
виду в евклидовом пространстве.
\item Обобщённый характеристический многочлен. Теорема об одновременном приведении одной квадратичной функции к каноническому
виду, а другой (положительно определённой) к нормальному виду.
\item Канонический изоморфизм линейного пространства и его второго двойственного. Тензоры. Полилинейные функции. Примеры.
\item Линейные операторы как тензоры. Матричные элементы обратной матрицы билинейной функции как тензоры.
\item Тензорное произведение тензоров. Координатное определение тензоров. Базис в пространстве тензоров.
\item Свёртка тензоров. Поднятие и опускание индексов в случае евклидова пространства.
\item Симметричные и кососимметричные тензоры. Симметризация и альтернирование. Внешнее умножение, его свойства.
\item Базис в пространстве кососимметрических тензоров.
\end{nums}

\medskip
\dmvntrail

\pagebreak


\section*{Программа 2005 года}

\begin{nums}{-2}
\item Линейное пространство. Определение, примеры. Линейная
оболочка. Аффинное пространство.
\item  Линейная  (не)зависимость системы векторов. Ранг  системы
векторов.  Размерность.  Базис. Координаты.
\item Подпространство. Факторпространство.   Теорема о сумме
размерностей подпространства и факторпространства.
\item Пересечение и сумма подпространств. Теорема об их размерностях.
Прямая сумма двух и более подпространств. Внешняя прямая сумма.
\item Двойственное пространство.   Двойственный базис.   Пример:
двойственное пространство   к пространству многочленов степени не
выше $n$ и его базис.
\item  Изоморфизм линейных   пространств. Изоморфность линейных
пространств одинаковой размерности. Второе двойственное
пространство. Канонический изоморфизм между пространством и его
вторым двойственным.
\item Линейные отображения. Ядро и образ линейного отображения.
Теорема о сумме размерностей ядра и образа. Матрица линейного
отображения. Зависимость от базиса.
\item Линейные операторы. Ядро и образ оператора. Инвариантное
подпространство. Ограничение оператора   и фактор оператор.    Вид
матрицы оператора, обладающего инвариантным подпространством.
\item Собственные значения и собственные векторы. Существование
нетривиальных инвариантных подпрост-\break ранств в случае
алгебраически замкнутого поля.
\item Операторы проектирования. Их алгебраическая и геометрическая
характеризация.
\item  Нильпотентные операторы. Теорема о нормальной форме для
нильпотентного оператора.
\item Собственные значения и собственные векторы. Корневые
подпространства. Аннулирующий многочлен. Минимальный многочлен.
\item  Теорема Гамильтона Кэли (доказательство для случая
алгебраически замкнутого поля).
\item Теорема о разложении в прямую сумму корневых подпространств
(для случая алгебраически замкнутого поля).
\item  Теорема Жордана о приведении к нормальной форме.
\item Овеществление и комплексификация. Канонические изоморфизмы
$(V_\mathbb{C})_\mathbb{R}\cong V\oplus V$,
$(V_\mathbb{R})_\mathbb{C}\cong V\oplus \overline{V}$.

\item Существование одномерных или двумерных инвариантных
подпространств у операторов над $\mathbb{R}$. Вещественная
жорданова нормальная форма.

\item Евклидовы   и эрмитовы    (унитарные) пространства.
Неравенство Коши Буняковского  и неравенство треугольника.

\item Процесс ортогонализации.     Ортогональное дополнение.
Проекция и ортогональная составляющая.

\item Расстояние от вектора до подпространства, угол между вектором
и подпространством.

\item Метод наименьших квадратов.

\item  Определитель Грама $G(a_1,\ldots, a_n)$. Объем $n$ мерного
параллелепипеда. Связь между $G\br{f(a_1),\ldots, f(a_n)}$ и
$G(a_1,\ldots, a_n)$, где $f$ оператор. Критерий
невырожденности матрицы Грама. Критерий для матрица быть матрицей
Грама.

\item  Изоморфизмы евклидовых   (унитарных) пространств. Операторы,
сохраняющие скалярное произведение. Изометрии. Частичные
изометрии.

\item Канонический вид унитарного оператора.

\item  Канонический вид ортогонального оператора.

\item Самосопряженные и кососимметрические операторы, их
канонический вид.

\item Нормальные операторы, связь нормальности с
диагонализируемостью.

\item Неотрицательные   операторы. Неотрицательность   квадратного
корня из   неотрицательного оператора.

\item  Полярное разложение операторов.

\item  Билинейные, полуторалинейные, квадратичные функции.
Канонический изоморфизм $B(V)\cong L(V, V')$. Правое и левое ядро.
Невырожденность.

\item Матрица   билинейной   (полуторалинейной)   функции,   ее
изменение   при   заменах   базиса. Симметричные и антисимметричные и
эрмитовы функции.

\item  Ортогональное    дополнение    относительно
(анти)симметричной билинейной (эрмитовой полуторалинейной)
функции. Его размерность. Сумма подпространства и его
ортогонального дополнения. Второе ортогональное дополнение.

\item Нормальный вид (анти)симметричных билинейных функций над
полями $\mathbb{R}$ и $\mathbb{C}$, эрмитовых полуторалинейных
функций.

\item  Теорема инерции. Теорема Якоби. Критерий Сильвестра.

\item  Вещественная и мнимая части псевдоэрмитова скалярного
произведения как псевдоевклидово и симплектическое  скалярные
произведения.

\item  Группы $\Ob(p,q)$, $\Ub(p, q)$, $\Sp(2m)$. Частные случаи. Связь
между этими группами.

\item Изотропные подпространства в симплектическом пространстве.
Лагранжевы подпространства. Существование   для   любого
изотропного   подпространства   содержащего   его   лагранжева
подпространства.

\item  Существование   для   лагранжева подпространства   $L$  в
симплектическом   пространстве   $V$ лагранжева подпространства
$L'$, дополняющего $L$ до $V$.

\item Приведение   симметрической   билинейной   функции   к
каноническому   виду   в   евклидовом пространстве.

\item Приведение     пары     квадратичных     функций     к
диагональному     виду.      Обобщенный характеристический
многочлен. Теорема об одновременном приведении одной квадратичной
функции к каноническому виду, а другой (положительно определенной)
к нормальному виду.

\item  Тензоры.   Полилинейные  функции. Примеры.   Тензорное
произведение  тензоров.   Базис  в пространстве тензоров.

\item   Свертка тензоров. Поднятие и опускание индексов в случае
евклидова пространства.

\item Симметричные и кососимметричные тензоры. Симметризация и
альтернирование. Внешнее умножение, его свойства.

\item  Базис в пространстве кососимметрических тензоров. Связь между
линейной зависимостью и тривиальностью внешнего произведения.
\end{nums}

\medskip\dmvntrail
\end{document}
