\documentclass[unicode, 10pt, a4paper, oneside, fleqn]{article}
\usepackage[xe]{dmvn}
\newcommand{\half}{\ensuremath{{\scriptstyle\frac{1}{2}}}}
\begin{document}
\dmvntitle{Краткая книга об исчислении}
          {Алгебры и Алмукабалы}
          {Реферат студента Богатова Дмитрия}
          {IV курс, кафедра математического анализа}
          {Москва, 2012~г}
\section{Введение}
Данная работа является конспектом труда
центральноазиатского математика Мухаммеда ибн Муса Хорезми~(783~-850)
«Краткая книга об исчислении Алгебры и Алмукабалы», приведенного к современным
терминам и обозначениям. Реферат повторяет структуру оригинального труда с целью
демонстрации, насколько невыразительны были математические средства того времени.
\section{Конспект}
\paragraph{Введение}
 Введение содержит длинную молитву Аллаху, благодарность
ему же, что наделил его покровителя имама (духовный сан) ал-Мамуна
«стремлением приближать к себе ученых, …». Далее следует аннотация
проделанной работы, перемежаемая восхвалением Аллаха, повествующая
о позиционной десятичной систему счисления. Также определяется понятие
корня и квадрата. Определяется контекстно зависимо, т.е если
\begin{displaymath}
  a² = b
\end{displaymath}
то в данном контексте $a$ это корень, а $b$ это квадрат. То, что величина
$b²$ существует, автор не рассматривает. Далее, определяется понятие простого числа.
Оно определяется как принципиально иной объект: «Простое число — это всякое число,
называемое словами без отношения к корню или квадрату.» Тот факт, что
$$ \sqrt{9} = 3 $$
автор опять таки не рассматривает.
\paragraph{Глава1}
 Рассматривается частный случай квадратных уравнений
 \begin{displaymath}
   ax² = bx ⇒x = b/a
 \end{displaymath}
Более точно, в самой работе эта формула не фигурирует, но демонстрируется, что конкретное
число действительно является корнем. Все рассмотрения происходят над множеством
натуральных чисел.
\paragraph{Глава2}
В сущности, записана следующая импликация
\begin{displaymath}
  x² = a² ⇒ x = a
\end{displaymath}
\paragraph{Глава3}
Автор демонстрирует, что ему известны значения квадратов небольших
натуральных чисел.
\paragraph{Глава4}
Приводится формула $\frac{D}{4}$:
\begin{displaymath}
  x²+bx = c ⇒ x = \sqrt{\left(\frac{b}{2}\right)^2 +c} - \frac{b}{2}
\end{displaymath}
Вновь, рассматриваются только положительные, натуральные решения уравнений с
натуральными коэффициентами. Неприведенные уравнения приводятся к приведенным.
\paragraph{Глава5}
Рассматривается квадратное уравнение, записанное в виде
\begin{displaymath}
 x² + c = bx
\end{displaymath}
Предлагается использовать следующую формулу
\begin{displaymath}
  x = \frac{b}{2} - \sqrt{\hr{\frac{b}{2}}^2 - c}
\end{displaymath}
\paragraph{Глава6}
В ней рассматривается уравнение
\begin{displaymath}
  c + bx = x²
\end{displaymath}
\begin{note}
  Очевидно, что автор использует формулу дискриминанта, однако
  обходит вопросы, связанные с отрицательными числам, поэтому
  рассматривает несколько видов записи квадратного уравнения. В труде отмечено,
  что задача неразрешима при отрицательном дискриминанте. Менее очевидно,
  что, несмотря на то, что задача ставится в денежном контексте
  «… Квадрат и число двадцать один дихрем равны десяти его корням», автор
  обходит вопрос размерности — он полагает, что размерность квадрата совпадает
  с размерностью корня.
\end{note}
\par
Далее следуют не совсем тривиальные рассмотрения, геометрически обосновывающие
метод выделения полного квадрата для рассмотренных случаев. По техническим причинам,
доказательство не приводится.
\paragraph{Глава об умножении}
В этой главе приводится ряд тривиальных равенств.
\begin{gather}
  m*n = \underbrace{m+\dots+m}_{n} \\
  (a+c)(b+d) = ab + ad + cb + cd \\
\end{gather}
\paragraph{Глава об увеличении и уменьшении}
В этой главе приведено утверждается, что в выражениях могут
взаимно уничтожаться даже иррациональные величины. Также
приводится соотношения
\begin{gather}
  (nx)² = n²x² \\
  \hr{\frac{x}{n}}² = \frac{x²}{n²} \\
  n\sqrt{a²} = \sqrt{n²a² } \\
  \frac{a²}{b²} = \frac{a}{b}
\end{gather}
Стоит отметить, что утверждение про взаимное уничтожение
в арифметических выражениях автор доказывает снова и снова,
используя числовую прямую.
\paragraph{Глава о шести задачах}
В этой главе приводятся задачи и их сведение к рассмотренным
типам уравнений.
\begin{enumerate}
\item $$ \bcase{ x + y = 10 \\ x² = 4xy} $$
  Метод подстановки, разумеется.
\item $$\bcase {x + y =10 \\ 2{\scriptstyle\frac{7}{9}}x² = 100 = 6{\scriptstyle\frac{1}{4}}y²} $$
  (!) В оригинале именно три условия на две переменные, но используется только два.
\item $$ \bcase{x+y = 10\\ x/y = 4} $$
  Метод подстановки, причем используется факт, что
  \begin{displaymath}
    \frac{a}{b} = c ⇒a = bc
  \end{displaymath}
  \item $$ ({\scriptstyle\frac{1}{3}}x +1)({\scriptstyle\frac{1}{4}}x+1) = 12 $$
    Квадратное уравнение в чистом виде.
  \item $$ \bcase{x+y = 10\\ x²+y² = 58} $$
  \item $$ ({\scriptstyle\frac{1}{4}}x)({\scriptstyle\frac{1}{3}}x) = x + 24 $$
\end{enumerate}
\paragraph{Глава об различных задачах}
В этой задаче также приводится задачи, сводимые
к квадратным уравнениям. Ввиду их количества,
пропустим их формулировки.
\paragraph{Глава о сделках}
В этой главе крайне многосложно излагается идея
пропорций.
\paragraph{Глава об измерении}
В этой главе упоминаются связи площадей и линейных размеров
различных геометрических фигур и приводятся следующие формулы
\begin{gather}
  S_△ = {\scriptstyle\frac{1}{2}}ah \\
  S_□ = a² \\
  S_◊ = {\scriptstyle\frac{1}{2}}d_1d_2 \\
  L_○= πd
\end{gather}
Для величины $π$ (это обозначение не используется),
приводятся следующие оценки
\begin{displaymath}
  π≈3{\scriptstyle\frac{1}{7}}≈\sqrt{10} ≈\frac{62832}{20000}
\end{displaymath}
Для площади круга предлагаются формулы
\begin{displaymath}
  S = \half d \half L ≈ \frac{11}{14}d²
\end{displaymath}
Также в главе представлены формулы для объема конусов,
теорема Пифагора. Также представлена следующая классификация
четырехугольников:
\begin{itemize}
\item Квадрат («С равными сторонами и прямыми углами»)
\item Прямоугольник («С прямыми углами и разными сторонами»)
\item  Ромб («Равные стороны и разные углы»)
\item Ромбоид (В классической терминологии — параллелограмм)
\item «С разными сторонам и углами»
\end{itemize}
Для них приводятся стандартные формулы площади и
краткое описание метода триангуляции.

Далее, рассматриваются треугольники — прямоугольные,
остроугольные и тупоугольные. Также  приводится формула
объема пирамиды и несколько численных примеров подстановки
в отмеченные формулы.
\section{Список литературы}
\par
Мухаммад ибн Муса аль Хорезми «Математические трактаты»,
издательство «ФАН», 1973

\end{document}
