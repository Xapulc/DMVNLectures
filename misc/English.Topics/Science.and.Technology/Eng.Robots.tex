\documentclass[a4paper]{article}
\usepackage[simple]{dmvn}

\begin{document}
\section{Robots}

A robot is an any automatically operated machine that replaces human effort,
though it may not resemble human beings in appearance or perform functions in a
humanlike manner. The term is derived from the Czech word robota, meaning
<<forced labour>>. Modern use of the term stems from the play R.U.R written in 1920
by the Czech author Karel Capek, which depicts society as having become
dependent on mechanical workers called robots that are capable of doing any kind
of mental or physical work.

Modern robot devices descend
through two distinct lines of development~--- the early automation
essentially mechanical toys, and the successive innovations and refinements
introduced in the development of industrial machinery. Possibly the earliest
ancestor of today's industrial-robot devices is the clepsydra, or water clock,
which improved upon the hourglass by employing a siphon principle to
automatically recycle itself. Ctesibius of Alexandria, a Greek physicist and
inventor, is reputed to have constructed such a clock about 250 ВС. Weight-
driven, pendulum-controlled clocks were invented in the European Middle Ages.
The spring-driven clock was finally developed during the 18th century, which
also witnessed the introduction of rudimentary forms of automatic machinery in
the textile industry.

The Industrial Revolution stimulated the invention of
elementary robot mechanisms to perfect the production of power itself. The steam
engine inspired the governor (actuated by rotating weights), which, when it
slowed under load, increased the flow of steam to the engine and, when its load
decreased, reduced it. The internal-combustion engine of the 19th century
brought a recycling innovation in the form of pistons that repositioned
themselves after each cycle.

The later 19th and early 20th centuries saw a rapid
proliferation of powerful machinery in industrial operations. These at first
required a person to position both the work and the machine, and later only the
work. Automatic cycle-repeating machines (automatic washers), self-measuring and
adjusting machines (textile colour-blending equipment), and machines with a
degree of self-programming (automatic elevators) soon followed. Since the late
1960s major developments in microelectronics and computer technology have led to
significant advances in robotics.

In the automobile industry, for example, the
introduction of computer-controlled robot devices has automated assembly lines
to a marked degree. These one-armed robots are capable of simulating the
articulation and movement of the human arm and hand and can be used to lift,
weld, and spray-paint automobile bodies. The mechanical arms are programmed by
physically moving them through the desired motions. The different movements are
recorded in the computer's memory so that they can be repeated precisely. Some
high-performance robots have built-in sensors that enable them to correct their
movements if they deviate from the programmed patterns. Others are equipped with
electronic digital cameras and are used to inspect automobile bodies to make
certain that their dimensions meet specified standards. In such robot devices,
the camera converts images into patterns of digital pulses, which in turn are
compared with pulse patterns stored in the controlling computer's memory. The
stored patterns represent two-dimensional geometric shapes that the computer has
been programmed to identify. Similar kinds of advanced high-performance robots
have been adopted by other assembly industries. Aircraft manufacturers employ
single-arm mechanisms for drilling and riveting body sections, while a growing
number of electronics firms are utilizing robot devices in conjunction with
other computerized instruments to sort or test finished products.

\end{document}
