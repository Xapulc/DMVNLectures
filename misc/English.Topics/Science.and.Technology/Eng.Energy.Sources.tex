\documentclass[a4paper]{article}
\usepackage[simple]{dmvn}

\begin{document}
Here are some topics on <<Science and Technology>>

\section{Energy Sources}

Solar energy is one the most resourceful sources of energy for the future. One of the reasons
for this is that the total energy we recieve each year from the sun is around 35,000 times
the total energy used by man. However, about 1/3 of this energy is either absorbed by the
outer atmosphere or reflected back into space (a proccess called albedo)1.
Solar energy is presently being used on a smaller scale in furnaces for homes and to heat up
swimming pools. On a larger scale use, solar energy could be used to run cars, power plants,
and space ships (like the picture you see above).

Geothermal energy is an alternative energy source, although it is not resourceful enough to
replace more than a minor amount of the future's energy needs. Geothermal energy is obtained
from the internal heat of the planet and can be used to generate steam to run a steam turbine.
This in turn generates electricity, which is a very useful form of energy.

The radius of the Earth is about 4000 miles, with an internal core temperature of about 4000
degrees celsius at the center. The mantle surrounds the outercore and is only about 45 miles
below the surface, depending on location. The temperature at the mantle-surface crust boundary is
about 375 degrees, celsius. (This is too deep to get to\dots as of today)

So, how does this help us? It turns out that if we drill down only three miles we can reach
temperatures of 100 degrees, celsius, which is enough to boil water to run a steam-powered electric
powerplant. Drilling three miles through the earth is possible, but not easy, so luckily there are
easier routes to access this power source, known as geothermal hotspots.

Geothermal hotspots are volcanic features which are found all around the world. Basically a hotspot
is an area of reduced thickness in the mantle which transmits excess internal heat from the interior
of the earth to the outer crust. These hotspots are well known for their unique effects on the surface,
such as the volcanic islands of Hawaii, the mineral deposits and gyesers in Yellowstone National Park,
or the hotsprings in Iceland. These geothermal hotspots can easily be used to generate electricity.

Some systems pump hot-water into permeable sedimentary hospots found underground and then use the steam
to generate electricity. Then the used steam is condensed and sent back down to the permeable
sedimentary stream. Another system utilizes volcanic magma which is still partly molten at around
650 degrees, celsius, to boil water which would generate electricity. Also there is a system which
uses hot dry rock, which is just hardened magma, but still is extremely hot. To recover this heat
from these rocks, a system is used which circulates water through the rock and transfers the heat
up to a steam generator. The first system listed here is not as useful as other methods because of
the acidic nature of the fluids found under the ground. These acidities require a lot of maintenance
and upkeep on the equipment, and this cost reduces the economic effectiveness of the system.
Therefore, geothermal energy systems are more inefficient than other alternative energy sources because
of the costs required in upkeep and the shortage of potential sites.

Even the power of the tides can be harnessed to produce electricity.

Similar to the more conventional hydroelectric dams, the tidal process utilizes the natural motion
of the tides to fill reservoirs, which are then slowly discharged through electricity-producing turbines.
The former USSR produced 300 MW in its Lumkara plant using this method.

Hydroelectricity comes from the damming of rivers and utilizing the potential energy stored in the water.
As the water stored behind a dam is released at high pressure, its kinetic energy is transferred onto
turbine blades and used to generate electricity. This system has enormous costs up front, but has
relatively low maintenance costs and provides power quite cheaply. In the United States approximately
180,000 MW of hydroelectric power potential is available, and about a third of that is currently
being harnessed.

\section{Alternative Energy (article dated 1996)}

It was predicted in 1995 that the increasing economic competitiveness of energy sources such as solar,
biomass, wind, geothermal, and tidal barrages would not be dependent on technological breakthroughs.
Within 20 years, it was thought, some alternative energy sources should reach competitive parity
with oil priced at \$15 a barrel. Limited market demand and the economics of production continued
to restrict the large-scale development of alternative sources in 1995, however.

Commercial applications of alternative energy generally remained confined to remote locations or areas
in which it had a distinct competitive advantage, as in solar-powered heating or the generation of
electricity in sunny climates. Even the international oil industry, however, began to use
alternative energy to bring down operating costs. The U.S. oil company Amoco, for example,
began installing wind-powered electrical generators on offshore natural gas platforms in the North Sea.
There also was growing interest in combining alternative energy sources with more conventional methods
of power generation. In the U.S. there was interest in using the high-quality gas produced at urban
landfills, and natural gas companies were looking into ways in which biomass gathered from land or
aquatic plant material could be processed to produce gas energy.

A study published in 1996 by the International Energy Agency (IEA), the Paris-based group that monitors
energy developments'on behalf of the Western industrialized countries, concluded that the demand for
alternative energy sources would grow strongly in the coming years. Even so, these sources would
account for only a small portion of the world's energy mix by 2010. The IEA estimated that
fossil-based fuels would account for almost 90\% of total demand in 2010. Nonhydroelectric
renewable sources, such as biomass, wind, wave, solar, and geothermal power, however, were expected
to register the highest growth rate. The IEA predicted that renewable sources would account for only
about 1\% of the total supply by 2010, compared with almost 3\% for hydropower.

The World Energy Council, a nongovernmental international group that promotes sustainable energy
policies, estimated that renewable sources could provide 5-8\% of the world's power requirements
by 2020 but only with additional spending on research and development. The current levels of government
support for alternative energy sources had resulted in steadily declining costs. The cost of
photovoltaic cells, for example, had fallen from tens of thousands of dollars per watt in the
1960s to about \$6. The world market for solar power remained small, however.

\section{Exploiting renewable energy sources}

Growing concern over the world's ever-increasing energy needs and the prospect of rapidly dwindling
reserves of oil, natural gas, and uranium fuel have prompted efforts to develop viable alternative
energy sources. The volatility and uncertainty of the petroleum fuel supply were dramatically
brought to the fore during the energy crisis of the 1970s caused by the abrupt curtailment of oil
shipments from the Middle East to many of the highly industrialized nations of the world. It also
has been recognized that the heavy reliance on fossil fuels has had an adverse impact on the
environment. Gasoline engines and steam-turbine power plants that burn coal or natural gas emit
substantial amounts of sulfur dioxide and nitrogen oxides into the atmosphere. When these gases
combine with atmospheric water vapour, they form sulfuric acid and nitric acids, giving rise to
highly acidic precipitation. The combustion of fossil fuels also releases carbon dioxide. The
amount of this gas in the atmosphere has steadily risen since the mid-1800s largely as a result
of the growing consumption of coal, oil, and natural gas. More and more scientists believe that
the atmospheric buildup of carbon dioxide (along with that of other industrial gases such as
methane and chlorofluorocarbons) may induce a greenhouse effect, raising the surface temperature
of the Earth by increasing the amount of heat trapped in the lower atmosphere. This condition
could bring about climatic changes with serious repercussions for natural and agricultural ecosystems.

Many countries have initiated programs to develop renewable energy technologies that would enable
them to reduce fossil-fuel consumption and its attendant problems. Fusion devices are believed
to be the best long-term option, since their primary energy source would be the hydrogen
isotope deuterium abundantly present in ordinary water. Other technologies that are being
actively pursued are those designed to make wider and more efficient use of the energy
in sunlight, wind, moving water, and terrestrial heat (i.e., geothermal energy). The amount
of energy in such renewable and virtually pollution-free sources is large in relation to world
energy needs, yet at the present time only a small portion of it can be converted to electric
power at reasonable cost. A variety of devices and systems have been created to better tap
the energy in sunlight. Among the most efficient are photovoltaic systems that transform radiant
energy from the Sun directly into electricity by means of silicon or gallium arsenide solar cells.
Large arrays consisting of thousands of these semiconductor cells can function as central power
stations (see Solar cells below). Other systems, which are still under development, are designed
to concentrate solar radiation not only to generate electric power but also to produce
high-temperature process heat for various applications. These systems employ a number of
different components, including large parabolic concentrators and heat engines of the Stirling
engine type. Another approach involves the use of flat-plate solar collectors to provide space
heating for commercial and residential buildings.

Although wind is intermittent and diffuse, it contains tremendous amounts of energy. Sophisticated
wind turbines have been developed to convert this energy to electric power. The utilization
of wind energy systems grew discemibly during the 1980s. For example, more than 15,000 wind
turbines are now in operation in Hawaii and California at specially selected sites. Their
combined power rating of 1,500 megawatts is roughly equal to that of a conventional
steam-turbine power installation.

Converting the energy in moving water to electricity has been a long-standing technology.
Yet, hydroelectric power plants are estimated to provide only about 2 percent of the world's
energy requirements. The technology involved is simple enough: hydraulic turbines change the
energy of fast-flowing or falling water into mechanical energy that drives power generators,
which produce electricity (see Water turbines below). Hydroelectric power plants, however,
generally require the building of costly dams. Another factor that limits any significant
increase in hydroelectric power production is the scarcity of suitable sites for additional
installations except in certain regions of the world. In certain coastal areas of the world,
as, for example, the Ranee River estuary in Brittany, Fr., hydraulic turbine-generator units
have been used to harness the great amount of energy in ocean tides (see Tidal plants below).
At most such sites, the capital costs of constructing damlike structures with which to trap and
store water are prohibitive, however. Geothermal energy flows from the hot interior of the Earth
to the surface in steam or hot water most often in areas of active volcanism. Geothermal
reservoirs with temperatures of 180 С or higher are suitable for power generation. The earliest
commercial geothermal power plant was built in 1904 in Larderello, Italy. Today, steam from wells
drilled to depths of hundreds of metres drives the plant's turbine generators to produce about
190 megawatts of electricity. Geothermal plants have been built in a number of other countries,
including El Salvador, Japan, Mexico, New Zealand, and the United States. The principal
U.S. plant, located at The Geysers north of San Francisco, can generate up to 1,900 megawatts,
though production may be restricted to prolong the life of the steam field.

\end{document}