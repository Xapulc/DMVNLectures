\documentclass[a4paper]{article}
\usepackage[simple]{dmvn}

\begin{document}
\cent{{\Large <<Science and Technology>>\\The Computer Verification of Calculation Models\par}}

I would like to speak about some field in mathematical logic connected with modeling of
calculation processes. The methods of pure logic become more and more practically applicable
now with the development of computer science.

First, we will discuss a notion of calculation process. This is a pair $\ha{\Sbb, \tau}$ where
$\Sbb$ is a model state space and $\tau$ is the transition function: $\tau\cln \Sbb\ra \Sbb$.
Whole calculation, hence, consists of sequence (maybe infinite) of steps when our model turns
from one state to another under action of transition function. For example, the Euclid's algorithm
of finding the greatest common divider is the calculation process. The state space of this model
is the Cartesian product $\N \times \N$ and transition function consists of division one
number by another and storing rest in the place of divisible. The process stops when division performs
with zero rest.

Another example of process that can be described in such way is the transferring file from one
workstation to another via the LAN. Some people, maybe disagree with me that such a complex processes
can be describes in logic formulas, but if we abstract from some things, it is really possible!

Now let us discuss a bit a purpose of such things as logical modeling. The first purpose is the real
model verification. Consider we have a very complicated program written, say, in C++ that contains
a bug, so it works incorrectly. There are two ways to fix it: first, and rather efficient is to use
the army of programmers and testers who will examine program flow and search a bug. It's now a good time
to remember that all bugs can be classified it two groups: logical bugs and mistakes that are completely
different. Sometimes, it is easy to fix a mistake, but it is very hard even to find the logical error in
program. The scientists proudly say that with increasing of speed of computers, it can be possible to find
such bugs using the computer verifier~--- the program that analyzes the logical structure of our buggy
program and finds errors. Note that the verification may took a plenty of time, but if such techniques were
used when Intel Corporation produced a Pentium microprocessor containing a FDIV bug, this company won't lost
\$1.000.000 that gone on replacement of buggy processors.

Now se should say a few words about the way that we use on turning our real process into a logical model.
In the case of buggy program, one can ask: why shouldn't we use, say, C++ code for building a skeleton of model?
The answer is easy: because all existing programming languages cannot supply needed pure logical structuring
and formality. To fix this problem, a new language was developed by the Logic Department of Microsoft Corporation:
it is called ASML for Abstract State Machines Language. This notion includes, in particular, the formal models
of calculation processes.

Returning to problem of verification I must add that some hardware manufacturers nowadays use such techniques.
The Advanced Micro Devices, firstly applied them to verification of their microprocessors. In conclusion I'd like to
cite one of ASML fathers: <<Soon comes a day when we will write computer programs that will be completely verified
using computer programs>>. This will ultimately lead to rising up the quality of all computer-connected production.

\end{document}
