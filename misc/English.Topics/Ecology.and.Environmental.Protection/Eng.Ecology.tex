\documentclass[a4paper]{article}
\usepackage[simple]{dmvn}

\begin{document}
\cent{\Large
<<Ecology and Environmental Protection>>\\
Campaign Launched To Eliminate 12 Hazardous Chemicals\par}

The 2001 Stockholm Convention on Persistent Organic Pollutants went into force on May 17,
marking the start of an ambitious international effort to rid the world of PCBs, dioxins and
furans, and nine highly dangerous pesticides.

"The Stockholm Convention will save lives and protect the natural environment -- particularly in
the poorest communities and countries -- by banning the production and use of some of the most toxic
chemicals known to humankind," said Executive Klaus Toepfer of the United Nations Environment Programme,
under whose auspices the Convention was adopted.

"Over the next several years national investments plus donor pledges of hundreds of millions will
channel more than five hundred million dollars into an overdue and urgently needed initiative to ensure
that future generations do not have to live as we do with measurable quantities of these toxic chemicals
stored in their bodies," he said.

Of all the pollutants released into the environment every year by human activity, POPs are amongst the
most dangerous. For decades these highly toxic chemicals have killed and sickened people and animals by
causing cancer and damaging the nervous, reproductive and immune systems. They have also caused uncounted
birth defects, according to the UNEP.

Governments will seek a rapid start to action against POPs when they meet for the first meeting of the
Conference of the Parties to the Convention (COP 1) in Punta del Este, Uruguay in the first week of May
2005.
\end{document}