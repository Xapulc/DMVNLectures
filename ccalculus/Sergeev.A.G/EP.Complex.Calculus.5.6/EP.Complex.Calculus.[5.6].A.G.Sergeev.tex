\documentclass[a4paper]{article}
\usepackage[utf,simple]{dmvn}

\title{Программа экзаменов по комплексному анализу}
\author{Лектор-- А.\,Г.\,Сергеев}
\date{V--VI семестры, 2002--2003 г.}

\begin{document}
\maketitle

\section*{V семестр}

\begin{nums}{-3}
\item Комплексные числа, полярная форма комплексного числа, стереографическая проекция.
\item Топология комплексной плоскости, пути и кривые, области на комплексной
плоскости; теорема об открыто-- замкнутом подмножестве связного множества.
\item Вещественная и комплексная дифференцируемость функций на комплексной
плоскости.
\item Комплексная производная; производная по направлению; условия Коши-- Римана.
\item Конформные отображения; геометрическая интерпретация комплексной
производной.
\item Дробно-- линейные преобразования $\ol\Cbb$; конформность дробно-- линейных отображений.
\item Группа дробно-- линейных отображений. Круговое свойство дробно-- линейных
отображений.
\item Свойство сохранения симметрии при дробно-- линейных отображениях; свойство
трех точек.
\item Дробно-- линейные автоморфизмы основных областей (круг, полуплоскость,
плоскость, $\ol\Cbb$).
\item Интеграл от функции вдоль кривой и его свойства (линейность;
аддитивность; независимость от параметризации; ориентированность). Оценка
интеграла вдоль кривой.
\item Лемма Гурса (теорема Коши для треугольников).
\item Комплексная первообразная; единственность; существование первообразной в
круге.
\item Первообразная вдоль пути; теорема существования. Формула
Ньютона-- Лейбница.
\item Отношение гомотопности; теорема Коши. Односвязные области.
\item Существование комплексной первообразной в односвязной области. Теорема
Коши в многосвязной области.
\item Интегральная формула Коши; теорема о среднем; формула Коши-- Грина (без
доказательства).
\item Разложение голоморфной функции и ряд Тейлора; неравенства Коши; теорема
Лиувилля.
\item Голоморфность суммы степенного ряда и круге его сходимости, формула
Коши-- Адамара.
\item Коэффициенты ряда Тейлора; бесконечная дифференцируемость голоморфной
функции; формула Коши для производных голоморфной функции.
\item Теорема Мореры. Три эквивалентных определения голоморфной функции.
\item Нули голоморфной функции; разложение голоморфной функции в окрестности
нуля; теорема единственности.
\item Ряды из голоморфных функций; теорема Вейерштрасса. Задача аппроксимации
и теорема Рунге.
\item Разложение голоморфной функции в ряд Лорана.
\item Ряды по целым степеням $z$; коэффициенты Лорана; неравенства Коши для
коэффициентов Лорана.
\item Изолированные особые точки; описание устранимых особых точек и полюсов в
терминах ряда Лорана.
\item Теорема Сохоцкого; теорема Лиувилля (для целой функции с полюсом);
рациональность мероморфной функции на расширенной плоскости.
\item Вычеты; теорема Коши о вычетах.
\item Вычет в терминах ряда Лорана; вычисление вычета; лемма Жордана; теорема
о полной сумме вычетов.
\item Аналитическое продолжение гамма-- функции Эйлера.
\item Понятие об аналитическом продолжении. Аналитическое продолжение
канонического элемента по цепочке и вдоль пути. Эквивалентность
аналитических продолжений по цепочке и вдоль пути.
\item Единственность продолжения канонического элемента вдоль пути.
\item Теорема о монодромии.
\item Понятие многозначной аналитической функции. Аналитические элементы и их
продолжение. Ветви аналитической функции.
\item Элементарные аналитические функции: корень $n$-- й степени, логарифм.
\item Особые точки аналитической функции. Точки ветвления конечного порядка.
Разложение аналитической функции в ряд Пюизо.
\item Римановы поверхности функций $\sqrt z$ и $\Ln z$.
\item Определение римановой поверхности аналитической функции. Структура
комплексного многообразия на римановой поверхности.
\end{nums}

\section*{VI семестр}
\begin{nums}{-2}
\item Логарифмический вычет. Число нулей и полюсов функции, мероморфной в
области. Принцип аргумента.
\item Теорема Руше. Основная теорема алгебры.
\item Принцип сохранения области.
\item Локальное обращение голоморфных функций. Необходимое условие локальной
однолистности.
\item Теорема Гурвица. Предел последовательности однолистных голоморфных
функций.
\item Принцип максимума модуля. Лемма Шварца.
\item Локально равномерно ограниченные семейства голоморфных функций.
\item Теорема Монтеля. Непрерывные функционалы на компактных семействах
функций.
\item Автоморфизмы основных областей (расширенная плоскость, комплексная
плоскость, единичный круг).
\item Теорема Римана.
\item Принцип соответствия границ. Теорема Каратеодори (без доказательства).
\item Принцип симметрии.
\item Конформное отображение верхней полуплоскости на прямоугольник. Интеграл
Кристоффеля-- Шварца.
\item Эллиптический синус. Периоды мероморфных функций.
\item Общие свойства эллиптических функций (теорема Лиувилля, вычеты и
значения эллиптической функции в фундаментальном параллелограмме, суммы
нулей и полюсов).
\item Функция Вейерштрасса.
\item Двоякопериодичность функции Вейерштрасса. Поведение функции Вейерштрасса
и ее производной в фундаментальном параллелограмме.
\item Поле эллиптических функций с заданной решеткой периодов и его
образующие.
\item Дифференциальное уравнение для функции Вейерштрасса.
\item Параметризация кубической кривой с помощью функции Вейерштрасса. Формула
сложения точек.
\item Модулярная функция. Теорема Пикара.
\item Гармонические функции. Бесконечная дифференцируемость, теорема о
среднем, принцип экстремума. Лемма Харнака (без доказательства).
\item Задача Дирихле. Единственность. Формула Пуассона, интеграл Шварца.
\end{nums}

\medskip
\dmvntrail
\end{document}


%% Local Variables:
%% eval: (setq compile-command (concat "latex  -halt-on-error -file-line-error " (buffer-name)))
%% End:
