\documentclass[a4paper]{article}
\usepackage[utf]{dmvn}

\title{Программа экзамена по комплексному анализу}
\author{Лектор-- Е.\,П.\,Долженко}
\date{V--VI семестры, 2004--2005 г.}
\begin{document}
\maketitle

\section*{V семестр}

\subsection*{Программа экзамена}

\begin{nums}{-2}
\item Основные понятия.

Комплексные числа, комплексная плоскость. Модуль и аргумент комплексного числа,
их свойства. Многозначные функции и их однозначные ветви на примере аргумента $\Arg z$.
Тригонометрическая и показательная формы записи комплексного числа. Метрика и топология комплексной плоскости:
последовательности и пределы, бесконечно удалённая точка и расширенная комплексная плоскость,
замкнутые и открытые множества, области, компакты. Пути и кривые на плоскости. Пример жордановой кривой
положительной площади, кривая Пеано. Спрямляемые кривые, натуральная параметризация.

\item Стереографическая проекция.

Формулы стереографической проекции. Хордальная метрика в $\ol \Cbb$.
Круговое свойство стереографической проекции. Постоянство растяжений. Угол с вершиной в бесконечности.
Сохранение углов при стереографическом проектировании. Конформность. Некоторые симметрии сферы Римана и
комплексной плоскости. Отображение~${w = \frac1z}$. Другие виды стереографической проекции.

\item Функции комплексного переменного.

Предел функции, непрерывность. Модуль
непрерывности, его свойства. Дифференцируемость по комплексному переменному, условия Коши-- Римана. Правила
дифференцирования по комплексному переменному, производная обратной функции. Аналитичность (голоморфность)
функции в области и в точке (в том числе в бесконечности). Теорема Лумана-- Меньшова (без доказательства).
Формальные производные по переменным~$z$ и~$\ol z$, комплексная форма условий Коши-- Римана. Понятие
о полианалитических функциях.

\item Геометрический смысл комплексной дифференцируемости, конформность.

Сохранение углов и постоянство
растяжений в точке $a$ при отображении $w = f(z)$ при условии $f'(a) \neq 0$. Конформные отображения. Теоремы
Х.\,Бора, Д.\,Е.\,Меньшова, теорема Римана о конформном отображении, соответствие границ при конформных
отображениях, теорема Каратеодори, принцип симметрии Римана-- Шварца, условия нормировки конформных
отображений (всё без доказательств).

\item Целые линейные и дробно-- линейные функции, их основные свойства.

Конформный гомеоморфизм, круговое
свойство, свойство симметрии. Сохранение сложного отношения четырёх точек, однозначная определённость
отображения тремя парами соответственных точек, неподвижные точки отображения. Групповые свойства. Общий вид
дробно-- линейного отображения круга на себя, верхней полуплоскости на себя и на круг.

\item Экспонента и логарифм, степень с произвольным показателем.

Экспонента, её периодичность и области
однолистности, конформные отображения с её помощью. Логарифм и степень с произвольным показателем, их
однозначные аналитические ветви и римановы поверхности, их производные. Конформные отображения с их помощью.

\item Функция Жуковского.

Функция Жуковского, её области однолистности и конформные отображения. Обратная
функция к функции Жуковского, её однозначные аналитические ветви и риманова поверхность.

\item Функции тригонометрические и гиперболические, обратные к ним функции.

Тригонометрические функции, их
периодичность и области однолистности, конформные отображения с их помощью. Обратные тригонометрические
функции. Функции гиперболической тригонометрии, связь с тригонометрическими функциями.

\item Интеграл по комплексному переменному.

Определение интеграла по комплексному переменному вдоль кривой. Ограниченность интегрируемой
фун-\break кции. Достаточные условия интегрируемости функции вдоль спрямляемой кривой. Оценки
уклонения интегральной суммы от интеграла. Примеры $\int 1\,dz$, $\int z\,dz$. Связь с
криволинейными интегралами от функций двух действительных переменных. Два критерия интегрируемости
функции вдоль спрямляемой кривой (без доказательства). Переход к пределу под знаком интеграла.
Интеграл как линейный непрерывный функционал. Аддитивность интеграла относительно пути
интегрирования. Сведение интеграла по комплексному переменному к интегралу по действительному
переменному. Производная и первообразная вдоль пути. Формула Ньютона-- Лейбница.

\item Интегральная теорема Коши, интегральная формула Коши, интеграл типа Коши.

Интегральная теорема Коши для односвязной и многосвязной области. Первообразная аналитической функции в
области. Теорема Мореры. Вычеты, две теоремы Коши о вычетах. Интегральная формула Коши. Интеграл типа Коши,
формулы для его производных. Бесконечная дифференцируемость аналитических функций, формулы Коши для
производных. Теорема Мореры. Первообразная аналитической функции в области.

\item Пространства аналитических функций.

Пространство $\As(G)$ всех функций, однозначных и
аналитических в области $G$, его счётная нормируемость и метризуемость. Теорема Вейерштрасса о
последовательностях и рядах аналитических функций. Полнота пространства $\As(G)$.

\item Степенные ряды и ряды по целым степеням.

Степенной ряд. Теорема Абеля, формула Коши-- Адамара.
Аналитичность суммы степенного ряда. Поведение ряда на границе круга сходимости. Ряди по целым степеням $(z-a)$,
их области сходимости, аналитичность их сумм. Разложение аналитических функций в ряды Тейлора и Лорана,
единственность разложения, формулы и неравенства Коши для коэффициентов. Теорема Лиувилля. Теорема Римана об
устранимой изолированной особой точке. Связь величины радиуса сходимости степенного ряда и величин
радиусов сходимости ряда Лорана с расположением особых точек сумм этих рядов. Теорема Фату о поведении
степенного ряда на границе круга сходимости (без доказательства). Действия со степенными рядами: арифметические
действия, почленное дифференцирование и интегрирование, ряд рядов, подстановка ряда в ряд, степенной ряд
обратной функции.

\item Теорема единственности и принцип максимума модуля.

Нули аналитической функции, порядок нуля, лемма о нулях.
Теорема единственности для аналитических функций. Принцип максимума модуля, его простейшие следствия. Лемма Шварца.

\item Изолированные особые точки однозначного характера.

Изолированные особые точки однозначного характера,
эквивалентность двух их классификаций. Полюс, порядок полюса, ряд Лорана в полюсе. Существенно особая точка,
теорема Сохоцкого-- Вейерштрасса. Теорема Пикара (без доказательства). Бесконечно удалённая точка как особая.
Формулы для вычисления вычетов. Применение вычетов к вычислению определённых интегралов. Лемма Жордана.
\end{nums}


\subsection*{Экзаменационные билеты}

\centerline{\footnotesize Заметим, что эти билеты (в отличие от программы экзамена) не соответствуют прочитанному курсу.}

\medskip

\begin{nums}{-2}
\item Комплексные числа, комплексная плоскость. Модуль и аргумент комплексного числа,
      их свойства. Многозначные функции и их однозначные ветви на примере аргумента $\Arg z$.
      Тригонометрическая и показательная формы записи комплексного числа. Метрика и топология комплексной плоскости:
      последовательности и пределы, бесконечно удалённая точка и расширенная комплексная плоскость,
      замкнутые и открытые множества, области и порядок их связности, компакты и континуумы, кривые.
\item Стереографическая проекция, её формулы. Круговое свойство. Хордальная метрика в $\ol \Cbb$.
\item Свойство конформности стереографической проекции: постоянство растяжений и сохранение углов.
      Угол с вершиной в бесконечности.
\item Симметрии сферы Римана и комплексной плоскости. Отображение~${w = \frac1z}$.
\item Функции комплексного переменного. Предел функции, непрерывность, дифференцируемость, условия Коши-- Римана.
      Правила дифференцирования. Аналитичность (голоморфность)
      функции в области и в точке (в том числе в бесконечности). Теорема Лумана-- Меньшова (без доказательства).
\item Геометрический смысл комплексной дифференцируемости. Конформное отображение. Теоремы
      Х.\,Бора, Д.\,Е.\,Меньшова (без доказательства).
\item Теорема Римана о конформном отображении, теорема Каратеодори о соответствии границ и принцип симметрии,
      условия нормировки конформного отображения (все без доказательства).
\item Целые линейные и дробно-линейные функции, их основные свойства: конформный гомеоморфизм, круговое
      свойство, свойство симметрии.
\item Целые линейные и дробно-линейные функции: сохранение сложного отношения четырех точек, однозначная
      определенность отображения тремя  парами  соответственных точек, неподвижные точки отображения,
      групповые свойства.
\item Общий вид дробно-линейного отображения круга на себя,  верхней  полуплоскости на себя и на круг.
\item Экспонента, ее периодичность и области однолистности, конформные отображения с ее помощью.
\item Логарифм и степень с произвольным показателем, их однозначные аналитические ветви и римановы поверхности,
      производные.  Применения  к  конформным отображениям.
\item Функция Жуковского, ее области однолистности; обратная к ней функция, ее однозначные аналитические
      ветви и риманова поверхность. Применения.
\item Функции тригонометрические и гиперболические, их периодичность  и области однолистности, обратные к ним
      функции. Конформные  отображения  с их помощью.
\item Области и кривые на плоскости. Определение интеграла по комплексному переменному вдоль кривой.
      Ограниченность интегрируемой функции.
\item Достаточные условия интегрируемости. Два критерия интегрируемости функции вдоль спрямляемой кривой
      (без доказательства).
\item Примеры интегрирования: интегралы от $f(z)=1$ и $f(z)=z$. Связь  интеграла по комплексному
      переменному с криволинейными интегралами 1-го и 2-го родов.
\item Переход к пределу под знаком интеграла. Интеграл как  линейный  непрерывный функционал. Аддитивность
      интеграла относительно пути интегрирования.
\item Сведение интеграла по комплексному переменному к интегралу по действительному переменному.
\item Производная и первообразная вдоль пути. Формула Ньютона-- Лейбница.
\item Интегральная теорема Коши в односвязной области.
\item Интегральная теорема Коши в области любой связности.
\item Вычеты, две теоремы Коши о вычетах. Интегральная формула Коши.
\item Интеграл типа Коши, формулы Коши для его производных. Бесконечная дифференцируемость аналитических функций.
\item Теорема Мореры. Первообразная аналитической функции.в области.
\item Пространство $\As(G)$ всех однозначных аналитических функций в области $G$, его счетная нормируемость и
      метризуемость. Теорема Вейерштрасса о последовательностях и рядах аналитических функций. Полнота
      пространства $\As(G)$.
\item Степенной ряд. Теорема Абеля и формула Коши-Адамара. Аналитичность суммы степенного ряда. Поведение
      ряда на границе его круга сходимости.
\item Ряды по целым степеням $(z-a)$, их области сходимости, аналитичность их сумм. Поведение ряда на границе
      его области сходимости.
\item Разложение аналитических функций в ряды Тейлора и Лорана, единственность разложения, формулы и
      неравенства Коши для коэффициентов.
\item Теорема Лиувилля, теорема Римана об устранимой изолированной особой точке. Связь величины радиуса
      сходимости степенного ряда и величин радиусов сходимости ряда Лорана с расположением особых точек сумм этих рядов.
\item Действия со степенными рядами: арифметические действия, почленное дифференцирование и интегрирование,
      ряд рядов, подстановка ряда в ряд, ряд обратной функции.
\item Нули аналитической функции, порядок нуля, лемма о нулях. Теорема единственности для аналитических функций.
\item Принцип максимума модуля, его простейшие следствия. Лемма Шварца.
\item Изолированные особые точки  однозначного характера,  эквивалентность двух их классификаций. Полюс,
      порядок полюса, ряд Лорана в полюсе.
\item Существенно особая точка, теорема Сохоцкого. Теорема Пикара (без доказательства). Бесконечно удаленная
      точка как изолированная особая.
\item Формулы для вычисления вычетов. Применение вычетов к вычислению интегралов. Лемма Жордана.
\end{nums}

\pagebreak

\section*{VI семестр}

\begin{nums}{-2}
\item Вычеты. Принцип аргумента и теорема Руше
\begin{items}{-2}
\item Формулы для вычисления вычетов. Применение вычетов к вычислению определенных интегралов. Лемма Жордана.
\item Логарифмический вычет, его обобщение.
\item Один способ вычисления нулей аналитической функции.
\item Принцип аргумента. Теорема Руше.
\end{items}
\item Аналитическое продолжение
\begin{items}{-2}
\item Устранимые множества неизолированных особых точек: хаусдорфовы меры, устранимые множества для
      ограниченных аналитических функций, устранимые множества точек для аналитических функций,
      непрерывных в своих особых точках.
\item Аналитическое продолжение через границу области. Принцип симметрии Римана-- Шварца.
\item Аналитическое продолжение по цепи и вдоль пути. Полная аналитическая функция, ее риманова
      поверхность и особые точки.
\item Теорема о монодромии. Области распадения функции $\sqrt z$ на однозначные аналитические ветви.
\item Модулярные функции для круга и полуплоскости. Малая теорема Пикара. Большая теорема Пикара (без доказательства).
\end{items}
\item Компактные семейства аналитических функций
\begin{items}{-2}
\item Компактные семейства однозначных аналитических функций, критерий Монтеля.
\item Теорема Витали о сходящихся последовательностях аналитических функций.
\item Теорема Гурвица о нулях сходящейся последовательности аналитических функций. Теорема о пределе
      последовательности однолистных функций.
\end{items}
\item Отображения посредством аналитических функций
\begin{items}{-2}
\item Локальное обращение аналитической функции: лемма о локальном обращении, принцип области
      (принцип открытости), его следствия. Обращение аналитической функции. Критерий локальной однолистности.
      Критерий конформности отображения в точке.
\item Дробно-линейность конформных отображений круговых областей друг на друга.
\item Теорема Римана о конформном отображении.
\item Соответствие границ при конформных отображениях областей. Понятие о теореме Каратеодори. Обратный
      принцип соответствия границ.
\item Конформно эквивалентные области. Условия единственности конформного отображения (условия нормировки).
\end{items}
\item Гармонические функции двух переменных
\begin{items}{-2}
\item Гармонические функции двух переменных, их связь с аналитическими функциями. Гармонически
      сопряженные функции. Бесконечная дифференцируемость гармонических функций. Поток и циркуляция градиента.
\item Инвариантность гармоничности при голоморфной замене переменных.
\item Принцип экстремума и теорема о среднем для гармонических функций.
\item Аналитичность комплексно сопряженного градиента гармонической функции. Теоремы единственности
      для гармонических функций.
\item Теорема Лиувилля для гармонических функций. Теорема Харнака об устранимой особой точке.
\item Гармонические полиномы, связь их с тригонометрическими полиномами.
\item Интегралы Пуассона и Шварца, их ядра. Задача Дирихле в круге и жордановых областях.
\item Гармоническое продолжение. Принцип отражения.
\item О приложениях к гидромеханике и электростатике.
\end{items}
\item Начала операционного исчисления
\begin{items}{-2}
\item Идея операционного исчисления и преобразование Лапласа.
\item Оригинал и изображение, аналитичность изображения, стремление его к нулю при $\Rea z\ra\bes$.
\item Линейность преобразования Лапласа, теоремы подобия, запаздывания оригинала и смещения изображения.
\item Дифференцирование и интегрирование оригинала и изображения.
\item Свертка оригиналов, её изображение. Изображение произведения оригиналов (без доказательства).
\item Формула обращения преобразования Лапласа, достаточное условие на функцию для того, чтобы она
      была изображением (всё без доказательств).
\item Операционный метод решения линейных дифференциальных уравнений.
\end{items}
\end{nums}

\section*{Литература}
\begin{nums}{-3}
\item Шабат\,Б.\,В. \emph{Введение в комплексный анализ.}-- М.: 1976.
\item Маркушевич\,А.\,И. \emph{Краткий курс теории аналитических функций.}-- М.: 1978.
\item Маркушевич\,А.\,И. \emph{Теория аналитических функций.}-- М.: 1968.
\item Привалов\,И.\,И. \emph{Введение в ТФКП.}-- М.: 1984.
\item Лаврентьев\,М.\,А., Шабат\,Б.\,В. \emph{Методы ТФКП.} М.: 1987.
\item Долженко\,Е.\,П., Николаева\,С.\,Н. \emph{ТФКП. Методические указания.}
\item Свешников\,А.\,Г., Тихонов А.Н., \emph{ТФКП.}-- М.: 1974.
\item Сидоров\,Ю.\,В., Федорюк\,М.\,В., Шабунин\,М.\,И., \emph{Лекции по ТФКП.}-- М.: 1989.
\end{nums}

\medskip\dmvntrail

\end{document}


%% Local Variables:
%% eval: (setq compile-command (concat "latex  -halt-on-error -file-line-error " (buffer-name)))
%% End:
