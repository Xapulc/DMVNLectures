\documentclass[unicode,10pt]{article}
\usepackage{dmvn}
\usepackage{polyglossia}
\usepackage{unicode-math}
\usepackage{fontspec}
\usepackage{epsfig}
\usepackage{ulem}
\usepackage{wrapfig}
\defaultfontfeatures{Scale=MatchLowercase, Mapping=tex-text}
\setmainfont{CMU Serif}
\setsansfont{CMU Sans Serif}
\setmonofont{CMU Typewriter Text}
\setmathfont{xits-math.otf}
\newcommand{\questionably}[1]{{
    \setmainfont{URW Chancery L}
    #1}}
\newcommand{\contrib}[1]{
  {
  \setmainfont{URW Chancery L}
  \uwave{#1}
  }}
\newcommand{\veca}{\vec A}
\newcommand{\vecb}{\vec B}
\newcommand{\vece}{\vec E}
\newcommand{\eX}{\vec e_x}
\newcommand{\eY}{\vec e_y}
\newcommand{\eZ}{\vec e_z}
\newcommand{\eR}{\vec e_r}
\newcommand{\unsafeIO}{
  \begin{petit}
    {\Large \bf\#!} Творчество верстальщика.
  \end{petit}
}
\begin{document}
\begin{tproblem}
  Построить электростатический потенциал (решение уравнения Пуассона)
  в $R^n$, $n≥2$ для
  \begin{itemize}
  \item Точечного заряда
  \item Заряженного шара (внутри и снаружи)
  \item Заряда, равномерно распределенного по двумерной плоскости ($n≥3$)
  \end{itemize}
\end{tproblem}

\begin{tproblem}
  Построить электростатический потенциал $φ(x,y,z,θ)$ точечного заряда
  в пространстве $M=R³×S¹$ с компактными дополнительным измерением:
  $dl² = dx² + dy² + dz² + ρ²_0dθ²$, $ρ_0=\const$, $0≤ θ ≤ 2π$,
  $(x,y,z) ∈R³$. Считать, что заряд покоится в точке
  $x=y=z=θ=0$. Рассмотреть предельные случаи $φ(x,y,z,0)$ при $r <\!<
  ρ_0$ и $r >\!> ρ_0$, где $r²=x²+y²+z²$.
\end{tproblem}
\begin{hint}
  Искомый потенциал является решением уравнения Пуассона в $M$
  \begin{equation*}
    ∂²_x +∂²_y+∂²_z+\frac{1}{ρ²_0}∂²_θ = \frac{q}{ρ_0}δ(x)δ(y)δ(z)δ(θ)
  \end{equation*}
\end{hint}
\begin{tproblem}
  Нерелятивистский заряд падает под углом на плоскость $z$, несущую
  заряд с плотностью
  \begin{equation*}
    ρ=λ\frac{d}{dz}δ(z)
  \end{equation*}
  (двойной электрический слой). Под каким углом заряд выйдет с другой
  стороны плоскости? (механического взаимодейтвия нет)
\end{tproblem}

\begin{tproblem}
  Равномерно заряженная сфера радиуса $R$ и заряда $Q$ разрезана на
  две половины. Найти силу взаимодействия половин.
\end{tproblem}

\begin{tproblem}
  Найти силу взаимодействия пластин с током в магнитном конденсаторе
  на единицу площади (противоположные токи в паралельных плоскостях).
\end{tproblem}
\begin{tproblem}
  Найти индукционное электрическое поле, порождаемое магнитным полем тока
  \begin{equation*}
    \begin{array}{cc}
      j_x = iδ(z)\cos ωt, &j_y = j_z = 0
    \end{array}
  \end{equation*}
\end{tproblem}

\begin{tproblem}
  Полупространство $z > 0$ заполнено однородным магнитным полем,
  направленным паралельно плоскост $z = 0$. Заряд падает из области $z
  < 0$ под углом $α$ к плоскости. Найти траекторию дальнейшего движения.
\end{tproblem}
\unsafeIO{}
\begin{solution}
  \newcommand{\vecr}{{\vec r}}
  Пусть положение частицы описывается её координатами $\vec r =
  x\eX+y\eY+z\eZ$. Не ограничивая общности, полагаем, что $\vecr(0) =
  0$, $\dot\vecr(0) = ξ\cos α\eX + ξ\sin α\eZ$, $ξ = \const$, $\vecb =
  B\eX$ На частицу действует сила Лоренца $F$, вычисляемая по формуле
  \begin{equation*}
    \vec F = \frac ec\mbmat{\eX & \eY & \eZ \\ \dot x & \dot y & \dot z \\ B & 0 & 0} = B\frac ec(\dot z\eX - \dot y\eZ)
  \end{equation*}
  Записываем второй закон Ньютона
  \begin{equation*}
    m(\ddot x\eX +\ddot y\eY + \ddot z\eZ) = B\frac ec(\dot z\eY - \dot y\eZ)
  \end{equation*}
  Покоординатно приравниваем, вводя обозначение $ ω  ≝ B\frac{e}{cm}$
  \begin{equation*}
    \begin{array}{ccc}
      \ddot x  = 0 & \ddot y = ω\dot z & \ddot z = -ω\dot y
    \end{array}
  \end{equation*}
  Решим систему на $y$,$z$, введя замену $ u = \dot y$, $v = \dot z$.
  Система переписывается как
  \begin{equation*}
    \bcase{\dot u = ωv\\\dot v = -ωu}
  \end{equation*}
  Продифференцировав первое уравнение и подставив второе, получаем уравнение на $u$:
  \begin{equation*}
    \ddot u + ω²u = 0
  \end{equation*}
  откуда, $u = a\cos ωt + b\sin ωt$, используя систему, $ v = -a\sin
  ωt + b\cos ωt $.  Возвращаемся к исходным переменным $x$,$y$,$z$
  (т.к $a$,$b$ -- произвольные констатны, то в них можно загнать $ω$
  при интегрировании)
  \begin{equation}
    \begin{array}{l}
      x = εt + β \\
      y = a\sin ωt - b\cos ωt + c_1 \\
      z = a\cos ωt + b\sin ωt +c_2
    \end{array}
  \end{equation}
  Подставим начальные условия. Из $\vecr = 0$ следует,
  что $β = 0$, $ c_1 = b$, $ c_2 = -a$. Тогда у нас остается три неизвестные:
  \begin{equation*}
    \begin{array}{l}
       x = εt\\
       y = a\sin ωt - b\cos ωt + b\\
       z = a\cos ωt + b\sin ωt -a
    \end{array}
  \end{equation*}
  Дифференцируем, подставляем $t = 0$ и используем начальную скорость
  \begin{equation*}
    \begin{array}{l}
      \dot x = ε\\
      \dot y = aω\cos ωt + bω\sin ωt\\
      \dot z = -aω\sin ωt + bω\cos ωt
    \end{array}
  \end{equation*}
  Откуда, $ε = ξ\cos α$, $aω = 0$, $bω = ξ\sin α$.
  Записываем ответ
  \begin{equation*}
    \begin{array}{ccc}
      x = ξ\cos α t & y = \frac{ξ\sin α}ω(1- \cos ωt)  & z = \frac{ξ\sin α}{ω}\cos ωt
    \end{array}
  \end{equation*}
  Получается занятный эффект -- частица описывает некую дугу в области
  $ z > 0$ и через время $ t= \frac{π}{2ω}$ вылетает из магнитного поля.
\end{solution}

\begin{tproblem}
  Рассчитать магнитное поле равномерно вращающейся сферы.
\end{tproblem}

\begin{tproblem}
  Найти вектор-потенциал магнитного поля, создаваемого прямолинейным тонким током.
\end{tproblem}
\unsafeIO{}
\begin{solution}
  Сначала найдем магнитное поле $\vecb$. Не ограничивая общности,
  можно считать, что плотность электрического тока равна $\vec\jmath =
  jδ(x)δ(y)\eZ$.  Из соображений симметричности, $\vecb = B_z\eZ +
  B_r\eR$. По теореме Стокса(стр. 23), взяв в качестве $S$ круг
  радиуса $r$ в плоскости $xy$
  \begin{equation*}
    ∮\limits_{S¹_r}(B_z\eZ + B_r\eR)dl =\frac{4π}c\intl{D²_r}{}jδ(x)δ(y)\eZ dS = \frac{4π}cj\eZ
  \end{equation*}
  откуда $B_r = 0$, $B_z = \frac{2j}{cr}$.  Т.е $\vecb = (0, 0,
  \frac{2j}{cr})$.  Вычислим поток магнитного поля через круг
  радиуса $r$ в плоскости $xy$.
  \begin{equation*}
    Φ ≡ \intl{D²_r}{}\vecb ·dS = \intl{S¹_r}{}\veca ·dl
  \end{equation*}
  Откуда заключаем, что $\veca = A(r)\eZ$. Вычислим интегралы.
  \begin{eqnarray*}
    \intl{D²_r}{}\frac{2j}{cr'}dS =\hc{\mat{ x = r'\sin φ\\y = r'\cos φ}} = \intl or\intl 0{2π} \frac{2j}c dr'dφ
  \end{eqnarray*}
  Бредово получается, что $\veca = \const$. \tbk
\end{solution}

\begin{tproblem}
  Найти магнитное поле постоянного тока плотности $j$, текущего по
  поверхности цилиндра радиуса $R$
  \begin{enumerate}
  \item паралельно оси
  \item вдоль окружностей.
  \end{enumerate}
\end{tproblem}
\unsafeIO{}
\begin{solution}
Заметим, что ввиду симметричности задачи, вектор магнитного поля имеет
вид $\vecb = B_r\vec e_r + B_z\vec e_z$. По формуле Стокса(стр. 23)  имеем
\begin{equation*}
   ∮_{S¹_r}(B_r\vec e_r + B_z\vec e_z)dl = \frac{4π}c\intl{D²_r}{}\vec\jmath dS
\end{equation*}
В первом случае, $\vec\jmath = jδ(r-R)\vec e_z$, откуда
\begin{equation*}
  \intl{D²_r}{}\vec\jmath dS = 2πj\vec e_zδ_{r≥R}
\end{equation*}
откуда заключаем,что
\begin{equation*}
   ∮_{S¹_r}(B_r\vec e_r + B_z\vec e_z)dl = \frac{4π}c 2πj\vec e_zδ_{r≥R}
\end{equation*}
что означает, что
\begin{equation*}
  \begin{array}{cc}
    B_r ≡ 0 &  2πrB_z =\frac{4π}c j\vec e_z2πRδ_{r≥R}
  \end{array}
\end{equation*}
  Упрощая, $B_z = \frac{4π}cj\frac Rrδ_{r≥R}$.  Во втором случае,
  вектор $\vec\jmath$ направлен по касательной к окружности, и в
  правой части уравнения будет $0$ вне зависимости от радиуса круга $r$.
  Потому во втором случае магнитное поле равно 0.
\end{solution}

\begin{tproblem}
  Найти связь между средним магнитным полем в круге и значением поля
  на окружности в бетатроне (ускорителе электронов, в котором
  ускоряющее электрическое поле создается благодаря электро-магнитной
  индукции неоднородным, зависящим от времени магнитным полем
  $B_z(ρ,t)$, $ρ² = x² + y²$, причем электроны ускоряются, оставаясь на окружности постоянного радиуса)
\end{tproblem}
 

\begin{tproblem}
  Записать преобразование Лоренца в форме $SL(2,\Cbb)$.
\end{tproblem}

\begin{tproblem}
  Показать, что композиция двух преобразований Лоренца с
  непаралельными скоростями не является чистым преобразованием Лоренца.
\end{tproblem}

\begin{tproblem}
  \newcommand{\ω}{ω\hr{t - z/c}}
  С помощью метода Гамильтона-Якоби найти закон движения и траекторию
  релятивистского заряда в поле плоской волны круговой поляризации,
  описываемой потенциалом
  \begin{equation*}
    \veca = A_0 (\cos\hr\ω, \sin\hr\ω, 0)
  \end{equation*}
\end{tproblem}

\begin{tproblem}
  Найти время жизни электрона, движущегося по круговой орбите в атоме
  водорода, при учете излучения. То же для заряда, вращающегося в
  однородном магнитном поле.
\end{tproblem}

\begin{tproblem}
  Показать, что радиальные геодезические в поле Шварцшильда
  приближаются горизонту событий за бесконечное время по часам
  удаленного наблюдателя. Каково время подения по собственным часам?
\end{tproblem}

\begin{tproblem}
  Рассчитать параметры всеъ круговых орбит массивных и безмассовых
  частиц в поле Шварцшильда.
\end{tproblem}

\begin{tproblem}
   Дать классификацию экваториальны изотропных геодезических в поле
   Шварцшильда, приходящих из бесконечности.
\end{tproblem}

\end{document}
