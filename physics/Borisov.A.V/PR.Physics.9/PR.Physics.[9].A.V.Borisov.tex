\documentclass[a4paper,12pt]{article}

\usepackage[simple]{dmvn}

\begin{document}

\begin{center}
\large{\textbf{ПРОГРАММА КУРСА ``ФИЗИКА. Ч. 2'' \\ \textit{для
студентов отделения математики\\ механико-математического
факультета}}}
\end{center}

\bigskip

\noindent ~~1. Уравнение Шрёдингера. [\textbf{С}, 2],
[\textbf{Л1}, 17], [\textbf{Ф}, 9]

\noindent ~~2. Волновая функция. [\textbf{С}, 1, 2, 3],
[\textbf{Л1}, 2], [\textbf{Ф}, 11]

\noindent ~~3. Наблюдаемые и операторы. [\textbf{С}, 2],
[\textbf{Л1}, 3], [\textbf{Ф}, 10, 13]

\noindent ~~4. Принцип суперпозиции. [\textbf{С}, 3],
[\textbf{Л1}, 2], [\textbf{Ф}, 6]


\noindent ~~5. Соотношение неопределённостей Гейзенберга.
[\textbf{С}, 6], [\textbf{Л1}, 16], [\textbf{Ф}, 7]

\noindent ~~6.  Изменение наблюдаемых со временем. [\textbf{С},
6], [\textbf{Л1}, 9], [\textbf{Ф}, 9]

\noindent ~~7. Гармонический осциллятор. [\textbf{С}, 7],
[\textbf{Л1}, 23], [\textbf{Ф}, 16, 17]

\noindent ~~8. Оператор момента импульса. [\textbf{С}, 10],
[\textbf{Л1}, 26-28], [\textbf{Ф}, 23-25, 29, 30]

\noindent ~~9. Спин. [\textbf{С}, 16], [\textbf{Л1}, 54-56],
[\textbf{Ф}, 27, 46]


\noindent 10. Уравнение Паули. [\textbf{С}, 16], [\textbf{Л1},
111]

\noindent 11. Движение в центрально-симметричном поле.
[\textbf{С}, 10], [\textbf{Л1}, 32], [\textbf{Ф}, 26, 31]

\noindent 12. Атом водорода. [\textbf{С}, 12], [\textbf{Л1}, 36],
[\textbf{Ф}, 32]


\noindent 13. Тождественные частицы. Принцип Паули. [\textbf{С},
24], [\textbf{Л1}, 61], [\textbf{Ф}, 48, 49]

\noindent 14. Каноническое распределение (распределение Гиббса).
[\textbf{Л2}, 28, 31]


\noindent 15. Термодинамические параметры и потенциалы.
[\textbf{Л2}, 9, 12, 13, 15]


\noindent 16. Идеальный классический газ. Распределение Больцмана.
[\textbf{Л2}, 29, 41, 42]

\noindent 17. Распределение Гиббса с переменным числом частиц.
[\textbf{Л2}, 35]


\noindent 18. Распределение Ферми--Дирака. Идеальный ферми-газ.
[\textbf{Л2}, 53, 56-58]

\noindent 19. Распределение Бозе--Эйнштейна. Идеальный бозе-газ.
[\textbf{Л2}, 54, 56, 62]


\noindent 20. Распределение Планка. Фотонный газ. [\textbf{Л2},
63]


\vspace{2cm}

\centerline{\bf Литература}

\bigskip

\noindent \textbf{С}. А. А. Соколов, И. М. Тернов, В. Ч.
Жуковский. \textit{Квантовая механика}. \newline
 М.: Наука, 1979. \newline
\textbf{Л1}. Л. Д. Ландау, Е. М. Лифшиц. \textit{Квантовая механика}. М.:
Физматлит, 2001. \newline
 \textbf{Ф}. Л. Д. Фаддеев, О. А. Якубовский. \textit{Лекции по квантовой
механике \newline
 для студентов-математиков}. Л.: Изд-во ЛГУ, 1980. \newline
\textbf{Л2}. Л. Д. Ландау, Е. М. Лифшиц. \textit{Статистическая физика. Ч.
1}. \newline
 М.: Физматлит, 2001.

\bigskip

\centerline{\bf Дополнительная литература}

\bigskip

\noindent 1. Ф. А. Березин, М. А. Шубин. \textit{Уравнение
Шрёдингера}. М.: Изд-во Моск. ун-та, 1983. \newline 2. И. А.
Квасников. \textit{Термодинамика и статистическая физика. Т. 2.
Теория равновесных систем.} М.: УРСС, 2002.
\newline 3. З. Флюгге. \textit{Задачи по квантовой механике}. Т.
1, 2. М.: Мир, 1974.
\newline 4. Р. Кубо. \textit{Статистическая механика}. М.: Мир,
1967. \newline 5. Л. де Бройль. \textit{Революция в физике}. М.:
Атомиздат, 1965.\newline 6. А. В. Борисов. \textit{Основы
квантовой механики}. М.: Изд-во физич. ф-та МГУ, 1999.

\newpage
\mbox{} %\vspace{-2cm}
\begin{center}
\textbf{ЗАДАЧИ \\ \textit{по курсу ``ФИЗИКА. Ч. 2'' для студентов
отделения математики механико-математического факультета}}
\end{center}

\bigskip

%%%%%%%%%%%%%%%%%%%%%%%%%%%%%%%%%%%%%%%%%%%%%%%%%%%%%%%%%%%%%%%%
\noindent\textbf{~1. Состояние свободной частицы при $t = 0$ имеет вид:
$\psi \left( {0,x} \right) = A\exp \left( { - x^2 / 2a^2 + ik_0 x}
\right).$ Найти при $t > 0$ средние значения $\left\langle x
\right\rangle$, $\left\langle {p_x } \right\rangle$, $\left\langle
{\left( {x - \left\langle x \right\rangle } \right)^2}
\right\rangle$, $\left\langle {\left( {p_x - \left\langle {p_x }
\right\rangle } \right)^2} \right\rangle $.}

Уравнение Шредингера имеет вид
$\displaystyle \hat H\psi(x,t)=i\hbar\frac{\partial\psi(x,t)}{\partial t}$,
$\displaystyle \hat H=\frac{\hat p^2}{2m}=-\frac{\hbar^2}{2m}\frac{\partial^2}
{\partial x^2}$.

Найдем $A$ из условия нормировки $\int|\psi(x,t)|^2dx=1$. В частности, при
$t=0$ получим $\int|\psi(x,0)|^2dx=A^2\int\exp\left(-\frac{x^2}{2a^2}\right)dx=
A^2\sqrt\pi=1$, откуда $\displaystyle A^2=\frac1{a\sqrt\pi}$
(использованы формулы $|e^z|=e^{\mathop{\rm Re}z}$ и
$\int\exp\left(-\frac{x^2}2\right)dx=\sqrt\pi$).

Найдем $\displaystyle\psi(p,t)=\frac1{\sqrt{2\pi\hbar}}\int\psi(x,t)
e^{\frac{ipx}\hbar}dp$. Для этой функции справедливо уравнение Шредингера
в импульсном представлении:
$\displaystyle i\hbar\frac{\partial\psi(p,t)}{\partial t}=
   \frac{p^2}{2m}\psi(p,t)$.

$$\psi(p,t)=C\exp\left(-\frac{ip^2}{2m}\hbar t\right),$$
$$C=\psi(p,0)=
  \frac1{2\pi\hbar}\int\psi(x,0)e^{-\frac{ipx}\hbar} dx=
  \frac A{\sqrt{2\pi\hbar}}\int\exp
     \left(-\frac{x^2}{2a^2}+i(k_0-\frac p\hbar)x\right)dx.$$

При $\mathop{\rm Re}\alpha>0$ верна формула
  $\displaystyle\int e^{-\alpha x^2+\beta x}dx=
   \sqrt{\frac\pi\alpha}e^{\frac{\beta^2}{4\alpha}}$, откуда
$\displaystyle C=\frac{Aa}{\sqrt\hbar}
  \exp\left(
    -\frac{\left(k_0-\frac p\hbar\right)^2}2
  \right)a^2$.

$$\rho(p,t)=|\psi(p,t)|^2=\frac{A^2a^2}\hbar\exp\left(-a^2\left(k_0-\frac{p^2}
\hbar\right)^2\right).$$

$$\langle p\rangle=\int p\rho(p,t)dp=\frac{A^2a^2}\hbar
\int p\exp\left(-a^2\left(k_0-\frac{p^2}\hbar\right)^2\right)=\ldots=\hbar k.$$
%
(интеграл вида
$\displaystyle p^ne^{-\alpha p^2+\beta p}dp$ можно найти как
$\displaystyle\left.\left(\frac{\partial^n}{\partial\lambda^n}
 \int e^{-\alpha p^2+\lambda p}dp\right)\right|_{\lambda=\beta}$).

Аналогично находим $\langle p^2\rangle$ и следом
$\langle(p-\langle p\rangle)^2\rangle=\langle p^2\rangle-\langle p\rangle^2=
\ldots=\frac{\hbar^2}{2a^2}$.

Найдем
$$\psi(x,t)=\frac1{\sqrt{2\pi\hbar}}\int\psi(p,t)e^{\frac{ipx}\hbar}dp=
  \frac{Aa}{\sqrt{2\pi\hbar}}\int\exp\left(
    -p^2\left(\frac12\frac{a^2}{\hbar^2}+\frac{it}{2m\hbar}\right)+\right.$$
$$\left.+p\frac{a^2k_0+ix}\hbar-\frac12a^2k_0^2\right)dp=
\ldots=\frac A{\sqrt{1+\frac{i\hbar t}{ma^2}}}\exp\frac
  {-\frac{x^2}{2a^2}+i\left(k_0x-k_0^2\frac{\hbar t}{2m}\right)}
  {1+\frac{i\hbar t}{ma^2}}.$$

$$\rho(x,t)=|\psi(x,t)|^2=\frac{A^2}{\sqrt{1+\frac{\hbar^2}{m^2a^4}}}
  \exp\left(-\frac{\left(x-\frac{k_0\hbar}{m}t\right)^2}
  {a^2\left(1+\frac{\hbar^2t^2}{m^2a^4}\right)}\right).$$

Вообще говоря, $\langle x\rangle=\int x\rho(x,t)dx$,
однако можно воспользоваться тем, что если $\xi$ распределена по
Гауссовскому закону
%
$\rho(\xi)=\frac1{\sqrt{2\pi\sigma^2}}e^{-\frac{(x-\bar x)^2}{2\sigma^2}},$
%
то $\langle\xi\rangle=\bar x$ и
$\langle(\xi-\langle\xi\rangle)^2\rangle=\sigma^2$. Отсюда
%
$$\langle x\rangle=\frac{k_0\hbar}mt,\quad
  \langle(x-\langle x\rangle)^2\rangle=\frac12a^2\left(
    1+\frac{\hbar^2t^2}{m^2a^4}\right).$$

%%%%%%%%%%%%%%%%%%%%%%%%%%%%%%%%%%%%%%%%%%%%%%%%%%%%%%%%%%%%%%%%
\noindent\textbf{~2. Найти коэффициент прохождения частицы через
потенциальный барьер
\[
U\left( x \right) = \left\{ {\begin{array}{l}
 U_0 > 0,\;x \in \left[ {0,a} \right]\,; \\
 0,\;x \notin \left[ {0,a} \right]\,. \\
 \end{array}} \right.
\]}

Рассмотрим отдельно три области:
   I $(-\infty,0)$, II $(0,a)$, III $(a,+\infty)$.
Частица, обладающая волновой функцией
   $\psi_{-\infty}=e^{i\left(kx-\frac{Et}\hbar\right)}$, $k:=\frac p\hbar$,
движется через потенциальный барьер.

Пусть $E>U_0$. В области I уравнение Шредингера $\hat H\varphi=E\varphi$
имеет вид: $-\frac{\hbar^2}{2m}\varphi''=E\varphi$, откуда
$\varphi_I=e^{ikx}+Ae^{-ikx}$ (коэффициент при $e^{ikx}$ равен 1 из-за
того, что $\varphi_I$ должно быть ассимптотически равно $\varphi_{-\infty}$
при $x\to-\infty$).

В области II уравнение Шредингера имеет вид
$-\frac{\hbar^2}{2m}\varphi''=(E-U_0)\varphi$, т.е.
$\varphi''+\frac{2m}{\hbar^2}(E-U_0)\varphi=0$. Обозначим
$k_2:=\frac{\sqrt{2m(E-U_0)}}\hbar$, тогда
$\varphi_{II}=Be^{ik_2x}+Ce^{-ik_2x}$.

В области III уравнение Шредингера имеет такой же вид, как и в области I,
$k_1:=\frac{\sqrt{2mE}}{\hbar}$ и $\varphi_{III}=Ge^{ik_1x}+Me^{-ik_1x}$ (!?!?
почему-то сразу полагаем $M:=0$?).

$\varphi''$ имеет в $0$ и $a$ конечный скачок, откуда $\varphi'$ и $\varphi$
обязаны быть непрерывными в $0$ и $a$:
\[
  \left\{\begin{array}{lcl}
    \varphi_I(0)=\varphi_{II}(0)      &\Rightarrow&1+A=B+C\\
    \varphi_I'(0)=\varphi_{II}'(0)    &\Rightarrow&k_1(1-A)=(B-C)k_2\\
    \varphi_{II}(a)=\varphi_{III}(a)  &\Rightarrow&B^{ik_2a}+Ce^{-ik_2a}=
                                          Ge^{ik_1a}\\
    \varphi_{II}'(a)=\varphi_{III}'(a)&\Rightarrow&k_2(B^{ik_2a}-Ce^{-ik_2a})=
                                          k_1Ge^{ik_1a}\\
  \end{array}\right.
\]

Имеем 4 уравнения и 4 неизвестных ($A$, $B$, $C$, $G$), откуда легко
найти $\varphi$.

Нам надо найти
$D(E)=\frac{|\vec j(\varphi_{III})|}{|\vec j(\varphi_{-\infty})|}$,
где
$\vec j(\varphi)=\frac{\hbar}{2mi}\left(\varphi^*\frac {d\varphi}{dx}-
\varphi\frac{d\varphi^*}{dx}\right)$~--- вектор плотности потока
вероятностей. Нетрудно видеть, что
$$D(E)=|G|^2=\ldots
=\frac{4k_1^2k_2^2}{(k_1^2-k_2^2)^2\sin^2k_2a+4k_1^2k_2^2},
k_1=\frac{\sqrt{2mE}}\hbar, k_2=\frac{\sqrt{2m(E-U_0)}}\hbar.$$

Случай $E<U_0$ аналогичен рассмотренному, только $k_2=i\varkappa$ является
чисто мнимой величиной, $\varphi_{II}=Be^{\varkappa x}+Ce^{-\varkappa x}$ и
$$D(E)=\frac{4k_1^2\varkappa^2}
     {(k_1^2+\varkappa^2)^2\mathop{\rm sh}^2\varkappa a+4k_1^2\varkappa^2},
k_1=\frac{\sqrt{2mE}}\hbar, \varkappa=\frac{\sqrt{2m(U_0-E)}}\hbar.$$

%%%%%%%%%%%%%%%%%%%%%%%%%%%%%%%%%%%%%%%%%%%%%%%%%%%%%%%%%%%%%%%%
\noindent\textbf{~3. Найти уровни энергии частицы в потенциальной яме
\[
U\left( x \right) = \left\{ {\begin{array}{l}
 - U_0 < 0,\;\left| x \right| < a; \\
 0,\;\left| x \right| > a. \\
 \end{array}} \right.
\]}

Лемма. Если $U$~--- четная функция ($U(x)=U(-x)$), то $\varphi_n$ либо
четная функция, либо нечетная.

В самом деле, рассмотрим оператор $\hat P$ $\hat Pf(x)=f(-x)$, тогда для
его собственной функции $g$ с собственным значением $\lambda$ верно
$\hat P^2g(x)=\lambda^2g(x)=g(x)$, откуда $\lambda=\pm1$. Далее, в силу
коммутативности оператора $\hat H=-\frac{\hbar^2d^2}{2m\,dx^2}+U(x),
U(x)=U(-x)$, с оператором $\hat P$ верно, что $\hat H\hat Pf(x)=
\hat P\hat Hf(x)$. Для собственной функции $\varphi$ оператора $\hat H$
рассмотрим
$\hat H\hat P\varphi=\hat P\hat H\varphi=\hat PE\varphi=E\hat P\varphi$,
откуда $\hat P\varphi$~--- собственная функция $\hat H$. Значит,
$\hat P\varphi$ пропорциональна $\varphi$ и
$\hat P\varphi(x)=\varphi(-x)=\lambda\varphi(x)=\pm\varphi(x)$.

!!! Найти продолжение решения!

%%%%%%%%%%%%%%%%%%%%%%%%%%%%%%%%%%%%%%%%%%%%%%%%%%%%%%%%%%%%%%%%
\noindent\textbf{~4. Найти уровни энергии частицы в поле $U\left( x
\right) = - g\delta \left( x \right),\;g > 0.$}

Как и в задаче 3, можно ограничиться только четными и нечетными решениями.

Уравнение Шредингера имеет вид:
%
$$-\frac{\hbar^2}{2m}\psi''(x)-g\delta(x)\psi(x)=E\psi(x).$$
%
Так как $\psi$ непрервына в 0, а $\psi''$ пропорциональна $\delta$, то
$\psi'$ разрывна в 0.

Рассмотрим две области: I $(-\infty,0)$, II $(0,+\infty)$.
В каждой из них уравнение Шре\-дин\-ге\-ра
имеет вид $\psi''-\frac{2m|E|}{\hbar^2}
\psi=0$. Обозначим $k:=\frac{\sqrt{2m|E|}}\hbar$. Тогда $\psi_I=Ae^{kx}+
Be^{-kx}$ и из требования $\lim\limits_{x\to-\infty}\psi(x)=0$
следует $B=0$. Аналогично
$\psi_{II}=Ce^{-kx}$. В силу четности/нечетности $C=\pm A$, более того,
нечетный случай невозможен, так как в этом случае для непрерывности $\psi$
в нуле необходимо $\psi(0)=0$, а это возможно лишь при $A=C=0$, т.е.
$\psi(x)\equiv0$. Таким образом, есть лишь четное решение $\psi(x)=Ae^{-k|x|}$.

Проинтегрируем обе части уравнения Шредингера
%
%$$-\frac{\hbar^2}{2m}\psi''(x)-g\delta(x)\psi(x)=E\psi(x)$$
%
на интервале $[-\varepsilon,\varepsilon]$.
%
  $$-\frac{\hbar^2}{2m}
     \int\limits_{-\varepsilon}^\varepsilon\psi''(x)dx-
    g\int\limits_{-\varepsilon}^\varepsilon\delta(x)\psi(x)dx=
    E\int\limits_{-\varepsilon}^\varepsilon\psi(x)dx;$$
  $$-\frac{\hbar^2}{2m}\psi'(x)\bigg|_{x=-\varepsilon}^\varepsilon-
    g\psi(0)=E\int\limits_{-\varepsilon}^\varepsilon\psi(x)dx.$$

Теперь возьмем предел при $\varepsilon\to0$.
%
  $$-\frac{\hbar^2}{2m}(\psi'(+0)-\psi'(-0))-g\psi(0)=0;$$
  $$-kA-kA=-\frac{2mg}\hbar A\mbox{, или }k=\frac{mg}{\hbar^2}.$$

Отсюда $E=-|E|=-\frac{\hbar^2k^2}{2m}=-\frac{mg^2}{\hbar^2}$.

(Изредка на экзамене спрашивают особый случай $E=0$, когда $\psi(x)\equiv0$)

%%%%%%%%%%%%%%%%%%%%%%%%%%%%%%%%%%%%%%%%%%%%%%%%%%%%%%%%%%%%%%%%
\noindent\textbf{~5. Найти уровни энергии частицы в поле $U\left( x
\right) = - g\left[ {\delta \left( {x + a} \right) + \delta \left(
{x - a} \right)} \right]$, $g > 0.$}

Пусть $E=-|E|<0$.
Как и в задаче 3, можно ограничиться рассмотрением только четных и нечетных
решений.

\begin{itemize}

\item[а)] Четный случай $\psi(x)=\psi(-x)$. Уравнение Шредингера имеет вид
   $\psi''-\frac{2m|E|}{\hbar^2}\psi=0$ в областях I: $(-\infty,-a)$,
   II: $(-a,a)$, III:$(a,+\infty)$. Обозначим
   $k:=\sqrt{\frac{2m|E|}{\hbar^2}}$ и найдем решения уравнения в каждой из
   областей I, II, III.

   $\psi_I=Ae^{kx}$ и в силу четности $\psi_{III}=Ae^{-kx}$
   ($e^{-kx}$ в $\psi_I$ отсутствует в силу необходимости
   стремления $\psi$ к нулю на бесконечности).

   $\psi_{II}=C_1e^{kx}+C_2e^{-kx}$ и в силу четности
     $\psi_{II}=C(e^{kx}+e^{-kx})$.

   Кроме того, следует записать условие $Ae^{-kx}=C(e^{ka}+e^{-ka})$
   непрерывности $\psi$ в точках $\pm a$ и условие постоянства интеграла
   от 0 по любой окрестности точки $a$
   $$\int\limits_{a-\varepsilon}^{a+\varepsilon}
     \left(
       -\frac{\hbar^2}{2m}\psi''-g[\delta(x-a)+\delta(x+a)]\psi+|E|\psi
     \right)dx=\mathop{\rm const}.$$
   Взяв предел при $\varepsilon\to0$, получим
   $$\psi'_{II}(a-0)-\psi'_{III}(a+0)=\frac{2mg}{\hbar^2}\psi_{III}(a+0)$$
   $$kC(e^{ka}-e^{-ka})+kAe^{-ka}=\frac{2mg}{\hbar^2}Ae^{-ka}.$$
   Вместе с условием непрерывности получается однородная система двух
   уравнений на два неизвестных $A$ и $C$. Она имеет нетривиальное решение
   только при условии $\mathop{\rm det}=0$:

   $$\left|\begin{array}{cc}
     e^{-ka} & -(e^{ka}+e^{-ka})\\
     -\left(\frac{2mg}{\hbar^2}-k\right)e^{-ka} & k(e^{ka}+e^{-ka})\\
     \end{array}\right| = 0
   \quad \Rightarrow \quad
     k=\frac{mg}{\hbar^2}(1+e^{-2ka})$$

\item[б)] Нечетный случай $\psi(x)=-\psi(-x)$. Аналогичные рассуждения приводят
   к получению $\psi_I=Ae^{kx}$, $\psi_{II}=C(e^{kx}-e^{-kx})$,
   $\psi_{III}=-Ae^{-kx}$, где $k=\frac{mg}{\hbar^2}(1-e^{-2ka})$.

\end{itemize}

В обоих случаях $E=-\frac{\hbar^2k^2}{2m}$ определяется значениями $k$,
удовлетворяющими одному из уравнений $k=\frac\eta2(1\pm e^{-2ka})$,
где $\eta:=\frac{2mg}{\hbar^2}$.

(Нарисуйте на одном рисунке графики функций $y(x)=\frac\eta2(1\pm e^{-2ax})$,
$y(x)=\frac\eta2$, $y(x)=x$ при $x>0$)

Из рисунка ясно, что в нечетном случае всегда есть вырожденное
решение $k_0=0$, и есть еще решение $k_-$, $0<k_-<\frac\eta2$, если
производная $y=\frac\eta2(1-e^{-2xa})$ больше 1. Также всегда существует
решение $k_+>\frac\eta2$.

Когда существует $k_-$? Когда $y'(0)=\eta a>1$, то есть $2mga>\hbar^2$.
Иначе нечетного решения нет.

Итого: при $2mga>\hbar^2$ есть два уровня энергии:
$E_1=-\frac{\hbar^2k_-^2}{2m}$ и $E_2=-\frac{\hbar^2k_+^2}{2m}$,
иначе есть только один уровень $E=-\frac{\hbar^2k_+^2}{2m}$.

(Изредка на экзамене спрашивают про особый случай $E=0$, когда
$2mga=\hbar^2$ и $\psi$ есть нечетная трехзвенная ломанная).

%%%%%%%%%%%%%%%%%%%%%%%%%%%%%%%%%%%%%%%%%%%%%%%%%%%%%%%%%%%%%%%%
\noindent\textbf{~6. Найти значения энергии частицы, при которых
обращается в нуль коэффициент отражения от потенциальной ямы
\[
U\left( x \right) = \left\{ {\begin{array}{l}
 - U_0 < 0,\;x \in \left[ {0,a} \right]\,; \\
 0,\;x \notin \left[ {0,a} \right]\,. \\
 \end{array}} \right.
\]}

Рассмотрим три области: I $(-\infty,0)$, II $(0,a)$, III $(a,+\infty)$.
Летит частица $\varphi_{\mbox{\scriptsize пад}}=e^{ik_1x}$.
В области I уравнение Шредингера имеет вид
$\displaystyle-\frac{\hbar^2}{2m}\varphi''+U\varphi=E\varphi$,
соответственно
 $\varphi_I=e^{ik_1x}+Ae^{-ik_1x}=
  \varphi_{\mbox{\scriptsize пад}}+\varphi_{\mbox{\scriptsize отраж}}$,
$k_1=\frac{\sqrt{2mE}}\hbar$.
В области II $\varphi_{II}=Be^{ik_2x}+Ce^{-ik_2x}$,
$\displaystyle k_2=\frac{\sqrt{2m(E-U_0)}}\hbar$.
В области III $\varphi_{III}=Ge^{ik_1x}$.

Надо найти $\displaystyle R(E)=
 \frac{|\vec j_{\mbox{\scriptsize отраж}}|}
      {|\vec j_{\mbox{\scriptsize пад}}|}$.
Легко видеть, что это просто $|A|^2$. Используя условия непрерывности
$\varphi$ и $\varphi'$ в точках $0$ и $a$, можно найти $A$. Но можно
использовать свойство $R(E)+D(E)=1$ и решение задачи 2:
$$D(E)=\frac{4k_1^2k_2^2}{(k_1^2-k_2^2)^2\sin^2k_2a+4k_1^2k_2^2}.$$
Тогда получим
$$R(E)=\frac
  {\displaystyle\sin^2\frac a\hbar\sqrt{2m(E+U_0)}}
  {\displaystyle\sin^2\frac a\hbar\sqrt{2m(E+U_0)}+4E(E+U_0)}.$$

Знаменатель при любом $E>0$ не равен нулю, значит, надо только решить
уравнение ``$\mbox{числитель}=0$'', или $\frac a\hbar\sqrt{2m(E+U_0)}=\pi n$,
$n\in\mathbb Z$. Отсюда $E_n=\frac{\pi^2n^2\hbar^2}{2ma^2}-U_0$ при условии
$n>\frac a{\pi\hbar}\sqrt{2mU_0}$ (это условие обусловлено требованием $E_n>0$).

%%%%%%%%%%%%%%%%%%%%%%%%%%%%%%%%%%%%%%%%%%%%%%%%%%%%%%%%%%%%%%%%
\noindent\textbf{~7. Вычислить матричный элемент для гармонического
осциллятора $\left( {\psi _0 ,\hat {x}^{2004}\psi _{2004} }
\right).$}

Гамильтониан квантового гармонического осциллятора:
$$\hat H=\frac{\hat p^2}{2m}+\frac{m\omega^2\hat x^2}2,
\hat p=-i\hbar\frac d{dx}, \hat x=x.$$

Его собственные функции и собственные значения:
$$\left\{\begin{array}{l}
 \displaystyle
  \varphi_n(x)=\left(\frac1{\pi x_0^2}\right)^{1/4}
               \frac1{\sqrt{2^nn!}}\exp\left(-\frac{x^2}{2x_0^2}\right)
               H_n\left(\frac x{x_0}\right), x_0=\sqrt{\frac\hbar{m\omega}};\\
 \displaystyle
  E_n=\hbar\omega\left(n+\frac12\right), n=0,1,2,\ldots;\\
\end{array}\right.$$
Здесь $\displaystyle H_n(\xi)=(-1)^ne^{\xi^2}\frac{d^n}{d\xi^n}e^{-\xi^2}$~---
полиномы Эрмита. Собственные функции образуют ортонормированную систему
в пространстве $L_2$, в частности, $(\psi_n,\psi_m)=\delta_{nm}.$

Вид собственных функций можно упростить, используя формализм повышающих и
понижающих операторов:
$$\hat a:=\frac1{\sqrt2}\left(\frac{\hat x}{x_0}+i\frac{\hat p}{p_0}\right),
\quad
 \hat a^+=\frac1{\sqrt2}\left(\frac{\hat x}{x_0}-i\frac{\hat p}{p_0}\right)
\mbox{ (здесь $p_0=\sqrt{\hbar m\omega}$)}$$.

Эти операторы очень полезны для описания квантового осциллятора.
Например, используя эти операторы, можно избавиться от $x$ и $p$ в
гамильтониане и записать
$\hat H=\hbar\omega\left(\hat a^+\hat a+\frac12\right)$.

Легко видеть, что
$\hat a\varphi_0\equiv0$,
$\hat a\varphi_n=\sqrt n\varphi_{n-1}$,
$\hat a^+\varphi_n=\sqrt{n+1}\varphi_{n+1}$.
Отсюда $\varphi_n=\frac{(\hat a^+)^n}{\sqrt{n!}}\varphi_0$.
%
Также можем написать
$\hat x=\frac{x_0}{\sqrt2}(\hat a+\hat a^+)$,
$\hat p=\frac{p_0}{i\sqrt2}(\hat a-\hat a^+)$.

Теперь мы можем найти $(\psi_0,\hat x^n\psi_n)$:
%
$$(\psi_0,\hat x^n\psi_n)=
  \frac{x_0^n}{\sqrt{2^n}}(\psi_0,(\hat a+\hat a^+)^n\psi_n)=
  \frac{x_0^n}{\sqrt{2^n}}\big(
    (\psi_0,\underbrace{\hat a\hat a\ldots\hat a}
            _{\mbox{\scriptsize $n$ штук}}\psi_n)+
    \ldots+
    (\psi_0,\hat a^+\hat a^+\ldots\hat a^+\psi_n)\big)$$
В силу $(\psi_n,\psi_m)=\delta_{nm}$ все скобки, кроме первой, равны нулю.
Тогда
$$(\psi_0,\hat x^n\psi_n)=
  \frac{x_0^n}{\sqrt{2^n}}(\psi_0,\hat a^n\psi_n)=
  \frac{x_0^n\sqrt{n!}}{\sqrt{2^n}}.$$

%%%%%%%%%%%%%%%%%%%%%%%%%%%%%%%%%%%%%%%%%%%%%%%%%%%%%%%%%%%%%%%%
\noindent\textbf{~8. Найти уровни энергии системы с гамильтонианом $$\hat
{H} = \frac{1}{2m}\left( {\hat {p}_x^2 + \hat {p}_y^2 } \right) +
\frac{m}{2}\omega ^2\left( {x^2 + y^2} \right) + \gamma
xy,\;\left| \gamma \right| < m\omega ^2.$$}

Уравнение Гамильтона имеет вид: $H^\Psi(x,y)=E\Psi(x,y)$. Условие
нормировки имеет вид $\iint\limits_{\mathbb{R}^2}|\Psi(x,y)|^2dxdy=1$.

Введем квадратичную форму
$\displaystyle B(x,y)=\frac{m\omega^2}2x^2+\gamma xy+\frac{m\omega^2}2y^2$
и найдем замену ${x\choose y}=Q{x'\choose y'}$, приводящую ее к главным осям.

$$B=\left(\begin{array}{cc}
   \frac{m\omega^2}2 & \frac\gamma2\\
   \frac\gamma2 & \frac{m\omega^2}2\\
 \end{array}\right), \qquad
  det(B-\lambda I)=
 \left(\frac{m\omega^2}2-\lambda\right)^2-\frac{\gamma^2}4=0;$$
%
$$\left\{\begin{array}{l}
  \lambda_1=\frac{m\omega^2}2+\frac\gamma2=\frac m2\Omega_1^2,\\
  \lambda_2=\frac{m\omega^2}2-\frac\gamma2=\frac m2\Omega_2^2;\\
\end{array}\right.\qquad
\left\{\begin{array}{ll}
  \vec u=\frac1{\sqrt2}{1\choose1} & \mbox{(соотв. $\lambda_1$)},\\
  \vec v=\frac1{\sqrt2}{-1\choose1}& \mbox{(соотв. $\lambda_2$)};\\
\end{array}\right.$$

Отсюда $Q=\left(\begin{array}{cc}1&-1\\1&1\\\end{array}\right)$~---
поворот на $45^\circ$. При таком преобразовании
$\hat p_x^2+\hat p_y^2=\hat p_{x'}^2+\hat p_{y'}^2$,
$\hat H=\left(\frac1{2m}\hat p_{x'}^2+\frac m2\Omega_1^2x'^2\right)+
        \left(\frac1{2m}\hat p_{y'}^2+\frac m2\Omega_2^2y'^2\right)=
 \hat H^{(1)}+\hat H^{(2)}$.
Теперь решим уравнение Гамильтона $\hat H\tilde\Psi(x',y')=E\tilde\Psi(x',y')$
методом разделения переменных.
%
$$\tilde\Psi(x',y')=\chi(x')\phi(y'),\quad
(\hat H^{(1)}(x')+\hat H^{(2)}(y')\chi\phi=E\chi\phi,
\phi\hat H^{(1)}\chi+\chi\hat H^{(2)}\phi=E\chi\phi,$$
%
$$\frac{\hat H^{(1)}\chi}\chi+\frac{\hat H^{(2)}\phi}\phi=
   E=\mathord{\rm const}\ \Rightarrow\ \left\{\begin{array}{l}
     \hat H^{(1)}\chi=E^{(1)}\chi,\\
     \hat H^{(2)}\phi=E^{(2)}\phi;\\
   \end{array}\right.\ \Rightarrow\ E_{nm}=\hbar\Omega_1(n+\frac12)+
     \hbar\Omega_2(m+\frac12)$$

%%%%%%%%%%%%%%%%%%%%%%%%%%%%%%%%%%%%%%%%%%%%%%%%%%%%%%%%%%%%%%%%
\noindent\textbf{~9. В состоянии $\psi $ с определенной проекцией момента
на ось $Oz$, $\hat{l}_z \psi = m\psi$, найти средние значения
$\left\langle {\hat {l}_x } \right\rangle ,\;\left\langle {\hat
{l}_y } \right\rangle ,\;\left\langle {\hat {l}_x \hat {l}_y }
\right\rangle $.}

Введем операторы $\hat l_\pm=\hat l_x\pm i\hat l_y$, тогда
$[\hat l_+,\hat l_-]=2\hat l_z$,
$\hat l^2=\frac12(\hat l_+\hat l_-+\hat l_-\hat l+)+\hat l_z^2$,
$\hat l_\pm\psi_m=\mathord{\rm const}\cdot\psi_{m\pm1}$,
$(\psi_m,\psi_{m'})=\delta_{mm'}$,
$\hat l_x=\frac12(\hat l_++\hat l_-)$,
$\hat l_y=\frac1{2i}(\hat l_+-\hat l_-)$.

$\langle\hat l_x\rangle=(\psi_m,\hat l_x\psi_m)=
  (\psi_m,\frac12(\hat l_+-\hat l_-)\psi_m)=
  \mathord{\rm const}(\psi_m,\psi_{m+1})+
  \mathord{\rm const}(\psi_m,\psi_{m-1})=0$.

$\langle\hat l_y\rangle=0$ аналогично.

$\langle\hat l_x\hat l_y\rangle=(\psi_m,\hat l_x\hat l_y\psi_m)=
  (\psi_m,\frac1{4i}(\hat l_+^2-\hat l_-^2+
    (\hat l_-\hat l_+-\hat l_+\hat l_-))\psi_m)=
  (\psi_m,\frac1{4i}(0-0-[\hat l_+,\hat l_-]))=
  -\frac{4i}m\cdot2(\psi_m,\psi_m)=-\frac1{2i}m=\frac{im}2$
(среднее является мнимым, так как $\hat l_x\hat l_y$~--- не эрмитов).

%%%%%%%%%%%%%%%%%%%%%%%%%%%%%%%%%%%%%%%%%%%%%%%%%%%%%%%%%%%%%%%%
\noindent\textbf{10. В состоянии $\psi $ с определенными ${\bf l}^2$ и
$l_z $, $\hat {\bf l}^2\psi = l\left( {l + 1} \right)\psi ,\;\hat
{l}_z \psi = m\psi $, найти средние значения $\left\langle {\hat
{l}_x^2 } \right\rangle ,\;\left\langle {\hat {l}_y^2 }
\right\rangle $.}

$\hat l_x^2=\frac14(\hat l_++\hat l_-)^2=
  \frac14(\hat l_+^2+\hat l_-^2+\hat l_+\hat l_-+\hat l_-\hat l_+)=
  \frac14(\hat l_+^2+\hat l_-^2+2(\hat l^2-\hat l_z^2))$,

$\hat l_y^2=-\frac14(\hat l_+-\hat l_-)^2=
  -\frac14(\hat l_+^2+\hat l_-^2-2(\hat l^2-\hat l_z^2))$,

$\langle\hat l_x^2\rangle=(\psi_{lm},\hat l_x^2\psi_{lm})=
  \frac14(\psi_{lm},(\hat l_+^2+\hat l_-^2+2(\hat l^2-\hat l_z^2))\psi_{lm})=
  \frac14(0+0+2(\psi_{lm},(l(l+1)-m^2)\psi_{lm}))=\frac12(l(l+1)-m^2)$,

$\langle\hat l_y^2\rangle=\frac12(l(l+1)-m^2)$ аналогично.

%%%%%%%%%%%%%%%%%%%%%%%%%%%%%%%%%%%%%%%%%%%%%%%%%%%%%%%%%%%%%%%%
\noindent\textbf{11. Найти плотность вероятности различных значений
импульса электрона в основном состоянии атома водорода.}

Атом водорода: вокруг ядра заряда $+e$ обращается электрон заряда
$-e$ и массы $m$. Будем считать, что масса ядра сильно больше массы
электрона.

Гамильтониан $\hat H=-\frac{\hbar^2}{2m}\nabla^2+U(r)$,
потенциальная энергия $U(r)=-\frac\alpha r$, $\alpha=e^2>0$.

В сферических координатах $\vec r=(r,\theta,\varphi)$
$\nabla^2=\frac1{r^2}\frac\partial{\partial r}
    \left(r^2\frac\partial{\partial r}\right)
  +\frac1{r^2}\left(
    \frac1{\sin\theta}\frac\partial{\partial\theta}
      \left(\sin\theta\frac\partial{\partial\theta}\right)+
      \frac1{\sin^2\theta}\frac{\partial^2}{\partial\varphi^2}\right)$.

Уравнение Шредингера $\hat H\Psi(\vec r)=E\Psi(\vec r)$ решается
разделением переменных: $\Psi(\vec r)=R(r)Y(\theta,\varphi)$. Решение:

$E_n=-\frac{m\alpha^2}{2\hbar^2}\frac1{n^2}$,\par
$\Psi_{nlm}(\vec r)=R_{nl}(r)Y_{lm}(\theta,\varphi)$,\par
$R_{nl}(r)=-\frac2{n^2a^{3/2}}\sqrt{\frac{(n-l-1)!}{((n+l)!)^2}}
  e^{-\frac r{na}}\left(\frac{2r}{na}\right)^lL_{n+l}^{2l+1}
  \left(\frac{2r}{na}\right)$,
$L_k^s(x)=e^xx^{-s}\frac{d^k}{dx^k}(e^{-x}x^{k+s})$,\par
$Y_lm(\theta,\varphi)=(-1)^{\frac{m+|m|}2}i^l
  \sqrt{\frac{2l+1}{4\pi}\frac{(l-|m|)!}{(l+|m|)!}}
  P_l^{|m|}(\cos\theta)e^{im\varphi}$,
$P_l^m(x)=(1-x^2)^{\frac m2}\frac{d^m}{dx^m}P_l(x)$,
$P_l(x)=2^{\frac1l}l!\frac{d^l}{dx^l}(x^2-1)^l$.

Здесь $n=1,2,3,\ldots$, $l=0,1,\ldots,n-1$, $m=-l,\ldots,l$, а основное
состояние получается $n=1$, $l=0$, $m=0$. При этом
$Y_{00}=\frac1{\sqrt{4\pi}}$, $R_{10}(r)=\frac2{a^{3/2}}e^{-\frac ra}$,
$a\equiv\frac{\hbar^2}{m\alpha}$, $U=-\frac\alpha r$, $\alpha=e^2$.
$\Psi_{100}(\vec r)=\frac1{\sqrt{4\pi}}\frac2{a^{3/2}}e^{-\frac ra}$,
$\rho(r)=|\Psi|^2$.

Теперь начинаются мучения как в задаче 1\ldots{} Надо найти
$\rho(\vec p)=|\Psi(\vec p)|^2$.

$\Psi(\vec p)=\frac1{(2\pi\hbar)^{3/2}}\int\Psi(\vec r)
    e^{-\frac{i\vec p\vec r}\hbar}d\vec r=
  \frac A{(2\pi\hbar)^{3/2}}
    \int\limits_0^{2\pi}d\varphi
    \int\limits_0^\infty dr
    \int\limits_0^\pi d\theta
    \sin\theta r^2e^{-\frac ra}e^{-\frac{\vec p\vec r}\hbar}=
  \frac{2\pi A}{(2\pi\hbar)^{3/2}}\ldots$

\ldots


%%%%%%%%%%%%%%%%%%%%%%%%%%%%%%%%%%%%%%%%%%%%%%%%%%%%%%%%%%%%%%%%
\noindent 12. Электрон находится в состоянии с определенным
значением $S_z = \hbar / 2$. Найти вероятности возможных значений
проекции спина на направление ${\bf n} = ( \sin \alpha \cos \beta
,$ $ \sin \alpha \sin \beta ,\cos \alpha )$.

%%%%%%%%%%%%%%%%%%%%%%%%%%%%%%%%%%%%%%%%%%%%%%%%%%%%%%%%%%%%%%%%
\noindent 13. Используя уравнение Паули, найти спектр энергии
электрона в постоянном однородном магнитном поле ${\bf B} = B{\bf
e}_z $, заданном вектор-потенциалом ${\bf A} = xB{\bf e}_y $.

%%%%%%%%%%%%%%%%%%%%%%%%%%%%%%%%%%%%%%%%%%%%%%%%%%%%%%%%%%%%%%%%
\noindent\textbf{14. Вычислить статистический вес состояния с энергией
$E$ системы $N$ независимых осцилляторов частоты $\omega $. Найти
теплоемкость системы как функцию температуры $T$.}

Уровни энергии $i$-го осциллятора имеют вид
$E_{in_i}=\hbar\omega(n_i+\frac12)$, $i=1\ldots N$, $n_i=0,1,2,\ldots$.
Тогда полная энергия системы есть $E=\sum\limits_{i=1}^NE_i=
\hbar\omega\sum\limits_{i=1}^N(n_i+\frac12)=
\hbar\omega\left(\sum\limits_{i=1}^Nn_i+\frac N2\right)=
\hbar\omega(M+\frac N2)$, где $M:=\sum\limits_{i=1}^Nn_i$.
Таким образом, задача нахождения статистического веса $\Gamma$ сводится к
нахождению числа способов представить $M$ как сумму $N$ неотрицательных
слагаемых, причем без учета порядка (в силу неразличимости квантовых частиц).
Например, при $M=4$, $N=2$ $\Gamma=3$ ($M=0+4=1+3=2+2$).
Однако такая комбинаторная задача
оказывается слишком сложной, поэтому немного нарушим условия, разрешим
квантовым частицам быть различимыми и будем искать с учетом порядка
(в нашем случае будет $M=0+4=1+3=2+2=3+1=4+0$)

Нам надо найти число способов расставить $N-1$ перегородок среди $M$ шариков.
Нарисуем $M+N-1$ клеточек и назовем $N-1$ из них перегородками, а в остальные
поместим шарики. Это можно сделать $C_{M+N-1}^{N-1}$ способами.

Найдем ассимптотику $\Gamma$ при больших $M$ и $N$, используя формулу
Стирлинга $N!=\left(\frac Ne\right)^N$:
$$\Gamma=\frac{(M+N-1)!}{M!(N-1)!}\approx\frac{(M+N)!}{M!N!}\approx
\frac{(M+N)^{M+N}}{M^MN^N}e^{-(M+N)}e^Me^N=\frac{(M+N)^{M+N}}{M^MN^N}.$$

Найдем энтропию системы: $S(E)=k\ln\Gamma(E)\approx k((M+N)\ln(M+N)-
M\ln M-N\ln N)$. Далее, $\frac 1T=\frac{\partial S}{\partial E}=
\frac{\partial S}{\partial M}\frac{\partial M}{\partial E}$,

$$\frac{\partial S}{\partial M}=k(\ln(M+N)+1-\ln M-1)=
  k\ln\frac{M+N}M=k\ln\frac{\frac E{\hbar\omega}+\frac N2}
                           {\frac E{\hbar\omega}-\frac N2},$$
%
$$\frac1{\hbar\omega}k\ln\frac{\frac E{\hbar\omega}+\frac N2}
                              {\frac E{\hbar\omega}-\frac N2}=\frac1T
\ \Rightarrow\ %
E(T)=N\left(\frac{\hbar\omega}2+\frac{\hbar\omega}
 {e^{\frac{\hbar\omega}{kT}}-1}\right).$$

Теперь осталось найти теплоемкость (при постоянном объеме) $C_V$.
%
$$C_V=\frac{dE}{dT}=N\hbar\omega\frac1
    {\left(e^{\frac{\hbar\omega}{kT}}-1\right)^2}
    \frac{\hbar\omega}k\frac1{T^2}e^{\hbar\omega}{kT}=
  Nk\left(\frac{\hbar\omega}{kT}\right)^2\frac
    {e^{\frac{\hbar\omega}{kT}}}
    {\left(e^{\frac{\hbar\omega}{kT}}-1\right)^2}.$$

Для полноты картины полезно нарисовать график, на котором по оси $Ox$
отложены $\frac{kT}{\hbar\omega}$, а по оси $Oy$~--- $\frac{C_V}{Nk}$.
График начинаеться в $(0,0)$ и на бесконечности стремится к прямой $y=1$.
Философское изучение графика проделайте самостоятельно ${}^*_*-)$.

%%%%%%%%%%%%%%%%%%%%%%%%%%%%%%%%%%%%%%%%%%%%%%%%%%%%%%%%%%%%%%%%
\noindent\textbf{15. Вычислить статистическую сумму системы $N$
независимых осцилляторов частоты $\omega $ при температуре $T$.
Найти теплоемкость системы.}

%%%%%%%%%%%%%%%%%%%%%%%%%%%%%%%%%%%%%%%%%%%%%%%%%%%%%%%%%%%%%%%%
По определению, $\displaystyle Z=\sum\limits_{n=0}^\infty e^{-\frac{E_n}{kT}}$.
При этом $E_n=\sum\limits_{i=1}^N E_{n_i}$, где $E_{n_i}=\hbar\omega(n_i+\frac12)$,
$n_i=0,1,2\ldots$~--- распределение энергий $i$-го осциллятора. Значит,
%
   $$Z=\sum_{n_1}\ldots\sum_{n_N}e^{-(\sum_{i=1}^N E_{n_i})/kT}=
       \sum_{n_1}e^{-E_{n_1}/kT}\ldots\sum_{n_N}e^{-E_{n_N}/kT}=$$
   $$=\left(\sum\limits_{m=0}^\infty e^{-\frac{E_m}{kT}}\right)^N=
       e^{-\frac{\hbar\omega N}{2kT}}\left(\sum_{m=0}^\infty
            e^{-\frac{\hbar\omega m}{kT}}\right)^N=\mbox{[геом. прогрессия]}=$$
   $$=\left(e^{-\frac{\hbar\omega}{2kT}}\frac1{1-e^{-\hbar\omega/kT}}\right)^N=
       \left(2\mathop{sh}\frac{\hbar\omega}{2kT}\right)^{-N}.$$

Для нахождения теплоемкости при постоянном объеме
   $C_V=\frac{\partial E}{\partial T}$
найдем ($\beta:=\frac1{kT}$, $w_n=\frac1Ze^{-\frac{E_n}{kT}}$~--- каноническое
распределение Гиббса)
%
   $$E=\langle E_n\rangle=\sum\limits_n E_nw_n=
          \frac{\sum E_ne^{-\beta E_n}}{\sum e^{-\beta E_n}}=
          -\frac\partial{\partial\beta}\ln\sum e^{-\beta E_n}=
          -\frac\partial{\partial\beta}\ln Z.$$

Тогда
$\displaystyle E(T)=
  \frac{N\hbar\omega}2\left(1+\frac2{e^{\frac{\hbar\omega}{kT}}-1}\right)$
и
$\displaystyle C_V=\frac{\partial E(T)}{\partial T}=
   Nk\left(\frac{\hbar\omega}{kT}\right)^2
   \frac
        {e^{\frac{\hbar\omega}{kT}}}
        {\left(e^{\frac{\hbar\omega}{kT}}-1\right)^2}$

%%%%%%%%%%%%%%%%%%%%%%%%%%%%%%%%%%%%%%%%%%%%%%%%%%%%%%%%%%%%%%%%
\noindent \textbf{16. Вычислить большую статистическую сумму для
идеального классического газа как функцию температуры $T$, объема
$V$ и химического потенциала $\mu $. Найти уравнение состояния и
теплоемкость системы.}

По определению,
$$
  \zeta(V,\mu,T)=
    \sum\limits_{N=0}^\infty\frac1{N!}e^{\frac{\mu N}{kT}}
    \int e^{-\frac{H_n(p,q)}{kT}}d\Gamma_N,\mbox{ }
  d\Gamma_N=\frac{\gamma^N}{(2\pi\hbar)^{3N}}
    \prod\limits_{i=1}^N d\vec p_i d\vec q_i
$$
($\gamma$~--- число степеней свободы, обычно равно 1, для электрона~--- 2).
У нас $H_N(p,q)=\sum\limits_{i=1}^N \frac{\vec p_i{}^2}{2m}$. Обозначим $\theta:=kT$~---
температура в термодинамических единицах~--- и запишем
$$
  \zeta(V,N,\theta)=\sum\limits_{N=0}^\infty\frac1{N!}e^{\frac{\mu N}\theta}
    \int\frac1{(2\pi\hbar)^{3N}}
       e^{-\sum\limits_{i=1}^N\textstyle\frac{\vec p_i{}^2}{2m\theta}}
       \prod\limits_{i=1}^N d\vec p_i d\vec q_i.
$$
Видим, что интеграл в этом выражении есть по сути $Z(V,N,\theta)$~---
статистическая сумма. Так как подынтегральное выражение не зависит от
$\vec q_i$, то можно безболезненно вынести $\int d\vec q_i$ за знак интеграла,
и воспользоваться тем, что $\int\limits_Vd\vec q_i=V$:
$$
  Z(V,N,\theta)=\frac{V^N}{(2\pi\hbar)^{3N}}
    \int e^{-\textstyle\sum\limits_{i=1}^N\frac{\vec p_i{}^2}{2m\theta}}
      \prod\limits_{i=1}^N d\vec p_i=
  \frac{V^N}{(2\pi\hbar)^{3N}}
    \left(
      \int e^{-\frac{\vec p^2}{2m\theta}}d\vec p
    \right)^N=
$$
$$=
  \frac{V^N}{(2\pi\hbar)^{3N}}
    \left(
      \int\limits_{-\infty}^{+\infty}e^{-\frac{t^2}{2m\theta}}dt
    \right)^{3N}=
  \frac{V^N}{(2\pi\hbar)^{3N}}(2\pi m\theta)^{\frac{3N}2},
$$
$$
  \zeta(V,\mu,\theta)=
  \sum\limits_{N=0}^\infty\frac{e^\frac{\mu N}\theta}{N!}
    \left(
      \frac {V(2m\theta)^{\frac32}} {(2\pi\hbar)^3}
    \right)^N=
  \exp\left(
         e^{\frac\mu\theta}\frac{V(2\pi m\theta)^{\frac32}}{(2\pi\hbar)^3}
      \right).
$$

Для нахождения уравнения состояния найдем $\Omega$~--- термодинамический
потенциал системы: $\Omega(V,\mu,\theta)=-\theta\ln\zeta(V,\mu,\theta)=
-AVe^{\frac\mu\theta}\theta^{\frac52}$, где
$A:=\frac{(2\pi m)^\frac32)}{(2\pi\hbar)^3}$.
Поскольку $$\Omega=F-\mu N=E-TS-\mu N\mbox{\quad и\quad}d\Omega=SdT-pdV-Nd\mu,$$ то
$p=-\frac{\partial\Omega}{\partial V}$, откуда $\Omega=-pV$.
Тогда $p(V,\mu,\theta)=A\theta^{\frac52}e^{\frac\mu\theta}$ будет уравнением
состояния идеального газа.

Однако мы хотим все же расписать это уравнение не через $p(V,\mu,\theta)$, а
через $p(V,\bar N,\theta)$, где $\bar N$~--- среднее число частиц. Напишем
$\bar N=-\frac{\partial\Omega(V,\mu,\theta)}{\partial\mu}=
VA\theta^{\frac32}e^{\frac\mu\theta}$. Отсюда
$\mu=\theta\ln\frac{\bar N}{VA\theta^{\frac32}}$.
Теперь все просто: $p=A\theta^{\frac32}e^{\frac\mu\theta}\big|_{\mu=\ldots}=
\frac{\theta\bar N}V=\bar nkT$ ($\bar n=\frac{\bar N}V$~--- средняя
концентрация частиц). Тогда $$p(V,\bar N,T)=\bar nkT$$ и есть уравнение
состояния идеального газа (известное еще с школьных уроков физики).

Чтобы найти теплоемкость при постоянном объеме и числе частиц $C_{VN}$, напишем
первое начало термодинамики: $\delta Q=dE+pdV-\mu dN=dE=TdS$, так как $V$ и $N$
постоянны. Тогда $C_{VN}=\frac{\delta Q}{\delta T}=\frac{dE}{dT}=T\frac{dS}{dT}$.

\noindent\llap{!!! }\textbf{Неверно, что
  $\frac{\partial S(V,\mu,\theta)}{\partial\theta}=C_{VN}(V,\mu,\theta)$,
  так как при замене переменной $\mu=\mu(N)$ и дифференцировании сложной
  функции выскочит лишний множитель!}

$S=-\frac{\partial\Omega}{\partial T}=S(V,T,\mu)$, подставим $\mu=\mu(T,N)$:
$$
  S(V,\mu,\theta)=-k\frac{\partial\Omega}{\partial\theta}=
    kVAe^{\frac\mu\theta}
      \left(\frac52\theta^{\frac32}-\mu\theta^{\frac12}\right),
\mbox{\quad}
  S(V,\bar N,\theta)=k\bar N
    \left(\frac52-\ln\frac{\bar N}{VA\theta^{\frac32}}\right)
$$

Итого $C_{VN}=\theta\frac{\partial S}{\partial\theta}=\frac32k\bar N$.

%%%%%%%%%%%%%%%%%%%%%%%%%%%%%%%%%%%%%%%%%%%%%%%%%%%%%%%%%%%%%%%%
\noindent 17. Найти химический потенциал $\mu \left( {T,L^2 / N}
\right)$ двумерного идеального квантового газа.

%%%%%%%%%%%%%%%%%%%%%%%%%%%%%%%%%%%%%%%%%%%%%%%%%%%%%%%%%%%%%%%%
\noindent 18. Вычислить давление идеального ферми-газа при $T =
0$.

%%%%%%%%%%%%%%%%%%%%%%%%%%%%%%%%%%%%%%%%%%%%%%%%%%%%%%%%%%%%%%%%
\noindent 19. Вычислить теплоемкость идеального бозе-газа при
низких температурах.

%%%%%%%%%%%%%%%%%%%%%%%%%%%%%%%%%%%%%%%%%%%%%%%%%%%%%%%%%%%%%%%%
\noindent 20. Вычислить среднее число фотонов в объеме $V$ при
температуре $T$.


\bigskip

\textbf{03.09.2003}
  \hfill\textbf{\textit{Профессор\hspace{2cm} А. В. Борисов}}


\end{document}
