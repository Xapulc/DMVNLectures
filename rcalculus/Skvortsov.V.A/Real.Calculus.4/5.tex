
% пятая лекция по комплану. сказания о интеграле Лебега

\begin{center}
\textbf{Лекция 5.}
\end{center}


$(X, \EuScript(M), \mu)$

$\mu$ --- $\delta$--конечна, если $X = \bigcup_{i = 1}^\infty X_i,
\; \mu X_i < +\infty$

$(L) \int\limits_{E} f d\mu = \sup\limits_{E = \bigcup_j E_j}
\left( \sum\limits_j (\inf\limits_{x \in E_j} f(x)) \mu E_j
\right)$, для меры и $f(x)$ разрешено принимать значение
$+\infty$.
%
%
%
%
%
%
%
%%%%%%%%%%%%%%%%%%%%%%%%%%%%%%%%%%%%%%%%%%%%%%%%%%%%%%%%%%%%%%%%%%
%%%%%%%%%%%%%%%%%%%%%%%%%%%%%%%%%%%%%%%%%%%%%%%%%%%%%%%%%%%%%%%%%%
%%%%%%%%%%%%%%%%%          стр 8.1    %%%%%%%%%%%%%%%%%%%%%%%%%%%%
%%%%%%%%%%%%%%%%%%%%%%%%%%%%%%%%%%%%%%%%%%%%%%%%%%%%%%%%%%%%%%%%%%
%%%%%%%%%%%%%%%%%%%%%%%%%%%%%%%%%%%%%%%%%%%%%%%%%%%%%%%%%%%%%%%%%%
%
%
%
%
%

Если интеграл принимает значение $\infty$, мы не будем говорить,
что функция интегрируема.

\textbf{Определение.} \quad Суммируемая функция --- интегрируемая
по Лебегу с конечным значением интеграла.

\textbf{Утверждение} \quad Если $f(x)$ измерима и $f(x) \geqslant
0$ (на измеримом множестве) и суммируема, то тогда $f(x)$ конечна
и суммируема, то тогда $f(x)$ конечна~п.в.

$\{ x \in E : f(x) = + \infty \} = E \: \backslash \: (\bigcup_{N
= 1}^\infty \{ f(x) < N \}) \quad \Rightarrow$

$\{ x \in E : f(x) = + \infty \}$ --- измеримо $\Rightarrow$

$\mu \: \{ x \in E : f(x) = + \infty \} = 0$

$E_1 \subset E, E_1$ --- измеримо, $f(x) \geqslant 0 \Rightarrow$

$\int_{E_1} f d\mu \leqslant \int_E f d\mu$

$f(x) \leqslant g(x) \Rightarrow \int_E f d\mu \leqslant \int_E g
d\mu$

$f(x) = a_j, \; x \in E_j \; (f(x) - \text{простая функция}) $

Тогда $(L) \int_E f d\mu = \sum_{j = 1}^n a_j \mu E_j$

Это можно считать определением интеграла для простой
неотрицательной функции.

$f, g$ --- простые функции

$f(x) = a_j, \quad x \in E_j$

$g(x) = b_i, \quad x \in E_i$

Тогда:

1. \quad $\int_E (f + g) d \mu = \sum_i \sum_j (a_j + b_i) \: \mu
(E_j \bigcap E'_i) = \int_E f d\mu + \int_E g d\mu$

2. \quad $\int_E (c  f) d\mu = c  \int_E f d\mu$

3. \quad $E = E_1 \bigcup E_2 \quad \int_{E_1} f d\mu + \int_{E_2}
f d\mu = \int_E f d\mu$


\textbf{Теорема (о вычислении интеграла Лебега через простые
функции)} \quad

%
%
%
%
%
%
%
%%%%%%%%%%%%%%%%%%%%%%%%%%%%%%%%%%%%%%%%%%%%%%%%%%%%%%%%%%%%%%%%%%
%%%%%%%%%%%%%%%%%%%%%%%%%%%%%%%%%%%%%%%%%%%%%%%%%%%%%%%%%%%%%%%%%%
%%%%%%%%%%%%%%%%%          стр 9.0    %%%%%%%%%%%%%%%%%%%%%%%%%%%%
%%%%%%%%%%%%%%%%%%%%%%%%%%%%%%%%%%%%%%%%%%%%%%%%%%%%%%%%%%%%%%%%%%
%%%%%%%%%%%%%%%%%%%%%%%%%%%%%%%%%%%%%%%%%%%%%%%%%%%%%%%%%%%%%%%%%%
%
%
%
%
%

$f(x) \geqslant \mbox{ на Е - измер., } f_k (x) \nearrow f(x),
\{f_k(x)\}$ --- последовательность простых измеримых функций.

Тогда $(L) \int_E f d\mu = lim_{k \rightarrow \infty} \int_E f_k d
\mu$

$\blacklozenge$

$f_k(x) \leqslant f(x)$

$(L) \int_E f_k d\mu \leqslant (L) \int_E f d\mu \; \Rightarrow \;
\lim_{k \rightarrow \infty} (L) \int_E f_k d\mu \leqslant (L)
\int_E f d\mu$

Докажем неравенство в другую сторону

$\inf\limits_{x \in E_j} f(x) = a_j, E = \bigcup_{j = 1}^n E_j$

$\lim\limits_{k \rightarrow \infty} f_k(x) \geqslant a_j \quad
\forall x \in E_j$

Возьмем $\varepsilon > 0$

$A_{jk} = \{ x \in E_j : f_k(x) > a_j - \varepsilon\}$

$E_j = \bigcup\limits_k A_{jk} \quad (\text{так как } lim_{k
\rightarrow \infty} f_k \geqslant a_j)$

$A_{jk} \subset A_{j(k + 1)}$ (так как $f_k$ --- монотонная), то
есть это монотонно растущая последовательность $\Rightarrow$ (по
непрерывности меры) $\mu A_{jk} \xrightarrow[k \rightarrow
\infty]{} \mu E_j$

$\int_{E_j} f_k d\mu \geqslant \int_{A_{jk}} f_k d\mu > (a_j -
\varepsilon) \mu A_{jk}$

$\lim_{k \rightarrow \infty} \int_{E_j} f_k d\mu \geqslant a_j \mu
E_j$

Получаем $\lim_{k \rightarrow \infty} \int_E f_k d\mu \geqslant
\sum_{j} a_j \mu E_j$, так как разбиение произвольно, то $\lim_{k
\rightarrow \infty} \int_E f_k d\mu \geqslant
\sup\limits_{\bigcup_{j = 1}^n E_j = E} \sum_j a_j \mu E_j = (L)
\int_E f d\mu$

$\lozenge$

Предельным переходом получаем для $f, g \geqslant 0$

$1. \int_E (f + g) d\mu = \int_E f d\mu + \int_E g d\mu$

$2. \int_E c f d\mu = c \int_E f d \mu$

$3. \int_{E_1} f d \mu + \int_{E_2}f d \mu = \int_{E_1 \bigcup
E_2} f d \mu f d\mu, \quad E_1 \bigcap E_2 = \varnothing$

Значение интеграла не зависит от значения $f(x)$ на множестве меры
0.

Определим интеграл Лебега для функций любого знака $f = f^+ + f^-$

$f^+ = max \{f(x); 0\}, \quad f^- = max\{f(x); 0\}$

%
%
%
%
%
%
%
%%%%%%%%%%%%%%%%%%%%%%%%%%%%%%%%%%%%%%%%%%%%%%%%%%%%%%%%%%%%%%%%%%
%%%%%%%%%%%%%%%%%%%%%%%%%%%%%%%%%%%%%%%%%%%%%%%%%%%%%%%%%%%%%%%%%%
%%%%%%%%%%%%%%%%%          стр 9.1    %%%%%%%%%%%%%%%%%%%%%%%%%%%%
%%%%%%%%%%%%%%%%%%%%%%%%%%%%%%%%%%%%%%%%%%%%%%%%%%%%%%%%%%%%%%%%%%
%%%%%%%%%%%%%%%%%%%%%%%%%%%%%%%%%%%%%%%%%%%%%%%%%%%%%%%%%%%%%%%%%%
%
%
%
%
%

Если $f$ --- изм. $\Rightarrow f^+, f^-$ --- измер.

$ (L) \int_E f d\mu =  (\text{по определению}) (L) \int_E f^+ d
\mu - (L) \int_E f^- d\mu$

Говорим, что интеграл существует, если хотя бы один из этих двух
интегралов (от $f^+$ и $f^-$) конечен.

$f$ называют интегрируемой, если оба интеграла, входящие в
определение, конечны.


\begin{center} \textbf{Свойства интеграла Лебега.} \end{center}

$1. \quad \mu E = 0 \Leftrightarrow \int_E f d\mu = 0$

$2. \quad f \thicksim g \; (f \text{ и } g \text{ совпадают п.в.})
 \Rightarrow \text{ существует} (L) \int_E f d\mu$

$ \Leftrightarrow \text{ существует } (L) \int_E g d \mu \text{ и
} (L) \int_E f d \mu = (L) \int_{E} g d \mu$

$3. \quad f(x) \leqslant g(x) \Rightarrow \int_E f d \mu \leqslant
\int_E g d \mu$

$4. \quad f(x) - \text{ интегрируемая } \Rightarrow f(x) \text{
конечна п.в. } (f(x) \in L(E)))$

$5. \quad E_1 \subset E, \int_E f d \mu \text{ существует}
\Rightarrow \int_{E_1} f d \mu \text{ существует } (f \text{
суммир. на } E \Rightarrow f \text{ суммир. на } E_1)$

$6. \quad f \text{ измерима на } E, \; f \in L(X) \Leftrightarrow
|f| \in L(E)$\\$|f| = f^+ + f^-$\\$|\int_E f d \mu| \leqslant
\int_E f^+ + \int_E f^- d\mu = \int_E |f| d \mu$

$7. \quad f, g - \text{ измеримы на } E \; |f(x)| \leqslant |g(x)|
\text{ п.в. } \Rightarrow \int_E |f| d \mu \leqslant \int_E |g| d
\mu $

$8. \quad \int_E (cf) d\mu = c \int_E f d \mu$

$9. \quad \int_{E_1} f d \mu + \int_{E_2} f d \mu = \int_{E_1
\bigcup E2} f d \mu \; (E_1 \bigcap E_2 = \varnothing)$

$10. \quad \int_E (f + g)d\mu = \int_E f d \mu + \int_E g d \mu$\\
Если $f$ и $g$ суммир. $\Rightarrow f + g$ суммируемы \\ $|f + g|
\leqslant |f| + |g|$

$11. \quad f(x) \geqslant 0 \text{ на } E, \; m \leqslant g(x)
\leqslant M \text{ п.в. на } E, \text{ тогда } m\int_E f d \mu
\leqslant M \int_E f d \mu$


%
%
%
%
%
%
%
%%%%%%%%%%%%%%%%%%%%%%%%%%%%%%%%%%%%%%%%%%%%%%%%%%%%%%%%%%%%%%%%%%
%%%%%%%%%%%%%%%%%%%%%%%%%%%%%%%%%%%%%%%%%%%%%%%%%%%%%%%%%%%%%%%%%%
%%%%%%%%%%%%%%%%%          стр 10.0    %%%%%%%%%%%%%%%%%%%%%%%%%%%%
%%%%%%%%%%%%%%%%%%%%%%%%%%%%%%%%%%%%%%%%%%%%%%%%%%%%%%%%%%%%%%%%%%
%%%%%%%%%%%%%%%%%%%%%%%%%%%%%%%%%%%%%%%%%%%%%%%%%%%%%%%%%%%%%%%%%%
%
%
%
%
%
\begin{center} \textbf{Предельный переход под знаком интеграла} \end{center}

$f(x) = \sum_{k = 1}^{\infty} f_k(x), \; f_k(x) \geqslant 0$

Тогда $(L) \int_E f  \mu = \sum_{k = 1}^{\infty} (L) \int_E f_k d
\mu$

$\blacklozenge \sum_{k = 1}^N f_k(x) \leqslant f(x)$

$\sum_{k=1}^N (L) \int f_k (x) d \mu \leqslant (L) \int_E f d \mu$

$\sum_{k = 1}^{\infty} (L) \int f_k d \mu \leqslant (L) \int_E f d
\mu$

Докажем неравенство в другую сторону:

$f_{k_j} \nearrow_{j \rightarrow \infty} f_k, \; f_{k_j}$ ---
простые функции

$(*) \; S_j = \sum_{k = 1}^j f_{k_j} \leqslant \sum_{k=1}^j f_k
\leqslant \sum_{k=1}^{\infty} f_k = f$

$lim_{j \rightarrow \infty} S_j(x) \leqslant f(x)$

Фиксируем $n$. Пусть $j > n$

$S_j \geqslant \sum_{k=1}^n f_{k_j}; lim_{j \rightarrow \infty}
S_j (x) \geqslant \sum_{k=1}^n f_k(x) \; \forall$ фиксированных
$n$

$\lim_{j \rightarrow \infty} S_j (x) \geqslant \sum_{k=1}^{\infty}
f_k(x) = f(x)$

Значит, $\lim_{j \rightarrow \infty} S_j (x) = f(x)$

Тогда $(L) \int f d\mu = \lim_{j \rightarrow \infty} \int_E S_j d
\mu \leqslant \sum_{k=1}^{\infty} \int_E f_k d \mu$

$\int_E S_j d \mu < \sum_{k=1}^j \int_E f_k d \mu $
$\blacklozenge$

\textbf{Следствие.} \quad  $f(x) = \sum_{k=1}^{\infty} f_k(x), \;
f_k \geqslant 0$, и сходится $\sum_{k=1}^{\infty} \int_E f_k d \mu
\Rightarrow \sum_{k=1}^{\infty} f_k(x)$ сходится п.в.

Верно, т.к. если $\int_E f d \mu$ конечен, то  $f$ конечен п.в.

\textbf{Задача.} \quad Пусть $\mathbf{R} = \{r_n\}$, доказать, что
$\sum_{k=1}^{\infty} \frac{1}{k^2\sqrt{x - r_k}}$ сходится п.в.

