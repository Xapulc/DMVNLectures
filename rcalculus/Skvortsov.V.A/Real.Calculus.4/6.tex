
% шестая лекция по комплану. сказания о интеграле Лебега

\begin{center} \textbf{Лекция 6.}


\textbf{$\delta-$ аддитивность интеграла Лебега}\end{center}

$E = \bigcup_{k=1}^{\infty} E_k, \; E_k$
%
%
%
%
%
%
%
%%%%%%%%%%%%%%%%%%%%%%%%%%%%%%%%%%%%%%%%%%%%%%%%%%%%%%%%%%%%%%%%%%
%%%%%%%%%%%%%%%%%%%%%%%%%%%%%%%%%%%%%%%%%%%%%%%%%%%%%%%%%%%%%%%%%%
%%%%%%%%%%%%%%%%%          стр 10.1    %%%%%%%%%%%%%%%%%%%%%%%%%%%%
%%%%%%%%%%%%%%%%%%%%%%%%%%%%%%%%%%%%%%%%%%%%%%%%%%%%%%%%%%%%%%%%%%
%%%%%%%%%%%%%%%%%%%%%%%%%%%%%%%%%%%%%%%%%%%%%%%%%%%%%%%%%%%%%%%%%%
%
%
%
%
%


$f,$ инт. для $f$ имеет смысл

Тогда $\int_E f d\mu = \sum_{k=1}^{\infty} \int_{E_k} f d \mu$

\textbf{Задача.} \quad Из правой части не следует, что интеграл
слева существует

\textbf{Доказательство.} Сначала докажем для неотрицательныхъ
функций $f \geqslant 0, \; f = \sum_{k=1}^{\infty} f \dot
\chi_{E_k}$

Так как этот ряд можно почленно интегрировать, получаем требуемое
равенство. Если $f(x)$ любого знака, то $f = f^+ + f^-$ , далее
примен. доказанную часть теоремы для $f^+$ и $f^-$ (примен., так
как предполагается, что интегрирование слева имеет смысл)


 $f_k \nearrow f$ п.в.,  $f_k \geqslant 0$,
измерим. (на измеримом множестве $E$) $\Rightarrow$ \\ $\lim_{k
\rightarrow \infty} \int_E f_k d \mu = \int_E f d \mu$

\textbf{Доказательство.} \quad Ясно, что $\lim_{k \rightarrow
\infty} \int_E f_k d \mu \leqslant \int_E f d \mu$

Если $\exists k: \; \int_E f_k d \mu = + \infty$, то будет
равенство. Оставшуюся часть доказательства можно провести в
предположении: $\int_E f_k d \mu$ конечен $\Rightarrow f_k$ кон.
п.в.

Пусть $f_k$ кон. вне $F_k, \: \mu F_k = 0$

$F_0$ --- множество, на котором не имеет места монотонная
сходимость

На $E \ (\bigcup_{k=0}^{\infty} F_k) \; f_k \nearrow f$ и все
$f_k$ конечны

Значит, $f = f_1 + \sum_{k=1}^{\infty} (f_{k+1} - f_k)$ $E \
(\bigcup_{k=0}^{\infty} F_k)$

$\int_E f d\mu = \int_E f_1 d \mu + \sum_{k=1}^{\infty} \int_E
(f_{k+1} - f_k) d \mu = \sum_{k=1}^{\infty}(\int_E f_{k+1} d \mu -
\int_E f_k d \mu) + \int_E f_1 d \mu$

%
%
%
%
%
%
%
%%%%%%%%%%%%%%%%%%%%%%%%%%%%%%%%%%%%%%%%%%%%%%%%%%%%%%%%%%%%%%%%%%
%%%%%%%%%%%%%%%%%%%%%%%%%%%%%%%%%%%%%%%%%%%%%%%%%%%%%%%%%%%%%%%%%%
%%%%%%%%%%%%%%%%%          стр 11.0    %%%%%%%%%%%%%%%%%%%%%%%%%%%%
%%%%%%%%%%%%%%%%%%%%%%%%%%%%%%%%%%%%%%%%%%%%%%%%%%%%%%%%%%%%%%%%%%
%%%%%%%%%%%%%%%%%%%%%%%%%%%%%%%%%%%%%%%%%%%%%%%%%%%%%%%%%%%%%%%%%%
%
%
%
%
%

$\lim_{k \rightarrow \infty} \int_E f_k d \mu = \int_E f d \mu$

\textbf{Теорема Б. Леви.} \quad

$f_k(x) \nearrow f(x)$ п.в. $f_k \in L(E) \Rightarrow \lim_{k
\rightarrow \infty} \int_E f_k d \mu = \int_E f d \mu$

$\lozenge$ Переходим к $f_k - f_1 \nearrow f - f_1$

$lim_{k \rightarrow \infty} (\int_E f_k - \int_E f_1) = \int_E f d
\mu - \int_E f_1 d \mu$ $\blacklozenge$


\textbf{Следствие.} \quad Если в условиях теоремы Б. Леви$\int_E
f_k d \mu \leqslant C \Rightarrow f$ кон. п.в. и интегрируема.

Если $\lim_{k \rightarrow \infty} \int_E f_k d \mu = + \infty
\Rightarrow \; f$ неинтегрируема.

\textbf{Теорема Фату.} \quad  $f_k \rightarrow f$ п.в. на $E, \;
f_k(x) \geqslant 0$. Тогда $\int_E f d \mu \leqslant \varliminf_{k
\rightarrow \infty} \int_E f_k d \mu$

\textbf{Задача.} \quad Показать, что этих условий недостаточно для
выполнения равенства

\textbf{Доказательство.} \quad $\varphi_k (x) = \inf_{n \geqslant
k} f_n (x)$. Это монотонно $\nearrow$ последовательность.

$\varphi_k(x) \nearrow f(x)$ $\varphi_k(x) \leqslant f_k(x)$

$\int_E f d \mu = \lim_{k \rightarrow \infty} \int_E \varphi_k d
\mu \leqslant \varliminf \int_E f_k d \mu$

\textbf{Теорема Лебега.} \quad

$f_k(x) \longrightarrow f(x) $ $|f_k(x)| \leqslant \varphi(x) \in
L(E) \; \Rightarrow \int_E f d \mu = \lim_{k \rightarrow \infty}
\int f_k d \mu$

$\lozenge$ $|f_k(x)| \leqslant \varphi(x) \Rightarrow(x) f(x)$ ---
интегрируема.

Сначала докажем для $f_k(x) \geqslant 0$

$\int_E f d \mu \leqslant \varliminf_{k \rightarrow \infty} \int_E
f_k d \mu \leqslant \int_E \varphi d \mu$ по теореме Фату.

$\varphi - f_k \geqslant 0$

$\int_E \varphi d \mu - \int_E f d \mu \leqslant \varliminf_{k
\rightarrow} \int_E (\varphi - f_k) d \mu \leqslant \int_E \varphi
d \mu -$
%
%
%
%
%
%
%
%%%%%%%%%%%%%%%%%%%%%%%%%%%%%%%%%%%%%%%%%%%%%%%%%%%%%%%%%%%%%%%%%%
%%%%%%%%%%%%%%%%%%%%%%%%%%%%%%%%%%%%%%%%%%%%%%%%%%%%%%%%%%%%%%%%%%
%%%%%%%%%%%%%%%%%          стр 11.1    %%%%%%%%%%%%%%%%%%%%%%%%%%%%
%%%%%%%%%%%%%%%%%%%%%%%%%%%%%%%%%%%%%%%%%%%%%%%%%%%%%%%%%%%%%%%%%%
%%%%%%%%%%%%%%%%%%%%%%%%%%%%%%%%%%%%%%%%%%%%%%%%%%%%%%%%%%%%%%%%%%
%
%
%
%
%
$ \varlimsup_{k \rightarrow \infty} \int_E f_K d \mu \Rightarrow
\int_E f d\mu \geqslant \varlimsup_{k \rightarrow \infty} \int_E
f_k d \mu \Rightarrow \int_E f d \mu = \lim_{k \rightarrow \infty}
f_k d \mu$

$-\varphi(x) \leqslant f_k(x) \leqslant \varphi(x)$

$0 \leqslant f_k(x) + \varphi(x) \leqslant 2 \varphi(x)$

Доказанная часть применима к этой последовательности:
 $\lim_{k \rightarrow \infty} (\int_E f_k d \mu + \int_E
\varphi d \mu) = \int_E (f + \varphi) d \mu$ $\blacklozenge$

\textbf{Задача.} \quad Показать, что в теореме Леви условие
интегрируемости нельзя заменить на условие существенной
интегрируемости

\textbf{Задача.} Доказать, что \quad $\lim_{n \rightarrow \infty}
\int_0^{\infty} \frac{1}{(1 + x/n)^n x^{1/n}} = 1$ (обосновать
предельный переход под знаком интеграла)

\textbf{Задача.} \quad $\mu E < \infty, f(x) \geqslant C$.
Доказать, что верна теорема Лебега.


\textbf{Определение.} \quad Срезка $f(x)$

$f^N(x) =
\begin {cases}
f(x), &\text |f(x)| \leqslant N  \\
N, &\text f(x) > N \\
-N, &\text f(x) < -N
\end {cases} $


$\int_E f d \mu = \lim_{N \rightarrow \infty} \int_E f^N d \mu,$
так как $|f^N| \leqslant f \in L(E)$

Если $f \geqslant 0$, то кон. $\lim_{N \rightarrow \infty} \int_E
f^N d\mu$ --- условие существования $\int_E f d \mu$. Это следует
из теоремы Леви.

\textbf{Теорема.} \quad Если $f$ --- интегрируема на $E$, то
$\int_E |f - f^N| d\mu \xrightarrow[N \rightarrow \infty]{} 0$

$\lozenge$ $f - f^N = f^+ - (f^N)^+ - (f^- - (f^N)^-)$

Применим утверждение для положительной и отрицательной части
функции и получим требуемое утверждение. $\blacklozenge$

\textbf{Лемма.} \quad $E \subset [a, b], \: E$ измерима.
$\Rightarrow$

$\EuScript{(M)} \int_a^b \chi_E d x = \mu(E) = (L) \int_{[a,b]}
\chi_E d \mu$

$\lozenge$ Пусть $E$ --- открытое множество, $E = G =
\bigcup_n(\alpha_n, \beta_n)$

$\chi_G(x) = \lim_{k \rightarrow \infty} (\bigcup_{n=1}^k
(\alpha_n, \beta_n))$

По теореме Б. Леви предельным переходом
%
%
%
%
%
%
%
%%%%%%%%%%%%%%%%%%%%%%%%%%%%%%%%%%%%%%%%%%%%%%%%%%%%%%%%%%%%%%%%%%
%%%%%%%%%%%%%%%%%%%%%%%%%%%%%%%%%%%%%%%%%%%%%%%%%%%%%%%%%%%%%%%%%%
%%%%%%%%%%%%%%%%%          стр 12.0    %%%%%%%%%%%%%%%%%%%%%%%%%%%%
%%%%%%%%%%%%%%%%%%%%%%%%%%%%%%%%%%%%%%%%%%%%%%%%%%%%%%%%%%%%%%%%%%
%%%%%%%%%%%%%%%%%%%%%%%%%%%%%%%%%%%%%%%%%%%%%%%%%%%%%%%%%%%%%%%%%%
%
%
%
%
%
получаем требуемую формулу. Переходом к дополнению получим такое
утверждение для замкнутых множеств.  $\blacklozenge$

$E$ --- измерим., $\{F_k\}$ --- замкнут.

$F_k \subset F_{k+1} \subset \ldots \subset E$

$\mu (E \ \bigcup_k F_k) = 0$ (так как для измеримого множества
возможно приближение с любой точностью)

$\chi_E = \chi_A + \chi_{\bigcup_k F_k} \; (A = E \ \bigcup_k
F_k)$

$\chi_E = \chi_A + \lim_{k \rightarrow \infty} \chi_{F_k}, \;
\chi_{F_k} \nearrow \chi$

Применим теорему Б. Леви (или теорему о неотрицательных функциях).
Интеграл Мак--Шейна и Лебега совпадают для $\chi_A$ и
$\chi_{\bigcup_k F_k} \Rightarrow$ они совпадают для $\chi_E$


\textbf{Теорема.} \quad На отрезке прямой интеграл Мак-Шейна
совпадает с интегралом Лебега.

$\lozenge$ Докажем сперва для неотрицательных функций. $f_k
\nearrow f, f_k$ --- простые

Интегралы $f_k$ совпадают. Применим теорему Леви и получим
совпадение пределов.

Если $f$ --- произвольного знака, то $f = f^+ + f^-$. Если $f$
интегрируема по Лебегу $\Rightarrow \; f^+, f^-$ интегрируемы по
Лебегу. $\Rightarrow \; f^+, f^-$ интегрируемы по Мак--Шейну
$\Rightarrow \; f$ интегрируема по Мак--Шейну $\Rightarrow \; |f|$
интегрируема по Мак-Шейну $\Rightarrow \; f^+, f^-$ интегрируемы
по Мак--Шейну$\Rightarrow f^+, f^-$ интегрируемы по Лебегу
$\Rightarrow f$ интегрируема по Лебегу. $\blacklozenge$



\textbf{Теорема.} \quad Если некоторые интегралы сходятся
абсолютно на отрезке $\Rightarrow$ функция интегрируема по Лебегу.

\textbf{Задача.} \quad Привести пример того, что обратное неверно.

\textbf{Задача.} \quad Обобщить утверждение теоремы для всей
прямой.
