
% восьмая лекция по действлану.

\begin{center} \textbf{Лекция 8.} \end{center}
%
%
%
%
%
%
%
%%%%%%%%%%%%%%%%%%%%%%%%%%%%%%%%%%%%%%%%%%%%%%%%%%%%%%%%%%%%%%%%%%
%%%%%%%%%%%%%%%%%%%%%%%%%%%%%%%%%%%%%%%%%%%%%%%%%%%%%%%%%%%%%%%%%%
%%%%%%%%%%%%%%%%%          стр 14.1    %%%%%%%%%%%%%%%%%%%%%%%%%%%%
%%%%%%%%%%%%%%%%%%%%%%%%%%%%%%%%%%%%%%%%%%%%%%%%%%%%%%%%%%%%%%%%%%
%%%%%%%%%%%%%%%%%%%%%%%%%%%%%%%%%%%%%%%%%%%%%%%%%%%%%%%%%%%%%%%%%%
%
%
%
%
%


$(X, \EuScript{A}, \EuScript{M})$

$\varphi: A \rightarrow \mathbb{R}$ --- аддитивная функция.

$\varphi$ абсолютно непрерывна относительно $\mu$, если $\forall E
\in \EuScript{A} \; \mu E = 0 $ имеем $\varphi(E) = 0$

\textbf{Определение.} \quad $\varphi$ сингулярна относительно
$\mu$, если $\exists z \in \EuScript{A}, \; \mu z = 0$, т. что
$\varphi(A) = 0$, если $A \bigcap z = \varnothing$ (или $\forall A
\varphi(A) = \varphi(A \bigcap z)$ )

\textbf{Задача.} \quad

1. \quad $\varphi$ абсолютно непрерывна и сингулярна $\Rightarrow
\varphi \equiv 0$

2. \quad линейная комбинация абсолютно непрерывных (сингулярных)
функций абсолютно непрерывна (сингулярна)

3. \quad $\varphi$ --- абсолютно непрерывна (сингулярна)
$\Leftrightarrow$ $V$, \={V}, \b{V} --- абсолютно непрерывны
(сингулярны)

4. \quad $\varphi_k$ --- последовательность абсолютно непрерывных
(сингулярных) аддитивных функций, $\varphi$ --- её предел (т.е.
$\varphi (A) = \lim_{k \rightarrow \infty} \varphi_k (A) \;
\forall A \in \EuScript{A}) \Rightarrow \varphi $ --- абсолютно
непрерывна (сингулярна)

$\varphi(E) = \int_E f d \mu$ --- абсолютно непрерывная функция

Далее рассматриваем все на $E, \; \mu E < \infty$

\textbf{Утверждение.} \quad $\varphi$ --- абсолютно непрерывна
относительно $\mu \Leftrightarrow \forall \varepsilon~>~0
\\ \exists \delta
> 0 \; \forall A: \: \mu A < \delta \Rightarrow \Rightarrow
|\varphi(A)| < \varepsilon$

Необходимость. $\mu A = 0, \mu A < \delta \; \forall \delta > 0$,
 тогда $\varphi(A) < \varepsilon \; \forall \varepsilon > 0$

Достаточность. Предположим $\varphi$ --- абсолютно непрерывна и
утверждение не выполнено. Сначала докажем для $\varphi \geqslant
0$.

 $\exists \varepsilon: \; \delta_k = \frac{1}{2^k}, \mu A_k <
 \frac{1}{2^k}$, $\varphi(A_k) \geqslant \varepsilon$

 $\varlimsup(A_k) = \bigcap_{m=1}^\infty \bigcup_{k \geqslant m} A_k$

 $\mu \varlimsup A_k = \mu (\bigcap_{m=1}^\infty \bigcup_{k \geqslant m} A_k) \leqslant \mu(\bigcup_{k \geqslant m} A_k)
 \leqslant \sum_{k \geqslant m} \mu A_k < \sum_{k \geqslant m} \frac{1}{2^k} = \frac{1}{2^{m-1}} \Rightarrow 0$

 То есть $\mu A = 0$ (так как не зависит от $m$)

 $\varphi(\varlimsup (A_k)) \geqslant \varlimsup_{k \rightarrow \infty} \varphi (A_k) \geqslant \varepsilon$
(было доказано для $\varphi \geqslant 0$)

%
%
%
%
%
%
%
%%%%%%%%%%%%%%%%%%%%%%%%%%%%%%%%%%%%%%%%%%%%%%%%%%%%%%%%%%%%%%%%%%
%%%%%%%%%%%%%%%%%%%%%%%%%%%%%%%%%%%%%%%%%%%%%%%%%%%%%%%%%%%%%%%%%%
%%%%%%%%%%%%%%%%%          стр 15.0    %%%%%%%%%%%%%%%%%%%%%%%%%%%%
%%%%%%%%%%%%%%%%%%%%%%%%%%%%%%%%%%%%%%%%%%%%%%%%%%%%%%%%%%%%%%%%%%
%%%%%%%%%%%%%%%%%%%%%%%%%%%%%%%%%%%%%%%%%%%%%%%%%%%%%%%%%%%%%%%%%%
%
%
%
%
%

То есть получаем противоречие ($\mu A = 0$ $\varphi(A) \geqslant
\varepsilon$)

Теперь докажем для $\varphi$ произвольной.

Если $\varphi$ абсолютно непрерывна $\Rightarrow V$ абсолютно
непрерывна.

$|\varphi(A)| \leqslant V(A) < \varepsilon \Rightarrow $ выполнено
утверждение

\textbf{Задача.} \quad $\varphi$ сингулярна относительно $\mu$ на
$E \Leftrightarrow \forall \varepsilon > 0 \; \exists A \subset F:
\: \mu A < \varepsilon$

$V (E \setminus A, \varphi) < \varepsilon $

\textbf{Теорема Хана.} \quad $\mu E < \infty, \; \varphi$ ---
аддитивная мера. Тогда $\exists P \subset E \; \forall \\ A
\subset P \; \varphi(A) \geqslant 0, \; \forall A \subset E
\setminus P \varphi(A) \leqslant 0$

\textbf{Доказательство.} \quad

\b{V}$(P) = 0$ равносильно первой части утверждения

\={V}$(E \setminus P) = 0 \Leftrightarrow \exists P \subset E \:
\forall A \subset E \setminus P \; \varphi(A) \leqslant 0$

Докажем утверждение для вариации

Замечание: если \b{V}$(P) = 0$, \={V}$(E \setminus P) = 0$, то
\={V}$(E) = $ \={V}$(P)$

$E \supset A_k : \varphi(A_k) >$ \={V}$(E) - 1/2^k \Rightarrow$
($\exists$ такое $A_k$  по определению \={V})

\={V}$(A_k)$ > \={V}$(E) - 1/2^k \Rightarrow $ \={V}$(E \setminus
A_k) < 1/2^k$

$\varphi(A_k)$ \={V}$(A_k) -$ \b{V}$(E) - 1/2^k$

\={V}$(A_k)$ - \b{V}$(A_k)$ > \={V}$(E) - 1/2^k$

\={V}$(A_k)$ < \={V}$(E)$ - \b{V}$(A_k)$

\b{V}$(A_k)$ < $1/2^k$

$p = \varliminf A_k = \bigcup_{m=1}^\infty \bigcap_{k \geqslant m}
A_k$

\b{V}$(P) \geqslant \varliminf_{k \rightarrow \infty}$ \b{V}$(A_k
= 0)$

\={V} $(E \setminus \varliminf A_k) = $ \={V}
$(\bigcap_{m=1}^\infty \bigcup_{k \geqslant m} (E \setminus A_k))
\leqslant$
 \={V}
$(\bigcup_{k \geqslant m} (E \setminus A_k))
\\ \leqslant \sum_{k = m}^\infty$
\={V}
$(E \setminus A_k) < \sum_{k=1}^\infty
\dfrac{1}{2^k} = \dfrac{1}{2^{m-1}} \rightarrow 0$

$V(E \setminus A_k) < 1/2^k$

\={V}$(E \setminus P) = 0$


\textbf{Теорема. (об обобщенном разложении Хана)} \quad

$\varphi(A) \geqslant 0, \mu $ --- мера, тогда $\forall a > 0 \; E
= Z \bigcup (\bigcup_{k = 1}^\infty E_k), \mu Z = 0, E_k: \;
\forall \: A \in E_k$ выполнено

$a(k - 1) \mu A \leqslant \varphi(A) \leqslant a k \mu A$


%
%
%
%
%
%
%
%%%%%%%%%%%%%%%%%%%%%%%%%%%%%%%%%%%%%%%%%%%%%%%%%%%%%%%%%%%%%%%%%%
%%%%%%%%%%%%%%%%%%%%%%%%%%%%%%%%%%%%%%%%%%%%%%%%%%%%%%%%%%%%%%%%%%
%%%%%%%%%%%%%%%%%          стр 15.1    %%%%%%%%%%%%%%%%%%%%%%%%%%%%
%%%%%%%%%%%%%%%%%%%%%%%%%%%%%%%%%%%%%%%%%%%%%%%%%%%%%%%%%%%%%%%%%%
%%%%%%%%%%%%%%%%%%%%%%%%%%%%%%%%%%%%%%%%%%%%%%%%%%%%%%%%%%%%%%%%%%
%
%
%
%
%

Достаточно доказать при $a = 1$ (иначе от $\varphi$ переходим к
$\varphi/a$)

Рассмотрим $\varphi$ --- $\mu$ -- функция множества.


 $\exists P = E^+ : \; \forall A \; \subset E^+ :
\varphi(A) \geqslant \mu(A)$ Пусть $E \setminus P = E^-: \;
\forall A \; \subset E^-: \varphi(A) \leqslant \mu(A)$

Пусть $E^- = E_1$

$\varphi - 2\mu$. Применяем к ней предыдущую теорему.

$E^+ \rightarrow E^{++} \quad \varphi(A) \geqslant 2 \mu(A)$

$\phantom{E^+.}^\searrow E^{+-}  \quad \mu A \leqslant \varphi(A)
\leqslant 2 \mu A$

$\phantom{xxx}$

$E^{\underbrace{+ + \ldots +}_{k-1}} \rightarrow E^{\underbrace{+
+ \ldots +}_{k}} \quad \quad \varphi(A) \geqslant k \mu (A)$


$\phantom{E^{\underbrace{+ + \ldots +}_{k-1}}} ^\searrow
E^{\underbrace{+ + \ldots + -}_{k}} \quad \quad (k-1) \mu A
\leqslant \varphi(A) \leqslant k \mu (A)$


$\phantom{xxxx} $

$Z = \bigcap_{k=1}^{\infty} E^{\underbrace{++ \ldots +}_k}$

$\phantom{xxxx} $

$\mu(z) \leqslant \dfrac{\varphi(Z)}{k} \xrightarrow[k \rightarrow
\infty]{} 0 \Rightarrow \mu Z = 0$

\textbf{Теорема. (о разложении Лебега)} \quad  $\varphi$ ---
аддитивная функция на $E$, $\mu - \delta$--конечна. Тогда
существует однозначное представление $\varphi: \; \lambda +
\delta$, где $\alpha$ --- абсолютно непрерывна,
$\delta$--сингулярна. Причем $\alpha(A) = \int_A f d \mu, \:
f$~определено однозначно $\delta (A) = \varphi(A \bigcap Z)$, где
$Z$
--- фиксирована, $\mu Z$. Если $\varphi \geqslant 0$, то $f~\geqslant 0$

$\lozenge$ Сначала рассмотрим случай $\mu < \infty$

Применим обобщенную теорему Хана для последовательности
$a_m~=~1/2^m$

$E = Z^m \bigcup (\bigcup_{k=1}^\infty E_k^m)$

$\dfrac{k-1}{2^m} \mu A \leqslant \varphi (A) \leqslant
\dfrac{k}{2^m} \mu A, \; A \subset E_k^m$

$[a,b] \quad b<c \quad [c,d]$

$a \mu A \leqslant \varphi(A) \leqslant b \mu A, c \mu A \leqslant
\varphi(A) \leqslant d \mu A \Rightarrow \mu A = 0$
%
%
%
%
%
%
%
%%%%%%%%%%%%%%%%%%%%%%%%%%%%%%%%%%%%%%%%%%%%%%%%%%%%%%%%%%%%%%%%%%
%%%%%%%%%%%%%%%%%%%%%%%%%%%%%%%%%%%%%%%%%%%%%%%%%%%%%%%%%%%%%%%%%%
%%%%%%%%%%%%%%%%%          стр 16.0    %%%%%%%%%%%%%%%%%%%%%%%%%%%
%%%%%%%%%%%%%%%%%%%%%%%%%%%%%%%%%%%%%%%%%%%%%%%%%%%%%%%%%%%%%%%%%%
%%%%%%%%%%%%%%%%%%%%%%%%%%%%%%%%%%%%%%%%%%%%%%%%%%%%%%%%%%%%%%%%%%
%
%
%
%
%
Имеем $E_k^m \subset E_{2k - 2}^{m+1} \bigcup E_{2k-1}^{m+1}
\bigcup E_{2k}^{m+1} \bigcup E_{2k + 1}^{m+1} \bigcup Y^m$

$Z = (\bigcup Z^m) \bigcup (\bigcup Y^m), \mu Z = 0$


$f_m(x) =
\begin {cases}
(k-1)/2^m, &\text x \in E_k^m \setminus Z  \\
z, &\text x \in Z
\end {cases} $

$| f_{m+1} (x) - f_m(x) | \leqslant 1/2^m, \quad f_m
\rightrightarrows f$

$A \subset E, \; A$ --- произвольное

$A \bigcap E_k^m$

$\varphi(A) = \varphi (A \bigcap Z) + \sum_k \varphi (A \bigcap \:
(E_k^m \setminus Z))$ \fbox{ $\leqslant$}

$\varphi(A \bigcap Z) + \sum \varphi (A \bigcap E_k^m) \geqslant
\int_A f_m d \mu + \varphi(A \bigcap Z)$

\fbox{ $\leqslant$} $\int_A f_m d \mu + 1/2^m \mu (A)$

$\varphi(A) = \varphi(A \bigcap Z) + \int_A f d \mu, \; \int_A f_m
d \mu = \sigma, \int_A f d \mu = \alpha$

$\varphi = \sigma_1 + \alpha_1 = \sigma_2 + \alpha_2 \Rightarrow
\alpha_1 - \alpha_2 = \sigma_2 - \sigma_1 \equiv 0$

(функция одновременно абсолютно непрерывна и сингулярна)
