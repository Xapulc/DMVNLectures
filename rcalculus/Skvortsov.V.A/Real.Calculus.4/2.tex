

\begin{center}
\textbf{Лекция 2.}
\end{center}

$$\mu ^* E = \inf\limits_{\bigcup_{k=1}^{\infty} \supset E}
\sum\limits_{k=1}^{\infty} m\:P_k, \; P_k \in S.$$

\textbf{Задача.} \quad Определить $\mu^*$ для покрытий
$\bigcup_{k=1}^{\infty} \supset E$, эквивалентных~исходному.

Если $A \in R(S)$, то $\mu^*(A) = m'(A)$

$A \subset \bigcup_{k=1}^{\infty}P_k \quad m'(A)\leqslant \sum_k
mP_k \Rightarrow m'(A) \leqslant \mu^*(A)$

Так как $A \in R(S)$, то $A = \bigcup_{i=1}^n A_i, \; A_i \in S.$

$m'(A) = \sum_{i=1}^n mA_i \geqslant \mu^*(A)$, (т.к.
$\bigcup_{i=1}^n A_i$ --- одно из покрытий $A$)

Далее будем считать, что есть функция $\lambda$, определенная на
всех подмножествах~Х и обладающая свойствами внешней меры $\mu^*$:

1) $\lambda(\o) = 0$

2) $A \subset B \Rightarrow \lambda(A) \leqslant \lambda(B)$

3) $A = \bigcup_{i=1}^{\infty} \Rightarrow \lambda(A) \leqslant
\sum_{i=1}^{\infty} \lambda(A_i)$

$E \subset X$ называется $\lambda$-измеримым, если $\forall A
\quad \lambda(A) = \lambda(A \bigcap E) + \lambda(A \setminus E)$


\textbf{Теорема.} \quad Семейство множеств, измеримых по
Каратеодори, образует $\delta$-алгебру множеств, и внешняя мера
$\lambda$ является $\delta$-аддитивной мерой на этой алгебре.

\textbf{Задача.} \quad Для проверки того, что семейство является
алгеброй, достаточно проверить его замкнутость относительно
дополнения и $\bigcup$.

Замкнутость относительно дополнения видна из определения:

$\lambda (A)~=~\lambda~(A \bigcap E)~+~\lambda(A\bigcap\;
(X\setminus E))$, т.е. выполнение равенства для Е влечет за собой
выполнение равенства для $X \setminus E$





%%%%%%%%%%%%%%%%%%%%%%%%%%%%%%%%%%%%%%%%%%%%%%%%%%%%%%%%%%%%%%%%%%%%
%%%%%%%%%%%%%%%%%%%%%%%%%%%%%%%%%%%%%%%%%%%%%%%%%%%%%%%%%%%%%%%%%%%%
%%%%%%%%%%%%%%%%%%%         стр 3.1     %%%%%%%%%%%%%%%%%%%%%%%%%%%%
%%%%%%%%%%%%%%%%%%%%%%%%%%%%%%%%%%%%%%%%%%%%%%%%%%%%%%%%%%%%%%%%%%%%
%%%%%%%%%%%%%%%%%%%%%%%%%%%%%%%%%%%%%%%%%%%%%%%%%%%%%%%%%%%%%%%%%%%%




Проверка замкнутости относительно $\bigcup\;$:

$\lambda(A \bigcap \; (E_1 \bigcup E_2)) + \lambda (A \setminus
(E_1 \bigcup E_2)) = \lambda (A \bigcap \; (E_1 \bigcup E_2)
\bigcap E_1) + \lambda (A \bigcap \; (E_1 \bigcup E_2) \setminus
E_1) + \lambda (A \setminus (E_1 \bigcup E_2)) = \lambda (A
\bigcap E_1) + \lambda (A \setminus E_1) = \lambda(A).$

Индуктивно доказывается замкнутость относительно любого конечного
количества объединений:

$\lambda (E_1 \bigcup E_2) = \lambda(E_1) + \lambda(E_2)$

$E = \bigcup_{i=1}^{\infty} E_i, \quad S_n = \bigcup_{i=1}^n E_i$

$\lambda(A) = \lambda (A \bigcap S_n) + \lambda (A \setminus S_n)
\geqslant \sum_{i=1}^n \lambda(A \bigcap E_i) + \lambda (A
\setminus \bigsqcup_{i=1}^{\infty} E_i) \Rightarrow
\lambda(A)~\geqslant~\sum_{i=1}^{\infty} \lambda(A \bigcap E_i) +
\lambda(A \setminus \bigsqcup_{i=1}^{\infty}E_i) \geqslant
\lambda(A \bigcap E) + \lambda (A \setminus E) \geqslant
\lambda(A)$, т.е. все неравенства можно заменить на равенства.

$\delta$-аддитивность: возьмем в качества $A$ само $E$. Получим:

$\lambda (E) = \sum_{i=1}^{\infty} \lambda(E_i)$

$E = \bigcup_{i=1}^{\infty} \lambda (E_i)$

$E = \bigcup_{i=1}^{\infty} E_i = E_1 \bigcup (E_2 \setminus E_1)
\bigcup \ldots \bigcup \;(E_i \setminus \bigcup_{k=1}^{i-1})
\bigcup \ldots = \\ E_1 \bigsqcup \;(E_2 \setminus E_1) \bigsqcup
\ldots \bigsqcup (E_i \setminus \bigcup_{k=1}^{i-1} E_k) \bigsqcup
\ldots \Rightarrow$ получается $\delta$-алгебра.

Применяя определение Каратеодори к $\mu^*$, будем называть
получившийся класс множеств \textbf{множествами, измеримыми по
Лебегу.}

Если $\lambda(E)=0 \Rightarrow E$ --- измеримо.

\textbf{Определение.} \quad Мера называется полной, если при
$\delta(E)=0 \; \forall \;E_1 \subset E$ имеем $\delta(E_1) = 0$

Для множеств, имеримых по Лебегу, $\mu^*$ обозначается $\mu$.

\textbf{Определение.} \quad Наименьшая $\delta$--алгебра множеств,
содержащая все открытые множества, называется \textbf{борелевской}
алгеброй. Каждое множество --- борелевским множеством.





%%%%%%%%%%%%%%%%%%%%%%%%%%%%%%%%%%%%%%%%%%%%%%%%%%%%%%%%%%%%%%%%%%%
%%%%%%%%%%%%%%%%%%%%%%%%%%%%%%%%%%%%%%%%%%%%%%%%%%%%%%%%%%%%%%%%%%%
%%%%%%%%%%%%%%%%%%%%%%%%    4 стр.  %%%%%%%%%%%%%%%%%%%%%%%%%%%%%%%
%%%%%%%%%%%%%%%%%%%%%%%%%%%%%%%%%%%%%%%%%%%%%%%%%%%%%%%%%%%%%%%%%%%
%%%%%%%%%%%%%%%%%%%%%%%%%%%%%%%%%%%%%%%%%%%%%%%%%%%%%%%%%%%%%%%%%%%






$\EuScript{F}$ --- замкнутые множества, $\EuScript{Y}$ ---
открытые множества.

$\EuScript{F_{\sigma}}$ --- объединение, $\EuScript{Y_{\delta}}$
--- пересечение.

$\EuScript{F_{\sigma \delta}}, \EuScript{Y_{\delta \sigma}}$

$\lambda$ определяем на подмножествах метрического пространства.


Мера~$\lambda$ называется~\textbf{метрической}, если $\forall A, B
\; \rho(A, B){>}0$ имеем: $\\ \lambda~(A \bigsqcup B)
=~\lambda(A)~+~\lambda(B) \\ \rho(A, B) = \inf\limits_{x \in A, y
\in B} \rho(x, y).$

\textbf{Лемма.} \quad $X$ --- метрическое пространство, $\lambda$
--- метрическая мера, $E~\in~G, \; G - \mbox{открытое}, \; E_k = \{ x \in E : \;
\rho(x, x\setminus G) \geqslant 1/k \} \Rightarrow \lim_{k
\rightarrow \infty} \lambda E_k = \lambda E$

Очевидно, что $\lim_{k \rightarrow \infty} \lambda E_k \leqslant
\lambda E, \;$ т.к. $E_k$ --- последовательность расширяющихся
множеств.

Обозначим $D_k = E_{k+1} \setminus E_k. \\ \rho(D_{k+1}, E_k) \\
x \in E_k, \; y \in D_{k+1},\; z \in X\backslash G \\ \frac{1}{k}
\leqslant \rho(x, z) \leqslant \rho (z, y) + \rho(y, x) \\
\rho(y, x) \geqslant \frac{1}{k} - \frac{1}{k+1} - \varepsilon >
\alpha > 0$

Тогда имеем: $\rho\:(D_{k+1}, E_k) > 0 \\ \lambda(E) \leqslant
\lambda(E_k) + \sum_{i=k}^{\infty}\lambda(D_i) \Rightarrow \\
\lambda(E) \leqslant \lim_{k \rightarrow \infty} \lambda(E_k)
\quad \mbox{(если ряд сходится)}$

$\rho (D_{k+1}, D_{k-1}) > 0 \\ \sum_{i=k}^{\infty} \lambda(D_i) =
\sum\limits_{i = 2t} + \sum\limits_{i = 2t + 1} \\
\mbox{Хотя бы один ряд сходится к +} \infty. \; \mbox{Допустим,
что } \sum_{i=1}^{\infty} \lambda(D_{2k}) = + \infty \\
\lambda(E_{2k+1}) \rightarrow {k \rightarrow \infty} + \infty,
\mbox{но тогда } lim_{k \rightarrow \infty} = + \infty \;
\Rightarrow \; lim_{k \rightarrow \infty} \lambda E_k \geqslant
\lambda E$



%%%%%%%%%%%%%%%%%%%%%%%%%%%%%%%%%%%%%%%%%%%%%%%%%%%%%%%%%%%%%%%%%%%
%%%%%%%%%%%%%%%%%%%%%%%%%%%%%%%%%%%%%%%%%%%%%%%%%%%%%%%%%%%%%%%%%%%
%%%%%%%%%%%%%%%%%%%%%        стр 4.1            %%%%%%%%%%%%%%%%%%%
%%%%%%%%%%%%%%%%%%%%%%%%%%%%%%%%%%%%%%%%%%%%%%%%%%%%%%%%%%%%%%%%%%%
%%%%%%%%%%%%%%%%%%%%%%%%%%%%%%%%%%%%%%%%%%%%%%%%%%%%%%%%%%%%%%%%%%%


\textbf{Теорема.} \quad Если $\lambda$ --- внешняя метрическая
мера, то класс борелевых множеств входит в $\delta$--алгебру
$\lambda$--измеримых множеств.

Достаточно доказать для замкнутых множеств. \\
$E$ --- замкнуто,
$\; A$ --- произвольное множество, $\; G = X \setminus E \\
B_k =
\{ x \in A: \rho (x, E) \geqslant \frac{1}{k} \} \\ \lim_{k
\rightarrow +\infty} \lambda B_k = \lambda (A \setminus E) \quad
\mbox{(по лемме)} \\
\mbox{В силу метричности } \lambda: \\
\lambda(B_k \bigsqcup \: (A
\bigcap E)) = \lambda (B_k) + \lambda (A \bigcap E),\quad
\lambda(B_k \bigsqcup \: (A \bigcap E)) \leqslant
\lambda (A)\\
\lambda(A) \geqslant \lambda(A \setminus E) + \lambda(A \bigcap
E)\\
\lambda(A) \leqslant \lambda(A \setminus E) + \lambda(A \bigcap E)
\mbox{ в силу полуаддитивности} \\ \Rightarrow \lambda(A) =
\lambda(A \setminus E) + \lambda(A \bigcap E)$

Покажем, что в $\mathbb{R}^* \; \mu^*$ является метрической, если
$\rho(E_1, E_2) = \alpha > 0$. (В~покрытии будем использовать
$k$-мерные интервалы диаметром меньше~$\alpha / 2$. Тогда покрытие
разделяется, и при переходе к inf получим аддитивность)

Если мера определена на~$\delta$-алгебре борелевских множеств,
будем называть её борелевской.
