
% тринадцатая лекция по действлану.

\begin{center} \textbf{Лекция 13.} \end{center}

\textbf{} \quad $f$ $X$ $E_k$
%
%
%
%
%
%
%%%%%%%%%%%%%%%%%%%%%%%%%%%%%%%%%%%%%%%%%%%%%%%%%%%%%%%%%%%%%%%%%%
%%%%%%%%%%%%%%%%%%%%%%%%%%%%%%%%%%%%%%%%%%%%%%%%%%%%%%%%%%%%%%%%%%
%%%%%%%%%%%%%%%%%          стр 23.0    %%%%%%%%%%%%%%%%%%%%%%%%%%%%
%%%%%%%%%%%%%%%%%%%%%%%%%%%%%%%%%%%%%%%%%%%%%%%%%%%%%%%%%%%%%%%%%%
%%%%%%%%%%%%%%%%%%%%%%%%%%%%%%%%%%%%%%%%%%%%%%%%%%%%%%%%%%%%%%%%%%
%
%
%
%
%
влечет удовлетворение условия Коши на $X$

$L_p, 1 < p < \infty$

Можно определить $L_\infty$

\textbf{Существенный супремум} \quad  $ ess \sup_x|f(x)| =
\inf\{C: \mu{|f(x)|> C}~=~0\} = \|f\|$ в $L_\infty$.

Элементы $L_\infty$ --- классы эквивалентных функций.

$f \sim g \Leftrightarrow f = g$

\textbf{Задача} \quad Проверить свойство нормы.

Пространство $L_o$ --- пространство всех измеримых функций

$L_0(x): \: \rho(f,g) = \int_X \dfrac{|f-g|}{1 + |f-g|} d \mu
\quad \mu X < \infty$

\textbf{Задача 1.} \quad Проверить $\rho(f,g)_{L_0}$ --- метрика.

\textbf{Задача 2.} \quad Сходимость относительно $\rho(f,g)
\Leftrightarrow$ сходимость по мере

В $L_2$ можно ввести скалярное произведение $(f,g) = \int_X f
\overline(g) d \mu$ (для комплексных функций)

$L_2$ --- полное пространство со скалярным произведением. (то есть
гильбертово пространство)

$(f,g) \leqslant \|f\|\cdot \|g\|$

\textbf{Определение.} \quad Метрическое протранство называется
сепарабельным, если в нем найдется счетное всюду плотное
множество.

Докажем, что $L_p([a,b])$ сепарабельно. В качестве счетного
множества берем множество всех многоленов с рациональными
коэффициентами. Покажем, что оно всюду плотно в$L_p$. Для $L_1$
ясно (применим теорему о приближении непрерывной функцией функции
интеграла по Лебегу, затем применим теорему Вейерштрасса о
приближении многочленами, затем приблизим полиномом с
рациональными коэффициtнтами)

Для $L_p : \: f \geqslant 0 \quad f_n$ --- простые. По теореме
Б.Леви $\|f_n - f\|_p \rightarrow 0$


%
%
%
%
%
%
%%%%%%%%%%%%%%%%%%%%%%%%%%%%%%%%%%%%%%%%%%%%%%%%%%%%%%%%%%%%%%%%%%
%%%%%%%%%%%%%%%%%%%%%%%%%%%%%%%%%%%%%%%%%%%%%%%%%%%%%%%%%%%%%%%%%%
%%%%%%%%%%%%%%%%%          стр 23.1    %%%%%%%%%%%%%%%%%%%%%%%%%%%%
%%%%%%%%%%%%%%%%%%%%%%%%%%%%%%%%%%%%%%%%%%%%%%%%%%%%%%%%%%%%%%%%%%
%%%%%%%%%%%%%%%%%%%%%%%%%%%%%%%%%%%%%%%%%%%%%%%%%%%%%%%%%%%%%%%%%%
%
%
%
%
%
Простые функции --- линейные комбинации характеристических
функций. Приближение полиномами хар. функций изм. множеств.
Измеримое пространство приближается открытым множеством. Открытое
множество --- счетное объединение интервалов. Приближение
характеристической функцией конечного числа интервалов (выбр.
счетное число интервалов суммарной длины < $\varepsilon$). Значит,
надо приближать характеристической функцией инт. с точностью до
$varepsilon$. Затем применить теорему Вейерштрасса.

\textbf{Свойства гильбертова пространства.} \quad

$f$ и $g$ ортогональны, если $(f,g) = 0$

$\|f\| = \sqrt{(f,f)}$

Можно рассматривать ортонормированные системы (О.Н.С.)

$X$ --- гильбертово пространство, $f$ ---  его элемент.

$\{e_n\}$ --- О

Коэффициенты Фурье $\hat{f_n} = (f, e_n)$

\textbf{Теорема.} \quad $f \in X, X$ --- гильбертово пространство.
Тогда $\|f - \sum_{k=1}^n \alpha_k e_k\|^2$  --- минимально, если
$\alpha_k = \hat{f_k}$

\textbf{Доказательство.} \quad

$(f - \sum_{k=1}^n \alpha_k e_k, f - \sum_{k=1}^n \alpha_k e_k) =
\|f\|^2 + \sum_{k=1}^n |\alpha_k - \hat{f_k}|^2 - \sum_{k=1}^n
|\hat{f_k}|^2 , min$ получается при $\alpha_k = \hat{f_k}$

Получаем равенство:

$\|f - \sum_{k=1}^n \hat{f_k} e_k\|^2 = \|f\|^2 - \sum_{k=1}^n
|\hat{f_k}|^2$, т.к. левая часть $\geqslant$ 0, то

$\|f\|^2 \geqslant \sum_{k=1}^n |\hat{f_k}|^2$

Переходим к пределу при  $n \rightarrow \infty$, получим
неравенство Бесселя:

$\|f\|^2 \geqslant \sum_{k=1}^\infty |\hat{f_k}|^2$

Если ряд Фурье сходится к функции, то

$\|f\|^2 \geqslant \sum_{k=1}^\infty |\hat{f_k}|^2$ --- равенство
Парсеваля.

%
%
%
%
%
%
%%%%%%%%%%%%%%%%%%%%%%%%%%%%%%%%%%%%%%%%%%%%%%%%%%%%%%%%%%%%%%%%%%
%%%%%%%%%%%%%%%%%%%%%%%%%%%%%%%%%%%%%%%%%%%%%%%%%%%%%%%%%%%%%%%%%%
%%%%%%%%%%%%%%%%%          стр 23.1    %%%%%%%%%%%%%%%%%%%%%%%%%%%%
%%%%%%%%%%%%%%%%%%%%%%%%%%%%%%%%%%%%%%%%%%%%%%%%%%%%%%%%%%%%%%%%%%
%%%%%%%%%%%%%%%%%%%%%%%%%%%%%%%%%%%%%%%%%%%%%%%%%%%%%%%%%%%%%%%%%%
%
%
%
%
%
Равенство Парсеваля $\Leftrightarrow$ сходимости ряда Фурье к
своей функции $f$

\textbf{Теорема (Мермера).} \quad Коэффициенты Фурье по
ограниченной (поточечно) ортонормированной системе в $L_1$
стремятся к нулю. (при $n \rightarrow \infty$)

$L_2(x)$ $\mu X < \infty$

$\{e_n\}$ $|e_n(x)| \leqslant M \quad \forall x \in X \; \forall
n$

$\hat{f_k} = \int f e_k d \mu$

$\forall \varepsilon > 0 \exists N: \int_X |f_N - f| d \mu <
\varepsilon$ ($f_N$ --- срезка)

$f_N \in L_2(x) \Rightarrow $ для  $f_N$ выполнено неравенство
Бесселя, т.е. коэффициенты Фурье $\rightarrow$ 0.

$|\hat{f_k}| = |\int_I f e_k d \mu| \leqslant |\int_X (f - f_N)e_k
d \mu| + |\int_X f_N e_k d \mu| \leqslant \int_X |f - f_N| d \mu +
|\hat{f_N}^k| \leqslant \dfrac{\varepsilon}{2M}\cdot M +
\varepsilon/2 = \varepsilon. \quad k \geqslant n_o$

О.Н.С. не ограничена в совокупности в $L_2([0,1])$


 $\chi_0 = 1$

$\chi_1 =
\begin{cases}
\phantom{-}1, &\text{$\chi \in [0, 1/2)$}\\
 -1,&\text{$\chi \in (1/2,0]$}
\end{cases} $

$\chi_2 =
\begin{cases}
\phantom{-}\sqrt{2}, &\text{$\chi \in [0, 1/4)$}\\
-\sqrt{2}, &\text{$\chi \in (1/4, 1/2)$}\\
 -- // --
\end{cases} $

$\chi_3 =
\begin{cases}
-- // --
\phantom{-}\sqrt{2}, &\text{$\chi \in (1/2, 3/4)$}\\
 - \sqrt{2},&\text{$\chi \in (3/4,1]$}
\end{cases} $


Далее определим функции от $2^k, \ldots, 2^{k+1} - 1$


$\chi_{2^k} =
\begin{cases}
\phantom{-}2^(k/2), &\text{$\chi \in [0, 1/2^{k+1})$}\\
-2^{k/2}, &\text{$\chi \in (1/2^{k+1}, 1/2^k)$}\\
-- // --
\end{cases} $

Далее сдвигаем. В точках разрыва --- как среднее арифметическое
пределов или по непрерывности справа.

\textbf{Теорема Риса--Фишера.} \quad

$\{e_k\}$

$\sum_{k=1}^\infty |c_k|^2 < \infty \quad (\{c_k\} \in L_2)
\Rightarrow \exists f \in X : c_k = \hat{f_k}$ по системе
$\{e_k\}$

И можно выбрать $f$ так, чтобы выполнялось равенство Парсеваля.

\textbf{Доказательство.} \quad $T_n = \sum_{k=1}^n c_k e_k$

$T_n$ удовлетворяет условию Коши:  $\|T_n - T_m\|^2 -
\sum_{k=m+1}^n |c_k|^2 \xrightarrow[m, n \rightarrow \infty]{} 0$

Значит, $T_n \rightarrow f$

$(f, e_k) = (f - T_n, e_k) + (T_n, e_k) = (\text{по неравенству
Коши--Бун.}) (f - T_n, e_k) + c_k$

$(f - T_n, e_k) \rightarrow 0$ $n \rightarrow \infty$

$(f, e_k) = c_k$

Т.к. $T_n \rightarrow f$, выполнено равенство Парсеваля (т.к.
$T_n$ --- частные суммы ряда Фурье)

$\{e_n\}$ полная, если $(f,e_n) = 0 \forall n \; f = 0$

\textbf{Задача.} \quad Полнота эквивалентна равенству Парсеваля
$\forall f$

\textbf{Признаки сходимости ряда Фурье.} \quad

Признак Дини:  $\dfrac{f(x+t) + f(x-t) - 2S}{t} \in L(o, \delta)$
в т. $x$

ряд Фурье по тригонометрической сумме сходится к $S$ (наиболее
часто встречается  $S = f(x)$)

 $S = \dfrac{f(x+0) + f(x-0)}{2}$
