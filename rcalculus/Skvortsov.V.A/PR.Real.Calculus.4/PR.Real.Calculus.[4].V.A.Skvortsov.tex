\documentclass[a4paper]{article}
\usepackage[xe,simple]{dmvn}
\usepackage{dmvnadd}

\title{Материалы к экзамену по действительному анализу}
\author{Лектор\т В.\,А.\,Скворцов}
\date{IV семестр, 2004--2009 г.}

\begin{document}
\maketitle
\begin{center}
  {\footnotesize
  Формулировки и определения: материал предоставлен Д.\,А.\,Григоренко
  и~пополнен в 2009~г. Ю.\,Г.\,Кудряшовым.\\
  Решения, набор и вёрстка: Д.\,Н.\,Вельтищев, М.\,Н.\,Вельтищев,
  В.\,А.\,Клепцын, Ю.\,Г.\,Кудряшов.\par}
\end{center}

\section{Соглашения, обозначения, формулировки и определения}

Символом $X$, как правило, будем обозначать основное пространство.
Будем обозначать через $\Bs(X)$ борелевскую $\si$\д алгебру на $X$.
Интегрируемость (суммируемость) функций подразумевается по Лебегу.
Совокупность интегрируемых на~$X$ функций обозначается~$L(X)$.

\begin{df}
  Пусть $f$\т измеримая функция. Для каждого $N \bw\in \N$ положим
  \eqn{[f]^N(x) = \case{f(x), & |f(x)| \le N;\\ \phantom{-}N, & f(x)>N;\\ -N, & f(x) < -N.}}
  Такие функции называются \emph{срезками} функции~$f$.
\end{df}

\begin{df}
  Множество~$S$ называется \emph{полукольцом}, если $\fa A,B\bw\in S$ имеем $A\bw\cap B \bw\in S$ и
  $\exi C_1\sco C_N \bw\in S\cln A \bw\wo B \bw= \bigsqcup C_i$.
\end{df}

\begin{df}
  Множество $R$ называется \emph{кольцом}, если оно замкнуто относительно $\cap$ и $\swo$.
\end{df}

\begin{df}
  Множество $X \bw\in S$ называется \emph{единицей}, если $\fa A \bw\in S$ имеем $A \bw\subs X$.
\end{df}

\begin{df}
  \emph{$\si$\д кольцо}\т кольцо, замкнутое относительно счётного объединения.
\end{df}

\begin{df}
  \emph{Алгебра}\т кольцо с единицей.
\end{df}

\begin{df}
  \emph{$\si$\д алгебра}\т $\si$\д кольцо с единицей.
\end{df}

\begin{theorem}
  Для данной системы множеств $S$ существует единственное минимальное кольцо,
  содержащее её. Обозначение: $R(S)$.
\end{theorem}

\begin{theorem}
  Пусть $S$\т полукольцо, тогда $R(S)$\т всевозможные конечные
  объединения непересекающихся элементов $S$.
\end{theorem}

\begin{df}
  Пусть на полукольце $S$ задана аддитивная \emph{мера}~$m$, \те задана функция
  $m\cln S \bw\ra [0,+\bes)$ со свойством $m(A \bw\sqcup B) \bw= m(A) \bw+ m(B)$ для любых
  $A,B \bw\in S$ таких, что $A \bw\sqcup B \bw\in S$.
  Мера $m$ называется \emph{$\si$\д аддитивной}, если для любого набора
  $\hc{A_i}_{i=1}^\bes \bw\subs S$ такого, что $\bigsqcup A_i \bw\in S$, имеем
  $m\hr{\bigsqcup A_i} \bw= \sum m(A_i)$.
\end{df}

\begin{theorem}
  Существует единственное продолжение $m'$ меры $m$ с полукольца $S$ на $R(S)$.
  При этом если $m$ $\si$\д аддитивна на $S$, то
  и $m'$ также $\si$\д аддитивна на $R(S)$.
\end{theorem}

\begin{theorem}[Монотонность меры]
  Если $A \bw\subs B$, то $m(A) \bw\le m(B)$.
\end{theorem}

\begin{theorem}[Счётная полуаддитивность $\si$\д аддитивной меры на кольце]
  Пусть $\hc{A_i} \bw\subs R, A \bw\subs \bigcup A_i$, тогда $m(A) \bw\le \sum m(A_i)$.
\end{theorem}

\begin{df}
  Пусть $m$\т $\si$\д аддитивная мера на полукольце $S$,
  $m'$\т её продолжение на $R(S)$, причём $R(S)$\т алгебра,
  тогда \emph{внешней мерой} множества $A\subs X$ называется
  \eqn{\mu^*(A):=\infl{A \subs \cupl{k=1}{\bes} P_k} \suml{k=1}{\bes} m(P_k),\quad\text{ где } P_k \in S.}
\end{df}

\begin{stm}
  Имеет место совпадение $m' \bw= \mu^*\evn{R(S)}$.
\end{stm}

\begin{theorem}[Счётная полуаддитивность $\mu^*$]
  Если $A \bw\subs \cupl{i=1}{\bes}A_i$, то $\mu^*(A) \bw\le \sum\mu^*(A_i)$.
\end{theorem}

\begin{df}
  \emph{Нижняя мера} $\mu_*(A) \bw{:=} m'(X) \bw- \mu^*(X \bw\wo A)$.
\end{df}

\begin{df}
  Множество~$A$ \emph{измеримо}, если $\mu^*(A) \bw= \mu_*(A) \bw{=:} \mu(A)$.
  Число~$\mu(A)$ называется \emph{мерой Лебега} множества~$A$.
\end{df}

\begin{theorem}[Критерий Вале\д Пуссена]
  Множество $A$ измеримо тогда и только тогда, когда
  $\fa\ep \bw> 0 \exi B \bw\in R(S)\cln \mu_*(A\bw\swo B) \bw< \ep$.
\end{theorem}

\begin{theorem}
  Класс измеримых по Лебегу множеств~$\Mg$ является $\si$\д алгеброй,
  а $\mu^*\evn{\Mg}$ является $\si$\д аддитивной.
\end{theorem}

\begin{df}
  Мера $\mu$ называется \emph{непрерывной}, если для любой последовательности
  $E_1 \bw\sups E_2 \bw\sups \ldots$, такой, что $\bigcap E_i \bw\in R$, имеем
  $\mu\hr{\bigcap E_i} \bw= \liml{i\ra\bes} \mu(E_i)$.
\end{df}

\begin{theorem}
  $\si$\д аддитивность меры $\mu$ равносильна непрерывности $\mu$.
\end{theorem}

\begin{df}
  Мера~$\mu$ называется \emph{$\si$\д конечной}, если $\exi\hc{X_i}_{i=1}^\bes \bw\subs S$, что
  единица $X \bw= \bigsqcup X_i$ и
  для $\fa i$ имеем $\mu(X_i) \bw< \bes$. Положим $\mu(X) \bw{:=} \sum \mu(X_i)$.
  Множество~$A$ называется \emph{измеримым}, если для $\fa i$ измеримо $A \bw\cap X_i$.
  В этом случае $\mu(A) \bw{:=} \sum\mu\hr{A\bw\cap X_i}$.
\end{df}

\begin{df}
  Класс \emph{борелевских} множеств $\Bs$\т
  минимальная $\si$\д алгебра, содержащая все открытые множества.
\end{df}

\begin{imp}
  $\Bs$ содержит все замкнутые множества.
\end{imp}

\begin{df}
  \emph{Мера Бореля}\т произвольная $\si$\д аддитивная мера на $\Bs$.
\end{df}

\begin{theorem}
  Пусть полукольцо~$S$ таково,
  что минимальная $\si$\д алгебра, порожденная им, совпадает с~$\Bs$, тогда
  продолжение на~$\Bs$ ограничения меры Бореля на кольцо совпадает с этой мерой Бореля.
\end{theorem}

\begin{stm}
  В любом множестве положительной меры есть неизмеримое подмножество.
\end{stm}

\begin{df}
  \emph{Мера Стилтьеса} на $\R$: рассмотрим $S \bw= \bc{[a,b)\cln a,b \bw\in \R}$ и $F$\т
  произвольную возрастающую функцию на $\R$,
  тогда $\mu_\mathrm{St} \br{[a,b)} \bw{:=} F(b)-F(a)$.
\end{df}

\begin{theorem}
  $F$~непрерывна слева, \те $\liml{\de\ra+0} F(t-\de) \bw= F(t)$, тогда и только тогда,
  когда $\mu_\mathrm{St}$ является $\si$\д аддитивной.
\end{theorem}

\begin{df}
  Пусть в тройке $\hr{X,\Sig,\mu}$ $X$\т основное пространство, $\Sig$\т $\si$\д алгебра
  подмножеств $X$, $\mu$\т $\si$\д аддитивная мера на $\Sig$. Тогда $f\cln X \bw\ra \R$
  \emph{измерима}, если $\fa c \bw\in\R$ имеем $\hc{x \bw\in X\cln f(x)\bw< c} \bw\in \Sig$.
  Или, эквивалентно: для $\fa B \bw\in \Bs(\R)$ имеем $f^{-1}(B) \bw\in \Sig$.
\end{df}

\begin{stm}
  Пусть $a \in \R$, а функции $f$ и $g$ измеримы. Тогда функции $a \bw\cdot f$, $a \bw+ f$,
  $f \bw+ g$, $f \bw\cdot g$, а если $0 \bw\notin g(X)$, то и $f/g$, будут измеримыми.
\end{stm}

\begin{df}
  Мера~$\mu$ называется \emph{полной}, если для всякого множества~$E$ меры нуль любое
  подмножество $F \bw\subs E$ измеримо.
\end{df}

\begin{stm}
Если $\mu$ полна, а~$f$ измерима, то любая функция~$g$, равная~$g$ почти всюду, также будет измеримой.
\end{stm}

\begin{stm}
  Пусть $\hc{f_n}$\т последовательность измеримых функций, тогда $\infl{n}f_n$, $\supl{n}f_n$, а, значит, и
  $\uliml{n} f_n$, $\lliml{n} f_n$, будут измеримы.
\end{stm}

\begin{theorem}
  Множество точек сходимости последовательности измеримых функций измеримо,
  а предел\т измеримая функция.
\end{theorem}

\begin{df}
  Измеримая функция~$f$ называется \emph{простой}, если она принимает конечное множество значений.
  Эквивалентное определение: $\exi E_1\sco E_n \bw\in \Sig$ и $a_1\sco a_n \bw\in \R$, что
  $\bigsqcup E_i \bw= X$ и~$f \bw\equiv a_i$ на~$E_i$.
\end{df}

\begin{theorem}
  Любая измеримая функция есть поточечный предел последовательности простых измеримых
  функций $\hc{f_n}$, причём если $f \bw\ge 0$, то~$f_n$ можно выбрать так,
  чтобы $f_i \bw\nearrow f$.
\end{theorem}

\begin{df}
  Сходимость \emph{по мере}: $f_n \xra{\mu} f$, если для $\fa\ep \bw> 0$ выполнено
  \eqn{\mu\hc{x\in X\cln |f_n(x)-f(x)|>\ep} \ra 0, \; n\ra\bes.}
\end{df}

\begin{theorem}
  Если $\mu(X) \bw< \bes$, то из сходимости почти всюду следует сходимость по мере.
\end{theorem}

\begin{theorem}[Егорова]
  Если $\mu(X) \bw< \bes$, а $f_n \bw\convae f$, то для
  $\fa\ep \bw>0 \exi E \bw\subs X\cln \mu(E) \bw< \ep$, а~$f_n \mskip-10mu \convu{X\wo E}\mskip-10mu f$.
\end{theorem}

\begin{note}
  Если $X \bw\subs \R^n$, то~$E$ можно взять открытым.
\end{note}

\begin{df}
  Функция $f$ обладает $\Cb$\д свойством Лузина в пространстве~$X$,
  если $\fa\ep \bw> 0$ найдётся замкнутое $F_\ep \bw\subs X$, что
  $\mu(X \bw\wo F_\ep)\bw< \ep$ и $f \bw\in \Cb(F_\ep)$.
\end{df}

\begin{theorem}[Лузина]
  Функция в~$\R^n$ измерима тогда и только тогда, когда она обладает $\Cb$\д свойством.
\end{theorem}

\begin{df}
  Пусть мера~$\mu$ из тройки $\hr{X,\Sig,\mu}$ полна.
  Пусть $f\cln X\bw\ra [0,+\bes]$ измерима, $E \bw\in \Sig$, тогда
  \eqn{(L)\ints{E}f\,d\mu := \supl{E=\cupsql{k=1}{n} E_k}
       \suml{k=1}{n}\Br{\infl{E_k}f}\mu(E_k).}
\end{df}

\begin{note}
  Если $\ints{E}f\,d\mu \bw< +\bes$, то~$f$ конечна почти всюду.
\end{note}

\begin{df}
  Если $\ints{E}f\,d\mu\bw<+\bes$, то $f$ называется
  \emph{интегрируемой} по Лебегу (\emph{суммируемой}).
\end{df}

\begin{stm}
  \begin{items}{-3}
    \item Если $\mu(E) \bw= 0$, то $\ints{E} f\,d\mu \bw=0$;
    \item Если $E_1\subs E_2$, то $\ints{E_1}f\,d\mu \bw\le \ints{E_2}f\,d\mu$;
    \item Если $f \bw\le g$ на $E$, то $\ints{E}f\,d\mu \bw\le \ints{E}g\,d\mu$;
    \item Для простых функций $\ints{E}f\,d\mu \bw= \suml{k=1}{n} a_i\mu(A_i)$.
  \end{items}
\end{stm}

\begin{theorem}
Пусть $f_k \bw\nearrow f$, тогда $\liml{k\ra\bes} \ints{E} f_k\,d\mu \bw= \ints{E}f\,d\mu$.
\end{theorem}

\begin{stm}
  \begin{items}{-3}
    \item Линейность интеграла: для всяких $f,g \bw\in L(E)$ и всяких $a,b \bw\in \R$
          имеем $\ints{E}(af \bw+ bg)\,d\mu \bw= a\ints{E}f\,d\mu \bw+ b\ints{E}g\,d\mu$.
    \item Аддитивность относительно множества:
          $\ints{E_1}f\,d\mu + \ints{E_2}f\,d\mu \bw= \ints{E_1\sqcup E_2}f\,d\mu$.
    \item Если $f\bw\eqae g$, то $\ints{E}f\,d\mu \bw= \ints{E}g\,d\mu$.
  \end{items}
\end{stm}

\begin{df}
  Пусть $f$\т измерима, тогда $\ints{E}f\,d\mu \bw{:=} \ints{E}f^+d\mu \bw- \ints{E}f^-d\mu$.
\end{df}

\begin{theorem}
  Пусть $f \bw= \suml{k=1}{\bes} f_k$, где $f_k$ измеримы на $E$ и неотрицательны.
  Тогда \eqn{\ints{E}f\,d\mu = \suml{k=1}\bes \ints{E}f_k\,d\mu.}
\end{theorem}

\begin{imp}
  Пусть $f_k$ измеримы и неотрицательны, а ряд $\suml{k=1}\bes \ints{E}f_k\,d\mu$ сходится.
  Тогда $\sum f_k$ сходится почти всюду.
\end{imp}

\begin{theorem}[Счётная аддитивность интеграла Лебега]
  Пусть $f \bw\in L\br{\bigsqcup E_k}$. Тогда
  \eqn{\ints{\bigsqcup E_k}f\,d\mu=\suml{k=1}\bes\ints{E_k}f\,d\mu.}
\end{theorem}

\begin{theorem}
  Пусть $f_k \bw\nearrow f$, $f_k \bw\ge 0$ на $E$, тогда
  $\liml{k}\ints{E}f_k\,d\mu \bw= \ints{E}f\,d\mu$.
\end{theorem}

\begin{theorem}[Беппо Леви]
  $f_n \bw\nearrow f$ почти всюду, $f_1 \bw\in L(E)$, тогда
  $\ints{E}f\,d\mu \bw= \liml{n}\ints{E}f_n\,d\mu$.
  Если $\ints{E}f_n\,d\mu \bw< \const$, то~$f$
  конечна почти всюду и~$f \bw\in L(E)$, в противном случае $f \bw\notin L(E)$.
\end{theorem}

\begin{theorem}[Фату]
  Пусть $f_n \bw\convae f$, причём $f_n \bw\ge 0$. Тогда
  $\ints{E}f\,d\mu \bw\le \lliml{n\ra\bes} \ints{E}f_n\,d\mu$.
\end{theorem}

\begin{imp}
  Пусть $f_n \bw\convae f$, $f_n \bw\ge 0$ и $\ints{E}f_n\,d\mu \bw\le \const$,
  тогда $\liml{n\ra\bes}f_n(x) \bw< +\bes$ и $f \bw\in L(E)$.
\end{imp}

\begin{theorem}[Лебега о мажоранте]
  Пусть $f_n \bw\convae f$ и интегрируемы. Пусть $\exi \ph(x) \bw\in L(E)$ такая, что $|f_n| \bw\le \ph$.
  Тогда $f \bw\in L(E)$ и $\liml{n}\ints{E}f_n\,d\mu \bw= \ints{E}f\,d\mu$.
\end{theorem}

\begin{note}
  Если $f \bw\in L(E)$, то $\liml{N}\ints{E}[f]^N d\mu \bw= \ints{E}f\,d\mu$.
\end{note}

\begin{theorem}
  Если $f \bw\in L(E)$, то $\liml{N}\ints{E}\hm{f \bw-[f]^N}d\mu \bw=0$.
\end{theorem}

\begin{stm}
  Если функция интегрируема по Риману, то она интегрируема по Лебегу.
\end{stm}
\begin{note}
  Обратное, вообще говоря, неверно.
\end{note}

\begin{theorem}
  Для функций $f\cln [a,b] \bw\ra \R$ имеем $\Mg[a,b] \bw= L[a,b]$.
\end{theorem}

\begin{imp}
  Если $F(x) \bw= (L)\intl{a}{x}f\,d\mu$, то~$F$ дифференцируема почти всюду на~$[a,b]$ и
  \eqn{f(x)=\liml{n\ra\bes}n\bbs{F\hr{x+\frac1n}-F(x)}.}
\end{imp}

\begin{df}
  Введём отношение эквивалентности для интегрируемых функций: $f \bw\sim g$
  тогда и только тогда, когда $f \bw\eqae g$. Факторпространство $L/\!\!\sim$ является линейным
  метрическим пространством, обозначаемым $L^1(X)$.
  Метрика на нём задаётся так: $\rho(f,g) \bw{:=} \ints{X}|f \bw- g|d\mu$.
  Норма в $L^1$ определяется как $\hn{f} \bw{:=} \ints{X} |f|d\mu$.
\end{df}

\begin{lemma}[Неравенство Чебышёва]
  $\mu\bc{x \bw\in X\cln |f(x)|\bw\ge c} \bw\le \frac1c\hn{f}$.
\end{lemma}

\begin{df}
  \emph{Заряд}\т $\si$\д аддитивная функция $\ph\cln\Sig \bw\ra \R$.
\end{df}

\begin{df}
  Заряд~$\ph$ \emph{абсолютно непрерывен} относительно меры~$\mu$,
  если для любого множества~$E$, для которого $\mu(E) \bw= 0$, имеем $\ph(E) \bw= 0$.
\end{df}

\begin{df}
  Заряд $\ph$ \emph{сингулярен} относительно меры~$\mu$, если найдётся множество $\mu$\д меры нуль,
  такое, что для $\fa Y \bw\subs (X \bw\wo Z)$ имеем $\ph(Y) \bw= 0$.
\end{df}

\begin{df}
  \emph{Верхняя, нижняя и полная вариации:}
  \eqn{\ol{V}(E,\ph) := \mskip-20mu \supl{\hc{A \in \Sigma | A \subs E}}\mskip-20mu \ph(A),\quad
       \ul{V}(E,\ph) := \ol{V}(E, -\ph), \quad
       V(E,\ph) := \ol{V}(E,\ph)+\ul{V}(E,\ph).}
\end{df}

\begin{stm}
  Верхняя вариация полуаддитивна: $\ol{V}\hr{\cupl{n=1}\bes E_n} \bw\le \suml{n=1}\bes \ol{V}(E_n)$.
\end{stm}

\begin{stm}
  $V(E) \bw< +\bes$.
\end{stm}

\begin{stm}
  $\ol{V}$, $\ul{V}$ и $V$ являются $\si$\д аддитивными.
\end{stm}

\begin{theorem}[Разложение Жордана]
  Для всякого $E \bw\in \Sig$ имеем $\ph(E) \bw= \ol{V}(E) \bw- \ul{V}(E)$.
\end{theorem}

\begin{theorem}[Разложение Хана]
  Пусть $E \bw\in \Sig$, тогда $\exi P\subs E$ такое, что для $\fa A \bw\subs P$
  имеем $\ph(A) \bw\ge 0$ и $\fa A \bw\subs (X \bw\wo P)$ имеем $\ph(A) \bw\le 0$.
  Эквивалентная формулировка: $\ul{V}(P) \bw= 0$ и $\ol{V}(E \bw\wo P) \bw= 0$.
\end{theorem}

\begin{theorem}[Обобщённое разложение Хана]
  Пусть $\ph \bw\ge 0$, а $\mu(E) \bw< +\bes$, тогда
  для $\fa a \bw\in \R$ существует разложение
  $E \bw= Z \bigsqcup \hr{\cupsql{k=1}\bes E_k}$, где $\mu(Z) \bw= 0$, а для $\fa A \bw\subs E_k$ имеем
  \eqn{(k-1)\cdot a\cdot \mu(A) \le \ph(A) \le k\cdot a\cdot\mu(A).}
\end{theorem}

\begin{theorem}[Разложение Лебега]
  Пусть $\ph$\т заряд, а мера $\mu$\т $\si$\д аддитивна, тогда существуют заряды $\al$ и $\si$ такие,
  что~$\al$ абсолютно непрерывен, а~$\si$\т сингулярен, причём $\al(E) \bw= \ints{E}f\,d\mu$,
  а~$\si(E) \bw= \ph(Z \bw\cap E)$, где $\mu(Z) \bw= 0$.
\end{theorem}

\begin{theorem}[Радона\ч Никодима]
  Каждая $\si$\д аддитивная, абсолютно непрерывная
  относительно некоторой $\si$\д конечной меры~$\mu$ функция
  представима в виде интеграла Лебега: $\al(A) \bw= \ints{A}f\,d\mu$.
  При этом~$f$ определена однозначно с точностью до множества меры нуль,
  а если $\al \bw\ge 0$, то $f \bw\ge 0$ почти всюду.
\end{theorem}

\begin{theorem}
  Если $f\cln [a,b]\bw\ra \R$ и~$f$ возрастает на~$[a,b]$, то почти всюду
  существует $f' \bw\in L[a,b]$.
\end{theorem}

\begin{imp}
  Функция ограниченной вариации дифференцируема почти всюду.
\end{imp}

\begin{df}
  Пусть $f\cln [a,b]\bw\ra \R$. Говорят, что~$f$ \emph{абсолютно непрерывна} на~$[a,b]$ и
  пишут $f \bw\in \AC[a,b]$, если $\fa\ep \bw> 0 \exi\de \bw> 0$ такое, что для любого
  конечного набора отрезков $[a_i,b_i] \bw\subs [a,b]$ с непересекающимися внутренностями
  и суммой длин меньше $\de$ имеем $\sum\hm{f(b_i) \bw- f(a_i)} \bw< \ep$.
\end{df}

\begin{theorem}
  $\AC \bw\subs \VB$.
\end{theorem}

\begin{df}
  Функция~$f$ обладает $\Nb$\д свойством Лузина, если для $\fa E \bw\subs X$ такого, что
  $\mu(E) \bw= 0$, имеем $\mu\br{f(E)} \bw= 0$.
\end{df}

\begin{theorem}
  Если функция $F \bw\in \AC[a,b]$, то~$F$ обладает $\Nb$\д свойством.
\end{theorem}

\begin{note}
  Если $f \bw\in \AC[a,b]$, то~$f$ почти всюду дифференцируема на~$[a,b]$.
\end{note}

\begin{theorem}
  Если $f \bw\in \AC[a,b]$ и $f' \bw\ge 0$ почти всюду на $[a,b]$, то~$f$ возрастает на~$[a,b]$.
\end{theorem}

\begin{theorem}
  Пусть $f \bw\in L[a,b]$, рассмотрим $F(x) \bw{:=} \intl{a}{x} f\,d\mu$, тогда $F \bw\in \AC[a,b]$.
\end{theorem}

\begin{df}
  Пусть $\ph(x)$ возрастает на $\R$, тогда $(LS) \intl{a}{b} f\,d\ph \bw{:=}
  (L)\intl{a}{b} f\,d\mu_\ph$, где $\mu_\ph$\т мера Стилтьеса,
  определяемая функцией~$\ph$.
\end{df}

\begin{df}
  Пусть $\ph(x) \bw\in \VB(\R)$, тогда $\ph \bw= \xi \bw- \psi$,
  где $\xi$ и $\psi$ возрастают на $\R$. Положим
  $(LS)\intl{a}{b}f\,d\ph \bw{:=} (LS) \intl{a}{b}f\,d\xi \bw-(LS)\intl{a}{b}f\,d\psi$.
\end{df}

\begin{df}
  Пусть $\mu_X$ и $\mu_Y$\т меры на~$X$ и~$Y$ соответственно, продолженные с полуколец~$S$ и~$T$
  соответственно. Заметим, что $S\bw\times T$\т полукольцо. Зададим на нём
  меру: $\mu(A\bw\times B) \bw{:=} \mu_X(A)\bw\cdot \mu_Y(B)$.
\end{df}

\begin{theorem}
  Если меры $\si$\д аддитивны на~$S$ и~$T$, то~$\mu$ тоже $\si$\д аддитивна.
\end{theorem}

\begin{theorem}[Фубини]
  Пусть $A \bw\subs X \bw\times Y$ и $f\cln A \bw\ra \R$, $f \bw\in L(A)$. Положим
  \eqn{A_X := \hc{y\cln \exi x\in X\cln (x,y)\in A}, \quad
       A_Y := \hc{x\cln \exi y\in Y\cln (x,y)\in A}.}
  Тогда
  \eqn{\ints{A}f(x,y)\,d\mu =\ints{X}\ints{A_X}f(x,y)\,d\mu_Y\,d\mu_X =
       \ints{Y}\ints{A_Y}f(x,y)\,d\mu_X\,d\mu_Y.}
\end{theorem}

\begin{df}
  Для любого $p\bw\ge1$ определим $L^p(X)$\т нормированное
  пространство классов эквивалентности функций $f\cln X\bw\ra \R$ таких, что $|f|^p \bw\in L(X)$,
  с нормой \eqn{\hn{f}_{L^p} :=\bbr{\ints{X}|f|^p d\mu}^{1/p}.}
  Отношение эквивалентности\т совпадение почти всюду.
\end{df}

\begin{note}
  $L^2$\т гильбертово пространство.
\end{note}

\begin{theorem}[Неравенство Гёльдера]
  Имеем $\hn{f g}_{L^1} \bw\le \hn{f}_{L^p} \hn{g}_{L^q}$ при $\frac1p \bw+ \frac1q \bw=1$.
\end{theorem}

\begin{theorem}[Неравенство Минковского]
  Имеем $\hn{f+g}_{L^p} \bw\le \hn{f}_{L^p} \bw+ \hn{g}_{L^p}$.
\end{theorem}

\begin{theorem}[Неравенство Юнга]
  Имеем $ab \bw \le\frac{a^p}p \bw+ \frac{b^q}q$ при $\frac1p \bw+ \frac1q \bw=1$.
\end{theorem}

\begin{stm}
  $L^p$\т полное (и потому банахово) пространство.
\end{stm}

\begin{theorem}[Рисса\ч Фишера]
  Пусть $\hc{\ph_n}$\т произвольная ортогональная нормированная система в гильбертовом пространстве~$\Hc$,
  а $\hc{c_k}_{k=1}^\bes$\т такой набор чисел, что $\sum c_k^2$ сходится.
  Тогда $\exi f\bw\in \Hc$, что $c_k \bw= (f,\ph_k)$ и $\sum c_k^2\bw=(f,f) \bw= \hn{f}^2$.
\end{theorem}

\pagebreak

\section{Задачи и их решения}

\begin{problem}
  Замкнутость относительно каких операций $\cup, \swo, \cap, \wo$ дает эквивалентное определение кольца?
\end{problem}
\begin{solution}
  \begin{items}{-3}
     \item Относительно $\swo, \cup$, так как $A \cap B = A \mathbin{\swo} B \swo (A\cup B)$.
     \item Относительно $\swo, \wo$, так как $A\cap B = A \wo (A \swo B)$.
     \item Относительно~$\cup, \wo$, так~как~$A\cap B = A \wo (A \wo B), A\swo B = (A\cup B) \wo (A\cap B)$.
  \end{items}

  Заметим, что никакой другой пары операций из~$\cup, \cap, \wo$ недостаточно. Действительно,
  при помощи операций~$\cap$ и~$\cup$ из~множеств~$A$  и~$B$ невозможно получить никакого
  множества, не~содержащего~$A\cap B$. При помощи операций~$\cap$ и $\wo$ из~множеств~$A$ и~$B$
  можно получить только множества~$A\cap B$, $A\wo B$, $B\wo A$, $\es$.
\end{solution}

\begin{problem}
  Проверить, что кольцо является полукольцом.
\end{problem}
\begin{solution}
  Если~множества~$A$ и~$B$ лежат в~кольце, то~$B=A\sqcup (B\wo A) =
  A\sqcup \br{B\swo (A\cap B)}$\Т разложение~$B$ в~сумму попарно
  непересекающихся множеств, одно из~которых равно~$A$.
\end{solution}

\begin{problem}
  Доказать, что если для любого $\al \bw \in I$ множество $\Rc_\al$ является кольцом, то
  $\caps{\al \bw\in I} \Rc_\al$\т кольцо.
\end{problem}
\begin{solution}
  Пусть~$X\in\Rc$ и~$Y\in\Rc$. Тогда для $\fa \al \in I$ выполнено $X \in\Rc_\al$ и~$Y\in\Rc_\al$.
  Но тогда для $\fa \al\in I$ выполнено $X\cap Y\in\Rc_\al$ и~$X\swo Y\in\Rc_\al$, а~значит
  $X\cap Y\in\Rc$ и~$X\swo Y\in \Rc$. Таким образом, $\Rc$\Т кольцо.
\end{solution}

\begin{problem}
  Доказать, что на полукольце прямоугольников площадь является $\si$\д аддитивной мерой.
\end{problem}
\begin{solution}
  Пусть~$\hc{A_n}$ и $A$\т прямоугольники, $A = \cupsqs{n} A_n$. Докажем, что~$S(A) =
  \sums{n} S(A_n)$. Сначала докажем, что~$S(A)\le \sums{n} S(A_n)$. Для каждого
  прямоугольника~$A_n$ фиксируем открытый прямоугольник~$A_n'$, площадь которого
  не~превосходит~$\frac{\ep}{2^n}$. Кроме того, фиксируем замкнутый
  прямоугольник~$A'\subs A$, площадь которого не~меньше~$S(A)-\ep$.
  Из~покрытия компакта~$A'$ открытыми множествами~$A_n'$ выделим конечное
  подпокрытие $\hc{A_{n_k}'}$. В силу конечной аддитивности площади, получим,
  что~$\sums{k} S(A_{n_k}')\ge S(A')$. Но~тогда
  \eqn{\sums{n} S(A_n)\ge \sums{n} S(A_n') - \ep\ge \sums{k} S(A_{n_k}')-\ep\ge S(A')-2\ep\ge S(A)-2\ep.}
  Так как это   выполнено для всех~$\ep>0$, то~$\sums{n} S(A_n)\ge S(A)$. Докажем теперь,
  что~$\suml{n=1}{\bes} S(A_n) \bw\le S(A)$. Каждое неравенство~$\suml{n=1}{N}
  S(A_n)\le S(A)$ следует из~аддитивности площади, а~доказываемое получается
  из~них предельным переходом.
\end{solution}

\begin{problem}
  Привести пример множества $A \bw\in \Bs(\R^n)$, не являющегося:
  \begin{items}{-3}
     \item $\Fc_\si$\д множеством;
     \item $\Gc_\de$\д множеством.
  \end{items}
\end{problem}
\begin{solution}
  Заметим, что дополнение к $\Gc_\de$ есть $\Fc_\si$, поэтому достаточно построить плохое множество только
  одного типа. Докажем, что множество $A \bw{:=} \Q \cap [0,1] \notin \Gc_\de$. Допустим, что
  $A \bw= \capl{i\bw=1}{\bes} U_i$, где $U_i$ открыты. Заметим, что $A$ всюду плотно
  в $[0,1]$, поэтому каждое из $U_i$ должно быть всюду плотным. Далее, заметим, что $A$ счётно.


\begin{df}
  Множество называется \emph{остаточным}, если оно является пересечением открытых всюду плотных множеств.
\end{df}

\begin{lemma}
  \label{RestSetsLemma}Остаточные множества в $\R^n$ имеют мощность континуум.
\end{lemma}
\begin{proof}
  В самом деле, пусть $E$\т остаточное множество, и $E \bw= \bigcap \hc{U_i}$. Выберем произвольную
  точку $x_1$ множества $U_1$. В силу открытости $U_1$ она содержится в нём с некоторой своей
  окрестностью, а значит, и вместе с некоторым отрезком $I_1$. В этом интервале можно найти
  некоторую точку $x_2$ из множества $U_2$, так как оно всюду плотно, и точно также можно
  выделить отрезок $I_2 \bw\subs I_1$, содержащий точку из $U_1 \bw\cap U_2$, \итд Таким
  образом можно построить последовательность вложенных отрезков. У них есть хоть одна общая
  точка по лемме о вложенных отрезках. Значит, $E \neq \es$. Эту же конструкцию можно применить
  к построению континуального подмножества в~$E$. А именно, на $k$\д м шаге можно выбрать $2^k$
  отрезков, содержащихся в пересечении $U_1 \sco U_k$. Для этого достаточно брать по одному
  отрезку в левой и правой окрестности точки, далее в каждом из них находить по точке
  из~$U_{k+1}$, \итд Будем кодировать каждый левый отрезок нулём, а каждый правый\т единицей.
  Тогда получим множество точек, принадлежащих всем $U_i$, закодированных \emph{всеми}
  последовательностями из нулей и единиц, а их континуум, что и требовалось.
\end{proof}

  Возвращаясь к задаче и применяя лемму \ref{RestSetsLemma}, заключаем, что $A$ является
  остаточным и в то же время счётным. Противоречие. Значит, $A \notin \Gc_\de$. Отсюда
  следует, что $\br{[0,1]\bw\wo A} \notin \Fc_\si$.
\end{solution}

\begin{problem}
  Определить борелевский класс множества точек непрерывности функции на отрезке.
\end{problem}

\begin{problem}
  Привести пример $A \bw\subs \R^2$, измеримого по Лебегу, но не измеримого по Жордану.
\end{problem}
\begin{solution}
  Годится множество $A\bw=\br{\Q \cap [0,1]}^2$. Очевидно, что минимальное простейшее множество,
  содержащее~$A$, есть квадрат $[0,1]^2$, а внутрь множества $A$ не уместится ничего, кроме
  конечного набора точек, ибо $\Q$ всюду плотно на $[0,1]$. Значит, внешняя мера Жордана не
  совпадает с внутренней, что свидетельствует о неизмеримости $A$.
\end{solution}

\begin{problem}
  Доказать, что если множество $E \bw\subs \R^n$ компактно и $\mu_L(E)\bw=0$, то оно измеримо по
  Жордану и $\mu_J(E)\bw=0$.
\end{problem}

\begin{problem}
  Пусть внешняя мера Лебега $\mu^*$ является продолжением меры, заданной на некотором кольце.
  Доказать, что множество $A$ $\mu^*$\д измеримо по Каратеодори тогда и только тогда, когда
  для $\fa \ep \bw> 0$ найдется элемент кольца $B_\ep$ такой, что $\mu^*(A \swo B_\ep) \bw< \ep$.
\end{problem}

\begin{problem}
  Пусть $S$\т полукольцо множеств на $X$, $\mu$\т $\si$\д аддитивная мера на $S$,
  $X\bw=\cupsql{i\bw=1}{\bes} X_i$, а $X_i \bw\in S$. По определению, множество $A \bw\subs X$ измеримо, если
  для $\fa i$ множество $A \cap X_i$ измеримо, и при этом $\mu(A) \bw= \suml{i\bw=1}{\bes}\mu(A\cap
  X_i)$. Доказать, что так определенная мера не зависит от разбиения $X$ на $X_i \bw\in S$, является
  $\si$\д аддитивной, и что измеримые множества образуют $\si$\д алгебру, а множества конечной меры\т
  $\si$\д кольцо.
\end{problem}

\begin{problem}\label{unmes.set.in.set.with.positive.measure}
  Построить в множестве положительной меры Лебега неизмеримое по Лебегу подмножество.
\end{problem}
\begin{solution}
  Мы изложим нашу конструкцию в случае, когда $X = [0,1]$.
  Пусть $E \bw\subs X$\т исходное множество. Введём отношение эквивалентности на~$E$:
  скажем, что $x \bw\sim y$, если $x \bw- y \bw\in \Q$. Выберем ровно по одному элементу из
  каждого класса эквивалентности и образуем из этих элементов множество $F$. Покажем, что~$F$
  неизмеримо по Лебегу. Через $A + q$ обозначим множество $\hc{x \bw+ q \vl x \bw\in A}$.
  Легко видеть, что для всяких $p,q \in \Q$ множества $F + p$ и $F + q$ не пересекаются.
  Действительно, если бы они пересеклись, то это означало бы, что в $F$ найдутся две точки,
  расстояние между которыми рационально, что противоречит его определению. Теперь рассмотрим
  множество $U := \cupsqs{q \in \Q \cap X}(F+q)$. Заметим, что $\mu(F + q) = \mu(F)$ для
  всякого $q\in \Q$. Если $\mu(F) > 0$, то, в силу счётной аддитивности меры, получим
  $\mu(U) = \sums{q \in \Q\cap X} \mu (F+q) = \sum \mu(F) = \bes$, что противоречит условию.
  С другой стороны, мера $F$ не может быть нулевой, поскольку тогда $\mu(U) = 0$, но тогда
  и $\mu(E) = 0$, а это невозможно по условию.
\end{solution}

\begin{problem}
  Построить на отрезке $[0,1]$ нигде не плотное множество наперёд заданной меры~$p$.
\end{problem}

\begin{problem}
  Пусть $\al \bw> 0$. Найти меру совершенного множества канторовского типа, в котором отношение
  длины выбрасываемого на $n$\д м шаге интервала к длине отрезка, из которого выбрасывается интервал,
  равно~$\frac{1}{(\al + n)^2}$.
\end{problem}

\begin{problem}
  Придумать множество меры нуль, не представимое в виде счетного объединения нигде не
  плотных множеств.
\end{problem}
\begin{solution}
  Рассмотрим отрезок $[0,1]$ и счётное объединение множеств положительной меры канторовского
  типа на нём, \те $E\bw{:=}\bigcup K_n$, где $K_n$\т канторово множество меры $1-\frac{1}{n}$.
  Заметим, что каждое $K_n$ нигде не плотно. Очевидно, что множество $G \bw{:=} [0,1] \bw\wo E$
  имеет меру нуль. Покажем, что оно обладает требуемым свойством. Действительно, допустим, что
  оно представимо в виде объединения нигде не плотных множеств $F_n$. Тогда
  $[0,1]\bw=\hr{\bigcup F_n} \cup \hr{\bigcup K_n}$. Но отрезок не представляется в виде
  счётного объединения нигде не плотных множеств, так как по аналогии с леммой о вложенных
  отрезках можно построить точку, которая не лежит в этом объединении (вспомните
  определение нигде не плотного множества).
\end{solution}

\begin{problem}
  Доказать, что в любом множестве положительной меры найдутся две точки, расстояние между
  которыми рационально.
\end{problem}
\begin{solution}
  Если мы живём на отрезке, то $E$ ограничено. Если же мы живём на прямой, то, в
  силу $\si$\д конечности меры Лебега на $\R$, можно выделить в $E$ ограниченное
  подмножество положительной меры и все рассуждения проводить для этого ограниченного
  подмножества. Докажем, что уже в этом подмножестве искомые точки существуют.

  Допустим противное, \те существует ограниченное множество $E \bw\subs \R$ положительной меры,
  у которого все расстояния между его точками иррациональны. Обозначим через $E + q$ сдвиг
  множества $E$ в точку $q \in \R$, иначе говоря, $E + q = \hc{x + q \vl x \in E}$. Тогда
  рациональные сдвиги $E$ не пересекаются: если бы случилось так, что $x + p = y + q$,
  где $x, y \in E$, а $p,q \bw\in \Q$, то $x-y = q-p \bw\in \Q$, что противоречит тому,
  что все расстояния между точками $E$ иррациональны. Рассмотрим $F = \cups{r_k} (E + r_k)$,
  где $\hc{r_k}$\т нумерация всех рациональных чисел отрезка $[0,1]$. С одной стороны,
  мера $F$ конечна, поскольку $F$ ограничено. С другой стороны, в силу $\si$\д аддитивности
  меры, имеем $\mu(F) = \sums{k} \mu(E) = \bes$. Противоречие.
\end{solution}

\begin{problem}
  Доказать, что $E \bw\subs \R^n$ измеримо по Лебегу тогда и только тогда, когда для
  $\fa \ep \bw> 0$ существуют замкнутое множество $F_\ep$ и открытое множество~$G_\ep$, такие что
  $F_\ep \bw\subs E \bw\subs G_\ep$ и $\mu(G_\ep)-\mu(F_\ep) \bw< \ep$.
\end{problem}

\begin{problem}
  Доказать эквивалентность определения измеримости функции через измеримость множеств любого из типов:
  $\hc{f(x) \bw\le C}$, $\hc{f(x) \bw< C}$, $\hc{f(x) \ge C}$, $\hc{f(x) \bw> C}$.
\end{problem}

\begin{problem}\label{unmes.f.with.mes.coim.of.every.point}
  Придумать неизмеримую функцию $f$, у которой для $\fa c \bw\in \R$ измеримо
  множество~$f^{-1}\br{\hc{c}}$.
\end{problem}
\begin{solution}
  Пусть $X \bw= [0,1]$. Рассмотрим неизмеримое по Лебегу множество $A \bw\subs X$, и построим
  функцию
  \eqn{f(x) \bw= \case{\phantom{-}x, & x \bw\in A;\\-x, & x \notin A.}}
  Тогда прообразом любой точки $c \bw\in \R$ будет либо $\es$, либо одна точка. Однако функция
  не является измеримой, поскольку $f^{-1}\br{[0,1]} \bw= A$.
\end{solution}

\begin{problem}
  Доказать, что функция $f$ измерима тогда и только тогда, когда для любого
  множества $E \bw\in \Bs(\R)$ его прообраз $f^{-1}(E)$ измерим.
\end{problem}

\begin{problem}
  Доказать измеримость $f'$, если измеримая функция $f$ дифференцируема почти всюду.
\end{problem}

\begin{problem}
  Будет ли измерима функция $g\br{f(x)}$, если:
  \begin{items}{-3}
    \item $g$ измерима, а $f$ непрерывна;
    \item $g$ непрерывна, а $f$ измерима?
  \end{items}
\end{problem}
\begin{solution}

  $\bullet$ Пусть $K(x)$\т лестница Кантора на $[0,1]$, тогда $\ph(x) \bw{:=} x + K(x)$
  непрерывна и монотонна. Пусть $f$\т функция, обратная к $\ph$, тогда $f\cln [0,2] \ra [0,1]$.
  Пусть $\Kc$\т множество Кантора, тогда легко видеть, что $\mu\br{\ph(\Kc)} \bw= 1$. Найдём
  в $\ph(\Kc)$ неизмеримое по Лебегу множество~$A$, тогда $B \bw{:=} f(A) \subs \Kc$ и потому
  имеет меру нуль. Рассмотрим $g \bw{:=} \chi_B$, тогда $g$ будет измерима, но композиция~$g\circ f$
  не будет измеримой, \тк $(g\circ f)^{-1}\br{\hc{1}} \bw= A$.

  $\bullet$ Одно из эквивалентных определений непрерывной функции\т прообраз открытого открыт.
  Следовательно, для $\fa c \bw\in \R$ при $C \bw= (-\bes, c)$ имеем
  $B \bw{:=} g^{-1}(C) \bw\in \Bs$, а в силу измеримости $f$, имеем $f^{-1}(B)$ измеримо.
  Поэтому композиция~$g\circ f$ измерима.
\end{solution}

\begin{problem}
  Пусть $f$\т простая функция, принимающая значения $a_k$ на множествах $E_k, \; k\bw=1\sco n$, и
  $E_i \cap E_j \bw= \es$. Показать, что она измерима $\Lra$ все $E_i$ измеримы.
\end{problem}

\begin{problem}
  Пусть $\mu(X) \bw= +\bes$. Построить последовательность измеримых функций, сходящуюся почти всюду,
  но не сходящуюся по мере.
\end{problem}
\begin{solution}
  Пусть $X \bw= \R$ с мерой Лебега. Рассмотрим $f_n \bw= \frac xn$. Ясно, что для $\fa x \bw\in \R$
  имеем $f_n(x) \ra 0$, но $\mu\hc{x\cln |f_n(x)| \bw> c} \bw= +\bes$ для $\fa c \bw> 0$ и всех $n$.
\end{solution}

\begin{problem}[Пример Рисса]
  Построить последовательность функций, сходящуюся по мере, но не сходящуюся почти всюду.
\end{problem}
\begin{solution}
  Рассмотрим отрезок $[0,1]$. Чтобы упростить обозначения, через $I[a,b]$ будем обозначать
  индикатор отрезка $[a,b]$. Мы будем строить искомую последовательность <<волнами>>, по~$n$
  функций на $n$\д ю волну. Первая волна состоит из одной функции $f_{11} := I[0,1]$. Вторая
  волна содержит две функции $f_{21} := I\hs{0,\frac12}$ и $f_{22} := I\hs{\frac12,1}$.
  Третья волна состоит из трёх функций $f_{31} := I\hs{0,\frac13}$, $f_{23} := I\hs{\frac13,\frac23}$
  и $f_{33} := I\hs{\frac23,1}$. Соответственно, $n$\д ая волна будет состоять из $n$ функций
  вида $I\hs{\frac in,\frac{i+1}n}$. Заметим, что для функций из $k$\д й волны имеем
  $\mu\hc{x\cln f_{ki} \neq 0} = \frac1k$. Расположим все функции в одну последовательность
  $\hc{f_k}$ следующим образом:
  \eqn{\ub{f_{11}}, \ub{f_{21}, f_{22}}, \ub{f_{31}, f_{32}, f_{33}},
       \ub{f_{41}, f_{42}, f_{43}, f_{44}}\ldots}
  Из сказанного выше следует, что $f_k \xra{\mu} 0$, но ясно, что эта последовательность нигде не сходится
  на~$[0,1]$.
\end{solution}

\begin{problem}[Теорема Рисса]
  Доказать, что последовательность измеримых функций, сходящаяся по мере, содержит
  подпоследовательность, сходящуюся почти всюду.
\end{problem}
\begin{solution}
  Откройте хороший учебник по теории меры и прочтите там доказательство.
\end{solution}

\begin{problem}
  Показать, что теорема Егорова неверна, если $\mu(X) \bw= +\bes$.
\end{problem}
\begin{solution}
  Пусть $X \bw= \R$ с мерой Лебега. Рассмотрим $f_n \bw= \frac xn$. Ясно, что
  если выкинуть из $\R$ множество конечной меры, оставшееся множество будет неограниченным,
  а для такой последовательности функций равномерная сходимость имеет место только
  на ограниченных подмножествах~$\R$.
\end{solution}

\begin{problem}
  Привести пример неизмеримой функции $f$, такой, что $|f|$ измерима.
\end{problem}
\begin{solution}
  Возьмём функцию из задачи \ref{unmes.f.with.mes.coim.of.every.point}. Ясно, что
  $|f| \bw= x$, а эта функция измерима.
\end{solution}

\begin{problem}
  Пусть $f \in L(E)$. Положим
  $E_c \bw{:=} \hc{x \bw\in E\cln |f(x)| \bw> c}$.
  Доказать, что $\mu(E_c) \bw= o(1/c)$.
\end{problem}

\begin{problem}
  Придумать измеримое множество $E \bw= \cupsql{k\bw=1}{\bes} E_k$, где $E_k$ измеримы, и
  не интегрируемую на $E$ функцию $f\cln E \ra \R$, такую что $f$ интегрируема на всех $E_k$, и ряд
  $\sumkui \ints{E_k} f\,d\mu$ абсолютно сходится.
\end{problem}
\begin{solution}
  Возьмём пример из задачи \ref{notint.f.with.zero.cuts.limit}, и положим
  $E_k \bw= \hsr{-\frac1k, -\frac1{k+1}} \cup \hrs{\frac1{k+1}, \frac1k}$. Точку $0$ отдельно
  накроем множеством~$E_0 \bw{:=} \hc{0}$. Очевидно, что $\ints{E_k} f\,d\mu \bw= 0$, поэтому
  ряд сходится.
\end{solution}

\begin{problem}
  Доказать, что суммируемость счётно\д простой функции, принимающей значения $a_k$ на $E_k$,
  где~$E \bw= \cupsql{k\bw=1}{\bes} E_k$, эквивалентна сходимости ряда $\sumkui |a_k| \mu(E_k)$.
\end{problem}

\begin{problem}
  Функция $f$ измерима на $E$, и $\mu(E) \bw< \bes$. Доказать, что сходимость любого из рядов
  \eqn{\sumkui k \cdot \mu \hc{x \bw\in E\cln k \bw\le \hm{f(x)} \bw< k+1},
       \qquad\sumkui \mu \hc{x \bw\in E\cln \hm{f(x)} \ge k}}
  является необходимым и достаточным условием интегрируемости $f$ на $E$.
\end{problem}

\begin{problem}\label{notint.f.with.zero.cuts.limit}
  Следует ли из существования конечного предела $\liml{N \ra \bes} \ints{E}[f]^N\,d\mu$
  суммируемость функции $f$?
\end{problem}
\begin{solution}
  Нет, не следует. Рассмотрим следующий пример: $E \bw= [-1,1]$, а
  \eqn{f(x) \bw= \case{\frac{1}{x}, & x \bw\in [-1,1]\bw\wo \hc{0};\\
       0, & x \bw= 0.}}
  Тогда для $\fa N \bw\in \N$ имеем $\ints{E} [f]^N\, d\mu \bw= 0$ в силу нечётности функции,
  но эта функция не интегрируема.
\end{solution}

\begin{problem}
  Показать, что в теореме Беппо Леви условие $f_1(x) \bw\in L(E)$ нельзя выбросить.
\end{problem}

\begin{problem}
  Пусть $\Q \bw= \hc{r_k}$. Доказать, что ряд $\sumkui \frac{1}{k^2\sqrt{|x-r_k|}}$ сходится почти всюду.
\end{problem}
\begin{solution}
  Применить теорему Беппо Леви к последовательности частных сумм ряда.
\end{solution}

\begin{problem}
  Привести пример, когда в заключении теоремы Фату имеет место строгое неравенство.
\end{problem}

\begin{problem}
  Доказать, что
  \eqn{\liml{n \ra \bes} \intl{0}{+\bes} \frac{1}{\hr{1+\frac{x}{n}}^n x^{1/n}}dx=1.}
\end{problem}

\begin{problem}
  Доказать, что для неотрицательных функций  из интегрируемости по Риману в несобственном смысле
  следует суммируемость.
\end{problem}

\begin{problem}
  Пусть функция $f$ интегрируема по Лебегу на измеримом множестве $E$. Обозначим $E_k\bw=\hc{x\in E\vl k<|f(x)|}$.
  Доказать, что $\mu(E_k)=o\Br{\frac1k}$ при $k\ra\infty$.
\end{problem}

\begin{problem}
  Придумать измеримую не интегрируемую функцию, для которой выполнено свойство
  \eqn{\mu \hc{|f(x)|\bw>k} \bw= o\Br{\frac1k}, \; k \ra \bes.}
\end{problem}

\begin{problem}
  Придумать функцию, такую, что любая функция, совпадающая с ней почти всюду,
  не интегрируема по Риману.
\end{problem}

\begin{problem}
  Доказать, что если аддитивная функция множества является одновременно абсолютно не\-прерывной
  и~сингулярной, то она тождественно равна нулю.
\end{problem}

\begin{problem}
 Доказать, что аддитивная функция множества абсолютно непрерывна (сингулярна) тогда и~только
 тогда, когда абсолютно непрерывны (сингулярны) её верхняя и нижняя вариации.
\end{problem}

\begin{problem}
  Пусть $\ph$ является $\si$\д аддитивной функцией на $\si$\д алгебре $\Sig$, а $\hc{E_k}$\т
  монотонная последовательность элементов~$\Sig$ (\те $E_k \nearrow E$ или $E_k \searrow E$).
  Доказать, что тогда $\ph(E)  \bw= \liml{k \ra \bes} \ph(E_k)$.
\end{problem}

\begin{problem}
  Пусть $\ph$\т неотрицательная аддитивная функция на $\si$\д алгебре $\Sig$, а $\hc{E_k}$\т
  произвольная последовательность элементов~$\Sig$. Доказать, что
  $\ph(\liminf E_k) \bw\le \liminf \ph(E_k) \bw\le \limsup \ph(E_k) \bw\le \ph(\limsup E_k)$.
\end{problem}

\begin{problem}
  Доказать, что канторова лестница не является $\AC$-функцией.
\end{problem}

\begin{problem}
  Доказать линейность классов $\AC$ и $\VB$.
\end{problem}

\begin{problem}
  Доказать, что для любой функции на прямой множество точек $\hc{x\vl D_-F(x) < D^+F(x)}$
  не~более\footnote{Смысл значков $D^+$ и $D_-$, если обозначения с 2003 года не поменялись, таков: $D_-$\т нижняя производная
  слева, $D^+$\т верхняя производная справа (есть ещё, соответственно, $D^-$ и $D_+$)\т \emph{Прим.~ред.}} чем счётно.
\end{problem}

\begin{problem}
  При каких $\al$ и  $\be$ функции $x^\al\sin x^\be, \; x^\al \cos x^\be$ принадлежат
  классам $\AC$ и $\VB$ на $[0,1]$?
\end{problem}

\begin{problem}
  Привести пример функции из класса $\mathbf{ACG}$, которая не принадлежит классу $\AC$.
\end{problem}

\begin{problem}
  Вычислить интеграл Лебега\ч Стилтьеса для функции скачков.
\end{problem}

\begin{problem}
  Доказать для случая $\ph \bw\in \AC$ равенство $(LS)\intl{a}{b}f\,d\ph \bw= (L)\intl{a}{b}f\ph'\,dx$.
\end{problem}

\begin{problem}
  Пусть $f, g\bw\in L^1(\R)$. Доказать, что для почти всех~$y$ существует свёртка
  \eqn{(f * g)(y) := \ints{\R}f(x)g(y-x)\,dx.}
\end{problem}

\begin{problem}
  Придумать последовательность интегрируемых по Лебегу функций~$f_n$, таких что~$f_n\xra{\mu} f$,
  но~$\hn{f_n - f}_{L^1} \nra 0$.
\end{problem}
\begin{solution}
  Рассмотрим $f_n \bw= n\cdot \chi_{\hs{0,\frac1n}}$ на $[0,1]$. Очевидно, что $f_n\map{\mu} 0$,
  но $\hn{f_n}_{L^1} \bw= 1 \nra 0$.
\end{solution}

\begin{problem}
  Показать несовпадение пространств $L^p(E)$ при различных $p$ для $E\bw=[0,1]$ и $E \bw= [1, +\bes)$.
\end{problem}

\begin{problem}
  Выяснить связь между сходимостью в $L^p(E)$, сходимостью почти всюду и по мере.
\end{problem}

\begin{problem}
  Доказать неравенство Чебышёва для $L^p(E)$:
  \eqn{\mu\hc{x\cln |f(x)| \ge \al} \bw\le \frac{1}{\al^p}\ints{E} |f(x)|^p \, d\mu.}
\end{problem}
\begin{solution}
  Пусть $L \bw= \hc{x\cln \hm{f(x)} \bw< \al}$, а $G \bw= E \bw\wo L$. Тогда
  \eqn{\ints{E} |f|^p\,d\mu
       \bw= \ints{L} |f|^p\, d\mu + \ints{G} |f|^p\, d\mu \ge \ints{G} |f|^p\,d\mu.}
  Последний интеграл оценим снизу числом $\al^p \mu(G)$, ибо $|f|^p \bw\ge \al^p$ на $G$.
  Значит, $\ints{E} \hm{f}^p\,d\mu \bw\ge \al^p \mu(G)$, что и требовалось доказать.
\end{solution}

\begin{problem}
  Доказать, что если $\hn{f}_{L^p} \bw\le C \bw< \bes$ при всех $p \ge 1$, то~$f$
  эквивалентна ограниченной функции.
\end{problem}
\begin{solution}
  Воспользуемся неравенством Чебышёва:
  \eqn{\mu\hc{x\cln |f(x)| \ge \al} \le \frac{1}{\al^p}\ints{E} |f(x)|^p\,d\mu =
       \frac{\hn{f}_{L^p}^p}{\al^p} \le \frac{C^p}{\al^p}.}
  Заметим, что при~$\al > C$ последняя дробь стремится к~нулю при~$p\ra\bes$. Значит,
  для $\fa \al > C$ имеем $\mu\hc{x\cln |f(x)| > \al} = 0$,
  \те функция~$f$ эквивалентна ограниченной функции~$[f]^\al$.
\end{solution}

\begin{problem}
  При каких $\al$ ряд $\sum \frac{\cos nx}{n^{\al}}$ будет рядом Фурье функции из $L^2$?
\end{problem}

\begin{problem}
  Будет ли $\sum \frac{\cos nx}{\ln n}$ рядом Фурье непрерывной функции?
\end{problem}

\medskip\dmvntrail
\end{document}
