\documentclass[10pt]{article}
\usepackage[utf]{dmvn}
\usepackage[matrix,arrow,curve]{xy}
\usepackage[dvips]{graphicx}


%\textwidth=17.0cm \oddsidemargin=-0.04cm \topmargin=-2.0cm
%\textheight=24.0cm

\newcommand{\ve}{\varepsilon}
\newcommand{\mr}{\mathrm}

\righthyphenmin=2

\CompileMatrices

\begin{document}

\renewcommand{\contentsname}{Оглавление}

\begin{titlepage}

\begin{center}

\vskip40pt

{\Large МОСКОВСКИЙ ГОСУДАРСТВЕННЫЙ УНИВЕРСИТЕТ

\medskip

имени М.В.\,ЛОМОНОСОВА

\bigskip

Механико-математический факультет}

\par\vskip30pt

\epsfbox{mmlogo.2}\\ %

\vskip30pt

{\huge\textbf{\textsf{Курс лекций по}}}

\medskip

{\huge\textbf{\textsf{действительному анализу}}}

\bigskip

{\large\textsf{Лектор --- Владимир Игоревич Богачёв}}

\vskip50pt

{\Large II курс, 4 семестр, отделение математики}

\vfill

{\large Москва 2008}

\end{center}

\end{titlepage}


\section*{От наборщика}

Этот документ представляет собой записки лекций, читавшихся весной
2004-го года, и основывается на конспекте (отсканированные лекции в
формате \texttt{djvu}), доступном на сайте
\texttt{http://dmvn.mexmat.net} с 2005-го года. По сравнению с
упомянутым конспектом в данном варианте исправлены неточности и
дописаны некоторые доказательства.

Весь текст прочитан и одобрен лектором.

О замечаниях, предложениях, а также найденных неточностях или
опечатках можете писать на адрес \texttt{suselr@yandex.ru}.

Данный документ набран с использованием стилевого пакета
\texttt{dmvn}.

\par{\rightline{\emph{Роман Авдеев}}}

\section*{От (в)редакции}

Были внесены косметические правки в исходные тексты. Сообщения об ошибках
и опечатках мы с радостью передадим автору для исправления! В заключение добавим,
что djvu-версия скорее всего будет убрана с сайта в пользу этой, более качественной версии.

\dmvntrail

\newpage \tableofcontents

\section{Введение}

Сия теория создана А. Лебегом. Суть: новая теория меры и интеграла,
призванная расширить как класс измеримых множеств, так и класс
интегрируемых функций.

Исходными объектами теории меры являются \emph{элементарные
множества}: на прямой $\R^1$ это конечные объединения промежутков
вида $[\al,\be]$, $(\al,\be)$, $(\al,\be]$, $[\al,\be)$; на
плоскости $\R^2$ и в $\R^n$--- произведения таковых. На элементарных
множествах естественным образом задаётся мера: в $\R^1$ это длина
промежутка, в $\R^2$ и в $\R^n$~--- соответственно площадь и
$n$-мерный объём. Далее нужно продолжить меру на более широкий класс
множеств так, чтобы выполнялось свойство аддитивности меры: если
$A\cap B=\varnothing$, то мера множества $A\cup B$ равняется сумме
мер множеств $A$ и $B$. Значительных результатов в этом направлении
достиг Жордан.

\begin{df}
Фигура $E$ \emph{измерима по Жордану}, если для любого $\ve>0$
существуют элементарные множества $A,B$, такие что $A\subset
E\subset B$ и $\lambda(B\backslash A)<\ve$. Здесь $\lambda$ ---
мера.
\end{df}

У этого определения есть один недостаток: класс измеримых по Жордану
множеств не замкнут относительно счётного объединения. В частности,
множество $\Q\cap[0,1]$ неизмеримо по Жордану. Поэтому потребовалось
уточнить понятие измеримости множества, и это было сделано Лебегом.
Для каждого множества $E\subset[0,1]^n$ определяется его внешняя
мера: $\lambda^*(E)\bw =\inf\{\sum\limits_{k=1}^\infty \lambda(E_k):
E_k\text{~--- элементарные и\,}E\subset\bigcup\limits_{k=1}^\infty
E_k\}$. Эта функция не является мерой на классе всех подмножеств
куба $[0,1]^n$, так как она, вообще говоря, неаддитивна. Тогда сужая
класс рассматриваемых множеств (с внешней мерой), мы приходим к
понятию измеримости по Лебегу.

\begin{df}
Множество $E$ \emph{измеримо по Лебегу}, если для любого $\ve>0$
найдётся элементарное множество $E_\ve$, такое что
$\lambda^*(E\triangle E_\ve)<\ve$, где $E\triangle E_\ve=(E\cup
E_\ve)\backslash (E\cap E_\ve)$.
\end{df}

Класс измеримых по Лебегу множеств значительно шире класса множеств,
измеримых по Жордану, и замкнут относительно счётного объединения.

Используя созданную им теорию меры, Лебег придумал совершенно новую
конструкцию интеграла, отличную от конструкции Римана. Оказывается,
что все интегрируемые по Риману функции интегрируемы и по Лебегу, но
при этом класс интегрируемых по Лебегу функций намного шире.

Но это было введение. Теперь мы приступаем к систематическому
изложению теории, содержащей анонсированные и многие другие
результаты.



\section{Основные понятия теории меры}
\subsection{Алгебры и $\sigma$-алгебры множеств}

\begin{df}
Пусть $X$ --- основное пространство. Класс $\As$ подмножеств
множества $X$ называется \emph{алгеброй}, если $\varnothing,X\in
\As$ и $\As$ замкнуто относительно конечных теоретико-множественных
операций. Алгебра называется \emph{$\si$-алгеброй}, если допускаются
счётные операции.
\end{df}

\begin{note}
В этом определении можно требовать замкнутости класса $\As$
относительно некоторых операций, через которые можно выразить все
остальные. Например, в определении $\si$-алгебры достаточно
требовать замкнутости относительно разности и счётных объединений. В
самом деле, имеем $X\backslash(\capn A_n)\bw =\cupn(X\backslash
A_n)$, $X\backslash(\cupn A_n)\bw =\capn(X\backslash A_n)$.

\end{note}

\textbf{Примеры:}

\pt{1}. $\As=\{\varnothing,X\}$ --- тривиальная $\si$-алгебра.

\pt{2}. $\As=2^X$ --- дискретная $\si$-алгебра (алгебра всех
подмножеств множества $X$).

\pt{3}. Элементарные множества на отрезке $[0,1]$ образуют алгебру,
но не $\si$-алгебру.

\pt{4}. $\As=\{\widetilde\As\subset\N:$ либо $\widetilde\As$
конечно, либо $\N\backslash\widetilde\As$ конечно$\}$ --- тоже
алгебра, но не $\si$-алгебра.

\pt{5}. $\As=\{E\subset X:$ либо $|E|\le\aleph_0$; либо
$|(X\backslash E)|\le\aleph_0\}$ является $\si$-алгеброй ($\aleph_0$
--- мощность множества натуральных чисел). Докажем это.
Если $E\in\As$, то $X\backslash E\in\As$ по определению. А если
$E_n\in\As$, то $\bigcup\limits_{n=1}^\infty E_n\in\As$. В самом
деле, если для всех $E_n$ выполнено неравенство $|E_n|\le\aleph_0$,
то $|\bigcup\limits_{n=1}^\infty E_n|\le\aleph_0$; а если для одного
из $E_n$ имеем $|X\backslash E_n|\le\aleph_0$, то получаем
$|X\backslash{\bigcup\limits_{n=1}^\infty E_n}|\le\aleph_0$.

\begin{theorem}
Если $\Fs$ --- семейство подмножеств множества $X$, то существует
наименьшая $\si$-алгебра, содержащая $\Fs$. Она обозначается через
$\si(\Fs)$.
\end{theorem}

\begin{proof}
Положим $\si(\Fs)=\bigcap\limits_{\Fs\subset\Ec}\Ec$, где
пересечение берётся по всем $\si$-алгебрам $\Ec$, содержащим $\Fs$.
Покажем, что множество $\si(\Fs)$ является $\si$-алгеброй. Если
$A\in\si(\Fs)$, то для любой $\si$-алгебры $\Ec$, содержащей
семейство $\Fs$, имеем $A\in\Ec$, поэтому $X\backslash A\in\Ec$ и,
значит, $X\backslash A\in\si(\Fs)$. Аналогично показываем, что из
$A_1,A_2,\ldots\in\si(\Fs)$ следует $\cupnui A_n\in\si(\Fs)$ и
$\capnui A_n\in\si(\Fs)$.
\end{proof}

\subsubsection{Борелевская $\si$-алгебра}

\begin{df}
\emph{Борелевской} $\si$-алгеброй пространства $\R^n$ (или его
подмножества) называется $\si$-алгебра, порождённая всеми открытыми
множествами.
\end{df}

\begin{note}
Поскольку дополнение к открытому множеству замкнуто, можно считать,
что борелевская $\si$-алгебра порождена всеми замкнутыми
множествами.
\end{note}
Обозначение для борелевской $\si$-алгебры: $\Bs(\R^n)$ или $\Bs(X)$,
где $X\subset \R^n$.

\begin{stm}\label{open_sets}
Всякое открытое в $\R^n$ множество есть не более чем счётное
объединение открытых шаров в $\R^n$ с рациональными центрами и
рациональными радиусами. Любое открытое множество на прямой --- это
конечное или счётное объединение попарно непересекающихся
(дизъюнктных) интервалов и лучей.
\end{stm}

\begin{proof}
Пусть $U\subset \R^n$ --- открытое множество. Для каждой
рациональной точки $p\in U$ берём все открытые шары $V(p,r)$
рационального радиуса $r$, такие что $V(p,r)\subset U$. Множество
всех таких шаров счётно. Осталось показать, что объединение этих
шаров есть всё $U$. Если точка $x\in U$ рациональна, то доказывать
нечего. Если же $x$ не является рациональной, то рассмотрим шар
$\Bb\subset U$ с центром в точке $x$. Поскольку сколь угодно близко
к точке $x$ имеются рациональные точки, существует шар с центром в
рациональной точке и рациональным радиусом, содержащий $x$ и
содержащийся в $\Bb$.

Для одномерного случая из приведённого выше рассуждения следует, что
любая связная компонента множества $U$ --- либо луч, либо интервал,
либо вся прямая $\R$.
\end{proof}

\begin{note}
Для $\R^2$ это уже не так.
\end{note}

\begin{problem}
Доказать, что открытый квадрат нельзя представить в виде счётного
объединения непересекающихся открытых кругов.
\end{problem}
\begin{problem}
Не существует паркета из замкнутых кругов для $\R^2$, если запретить
пересечения внутренностей.
\end{problem}

\begin{theorem}\label{borel_alg}
$\si$-алгебра $\Bs(\R)$ порождается каждым из следующих классов:

1) лучи $(-\infty,r)$, где $r\in\Q$;

2) лучи $(-\infty,r]$, где $r\in\Q$;

3) промежутки вида $(a,b]$, где $a,b\in\Q$;

4) промежутки вида $[a,b]$, где $a,b\in\Q$;

5) промежутки вида $(a,b)$, где $a,b\in\Q$.
\end{theorem}
Доказательство этой теоремы несложное и следует из второй части
утверждения~\ref{open_sets}.

\subsection{Измеримые функции}

Пусть $X$ --- измеримое пространство (т.е. пространство $X$ с
$\si$-алгеброй $\As$ его подмножеств).

\begin{df}
Функция $f:X\to\R^1$ называется \emph{измеримой относительно $\As$},
если для всякого множества $B\in\Bs(\R^1)$ имеем $f^{-1}(B)\in\As$.
\end{df}

\begin{theorem}\label{measurable}
Функция $f$ измерима относительно $\As$ тогда и только тогда, когда
$\{x:f(x)<c\}\in\As$ для всякого $c\in\R$.
\end{theorem}

\begin{proof}
Указанное множество есть $f^{-1}(E)$, где $E=(-\infty,c)$. Значит,
если $f$ измерима относительно $\As$, то $f^{-1}(E)\in\As$. Обратно,
если для всякого $c\in\R$ имеем $f^{-1}(E)\in\As$, то класс
$\Ec=\{E\subset \R:f^{-1}(E)\in\As\}$ содержит все открытые лучи
вида $(-\infty,c)$. Далее, $\Ec$ является $\si$-алгеброй
(докажите!), поэтому $\si(\{(-\infty, c):c\in\R\})\subset\Ec$.
Отсюда по теореме~\ref{borel_alg} получаем $\Bs(\R)\subset\Ec$, что
и требовалось.
\end{proof}

Пусть функция $f:\R^n\to\R^1$ измерима относительно борелевской
$\si$-алгебры. Тогда $f$ называется \emph{измеримой по Борелю} или
\emph{борелевской}.

Если $(X,\As)$ и $(Y,\Bs)$ --- измеримые пространства, то
отображение $f: X\to Y$ называют \emph{$(\As,\Bs)$-измеримым}, если
$f^{-1}(B)\in\As$ для любого $B\in\Bs$.

\begin{note}
Непрерывная функция является борелевской, так как множество
$\{x:f(x)<c\}$ открыто для любого $c\in\R$.
\end{note}

\begin{theorem}\label{measureable}
Пусть функции $f_n$ измеримы относительно $\si$-алгебры $\As$.
Тогда:

1) $\al_1f_1+\al_2f_2$ есть $\As$-измеримая функция для любых
$\al_1,\al_2\in\R$;

2) $f_1\cdot f_2$ --- $\As$-измеримая функция;

3) $f_1/f_2$ --- $\As$-измеримая функция, если $f_2\ne0$;

4) Если $\ph:\R\to\R$ --- борелевская функция, то функция $\ph\circ
f_1$ является $\As$-измеримой;

5) Если $f_n\to f$ при $n\to\infty$, то $f$ измерима относительно
$\As$;

6) $\max\{f_1,f_2\}$, $\min\{f_1,f_2\}$ --- $\As$-измеримые функции.
\end{theorem}

\begin{proof}
1) Если $\al\in\R$, то функция $\al\cdot f$ является
$\As$-измеримой, так как при $\al\ne 0$ имеем $(\al\cdot
f)^{-1}(-\infty,c)\bw=f^{-1}(-\infty,c/\al)\in\As$. Осталось
доказать, что сумма $f_1+f_2$ является $\As$-измеримой. Это следует
из цепочки равенств $\{x:(f_1+f_2)(x)<c\}\bw
=\{x:f_1(x)<c-f_2(x)\}\bw
=\cupnui(\{x:f_1(x)<r_n\}\cap\{x:r_n<c-f_2(x)\})$, где $\Q=\{r_n\}$.
Ясно, что последнее объединение входит в $\As$.

2),3) Следуют из п. 4).

4) Если $B\in\Bs(\R)$, то $\ph^{-1}(B)\in\Bs(\R)$, поэтому
$(\ph\circ f_1)^{-1}(B)\bw=f^{-1}_1(\ph^{-1}(B))\bw\in\As$. Значит,
функция $\ph\circ f_1$ $\As$-измерима. В частности, если $f$ ---
$\As$-измеримая функция, то функции $f^2$ и $1/f$ (при $f\ne0$)
также $\As$-измеримы. Теперь из равенства $f\cdot
g=\cfrac12((f+g)^2-f^2-g^2)$ следует пункт 2), а пункт 3) --- из
равенства $\cfrac{f}{g}=f\cdot\cfrac1g$.

5) Пусть $f=\lim\limits_{n\ra\infty}f_n$. Тогда $\{x:f(x)<c\}
=\cupkui\cupnui \bigcap\limits_{m>n}^\infty \{x:f_m(x)<c-\frac1k\}$,
что означает, что $f(x)<c$ тогда и только тогда, когда существуют
числа $k,n\in\N$, такие что для любого $m>n$ выполнено неравенство
$f_m(x)<c-\frac1k$. Так как для любых $m,k\in\N$ множество
$\{x:f_m(x)<c-\frac1k\}$ лежит в $\As$, то и множество
$\{x:f(x)<c\}$ лежит в $\As$.

6) Оставляется в качестве упражнения.
\end{proof}

\begin{imp}
Если начать с непрерывных функций и применять к ним операции
сложения, вычитания, умножения, деления и взятия предельных
переходов, то будут получаться $\Bs$-измеримые, т.е. борелевские
функции.
\end{imp}

\begin{note}
Пусть $B_0$ --- класс всех непрерывных функций. При $n=1,2,\ldots$
определим класс $B_n$ как множество всех функций, не лежащих в
классах $B_0,B_1,\ldots,B_{n-1}$, но являющихся поточечными
пределами последовательностей функций из этих классов. Множества
$B_n$ называются \emph{классами Бэра (Baire)}. Их объединение по $n$
от нуля до бесконечности ещё не даёт всего класса борелевских
функций.
\end{note}

\subsection{Меры и их продолжения}

\subsubsection{Меры}\label{measures}

Напомним, что если два множества $A$ и $B$ не пересекаются, то их
объединение может быть также обозначено через $A\sqcup B$. Такое
объединение называется \emph{дизъюнктным}. Символ $\sqcup$
подчёркивает, что пересечение объединяемых им множеств пусто.

\begin{df}
Пусть $\As$ --- алгебра множеств в пространстве $X$. Функция
$m:\As\to\R$ называется \emph{аддитивной}, если $m(A\sqcup
B)=m(A)+m(B)$ для любых непересекающихся множеств $A,B\in\As$.
\end{df}

\begin{df}
Функция $\mu:\As\to\R$ называется \emph{счётно-аддитивной} (или
\emph{$\si$-аддитивной}), если $\mu(\bigsqcup\limits_{n=1}^\infty
A_n) \bw=\sum\limits_{n=1}^\infty\mu(A_n)$ при условии, что
$\bigsqcup\limits_{n=1}^\infty A_n\in\As$ (Если $\As$ ---
$\si$-алгебра, то последнее условие излишне).
\end{df}

\begin{note}
Это важно. Такие счётно-аддитивные функции мы будем называть
\emph{мерами}.
\end{note}

\textbf{Примеры:}

\pt{1}. Пусть $\As=\{A\subset\N:$ либо $A$, либо $\N\backslash A$
конечно$\}$. Положим $\mu(n)=2^{-n}$. Тогда будем иметь
$$\mu(\{n_1,\ldots,n_k\})\bw =2^{-n_1}+\ldots+2^{-n_k}$$
и $\mu(\N)=2$. Получили, что $\mu$ --- счётно-аддитивная функция на
алгебре~$\As$.

\pt{2}. Рассмотрим множество $X$, счётное подмножество
$\{x_n\}\subset X$ и последовательность чисел $\al_n>0$, таких что
$\sum\limits_{n=1}^\infty\al_n=1$. Положим $\As=2^X$. Определим
функцию $\mu:\As\to\R$ формулой
$\mu=\sum\limits_{n=1}^\infty\al_n\de_{x_n}$, где
\equ{\de_{x_n(A)}=\case{1, & \text{если}\:\: x_n\in A,\\0, &
\text{если}\:\: x_n\notin A.}} Функция $\de_{x_n}$ называется
\emph{мерой Дирака} в точке $x_n$. Таким образом,
$\mu(A)=\sum\limits_{n:x_n\in A}\al_n$ для любого $A\in\As$. Можно
проверить, что функция $\mu$ счётно-аддитивна.

\pt{3}. Пусть $\As=\{A\subset[0,1]:$ либо $|A|\le\aleph_0$, либо
$|[0,1]\backslash A|\le\aleph_0\}$. Положим \equ{\mu(A)=\case{1, &
\text{если}\;|[0,1]\backslash A|\le\aleph_0,\\0, & \text{если}\;
|A|\le\aleph_0.}} Очевидно, что $\mu$ --- счётно-аддитивная мера.

\pt{4}. $\As$ --- алгебра элементарных множеств на отрезке [0,1].
Напомним, что $\As$ состоит из конечных объединений промежутков вида
$(a,b),[a,b),(a,b],[a,b]$. Пусть $\la$~--- функция длины
промежутков. Аддитивность этой функции очевидна. Чтобы доказать её
счётную аддитивность, нам потребуется ввести одно понятие.

\subsubsection{Компактные классы}

\begin{df}
Система $\Ks$ подмножеств в $X$ называется \emph{компактным
классом}, если из условия $\capnui K_n\bw =\varnothing\;
(K_n\in\Ks)$ следует, что существует $N\in\N$, такое что
$\bigcap\limits_{n=1}^N K_n =\varnothing$.
\end{df}
Иными словами, если каждое конечное пересечение непусто, то и
счётное непусто.

Основной пример компактных классов даётся следующей леммой.

\begin{lemma}Если любое множество $K\in\Ks$ является компактом, то
$\Ks$~--- компактный класс.
\end{lemma}

\begin{proof}
Пусть $K_n\in\Ks$ и $\bigcap\limits_{n=1}^N K_n\ne\varnothing$ для
любого $N\in\N$. Тогда существует точка
$x_N\in\bigcap\limits_{n=1}^N K_n$. Если последовательность
$\{x_n\}$ стабилизируется на элементе $x$, то $x$~--- это общий
элемент всех $K_n$. Иначе последовательность $\{x_n\}$ имеет
предельную точку $x$, эта точка лежит во всех $K_n$.
\end{proof}

\begin{theorem}\label{inf_add_comp}
Пусть $\mu$ --- неотрицательная аддитивная функция на алгебре $\As$
и существует компактный класс $\Ks\subset\As$, такой что для всякого
множества $A\in\As$ имеем $\mu(A)=\sup\{\mu(K): K\bw\subset A,K\bw
\in\Ks\}$. Тогда функция $\mu$ счётно-аддитивна.
\end{theorem}

Прежде чем доказывать эту теорему, установим два вспомогательных
результата.

\begin{prop}\label{semiadd}
Для любой неотрицательной аддитивной функции $\mu$ на алгебре $\As$
имеем $\mu(\bigcup\limits_{n=1}^N A_n)\bw \le \sum\limits_{n=1}^N
\mu(A_n)$, где $A_1,\ldots, A_N$ --- произвольные множества из
$\As$.
\end{prop}

\begin{proof}
Индукция по $N$. Для $N=1$ доказывать нечего. Пусть для $N=k\ge1$
утверждение доказано. Тогда $\mu(\bigcup\limits_{n=1}^{k+1}A_n)
=\mu((\bigcup\limits_{n=1}^{k}A_n)\sqcup(A_{k+1}\backslash
\bigcup\limits_{n=1}^{k}A_n))=\mu(\bigcup\limits_{n=1}^{k}A_n)
+\mu(A_{k+1}\backslash \bigcup\limits_{n=1}^{k}A_n)$. По
предположению индукции имеем $\mu(\bigcup\limits_{n=1}^k
A_n)\bw\le\sum\limits_{n=1}^k \mu(A_n)$. Кроме того, из равенства
$A_{k+1}=(A_{k+1}\backslash \bigcup\limits_{n=1}^{k}A_n)
\bw\sqcup(A_{k+1}\cap \bigcup\limits_{n=1}^{k}A_n)$ следует, что
$\mu(A_{k+1}\backslash \bigcup\limits_{n=1}^{k}A_n)\le\mu(A_{k+1})$.
Отсюда окончательно имеем $\mu(\bigcup\limits_{n=1}^{k+1}A_n)
\bw\le\sum\limits_{n=1}^{k+1} \mu(A_n)$.
\end{proof}

\begin{prop}\label{inf_add}
Аддитивная функция $\mu$ на алгебре $\As$ счётно-аддитивна тогда и
только тогда, когда $\lim\limits_{n\ra\infty}\mu(A_n)=0$ при
$A_n\downarrow\varnothing$ \upshape{(}т.е. $A_{n+1}\subset A_n$ и
$\capnui A_n=\varnothing$, это свойство называется
\emph{непрерывностью функции $\mu$ в нуле}\upshape{)}.
\end{prop}

\begin{proof}
Пусть есть непрерывность в нуле. Рассмотрим семейство множеств
$B_n\in\As$, причём $B_i\cap B_j=\varnothing$ при $i\ne j$. Пусть
$B=\cupnui B_n\in\As$. Положив $C_n=\bigcup\limits_{i=n}^\infty
B_i$, имеем $C_n\downarrow\varnothing$. Поэтому
$\mu(C_n)\xra{n\ra\infty} 0$. Отсюда
$\mu(B)-\mu(\bigsqcup\limits_{i=1}^{n-1}B_i)\xra{n\ra\infty}0$. Так
как $\mu(\bigsqcup\limits_{i=1}^{n-1}B_i)
=\sum\limits_{i=1}^{n-1}\mu(B_i)$, то получаем
$\mu(B)=\sumiui\mu(B_i)$.

Обратно. Имеем $A_1=(A_1\backslash A_2)\sqcup(A_2\backslash
A_3)\sqcup...$, поэтому сходится ряд $\sumnui\mu(A_n\backslash
A_{n+1})$, откуда \\$\sum\limits_{n=N}^\infty \mu(A_n\backslash
A_{n+1})\bw\ra0$ при $N\ra\infty$. Но $\sum\limits_{n=N}^\infty
\mu(A_n\backslash A_{n+1})=\mu(A_N)$, ибо
$A_N=\bigsqcup\limits_{n=N}^\infty(A_n\backslash A_{n+1})$. Значит,
$\mu(A_n)\ra0$ при $n\ra\infty$.
\end{proof}

Теперь докажем теорему~\ref{inf_add_comp}.

\begin{proof}
Воспользуемся предложением~\ref{inf_add}. Пусть $\{A_n\}\in\As$,
$A_{n+1}\subset A_n$ и $\capnui A_n=\varnothing$. Докажем, что
$\mu(A_n) \bw{\mathbin{\xra{n\ra\infty}}}0$, это и даст нам счётную
аддитивность функции $\mu$. В самом деле, если $\mu(A_n)\nra0$ при
$n\ra\infty$, то существует $\ve>0$, такое что $\mu(A_n)\ge\ve$ при
любом $n$, поскольку $\mu(A_{n+1})\le\mu(A_n)$. Далее, существует
множество $K_n\in\Ks$, такое что $K_n\subset A_n$ и $\mu(A_n)\bw\le
\mu(K_n)+\frac{\ve}{2^n}$. Заметим, что $\capnui K_n\subset\capnui
A_n=\varnothing$, т.е. $\capnui K_n=\varnothing$. Поэтому в силу
компактности класса существует $N\in\N$, такое что
$\bigcap\limits_{n=1}^N K_n=\varnothing$. Теперь заметим, что
$A_N\bw =\bigcap\limits_{n=1}^NA_n\bw \subset
\bigcup\limits_{n=1}^N(A_n\backslash K_n)$. В самом деле, пусть
$x\in A_N$. Тогда $x\in A_1,\ldots,A_{N-1},A_N$. Если
$x\notin\bigcup\limits_{n=1}^N(A_n\backslash K_n)$, то $x\notin
A_n\backslash K_n$ при каждом $n\le N$. Но тогда $x\in K_n$ для
каждого $n\le N$, откуда
$x\in\bigcap\limits_{n=1}^NK_n=\varnothing$. Противоречие. Теперь
имеем $\mu(A_N)\le\mu(\bigcup\limits_{n=1}^N(A_n\backslash K_n))
\bw\le\sum\limits_{n=1}^N\mu(A_n\backslash K_n)<\ve$. Противоречие.
\end{proof}

Теперь вернёмся к примеру~\pt{4} п.~\ref{measures}. Пусть $\la$~---
функция длины на алгебре $\As$ элементарных множеств. Тогда $\la$
счётно-аддитивна. В самом деле, в качестве компактного класса $\Ks$
можно взять конечные объединения отрезков.

Аналогично обстоит дело в $\R^2$ ($\R^N$) с функцией площади
($n$-мерного объёма) $\la$ на алгебре элементарных множеств.

\begin{note}
Условия теоремы~\ref{inf_add_comp} не являются необходимыми для
счётной аддитивности, но в практических ситуациях они выполняются.
\end{note}

\begin{problem}
Пусть $\mu$ --- непрерывная мера на $\Bs(\R^n)$. Тогда для любого
множества $B\in\Bs(\R^n)$ и любого $\ve>0$ найдутся компактное
множество $K_{\ve}\subset B$ и открытое множество $U_{\ve}\supset
B$, такие что $\mu(U_{\ve}\backslash K_{\ve})<\ve$.
\end{problem}

\begin{note}
Это же верно для любого полного сепарабельного метрического
пространства.
\end{note}

Итак, аддитивность, вообще говоря, не означает счётную аддитивность,
однако влечёт её при некоторых дополнительных условиях (например,
наличие компактных классов). Однако есть и условия, которые являются
и необходимыми, и достаточными.

\subsubsection{Эквивалентные условия счётной аддитивности меры}

\begin{prop}\label{count_add}
Пусть неотрицательная функция $\mu$ аддитивна на алгебре $\As$.
Тогда следующие условия равносильны:

\pt{1}. Функция $\mu$ счётно-аддитивна.

\pt{2}. Функция $\mu$ непрерывна в нуле (т.е. $\mu(A_n)\ra0$ при
$n\ra\infty$, если $A_n\downarrow\varnothing; A_n\in\As$).

\pt{3}. Если $B_n\in\As$ и $B_n\uparrow B\in\As$ (т.е. $B_n\subset
B_{n+1}$ и $\cupnui B_n=B$), то $\mu(B_n)\xra{n\ra\infty}\mu(B)$.

\pt{4}. Если $A_n\in\As$ и $\cupnui A_n\in\As$, то $\mu(\cupnui
A_n)\le\sumnui\mu(A_n)$ (это свойство функции $\mu$ называется
\emph{счётной полуаддитивностью}).
\end{prop}

\begin{proof}
Равносильность \pt{1} $\Lra$ \pt{2} уже доказана (см.
предложение~\ref{inf_add}).

\pt{2} $\Lra$ \pt{3}, ибо если $A_n\downarrow\varnothing$, то
$X\backslash A_n\uparrow X$; кроме того, $B_n\uparrow B\Lra
(B\backslash B_n)\downarrow\varnothing$.

Осталось доказать равносильность пунктов \pt{1}, \pt{2}, \pt{3}
пункту \pt{4}. Пусть $A_n\in\As$ и $\cupnui A_n\in\As$. Положим
$B_n=\bigcup\limits_{k=1}^nA_k\in\As$. Имеем $B_n\uparrow\cupnui
A_n$, поэтому $\mu(\cupnui A_n)=\lim\limits_{n\ra\infty}\mu(B_n)$.
Далее, по лемме~\ref{semiadd} получаем $\mu(B_n)\bw
\le\sum\limits_{k=1}^n\mu(A_k)$.
% Ясно, что $B_n=\widetilde
%A_1\sqcup\ldots\sqcup\widetilde A_n$, где $\widetilde
%A_i=A_i\backslash(A_1\cup\ldots\cup A_{i-1})\in \As$ ($\widetilde
%A_1=A_1$), $\widetilde A_i\subset A_i$.
Отсюда $\mu(B_n)\le\sumkui\mu(A_k)$. Переходя к пределу при
$n\ra\infty$, получаем $\mu(\cupkui A_k)\bw\le\sumkui\mu(A_k)$.

Обратно, пусть выполнено условие пункта \pt{4}. Рассмотрим множества
$A_n\in\As$, такие что $\bigsqcup\limits_{n=1}^\infty A_n\in\As$. В
силу счётной полуаддитивности функции $\mu$ имеем
$\mu(\bigsqcup\limits_{n=1}^\infty A_n)\le\sumnui\mu(A_n)$. С другой
стороны, в силу неотрицательности функции $\mu$ имеем
$\mu(\bigsqcup\limits_{n=1}^\infty A_n)\bw
\ge\mu(\bigsqcup\limits_{n=1}^k A_n)
\bw=\sum\limits_{n=1}^k\mu(A_n)$ при любом натуральном $k$. Значит,
$\mu(\bigsqcup\limits_{n=1}^\infty A_n)\bw\ge\sumnui\mu(A_n)$, что и
доказывает счётную аддитивность функции $\mu$.
\end{proof}

Как строить счётно-аддитивные меры на $\si$-алгебрах? Оказывается,
что счётно-аддитивная мера на алгебре $\As$ продолжается до
счётно-аддитивной меры на $\si(\As)$.

\subsection{Внешняя мера и продолжение мер}

\subsubsection{Внешняя мера}

Определим внешнюю меру $\mu^*$ для неотрицательной $\si$-аддитивной
меры $\mu$ на алгебре $\As$.

\begin{df}
Пусть $E$ --- \underline{любое} множество из $2^X$. \emph{Внешней
мерой} множества $E$ называется величина $\mu^*(E)\bw
=\inf\{\sumnui\mu(A_n): A_n\in\As; E\subset\cupnui A_n\}$.
\end{df}
Покрывать есть чем, поэтому $0\le\mu^*(E)\le\mu(X)$. Вообще говоря,
внешняя мера неаддитивна на классе всех множеств.

\begin{lemma}\label{sym_diff}

\pt{1}. $|\mu^*(A)-\mu^*(B)|\le\mu^*(A\triangle B)$ для любых
$A,B\in 2^X$.

\pt{2}. Функция $\mu^*$ обладает свойством счётной полуаддитивности
на $2^X$.
\end{lemma}

\begin{proof}
\pt{1}. Покажем, что $\mu^*(A)\le\mu^*(B)+\mu^*(A\triangle B)$.
Имеем: $A\subset B\cup(A\triangle B)$, поэтому достаточно проверить,
что внешняя мера полуаддитивна, т.е. $\mu^*(E_1\cup
E_2)\le\mu^*(E_1)+\mu^*(E_2)$ для любых $E_1,E_2\in2^X$. В самом
деле, пусть $E_1\subset\cupnui A'_n$ и $E_2\subset\cupnui A''_n$,
причём $\mu^*(E_1)\ge\sumnui\mu(A'_n)-\ve$ и
$\mu^*(E_2)\ge\sumnui\mu(A''_n)-\ve$. Тогда $E_1\cup
E_2\subset\cupnui(A'_n\cup A''_n)$ и $\mu^*(E_1\cup E_2)\bw
\le\sumnui\mu(A'_n\cup A''_n)\bw
\le\sumnui\mu(A'_n)+\sumnui\mu(A''_n)\bw
\le\mu^*(E_1)+\mu^*(E_2)+2\ve$. Осталось устремить $\ve$ к нулю.

\pt{2}. Пусть $A_n\subset X$, $n\in\N$. Пусть $\ve>0$ ---
произвольное число. Для каждого $n\in\N$ существует набор множеств
$\{P_{nk}\}_{k=1}^\infty\subset\As$, такой что $A_n\subset\cupkui
P_{nk}$ и $\sumkui\mu(P_{nk})\bw \le\mu^*(A_n)+\frac{\ve}{2^n}$.
Тогда $\cupnui A_n\subset\cupnui\cupkui P_{nk}$, откуда
$\mu^*(\cupnui A_n)\le\sumnui\sumkui\mu(P_{nk})\bw
\le\sumnui\mu^*(A_n)+\ve$. В силу произвольности $\ve$ получаем
$\mu^*(\cupnui A_n)\bw \le\sumnui\mu^*(A_n)$, что и требовалось.
\end{proof}

\begin{df}
Обозначим через $\Ls_{\mu}$ класс таких множеств $E\subset X$, что
для любого $\ve>0$ существует множество $A_{\ve}\in\As$,
удовлетворяющее условию $\mu^*(E\triangle A_{\ve})<\ve$. Множества
из $\Ls_{\mu}$ называются \emph{измеримыми по Лебегу} относительно
меры $\mu$ (или просто \emph{$\mu$-измеримыми}).
\end{df}

\subsubsection{Основная теорема о продолжении меры}

\begin{theorem}\label{continue_measure}
Пусть $\mu$ --- счётно-аддитивная неотрицательная мера на алгебре
$\As$. Тогда:

\pt{1}. Внешняя мера $\mu^*$ совпадает с $\mu$ на $\As$.

\pt{2}. Множество $\Ls_{\mu}$ является $\si$-алгеброй, содержащей
$\si(\As)$, и функция $\mu^*$ счётно-аддитивна на $\Ls_{\mu}$. В
частности, $\mu^*$ даёт счётно-аддитивное продолжение меры $\mu$ на
$\si(\As)$.
\end{theorem}

\begin{note}
Это самая трудная теорема в курсе.
\end{note}

\begin{proof}
\pt{1}. Пусть $E\in\As$. Тогда $\mu^*(E)\bw \le\mu(E)$, ибо $E$ себя
покрывает. С другой стороны, если $E\subset\cupnui A_n$,
$A_n\in\As$, то $\mu(E)=\mu(\cupnui(A_n\cap
E))\bw\le\sumnui\mu(A_n\cap E)\bw \le\sumnui\mu(A_n)$, откуда
$\mu(E)\le\mu^*(E)$. Таким образом, $\mu(E)=\mu^*(E)$ и $\As\subset
\Ls_\mu$.

\pt{2}. Если $A\in\Ls_{\mu}$, то $X\backslash A\in\Ls_{\mu}$, ибо
для любого $E\in\As$ имеем $(X\backslash A)\triangle(X\backslash
E))=A\triangle E$, и потому $\mu^*\BS{(X\backslash A)
\triangle(X\backslash E)}=\mu^*(A\triangle E)$, т.е. класс
$\Ls_{\mu}$ замкнут относительно дополнений.

Теперь пусть $A,B\in\Ls_{\mu}$. Покажем, что $A\cup B\in\Ls_{\mu}$.
Возьмём произвольное $\ve>0$. Тогда существуют множества
$E_1,E_2\in\As$, такие что $\mu^*(A\triangle E_1)<\ve$ и
$\mu^*(A\triangle E_2)<\ve$. Поскольку $(A\cup B)\triangle (E_1\cup
E_2)\bw\subset(A\triangle E_1)\cup(B\triangle E_2)$, имеем
$\mu^*\BS{(A\cup B)\triangle (E_1\cup E_2)}\bw \le\mu^*(A\triangle
E_1)+\mu^*(B\triangle E_2)<2\ve$ по лемме~\ref{sym_diff}.

Раз множество $\Ls_{\mu}$ замкнуто относительно операций
$\backslash$ и $\cup$, следовательно, $\Ls_{\mu}$ является алгеброй.

Теперь докажем аддитивность функции $\mu^*$. Пусть $A,B\in\Ls_{\mu}$
и $A\cap B=\varnothing$. Нужно показать, что $\mu^*(A\sqcup B)\bw
=\mu^*(A)+\mu^*(B)$. Неравенство $\mu^*(A\sqcup B)\bw
\le\mu^*(A)+\mu^*(B)$ следует из леммы~\ref{sym_diff}. Пусть
$\ve>0$~--- произвольное число. Существуют множества
$E_1,E_2\in\As$, такие что $\mu^*(A\triangle E_1)<\ve$ и
$\mu^*(A\triangle E_2)<\ve$. По лемме~\ref{sym_diff} имеем
$\mu^*(A\cup B)\bw \ge\mu^*(E_1\cup E_2)\bw -\mu^*\BS{(A\cup
B)\triangle(E_1\cup E_2)}$. Далее, выше было показано, что
$\mu^*\BS{(A\cup B)\triangle (E_1\cup E_2)}\bw<2\ve$. Поскольку
$E_1,E_2\in\As$, с учётом пункта~\pt{1} имеем $\mu^*(E_1\cup
E_2)\bw=\mu(E_1\cup E_2)\bw =\mu(E_1)+\mu(E_2)-\mu(E_1\cap E_2)\bw
\le\text{(лемма~\ref{sym_diff})}\bw \le \mu^*(A)-\ve+\mu^*(B)
-\ve-\mu(E_1\cap E_2)$. При этом $E_1\cap E_2\subset (A\triangle
E_1)\cup(B\triangle E_2)$, откуда $\mu(E_1\cap E_2)\bw \le
\ve+\ve=2\ve$. Итак, $\mu(E_1\cup E_2)\bw\ge\mu^*(A) +\mu^*(B)-4\ve$
и поэтому $\mu^*(A\cup B)\bw \ge\mu^*(A)+\mu^*(B)-6\ve$. Так как
$\ve>0$ произвольно, то $\mu^*(A\cup B)\bw \ge\mu^*(A)+\mu^*(B)$,
что и доказывает аддитивность функции $\mu^*$.

В силу счётной полуаддитивности функции $\mu^*$ на алгебре
$\Ls_{\mu}$ (лемма~\ref{sym_diff}) из предложения~\ref{count_add}
следует счётная аддитивность $\mu^*$ на $\Ls_{\mu}$.

Осталось доказать, что $\Ls_{\mu}$ --- это $\si$-алгебра. Пусть
$\{A_n\}\subset\Ls_{\mu}$. Надо доказать, что $\cupnui
A_n\in\Ls_{\mu}$. Дело сводится к дизъюнктному объединению, если
взять $\widetilde A_1=A_1$, $\widetilde A_2=A_2\backslash A_1$,
$\ldots$, $\widetilde A_n=A_n\backslash\bigcup
\limits_{k=1}^{n-1}A_k$, $\ldots$. Ясно, что $\widetilde
A_n\in\Ls_{\mu}$ при любом $n\in\N$ и что $\cupnui
A_n=\cupnui\widetilde A_n$. Теперь считаем, что множества $A_n$
попарно не пересекаются. Для любого $n\in\N$ имеем
$\sum\limits_{k=1}^n\mu^*(A_k)=\mu^*(\bigsqcup\limits_{k=1}^nA_k)\bw
\le\mu^*(\bigsqcup\limits_{k=1}^\infty A_k)$. Поэтому ряд
$\sumkui\mu^*(A_k)$ сходится, а значит, для всякого $\ve>0$
существует $N\in\N$, такое что $\sum\limits_{k>N}\mu^*(A_k)<\ve$. По
уже доказанному имеем $A_1\sqcup\ldots\sqcup A_N\in\Ls_{\mu}$.
Значит, существует множество $E\in\As$, такое что
$\mu^*(E\triangle(\bigsqcup\limits_{k=1}^N A_k))<\ve$. Покажем, что
множество $E$ хорошо аппроксимирует и множество
$\bigsqcup\limits_{k=1}^\infty A_k$. Имеем
$(\bigsqcup\limits_{k=1}^\infty A_k)\bw \triangle E \bw
\subset((\bigsqcup\limits_{k=1}^N A_k) \bw \triangle E)\bw
\cup(\bigsqcup\limits_{k>N}A_k)$. Тогда
$\mu^*\BS{(\bigsqcup\limits_{k=1}^\infty A_k) \triangle E} \bw \le
\ve+\mu^*(\bigsqcup\limits_{k>N}A_k)
\bw\le\ve+\sum\limits_{k>N}\mu^*(A_k)< 2\ve$. Таким образом,
множество $\cupkui A_k$ по определению принадлежит $\Ls_{\mu}$,
поэтому $\Ls_\mu$~--- $\si$-алгебра. Так как $\As\subset\Ls_\mu$, то
$\si(\As)\subset \Ls_\mu$.
\end{proof}

\begin{note}
Более общую конструкцию Каратеодори см. в книжке В.И. Богачёва
<<Основы теории меры>>.
\end{note}

\begin{imp}
Неотрицательная счётно-аддитивная мера на алгебре $\As$ продолжается
на $\Ls_{\mu}$ и на $\si(\As)$ однозначно (с требованием счётной
аддитивности).
\end{imp}

\begin{proof}
Пусть $A\in\Ls_{\mu}$ и $\la\ge0$ --- какое-нибудь счётно-аддитивное
продолжение $\mu$ на $\si$-алгебру, содержащую множество~$A$.
Зафиксируем произвольное $\ve>0$. Тогда по определению измеримости
существует множество $B\bw\in\As$, такое что $\mu^*(A\triangle
B)<\ve$. Это означает, что существуют множества $C_n\in\As$, такие
что $A\triangle B\bw \subset\cupnui C_n$ и $\sumnui\mu(C_n)<\ve$.
Тогда имеем $\la(A\triangle
B)\le\sumnui\la(C_n)=\sumnui\mu(C_n)<\ve$, поскольку $\mu\equiv\la$
на $\As$. Отсюда следует, что $|\la(A)-\la(B)|\le\la(A\triangle
B)<\ve$. С другой стороны, также справедливо неравенство
$|\mu^*(A)-\mu^*(B)|<\ve$. Учитывая, что $\la(B)=\mu(B)=\mu^*(B)$,
получаем $|\la(A)-\mu^*(A)|<2\ve$, откуда в силу произвольности
$\ve$ следует $\la(A)=\mu^*(A)$.
\end{proof}

\begin{problem}
 Пусть $\mu$ --- счётно-аддитивная неотрицательная мера на
 $\si$-алгебре $\As$ и $E\notin\As$. Тогда существует
 счётно-аддитивная неотрицательная мера $\nu$ на
 $\si(\As\cup\{E\})$, которая совпадает с $\mu$ на $\As$.
\end{problem}

\subsubsection{Применения основной теоремы}

Применим предыдущую теорему к алгебре элементарных множеств в
отрезке (в случае $\R^n$~--- в $n$-мерном кубе) $I$ с мерой $\mu$,
равной длине (соответственно $n$-мерному объёму). На этой алгебре
функция $\mu$ счётно-аддитивна (так как существует компактный
класс). Такое продолжение называется \emph{мерой Лебега}, а
множества из $\Ls_{\mu}$ --- \emph{измеримыми по Лебегу}. Заметим,
что $\Ls_{\mu}\supset\Bs(I)$.

Теорему~\ref{continue_measure} можно применить к любой
неотрицательной аддитивной функции на алгебре элементарных множеств
в $I$. В результате получим неотрицательную меру на некоторой
$\si$-алгебре, содержащей $\si$-алгебру $\Bs(I)$. Ограничение этой
меры на $\Bs(I)$ называется \emph{борелевской мерой}. Ясно, что
любая неотрицательная счётно-аддитивная мера на $\Bs(I)$ является
борелевской.

\begin{note}
Кроме алгебр рассматривают ещё кольца и полукольца. \emph{Кольцо}
$\Rg$ допускает операции $\cup$, $\cap$, $\backslash$, и, кроме
того, $\varnothing\in\Rg$. Не всегда имеем $X\in\Rg$, поэтому
кольцо, вообще говоря, не является алгеброй. Например, ограниченные
множества в $\R$ образуют кольцо. \emph{Полукольцо} $\Rg_0\subset
2^X$ --- это система множеств, которая вместе с любыми двумя
множествами $A,B$ содержит их пересечение $A\cap B$, а разность
$A\backslash B$ (которая не обязательно сама лежит в $\Rg_0$) может
быть представлена в виде $A\backslash B=R_1\sqcup\ldots\sqcup R_m$
для некоторых множеств $R_1,\ldots,R_m\in\Rg_0$. Примером полукольца
является класс $\{[\al,\be)\}\subset\R$, причём он не является
кольцом. Можно легко проверить, что счётно-аддитивная мера на
$\Rg_0$ продолжается до счётно-аддитивной меры на кольце $\Rg$,
порождённом полукольцом $\Rg_0$ (это кольцо составляют конечные
объединения элементов из $\Rg_0$). Далее, с кольца $\Rg$ эта мера
продолжается на $\si$-алгебру $\si(\Rg)$.
\end{note}

\begin{ex}\label{Lebeg-Stilt'es}
Мера Лебега-Стилтьеса. Пусть дана вероятностная борелевская мера
$\mu$ на $\R$ (термин <<вероятностная мера>> означает, что мера
всего пространства равна $1$). \emph{Функция распределения} этой
меры есть $F_{\mu}(t)=\mu((-\infty,t))$. Функция распределения
обладает следующими свойствами:

1) $F_{\mu}$ монотонно не убывает;

2) $F_{\mu}(t)\ra 1$ при $t\ra+\infty$;

3) $F_{\mu}(t)\ra0$ при $t\ra-\infty$;

4) $F_{\mu}$ непрерывна слева.
\end{ex}

\textbf{Примеры} мер Лебега-Стилтьеса:

\pt{1}. Мера Лебега $\la$ на отрезке $[0,1]$. Функция распределения
для этой меры имеет вид \equ{F_{\la}(t)=\case{0, & t\le0; \\t, &
t\in[0,1];\\1, & t\ge1.}}

\pt{2}. Мера $\de$ Дирака в нуле. Функция распределения
\equ{F_{\de}(t)=\case{0,\: & t\le0; \\ 1,\; & t>0}} имеет скачок в
точке $t=0$. Вообще из равенства
$\mu((-\infty,t])=\mu((-\infty,t))+\mu(\{t\})$ следует, что функция
$F_{\mu}$ имеет скачок в точке $t$ тогда и только тогда, когда
$\mu(\{t\})>0$. В нашем случае $\de(\{0\})=1$.

Обратно, если функция $F(t)$ обладает свойствами 1)--4) функций
распределения, то существует единственная вероятностная борелевская
мера $\mu$, такая что $F_{\mu}(t)=F(t)$. В самом деле, заметим, что
$F$ имеет не более чем счётное число точек разрыва. Тогда возьмём в
$\R$ счётное всюду плотное множество $S$ точек непрерывности функции
$F$ и рассмотрим алгебру $\As$ конечных объединений промежутков вида
$\{\al,\be\}$, $(-\infty,\al\}$, $\{\al,+\infty)$ (скобки "$\{$" и
"$\}$" заменяют любые из скобок "$($", "$[$" \,и "$)$", "$]$"\,
соответственно), где $\al,\be\in S$. Для любых $\al,\be\in S$,
$\al<\be$, положим $\mu(\{\al,\be\})\bw =F(\be)-F(\al)$,
$\mu((-\infty,\al\})=F(\al)$, $\mu(\{\al,+\infty))=1-F(\al)$. Таким
образом, мы задали меру на алгебре $\As$. Ясно, что эта мера
конечно-аддитивна.

\begin{note}
На самом деле $\mu$ счётно-аддитивна, ибо существует компактный
класс (для лучей всё следует из определений функции $F$ на
$\pm\infty$).
\end{note}

Основная теорема о продолжении меры (теорема~\ref{continue_measure})
даёт меру $\mu$ на $\Bs(\R)$ (так как $\si(\As)=\Bs(\R)$), её-то и
называют мерой Лебега-Стилтьеса.

\subsubsection{Свойства меры Лебега в $\R^n$}

%Теперь отметим некоторые свойства меры Лебега в $\R^n$.

\begin{prop} Пусть множество $A\subset\R^n$ измеримо по Лебегу и
$\la_n$ --- мера Лебега в $\R^n$. Тогда:

1. Для любого вектора $\overline h\in\R^n$ множество $A+\overline h$
тоже измеримо по Лебегу и $\la_n(A+\overline h)=\la_n(A)$.

2. Для любого $\ve>0$ существуют компактное множество
$K_{\ve}\subset A$ и открытое множество $U_{\ve}\supset A$, такие
что $\la_n(U_{\ve})-\ve\le\la_n(A)\le\la_n(K_{\ve})+\ve$.
\end{prop}

\begin{proof}
1. Следует из того, что доказываемое верно для элементарных
множеств, поэтому это верно и для внешней меры.

2. Из определения следует, что существуют элементарные множества
$E_k$, такие что $\la_n(A)\ge\sumkui\la_n(E_k)\bw -\ve$ и
$A\subset\cupkui E_k$. Более общим образом, можно считать, что все
$E_k$ открыты. Поэтому можно положить $U_{\ve}=\cupkui E_k$. Взяв
дополнение множества $A$ до куба и применив к нему доказанное,
получим вписанный в $A$ компакт близкой меры.
\end{proof}

\begin{problem}
Пусть $O$ --- ортогональный оператор в $\R^n$. Тогда для любого
измеримого множества $A$ имеем $\la_n(A)=\la_n(O(A))$.
\end{problem}

\begin{ex}
Пусть $I$ --- бесконечномерный куб:
$I=\{(x_1,x_2,x_3,\ldots):\,x_i\in[0,1]\}$. В $I$ есть алгебра
цилиндров вида $C=\{(x_i)\in I:(x_1,\ldots,x_n)\in B\}$, где
$B\in\Bs([0,1]^n)$, $n\in\N$. Множество $B$ называется
\emph{основанием} цилиндра $C$.
\end{ex}

\begin{problem}
Это действительно алгебра.
\end{problem}
\begin{problem}
Пусть $\la_\infty(C)=\la_n(B)$. Показать, что эта функция корректно
определяет меру на алгебре цилиндров.
\end{problem}

\begin{problem}
Класс цилиндров с компактными основаниями компактен. Следствие: мера
$\la_\infty$ счётно-аддитивна.
\end{problem}

\textbf{Пример} неизмеримого по Лебегу множества. Рассмотрим отрезок
$[0,1]$ и введём на нём отношение $x\thicksim y \bw\Lra x-y\in\Q$.
То, что это отношение эквивалентности, сомнения не вызывает. В
каждом классе эквивалентности число точек счётно. По аксиоме выбора
существует множество $E$, содержащее ровно по одному элементу
каждого класса. Покажем, что множество $E$ неизмеримо по Лебегу.
Предположим противное. Заметим сначала, что $(E+r_1)\cap
(E+r_2)=\varnothing$ при любых $r_1,r_2\in\Q$. Далее,
$\la(E)=\la(E+r)$ при любом $r\in\Q$ в силу инвариантности меры
Лебега при сдвиге. Пусть $\la(E)=0$. Тогда заметим, что отрезок
$[0,1]$ содержится в объединении множеств вида $E+r$ при
$r\in[-1,1]\cap\Q$. Значит, $1=\la([0,1])
\bw\le\la(\bigcup\limits_{r\in[-1,1]\cap\Q}(E+r))\bw
\le\sum\limits_{r\in[-1,1]\cap\Q}\la(E+r)\bw
=\sum\limits_{r\in[-1,1]\cap\Q}0=0$ --- противоречие. Теперь
предположим, что $\la(E)=d>0$. Тогда имеем $[-1,2]\subset
\bigsqcup\limits_{r\in[0,1]\cap\Q}(E+r)$. Значит, $3\bw =\la([-1,2])
\bw\ge\la(\bigcup\limits_{r\in[0,1]\cap\Q}(E+r))\bw
\ge\sum\limits_{r\in[0,1]\cap\Q}\la(E+r)\bw
=\sum\limits_{r\in[0,1]\cap\Q}d=\infty$. Противоречие.

\begin{problem}
Построить пример множества $E\subset[0,1]$, такого что $\la_*(E)=0$,
$\la^*(E)=1$, где $\la_*(E)\bw =\sup\{\sumnui\la(A_n):A_n\in\As,
E\supset\bigsqcup\limits_{n=1}^\infty A_n\}$.
\end{problem}

Теперь приведём пример множества, неизмеримого относительно
какой-либо меры. Возьмём $X\bw=\{0,1\}$, положим
$\mu(\varnothing)=0$, $\mu(\{X\})=1$, $\As=\{\varnothing, X\}$.
Тогда ясно, что множество $\{1\}$ неизмеримо.

\subsubsection{Описание измеримых множеств}

\begin{prop}\label{close_set_from_A}
Пусть $\As$ --- некоторая $\si$-алгебра подмножеств $X$ с заданной
на ней $\si$-аддитивной мерой $\mu$ и $A\subset X$. Тогда
$A\in\Ls_\mu\,\Lra\,A=A_0\cup Z$, где $A_0\in\As$ и $\mu^*(Z)=0$.
Иными словами, существуют множества $A_1,A_2\in\As$, такие что
$A_1\subset A\subset A_2$ и $\mu(A_1)=\mu(A_2)$.
\end{prop}

\begin{proof}
В обратную сторону доказываемое утверждение ясно. Докажем его в
прямую. Если $A\in\Ls_\mu$, то для всякого $\ve>0$ существуют
множества $B_k\in\As$, такие что $\mu^*(A)\ge\sumkui\mu(B_k)-\ve$ и
$A\subset\cupkui B_k$. Положим $A_{2,\ve}=\cupkui B_k\in\As$. Тогда
имеем $\mu^*(A)\bw\ge\mu(A_{2,\ve})-\ve$. Теперь берём
$A_2=\bigcap\limits_{m=1}^\infty A_{2,\frac1m}\in\As$, тогда
$\mu^*(A)=\mu(A_2)$. Дополнение к множеству $A$ входит в $\Ls_\mu$,
применим к нему доказанное. Получим множество $B\in\As$, такое что
$X\backslash A\subset B$ и $\mu^*(X\backslash A)=\mu(B)$. Тогда
множество $A_1=X\backslash B$ удовлетворяет нашим требованиям.
\end{proof}


\subsection{Измеримые функции на пространствах с мерами}

Пусть $\mu\ge0$ --- мера на $\si$-алгебре $\As$ в $X$. Отныне
$\si$-алгебру $\Ls_\mu$ множеств, измеримых по Лебегу относительно
меры $\mu$, будем обозначать через $\As_\mu$, а продолжение $\mu^*$
меры $\mu$ на $\si$-алгебру $\As_\mu$ будем обозначать тем же
символом $\mu$.

Отметим, что развиваемая нами теория в основном относится к случаю,
когда $\mu(X)\bw <\infty$, на это следует обратить внимание. Но
большинство теорем, которые мы докажем, также справедливы и для
случая $\mu(X)\bw =\infty$. Об этом см. п.~\ref{infty_measures}.

\begin{df}
Функции, измеримые относительно $\As_\mu$, назовём
\emph{$\mu$-измеримыми}. Кроме того, функцию~$f$ будем считать
$\mu$-измеримой, если она определена на множестве $X_0\in\As_\mu$,
таком что $\mu(X\backslash X_0)=0$ (это называется <<$f$
определена $\mu$-почти всюду>>), и функция~$f\left|_{X_0}\right.$
является $\mu$-измеримой, где $\mu$ рассматривается на $X_0$. При
этом на $X\backslash X_0$ функция $f$ может быть не определена или
принимать значения $\pm\infty$. Фактически требуется выполнение
условия $\{x\in X_0: f(x)<c\}\in\As_\mu$ для любого $c\in\R$.
\end{df}

\begin{note}
В дальнейшем фразы <<почти всюду>>, <<почти всех>> и т.д.
будем часто заменять сокращением <<п.в.>>
\end{note}

\begin{df}
Пусть есть последовательность функций $f_1,f_2,\ldots$, определённых
$\mu$-почти всюду. Говорят, что $f_n\ra f$ $\mu$-почти всюду
(обозначения: $f_n\xra{\mu-\text{п.в.}}f$ или
$f_n\xra{\text{п.в.}}f$, если ясно, о какой мере идёт речь), если
$f(x)=\lim\limits_{n\ra\infty}f_n(x)$ для $\mu$-почти всех $x$.
\end{df}
По доказанному ранее (теорема~\ref{measureable}, п. 5)), если
функции $f_n$ $\mu$-измеримы и $f_n\ra f$ $\mu$-п.в., то функция $f$
также $\mu$-измерима.


\subsubsection{Сходимость по мере}

Пусть $f_1,f_2,\ldots$ и $f$ --- $\mu$-измеримые функции.
Последовательность $\{f_n\}_{n=1}^\infty$  называется
\emph{фундаментальной по мере}, если для всяких $c, \ve>0$
существует $n_\ve\in\N$, такое что $\mu(\{x:|f_n(x)-f_m(x)|\ge
c\})<\ve$ для всех $n,m>n_\ve$. Другими словами, для каждого $c>0$
имеем $\mu(\{x:|f_n(x)-f_m(x)|\ge c\})\ra0$ при $n,m\ra\infty$.

\begin{df}
Последовательность функций $f_1,f_2,\ldots$ сходится к функции $f$
\emph{по мере} (обозначение: $f_n\xra{\mu}f$), если для каждого
$c>0$ имеем $\mu(\{x:|f_n(x)-f(x)|\ge c\})\ra0$ при $n\ra\infty$.
\end{df}
Зададимся вопросом: как эта сходимость влияет на остальные виды
сходимости?

\begin{prop}\label{converge}
Пусть $\{f_n\}$ --- последовательность функций. Тогда:

I. Если $f_n\xra{\mu}f$ при $n\ra\infty$, то последовательность
$\{f_n\}$ фундаментальна по мере.

II. Если $f_n\xra{\text{п.в.}}f$, то $f_n\xra{\mu}f$.
\end{prop}

\begin{proof}

I. Пусть $c>0$. Имеем $\{x:|f_n(x)-f_m(x)|\ge c\}\bw \subset
\{x:|f_n(x)-f(x)|\ge \frac{c}{2}\} \cup \{x:|f_m(x)-f(x)|\ge
\frac{c}{2}\}$, так как $|f_n(x)-f_m(x)|\bw
\le|f_n(x)-f(x)|+|f_m(x)-f(x)|$. Отсюда всё и следует.

II. Пусть $c>0$. Рассмотрим при $N\in\N$ множества
$A_N=\{x:|f_n(x)-f(x)|<c\;\text{при любом}\; n\ge N\}$. Имеем
$A_N\subset A_{N+1}$ и, с учётом условия, $X=\cupnui A_N$ с
точностью до множества меры нуль, поэтому $\mu(A_N)\ra \mu(X)$ при
$N\ra\infty$. Поэтому $\mu(X\backslash A_N)\ra0$ при $N\ra\infty$,
откуда $\mu(\{x:|f_n(x)-f(x)|\ge c\})\ra0$ при $n\ra\infty$.
\end{proof}

\textbf{Примеры.}

\pt{1}. Пусть $f_n=\frac1n$ на $[0,1]$, тогда $f_n\ra0$ поточечно и
по мере. Имеем $\la(\{x:|f_n(x)|>0\})=1$ при любом $n\in\N$. Таким
образом, в определении сходимости по мере условие $c>0$ существенно.

\pt{2}. Построим последовательность $\{f_n\}$ на отрезке $[0,1]$ с
мерой Лебега $\la$, не сходящуюся ни в одной точке, но сходящуюся по
мере к нулю. Положим $f_0\equiv1$. На $n$-м шаге разобьём отрезок
$[0,1]$ на $2^n$ равных отрезков $I_1,\ldots, I_{2^n}$ и при
$k=1,\ldots,2^n$ положим \equ{f_{n,k}(x)= \case{1, & x\in I_k, \\0,
& x\notin I_k.}} Затем расположим полученные функции $f_{n,k}$ в
одну последовательность $f_0, f_{11}, f_{12}, f_{21}, f_{22},
f_{23}, f_{24}, f_{31}, \ldots$. Эта последовательность не сходится
ни в одной точке, однако она сходится по мере к нулю.

\subsubsection{Теорема Рисса}

Частичное обращение импликации <<сходимость п.в.>>, $\Ra$ <<сходимость по мере>>
даёт теорема Рисса.

\begin{theorem}\label{Riss}

I. (Рисс) Пусть $f_n\xra{\mu}f$. Тогда существует
подпоследовательность $\{f_{n_k}\}$, такая что
$f_{n_k}\xra{\text{п.в.}}f$ при $k\ra\infty$.

II. Если последовательность $\{f_n\}$ фундаментальна по мере, то
существует $\mu$-измеримая функция~$f$, такая что $f_n\xra{\mu}f$.
\end{theorem}

\begin{proof}
Пусть последовательность $\{f_n\}$ фундаментальна по мере (в I это
следует из условия, в II это дано). Тогда для любого $k\in\N$
существует $N_k\in\N$, такое что
$\mu(\{x:|f_n(x)-f_m(x)|\ge2^{-k}\})<2^{-k}$ при всех $n,m>N_k$.
Ясно, что можно считать $N_k<N_{k+1}$. Теперь покажем, что
последовательность $\{f_{N_k}\}$ сходится п.в., т.е. фундаментальна
п.в. Действительно, пусть $E_k\bw
=\{x:|f_{N_k}(x)-f_{N_{k+1}}(x)|\bw \ge\frac1{2^k}\}$. Тогда
$\mu(E_k\cup E_{k+1}\cup\ldots) \bw\le\frac1{2^k} +\frac1{2^{k+1}}
+\ldots \bw\le\frac2{2^k}\ra0$ при $k\ra\infty$. Поэтому $\mu(M)$=0,
где $M\bw =\capkui\bigcup\limits_{m=k}^\infty E_m$. Если $x\notin
M$, то $x\bw \notin\bigcup\limits_{m=k}^\infty E_m$ при некотором
$k$. Значит, при любом $m\ge k$ выполнено неравенство
$|f_{N_m}(x)\bw -f_{N_{m+1}}(x)|\bw <\frac1{2^m}$. Отсюда получаем,
что при любых $i\ge j\ge k$ справедливо неравенство
$|f_{N_i}(x)-f_{N_j}(x)|<\frac1{2^{j-1}}$, откуда и следует
фундаментальность последовательности $\{f_{N_k}(x)\}$ при $x\notin
M$. Тем самым доказано~I. Теперь докажем~II. Положим $f(x)\bw
=\lim\limits_{k\ra\infty}f_{N_k}(x)$ при $x\notin M$. Тогда функция
$f(x)$ является $\mu$-измеримой как предел сходящейся почти всюду
последовательности измеримых функций (теорема~\ref{measureable},
п.5)). Далее, $f_{N_k}\xra{\mu}f$ из предложения~\ref{converge}, II.
Выведем теперь, что $f_n\xra{\mu}f$. Зафиксируем $c>0$ и $\ve>0$.
Тогда по условию существует число $N\in\N$, такое что для всех
$m,n>N$ выполнено неравенство $\mu(\{x:|f_m(x)\bw -f_n(x)|\bw \ge
\frac{c}2\})\bw <\frac{\ve}2$. Можно также считать, что при всех
$N_k>N$ выполнено неравенство $\mu(\{x:|f_{N_k}(x)\bw -f(x)|\bw \ge
\frac{c}2\})\bw <\frac{\ve}2$. Тогда при всех $n>N$ имеем
$\mu(\{x:|f_n(x)\bw -f(x)|\ge c\})\bw \le\mu(\{x:|f_n(x)\bw
-f_{N_k}(x)|\bw \ge \frac{c}2\}\bw \cup\{x:|f_n(x)\bw -f(x)|\ge
\frac{c}2\})\bw \le\mu(\{x:|f_n(x)\bw -f_{N_k}(x)|\bw \ge
\frac{c}2\})\bw +\mu(\{x:|f_n(x)-f(x)|\bw \ge \frac{c}2\})\bw \le
\frac{\ve}2+\frac{\ve}2=\ve$. Это и означает, что $f_n\xra{\mu}f$
при $n\ra\infty$.
\end{proof}

\begin{problem}
Доказать, что сходимость по мере можно задать метрикой, т.е.
существует метрика $\rho$ на множестве классов измеримых функций (по
отношению эквивалентности, задаваемому равенством почти всюду),
такая что $f_n\xra{\mu}f\, \Lra\, \rho(f,f_n)\ra0$.
\end{problem}

\begin{problem}
Сходимость почти всюду нельзя задать метрикой даже на множестве
непрерывных функций.
\end{problem}

\begin{note}
Эту сходимость нельзя задать даже топологией.
\end{note}

\subsubsection{Теорема Егорова}

\begin{theorem}[Егоров]
Пусть $\mu(X)\bw <\infty$ и последовательность $\{f_n\}$
$\mu$-измеримых функций сходится почти всюду на $X$ к функции $f$.
Тогда для всякого $\ve>0$ существует измеримое множество
$X_\ve\subset X$, такое что $\mu(X\backslash X_\ve)<\ve$ и $f_n\rra
f$ на $X_\ve$.
\end{theorem}

\begin{proof}
Функция $f$ является $\mu$-измеримой по теореме~\ref{measureable},
п. 5). Рассмотрим множества $X_n^m\bw =\bigcap\limits_{i\ge
n}\{x:|f_i(x)\bw -f(x)|<\frac1m\}$. Имеем $X_1^m\subset X_2^m\subset
\ldots$ при любом $m$, и все множества $X_n^m$ являются
$\mu$-измеримыми. Положим $X^m=\cupnui X_n^m$. Для любого $m$
существует такое $k(m)$, что $\mu(X^m\backslash
X_{k(m)}^m)<\frac{\ve}{2^m}$ (это следует из счётной аддитивности
меры $\mu$). Положим $X_\ve\bw =\bigcap\limits_{m=1}^\infty
X_{k(m)}^m$. Покажем, что это и есть искомое множество. Имеем
$\mu(X\backslash X_\ve)\bw =\mu(X\backslash
\bigcap\limits_{m=1}^\infty X_{k(m)}^m)\bw
=\mu(\bigcup\limits_{m=1}^\infty(X\backslash X_{k(m)}^m))\bw
\le\sum\limits_{m=1}^\infty\mu(X\backslash X_{k(m)}^m)$. Заметим,
что $\mu(X\backslash X^m)=0$ при любом $m$, ибо из сходимости почти
всюду следует, что для почти всех $x\in X$ существует $n=n(x)$,
такое что при всех $i\ge n$ выполнено неравенство
$|f_i(x)-f(x)|<\frac1m$, поэтому $x\in X_n^m\subset X^m$. Итак,
множества $X$ и $X^m$ отличаются на множество меры нуль при любом
$m$, поэтому множества $X\backslash X_{k(m)}^m$ и $X^m\backslash
X_{k(m)}^m$ также отличаются на множество меры нуль. Поэтому
$\mu(X\backslash X_\ve)\bw
\le\sum\limits_{m=1}^\infty\mu(X\backslash X_{k(m)}^m)\bw =
\sum\limits_{m=1}^\infty\mu(X^m\backslash X_{k(m)}^m)\bw
<\sum\limits_{m=1}^\infty\ve\cdot 2^{-m}=\ve$. Теперь проверим, что
$f_n\convu{X_\ve}f$. Действительно, при $x\in X_\ve$ для любого $m$
получаем $|f_i(x)-f(x)|<\frac1m$ при всех $i\ge k(m)$, ибо
$X_\ve\subset X_{k(m)}^m$. Это и означает равномерную сходимость
последовательности $\{f_n\}$ на $X_\ve$.
\end{proof}

\begin{note}
В теореме Егорова нельзя взять $\ve=0$. Действительно, пусть
$X=(0,1)$ с мерой Лебега. Положим $f_n(x)=x^n$. Тогда $f_n\ra0$ на
$(0,1)$ неравномерно, и потому нет множества $Z$ меры нуль, такого
что $x^n\rra0$ на $(0,1)\backslash Z$.
\end{note}

\subsubsection{Теорема Лузина}

\begin{theorem}[Лузин]
Пусть $f$ --- измеримая функция на отрезке $[0,1]$ с мерой Лебега.
Тогда для всякого $\ve>0$ существует функция $f_\ve\in C[0,1]$,
такая что $\la(\{x:f\ne f_\ve\})<\ve$.
\end{theorem}

\begin{proof}
Заметим, что достаточно доказать следующее: для любого $\ve>0$
существует компакт $K_\ve$, такой что $\la([0,1]\backslash
K_\ve)<\ve$ и функция $f\left|_{K_\ve}\right.$ непрерывна. Если это
сделано, то сужение функции $f$ на $K_\ve$ можно доопределить по
непрерывности. Множество $[0,1]\backslash K_\ve$ открыто и потому
есть объединение попарно непересекающихся интервалов:
$[0,1]\backslash K_\ve=\bigsqcup\limits_{n=1}^\infty(\al_n,\be_n)$.
На каждом интервале $(\al_n,\be_n)$ возьмём линейную интерполяцию
нашей функции $f$, и этого будет достаточно. Далее, можно считать,
что функция $f$ ограниченна, так как иначе вместо $f$ можно
рассмотреть функцию $\widetilde f=\arctg f$. Итак, пусть $|f|\le c$.
Заметим, что существует последовательность $\{f_n\}$ измеримых
функций с конечными множествами значений, равномерно сходящаяся к
$f$. Действительно, делим отрезок $[-c,c]$ на $k_n$ равных частей
длины $\le\frac1n$ и положим $f_n$ равным середине $j$-го промежутка
$I_j$ на множестве $f^{-1}(I_j)$. Тогда имеем $|f-f_n|\le\frac1n$.
Ясно, что множества $A_j=f^{-1}(I_j)$ измеримы и дизъюнктны. В них
впишем компакты $S_j\subset A_j$, такие что $\la(A_j\backslash
S_j)\bw <\frac{\ve2^{-n}}{k_n}$. Тогда
$\la([0,1]\backslash\bigcup\limits_{j=1}^{k_n} S_j)<\ve2^{-n}$. На
множестве $K_n=\bigcup\limits_{j=1}^{k_n}S_j$ функция $f_n$
непрерывна. Пусть $K_\ve=\capnui K_n$, тогда $\la([0,1]\backslash
K_\ve)\le\sumnui\ve2^{-n}=\ve$. При этом имеем $f_n\in C(K_\ve)$ при
любом $n$ и $f_n\rra f$ на $K_\ve$, поэтому функция
$\left.f\right|_{K_\ve}$ непрерывна.
\end{proof}

\begin{note}
Хотя функция $f\left|_{K_\ve}\right.$ непрерывна, $f$ может не иметь
точек непрерывности на $[0,1]$. Пример тому --- функция Дирихле
$D(x)$ на $[0,1]$.
\end{note}


\begin{problem}
Доказать аналог теоремы Лузина для произвольной борелевской меры на
$[0,1]$.
\end{problem}

%\vspace{0.25em}


\begin{imp}[вытекающее из доказательства теоремы Лузина]\label{bounded_function}
Пусть $f$ --- ограниченная $\As$-измеримая функция на измеримом
пространстве $(X,\As)$. Тогда существует последовательность
$\{f_n\}$ $\As$-измеримых функций с конечными множествами значений,
равномерно сходящаяся к $f$.
\end{imp}
%\vspace{0.25em}

\begin{proof}
Пусть $|f|<c$. Как и в доказательстве теоремы Лузина, делим отрезок
$[-c,c]$ на промежутки $I_1^n,\ldots, I_{k_n}^n$ длины менее
$\frac1n$. Положим $A_j^n=f^{-1}(I_j^n)$ и
$\left.f_n\right|_{A_j^n}=c_j^n$, где $c_j^n$ --- середина
промежутка $I_j^n$. Тогда $|f-f_n|<\frac1n$, т.е. $f_n\rra f$ при
$n\ra\infty$.
\end{proof}

\begin{note}
Вообще говоря, функции $f_n$ не являются ступенчатыми, потому что
прообразы промежутков $I_j^n$ могут быть плохими.
\end{note}

\subsubsection{Связь $\mu$-измеримых функций с $\As$-измеримыми}

\begin{prop}\label{A_mu_measurable}
Пусть $(X,\As,\mu)$ --- пространство с мерой, $f$ ---
$\mu$-измеримая функция на $X$. Тогда существует $\As$-измеримая
функция $g$, такая что $f=g$ почти всюду на $X$. В частности, всякая
измеримая по Лебегу функция на $[0,1]$ почти всюду равна некоторой
борелевской функции.
\end{prop}

\begin{proof}
На множестве меры нуль, где функция $f$ не была определена или была
бесконечна, мы определим её нулём, поэтому без ограничения общности
можно считать, что $f$ всюду определена и конечна. Перейдя к $\arctg
f$, можно считать, что $f$ ограниченна. В силу предыдущего факта
существует последовательность $\As_\mu$-измеримых функций $f_n$ с
конечными множествами значений, равномерно сходящаяся к $f$. Пусть
$f_n\bw =c_1,\ldots,c_{k(n)}$ на дизъюнктных множествах
$A_1,\ldots,A_{k(n)}\in\As_\mu$. Тогда существуют множества
$B_j\subset A_j$, такие что $B_j\in\As$ и $\mu(B_j)=\mu(A_j)$ (см.
предложение~\ref{close_set_from_A}). Положим $g_n=c_j$ на $B_j$ при
всех $j=1,\ldots,k(n)$, а во всех остальных точках положим $g_n=0$.
Тогда имеем $g_n=f_n$ почти всюду и, кроме того, функции $g_n$
$\As$-измеримы. При этом $g_n(x)\ra f(x)$ на множестве
$B=\capnui(B_1\cup\ldots\cup B_{k(n)})$. Имеем $\mu(B)=\mu(X)$, ибо
$\mu(X\backslash (B_1\cup\ldots\cup B_{k(n)}))=0$. Тогда функция $f$
является $\As$-измеримой на множестве $B$ как поточечный предел
$\As$-измеримых функций. Полагая $g=f$ на $B$ и $g=0$ вне $B$,
получаем, что $f=g$ п.в. и функция $g$ $\As$-измерима.
\end{proof}


\section{Интеграл Лебега}

\subsection{Определение Интеграла Лебега}
\subsubsection{Простые функции}

Пусть $(X,\As,\mu)$ --- пространство с конечной неотрицательной
мерой.

Напомним, что индикатором множества $A\subset X$ называется функция
$\chi_A: X \to \R^1$, такая что $\chi_A(x)\bw =\case{1, &
\text{если}\; x\in A;\\0, & \text{если}\; x\notin A.}$

\begin{df}
\emph{Простой функцией} на $X$ называется $\As$-измеримая функция с
конечным числом значений. Она имеет вид
$f=\sum\limits_{i=1}^nc_i\chi_{A_i}$, где $A_i\in\As$, а $c_i\in\R$.
\end{df}

\begin{note}
Без ограничения общности можно считать, что множества $A_i$
дизъюнктны.
\end{note}

\begin{df}
Если $f$ --- простая функция вида $f=
\sum\limits_{i=1}^nc_i\chi_{A_i}$, то её \emph{интегралом Лебега по
пространству $X$} называется величина $\int\limits_X f\,
d\mu=\sum\limits_{i=1}^nc_i\mu(A_i)$ (иногда пишут просто $\int f\,
d\mu$).
\end{df}

\begin{note}
Это определение корректно в силу аддитивности меры $\mu$, т.е. не
зависит от представления $f$ указанным способом. Действительно,
пусть $f=\sum\limits_{j=1}^md_j\chi_{B_j}$. Тогда \equ{\int\limits_X
f\,d \mu=\sum\limits_{j=1}^md_j\mu(B_j)\bw
=\sum\limits_{j=1}^md_j\mu(\bigsqcup\limits_{i=1}^n(B_j\bw\cap
A_i))\bw =\sum\limits_{j=1}^md_j\sum\limits_{i=1}^n\mu(B_j\cap
A_i)\bw=\sum\limits_{i=1}^n\sum\limits_{j=1}^mc_i\mu(A_i\cap
B_j)=\sum\limits_{i=1}^nc_i\mu(A_i),} поскольку $c_i=d_j$ на
множестве $A_i\cap B_j$.
\end{note}


\subsubsection{Свойства интеграла на простых функциях}

\begin{prop}
Пусть $f,g$ --- простые функции. Тогда:

1) если $f\ge0$, то $\int f\,d\mu\ge0$;

2) (линейность) $\int(\al f+\be g)\,d\mu=\al\int f\,d\mu+\be\int
g\,d\mu$ при любых $\al,\be\in\R$;

3) $\left|\int f\,d\mu\right|\le\int|f|\,d\mu\le\sup f\cdot\mu(X)$.
\end{prop}

\begin{proof}
1) Ясно из определения.

2) Из определения ясно, что постоянный множитель можно вынести за
знак интеграла. Поэтому осталось доказать, что $\int(f+g)\,d\mu
\bw=\int f\,d\mu+\int g\,d\mu$. Во-первых, покажем, что функция
$f+g$ тоже простая. Действительно, пусть функция $f$ принимает
значения $a_1,\ldots,a_n$ на дизъюнктных множествах
$A_1,\ldots,A_n$, функция $g$ принимает значения $b_1,\ldots,b_m$ на
дизъюнктных множествах $B_1,\ldots B_m$. Тогда функция $f+g$
принимает значение $a_i+b_j$ на множестве $A_i\cap B_j$. По
определению, $\int(f+g)\,d\mu=\sum\limits_{i,j}(a_i+b_j)\mu(A_i\cap
B_j)$, а при фиксированном $i$ имеем
$\sum\limits_{j=1}^ma_i\mu(A_i\cap B_j)\bw =a_i\mu(A_i)$ в силу
условия $X=B_1\sqcup\ldots\sqcup B_m$. Аналогично, при фиксированном
$j$ имеем $\sum\limits_{i=1}^nb_j\mu(A_i\bw \cap B_j)=b_j\mu(B_j)$.
Отсюда $\sum\limits_{i,j}(a_i+b_j)\mu(A_i\bw \cap B_j)\bw
=\sum\limits_{i=1}^n \sum\limits_{j=1}^ma_i\mu(A_i\bw \cap B_j)\bw
+\sum\limits_{j=1}^m \sum\limits_{i=1}^nb_j\mu(A_i\bw \cap B_j)\bw
=\sum\limits_{i=1}^na_i\mu(A_i)\bw
+\sum\limits_{j=1}^mb_j\mu(B_j)\bw =\int f\,d\mu+\int g\,d\mu$.

3) Очевидно, что функция $|f|$ простая. Имеем $|\int f\,d\mu|\bw\
=|\sum\limits_{i=1}^na_i\mu(A_i)|\bw
\le\sum\limits_{i=1}^n|a_i|\mu(A_i)\bw =\int|f|\,d\mu$. Далее,
$\int|f|\,d\mu\bw =\sum\limits_{i=1}^n|a_i|\mu(A_i)\bw
\le\max\limits_i|a_i|\sum\limits_{i=1}^n\mu(A_i)=\max f\cdot
\mu(X)$.
\end{proof}

\begin{df}
Последовательность $\{f_n\}$ называется \emph{фундаментальной в
среднем}, если для всякого $\ve>0$ существует $N\in\N$, такое что
$\int|f_i-f_j|\,d\mu<\ve$ для всех $i,j\ge N$.
\end{df}

\begin{note}
Если последовательность $\{f_j\}$ фундаментальна в среднем, то
последовательность $\left\{\int f_j\,d\mu\right\}$ сходится,
поскольку для неё выполнен критерий Коши.
\end{note}

\begin{lemma}\label{fund_average}
Пусть последовательность $\{f_n\}$ простых функций фундаментальна в
среднем. Тогда для всякого $\ve>0$ существует $\de>0$, такое что
$\int\limits_A|f_n|\,d\mu<\ve$ для любого множества $A\in\As$ с
$\mu(A)<\de$ и любого $n$. Здесь $\int\limits_A
f\,d\mu:=\int\limits_X f\cdot\chi_A\,d\mu$, $A\in\As$ (если $f$
простая, то и $f\cdot \chi_A$ тоже простая).
\end{lemma}

\begin{proof}
Пусть задано произвольное $\ve>0$. Берём $N\in\N$, такое что
$\int\limits_X|f_n-f_k|\, d\mu<\frac{\ve}2$ для любых $n,k\ge N$.
Возьмём $C=\max\{|f_1|,\ldots,|f_N|\}+1$. Пусть множество $A$
таково, что $\mu(A)<\de$, где $\de=\frac{\ve}{2C}$, и $n\ge N$.
Имеем $\int\limits_A|f_n|\,d\mu\bw
\le\int\limits_A|f_n-f_N|\,d\mu\bw +\int\limits_A|f_N|\,d\mu\bw \le
\int\limits_X|f_n-f_N|\,d\mu\bw
+\max|f_N|\cdot\int\limits_X\chi_A\,d\mu\bw
<\frac{\ve}2+\frac{\ve}{2C}\cdot C=\ve$, что и требовалось.
\end{proof}

\begin{imp}
В условиях предыдущей леммы имеем $\int\limits_A|f_n|\,d\mu\bw
=\int\limits_{A\cap A^+}f_n\,d\mu\bw -\int\limits_{A\cap
A^-}f_n\,d\mu\bw <\ve$, где $A^+\bw =\{x: f_n(x)\ge0\}$, $A^-\bw
=\{x: f_n(x)<0\}$.
\end{imp}


\subsubsection{Общее определение интеграла Лебега}

Пусть $(X,\As,\mu)$ --- пространство с конечной неотрицательной
мерой. Пусть $f$ --- $\mu$-измеримая функция, т.е. область
определения $D(f)$ функции $f$ содержит множество $X_0$, такое что
$\mu(X\backslash X_0)=0$, и функция $\left.f\right|_{X_0}$ измерима
относительно $\As_\mu\cap X_0$ (вне $X_0$ функция $f$ может
принимать какие угодно значения).

\begin{df}
Функция $f$ называется \emph{интегрируемой по Лебегу по мере $\mu$
($\mu$-интегрируемой)}, если существует последовательность $\{f_n\}$
простых функций, которая фундаментальна в среднем и почти всюду
сходится к $f$. \emph{Интегралом (Лебега)} функции $f$ называется
величина $\int\limits_Xf\,d\mu\bw
=\lim\limits_{n\ra\infty}\int\limits_Xf_n\,d\mu$.
\end{df}

Множество $\mu$-интегрируемых функций будем обозначать через
$\Ls^1(\mu)$ или через $\Ls^1(X)$, когда ясно, о какой мере идёт
речь.

Покажем, что определение интеграла корректно, т.е. другого предела
нет и быть не может ($\int\limits_Xf\,d\mu$ определён однозначно). В
самом деле, пусть $\{g_n\}$ --- другая фундаментальная в среднем
последовательность простых функций, почти всюду сходящаяся к $f$.
Докажем, что $\left|\int\limits_Xf_n\,d\mu\bw
-\int\limits_Xg_n\,d\mu\right|\ra0$ при $n\ra\infty$. Пусть $\ve>0$
--- произвольное число. Тогда по лемме~\ref{fund_average} существует
$\de>0$, такое что $\left|\int\limits_Af_n\,d\mu\right|\bw
<\frac{\ve}4$, $\left|\int\limits_Ag_n\,d\mu\right|\bw<\frac{\ve}4$
для любого множества $A$ c $\mu(A)<\de$. По теореме Егорова
существует множество $A$ с мерой меньше $\de$, такое что
$\left.f_n\right|_{X\backslash A}\bw \rra
\left.f\right|_{X\backslash A}$ и $\left.g_n\right|_{X\backslash
A}\rra \left.f\right|_{X\backslash A}$. Тогда существует $N\in\N$,
такое что при любом $n\ge N$ справедливо неравенство
$\sup\limits_{x\in X\backslash A}|f_n-g_n|\bw
<\frac{\ve}{4(\mu(X)+1)}$. Отсюда при $n\ge N$ имеем
$\left|\int\limits_X(f_n-g_n)\,d\mu\right|\bw \le
\left|\int\limits_A(f_n-g_n)\,d\mu\right|\bw
+\left|\int\limits_{X\backslash A}(f_n-g_n)\,d\mu\right|\bw
\le\left|\int\limits_Af_n\,d\mu\right|\bw
+\left|\int\limits_Ag_n\,d\mu\right|\bw +\int\limits_{X\backslash
A}|f_n-g_n|\,d\mu\bw
<\frac{\ve}4+\frac{\ve}4+\frac{\ve\mu(X\backslash
A)}{4(\mu(X)+1)}\bw <\ve$. Таким образом,
$\lim\limits_{n\ra\infty}\int\limits_Xf_n\,d\mu\bw
=\lim\limits_{n\ra\infty}\int\limits_Xg_n\,d\mu$ и тем самым
интеграл Лебега функции $f$ определён корректно.

\begin{note}
Пусть $A\in\As$ и функция $f$ $\mu$-измерима. Если функция
$f\cdot\chi_A$ $\mu$-интегрируема, то будем говорить, что функция
$f$ $\mu$-интегрируема на множестве $A$, и писать $f\in
\Ls^1(A,\mu)$. Положим $\int\limits_Af\,d\mu\bw
=\int\limits_Xf\cdot\chi_A\,d\mu$.
\end{note}

\begin{ex}
Всякая ограниченная $\As$-измеримая функция $f$ является
$\mu$-интегрируемой. Действительно, было доказано
(следствие~\ref{bounded_function}), что $f$ есть равномерный предел
последовательности простых функций, а такая последовательность
фундаментальна в среднем. В частности, любая ограниченная
борелевская функция (или хотя бы ограниченная и измеримая по Лебегу
функция) на отрезке интегрируема относительно меры Лебега.
\end{ex}

Простейшие свойства общего интеграла Лебега даёт следующая

\begin{theorem}\label{int_properties}

(1) Если $f\in \Ls^1(\mu)$ и $f\ge0$, то $\int f\,d\mu\ge0$.

(2) Если $f,g\in \Ls^1(\mu)$, то при любых $\al,\be\in\R$ имеем $\al
f+\be g\in \Ls^1(\mu)$, причём $\int(\al f+\be g)\,d\mu\bw =\al\int
f\,d\mu\bw +\be\int g\,d\mu$.

(3) Если $f\in \Ls^1(\mu)$, то $|f|\in \Ls^1(\mu)$ и $\left|\int
f\,d\mu\right|\le\int|f|\,d\mu$.

(4) Если $f$ --- ограниченная $\As$-измеримая функция, то $f\in
\Ls^1(\mu)$ и $\left|\int\limits_Xf\,d\mu\right|\bw \le
\sup|f|\cdot\mu(X)$.
\end{theorem}

\begin{proof}
(1) По определению интегрируемости существует последовательность
$\{f_n\}$ простых функций, сходящаяся к $f$ почти всюду на $X$ и
фундаментальная в среднем. Тогда все функции $|f_n|$ простые и
$|f_n|\bw{ \mathbin{\xra{\text{п.в.}}}} |f|=f$. При этом
последовательность $\{|f_n|\}$ фундаментальна в среднем.
Действительно, справедливо неравенство $\bigl||f_n|-|f_k|\bigr|\bw
\le |f_n-f_k|$, откуда $\int\bigl||f_n|-|f_k|\bigr|\,d\mu\ra0$.
Отсюда $\int|f_n|\,d\mu\ra\int f\,d\mu\bw\ge0$, ибо
$\int|f_n|\,d\mu\ge0$.

(3) В этом случае доказательство аналогично случаю (1). Если
последовательность $\{f_n\}$ интегрируемых простых функций сходится
почти всюду к $f$, причём $\{f_n\}$ фундаментальна в среднем, то
$|f_n|\xra{\text{п.в.}}|f|$, и тогда функции $|f_n|$ простые и
последовательность $\{|f_n|\}$ фундаментальна в среднем. Значит,
функция $|f|$ интегрируема и
$\int|f|\,d\mu=\lim\limits_{n\ra\infty}\int|f_n|\,d\mu\bw
\ge\lim\limits_{n\ra\infty}\int f_n\,d\mu=\int f\,d\mu$. Аналогично
получаем $\int|f|\,d\mu\bw \ge\int f\,d\mu$.

(2) Очевидно, что константу можно выносить из-под знака интеграла.
Осталось доказать требуемое при $\al=\be=1$. Далее, пусть
$f_n\xra{\text{п.в.}}f$ и $g_n\xra{\text{п.в.}}g$, где обе
последовательности $\{f_n\}$ и $\{g_n\}$ состоят из простых функций
и фундаментальны в среднем. Все функции $f_n+g_n$ являются простыми,
причём $f_n+g_n\xra{\text{п.в.}}f+g$. Последовательность
$\{f_n+g_n\}$ фундаментальна в среднем, так как
$\int|f_n+g_n-f_k-g_k|\,d\mu\bw \le\int|f_n-f_k|\,d\mu\bw
+\int|g_n-g_k|\,d\mu$. Отсюда, так как $\int(f_n+g_n)\,d\mu\bw =\int
f_n\,d\mu\bw +\int g_n\,d\mu$, имеем $\int f\,d\mu\bw +\int
g\,d\mu\bw =\int(f+g)\,d\mu$.

(4) Интегрируемость функции $f$ уже пояснялась, а поскольку
$|f|-\sup|f|\bw \le0$, имеем $\int(|f|\bw -\sup|f|)\,d\mu\bw \le0$,
откуда $\int|f|\,d\mu\bw \le\int\sup|f|\,d\mu=\sup|f|\cdot\mu(X)$.
\end{proof}

\begin{imp}[признак сравнения]\label{compar_intable}
Пусть $f,g$ --- $\mu$-измеримые функции, причём $|f-g|\le C$ п.в. на
$X$. Тогда функции $f$ и $g$ $\mu$-интегрируемы либо
$\mu$-неинтегрируемы одновременно.
\end{imp}

\begin{proof}
Функция $f-g$ является $\mu$-измеримой, поэтому по
предложению~\ref{A_mu_measurable} существует $\As$-измеримая функция
$h$, такая что $f-g=h$ п.в. на $X$, причём $h$ также можно считать
ограниченной (переопределив $h$ нулём на множестве $\{x: |h(x)|\ge
C\}\in\As$). По п. (4) предыдущей теоремы функция $h$ является
$\mu$-интегрируемой. Тогда существует фундаментальная в среднем
последовательность $\{h_n\}$ простых функций, сходящаяся п.в. к $h$.
Но тогда $h_n\xra{\text{п.в.}}f-g$, ибо $h\bw= f-g$ п.в., поэтому
функция $f-g$ является $\mu$-интегрируемой. Теперь доказываемое
утверждение следует из п. (2) предыдущей теоремы, так как $f= g+
(f-g)$, $g= f-(f-g)$.
\end{proof}

Пусть $A$ --- $\mu$-измеримое множество и $f$ ---
$\mu$-интегрируемая функция. Тогда функция $\chi_A\cdot f$ тоже
$\mu$-интегрируема. Покажем это. Во-первых, по
предложению~\ref{close_set_from_A} существует множество $A_0\in\As$,
такое что $\chi_A(x)=\chi_{A_0}(x)$ п.в. Во-вторых, если
$\{f_n\}$~--- фундаментальная в среднем последовательность простых
функций, почти всюду сходящаяся к $f$, то функция $\chi_A\cdot f$
есть п.в. предел простых функций $\chi_{A_0}\cdot f_n$. При этом
последовательность $\{\chi_{A_0}\cdot f\}$ фундаментальна в среднем,
так как $|\chi_{A_0}\cdot f_n-\chi_{A_0}\cdot f_k|\bw \le|f_n-f_k|$.
Значит, по определению функция $\chi_A\cdot f$ $\mu$-интегрируема.
Тогда положим $\int\limits_A f\,d\mu:=\int\limits_X\chi_A\cdot
f\,d\mu$.

Из определения ясно, что если $A\cap B=\varnothing$, то
$\int\limits_Af\,d\mu+\int\limits_Bf\,d\mu\bw =\int\limits_{A\sqcup
B}f\,d\mu$.

\begin{ex}\label{series_int}
Пусть множества $A_n$ образуют покрытие множества $X$ и дизъюнктны.
Пусть $f(x)\bw =c_k$ при $x\in A_k$. Тогда функция $f$ является
$\As$-измеримой. При этом $f$ $\mu$-интегрируема тогда и только
тогда, когда $\sumnui|c_n|\mu(A_n)\bw <\infty$. В этом случае $\int
f\,d\mu=\sumnui c_n\mu(A_n)$.
\end{ex}

\begin{proof}
Если $f\in \Ls^1(\mu)$, то пусть \equ{f_n(x)=\case{f(x)
&\text{при}\;x\in\bigcup\limits_{k=1}^nA_k,\\ 0 & \text{иначе.}}}
Тогда $|f_n|\bw \le|f|$ и поэтому $\int|f_n|\,d\mu\bw
\le\int|f|\,d\mu$ при любом $n$, но
$\int|f_n|=\sum\limits_{k=1}^n|c_k|\mu(A_k)$, значит,
$\sumkui|c_k|\mu(A_k)\bw \le\int|f|\,d\mu$.

Обратно, если $\sumkui|c_k|\mu(A_k)\bw <\infty$, то
последовательность $\{f_n\}$ фундаментальна в среднем, причём она
сходится к $f$ почти всюду. Все функции $f_n$ являются простыми и
интегрируемы, при этом $\int f_n\bw
=\sum\limits_{k=1}^nc_k\mu(A_k)$. Отсюда следует, что $f\in
\Ls^1(\mu)$ и $\int f\,d\mu\bw =\sumkui c_k\mu(A_k)$.
\end{proof}


\begin{note}
Лебег так определял интеграл $\int f\,d\mu$: пусть при некотором
$\ve>0$ сходится ряд $S(\ve)\bw
=\sum\limits_{k=-\infty}^{+\infty}(\ve k\bw \times\mu(\{x:\ve k\bw
\le f\bw <\ve(k+1)\}))$, то есть сходятся по отдельности ряды при
$k>0$ и $k<0$. Тогда такой ряд сходится при любом $\ve>0$ и
существует предел $S(f)=\lim\limits_{\ve\ra0}S(\ve)$. Этот предел и
называется интегралом Лебега.

Покажем, что это то же самое, т.е. определение Лебега равносильно
нашему определению интеграла.  Рассмотрим функцию $f_\ve$, равную
$\ve k$ при тех значениях $x$, для которых $\ve k\bw \le f(x)\bw
<\ve(k+1)$. В силу примера~\ref{series_int} имеем $f_\ve\in
\Ls^1(\mu)$ $\Lra$ ряд $S(\ve)$ сходится. Так как $|f_\ve-f|\le\ve$,
то по признаку сравнения из интегрируемости функции $f$ следует
интегрируемость функции $f_\ve$ и обратно. При этом в случае
интегрируемости $\left|\int f_\ve\,d\mu\bw -\int f\,d\mu\right|\bw
\le\int|f_\ve-f|\,d\mu\bw \le\ve\cdot\mu(X)$, откуда $\int
f\,d\mu\bw =\lim\limits_{\ve\ra0}\int f_\ve\,d\mu\bw
=\lim\limits_{\ve\ra0}S(\ve)\bw =S(f)$, что и доказывает
эквивалентность двух определений интеграла.
\end{note}

\subsection{Свойства интеграла Лебега}
\subsubsection{Абсолютная непрерывность интеграла Лебега и неравенство
Чебышёва}

\begin{theorem}[абсолютная непрерывность интеграла Лебега]
Пусть $f$ --- $\mu$-интегрируемая функция. Тогда для всякого $\ve>0$
существует $\de>0$, такое что для любого множества $A\in\As_\mu$ с
условием $\mu(A)<\de$ выполнено неравенство
$\int\limits_A|f|\,d\mu\bw <\ve$.
\end{theorem}

\begin{proof}
Пусть задано произвольное $\ve>0$. По определению интегрируемости
существуют неотрицательные простые функции $f_n$, такие что
$f_n\xra{\text{п.в.}}|f|$, причём последовательность $\{f_n\}$
фундаментальна в среднем. По доказанному ранее
(лемма~\ref{fund_average}), существует $\de>0$, такое что
$\int\limits_{A_0}f_n\,d\mu\bw <\frac{\ve}2$ при условиях
$A_0\in\As$ и $\mu(A_0)<\de$ сразу для всех~$n$. Пусть теперь
$A\in\As_\mu$ --- произвольное множество с $\mu(A)<\ve$. Из
предложения~\ref{close_set_from_A} следует, что существует множество
$A_0\in\As$, такое что $A_0\subset A$ и $\mu(A_0)=\mu(A)$. Тогда
ясно, что для любой функции $h\in \Ls^1(\mu)$ имеем
$\int\limits_{A_0}h\,d\mu=\int\limits_Ah\,d\mu$. Поскольку
$\int\limits_A|f|\,d\mu=\lim\limits_{n\ra\infty}
\int\limits_Af_n\,d\mu$, получаем $\int\limits_A|f|\,d\mu\bw
\le\frac{\ve}2\bw <\ve$, что и требовалось доказать.
\end{proof}

\begin{theorem}[неравенство Чебышёва]
Пусть функция $f$ $\mu$-интегрируема. Тогда для любого $R>0$ имеем
$\mu(\{x:|f(x)|\ge R\})\bw \le\frac{\int\limits_X|f|\,d\mu}R$.
\end{theorem}

\begin{proof}
Пусть $G=\{x:|f(x)|\ge R\}$, $L=\{x:|f(x)|< R\}$. Тогда
$\int\limits_X|f|\,d\mu\bw =\int\limits_G|f|\,d\mu\bw
+\int\limits_L|f|\,d\mu\bw \ge\int\limits_G|f|\,d\mu\bw
\ge\int\limits_R\,d\mu\bw =R\cdot\mu(G)$, что и требовалось
доказать.
\end{proof}

\subsubsection{Критерий интегрируемости}

\begin{theorem}[критерий интегрируемости]\label{crit_int}
Пусть $f$~--- $\mu$-измеримая функция. Тогда $f\in \Ls^1(\mu) \bw
\Lra \sum\limits_{n=1}^{+\infty}\mu(\{x:|f(x)|\ge n\})$ ---
сходящийся ряд.
\end{theorem}

\begin{proof}
Рассмотрим функцию $g(x)$, равную $n$ на множестве $\{x:n\le
|f(x)|<n+1\}$ при любом $n\ge0$. Эта функция имеет счётное множество
значений. Кроме того, имеем $g\le|f(x)|< g+1$. Поэтому функции $g$ и
$|f(x)|$ одновременно интегрируемы или неинтегрируемы по признаку
сравнения (следствие~\ref{compar_intable}). Но $\int g\,d\mu\bw
=\sum\limits_{n=0}^{\infty}n\cdot\mu(\{x:n\le |f(x)|\le n+1\})$.
Значит, $f\in \Ls^1(\mu) \bw\Lra \sum\limits_{n=1}^\infty
n\cdot\mu(\{x:n\le |f(x)|\le n+1\})$ --- сходящийся ряд. Но
$\mu(\{x:|f(x)|\ge
n\})=\sum\limits_{k=n}^\infty\mu(\{x:k\le|f(x)|<k+1\})$, поэтому
$\sum\limits_{n=1}^{+\infty}\mu(\{x:|f(x)|\ge n\})\bw
=\sum\limits_{n=1}^\infty n\cdot\mu(\{x:n\le |f(x)|\bw < n+1\})$,
откуда всё и следует.
\end{proof}

\begin{imp}\label{crit_Weierst}
Пусть $f,g$ --- $\mu$-измеримые функции, $g\in \Ls^1(\mu)$ и $|f|\le
g$ п.в. Тогда $f\in \Ls^1(\mu)$.
\end{imp}
\begin{proof}
По критерию интегрируемости ряд
$\sum\limits_{n=1}^{+\infty}\mu(\{x:g(x)\ge n\})$ сходится. Осталось
заметить, что $\{x:|f(x)|\bw \ge n\}\subset\{x:g(x)\bw \ge n\}$ при
любом $n\in\N$, и воспользоваться критерием интегрируемости для $f$.
\end{proof}

\begin{ex}
Функция $f(x)=\ln x$ на отрезке $[0,1]$ интегрируема по Лебегу.
Действительно, $\{x:|\ln x|\ge n\}\bw =(0, e^{-n}]$, а ряд $\sumnui
e^{-n}$ сходится.
\end{ex}

\subsubsection{Предельный переход в интеграле}

\begin{theorem}[Лебег]
Пусть последовательность $\{f_n\}$ $\mu$-измеримых функций почти
всюду сходится к функции $f$. Пусть также существует
$\mu$-интегрируемая функция $\Phi$, такая что $|f_n|\le\Phi$ п.в.
при любом $n$. Тогда $f\in \Ls^1(\mu)$ и $\int f\,d\mu\bw
=\lim\limits_{n\ra\infty}\int f_n\,d\mu$. Кроме того,
$\int|f-f_n|\,d\mu\ra0$.
\end{theorem}

\begin{proof}
Из условия имеем $|f|\le\Phi$ п.в., поэтому функция $f$ является
$\mu$-интегрируемой по следствию~\ref{crit_Weierst}. Поскольку
$\left|\int f\,d\mu\bw -\int f_n\,d\mu\right|\bw
\le\int|f-f_n|\,d\mu$, осталось доказать, что
$\int|f-f_n|\,d\mu\bw\ra0$ при $n\ra\infty$. Заметим, что
$|f-f_n|\bw\le 2\Phi$ п.в. Пусть $\ve>0$ --- произвольное число. В
силу абсолютной непрерывности интеграла Лебега существует $\de>0$,
такое что $\int\limits_A\Phi\,d\mu\bw <\frac{\ve}4$ для любого
множества $A\in\As_\mu$ с $\mu(A)<\de$. По теореме Егорова
существует множество $X_\de$, такое что $\mu(X\backslash X_\de)<\de$
и $|f-f_n|\rra0$ на $X_\de$. Поэтому существует $N\in\N$, такое что
для любого $n\ge N$ выполнено неравенство $|f-f_n|\bw
\le\frac{\ve}{2(\mu(X)+1)}$ на множестве $X_\de$. Тогда
$\int\limits_X|f-f_n|\,d\mu\bw =\int\limits_{X_\de}|f-f_n|\,d\mu\bw
+\int\limits_{X\backslash X_\de}|f-f_n|\,d\mu\bw
\le\frac{\ve}{2(\mu(X)+1)}\mu(X)\bw +\int\limits_{X\backslash
X_\de}2\Phi\,d\mu\bw <\frac{\ve}2\bw +\frac{\ve}2=\ve$, откуда
следует требуемое.
\end{proof}

\begin{ex}
Пусть $f_n=n$ на $[0,\frac1n]$ и $f_n=0$ вне $[0,\frac1n]$. Тогда
$f_n\ra0$ п.в., но $\int\limits_{[0,1]}f_n\,d\mu=1\ne0$, поскольку у
функций $f_n$ нет общей мажоранты.
\end{ex}

Эту теорему называют также теоремой \emph{о мажорируемой
сходимости}.

\begin{theorem}[Беппо Леви, о монотонной сходимости]
Пусть функции $f_n$ $\mu$-интегрируемы, причём последовательность
$\{f_n(x)\}$ монотонна для почти всех $x\in X$. Пусть
$\sup\limits_n\int\limits_Xf_n\,d\mu\bw <\infty$. Тогда функция $f:
\bw=\lim\limits_{n\ra\infty}f_n$ почти всюду конечна,
$\mu$-интегрируема и $\int f\,d\mu=\lim\limits_{n\ra\infty}\int
f_n\,d\mu$.
\end{theorem}

\begin{proof}
Без ограничения общности можно считать, что последовательность
$\{f_n(x)\}$ монотонна при всех $x\in X$. Последовательность $\{\int
f_n\,d\mu\}$ возрастает и ограниченна, поэтому она фундаментальна.
Значит, последовательность $\{f_n\}$ фундаментальна в среднем, так
как при $n\ge k$ имеем $f_n\ge f_k$, откуда $\int|f_n-f_k|\,d\mu\bw
=\int f_n\,d\mu\bw -\int f_k\,d\mu\bw \ra 0$ при $n,k\ra\infty$. Из
неравенства Чебышёва следует, что последовательность $\{f_n\}$
фундаментальна по мере: $\mu(\{x:|f_n(x)-f_k(x)|>c\})\bw
\le\frac1c\int\limits_X|f_n\bw -f_k|\,d\mu\ra0$ при $n,k\ra\infty$
при любом $c>0$. По теореме Рисса (теорема~\ref{Riss}, I)
последовательность $\{f_n\}$ содержит подпоследовательность,
сходящуюся п.в. к конечной измеримой функции. Но тогда к этой
функции сходится п.в. вся последовательность $\{f_n\}$ в силу её
монотонности п.в.

Покажем, что $f\in \Ls^1(\mu)$. В самом деле, применим критерий
интегрируемости (теорема~\ref{crit_int}). Перейдя к функциям вида
$f_n-f_1$, можно считать, что $f_n\ge0$ при любом $n\in\N$. Пусть
$\int f_n\,d\mu\bw \le C$ для любого $n\in\N$. При любом
фиксированном $N$ имеем $\sum\limits_{n=1}^N\mu(\{x:f(x)> n\})\bw
=\lim\limits_{k\ra\infty}\sum\limits_{n=1}^N\mu(\{x:f_k(x)\ge n\})$,
так как $\{x:f_k(x)\ge n\}\subset\{x:f_{k+1}(x)\ge n\}$ при любом
$k$ и $\cupkui\{x:f_k(x)\ge n\}=\{x:f(x)> n\}$, т.е.
$\mu(\{x:f(x)>n\})=\lim\limits_{k\ra\infty}\mu(\{x:f_k(x)\ge n\})$.
Далее, $\sum\limits_{n=1}^N\mu(\{x:f_k(x)\ge n\})\bw
\le\sumnui\mu(\{x:f_k(x)\ge n\})\bw =\sum\limits_{n=1}^\infty
n\cdot\mu(\{x:n\le f_k(x)< n+1\})\bw \le\int f_k\,d\mu\bw\le C$.
Поэтому $\sum\limits_{n=1}^N\mu(\{x:f(x)>n\})\le C$ при любом~$N$,
поэтому сходится ряд $\sumnui\mu(\{x:f(x)>n\})$, поэтому сходится и
ряд $\sumnui\mu(\{x:f(x)\ge n\})$, а потому $f\in \Ls^1(\mu)$.
Теперь $\int f_n\,d\mu\bw \ra\int f\,d\mu$ по теореме Лебега.
\end{proof}

\begin{theorem}[Фату]
Пусть $\{f_n\}$ --- последовательность $\mu$-интегрируемых функций и
$f_n\ge0$. Предположим, что $f_n\xra{\text{п.в.}}f$ и
$\sup\limits_n\int f_n\,d\mu\bw <\infty$. Тогда $f\in \Ls^1(\mu)$ и
$\int f\,d\mu\bw \le\sup\limits_n\int f_n\,d\mu$.
\end{theorem}

\begin{proof}
Введём функции $g_n(x)=\inf\limits_{k\ge n}f_k(x)$. Тогда все они
$\mu$-измеримы по теореме~\ref{measurable}, ибо $\{x:g_n(x)<c\}\bw
=\bigcup\limits_{k\ge n}\{x:f_k(x)<c\}\in\As_\mu$ при любом
$c\in\R$. При этом $0\le g_n\le f_n$ и $g_n\le g_{n+1}$. По теореме
Б. Леви почти всюду существует функция
$g=\lim\limits_{n\ra\infty}g_n$ и она является $\mu$-интегрируемой,
причём $\int g\,d\mu\bw \le\sup\limits_n\int f_n\,d\mu$, ибо $\int
g_n\,d\mu\bw \le\int f_n\,d\mu$. Осталось заметить, что $g(x)=f(x)$
почти всюду.
\end{proof}

\subsubsection{Связь интегралов Лебега и Римана}

Если функция $f$ интегрируема по Риману на отрезке $[a,b]$, то будем
писать $f\in R[a,b]$.

\begin{theorem}
Пусть $f\in R[a,b]$, тогда $f\in \Ls^1([a,b])$ и
$\int\limits_a^bf\,dx\bw =\int f\,d\mu$.
\end{theorem}

\begin{proof} Введём обозначения $(R)\int\limits_a^bf\,dx$ и
$(L)\int\limits_{[a,b]}f\,d\mu$ для интегралов Римана и Лебега
соответственно от функции $f$ по отрезку $[a,b]$.

Без ограничения общности будем считать, что $[a,b]\bw =[0,1]$. При
любом $n\in\N$ разделим отрезок $[0,1]$ точками вида $\frac{k}{2^n}$
на $2^n$ отрезков $J_{n,1},\ldots,J_{n,2^n}$. Рассмотрим функции
(они будут ступенчатыми) $f_n$ и $g_n$, такие что
$f_n(x):=\inf\limits_{y\in J_{n,k}}f(y)$ при $x\in J_{n,k}$,
$g_n(x):=\sup\limits_{y\in J_{n,k}}f(y)$ при $x\in J_{n,k}$. Такое
определение корректно, поскольку любая интегрируемая по Риману
функция ограниченна. Последовательность
$\left\{\int\limits_0^1f_n\,dx\right\}$ возрастает и стремится к
$\left\{\int\limits_0^1f\,dx\right\}$ при $n\ra\infty$, аналогично
последовательность $\left\{\int\limits_0^1g_n\,dx\right\}$ убывает и
стремится к $\left\{\int\limits_0^1f\,dx\right\}$. При этом $f_n\bw
\le f\bw \le g_n$ и $f_n\bw \le f_{n+1}$, $g_n\bw \ge g_{n+1}$ (при
измельчении разбиения отрезка нижняя сумма Дарбу не уменьшается, а
верхняя сумма Дарбу не увеличивается). Положим
$f^*:=\lim\limits_{n\ra \infty}f_n$, $g^*:=\lim\limits_{n\ra
\infty}g_n$. Функции $f^*$ и $g^*$  измеримы и ограниченны как
поточечные пределы ступенчатых функций. По теореме Б. Леви функции
$f^*$ и $g^*$ являются $\mu$-интегрируемыми, причём
$(L)\int\limits_{[0,1]}f^*\,d\mu\bw =\lim\limits_{n\ra \infty}(L)
\int\limits_{[a,b]}f_n\,d\mu\bw =\lim\limits_{n\ra \infty}(R)
\int\limits_0^1f_n\,d\mu$ и $(L)\int\limits_{[0,1]}g^*\,d\mu\bw
=\lim\limits_{n\ra \infty}(L) \int\limits_{[0,1]}g_n\,d\mu\bw
=\lim\limits_{n\ra \infty}(R) \int\limits_0^1g_n\,d\mu$. Отсюда
$(L)\int\limits_{[0,1]}f^*\,d\mu\bw
=(L)\int\limits_{[0,1]}g^*\,d\mu\bw =(R)\int\limits_0^1f\,dx$,
поэтому $f^*=g^*$ п.в., так как $g^*-f^*\bw\ge0$ и
$\int\limits_{[0,1]}(g^*\bw -f^*)\,d\mu=0$, а по неравенству
Чебышёва $\mu(\{x:g^*(x)-f^*(x)\bw >\frac1k\})\bw \le
k\int\limits_{[0,1]}(g^*-f^*)\,d\mu$. Отсюда $f=f^*=g^*$~п.в.
\end{proof}

\begin{problem}\label{zadacha}
Доказать, что $f\in R[a,b]$ тогда и только тогда, когда $f$ ---
ограниченная функция и множество её точек разрыва имеет лебегову
меру нуль.
\end{problem}

\begin{note}
Существуют несобственно интегрируемые по Риману функции, которые не
интегрируемы по Лебегу. Пример: $f(x)=\frac1x\sin\frac1x$ на
$[0,1]$.
\end{note}

\begin{theorem}
Пусть функция $f$ задана на интервале $(a,b)$ и $f\in
R[a+\ve,b-\ve]$ при любом $\ve>0$. Тогда $|f|\in\widetilde R(a,b)
\bw \Lra f\in \Ls^1([a,b])$ {\upshape($\widetilde R(a,b)$ --- класс
функций, интегрируемых на $(a,b)$ по Риману в несобственном
смысле)}. При этом $\int\limits_a^b|f|\,dx\bw
=\int\limits_{[a,b]}|f|\,d\mu$, где интеграл слева --- несобственный
интеграл Римана.
\end{theorem}

\begin{proof}
Если $|f|\in\widetilde R(a,b)$, то существует предел
$\lim\limits_{n\ra\infty}\int\limits_{a+\frac1n}^{b-\frac1n}|f|\,dx$.
По предыдущей теореме функции $f_n(x):\bw
=f(x)\cdot\chi_{[a+\frac1n,b-\frac1n]}$ интегрируемы по Лебегу и
$(R)\int\limits_{a+\frac1n}^{b-\frac1n}f_n\,dx\bw
=(L)\int\limits_{[a+\frac1n,b-\frac1n]}f_n\,d\mu$ при любом
$n\in\N$. Далее, $f_n\ra f$ для любого $x\in(a,b)$. Поскольку
$\sup\limits_n\int|f_n|\,d\mu \bw<\infty$, по теореме Фату получаем
$f\in \Ls^1([a,b])$. Теперь по теореме Лебега имеем
$\int\limits_{[a,b]} f\,d\mu\bw
=\lim\limits_{n\ra\infty}\int\limits_{[a,b]}f_n\,d\mu\bw
=\lim\limits_{n\ra\infty}\int\limits_{a+\frac1n}^{b-\frac1n}|f|\,dx\bw
=\int\limits_a^b|f|\,dx$.

Обратно, пусть $f\in \Ls^1([a,b])$. Тогда для тех же функций $f_n$
получаем $\int\limits_{[a,b]}|f_n|\,d\mu\bw
\le\int\limits_{[a,b]}|f|\,d\mu$, поэтому функция $|f|$ несобственно
интегрируема по Риману.
\end{proof}

\begin{problem}
Привести пример компакта $K\subset[0,1]$, такого что функция
$\chi_K$ не может быть равна почти всюду интегрируемой по Риману
функции.
\end{problem}


\subsection{Пространства $\Ls^p$}

\subsubsection{Пространство $\Ls^1(\mu)$}\label{L1}

Пусть опять $(X,\As,\mu)$ --- пространство с конечной
неотрицательной мерой. Напомним, что $\Ls^1(\mu)$~--- это
пространство всех $\mu$-интегрируемых функций на $X$.

\begin{note}
Пространство $\Ls^1(\mu)$ не является линейным пространством, потому
что мы допускаем функции, не определённые в некоторых точках. Тогда
невозможно корректно определить, например, сумму двух функций, не
определённых в одной и той же точке. Поэтому обычно рассматривают
факторпространство $L^1(\mu)=\Ls^1(\mu)/\sim$, где $\sim$ ---
отношение эквивалентности, $f\sim g\Lra f=g$ п.в. Ясно, что
пространство $L^1(\mu)$ является линейным пространством.
\end{note}

В пространстве $L^1(\mu)$ определяется \emph{норма} функции $f$ по
формуле $\|f\|_{L^1(\mu)}=\int\limits_X|f|\,d\mu$ (иногда индекс
$L^1(\mu)$ у нормы будем опускать, когда ясно, о какой норме идёт
речь). Очевидно, что норма корректно определена, поскольку от
изменения значений функции на множестве нулевой меры результат не
поменяется.

Введённая норма действительно является нормой, поскольку для неё
справедливы все свойства нормы:

1) $\|f\|\ge0$;\, причём $\|f\|=0 \Lra f=0\in L^1(\mu)$;

2) $\|\al f\|=|\al|\cdot \|f\|$ при любом $\al\in\R$;

3) $\|f+g\|\le\|f\|+\|g\|$.

Справедливость свойств 1)--3) очевидно следует из свойств интеграла
Лебега.

Таким образом, пространство $L^1(\mu)$ является нормированным
линейным пространством.

\begin{theorem}\label{complete_l1}
Пространство $L^1(\mu)$ полно, т.е. оно ещё и банахово пространство.
\end{theorem}

\begin{proof}
Пространство полно тогда и только тогда, когда любая фундаментальная
последовательность в нём сходится. Пусть последовательность
$\{f_n\}$ фундаментальна. Тогда она фундаментальна по мере:
$\mu(\{x:|f_n(x)\bw -f_m(x)|\ge c\})\ra 0$ при $n,m\ra\infty$ для
любого $c>0$. Это ясно из неравенства Чебышёва: $\mu(\{x:|f_n(x)\bw
-f_m(x)|\ge c\})\bw \le\cfrac{\int|f_n-f_m|\,d\mu}c\bw
=\frac1c\|f_n-f_m\|$. Поэтому существует подпоследовательность
$f_{n_k}$, сходящаяся п.в. к функции $f$. Пусть задано произвольное
$\ve>0$. Тогда существует $N$, такое что для всех $n,m>N$ выполнено
неравенство $\|f_n-f_m\|\bw < \ve$. При любом фиксированном $n_0>N$
имеем $|f_{n_0}-f_{n_k}|\xra{\text{п.в.}}|f_{n_0}-f|$. По теореме
Фату $\int|f_{n_0}-f|\,d\mu<\ve$. Тогда
$\|f\|\le\|f_{n_0}-f\|+\|f_n\|<\infty$, поэтому $f\in L^1(\mu)$ и
$\|f_n-f\|\ra0$ при $n\ra\infty$.
\end{proof}

\begin{problem}
Доказать, что пространство функций, абсолютно интегрируемых на
$[0,1]$ по Риману, неполно.
\end{problem}

\subsubsection{Неравенства Гёльдера и Минковского}

\begin{theorem}[неравенство Гёльдера (H\"older)]
Пусть $p>1$ и $q$ таково, что $\frac1p+\frac1q=1$, т.е.
$q=\frac{p}{p-1}$. Пусть функции $f,g$ $\mu$-измеримы, причём
функции $|f|^p$, $|g|^q$ $\mu$-интегрируемы. Тогда функция $fg$
является $\mu$-интегрируемой и $\|fg\|\bw\le\|f\|_p\cdot\|g\|_q$
{\upshape($\|f\|_p=\left(\int|f|^p\,d\mu\right)^{1/p}$,
$\|g\|_q=\left(\int|g|^q\,d\mu\right)^{1/q}$)}.
\end{theorem}

\begin{proof}
Для любых неотрицательных чисел $a,b$ справедливо неравенство $ab\bw
\le\frac{a^p}p+\frac{b^q}q$ (доказать в качестве задачи!). Поэтому
$\cfrac{|f|}{\|f\|_p}\cdot\cfrac{|g|}{\|g\|_q}\bw \le\cfrac1p
\cdot\cfrac{|f|^p}{\|f\|_p^p}\bw
+\cfrac1q\cdot\cfrac{|g|^q}{\|g\|_q^q}$ (естественно, мы
предполагаем, что нормы $\|f\|_p$ и $\|g\|_q$ ненулевые, иначе
утверждение очевидно). Отсюда сразу следует интегрируемость функции
$fg$. Домножив это неравенство на $\|f\|_p\cdot\|g\|_q$ и затем
проинтегрировав по множеству $X$, получим $\|fg\|\bw
\le\cfrac1p\cdot\cfrac{\int|f|^p\,d\mu}{\|f\|_p^{p-1}}
\cdot\|g\|_q\bw +\cfrac1q \cdot
\cfrac{\int|g|^q\,d\mu}{\|g\|_q^{q-1}} \cdot\|f\|_p\bw
=\Br{\cfrac1p\bw +\cfrac1q}\cdot\|f\|_p\cdot\|g\|_q\bw
=\|f\|_p\cdot\|g\|_q$.
\end{proof}

\begin{theorem}[неравенство Минковского]
Пусть $p\ge1$, а функции $f,g$ $\mu$-измеримы, причём
$|f|^p,|g|^p\bw \in \Ls^1(\mu)$. Тогда $|f+g|^p\bw \in \Ls^1(\mu)$ и
$\|f+g\|_p\bw \le\|f\|_p+\|g\|_p$, где обозначение $\|\cdot\|_p$
означает то же самое, что и в неравенстве Гёльдера.
\end{theorem}

\begin{proof}
При $p=1$ доказывать нечего. Пусть $p>1$. Возьмём $q=\frac{p}{p-1}$.
Заметим, что $|f+g|^p\bw \le2^p(|f|^p+|g|^p)$, ибо при любом $x\in
X$ справедливо неравенство $|f(x)+g(x)|\bw
\le2\max\{|f(x)|,|g(x)|\}$. Отсюда $|f+g|^p\in \Ls^1(\mu)$. Далее,
имеем $|f+g|^p\bw\le|f+g|^{p-1}\cdot|f|\bw +|f+g|^{p-1}\cdot|g|$.
Заметим, что функция $|f+g|^{p-1}$ интегрируема в степени $q$.
Поэтому по неравенству Гёльдера $\int|f+g|^{p-1}\cdot|f|\,d\mu\bw
\le(\int|f+g|^p\,d\mu)^{1/q}\cdot(\int|f|^p)^{1/p}$, аналогично
$\int|f+g|^{p-1}\cdot|g|\,d\mu\bw
\le(\int|f+g|^p\,d\mu)^{1/q}\cdot(\int|g|^p)^{1/p}$. Значит,
$\int|f+g|^p\,d\mu\bw \le(\int|f+g|^p)^{1/q}\cdot(\|f\|_p\bw
+\|g\|_p)$. Осталось заметить, что $\int|f+g|^p\,d\mu=\|f+g\|_p^p$,
$(\int|f+g|^p\,d\mu)^{1/q}\bw
=\|f+g\|_p^{p/q}=\|f+g\|_p^{p(1-\frac1p)}\bw =\|f+g\|_p^{p-1}$ и
разделить неравенство на $\|f+g\|_p^{p-1}$.
\end{proof}

\subsubsection{Пространство $\Ls^p(\mu)$}

По определению пространство $\Ls^p(\mu)$ --- множество
$\mu$-измеримых функций, таких что $|f|^p\in \Ls^1(\mu)$. Как и в
$\Ls^1$-случае, вводим факторпространство
$L^p(\mu)=\Ls^p(\mu)/\sim$. Из неравенства Минковского следует, что
$L^p(\mu)$ --- линейное пространство. Теперь честно заведём
$L^p$-норму: $\|f\|_p:=\|f\|_{L^p}=(\int|f|^p\,d\mu)^{1/p}$. Это
действительно норма, потому что все свойства 1)--3) (см.
п.~\ref{L1}) для $\|\cdot\|_p$ справедливы.

\begin{theorem}
Пространство $L^p(\mu)$ полно.
\end{theorem}


\begin{proof}
Рассуждения аналогичны рассуждениям при доказательстве
теоремы~\ref{complete_l1}. Пусть последовательность $\{f_n\}\subset
L^p(\mu)$ фундаментальна по норме $\|\cdot\|_p$. Тогда она
фундаментальна по мере, так как по неравенству Чебышёва имеем
$\mu(\{x:|f_n(x)-f_m(x)|\ge c\})\bw =\mu(\{x:|f_n(x)-f_m(x)|^p\bw
\ge c^p\})\bw \le\frac1{c^p}\|f_n-f_m\|_p^p\bw \ra 0$ при
$m,n\ra\infty$. По теореме~\ref{Riss} последовательность $\{f_n\}$
содержит подпоследовательность $f_{n_k}$, сходящуюся п.в. к
некоторой функции $f$. Пусть $\ve>0$ --- произвольное число. Тогда
существует $N\in\N$, такое что $\|f_n-f_m\|_p<\ve$ при любых
$n,m>N$. При любом фиксированном $n_0>N$ имеем $|f_{n_0}-f_{n_k}|^p
\xra{\text{п.в.}}|f_{n_0}-f|^p$, откуда по теореме Фату
$\|f_{n_0}-f\|_p\bw \le\ve$. Далее по неравенству Минковского имеем
$\|f\|_p\bw \le\|f_{n_0}-f\|_p+\|f_{n_0}\|_p\bw <\infty$, поэтому
$f\in L^p(\mu)$. Наконец, при любом $n>N$ имеем $\|f-f_n\|_p\bw
\le\|f-f_{n_0}\|_p\bw +\|f_{n_0}-f_n\|_p\bw <\ve+\ve=2\ve$, откуда
$\|f-f_n\|_p\ra0$ при $n\ra\infty$.
\end{proof}

\subsubsection{Связь разных видов сходимости измеримых функций}

Эта связь может быть проиллюстрирована следующей схемой:

$$
\xymatrix{ {\text{равномерная}}\ar@{=>}[rr] \ar@{=>}[d]
&&{\text{сходимость}\:
L^p}\ar@{=>}[dd] \ar@{=>}[ddl]\\
{\text{поточечная}} \ar@{=>}[d] &
&\\
{\text{почти всюду}}\ar@{=>}[r] \ar@/^3pc/@{-->}[uu]^{\text{\tiny
{\begin{tabular}{c} Егоров\\(на множестве\\почти полной\\ меры)
\end{tabular}\hspace{-1em}}
}} &{\text{по мере}}\ar@/^1pc/@{-->}[l]^{\text{\tiny
{\begin{tabular}{c} Рисс\\(подпоследовательность)
\end{tabular}} }} &{\text{сходимость}\:L^1}\ar@{=>}[l] }
$$

Нуждается в пояснении только импликация $\text{<<сходимость}\: L^p
\Ra \text{сходимость}\: L^1\text{>>}$. Она следует из неравенства
Гёльдера: $\|f\|_1=\|f\cdot1\|_1\bw \le
\|f\|_p\cdot\|1\|_q=(\mu(X))^{1/q}\cdot\|f\|_p$, где
$q=\frac{p}{p-1}$.

\begin{problem}
Если $f_n\xra{\mu}f$ и $|f_n|\le \Phi$, где $\Phi$ ---
$\mu$-интегрируемая функция, то $f_n\xra{L^1}f$.
\end{problem}

\begin{problem}[теорема Витали--Фихтенгольца--Юнга (Young)]
Пусть $f_n\xra{\text{п.в.}}f$. Пусть функции $f_n$ и $f$
$\mu$-интегрируемы. Тогда $f_n\xra{L^1}f \Lra \int|f_n|\,d\mu\bw
\ra\int|f|\,d\mu$.
\end{problem}

\subsubsection{О пространстве $L^\infty(\mu)$}

Пусть $L^\infty(\mu)$ --- это множество классов эквивалентных
функций, которые обладают ограниченной модификацией, т.е. любую
такую функцию $f$ можно переопределить почти нигде (= на множестве
меры нуль) и получить ограниченную функцию. Норма вводится по
формуле $\|f\|_\infty=\|f\|_{L^\infty}:\bw =\inf\limits_{\widetilde
f\sim f}\,\sup\limits_{x\in X}|\widetilde f(x)|$.


\begin{problem}
Проверить, что это норма, и доказать полноту $L^\infty$.
\end{problem}

\begin{problem}
Доказать, что $f\in L^\infty(\mu)\Lra \sup\limits_{1\le
p<\infty}\|f\|_p\bw <\infty$, т.е. $f\in L^p(\mu)$ для любого
$p\ge1$; и привести пример недостаточности только принадлежности
функции $f$ всем $L^p$ без равномерной ограниченности $\|f\|_p$.
\end{problem}

\subsubsection{Пространство $L^2(\mu)$ и его свойства}

Это пространство со скалярным произведением $(f,g)=\int fg\,d\mu$.
Все свойства скалярного произведения очевидны. Полное евклидово
пространство по определению называется \emph{гильбертовым
пространством}, поэтому пространство $L^2(\mu)$ гильбертово.

\begin{problem}
Доказать, что в $L^2(\mu)$ выполняется \emph{неравенство
Коши--Буняковского:} $|(f,g)|\bw \le\|f\|_2\bw \cdot\|g\|_2$.
\end{problem}

\begin{prop}
Пусть $H$ --- гильбертово пространство и $L\subset H$ --- замкнутое
линейное подпространство в нём. Пусть $a\notin L$, тогда существует
единственный элемент $h\in L$, такой что $\|a-h\|_2\bw
=\inf\{\|x-a\|_2:x\in L\}$.
\end{prop}

\begin{proof}
\emph{Единственность.} Пусть есть ещё один элемент $\widetilde h\in
L$, $\widetilde h\ne h$, удовлетворяющий условию теоремы. Тогда
рассмотрим (не более чем трёхмерное) подпространство
$\ba{a,h,\widetilde h}$. В нём минимум расстояния от $a$ до
подпространства $\ba{h,\widetilde h}$ достигается в единственной
точке, поэтому $h=\widetilde h$.

\emph{Существование.} Пусть $d:=\inf\{\|x-a\|_2:x\in L\}$. Тогда
существует последовательность элементов $x_n\bw \in L$, такая что
$\|x_n-a\|_2^2\bw \le d^2+\frac1n$. Рассмотрим пространство
$S=S_{n,k}=\ba{a,x_n,x_k}$ и ортогонально спроектируем элемент $a$
на плоскость $\ba{x_n,x_k}$. Обозначим полученный элемент через $p$.
Тогда, используя конечномерность пространства $S$, получаем цепочку
неравенств $d^2\le\|a-p\|^2_2\le\|a-x_n\|^2\le d^2+\frac1n$.
Аналогично $d^2\le\|a-p\|^2_2\le\|a-x_k\|^2\le d^2+\frac1k$. Тогда
$\|x_n-x_k\|_2\bw \le\|x_n-p\|_2\bw +\|x_k-p\|_2\bw =(\text{теорема
Пифагора})\bw =\sqrt{\|a-x_n\|_2^2-\|a-p\|_2^2}+
\sqrt{\|a-x_k\|_2^2-\|a-p\|_2^2}\bw\le \frac1{\sqrt n}\bw
+\frac1{\sqrt k}$. Следовательно, последовательность $\{x_n\}$
фундаментальна. Так как пространство $H$ полно, а подпространство
$L$ замкнуто, то существует элемент $h\in L$, такой что $x_n\ra h\in
L$ при $n\ra\infty$. Покажем, что $h$ --- искомый вектор.
Действительно, имеем $\|a-h\|_2\bw \le\|a-x_n\|_2\bw +\|x_n-h\|_2\bw
\le\sqrt{d^2+\frac1n}+\|x_n-h\|_2$. Переходя к пределу при
$n\ra\infty$, получим $\|a-h\|_2\bw \le d$, т.е. $\|a-h\|_2=d$.
\end{proof}

В частности, утверждение теоремы верно и для $L^2(\mu)$.

\begin{imp}\label{functional}
Пусть $H$ --- гильбертово пространство, $l$ --- непрерывная линейная
функция на $H$. Тогда существует единственный элемент $v\in H$,
такой что $l(x)=(x,v)$, и обратно: при любом $v\in H$ линейная
функция $l(x)=(x,v)$ непрерывна. В частности, в случае $L^2(\mu)$
имеем $l(f)=\int fg\,d\mu$ для некоторого фиксированного $g\in
L^2(\mu)$.
\end{imp}
\begin{proof}
Если $v\in H$ и $l(x)=(x,v)$, то $|l(x)-l(y)|=|(x-y,v)|\bw
\le\|x-y\|\cdot\|v\|$ по неравенству Коши--Буняковского, откуда
следует непрерывность $l$.

Обратно, пусть $l$ --- непрерывная функция на $H$. Если $l\equiv0$,
то можно взять $v=0$. Пусть $l\not\equiv 0$. Тогда положим $L=\Ker
l$. Очевидно, что $L$ --- замкнутое подпространство. Возьмём элемент
$a$, такой что $l(a)=1$ (такой элемент существует, ибо $l\not\equiv
0$). Пусть $h$ --- проекция элемента $a$ на пространство $L$. Пусть
$v=\frac{a-h}{\|a-h\|^2}$. Рассмотрим линейную функцию
$l_0(x)=(x,v)$. Имеем $l_0(v)=(v,v)=\frac1{\|a-h\|^2}$. Кроме того,
$l(v)=\frac1{\|a-h\|^2}$, так как $l(h)=0$. Значит, $l_0(v)=l(v)$.
Если $x\in L$, то $l(x)=0$ и $l_0(x)=0$, причём последнее следует из
того, что $v\perp L$. Отсюда получаем, что $l\equiv l_0$, ибо любой
элемент $x\in L$ представим в виде $x=\la v+z$, где $z\in L$.
Действительно, $x=(x-l(x)(a-h))+l(x)(a-h)$, причём
$l(x-l(x)(a-h))\bw =l(x)-l(x)l(a)=0$ и $l(x)(a-h)\in \ha{v}$.
\end{proof}

\subsection{Теорема Радона--Никодима}

\begin{df}
Пусть $\mu,\nu$ --- неотрицательные меры на $\si$-алгебре $\As$.
Говорят, что мера $\nu$ \emph{абсолютно непрерывна относительно меры
$\mu$}, если $\nu(A)=0$ для всякого множества $A\in\As$, такого что
$\mu(A)=0$.
\end{df}
\begin{denote}
$\nu\ll\mu$.
\end{denote}
\begin{df}
Говорят, что меры $\mu$ и $\nu$ \emph{взаимно сингулярны}, если
существуют множества $X_1, X_2\in \As$, такие что $X_1\cap
X_2=\varnothing$, $X_1\cup X_2=X$, а $\mu(X_2)=\nu(X_1)=0$.
\end{df}
\begin{denote}
$\mu\perp\nu$.
\end{denote}
Это означает, что меры $\mu$ и $\nu$ сосредоточены на
непересекающихся множествах.

\begin{ex}
Пусть $\rho$ --- неотрицательная $\mu$-интегрируемая функция.
Положим $\nu(A)=\int\limits_A\rho(x)\,d\mu$. Тогда $\nu$ --- мера на
$\As$ и $\nu\ll\mu$.
\end{ex}

\begin{proof}
Если множества $A_i$ дизъюнктны и $A=\bigsqcup\limits_{i=1}^\infty
A_i$, то $\chi_A\bw = \sumnui\chi_{A_n}$. При этом
$\sum\limits_{n=1}^N\chi_{A_n}\bw\le 1$ при любом $N\in\N$. По
теореме Лебега $\nu(A)\bw =\int\limits_X\chi_A\rho\,d\mu\bw
=\lim\limits_{N\ra\infty} \int\limits_X\sum\limits_{n=1}^N
\chi_{A_n}\rho\,d\mu\bw =\lim\limits_{N\ra\infty}
\nu(\bigsqcup\limits_{n=1}^NA_n)\bw =\sumnui\nu(A_n)$, что и
требовалось.
\end{proof}

\begin{ex}
Пусть $\mu$ --- мера Лебега на $[0,1]$, а $\nu$ --- мера Дирака в
нуле: $\nu(A)\bw =\case{1, & 0\in A,\\ 0, & 0 \notin A.}$\; Тогда
$\mu\perp\nu$, можно взять $X_1=(0,1]$ и $X_2=\{0\}$.
\end{ex}


\begin{theorem}[Радон--Никодим]
Пусть $\nu,\mu$ --- неотрицательные меры на $\si$-алгебре $\As$ и
$\nu\ll\mu$. Тогда существует такая неотрицательная
$\mu$-интегрируемая функция $\rho$, что
$\nu(A)=\int\limits_A\rho\,d\mu$ для любого $A\in\As$.
\end{theorem}

\begin{note}
Функция $\rho$ называется \emph{плотностью Радона--Никодима} меры
$\nu$.
\end{note}

\begin{denote}
$\rho=\cfrac{d\nu}{d\mu}$.
\end{denote}

\begin{proof}
Рассмотрим пространство $L^2(\mu+\nu)$. На нём есть линейная функция
$l(\ph)=\int\limits_X\ph\,d\nu$, где $\ph\in L^2(\mu\bw +\nu)$. Эта
функция определена корректно, поскольку если функция $\ph$ измерима
относительно меры $\mu+\nu$, то она измерима относительно $\nu$,
причём если $\ph=\widetilde\ph$ п.в. относительно $\mu+\nu$, то
$\ph=\widetilde\ph$ п.в. относительно $\nu$. Линейность функции $l$
очевидна. Покажем, что $l$ непрерывна: $|l(\ph)-l(\psi)|\bw
=\left|\int\limits_X(\ph-\psi)\,d\nu\right|\bw
\le\int|\ph-\psi|\,d\nu \bw\le\int|\ph-\psi|\,d(\mu+\nu)\bw
\le{\text{(неравенство Коши--Буняковского)}}\bw
\le\|\ph-\psi\|_{L^2(\mu+\nu)}\bw \cdot\|1\|_{L^2(\mu+\nu)}\bw
=\sqrt{(\mu+\nu)(X)}\bw \cdot\|\ph\bw -\psi\|_{L^2(\mu+\nu)}$. По
следствию~\ref{functional} существует функция $g\in L^2(\mu+\nu)$,
такая что $l(\ph)\bw =\int\limits_X\ph g\,d(\mu+\nu)$ для любой
функции $\ph\in L^2(\mu+\nu)$. Получаем $\int\ph\,d\nu\bw =\int\ph
g\,d\nu+\int\ph g\,d\mu$ для любой $\ph\in L^2(\mu+\nu)$. Отсюда
$\int\ph(1-g)\,d\nu\bw =\int\ph g\,d\mu$. Покажем, что в качестве
$\rho$ можно взять функцию $\rho:=\cfrac{g}{1-g}$. Подставим в
последнее равенство функцию $\ph=\chi_{\{x:g(x)=1\}}$. Слева получим
ноль, поэтому и справа ноль, откуда $\mu(\{x:g(x)=1\})=0$, т.е.
$g\ne1$ $\mu$-почти всюду. Теперь подставив в то же равенство
$\ph=\chi_{\{x:g(x)\ge1\}}$, получим слева неположительное значение,
а справа неотрицательное, поэтому оба этих значения --- нули и
$\mu(\{x:g(x)\ge1\})=0$, т.е. $g<1$ $\mu$-почти всюду. Теперь
подставим в то же равенство $\ph=\chi_{\{x:g(x)<0\}}$. Слева получим
неотрицательное значение, справа --- неположительное, откуда
$\mu(\{x:g(x)<0\})=0$, т.е. $g\ge0$ $\mu$-почти всюду. Итак, функция
$\rho$ определена $\mu$-почти всюду, $\mu$-измерима как отношение
$\mu$-измеримых функций и неотрицательна.

Пусть $X_n: =\{x:g(x)\bw \le 1-\frac1n\}$. Тогда возьмём функцию
$\ph_n=\cfrac{\chi_{X_n}}{1-g}$. Имеем $\int\ph_ng\,d\mu\bw
=\int\ph_n(1-g)\,d\nu\bw =\int\chi_{X_n}\,d\nu\bw \le\nu(X)$. По
теореме Фату функция $\rho$ является $\mu$-интегрируемой и
$\int\limits_X\rho\,d\mu\bw \le \nu(X)$. Далее, $\chi_{X_n}(x)\ra1$
$\mu$-почти всюду и потому $\chi_{X_n}(x)\ra1$ $\nu$-почти всюду.
Поэтому $\nu(A)\bw
=\lim\limits_{n\ra\infty}\int\limits_A\chi_{X_n}\,d\nu\bw
=\lim\limits_{n\ra\infty}\int\limits_X\chi_A
\cdot\chi_{X_n}\cdot\frac1{1-g}\cdot(1\bw -g)\,d\nu\bw
=\lim\limits_{n\ra\infty}\int\limits_X\chi_A
\cdot\chi_{X_n}\cdot\frac{g}{1-g}\,d\mu\bw =\lim\limits_{n\ra\infty}
\int\limits_A\chi_{X_n}\cdot\rho\,d\mu\bw =
\int\limits_A\rho\,d\mu$, где последнее равенство следует из теоремы
Лебега, так как функция $\rho$ $\mu$-интегрируема.
\end{proof}

\begin{note}
Доказательство теоремы Радона--Никодима даёт большее: пусть
$\mu,\nu$ --- неотрицательные меры на $\As$, но теперь мы не требуем
абсолютной непрерывности. Тогда существует функция $\rho\in
L^1(\mu)$ и неотрицательная мера $\mu_0$ на $\As$, такие что $\nu\bw
=\rho\cdot\mu+\mu_0$, причём $\mu_0\perp\mu$.
\end{note}

\begin{problem}
Доказать это, используя аналогичные рассуждения, как и в
доказательстве теоремы Радона--Никодима.
\end{problem}

\begin{ex}
Пусть $\la$ --- мера Лебега на $[0,1]$, $\de$ --- мера Дирака в
нуле. Тогда $\la+\de=1\cdot\la+\de$.
\end{ex}
\begin{note}
В случае борелевских мер на $\R$ всякая мера $\nu$ может быть
записана в виде $\nu=\rho\cdot\la+\nu_0$, где $\la$ --- мера Лебега,
$\rho$ --- $\la$-интегрируемая функция и $\nu_0\perp\la$. Кроме
того, мера $\nu_0$ имеет не более чем счётное множество точек $t_n$,
для которых $\nu_0(t_n)>0$ (для каждого $k$ количество точек меры
больше $1/k$ конечно, так как мера всего пространства конечна).
Поэтому $\nu_0\bw =\nu_c+\sumnui c_n\de_{t_n}$, где $\de_t$ --- мера
Дирака в точке $t$ и $\nu_c\perp\la$, причём мера $\nu_c$ не имеет
точек положительной меры. Это мера является \emph{чисто непрерывной
сингулярной компонентой} меры $\nu$.
\end{note}

\subsection{Теорема Фубини и смежные вопросы}

\subsubsection{Произведение мер}

Пусть $(X,\As,\mu)$ и $(Y,\Bs,\nu)$ --- измеримые пространства с
конечными неотрицательными мерами. При $A\in\As$ и $B\in\Bs$ положим
$(\mu\times\nu)(A\times B)=\mu(A)\nu(B)$. Такие множества $A\times
B$ называются \emph{измеримыми прямоугольниками}. Они не образуют
алгебру, но $\As\otimes\Bs$ --- $\si$-алгебра, порождённая
измеримыми прямоугольниками, --- достойна нашего рассмотрения.
Алгебра, порождённая измеримыми прямоугольниками, состоит из
дизъюнктных конечных объединений $\bigsqcup\limits_{n=1}^NA_n\times
B_n$, где $A_n\in\As$, $B_n\in\Bs$. На такие множества мера
$\mu\times\nu$ естественно продолжается:
$(\mu\times\nu)[\bigsqcup\limits_{n=1}^NA_n\times B_n]\bw
=\sum\limits_{n=1}^N\mu(A_n)\nu(B_n)$. Докажем счётную аддитивность
этой меры.

\begin{theorem}
Мера $\mu\times\nu$ счётно-аддитивна на алгебре, порождённой
измеримыми прямоугольниками.
\end{theorem}

\begin{proof}
Пусть сначала $C=\bigsqcup\limits_{n=1}^\infty C_n$, где $C=A\times
B$ и $C_i=A_i\times B_i$; $A,A_i\in\As$, $B,B_i\in\Bs$. Введём
функции $f_n: X\ra\R$ следующим образом: $f_n(x)\bw =\case{\nu(B_n),
& x\in A_n,\\ 0, & x\notin A_n.}$ Ясно (проверьте!), что $\sumnui
f_n\bw =\nu(B)$ на множестве $A$. По теореме Б. Леви
$\sumnui\int\limits_Af_n\,d\mu=\int\limits_A\nu(B)\,d\mu\bw
=\mu(A)\nu(B)$. Но $\int\limits_Af_n\,d\mu\bw =\nu(B_n)\mu(A_n)\bw
=(\mu\times\nu)(C_n)$, и тем самым в частном случае счётная
аддитивность доказана.

Пусть теперь $C\bw = \bigsqcup\limits_{j=1}^NC_j$, где $C_j$ ---
измеримые прямоугольники, и $C=\bigsqcup\limits_{n=1}^\infty D_n$,
где $D_n\bw =\bigsqcup\limits_{i=1}^{M_n}D_{n,i}$, $n=1,2,\ldots$, и
$D_{n,i}$ --- измеримые прямоугольники. Пусть $D_{n,i,j}\bw
=D_{n,i}\cap C_j$. Тогда множества $D_{n,i,j}$ дизъюнктны и $C_j\bw
=\bigsqcup\limits_{n=1}^\infty\bigsqcup\limits_{i=1}^{M_n}D_{n,i,j}$,
$D_{n,i}\bw =\bigsqcup\limits_{j=1}^N D_{n,i,j}$. По уже доказанному
имеем $\mu(C_j)\bw =\sumnui\sum\limits_{i=1}^{M_n}\mu(D_{n,i,j})$,
$\mu(D_{n,i})\bw =\sum\limits_{j=1}^N\mu(D_{n,i,j})$ и
$\mu(C)=\sum\limits_{j=1}^N\mu(C_j)$, $\mu(D_n)\bw
=\sum\limits_{i=1}^{M_n}\mu(D_{n,i})$. Ввиду абсолютной сходимости
всех рядов получаем $\mu(C)\bw =\sumnui\mu(D_n)$.
\end{proof}

$\si$-Алгебра, порождённая прямоугольниками, обозначается через
$\As\otimes\Bs$. По доказанному (теорема~\ref{continue_measure})
мера $\mu\otimes\nu$ с алгебры, порождённой измеримыми
прямоугольниками, продолжается до счётно-аддитивной меры на
$\As\otimes\Bs$. Далее, эту меру можно продлить на лебегово
пополнение $\Ls(\As\otimes\Bs):=\Ls_{\mu\times\nu}$.

Пополненная счётно-аддитивная мера на лебеговом пополнении
$\Ls(\As\otimes\Bs)$ обозначается через $\mu\otimes\nu$ и называется
\emph{произведением мер} $\mu$ и $\nu$.

\begin{df}
Мера $\mu$ на $\si$-алгебре $\Ec$ называется \emph{полной}, если для
любого множества $E\in\Ec$ с условием $\mu(E)=0$ и любого
подмножества $D\subset E$ имеем $D\in\Ec$ (и тогда $\mu(D)=0$).
\end{df}

По построению $\mu\otimes\nu$ --- полная мера, даже если меры $\mu$
и/или $\nu$ не были полными.

\begin{note}
При построении счётно-аддитивной меры, продолженной с алгебры на
$\si$-алгебру всех измеримых множеств, получается полная мера.
\end{note}
На $\As\otimes\Bs$ мера $\mu\otimes\nu$ не обязана быть полной.

\begin{ex}
Пусть $\mu=\nu=\la$ --- мера Лебега на $\si$-алгебре $\As=\Bs=\Ls$
всех измеримых по Лебегу множеств на отрезке $[0,1]$. Тогда
$\si$-алгебра $\Ls\otimes\Ls$ меньше, чем $\si$-алгебра $\Ls_2$ всех
измеримых множеств в $[0,1]^2$.
\end{ex}

\begin{lemma}
Пусть $(X,\As)$ и $(Y,\Bs)$ --- измеримые пространства. Тогда для
любого множества $E\in\As\otimes\Bs$ и любого $y\in Y$ имеем
$E_y\in\As$, где $E_y:\bw =\{x\in X: (x,y)\in E\}$.
\end{lemma}

\begin{note}
Множества $E_y$ и аналогичные им множества $E_x:=\{y\in Y: (x,y)\in
E\}$ называются \emph{сечениями} множества $E$.
\end{note}

\begin{proof}
Если $E=A\times B$, где $A\in\As$, $B\in\Bs$, то это очевидно.
Обозначим через $\Ec$ класс всех множеств $E\in \As\otimes\Bs$, для
которых это верно. Утверждается, что $\Ec$ --- $\si$-алгебра.
Действительно, $(X\times Y)\backslash E\in\Ec$, ибо $((X\times
Y)\backslash E)_y=X\backslash E_y$. Далее, если $E_n\in\Ec$, то
$\cupnui E_n\in\Ec$, потому что $(\cupnui E_n)_y\bw =\cupnui
(E_n)_y$. Так как $\Ec$ содержит $\si$-алгебру, порождённую
измеримыми прямоугольниками, то $\Ec\bw =\As\otimes\Bs$.
\end{proof}

\begin{imp}
Если $\Ns$ --- неизмеримое по Борелю подмножество на отрезке
$[0,1]$, то оно будет неизмеримо по Борелю и в квадрате $[0,1]^2$.
При этом множество $\Ns$ измеримо по Лебегу в квадрате и имеет меру
нуль.
\end{imp}

\subsubsection{Замечание о бесконечных мерах} \label{infty_measures}

Можно рассматривать счётно-аддитивные меры
$\mu:\As\ra\overline\R^+$. В этом случае счётная аддитивность
определяется так же, как и для пространств с конечными мерами:
$\mu(\bigsqcup\limits_{n=1}^\infty(A_n))\bw
=\sumnui\mu(A_n)\in\overline\R$. Но в этом случае есть одно
нововведение: обязательным требованием является $\mu(\varnothing)=0$
(в пространствах с конечными мерами это выполняется автоматически).

\begin{ex}
Положим $\mu(\varnothing)=0$, $\mu(A)=\infty$ для любого множества
$A\ne\varnothing$. Тогда $\mu$ --- счётно-аддитивная бесконечная
мера.
\end{ex}

\begin{ex}
Множество $A\subset\R^n$ называется измеримым по Лебегу, если для
всякого куба $K\subset \R^n$ множество $A\cap K$ измеримо. Разделим
$\R^n$ на кубосетку, образованную всевозможными сдвигами куба
$I=[-1,1]^n$ на целочисленные векторы с чётными координатами.
Положим $\la_n(A)\bw =\sum\limits_{j=1}^\infty \la_n(A\cap I_j)$ для
любого измеримого множества $A\subset\R^n$. Получена
счётно-аддитивная мера на $\R^n$ со значениями в $\overline\R^+$.
\end{ex}

\begin{df}
Мера $\mu: X\ra\overline\R^+$ называется $\si$-конечной, если
$X=\bigsqcup\limits_{n=1}^\infty X_n$, где $X_n\in\As$ и
$\mu(X_n)<\infty$ для всякого $n\in\N$. В этом случае $\mu(A)\bw
=\sumnui\mu(A\cap X_n)$ для всякого $A\in\As$.
\end{df}
В частности, мера Лебега $\la_n$ на $\R^n$ является $\si$-конечной.

Интеграл Лебега для множеств с бесконечными мерами определяется так
же, как и для множеств с конечными мерами, только теперь функция $f$
называется простой, если $f=\sum\limits_{i=1}^nc_i\chi_{A_i}$, где
$\mu(A_i)<\infty$ при $c_i\ne 0$. Основное определение интеграла
Лебега остаётся без изменений (полагаем при этом, что
$c_i\mu(A_i)\bw =0$, если $c_i\bw =0$, $\mu(A_i)\bw =\infty$). Если
мера $\mu$ $\si$-конечна, то интеграл Лебега сводится к случаю
конечной меры следующим образом. Положим
$\nu(A):=\sumnui2^{-n}\frac1{1+\mu(X_n)}\mu(X_n\cap A)$. Пусть
$\rho(X):=2^n(1+\mu(X_n))$ при $x\in X_n$. Тогда
$\int\limits_Xf\,d\mu=\int\limits_Xf\rho\,d\nu$.

\begin{problem}
Случай интеграла по бесконечной мере $\mu$ сводится к случаю
$\si$-конечной меры, так как если $f$ --- $\mu$-интегрируемая
функция, то мера $\mu$ является $\si$-конечной на множестве
$\{x:f(x)\ne 0\}$.
\end{problem}

Для бесконечных мер верны теоремы Лебега, Б. Леви, Фату, Гёльдера, о
полноте пространств $L^p$. Теорема Радона-Никодима верна только для
$\si$-конечных мер.

Если $\mu$ и $\nu$ --- $\si$-конечные меры, то меру $\mu\otimes\nu$
можно определить так. Пусть $X=\bigsqcup\limits_{n=1}^\infty X_n$,
$Y=\bigsqcup\limits_{k=1}^\infty Y_k$, $\mu(X_n)<\infty$,
$\nu(Y_k)<\infty$. Тогда положим $\mu\otimes\nu:\bw
=\sum\limits_{n,k}\left.\mu\right|_{X_n}\otimes
\left.\nu\right|_{Y_k}$.

\subsubsection{Теорема Фубини}

\begin{theorem}\label{predecessor_Fubini}
Пусть множество $A\subset X\times Y$ измеримо относительно меры
$\mu\otimes\nu$, где $\mu,\nu$ --- меры на $X,Y$ соответственно.
Положим $A_x\bw =\{y: (x,y)\in A\}\bw\subset Y$, $A_y\bw =\{x:
(x,y)\in A\}\bw\subset X$. Тогда для $\mu$-почти всех $x$ множество
$A_x$ $\nu$-измеримо и функция $x\mapsto\nu(A_x)$ $\mu$-измерима.
Далее, для $\nu$-почти всех $y$ множество $A_y$ $\mu$-измеримо и
функция $y\mapsto\mu(A_y)$ $\nu$-измерима, причём
$(\mu\otimes\nu)(A)=\int\limits_X\nu(A_x)\,d\mu\bw
=\int\limits_Y\mu(A_y)\,d\nu$.
\end{theorem}

\begin{proof}
Достаточно доказать, что $(\mu\otimes\nu)(A)\bw
=\int\limits_X\ph_A\,d\mu$, где $\ph_A(x)\bw =\nu(A_x)$. Второе
равенство доказывается аналогично. Для прямоугольников доказываемое
верно, поэтому верно и для конечных объединений прямоугольников.
Пусть $A$ --- произвольное $(\mu\otimes\nu)$-измеримое множество.
Тогда из конструкции измеримых множеств существует множество
$B\in\As\otimes\Bs$, такое что $B\supset A$ и $(\mu\otimes\nu)(A)\bw
=(\mu\otimes\nu)(B)$ (см. предложение~\ref{close_set_from_A}). При
этом $B=\capnui B_n$, где $B_n\supset B_{n+1}$, $B_n\bw =\cupkui
B_{n,k}$, $B_{n,k}$~--- конечное объединение дизъюнктных измеримых
прямоугольников и $A\subset B_n$ для любого $n\in\N$ (ср. с
доказательством предложения~\ref{close_set_from_A}). Докажем наше
утверждение для множества $B$. Для каждого из множеств $B_{n,k}$ оно
верно; для конечных объединений вида
$\bigcup\limits_{k=1}^NB_{n,k}\bw =:C_{n,N}$ тоже верно, поскольку
такие множества представимы в виде дизъюнктного объединения
измеримых прямоугольников. Далее, последовательность функций
$\ph_{C_{n,N}}$ возрастает и стремится к $\ph_{B_n}$ при
$N\ra\infty$. По теореме Б. Леви о монотонной сходимости
доказываемое утверждение верно для множеств $B_n$.
Последовательность $\ph_{B_n}$ убывает и стремится к функции $\ph_B$
при $n\ra\infty$, поэтому утверждение теоремы верно для множества
$B$ снова по теореме Б. Леви (если множества $E_k$ убывают или
возрастают к множеству $E$, то их сечения возрастают или убывают к
сечению $E$). Осталось доказать наше утверждение для множества $C\bw
= B\backslash A$, которое имеет $(\mu\otimes\nu)$-меру нуль.
Существует множество $\widetilde B\in \As\otimes\Bs$ такого же вида,
как и $B$, такое что $C\subset \widetilde B$ и
$(\mu\otimes\nu)(\widetilde B)=0$. При этом $C_x\subset \widetilde
B_x$ для любого $x\in X$. Для множества $\widetilde B$ утверждение
теоремы верно (уже доказано), поэтому $0\bw
=(\mu\otimes\nu)(\widetilde B)\bw =\int\limits_X\nu(\widetilde
B_x)\,d\mu$. Отсюда $\nu(\widetilde B_x)\bw =0$ $\mu$-п.в. и потому
$\nu(C_x)\bw =0$ $\mu$-п.в. Таким образом, утверждение верно для
множества $C$, а значит, верно и для $A$.
\end{proof}
\begin{imp}\label{subgraphic}
Пусть $\mu$ --- конечная мера, $f\ge 0$ --- $\mu$-интегрируемая
функция на $X$. Пусть $Y=\R^1$ с мерой Лебега $\la$. Пусть $A$ ---
подграфик функции $f$, т.е. множество $\{(x,y)\in X\times Y:0\le
y\le f(x)\}$. Тогда $\int\limits_Xf\,d\mu\bw =(\mu\otimes\la)(A)$.
\end{imp}
\begin{proof}
Имеем $\la(A_x)=f(x)$ для любого $x\in X$. По предыдущей теореме
получаем требуемое, если проверить измеримость множества $A$.
Проверим её. Так как функция $f$ $\As$-измерима, то множество
$\{(x,t):0\le f(x)\le t\}\bw =\{(x,t):f(x)\le t\}$ входит в
$\si$-алгебру $\As\otimes\Bs(\R^1)$, поскольку функция
$g(x,t)=f(x)-t$ измерима относительно этой $\si$-алгебры (как сумма
двух измеримых функций).
\end{proof}

\begin{theorem}[Фубини]
Пусть функция $f: X\times Y\ra \R$ интегрируема по мере
$\mu\otimes\nu$. Тогда для $\nu$-п.в. $y$ функция $f(x,y)$
$\mu$-интегрируема и функция $y\mapsto \int\limits_Xf(x,y)\,d\mu$
является $\nu$-интегрируемой. Далее, для $\mu$-п.в. $x$ функция
$f(x,y)$ $\nu$-интегрируема и функция $x\mapsto
\int\limits_Yf(x,y)\,d\nu$ является $\mu$-интегрируемой. При этом
$\int\limits_{X\times Y}f\,d(\mu\otimes\nu)\bw =\int\limits_Y
\br{\int\limits_Xf\,d\mu} d\nu\bw =\int\limits_X
\br{\int\limits_Yf\,d\nu} d\mu$.
\end{theorem}

\begin{proof}
Достаточно доказать утверждение для неотрицательной функции $f$.
Рассмотрим пространство $X\times Y\times\R^1$ с мерой
$\mu\otimes\nu\otimes\la$, где $\la$ --- мера Лебега. Пусть
$A=\{(x,y,t):0\le t\le f(x,y)\}$. По следствию~\ref{subgraphic}
получаем $\int\limits_{X\times Y}f\,d(\mu\otimes\nu)\bw
=(\mu\otimes\nu\otimes\otimes\la)(A)$, а эта величина по
теореме~\ref{predecessor_Fubini} равняется
$\int\limits_X(\nu\otimes\la)(A_x)\,d\mu$, а это, в свою очередь, по
следствию~\ref{subgraphic} равно $\int\limits_X
\br{\int\limits_Yf\,d\nu} d\mu$, что и требовалось доказать.
\end{proof}

\begin{note}
Если $f$ измерима относительно пополненной $\si$-алгебры
$\Ls(\As\otimes\Bs)$, то при $\nu$-п.в. $y\in Y$ функция $x\mapsto
f(x,y)$ $\mu$-измерима. Это следует из того, что при фиксированном
$r$ множество $\{(x,y):f(x,y)< r\}$ является
$(\mu\otimes\nu)$-измеримым, откуда по
теореме~\ref{predecessor_Fubini} множество $\{x:f(x,y)<r\}$ является
$\mu$-измеримым при $\nu$-п.в. $y\in Y$.
\end{note}

\begin{note}
В доказательстве теоремы Фубини можно иметь дело с ограниченными
функциями, так как потом легко перейти к неограниченным с помощью
теоремы Б. Леви, рассматривая срезки $f_n:\bw =\min\{f,n\}$ (считаем
$f\ge 0$).
\end{note}

\begin{theorem}[Тонелли]
Пусть неотрицательная функция $f$ измерима относительно меры
$\mu\otimes\nu$. Пусть $\int\limits_Y \br{\int\limits_Xf\,d\mu}
\,d\nu\bw <\infty$. Тогда функция $f$ является
$(\mu\otimes\nu)$-интегрируемой.
\end{theorem}

\begin{proof}
Пусть $f_n=\min\{f,n\}$. Тогда каждая из функций $f_n$ $(\mu\bw
\otimes\nu)$-интегрируема и по теореме Фубини
$$\int\limits_{X\times
Y}f_n\,d(\mu\bw \otimes\nu)\bw =\int\limits_Y
\br{\int\limits_Xf_n\,d\mu}\,d\nu\bw \le\int\limits_Y
\br{\int\limits_Xf\,d\mu}\,d\nu.$$
Отсюда по теореме Фату функция $f$
$(\mu\bw \otimes\nu)$-интегрируема.
\end{proof}

\begin{problem}
Построить пример функции, не являющейся неотрицательной, для которой
теорема Тонелли неверна.
\end{problem}

\begin{problem}
Привести пример неотрицательной функции $f: X\times Y\ra \R$,
которая не интегрируема относительно меры $(\mu\times\nu)$, но
$\int\limits_Y \br{\int\limits_Xf\,d\mu}\,d\nu$ существует.
\end{problem}

\begin{problem}
Привести пример функции $f: X\times Y\ra \R$, такой что
$\int\limits_Y \br{\int\limits_Xf\,d\mu}\,d\nu\bw =\int\limits_X
\br{\int\limits_Yf\,d\nu}\,d\mu$, но $f$ не является
$(\mu\otimes\nu)$-интегрируемой.
\end{problem}

\begin{problem}
Привести пример функции $f: X\times Y\ra \R$, такой что
$\int\limits_Y \br{\int\limits_Xf\,d\mu}\,d\nu\bw \ne\int\limits_X
\br{\int\limits_Yf\,d\nu}\,d\mu$.
\end{problem}

\subsection{О замене переменных}

Пусть $(X,\As,\mu)$ --- пространство с конечной мерой и $f:
(X,\As)\ra (Y,\Bs)$ функция, являющаяся $(\As,\Bs)$-измеримой. Тогда
определена мера $B\mapsto (\mu\circ f^{-1})(B):\bw =\mu(f^{-1}(B))$
для любого множества $B\in\Bs$.

\begin{df}
Мера $\mu\circ f^{-1}$ называется \emph{образом меры $\mu$ при
отображении} $f$.
\end{df}

Ясно, что мера $\mu\circ f^{-1}$ счётно-аддитивна.

\begin{problem}
Пусть $\ph$ --- ограниченная $\Bs$-измеримая функция на $Y$. Тогда
$\int\limits_Y\ph(y)\,d(\mu\circ f^{-1})\bw =\int\limits_X\ph
\bw\circ f\,d\mu$.
\end{problem}
\begin{hint}
Сначала доказать это для характеристических функций.
\end{hint}

Если $\mu$ --- мера Лебега на $\R^n$ и $f:\R^n\ra\R^n$, то можно
поинтересоваться, когда мера $\mu\circ f^{-1}$ задаётся плотностью.
Достаточным условием для этого является дифференцируемость $f$ и
невырожденность якобиана $f'(x)$ почти всюду. Пусть $\Om_1$ и
$\Om_2$ --- открытые множества в $\R^n$, а $f$ --- диффеоморфизм
$\Om_1\ra\Om_2$. Тогда для любой ограниченной измеримой функции
$\ph:\Om_2\ra\R$ имеем $\int\limits_{\Om_1}\bw \ph\circ f\,dx\bw
=\int\limits_{\Om_2}\ph(y)\cfrac1{|\mr{Det}\,f'(f^{-1}(y))|}\,dy$.

\begin{lemma}
Пусть $\mu$ --- борелевская мера на кубе $K\subset\R^n$, причём
$\mu(B+h)=\mu(B)$ для любых $B\in\Bs(\R^n)$ и $h\in\R^n$. Тогда
$\mu\bw =k\cdot\la_n$, где $k\in\R$, $\la_n$ --- мера Лебега на
$\R^n$. Иными словами, с точностью до нормировки мера Лебега ---
единственная мера на $\R^n$, сохраняемая при сдвигах.
\end{lemma}

\begin{proof}
Достаточно это доказать для $n$-мерных параллелепипедов вида
$(\al_1,\be_1]\times(\al_2,\be_2] \times \ldots \times
(\al_n,\be_n]\subset K$, поскольку они порождают $\si$-алгебру
$\Bs(\R^n)$. Более того, достаточно ограничиться теми
параллелепипедами, для которых $\al_i,\be_j\in\Q$ (ср. с
теоремой~\ref{borel_alg}). А для любых двух таких параллелепипедов
$P_1,P_2$ найдётся натуральное число $n$, такое что каждое из
множеств $P_1$ и $P_2$ можно разбить на целое число непересекающихся
кубиков размера $(\frac1n,\frac1n]\times (\frac1n,\frac1n]\times
\ldots \times(\frac1n,\frac1n]$. Из этого получаем, что
$\mu(P_1)\la_n(P_2)\bw =\mu(P_2)\la_n(P_1)$, откуда и следует
доказываемое утверждение.
\end{proof}

\begin{imp}
Если $T$ --- ортогональный линейный оператор, то $\la_n\circ
T^{-1}=\la_n$.
\end{imp}

\begin{proof}
Из предыдущей леммы следует, что $\la_n\circ T^{-1}=k\cdot \la_n$.
Но эти две меры совпадают на любом шаре, поэтому $k=1$.
\end{proof}


\subsection{Свёртки}

\begin{df}
Пусть $f,g$ --- функции на $\R^n$, интегрируемые по мере Лебега.
Тогда \emph{свёрткой} функций $f$ и $g$ называется функция
$(f*g)(x):=\int\limits_{\R^n}f(x-y)g(y)\,dy$. Если функция $f$
ограниченна, то свёртка определена для почти всех $x$.
\end{df}

\begin{theorem}
Пусть $f,g\in L^1(\R^n)$. Тогда свёртка $f*g$ определена для почти
всех $x$, причём \equ{\|f*g\|_{L^1} \le\|f\|_{L^1}\cdot\|g\|_{L^1}.}
\end{theorem}

\begin{proof}
Рассмотрим функцию $\widetilde g(x,y):=f(x)\cdot g(y)$. Имеем
$\widetilde g(x,y) \in L^1(\R^{2n})$ по теореме Б. Леви, ибо
$|\widetilde g|\bw =\lim\limits_{k\ra\infty}
\min\{|f|,k\}\chi_{\{|x|\le k\}}\bw \cdot
\min\{|g|,k\}\chi_{\{|y|\le k\}}$, а интеграл от правой части по
$\R^n$ по теореме Фубини не превосходит
$\int\limits_{\R^n}|f|\,dx\cdot\int\limits_{\R^n}|g|\,dy
=\|f\|_{L^1}\cdot\|g\|_{L^1}$. Делаем замену $\case{u=x+y,\\v=y.}$
Тогда $\widetilde g(u,v)=f(u-v)g(v)$ на $\R^{2n}$. По теореме Фубини
при почти всех $u\in\R^n$ функция $v\mapsto f(u-v)g(v)$ интегрируема
по $v$, а её интеграл интегрируем по $u$. Значит, функция $f*g$
определена при почти всех $u$ и интегрируема на $\R^n$. Наконец,
$\|f*g\|_{L^1}\bw = \int\limits_{\R^n}
\bm{\int\limits_{\R^n}f(u-v)g(v)\,dv} \,du\bw \le
\int\limits_{\R^n}\br{\int\limits_{\R^n}|f(u-v)g(v)|\,dv}\,du\bw
=\text{(формула замены переменных)}\bw =
\int\limits_{\R^n}\br{\int\limits_{\R^n}|f(x)g(y)|\,dy}\,dx\bw
=\int\limits_{\R^n}|f|\,dx\cdot\int\limits_{\R^n}|g|\,dy
=\|f\|_{L^1}\cdot\|g\|_{L^1}$.
\end{proof}

\begin{problem}
Если $f\in L^p(\R^n)$, $g\in L^1(\R^n)$, то $f*g\in L^p(\R^n)$.
\end{problem}

\subsection{Связь интеграла и производной}

\subsubsection{Функции ограниченной вариации}

\begin{df}
Функция $f$ на отрезке $[a,b]$ называется \emph{функцией
ограниченной вариации}, если конечна величина (называемая
\emph{вариацией} функции $f$ на отрезке $[a,b]$)
$V_{[a,b]}(f)=\sup\limits_T\sum\limits_{i=1}^n |f(a_i)-f(a_{i-1})|$,
где $T$ --- разбиение отрезка на $n=n(T)$ частей точками $a=a_0<a_1<
\ldots <a_{n-1}<a_n=b$.
\end{df}

\begin{denote}
$f\in VB([a,b])$ или просто $f\in VB$, когда ясно, о каком отрезке
идёт речь.
\end{denote}

\begin{problem}
Вариация линейна, т.е. если $a\le c\le b$, то $V_{[a,b]}(f)\bw
=V_{[a,c]}(f)\bw +V_{[c,b]}(f)$.
\end{problem}

\begin{problem}\label{bounded_variation}
Если $f\in VB([a,b])$, то $f$ есть разность двух монотонных функций.
\end{problem}

\begin{theorem}[Лебег, без доказательства]\label{bound_var_diff}
Любая функция $f$ ограниченной вариации п.в. дифференцируема, и
функция $f'$ интегрируема.
\end{theorem}

Можно задаться вопросом: верно ли в этом случае, что $f(b)-f(a)\bw =
\int\limits_a^bf'\,dx$? Ответ: вообще говоря, нет. В качестве
примера можно взять на отрезке $[0,1]$ функцию $k(x)\bw =\case{0,
&x\in[0,\frac12],\\1, & x\in(\frac12,1].}$ Ясно, что $k'(x)\bw =0$
п.в., но формула Ньютона--Лейбница неверна. На самом деле это,
вообще говоря, неверно даже для непрерывных функций.

\subsubsection{Абсолютно непрерывные функции и формула
Ньютона--Лейбница}

\begin{df}
Функция $f$ называется\emph{ абсолютно непрерывной} на отрезке
$[a,b]$, если для всякого $\ve>0$ существует $\de>0$, такое что
$\sum\limits_{i=1}^n|f(b_i)-f(a_i)|\bw <\ve$ для всякого конечного
набора дизъюнктных интервалов $\{(a_i,b_i)\}$ с условием
$\sum\limits_{i=1}^n|b_i-a_i|\bw <\de$.
\end{df}
\begin{denote}
$f\in AC([a,b])$.
\end{denote}

Непосредственно из определения следует

\begin{stm}
Если $f\in AC([a,b])$, то $f\in C([a,b])$.
\end{stm}

\begin{note}
Обратное неверно. Пример --- функция $f$ на отрезке $[0,1]$, на
каждом из отрезков $[\frac1{n+1},\frac1n]$ заданная следующим
образом: от левого конца отрезка до его середины она линейно
возрастает от нуля до $\frac1n$, от середины до правого конца ---
линейно убывает от $\frac1n$ до нуля; кроме того, $f(0)=0$. Несложно
проверить, что $f$ не является ни функцией ограниченной вариации, ни
абсолютно непрерывной функцией.
\end{note}

\begin{stm}\label{ac=>vb}
Если $f\in AC([a,b])$, то $f\in VB([a,b])$.
\end{stm}

\begin{proof}
Для $\ve=1$ найдётся соответствующее $\de$ из определения абсолютной
непрерывной функции. Пусть $n\in\N$ таково, что $\frac{b-a}n<\de$.
Тогда нетрудно убедиться, что $V_{[a,b]}(f)\le n$.
\end{proof}

Теперь докажем основную теорему.

\begin{theorem}\label{abs_cont}
Пусть функция $f$ задана на отрезке $[a,b]$. Тогда $f\in
AC([a,b])\bw \Lra $ существует функция $g\in \Ls^1([a,b])$, такая
что $f(x)= f(a)\bw +\int\limits_a^xg(t)\,dt$. При этом функция $f$
п.в. дифференцируема и $f'=g$ п.в. В частности, $f'(x)\in
\Ls^1([a,b])$ и $f(x)-f(a)\bw =\int\limits_a^xf'(t)\,dt$.
\end{theorem}

\begin{note}
Эта теорема даёт альтернативную классификацию абсолютно непрерывных
функций.
\end{note}

\begin{proof}(Частичное доказательство)
Пусть $g\in \Ls^1([a,b])$. Докажем, что функция
$f(x)=\int\limits_a^xg(t)\,dt$ абсолютно непрерывна. Пусть $\ve>0$
--- произвольное число. Тогда в силу абсолютной непрерывности
интеграла Лебега существует $\de>0$, такое что
$\int\limits_E|g|\,dt<\ve$ для всякого множества $E$ с $\la(E)<\de$.
Пусть $\{(a_i,b_i)\}$ --- дизъюнктный набор интервалов на $[a,b]$ с
суммой длин меньше $\ve$. Тогда $\sum\limits_i|f(b_i)-f(a_i)|\bw
=\sum\limits_i\Bm{\int\limits_{a_i}^{b_i}g\,dt}\bw
\le\sum\limits_i\int\limits_{a_i}^{b_i}|g|\,dt\bw
=\int\limits_{\bigcup\limits_i(a_i,b_i)}|g|\,dt\bw <\ve$.

Пусть $f\in AC([a,b])$. Можно считать, что $f(a)=0$. Из
утверждения~\ref{ac=>vb} следует, что $f\in VB([a,b])$, значит, по
задаче~\ref{bounded_variation} $f$ представима в виде разности двух
монотонных функций $f_1$ и $f_2$. Достаточно доказать требуемое
утверждение для каждой из функций $f_1,f_2$, поэтому можно считать,
что $f$ монотонна и не убывает. Тогда существует неотрицательная
мера $\mu$ на $\Bs([a,b])$, такая что $f(x)=\mu([a,x))$ (ср. с
примером~\ref{Lebeg-Stilt'es}). Докажем, что $\mu\ll\la$. Тогда по
теореме Радона--Никодима получим функцию $g\bw \in L^1([a,b])$,
такую что $\mu(B)=\int\limits_Bg\,dt$ для любого множества
$B\in\Bs([a,b])$. В частности, $\mu([a,x))\bw
=\int\limits_a^xg\,dt$. Осталось проверить, что $\mu\ll\la$. Пусть
$B\in\Bs([a,b])$ --- множество с $\la(B)=0$. Покажем, что
$\mu(B)=0$. Зафиксируем произвольное $\ve>0$ и найдём
соответствующее ему $\de>0$ из определения абсолютной непрерывности
функции $f$. Из определения множества нулевой меры следует, что
существует не более чем счётное множество интервалов
$\{(\al_i,\be_i)\}$ суммарной длины меньше $\de$, покрывающих
множество $B$. Можно считать, что эти интервалы дизъюнктны (см.
утверждение~\ref{open_sets}). Тогда для любого $N\in\N$ имеем
$\sum\limits_{i=1}^N(\be_i-\al_i)\bw <\de$, поэтому
$\sum\limits_{i=1}^N(f(\be_i)-f(\al_i))\bw
=\sum\limits_{i=1}^N\mu((\al_i,\be_i))\bw <\ve$. Значит,
$\sum\limits_{i=1}^\infty\mu((\al_i,\be_i))\bw \le\ve$, т.е.
$\mu^*(B)\le\ve$, откуда получаем $\mu(B)=0$ в силу произвольности
выбора $\ve>0$.

Утверждение о дифференцируемости функции $f$ почти всюду и
интегрируемости $f'$ следует из теоремы~\ref{bound_var_diff},
доказательство которой можно прочитать в книге лектора <<Основы
теории меры>>.
\end{proof}

\begin{note}
Если $f$ --- произвольная абсолютно непрерывная функция, то в
качестве монотонно неубывающих функций $f_1,f_2$, таких что
$f=f_1-f_2$, можно взять $f_1(x):=V_{[a,x]}(f)$ и
$f_2(x):=V_{[a,x]}(f)-f$.
\end{note}
\begin{problem}
Доказать, что функция $V_{[a,x]}(f)$ является абсолютно непрерывной.
\end{problem}

\begin{imp}[Формула интегрирования по частям]
Пусть $f,g\in AC([a,b])$. Тогда верна следующая формула:
$\int\limits_a^bf'g\,dt\bw
=\left.fg\right|_a^b-\int\limits_a^bfg'\,dt$.
\end{imp}

\begin{proof}
Функция $fg$ является абсолютно непрерывной (упражнение). Тогда по
недоказанной теореме~\ref{abs_cont} существуют п.в. функции $(fg)'$,
$f'$ и $g'$. Поэтому п.в. функции $f$ и $g$ одновременно
дифференцируемы и $(fg)'\bw =f'g+fg'$. Проинтегрировав это равенство
от $a$ до $b$, получим требуемое.
\end{proof}

\subsubsection{Несколько заключительных замечаний}

Через $D([a,b])$ обозначается множество функций, определённых на
отрезке $[a,b]$ и всюду дифференцируемых на нём.

(1) Функция $\int\limits_a^xg(t)\,dt$ НЕ обязана быть
дифференцируемой всюду. Пример: разрывная функция $g(t)$.

(2) Равенство $\Br{\int\limits_a^xg(t)\,dt}'=g(x)$ не обязано
выполняться во всех точках, где функция $f\bw =
\int\limits_a^xg(t)\,dt$ дифференцируема, ибо $g$ можно исправить на
множестве меры нуль, а интеграл этого не заметит.

(3) Из того, что $f\in D([0,1])$, вообще говоря, не следует, что
$f\in AC([0,1])$. Может оказаться, что $f'\bw \notin \Ls^1([0,1])$.
Пример: $f(x)=x^2\sin\frac1{x^2}$, $f(0)=0$. Упражнение: проверить,
что $f'\notin \Ls^1([a,b])$. При этом $f'$ интегрируема в
несобственном смысле по Риману.

(4) Задача: если $f\in D([a,b])$ и $f'\in \Ls^1([a,b])$, то $f\in
AC([a,b])$.


\end{document}
