\documentclass[a4paper,12pt]{article}
\usepackage[utf]{dmvn}

%%\input marginmarker

\def\x{\ddot x}
\def\F{\ol F}
\def\w{\ol\om}
\def\g{\ol\ga}
\def\p{\ol\rho}
\def\a{\ol a}
\def\n{\ol n}
\def\r{\ol r}
\def\v{\ol v}
\def\e{\ol e}
\def\vR{\ol R}
\def\vK{\ol K}
\def\K{\ol K_O}
\def\MO{\ol M_O}
\def\M{\ol M}
\def\P{\ol P}
\def\J{\mathbb{J}}
\def\JO{{\mathbb{J}_O}}
\def\JS{{\mathbb{J}_S}}
\def\vtau{\ol\tau}
\def\vnu{\ol\nu}
\def\vbe{\ol\be}
\def\vep{\ol\ep}

\def\dd{\ddot}
\def\d{\dot}

\def\pic#1{ \hbox{\textbf{\textsc{[Рис.~#1]}}}}


\def\md{\text{~---}}

\newcommand{\ez}{\equiv0}
\newcommand{\nez}{\not\equiv0}

\tocsubsubsectionparam{3.4em}

\begin{document}
\dmvntitle{Курс лекций по}{классической механике}{Лектор Александр Владиленович Карапетян}
{IV курс, 7 семестр, поток математиков}{Москва, 2006~г.}

\pagebreak

\tableofcontents

%\pagebreak

\section*{Предисловие}

Набрано Ромой и Машей Ждановыми. Рисунков, к сожалению, пока нету. Текст был частично подправлен DMVN Corp.
Пока тщательной переработке подверглись только первые две главы, последняя на предмет опечаток
и лажи не верифицировалась. Разделение на главы и параграфы условно и не претендует на точное соответствие
названий и содержания. В порядке исключения шрифт здесь будет 12pt, потому что легче читать
(очень много мелких, но важных индексов).

\medskip
\dmvntrail

%\subsection*{Условные обозначения и соглашения}

%\begin{thebibliography}{АТФ}
%\setlength{\itemsep}{-2pt}
%\bibitem[АТФ]{atf} \fio.Алексеев.ВМ., \fio.Тихомиров.ВМ., \fio.Фомин.СВ. \emph{Оптимальное управление}. М.: Наука, 1979.
%\end{thebibliography}

\newpage

%\makeatletter
%  \renewcommand{\headheight}{11mm}
%  \renewcommand{\headsep}{2mm}
%  \renewcommand{\sectionmark}[1]{}
%  \renewcommand{\subsectionmark}[1]{}
%  \renewcommand{\subsubsectionmark}[1]{\markright{\thesubsubsection. #1}}
%  \renewcommand{\@oddhead}{\vbox{\hbox to \textwidth{\scriptsize\hbox to .5cm{\thepage\hfil}\textsc{[егодня~года, \now]}\hfil\rightmark\strut}\hrule}}
%  \renewcommand{\@oddfoot}{\hfil\thepage\hfil}
%\makeatother

%---------------------------------------------------------------------
\section{Кинематика}
%---------------------------------------------------------------------

\subsection{Основные понятия}

\subsubsection{Напоминание из аналитической и дифференциальной геометрии}

Мы часто будем использовать формулу <<бац минус цаб>>:
$$[a,[b,c]] = b(a,c) - c(a,b).$$

Кроме того, если $a \perp b$, то есть $(a,b) = 0$, то, с использованием предыдущей формулы
(хотя это легко проверить и непосредственно) получаем
$$[a,[a,b]] = a\ub{(a,b)}_0 - b(a,a) = -\hm{a}^2 b.$$

Ещё одно полезное соображение: если $e$ и $f$ векторы, зависящие от времени,
но $(e,f) \equiv \const$,
то $(\d e, f) + (e, \d f) \equiv 0$.

\subsubsection{Совсем основные понятия}

Пусть $M$ точка в $\R^3$,  $\r$ её радиус-вектор относительно
некоторой системы отсчета $Oxyz$ с ортами $\e_x$, $\e_y$, $\e_z$.
Время $t\in\R^+=[0,+\infty)$, функция $\r=\r(t)$ закон движения.
Кривая $\ga=\hc{\r(t)| t\in[a,b]}$ траектория движения. Все функции
дифференцируемы столько раз, сколько нужно.

\begin{df}
\emph{Скоростью} точки в данный момент времени
называется производная ее радиус вектора по времени:
$$\v=\d {\r}=\frac{d\r}{dt}.$$
\end{df}

\begin{df}
\emph{Ускорением} точки в данный момент времени
называется первая производная её вектора скорости по времени или,
что то же самое, вторая производная ее радиус-вектора по времени:
$$\a=\d \v = \dd \r=\frac{d^2\r}{dt^2}.$$
\end{df}

Таким образом,
$$\r=x\e_x+y\e_y+z\e_z,\qquad \v=\d x\e_x+\d y\e_y+\d z\e_z,\qquad \a=\dd x\e_x+\dd y\e_y+\dd z\e_z.$$

\begin{ex}
Простейшие виды движения точки:

\begin{enumerate}
\item Движение вдоль неподвижной прямой:
$$\r=\r_0+s\e.$$
\item Движение вдоль окружности $x^2+y^2=R^2$ в плоскости $z =0$ \pic{1}.
Введём (подвижную) систему координат $(\e_r,\e_\ph)$.
$$\e_r:=\e_x\cos\ph+\e_y\sin\ph,\qquad\e_\ph:=-\e_x\sin\ph+\e_y\cos\ph.$$
Тогда
$$\r=R\e_r,\quad \v=R\d\e_r=R \d\ph \e_\ph,\quad \a=R \dd\ph \e_\ph + R\d\ph\d\e_\ph = R\dd\ph\e_\ph -R\d\ph^2\e_r.$$
\end{enumerate}
\end{ex}

\subsubsection{Цилиндрические и сферические  координаты}

\begin{enumerate}
\item Цилиндрические \pic{2}: $\ph,\,\rho,\,z,\qquad\e_\ph,\,\e_\rho,\,\e_z,\qquad \r=\rho\e_\rho+z\e_z.$
\item Сферические \pic{3}: $r,\,\ph,\,\ta,\qquad\e_r,\,\e_\ph,\,\e_\ta,\qquad \r= r \e_r.$
\end{enumerate}

\subsubsection{Натуральный параметр и естественные оси (репер Френе)}

Введем обозначения: $\vtau=\frac{\v}{|\v|}$ орт касательной
к траектории. Пусть $\Pi$ cоприкасающаяся плоскость,
то есть плоскость, что $\vtau,\vnu\in\Pi$, где $\nu$ вектор главной
нормали, $\vnu\perp\vtau$.
Положим $\ol\be=[\vtau,\vnu]$ вектор бинормали.

\begin{df}
Оси $\hc{\vtau,\vnu, \ol\be}$
называются \emph{естественными осями} или \emph{репером
Френе}.
\end{df}

РЕПЕР ФРЕНЕ: \pic{4}

\begin{df}
Параметр называется натуральным, если $\hm{\frac{d\r}{ds}}=1$.
\end{df}

Здесь будем считать, что штрих это производная по натуральному параметру.
Имеем
$$\v=\d\r=\frac{d\r}{ds}\d s=\d s\r\,'=v\vtau,$$
так как  $|\vtau|=|\r'|=1$, то $\d s=v;\quad
\r'=\vtau;\quad s=\int vdt$.
$$\a=\d{\v}=\d
v\vtau+v\d{\vtau}=\d
v\vtau+v^2\vtau'=A\vtau+B\vnu.$$

Очевидно, $\vtau'\perp\vtau$, поэтому $A=\d v$, и $\vtau'=k\vnu$.
Из курса дифференциальной геометрии известно, что $k$
кривизна кривой. Итак,
$$\v=v\vtau,\qquad \a=\d v\vtau+kv^2\vnu.$$

Пусть $\rho:=\frac1k$ радиус кривизны, $\ka$ кручение кривой.
Напомним формулы Френе из дифференциальной геометрии:
$$
\begin{cases}
\vtau'=k\vnu,\\
\vnu'=-k\vtau+\ka\ol\be,\\
\ol\be'=-\ka\vnu.
\end{cases}
$$

\begin{imp} В наших обозначениях
$$
\begin{cases}
\d{\vtau}=kv\vnu,\\
\d{\vnu}=-kv\vtau+\ka v\ol\be,\\
\d{\ol\be}=-\ka v\vnu.
\end{cases}
$$
\end{imp}

\subsubsection{Угловая скорость подвижного репера}

Пусть $\hc{\e_1,\e_2,\e_3}$ репер подвижной системы координат,
а $\hc{\e_x,\e_y, \e_z}$ репер неподвижной системы координат.

\begin{df}
\emph{Угловой скоростью} подвижного репера называется вектор
$$\w:=\frac12\br{ [\e_1,\d\e_1]+ [\e_2,\d\e_2]+[\e_3,\d\e_3]}.$$
\end{df}

\begin{ex}
Рассмотрим цилиндрическую систему координат:
$$\begin{cases}
\e_r=\phm\e_x\cos\ph+\e_y\sin\ph,\\
\e_\ph=-\e_x\sin\ph+\e_y\cos\ph,\\
\e_z=\phm\e_z.
\end{cases}
\quad \Ra\quad
\begin{cases}
\d\e_r=\phm\d\ph\e_\ph,\\
\d\e_\ph=-\d\ph\e_r,\\
\d\e_z=\phm0.
\end{cases}
$$
Отсюда получаем, что $\w:=\frac12\br{[\e_r,\d\ph\e_\ph]+ [\e_\ph,-\d\ph\e_r]}= \d\ph[\e_r,\e_\ph] = \d\ph\e_z$.
\end{ex}

\begin{theorem}[Формулы Пуассона]
$\d\e_i=[\w,\e_i]$.
\end{theorem}
\begin{proof}
Воспользуемся формулой <<бац минус цаб>> и тем, что если
$(e_1,e_j)=0$, то $(\d\e_1,\e_j)+(\e_1,\d\e_j)=0$.
Имеем
\begin{multline*}
[\w,\e_1]=-[\e_1,\w]=-\frac12\br{\bs{\e_1,[\e_1,\d\e_1]}+\bs{\e_1,[\e_2,\d\e_2]}+\bs{\e_1,[\e_3,\d\e_3]}}=\\
=-\frac12\hr{-\d{\e}_1-\e_1(\d\e_1,\e_1)-\e_2(\d\e_1,\e_2)-\e_3(\d\e_1,\e_3)}=\d\e_1,
\end{multline*}
так как последних три слагаемых суть разложение вектора $\d\e_1$
по базису $\hc{\e_1, \e_2, \e_3}$.
\end{proof}

\subsubsection{Связь абсолютной и локальной производных по времени}

Через $\frac{d_a}{dt}$ будем обозначать абсолютную производную по времени,
то есть по отношению к неподвижному реперу $Oe_xe_ye_z$.
Через $\frac{d_r}{dt}$ относительную (по отношению к подвижному реперу $Se_1e_2e_3$).

\begin{df}
Абсолютная и относительная производные радиус вектора точки это её \emph{абсолютная}
и \emph{относительная} скорости.
\end{df}

\begin{stm} Имеет место соотношение
$$\frac{d_a\ol{AB}}{dt}=\frac{d_r\ol{AB}}{dt}+[\w,\ol{AB}],$$
где $\w$ угловая скорость репера $e_1e_2e_3$.
\end{stm}
\begin{proof}
Пусть $\ol{AB}=\xi_1\e_1+\xi_2\e_2+\xi_3\e_3$.
Тогда
$$
\frac{d_a\ol{AB}}{dt}=\sum\d\xi_i\e_i+\sum\xi_i\d\e_i\stackrel{!}{=}
\frac{d_r\ol{AB}}{dt}+\sum\xi_i[\w,\e_i] =\frac{d_r\ol{AB}}{dt}+\hs{\w, \sum\xi_i \e_i}=
\frac{d_r\ol{AB}}{dt}+[\w,\ol {AB}].$$
Переход <<!>> следует из формулы Пуассона.
\end{proof}

\subsection{Кинематика абсолютно твердого тела}

\subsubsection{Абсолютно твёрдое тело. Формула Эйлера и Ривальса}

\begin{df}
\emph{Абсолютно твердым телом} (ТТ) называется система
из $n\ge 3$ точек $M_i$ (среди которых существуют три точки, не
лежащие на одной прямой), такая что при любых движениях этой системы
имеют место соотношения $\hm{\ol{M_iM_j}}=\rho_{ij}=\const$.
\end{df}

Для того, чтобы знать закон движения твердого тела (то есть знать
законы движения всех его точек), достаточно знать законы движения трёх его точек,
не лежащих на одной прямой. Они задаются 6~параметрами.
Группа движений: $\R^3\times \SO(3)$.

\begin{df}
\emph{Угловой скоростью} твердого тела $\w_{TT}$
называется угловая скорость репера, связанного с телом.
\end{df}

\begin{stm}[Формула Эйлера]
Имеет место формула $\v_B=\v_A+[\w_{TT},\ol{AB}]$.
\end{stm}
\begin{proof}
Имеем $\v_B-\v_A=\frac{d_a\ol{AB}}{dt}=\frac{d_r\ol{AB}}{dt}+[\w,\ol{AB}]$. Но
первое слагаемое равно нулю, так как тело твердое, и его точки не движутся друг относительно друга.
\end{proof}

\begin{imp}[Корректность определения $\w_{TT}$]
Вектор $\w_{TT}$ корректно определён.
\end{imp}
\begin{proof}
Допустим, существует два репера, в которых $\w_1\ne\w_2$. Тогда
\begin{gather*}
\v_B=\v_A+[\w_1,\ol{AB}],\\
\v_B=\v_A+[\w_2,\ol{AB}].
\end{gather*}
Поэтому $[\w_1-\w_2, \ol{AB}]=0$ для любых точек $A$ и $B$ тела.
Но поскольку среди точек твердого тела есть три, не лежащие на одной
прямой, получаем $\w_1-\w_2=0$.
\end{proof}

\begin{stm}
Угловое ускорение не зависит от репера.
\end{stm}
\begin{proof}
Мы знаем, как связаны абсолютная и относительная производные. Подставим
в ту формулу угловую скорость $\w$ вместо вектора $\ol{AB}$:
$$\frac{d_a\w}{dt}=\frac{d_r\w}{dt}+[\w,\w]=\frac{d_r\w}{dt},$$
значит, угловое ускорение можно измерять, как сидя в
неподвижной системе, так и сидя в подвижной.
\end{proof}

Угловое ускорение $\d\w$ (производную можно обозначать просто точкой в силу
только что доказанного утверждения) мы будем обозначать через~$\vep$.

\begin{imp}[Формула Ривальса]
$$\a_B=\a_A+[\vep,\ol{AB}]+[\w,[\w,\ol{AB}]].$$
\end{imp}
\begin{proof}
Из формулы Эйлера имеем $\v_B=\v_A+[\w,\ol{AB}]$. Продифференцируем это соотношение,
и учтём то, что
$$\frac{d_a \ol{AB}}{dt} = \ub{\frac{d_r\ol{AB}}{dt}}_0 + [\w,\ol{AB}],$$
потому что тело твёрдое.
Получаем в точности доказываемую формулу.
\end{proof}

\subsubsection{Примеры движений твердого тела}

\pt1. \textbf{Поступательное движение.} Для него имеем $\w=0$.
Откуда для всех $A,B\in TT$ имеем $\v_A=\v_B$, $\a_B=\a_A$. Если
существует точка $t_*$ такая, что $\w(t_*)=0$, то говорят, что
тело совершает \emph{мгновенно поступательное} движение, при этом
$\v_A(t_*)=\v_B(t_*)$. Но для ускорений это равенство, вообще
говоря, неверно.

\medskip

\pt2. \textbf{Вращение вокруг неподвижной оси.}

Тут должен быть \pic{7}.

Если существуют две различные
точки $A$ и $B$ твердого тела такие, что $\v_A=\v_B=0$, то
из формулы Эйлера получаем, что $\w=\la \ol{AB}$. Отсюда $\v_C=0$ для любой точки $C$,
принадлежащей прямой $AB$. Без ограничения общности, $e_z = e_3$.
Запишем соотношения для точки тела $P$. Пусть $P'$ её проекция
на ось вращения $Oz$. Тогда
$$\w=\om\e_z,\quad \v_P=[\w,\ol{OP}]=\om[\e_z,\ol{OP}]=\om[\e_z,\ol{P'P}].$$
$$\a_P=\d\om[\e_z,\ol {P'P}]+\om^2[\e_z,[\e_z, \ol{P'P}]]=\d\om[\e_z,\ol{P'P}]-\om^2\ol{P'P}.$$

 Если при $t=t_*$ существует прямая $\ell=\ell(t)$ такая, что
$\v_C=0$ для всех $C\in \ell$, то говорят, что тело совершает
мгновенное вращение вокруг оси $\ell$.

\medskip

\pt3. \textbf{Плоско-параллельное движение твердого тела.}

Тут должен быть \pic{7}.

Это такое движение, при котором все точки ТТ движутся в плоскостях,
параллельных какой-либо неподвижной плоскости. В этом случае движение
полностью описывается движением сечения.

Пусть $TT^*$ сечение ТТ плоскостью $Oxy$ (мы считаем, что система координат
выбрана именно так). Подвижная система координат, связанная с телом $S\e_1\e_2$.
Тогда имеем $\r_s=x_s\e_x+y_s\e_y$.
Домножая скалярно равенство $\v_B-\v_A=[\w,\ol{AB}]$ на~$\w$,
имеем
$$(\w,\v_B-\v_A)=(\w,[\w,\ol{AB}])=0,$$
откуда $\w=\om\e_z$. Подставляя вектор угловой скорости в формулу для скоростей, получаем
$$\v_B=\v_A+\om[\e_z,\ol{AB}],$$
а для ускорений
$$\a_B=\a_A+\d\om[\e_z,\ol{AB}]+\om^2[\e_z,[\e_z,\ol{AB}]]=\a_A+\d\om[\e_z,\ol{AB}]-\om^2\ol{AB}.$$

\begin{stm}
Если $\w\ne0$, то существует $C\in TT^*$ такая, что $\v_C=0$.
\end{stm}
\begin{proof}
Пусть $S$ любая точка в теле. Если $\v_S = 0$, то всё ясно мы её нашли.
Если же $\v_S \ne 0$, то хотим найти такую точку $C$, что
$0=\v_C=\v_S+[\w,\ol{SC}]$. Умножим это равенство векторно на $\w$ слева.
Получим $[\w,\v_S]+[\w,[\w,\ol{SC}]]=0$, то есть $[\w,\v_S]=\om^2\ol{SC}$.
Отсюда можно найти вектор $\ol{SC}$:
$$\ol{SC}=\frac{[\w,\v_S]}{\om^2},$$
то есть неподвижная точка найдена.
\end{proof}

\begin{df}
Точка $C=C(t)$, для которой $\v_C = 0$ называется \emph{мгновенным центром скоростей}.
Геометрическое место точек $C$ в теле называется \emph{подвижной центроидой}.
Геометрическое место точек $C$ в на плоскости $Oxy$
называется \emph{неподвижной центроидой}.
\end{df}

\medskip

\pt4. \textbf{Вращение вокруг неподвижной точки (сферическое  движение).}

Тут должен быть \pic{8}.

В этом случае существует точка $O\in TT$ такая, что $\v_O\equiv 0$.
Примем её за начало системы отсчета.
Тогда для всякой точки $P \in TT$ имеем
$$\v_P=\v_O + [\w,\ol{OP}] = [\w,\ol{OP}],\quad
\a_P=[\vep,\ol{OP}]+[\w,[\w,\ol{OP}]].$$
Пусть $\ell$ прямая, проходящая через точку $O$ с направляющим вектором $\w$.
Тогда очевидно, что $\v_P=0$ для всех $P\in \ell$.

\begin{df}
Прямая $\ell$ называется \emph{мгновенной осью вращения}.
Геометрическое место прямых $\hc{\ell(t)}$ в теле (или в неподвижном
пространстве) называется \emph{подвижным (или неподвижным) аксоидом}.
\end{df}

Пример такого движения демонстрирует гироскоп.

\begin{df}
Если существует момент времени $t_*$ и $Q\in TT$ такая, что
$\v_Q(t_*)=0$, то говорят, что при $t=t_*$ тело совершает мгновенно
вращательное движение вокруг точки $Q$.
\end{df}

\medskip

\pt5. \textbf{Мгновенно винтовое движение (случай общего положения).}

\begin{stm}
Если $\w\ne0$ и $\v_P\ne0$ для всех точек $P\in TT$, то существует
прямая $\ell(t)\in TT$ с направляющим вектором $\e = \frac{\w}{\om}$ и
для всех точек прямой $\ell$ выполнено $\v_C=v\e$.
\end{stm}

\begin{proof}
Возьмём произвольную точку $S\in TT$. Тогда $\v_S\ne 0$.
Проведем плоскость $\Pi$ через точку $S$, перпендикулярную вектору $\w$.
Покажем, что найдётся точка $Q\in\Pi$, для которой $\v_Q=\si\w$.
Действительно, уравнение для неё имеет вид $\v_Q=\v_S+[\w,\ol{SQ}]=\si\w$,
умножая его векторно на $\w$, получаем
$$[\w,\v_S]+[\w,[\w,\ol{SQ}]]=0,$$
а отсюда
$[\w,\v_S]-\om^2\ol{SQ}=0$, ибо $\ol{SQ}\perp\w$.
Отсюда $\ol{SQ}=\frac{[\w,\v_S]}{\om^2}$, то есть точка $Q$ найдена.
Пусть $\ell$ прямая с направляющим вектором $\w$, проходящая через точку $Q$.
Далее, рассмотрим произвольную точку $C \in \ell$, тогда
$$
\v_C=\v_Q+[\w,\ol{QC}]=\v_Q = \si\w = v\e,
$$
что и требовалось доказать.
\end{proof}

\begin{df}
Такое движение называется \emph{мгновенно винтовым}, а прямая $\ell$ \emph{мгновенной винтовой осью}.
\end{df}


\begin{theorem}
Произвольное движение твердого тела либо
мгновенно-поступательное, либо мгновенно вращательное, либо
мгновенно-винтовое.
\end{theorem}

\subsubsection{Сложное движение точки}

Картинка \pic{10}.

Рассмотрим движение точки $P \in TT$.
Пусть $O\e_x\e_y\e_z$ неподвижный репер, $S\e_1\e_2\e_3$ подвижный репер.
Пусть $\w$ угловая скорость подвижного репера.
Радиус вектор точки $P$ в подвижной системе координат обозначим через $\p$.
Ясно, что $\r=\r_S+\p$.

\begin{df}
Переносной скоростью точки $P$ называется вектор $\v_e := \v_a + [\w,\p]$.
\end{df}

\begin{theorem}[О сложении скоростей]
Имеет место равенство
$$\v_a=\v_r+\v_e,$$
где $\v_a,\v_r,\v_e$ абсолютная, относительная и переносная
скорости точки $P$ соответственно.
\end{theorem}
\begin{proof}
В самом деле,
$$\v_a^P=\frac{d\r}{dt}=\frac{d_a\r_S}{dt}+\frac{d_a\p}{dt}=
\v_a^S+\frac{d_r\p}{dt}+[\w,\p]=\v_r^P+\v_a^S+[\w,\p]=: \v_r^P+\v_e^P,$$
что и требовалось доказать.
\end{proof}

\begin{note}
Переносная скорость точки $P$ это скорость той точки подвижного репера,
с которой в данный момент совпадает точка $P$.
\end{note}

\begin{df}
Вектор $\a_e^P := \a_a^S+[\d\w,\p]+[\w,[\w,\p]]$ называется переносным ускорением.
\end{df}

\begin{df}
Вектор $\a^P_c := 2[\w,\v_r]$ называется кориолисовым ускорением.
\end{df}

\begin{theorem}[Кориолис]
Имеет место равенство:
$$\a_a=\a_r+\a_e+\a_c.$$
\end{theorem}
\begin{proof}
Рассмотрим формулу $\v_a^P=\v_r^P+v_a^S+[\w,\p]$. Имеем
\begin{multline*}
\a_a^P=\frac{d_a\v_a^P}{dt}=\frac{d_a}{dt}\br{\v_r^P+\v_a^S+[\w,\p]}=
\frac{d_r\v_r^P}{dt}+[\w,\v_r^P]+\frac{d_a\v_a^S}{dt}+[\d\w,\p]+\hs{\w,\frac{d_r\p}{dt}+[\w,\p]}=\\=
\a_r^P+[\w,\v_r^P]+\a_a^S+[\d\w,\p]+[\w,\v_r]+[\w,[\w,\p]]=
\a_r^P+\ub{\a_a^S+[\d\w,\p]+[\w,[\w,\p]]}_{\a_e^P} +\ub{2[\w,\v_r]}_{\a_c^P},
\end{multline*}
что и требовалось доказать.
\end{proof}

\subsubsection{Теорема  о сложении угловых скоростей.\\
Кинематические формулы Эйлера}

Пусть $O\e_x\e_y\e_z$ неподвижный репер,
$\Om\e_\xi\e_\eta\e_\ze$ подвижный репер, $S\e_1\e_2\e_3$ репер,
жёстко связанный с твёрдым телом.

Напомним, что
$$\w_a=\frac12\sum\hs{\e_i,\frac{d_a\e_i}{dt}}\md$$
это абсолютная угловая скорость,
$$\w_r=\frac12\sum\hs{\e_i,\frac{d_r\e_i}{dt}}\md$$
это относительная угловая скорость (по отношению к реперу $\Om\e_\xi\e_\eta\e_\ze$), а
$$\w_e=\frac12\hr{\hs{\e_\xi,\frac{d_a\e_\xi}{dt}}+\hs{\e_\eta,\frac{d_a\e_\eta}{dt}}+\hs{\e_\ze,\frac{d_a\e_\ze}{dt}}}\md$$
это переносная угловая скорость, то есть угловая скорость репера $\xi\eta\ze$ в неподвижной системе отсчета.

\begin{theorem}[О сложении угловых скоростей]
Имеет место формула $\w_a=\w_r+\w_e$.
\end{theorem}
\begin{proof}
Будем под $*$ понимать одну из букв $a,r,e$. Пусть $A,B$ произвольные точки твердого
тела, тогда
$$\v^B_*=\v^A_*+[\w_*,\ol{AB}],\eqno(1)$$
Сложим эти три равенства (два последних возьмём со знаком <<$-$>>), получим
$$\ub{\v_a^B-\v_r^B-\v_e^B}_0=\ub{\v_a^A-\v_r^A-\v_e^A}_0+[\w_a-\w_r-\w_e,\ol{AB}],$$
то есть $[\w_a-\w_r-\w_e,\ol{AB}] = 0$.
А поскольку это равенство верно для любых точек, то на самом деле получаем
$$\w_a-\w_r-\w_e=0,$$
что и требовалось доказать.
\end{proof}

\begin{imp}
Для угловых ускорений справедлива формула $\vep_a=\vep_r+\vep_e$.
\end{imp}

\begin{imp}\label{imp:angleSpeed}
Пусть $S_0$ неподвижная система отсчета, а
$S_1,\dots, S_n$ подвижные. Пусть $\w$ угловая скорость $S_n$
относительно $S_0$, а $\w^i$ угловая скорость $S_i$ относительно
$S_{i-1}$ ($i=1\sco n$). Тогда $$\w=\sum_{i=1}^n\w^i.$$
\end{imp}

\subsubsection{Углы Эйлера}

Картинка \pic{11}.

Определим \emph{углы Эйлера} подвижного репера $O\e_1\e_2\e_3$ относительно
неподвижного репера $Oxyz$.

Пусть $[\e_3,\e_z]\ne0$.
Пусть $\xi=Oxy\cap Oe_1e_2$.

\begin{df}
Угол $\psi=\angle(\e_x,\e_\xi)$ называется углом \emph{прецессии},
угол $\ta=\angle(\e_z,\e_3)$ углом \emph{нутации},
угол $\ph=\angle(\e_\xi,\e_1)$ углом \emph{собственного вращения}.
\end{df}

При этом имеем $\ta\in[0,\pi)$, $\ph\in[0,2\pi)$, $\psi\in[0,2\pi)$.

Переведем репер $O\e_1\e_2\e_3$ в $Oxyz$ с помощью композиции трёх поворотов
с угловыми скоростями $\w_I$, $\w_{I\!I}$ и $\w_{I\!I\!I}$:

\begin{points}{-2}
\item $Oxyz$ переходит в $O\xi\eta z$, (при этом $O\eta\in Oxy$) под
      действием поворота вокруг $Oz$ на угол~$\psi$. Угловая скорость
      репера $O\xi\eta z$: $\w_{I}=\d\psi\e_z$.
\item $O\xi\eta z$ переходит в $O\xi\eta'e_3$ под действием поворота вокруг
      $O\xi$ на угол $\ta$ (при этом $O\eta'\in Oe_1e_2$). Угловая скорость
      репера $O\xi\eta'e_3$ относительно $O\xi\eta z$:
      $\w_{I\!I}=\d\ta\e_\xi$.
\item $O\xi\eta'e_3$ переходит в $O\e_1\e_2\e_3$ с помощью поворота вокруг $O\e_3$
      на угол $\ph$. Угловая скорость: $\w_{I\!I\!I}=\d\ph\e_3$.
\end{points}

По следствию~\ref{imp:angleSpeed} из теоремы о сложении скоростей получаем, что угловая
скорость ТТ (то~есть репера $O\e_1\e_2\e_3$) равна
$$\w=\w_1+\w_2+\w_3=\d\psi\e_z+\d\ta\e_\xi+\d\ph\e_3.$$
Очевидно,
$$\begin{aligned}\e_\xi&=\cos\ph\,\e_1-\sin\ph\,\e_2,\\
\e_z&=\sin\ta\sin\ph\,\e_1+\sin\ta\cos\ph\,\e_2+\cos\ta\,\e_3.
\end{aligned}$$
Поэтому $\w=\om_1\e_1+\om_2\e_2+\om_3\e_3$, где
$\om_1,\om_2,\om_3$ задаются с помощью \emph{кинематических формул
Эйлера}:
$$
\begin{aligned}
\om_1&=\phm\d\ta\cos\ph+\d\psi\sin\ta\sin\ph,\\
\om_2&=-\d\ta\sin\ph+\d\psi\sin\ta\cos\ph,\\
\om_3&=\phm\d\ph+\d\psi\cos\ta.
\end{aligned}
$$

\subsubsection{Замечание о качении тел}

Пусть $\Sig$ неподвижная поверхность, $S$ подвижная поверхность (поверхность,
ограничивающая твердое тело).

\begin{df}
Говорят, что тело совершает \emph{качение без скольжения},
если $\v_K=0$ ($K\in S$). Если $\Sig$ подвижная поверхность,
то тело совершает качение без скольжения, если $\v_{K_s}=\v_{K_\Sig}$.
\end{df}

%---------------------------------------------------------------------
\section{Динамика точки}
%---------------------------------------------------------------------

\subsection{Движение точки под действием сил}

\subsubsection{Принцип детерминированности Ньютона. Прямая и обратная задачи динамики}

\emph{Принцип детерминированности Ньютона} гласит, что:
$$\begin{aligned}
\r(t)&=\Phi \br{\r(t_0),\d \r(t_0), t_0, t},\\
\dd\r(t)&=\frac{\pd^2\Phi}{\pd t^2}\br{\r(t_0),\d\r(t_0), t_0, t}.
\end{aligned}
$$
Подставим $t=t_0$ во второе уравнение, получим:
$$\dd \r(t_0)=\ol f(\r(t_0), \d \r(t_0), t_0).$$
Но так как $t_0$ может быть любым, то
$$
\dd \r(t)=\ol f(\r(t), \d \r(t), t),
$$
В других обозначениях (просто умножим последнее уравнение на константу $m$, называемую массой),
получим
$$m\dd \r(t)=\ol F(\r(t), \d \r(t), t).$$
Это равенство называется \emph{вторым законом Ньютона}. Его правая часть это сила (мера взаимодействия тел).
В динамике фактически постулируется, что сила зависит только от положения точки, скоростей и времени.

\emph{Прямая задача динамики} состоит в том, что нужно найти закон движения, если заданы силы и начальные условия.
\emph{Обратная задача динамики} состоит в нахождении сил при заданных свойствах движения.

\begin{ex}
Если точка изолирована (не взаимодействует ни с какими материальными системами),
то $\ol F=0$ и
$$\r(t)=\r(t_0)+\v_0\cdot(t-t_0).$$
\end{ex}

\begin{ex}
Рассмотрим так называемую задачу Галилея движение точки в однородном силовом поле.
Тогда $m\dd \r=m\ol g$ ($g$ постоянный вектор в $Oxyz$).
Общее решение этой задачи таково:
$$\r(t)=\r_0+\v_0(t-t_0)+\frac{g(t-t_0)^2}{2}.$$
\end{ex}

\subsubsection{Примеры сил}

\begin{nums}{-2}
\item \emph{Стационарные}: $\F=\F(\r,\v)$ (то есть $\F$ зависит только от положения и скорости).
\item \emph{Позиционные}: $\F=\F(\r)$ (то есть $\F$ зависит только от положения).
      Если $\F=-\grad V(\r)$, тогда $V(\r)$ называется потенциальной энергией,
      а сила $\F$ потенциальной силой.
\item \emph{Диссипативные}: $\br{\F(\r,\v),\v} \le 0$ (или $\nez$).
      Примером такой силы служит \emph{сила трения}. например, если $\F=-\ka\v$,
      то это \emph{вязкое} трение. Если $\F=-\ka\frac{\v}{|\v|}$ при $\v\ne0$, то это \emph{сухое} трение.
\item \emph{Гироскопические}: $\br{\F(\r,\v),\v}=0$. Например, такова кориолисова сила,
      потому что $\F_c=2m[\w,\v_r]$, поэтому $(\F_c,\v_r)\ez$.
\end{nums}



\subsubsection{Основные динамические величины. Работа и момент силы}

\begin{df}
Вектор $\P:=m\v=m\d\r$ называется \emph{импульсом точки} (или \emph{количеством движения}).
\end{df}

\begin{df}
Вектор $\K:=[\r,m\v]=[\r,\P]$ называется \emph{кинетическим моментом}
(или \emph{моментом количества движения}). Здесь индекс указывает на то,
относительно какой точки рассматривается кинетический момент.
\end{df}

\begin{df}
Число $T=\frac12mv^2=\frac12m(\v,\v)$ \emph{кинетическая энергия}.
\end{df}

Работа силы: $A_{эл}=(\F,d\r)$ элементарная работа. Тогда
$$A=\intl{M_1}{M_2}A_{эл}=\intl{M_1}{M_2}(\F,d\r)\md$$
это работа на отрезке $M_1M_2$.

\begin{df}
Вектор $\MO(\F):=[\r,\F]$ \emph{момент} силы $\F$, приложенной в точке с радиус вектором $\r$,
относительно точки $O$.
\end{df}

Если из контекста ясно, о какой силе идёт речь, аргумент $\F$ писать не будем.

Пусть $\ell$ ось, проходящая через точку $O$ с направляющим вектором $\e$.
Определим величины $M_\ell:=(\MO,\e)$ момент силы относительно прямой $\ell$,
$K_\ell=(\K,\e)$.

Легко видеть, что $M_\ell$ и $K_\ell$ не зависят от выбора точки $O$ на
прямой $\ell$, потому что если $O'\in \ell$, то
$\M_{O'}=[\r-\ol{OO'},\F]$, поэтому
$$M_\ell=(\M_{O'},\e)=(\MO,e) + \br{[\ol{OO'},\F], \e} =  (\MO,\e),$$
потому что $\ol{OO'} \parallel \e$.

\subsubsection{Основные теоремы динамики точки}

\begin{theorem}[Следствие из второго закона Ньютона]
$\d\P=\F$.
\end{theorem}
\begin{proof}
Сразу следует из определения импульса.
\end{proof}

\begin{theorem}[2]
$\d{\ol{K}}_O=\MO$.
\end{theorem}

\begin{proof}
$\K=[\r,m\d\r]$, поэтому $\d\K=[\r,m\dd\r]+[\d\r,m\d\r]=[\r,m\dd\r]=[\r,\F]=\MO$.
\end{proof}

\begin{theorem}[3]
$dT=A^{\text{эл}}=(\F,d\r)$, что равносильно
равенству $\d T=(\F,\v)$ мощность.
\end{theorem}

\begin{proof}
$T=\frac12m(\v,\v)$, поэтому $\d T=m(\v,\d\v)=(\v,\F)$.
\end{proof}


Выведем из этих теорем несколько полезных \textbf{следствий}
и проиллюстрируем их на задаче Галилея.

Пусть $\e$ неподвижный единичный вектор,
$\ell$ прямая с направляющим вектором $\e$. Тогда:

\begin{points}{-2}
\item $\d P_\ell=F_\ell$, где $P_\ell=(\P,\ol e)$, $F_\ell=(\F,\e)$ проекции
      силы и импульса на прямую $\ell$).
      В~частности, если $F_\ell=0$, то $P_\ell=\const$.

      \begin{ex}
        Задача Галилея при $\ol g=-g\e_z$: $P_x=c_1$, $P_y=c_2$.
      \end{ex}

\item $\d K_\ell=M_\ell$. В частности, если $M_\ell=0$, то $K_\ell=\const$.

      \begin{ex} Рассмотрим ту же задачу, тогда $K_z=c_3$. \end{ex}

\item Если $\F=-\grad V(\r)$, то $T+V=\const$.
      В самом деле, имеем
      $$\d T=(\F,\v)=-(\grad V, \d\r)=-\frac{\pd V}{\pd r_i}\frac{d r_i}{dt}=-\d V,$$
      значит, $\d T+\d V=0$, откуда $T+V=\const$.

      \begin{ex} В той же задаче Галилея получаем
      $$
        \frac12m(\d x^2+\d y^2+\d z^2)+mgz=h=\const.
      $$
      \end{ex}
\end{points}

\subsubsection{Одномерное движение точки}

Пусть сила $F$ позиционная, а точка живёт на прямой.
Тогда
$$
m\dd x=F(x),\quad V(x)=-\int F(x)dx,\quad F=-V'=-\frac{dV}{dx}.
$$

В этом случае имеет место первый интеграл (интеграл энергии):
$$\frac12m\d x^2+V(x)=h=\const.\eqno(1)$$
Это дифференциальное уравнение несложно решить. Проведём выкладки:
$$
\d x=\pm{\sqrt{\frac{2(h-V(x))}{m}}},\qquad
\intl{x_0}{x}\frac{dx}{\sqrt{\frac{2(h-V(x))}{m}}}=t-t_0.
$$

\begin{df}
Те точки, в которых уравнение имеет решения, называются областью возможности движения (ОВД).
\end{df}

\begin{ex}
В нашем случае ОВД это множество $\hc{x\cln V(x)\le h}$.
\end{ex}

\begin{df}
Пара переменных $(x,v)$ (или, что то же самое, $(x,\d x)$) это фазовые переменные. Плоскость
$\hc{(x,v)\vl x\in\R,\d x\in\R}$ называется фазовой плоскостью.
Фазовый портрет это множество кривых, определяемых интегралом
энергии~$(1)$, то есть его линий уровня.
\end{df}

\begin{ex}
Пружинка со связанным концом: $\dd x=kx$ ($k=\const$).

Пусть $k<0$, тогда имеем уравнение $\d x^2-kx^2=2h$, и фазовые кривые эллипсы.

Пусть $k>0$, тогда фазовые кривые гиперболы.

\begin{note}
Случай $k>0$ лишён физической интерпретации: он соответствует пружине
с отрицательным коэффициентом жёсткости!
\end{note}

Если $V'(x)=0$, то это критическая точка потенциала.
При этом, если $V''(x_1)>0$, то наблюдается минимум
потенциальной энергии, а если $V''(x_1)<0$, то максимум.

Период колебаний такой системы это период между двумя минимумами потенциальной энергии.
$$
\tau=2\int_{x_0}^x\frac{dx}{\sqrt{\frac{2(h-V(x))}{m}}},\quad h=V(a)=V(b).
$$
\end{ex}

\subsection{Задача Кеплера}

\subsubsection{Вывод закона всемирного тяготения из законов Кеплера}

Это так называемая \emph{обратная задача Кеплера}.

Опишем условия (\emph{законы Кеплера}):
\begin{nums}{-2}
\item Планеты движутся по эллипсам, в одном из фокусов Солнце.
\item За одинаковое время планеты заметают в эллипсах равные площади.
\item Число $\frac{T^2}{a^3}=\const$ (не зависит от планеты), где $T$ период, $a$ большая полуось.
\end{nums}

Выведем закон Всемирного тяготения из этих условий. Запишем в полярных
координатах уравнение эллипса.
$$
r=\frac{p}{1+e\cos \ph},\quad b^2=a^2(1-e^2),\quad p:=a(1-e^2),
$$
Из второго условия следует, что $r^2\d\ph=c=\const$. Эта величина есть удвоенная \emph{секторная скорость}.
Площадь эллипса: $\pi ab=\frac12 c\cdot T$.

Пусть $r,\ph,z$ цилиндрические координаты (по сути дела, полярные, конечно, потому что движение плоское)
с центром в фокусе, в котором находится Солнце. Тогда
$$
\begin{cases}
m(\dd r-r\d\ph^2)&=F_r=(\ol F,\e_r),\\
m(2\d r\d \ph+r\dd \ph)&=F_{\ph}=(\ol F,\e_\ph),\\
m\dd z&=F_z=(\ol F,\e_z).
\end{cases}
$$
Поясним, как получены первые две формулы (левые части). В полярной системе координат
берём радиус вектор $\r = r \e_r$ и дифференцируем его два раза, помня о том, что
$\d\e_r = \d\ph\e_\ph$ и $\d\e_\ph = -\d\ph \e_r$. После того, как это будет сделано без
ошибок, в качестве координат получим как раз компоненты ускорения, и останется только написать второй
закон Ньютона.

Так как движение плоское, то $z\ez\Ra F_z=0$. Так как
$r^2\d\ph=\const$, то имеем $2r\d r\d\ph+r^2\dd \ph\ez$, откуда
$F_\ph=0$. Поэтому $\F=F_r\e_r$ центральная сила.

Пусть штрих это производная по $\ph$, а $\rho:=\frac1r$. Тогда $\d\ph = c\rho^2$ и\label{page:KeplerReverse}
\eqn{\label{eqn:KeplerReverse}
\begin{aligned}
\d r&=\frac{d}{dt}\frac1{\rho}=-\frac{\d\rho}{\rho^2}=-\frac{\rho'\d\ph}{\rho^2}=-\rho'c.\\
\dd r&=(\d r)'\d\ph=-c\rho''(c\rho^2)=-c^2\rho^2\rho''.
\end{aligned}}
\begin{gather*}
r\d\ph^2=\frac1\rho\rho^4c^2=c^2\rho^3.\\
F_r=m(-c^2\rho^2\rho''-c^2\rho^3)=-mc^2\rho^2(\rho''+\rho)\stackrel{!}{=}-\frac{mc^2\rho^2}{p}.
\end{gather*}
Справедливость последнего равенства, отмеченного <<!>>, вытекает из
того, что $\rho=\frac{1+e\cos\ph}{p}$ и потому $\rho''=-\frac{e\cos \ph}{p}$.

Теперь начинаем подставлять константы:
$$F_r=-\frac{mc^2}{pr^2}=-\frac{4\pi^2a^2b^2}{T^2}\cdot \frac1{a(1-e^2)}\cdot \frac{m}{r^2}=
-\frac{4\pi^2a^3}{T^2}\frac{m}{r^2}=:-\mu\frac m{r^2}.$$
Поэтому $$\ol F=-\ga\frac{Mm}{r^2}\frac{\r}{r},$$ где
$\ga$ и $\mu$ связаны соотношением: $\mu=\ga M$ ($M$ масса
Солнца).

В задаче Кеплера потенциал равен $V=-\mu\frac{m}{r}=-\ga\frac{Mm}{r}$.

\begin{stm}
$\F$ потенциальная сила.
\end{stm}
\begin{proof}
Имеем
$$
A=\intl{p_1}{p_2}(\F,d\r)=-\mu m \intl{p_1}{p_2}\frac{(\r,d\r)}{r^3}\stackrel{!}{=}
-\mu m\intl{p_1}{p_2}\frac{r\,dr}{r^3}=
-\mu m\int\frac{dr}{r^2}=\frac{\mu m}{r}=-V.
$$
Здесь переход <<!>> следует из формулы $(\r,\,d\r)=\frac12d(\r,\r)=\frac12d(r^2)=r\,dr$.
Таким образом, $\F\bw=-\grad V$.
\end{proof}

\begin{note}
Если $\ol F=F(r)\frac{\r}r$ (то~есть сила центральная), то $\ol F\bw=\grad V(r)$,
где $V\bw=-\int F(r)\,dr$.
\end{note}

\subsubsection{Прямая задача Кеплера}

Пусть тело движется под действием центральной
силы $F=-\mu\frac{m}{r^2}\e_r$ ($S$ начало координат).

\begin{stm}
Движение точки под действием центральной силы
происходит в одной и той же неподвижной плоскости, которая зависит
от начальных условий.
\end{stm}

\begin{proof}
$\M_S=[\r,\F]= 0$ (потому что $\r$ и $\F$ просто коллинеарны), поэтому $\ol K_S=:\ol k=\const$
(потому что момент силы это производная кинетического момента).
Поскольку $\ol K_S=[\r,m\v]$, то точка движется в плоскости, перпендикулярной вектору~$\ol k$.
\end{proof}

\begin{stm}
Радиус-вектор точки, движущейся под действием
центральной силы, за равные промежутки времени заметает равные площади.
\end{stm}

\begin{proof}
Поскольку сила $\F$ центральная, выполнено равенство $m \a_\ph = F_\ph = 0$,
где $\a_\ph = 2\d r\d\ph+r\dd\ph$.
Мы хотим получить соотношение $r^2\d\ph = \const$.
Оно означает, что $\frac{d}{dt}(r^2\d\ph) = 0$.
Расписывая производную, получаем
$$\frac{d}{dt}(r^2\d\ph) = 2r\d r\d \ph+r^2\dd \ph= r\ub{(2\d r\d\ph+r\dd\ph)}_{\a_\ph}=0.$$
Значит, искомое соотношение является следствием второго закона Ньютона.
\end{proof}

\subsubsection{Качественный анализ уравнения}

Теперь выпишем компоненту уравнения второго закона Ньютона по координате $r$.
$$m(\dd r-r\d\ph^2)=-\mu\frac m{r^2}.\eqno(1)$$
C учетом соотношения $\d\ph r^2 = c$ имеем
$$\dd r-\frac{c^2}{r^3}+\frac{\mu}{r^2}=0\quad\Ra\quad
\dd r=F_c=\frac{c^2}{r^3}-\frac\mu{r^2}\quad\Ra\quad
\dd r=-\frac{dV_c}{dr},\text{ где }V_c=-\frac{\mu}{r}+\frac12\frac{c^2}{r^2}.$$
Здесь $mV_c$ так называемый приведенный (эффективный) потенциал исходной системы.
Он достигает своего минимума в точке $r_0=\frac{c^2}{\mu}$.

Фазовый портрет: \pic{?}.

Первый интеграл: $\frac12\d r^2+V_c=h$. Величина $mh$ постоянная
интеграла энергии. При $h<0$ траектории замкнутые, при $h=0$
незамкнутые ($\d r\to 0,r\to\infty$), при $h>0$ незамкнутые
($\d r\to q,r \to \infty$).

\subsubsection{Аналитическое исследование}

Преобразованиями, которые мы уже проделывали в формулах~\eqref{eqn:KeplerReverse} на с.~\pageref{page:KeplerReverse},
приведем уравнение~$(1)$ к виду
$$
-m\rho^2(\rho''+\rho)=-\frac{1}{p}m\rho^2\quad\Ra\quad
\rho''+\rho=\frac{1}{p},\eqno(2)
$$
где $\rho=\frac 1r$, $p=\frac{c^2}{\mu}$.

Решение имеет вид: $\rho=\frac1p+A\cos(\ph-\ph_\pi)$, где $A$ и
$\ph_\pi$ произвольные постоянные. Без ограничения общности
положим $A=\frac ep$, $\ph_\pi=0$. Тогда
\begin{gather*}
\rho=\frac{1+e\cos \ph}{p},\qquad r=\frac p{1+e\cos \ph}.
\end{gather*}

Это уравнение конических сечений:
\ctab{|c|c|c|}{
\hline \textbf{Эксцентриситет} & \textbf{Кривая} & \textbf{Энергия}\\
\hline $e=0$ & окружность & $h<0$ \\
\hline $0<e<1$ & эллипс & $h<0$ \\
\hline $e=1$ & парабола & $h=0$ \\
\hline $e>1$ & гипербола & $h>0$ \\
\hline}

Посчитаем $h$ при $\ph=0$: имеем
$$\d r=r'\d \ph\quad\Ra\quad\d r\evn{\ph=0}= -\frac{e\d\ph\sin\ph}{(1+e\cos\ph)^2}\evn{\ph=0} = 0,$$
$$h=V_c\evn{\ph=0}=V_c\hr{\frac p{1+e}}=
-\frac{c^2(1+e)}{p\cdot p}+\frac12\frac{c^2(1+e)^2}{p^2}=\frac{c^2(e^2-1)}{2p^2}.$$

\subsubsection{Исследование эллиптического движения}

Положим $h=-\frac{\mu}{2a}$. Выведем третий закон Кеплера:
$$T=\frac{2\pi ab}c\quad\Ra\quad
T^2=\frac{4\pi^2a^2b^2}{c^2}=\frac{4\pi^2a^4(1-e^2)}{\mu\rho}=
\frac{4\pi^2a^4(1-e^2)}{\mu a(1-e^2)}=\frac{4\pi^2a^3}{\mu}\Ra
\frac{T^2}{a^3}=\frac{4\pi^2}{\mu},$$
где $\mu$ постоянная солнечной системы.

\textbf{Определение закона движения по эллиптической траектории.} Имеем
$$r^2\d\ph=c \quad\Ra\quad  \frac{\d\ph}{(1+e\cos\ph)^2}=
\frac{c}{p^2} \quad\Ra\quad \frac{d\ph}{(1+e\cos\ph)^2}=\frac c{p^2}dt.$$
Пусть $P$ точка на эллипсе, $Q$ ее проекция на главную ось
эллипса, $S$ и $O$ соответственно фокус и центр эллипса.
Интегрировать будем с помощью подстановки:
$$\cos u=\frac{OQ}{a}=\frac{OS+SQ}{a}=\frac{ae+r\cos\ph}a=
e+\frac{1-e^2}{1+e\cos\ph} \cos \ph=\frac{e+\cos\ph}{1+e\cos\ph}.$$
Уравнение тогда приводится к виду
\begin{gather*}
(1-e\cos u)du=\frac c{p^2}(1-e^2)^{3/2}dt\quad\Ra\\
\intl{0}{u}(1-e\cos u)\,du=n\intl{t_\pi}{t}dt,\qquad \hr{n=\frac c{p^2}(1-e^2)^{3/2}},
\end{gather*}
Получаем уравнение Кеплера:
$$u-e\sin u=n(t-t_{\pi}).$$
В элементарных функциях выразить решение нельзя.

\textbf{Первая и вторая космические скорости.}
Пусть $M$ масса Земли, а $m$ масса искусственного спутника Земли.
Запишем интеграл энергии:
$$\frac12mv^2-\ga\frac{M m}{r}=h.$$

Мы знаем, что значение $h$, при котором траектория представляет собой
окружность, равно
$$h_{\text{кр}}=-\frac{\ga M m}{2R}.$$
Первая космическая скорость:
$\frac12mv_1^2=m\frac{\ga M}{2R}(-1+2)$.
Отсюда
$$v_1^2=\frac{\ga M}{R}.$$

Чтобы оторваться от Земли, нужно, чтобы траектория стала разомкнутой кривой (то есть хотя бы параболой).
Как мы знаем, $h_{\text{пар}}=0$.
Отсюда получаем вторую космическая скорость:
$$v_2^2=2\frac{\ga M}{R}=2v_1^2.$$
Все промежуточные скорости соответствуют всё более и более вытянутым эллипсам,
а большие значения гиперболам.

\subsection{Динамика материальной точки при наличии связей}

\subsubsection{Движение точки по поверхности}

Пусть траектория точки определяется не только силами, которые действуют на эту точку,
но и какими то другими соотношениями. Например, точка движется по неособой поверхности,
заданной уравнением $f(\r,t) = 0$ (неособая поверхность это поверхность, у которой ${\grad f \ne 0}$).
Тогда её траектория имеет вид
$$\Sig_t=\hc{\r(t)\in\R^3\cln f(\r,t)=0}.$$
В этом случае, вообще говоря, $m\a\ne\F$, то есть закон Ньютона не обязательно справедлив.

Это, однако, неудобно, поэтому введём аксиому освобождения от связей:
уравнение связи убирается, но добавляются силы реакции связи:
\eqn{\label{eqn:newtonReaction}
\bcase{
&m\dd\r=\F(\d\r,\r,t)+\vR(\d\r, \r, t), \\
&f(\r,t)=0.}}
Здесь $\F$ заданная сила, $\vR$ реакция связи (неизвестная).
Реакцию можно разложить на касательную и нормальную компоненты:
$$\vR=\vR_t+\vR_n,\quad \vR_n=R_n\n,\quad \n:=-\frac{\grad f}{\hm{\grad f}}.$$

Мы будем для простоты считать, что связь не зависит от времени.
Иначе говоря, это некоторая неподвижная гиперповерхность в пространстве.

\begin{stm}
Нормальная реакция $\vR_n$ однозначно определяется заданными силами и связью.
\end{stm}
\begin{proof}
Имеем $f\br{\r(t)} \equiv 0$. Продифференцируем, получим $\hr{\grad f,\d\r}\ez$, а теперь ещё раз:
$$\hr{\grad f,\dd\r}+\Br{\pf{^2f}{\r^2}\d\r, \d\r}=0.$$
Отсюда следует, что $$(\ol n,\dd \r)=\frac{1}{|\grad f|}\hr{\pf{^2f}{\r^2}\d\r,\d\r}.$$
Умножим первое уравнение системы \eqref{eqn:newtonReaction} скалярно на $\n$. Пусть $F_n=(\F,\n)$ координата
силы $F$ по нормали к поверхности. Тогда получим
$$R_n=-F_n+m(\dd\r,\n)=-F_n+\frac{m}{|\grad f|}\hr{\pf{^2f}{\r^2}\d\r,\d\r}.$$
\hfill\end{proof}

\begin{ex} Рассмотрим частные случаи для касательной составляющей реакции связи:
\begin{items}{-2}
\item $\vR_t=0$ связь \emph{идеальная}.
\item $\vR_t=-kR_n\frac{\v}{v}$ \emph{сухое трение}.
\end{items}
\end{ex}

Напомним, что через $T$ мы обозначаем кинетическую энергию точки.
\begin{theorem}\label{thm:idealLink}
 Если связь не зависит от времени и является идеальной, то $\d T=(\F,\v)$.
\end{theorem}
\begin{proof}
Умножим второе уравнение системы \eqref{eqn:newtonReaction} скалярно на $\d \r=\v$, учитывая, что $(\n, \v)=0$:
$$\frac d{dt}(mv^2)=(m\dd\r,\d\r)=(\F,\v)+(\vR,\v)=(\F,\v)+R_n(\n,\v)=(\F,\v),$$
что и требовалось доказать.
\end{proof}

\begin{imp}
Если связь не зависит от времени и является идеальной, а заданные силы потенциальны: $\F=-\grad V(\r)$
и не зависят от времени, то имеет место интеграл энергии: $T+V=\const$.
\end{imp}

\subsubsection{Движение точки по кривой}

Рассмотрим кривую $\Ga=\hc{\r\in\R^3 \vl f_1(\r)=0, f_2(\r)=0, \r=\r(s)}$, где $s$ натуральный параметр (кривая задана
как пересечение двух гиперповерхностей $f_1$ и $f_2$). Пусть радиус кривизны $\rh$ отличен от нуля.

Освободимся от связи:
\eqn{\bcase{
&m\dd\r=\F+\vR,\\
&f_1=0,\quad f_2=0,\quad \r=\r(s),\\
&\vR=R_\tau \vtau+R_\nu\vnu+R_\be\vbe.}}

Первое уравнение системы распишем в проекциях на оси $\vtau,\vnu,\vbe$:
\begin{gather*}
m\d v=F_\tau+R_\tau,\\
\frac{mv^2}{\rho}=F_\nu+R_\nu,\\
0=F_\be+R_\be.
\end{gather*}
Отсюда $$R_\be=-F_\be;\quad R_\nu=-F_\nu+\frac{mv^2}{\rho};\quad R_n=\sqrt{F_\be^2+F_\nu^2}.$$

Для $R_\tau$ возможны следующие частные случаи:

\begin{items}{-2}
\item $R_\tau=0$ (связь идеальная);
\item $R_\tau=-kR_n$ (сухое трение).
\end{items}

В случае идеальной связи $m\dd s=F_\tau(\d s, s, t)$.

\begin{note}
Теорема \ref{thm:idealLink} справедлива и для движения по кривой. Справедливо также
следствие, и интеграл энергии имеет вид: $H=\frac12mv^2+V=h$. Тогда
$R_\nu=-F_\nu+\frac{2(h-V)}\rho$ и $R_\be=-F_\be$.
\end{note}

\begin{equation*}
\begin{cases}
R_\nu=R_\nu(s,h)\\
R_\be=R_\be(s,h)
\end{cases}
\quad \Ra \quad R_n=R_n(s,h).
\end{equation*}

\subsubsection{Математический маятник}

Пусть $\ol g := g\e_z$ вектор ускорения свободного падения.

Рассмотрим груз массы $m$, подвешенный на нерастяжимом и невесомом стержне длины $l$.
Связи: $x^2+y^2=l^2$, а $z\ez$.

Предполагаем, что связь идеальная. Пусть $\ph$ угол отклонения маятника от вертикали. Тогда $v=l\d\ph$. Имеем
\eqn{\bcase{
ml\dd\ph&=-mg\sin\ph,\\
\frac{m(l\d\ph)^2}l&=-mg\cos\ph+R_\nu,\\
0&=0+R_\be.}}

Интеграл энергии: $\frac12m(l\d\ph)^2-mgl\cos\ph=h$. Выразим $\d\ph$ из этого уравнения:
$$\d\ph=\pm\sqrt{\frac{2(h+mgl\cos\ph)}{ml^2}}\quad \Ra\quad
\pm\frac{d\ph}{\sqrt{\frac{2(h+mgl\cos\ph)}{ml^2}}}=dt \quad \Ra \quad \intl\ph{\ph_0}\frac{d\ph}{\sqrt{\frac{2(h+mgl\cos\ph)}{ml^2}}}=t-t_0.$$

Нарисуем фазовый портрет. Область возможности движения:
\begin{items}{-1}
\item Если $h<-mgl$, то движение невозможно, ибо под корнем отрицательное число.
\item Если $h=-mgl$, то $\ph=\d\ph\ez$ (положение устойчивого равновесия).
\item Если $h\in(-mgl;mgl)$, то $\ph\in(-\al;\al)$, где $\al=\arccos\hr{-\frac{h}{mgl}}$ амплитуда колебаний.
\item Если $h=mgl$, то:

a) $\ph\equiv\pi, \d\ph\equiv0$ неустойчивое положение равновесия (<<конус на вершине>>).

b) $\ph=\ph(t)$, $\liml{t\ra\pm\bes}\ph(t)=\pi$ маятник стремится к верхнему положению равновесия за бесконечное
время и останавливается там.
\item Если $h>mgl$ неравномерное вращение.
\end{items}

\begin{problem}
Доказать, что при малых колебаниях ($\al \ra 0$) имеет место формула периода колебаний
$$\liml{\al\ra+0}\tau(\al)=2\pi \sqrt{\frac{l}{g}}=\tau_0,$$
где
$$\tau(\al)=2\intl{-\al}{\al}\frac{d\ph}{\sqrt{2\frac gl(\cos\ph+\cos\al)}}.$$
\end{problem}
\begin{problem}
$\tau(\pi-\ep)\sim\ln\frac1\ep$ (то~есть существует предел: $\lim\frac{\tau(\pi-\ep)}{\ln\frac1\ep}$).
\end{problem}
\begin{problem} Вывести уравнения малых колебаний:
$$\dd \ph=-\frac gl\ph, \quad \ph\in(-\al;\al), \quad 0<\al\ll 1, \quad \tau_0=2\pi\sqrt{\frac lg}.$$
\end{problem}

\subsubsection{Сферический маятник}

А теперь рассмотрим тот же груз, но разрешим ему качаться не только в плоскости, но и в пространстве.
Как обычно, $l$ длина стержня. Связь $|\r|=l$ идеальна.
Пусть $\hc{\e_\ph,\e_\ta,\e_r}$ репер сферических координат. Тогда
\eqn{\label{eqn:spherePendulum}m\dd\r=\ub{m\ol g+R\e_r}_{\F},\quad \frac12mv^2+V=h,\quad V=mgl\sin\ta.}
Напомним, что $\ta$ <<широта>>, а $\ph$ <<долгота>>.

Введём обозначение $K_z := (\K, \e_z)$.


\begin{stm}
$K_z = \const$.
\end{stm}
\begin{proof}
В самом деле, для силы реакции имеем $\MO\vR = [\r,\vR] = 0$, так как реакция коллинеарна~$\r$.
Далее, $\MO(m\ol g) = [\r,m\ol g] = mg[\r,\e_z]$. поэтому $\br{\MO(m\ol g),\e_z} = mg([\r,\e_z],\e_z) = 0$.
Итак, мы показали, что $(\MO(\F),\e_z) = 0$, а потому $K_z=\const$.
\end{proof}

Скорость точки  будет равна $\v= l(\d\ta \e_\ta + \d\ph \cos\ta \e_\ph)$. Поэтому закон сохранения энергии будет таким:
$$\frac12ml^2(\d\ta^2+\d\ph^2\cos^2\ta)+mgl\sin\ta=h.$$
Кроме того,
$$\K=[\r,m\v] = ml^2[\e_r,\d\ta\e_\ta+\d\ph\cos\ta\e_\ph]=ml^2(-\d\ta\e_\ph+\d\ph\cos\ta\e_\ta),$$
поэтому, переходя к проекции, получаем (с учётом того, что $\e_\ph \perp \e_z$)
$$K_z=(\K,\e_z)=ml^2\d\ph\cos\ta(\e_\ta,\e_z)=ml^2\d\ph\cos^2\ta.$$
Следовательно, $ml^2\d\ph\cos^2\ta=k=:ml^2c$, где $c = \const$, откуда при $\ta \ne \pm \frac\pi2$
получаем $\d\ph=\frac{c}{\cos^2\ta}$.
Теперь подставим полученное значение для $\d\ph$ в закон сохранения энергии:
$$
\frac12ml^2\d\ta^2+V_c=h, \quad
V_c:=mgl\sin\ta+\frac12\frac{ml^2c^2}{\cos^2\ta}.
$$
Здесь $V_c$ приведённый потенциал.

Проанализируем решение. Если $c=0$, то $\d\ph=0\Ra\ph=\const$, то есть движение плоское.
Теперь посмотрим на график приведённого потенциала. Эта функция имеет полюса в точках $\pm \frac\pi2$,
а между ними имеется (единственный) минимум, обозначим его $\ta_c$.

Тут имеется картинка \pic{?} и сфера с вычерченной на ней синусоидальной кривой.

Пусть $h = V_c(\ta_c)$. Тогда маятник заметает окружность, высеченную на сфере некоторой
горизонтальной плоскостью. Если же $h > V_c(\ta_c)$, то траектория на сфере будет
иметь синусоидальную форму.



\textbf{Малые колебания сферического маятника в окрестности
нижнего положения равновесия.} Пусть наш маятник колеблется вблизи южного полюса сферы,
$(x,y,z)$ его координаты. Имеем
$$z=-\sqrt{l^2-x^2-y^2},$$
и при $0<|x|+|y| \ll 1$ имеем
$z=-l+O(x^2+y^2)$.
Тогда получаем уравнения
$$m\dd\r=m\ol g+R\e_z
$$
Поскольку $f=x^2+y^2+z^2-l^2=0$,
получаем $\e_r=\frac{\grad f}{\hm{\grad f}}$, имеем
$$\begin{aligned}
m\dd x&=R\frac xl+o\br{|x|+|y|},\\
m\dd y&=R\frac yl+o\br{|x|+|y|},\\
0&=-mg-R,
\end{aligned}$$
таким образом
$$\case{
\dd x=-\frac glx,\\
\dd y= -\frac gly.}
$$
Итак, в окрестности полюса колебания сферического маятника в первом приближении
совпадают по каждой из осей с обычными гармоническими колебаниями.

\subsection{Относительное движение точки}

\subsubsection{Относительное движение материальной точки}

Пусть $O\e_x\e_y\e_z$ неподвижный репер, $S\e_1\e_2\e_3$ подвижный репер.
Траектория начала отсчёта подвижного репера это
$\r_S=\r_S(t)$, а $\Ga=\Ga(t)$ матрица, задающая ориентацию
$\e_1\e_2\e_3$ в $O\e_x\e_y\e_z$.
Напомним, что
$$
\v_S=\frac {d_a\r_S}{dt},\quad \a_S=\frac{d^2_a\r_S}{dt^2},\quad
\ol\ep= \frac{d_a\w}{dt}=\frac{d_r\w}{dt}=\d\w.
$$
Здесь точка означает производную по времени в относительных координатах.

$$m\a_a=\F, \eqno(1)$$
$$m\a_r=\F-m\a_e-m\a_c,\eqno(2)$$
и, соответственно, $\F_e:=-m\a_e$, а $\F_c:=-ma_c$.

Через $\p$ обозначим радиус вектор точки относительно
подвижной системы координат. Тогда~$(2)$ записывается в виде
$$m\dd\p=\F(t,\rho,\d\rho)+\F_e+\F_c,\eqno(3)$$
где
$$\F_e=-m\br{\a_S+[\d\w,\p]+[\w,[\w,\p]]}\md$$
переносная сила, а
$$\F_c=-2m[\w,\d\rho]\md$$
кориолисова сила.

\begin{theorem}
$\d T=(\F+\F_e,\d\p)$, где $T=\frac12 m\d{\p}^2$.
\end{theorem}

\begin{proof}
Заметим, что $(\F_c,\d\p)\ez$. Умножая $(3)$ скалярно на $\d\p$, получаем
требуемое.
\end{proof}

\begin{lemma}
Если $\d\a_S=0$ и $\d\w=0$, то $\F_e=-\grad V_e(\p)$, где
$$V_e=m\Br{(\a_S,\p)+\frac 12 (\w,\p)^2-\frac 12 \om^2\rho^2}.$$
\end{lemma}

\begin{proof}
Посчитаем ускорение, на массу потом домножим:
\begin{multline*}
-(\a_e,d_r\p)=(\a_S,d_r\p)+([\w,[\w,\p]],d_r\p)=
d_r(\a_S,\p)+(\w,\p)(\w,d_r\p)-\om^2(\p,d_r\p)=\\=
d_r(\a_S,\p)+\frac12d_r(\w,\p)^2-\frac12\om^2d_r(\rho^2)=
d_r\BS{(\a_S,\p)+\frac12(\w,\p)^2-\frac12\om^2\rho^2}.
\end{multline*}
Лемма доказана.
\end{proof}

\begin{theorem}
Если связи, наложенные на относительное движение
точки, идеальны и не зависят от времени, заданные силы потенциальны
и не зависят от времени, то есть $\F=-\grad V(\p)$, и выполнены
условия леммы, то уравнения (1) допускают обобщенный интеграл энергии:
$$T+V+V_e=h=\const.$$
\end{theorem}

\begin{proof}
Следует из предыдущей теоремы и леммы.
\end{proof}

\begin{imp}
Если $S = O$, $\e_\ze=\e_z$ и
$\w=\om\e_z =\const$, то
$$V_e=-\frac12m(\rho_1^2+\rho_2^2)\om^2,$$
где $\p = (\rho_1,\rho_2,\rho_3)$.
\end{imp}

\begin{ex} \textbf{Математический маятник во вращающейся системе отсчета.}
Применим полученное только что следствие. В нашем случае $\p = (\rho_1,\rho_2,0)$,
причём $\rho_1^2 + \rho_2^2 = (r\sin\ph)^2$.
Имеем
$$
\begin{aligned}
\w&=\om\e_z=\om\e_\ze=\const,\\
V&=-mgr\cos\ph,\\
T&=\frac12mr^2\d\ph^2,\\
V_e&=-\frac12m(r\sin\ph)^2\om^2,\\
T+V+V_e&=:T+V_\om=h,
\end{aligned}
$$
где
$$V_\om=-mgr\cos\ph-\frac12\om^2mr^2\sin^2\ph\md$$
измененный потенциал.

Вычислим момент относительного равновесия (найдём критическую точку изменённого потенциала):
$$0=V_\om'(\ph) =\frac{dV_\om}{d\ph}=mgr\sin\ph\hs{1-\frac{r\om^2}{g}\cos\ph},$$
Таким образом, потенциал имеет до четырёх  критических точек, в зависимости от параметра~$\om$.
Точки $\ph_1=0$ и $\ph_2 = \pi$ являются критическими при всех значениях~$\om$,
а точки
$$\ph_{3,4}=\pm\arccos\frac{g}{r\om^2}$$
будут критическими при $\om^2>\frac gr$.

Но нам и этого мало. Теперь будем исследовать критические точки на устойчивость.
Для краткости обозначим $u := \frac{r\om^2}{g}$.
Считаем вторую производную:
$$V_\om''=mgr\hs{\cos\ph\hr{1-u\cos\ph}+ru\sin^2\ph},$$
Отсюда
$$
\begin{aligned}
\frac1{m g r}V_\om''(\ph_1)&=1-u \gtrless 0 \text{ при }u \lessgtr 1;\\
\frac1{m g r}V_\om''(\ph_2)&=-(1+u)<0;\\
\frac1{m g r}V_\om''(\ph_{3,4})&=u\sin^2\ph_{3,4}>0,
\end{aligned}
$$
потому что критические точки $\ph_{3,4}$ существуют только при $u > 1$.

Бифуркационная диаграмма Пуанкаре: \pic{?}
картинка в осях $(u,\ph)$: при каждом значении параметра $u$ рисуем множество критических точек,
получается такая <<многозначная функция>> (при некоторых значениях прообраз вообще может быть пуст,
но в нашем случае он всегда состоит хотя бы из двух точек).


Фазовый портрет: (это \pic{?}).

При $u = 1$ можно проверить, что $V_\om''(0)=V_\om^{(3)}(0) = 0$,
но $V_\om^{(4)}(0)>0$.
При $u = 1$ происходит перестройка фазового портрета.

Уровни энергии: $h=h_i(\om^2)=V_\om(\ph_i)$, тогда
$$h_1=-mgr,\quad h_2=mgr,\quad h_{3,4} = -\frac{mgr}2\hr{u + \frac{1}{u}}.$$

Бифуркационная диаграмма Смейла: \pic{?}.
\end{ex}

\subsection{Движение точки в поле тяготения Земли}

\subsubsection{Движение точки с учетом вращения Земли}

Модель Земли однородный шар массы $M$ и радиуса~$R$.
Мы будем изучать, как влияет вращение Земли вокруг своей оси
на движение точки.

\begin{lemma}
Гравитационный потенциал однородного шара совпадает с
гравитационным потенциалом центра шара, в котором сосредоточена вся
его масса.
\end{lemma}
\begin{proof}
Во первых, можно считать только по сечению, потому что вся картинка является осе симметричной.
Далее,
$$V=-\ga\frac{Mm}{\frac 43\pi R^3}\cdot\intl{-R}R
\intl0{\sqrt{R^2-x^2}} \frac{(2\pi y)\,dy\,dx}{\sqrt{(r-x)^2+y^2}}\stackrel{!}{=}-\ga\frac{Mm}r,$$
при этом проверка последнего равенства предоставляется читателю.
\end{proof}

Обозначим $\e := \e_z = \e_\ze$. Угловая скорость вращения Земли:
$$\w=\Om\e,\quad \Om=\frac{2\pi}{24\cdot60\cdot60}\text{ сек}^{-1}.$$
Пусть $\M$ наша точка, а $\M_0$ центральная проекция точки на поверхность Земли.
Обозначим через $M$ массу Земли,
$$\p = \ol{OM}, \quad \vR = \ol{OM}_0,\quad R = \hm{\vR} \sim6.4\cdot10^6\text{ м},\quad \mu=\ga M.$$
Напишем основное уравнение:
$$m\dd\p=-\mu\frac{m}{\rho^3}\p-m[\w,[\w,\p]]-2m[\w,\d\p].\eqno(1)$$
Если $\M$ принадлежит поверхности Земли, то $|\p|=R$, и тогда
$$m\dd\p=-\mu\frac{m}{\rho^3}\p-m[\w,[\w,\p]]-2m[\w,\d\p]-\P,$$\
где $(-P)$ реакция. $\P$ сила, которая действует со
стороны точки на Землю <<давление>> точки на Землю.

\begin{df}
\emph{Весом} точки на Земле называется сила, которая
действует со стороны покоящейся (относительно Земли) точки на Землю.
\end{df}
Имеем
$$\P_0=\P\evn{|\p|=R,\d\p=0,\dd\p=0}, \quad \P_0=-\mu\frac m{R^3}\vR+m\Om^2[[\e,\vR],\e].$$
Вычислим $P_0$: имеем
\begin{gather*}
\vR=R\cos\ta\e_\eta+R\sin\ta\e_\ze.\\
[\e,[\e,\vR]]=\e(\e,\vR)-\vR=R\sin\ta\e-R\sin\ta\e-R\cos\ta\e_\eta=-R\cos\ta\e_\eta,\\
\end{gather*}
отсюда находим
$$\P_0=-\mu\frac{m}{R^2}(\cos\ta\e_\eta+\sin\ta\e_\ze)+m\Om^2R\cos\ta\e_\eta.$$

Пусть $\ol g_P=-\mu\frac{\vR}{R^3}$ гравитационное ускорение,
$g_P=|\ol g_P|=\frac{\mu}{R^2}$. Тогда
$$\P_0=-mg_P\hs{\sin\ta\e_\ze+\cos\ta\hr{1-\frac{\Om^2R}{g_P}}\e_\eta},$$
$$\ol g=-g_P\hs{\sin\ta\e_\ze+\cos\ta\hr{1-\frac{\Om^2R}{g_P}}\e_\eta}\md$$
ускорение силы тяжести на Земле.
Значит,
$$
g=|\ol g|=g_P\sqrt{1-2\cos^2\ta\frac{\Om^2R}{g_P}+\cos^2\ta\hr{\frac{\Om^2R}{g_P}}^2}\approx
\hr{1-\cos^2\ta\frac{\Om^2R}{g_P}}g_P.$$
\begin{equation*}
\begin{cases}
g_p\sin\ta=g\sin\ph,\\
g_p\cos\ta=g\cos\ph+\Om^2R\cos\ta
\end{cases}
\Ra
\end{equation*}
$$\tg\ph=\tg\ta\frac1{1-\frac{\Om^2R}{g_p}}\approx\frac{288}{289}\tg\ta,$$
где $\ph$ угол местной вертикали \pic{?}.

\subsubsection{Падение точки на Землю с учетом вращения Земли}

Пусть $Sxyz$ система координат, <<вмороженная в Землю>>. Координата $Sx$ на
запад, $Sy$ на юг, $Sz$ местная вертикаль.

$$m\dd\r=m\ol g(\r)-2m[\w,\d\r]\eqno(1)$$
($()^{\cdot}$ производная в системе отсчета, связанной с Землей,
т.е. локальная). Переносная сила инерции учтена уже в $m\ol g$.

\begin{gather*}
\ol g(\r)\approxeq\ol g(\ol0)=-g\e_z\quad \ol g(M)\approxeq\ol g(S)=-g\e_z,\quad g=\const.\\
\w=\Om\e, \quad
e=-\cos\ph\e_y+\sin\ph\e_z;\quad\Om=\frac{2\pi}{24\cdot60\cdot60}\text{сек}^{-1}.
\end{gather*}

Начальные условия: $\r(0)=z_0\e_z;\d\r(0)=\ol0.$ $\d\r=\v$ относительная скорость.
$$\d\v=\ol g-2\Om[\e,\v]\eqno (2)$$

Это система линейных дифференциальных уравнений с постоянными
коэффициентами. Будем искать решение уравнения (2) в виде
$$\v=\v_0+\Om\v_1+\Om^2\v_2+…\eqno(3)$$
Тогда (2) примет вид
$$\d\v_0+\Om\d\v_1+\Om^2\d\v_2+…=\ol g-2\Om[\e,(\v_0+\Om\v_1+…)].$$

Получаем систему
$$
\begin{cases}
\d\v_0=\ol g,\\
\d\v_1=-2[\e, \v_0],\\
\d\v_2=-2[\e,\v_1] ...
\end{cases}\eqno(4)
$$

Решая систему (4) при начальных условиях $\v_i(0)=\ol0, i=0,1,2…$,
получаем
$$
\begin{cases}
\v_0=\ol gt,\\
\v_1=-[\e,\ol g]t^2,\\
\v_2=\frac23[\e,[\e,\ol g]]t^3,...
\end{cases}\eqno(5)
$$

Значит, $$\v=\ol gt-\Om[\e,\ol
g]t^2+\frac{2\Om^2}3[\e,[\e,\ol g]]t^3, …$$
$$\r=\r(0)+\frac{\ol gt^2}2-\frac{\Om}3[\e,\ol g]t^3+\frac{\Om^2}6[\e,[\e,\ol g]]t^4, …$$

Это закон движения точки, падающей на Землю. В проекциях на оси
$x,y,z$ он записывается следующим образом:

\begin{gather*}
[\e,\ol g]=-g[\e,\e_z]=g\cos\ph\e_x,\\
[\e,[\e,\ol g]]=[-\cos\ph\e_y+\sin\ph\e_z,g\cos\ph\e_x]=g\cos^2\ph\e_z+g\sin\ph\cos\ph\e_y;\Ra\\
\r=-\frac{\Om t}3\cos\ph\cdot gt^2\e_x+\frac{(\Om t)^2}6\sin\ph\cos\ph\cdot
gt^2\e_y+\hs{z_0-\frac{gt^2}2\hr{1-\frac{(\Om t)^2}3\cos^2\ph}}\e_z+o((\Om t)^2).
\end{gather*}

\begin{note}
Вывод несколько некорректен, так как нельзя
раскладывать по величинам, имеющим размерность. (Мы раскладывали по
$\Om$, а надо раскладывать по степеням безразмерной величины
$\Om t$. Но ответ будет таким же).
\end{note}

В первом приближении точка отклоняется на Восток; во
втором приближении на юг; точка упадет чуть позже, чем если бы
Земля не вращалась (тоже во втором приближении).

\begin{ex}
Рассмотрим падение тела с Останкинской телебашни.
$$z_0=500{\text{ м}}, g=10\frac{\text{м}}{\text{с}^2};\ph=60^0;  500-\frac{10t_{\text{пад}}^2}2=0\Ra t_{\text{пад}}=10{\text{ сек}}$$
$$\Delta x=\frac{2\pi\cdot10 }{24\cdot60\cdot60}\cdot\frac{1}{3\cdot2}\approx10^{-1}{\text{м}}=10{\text{ см}}$$
$$\Delta y=\frac{(2\pi)^2\cdot10^2 }{24^2\cdot60^2\cdot60^2}\cdot\frac{1.73}{6\cdot4}\approx10^{-3}{\text{м}}=1{\text{ мм}}.$$
\end{ex}

\begin{note} По местной вертикали точка падает на полюсе, не
отклоняется на Юг на экваторе.
\end{note}

\subsubsection{Маятник Фуко}

\begin{df}
\emph{Маятник Фуко} это сферический маятник, подвешенный на Земле,
с учетом ее вращения.
\end{df}

Введем систему координат $Sxyz$: $S$ центр сферы, $Sx$ на
запад, $Sy$ на юг, $Sz$ местная вертикаль.

$$ m\dd\r=m\ol g-2m[\w,\d\r]+N\frac{\r}{l};\eqno(1)
$$
$|\r|=l$ идеальная связь, $N$ нормальная реакция. Частное
решение уравнения (1) имеет вид
\begin{gather*}
\r=-l\e_z, \qquad N=-m g\\
x=y=0, \qquad z=-l, \qquad N=-m g.
\end{gather*}
Выпишем линеаризованное уравнение вблизи нижнего положения
равновесия ($x$ и $y$ малые, отбрасываем члены порядка
$x^2+y^2$):
\begin{gather*}
z=-l\sqrt{1-\frac{x^2+y^2}{l^2}}=-l+o(|x|+|y|), \qquad N=-m g+\nu
\text{( $\nu$
-- малая величина)}\\
\r=x\e_x+y\e_y-l\e_z, \qquad \d\r=\d x\e_x+\d y\e_y;\\
[\w,\d\r]=\Om[(-\cos\ph\e_y+\sin\ph\e_z),(\d x\e_x+\d
y\e_y)]=-\Om\d y\sin\ph\e_x+\Om\d x\sin\ph\e_y+\Om\d
x\cos\ph\e_z.
\end{gather*}
Уравнение (1) в проекциях на оси $x,y,z$:
$$
\begin{cases}
\dd x=2\Om\d y\sin\ph-\frac{g}{l}x,\\
\dd y=-2\Om\d x\sin\ph-\frac{g}{l}y, \\
0=\dd z=2\Om\d x\cos\ph-\nu;
\end{cases}\eqno(2)
$$
Откуда получаем,  что в первом приближении $N=-mg+\Om\d
x\cos\ph$. Положим $w=x+iy$. Интегрирование системы (2) дает
$$
\dd w+2\Om\sin\ph i\la+\frac{g}{l}w=0.\eqno(3)
$$
Характеристическое уравнение:
\begin{gather*}
\la^2+2\Om\sin\ph i\la+\frac{g}{l}=0,\\
\la_{1,2}=\hr{-\Om\sin\ph\pm\sqrt{\Om^2\sin^2\ph+\frac{g}{l}}}i=(-\om_1\pm\om_2)i\\
\om_1=\Om\sin\ph, \qquad
\om_2=\sqrt{\Om^2\sin^2\ph+\frac{g}{l}}; \quad
\om_1\ll\om_2
\end{gather*}
Тогда
$$
w_1=\cos(\om_2-\om_1)t+i\sin(\om_2-\om_1)t, \qquad
w_2=\cos(\om_2+\om_1)t-i\sin(\om_2+\om_1)t.
$$
Пусть
\begin{gather*}
\tilde{w_1}=\frac{w_1+w_2}2=\cos\om_2t(\cos\om_1t-i\sin\om_1t), \\
\tilde{w_2}=\frac{w_1-w_2}2=\sin\om_2t(\cos\om_1t-i\sin\om_1t)
\end{gather*}
Общее решение уравнения (3):
$$
w=c_1\tilde w_1+c_2\tilde
w_2=(\cos\om_1t-i\sin\om_1t)(c_1\cos\om_2t+c_2\sin\om_2t).
$$
{\bf Пример:} начальные условия:
\begin{gather*}
x(0)=x_0,y(0)=0,\d x(0)=\d y(0)=0;\\
w(0)=x_0,\d w(0)=0,c_1=x_0,c_2=i\frac{\om_1}{\om_2}x_0;\\
x=x_0(\cos\om_1t\cos\om_2t+\frac{\om_1}{\om_2}\sin\om_1t\sin\om_2t)=\mathrm{Re}w\\
y=x_0(\frac{\om_1}{\om_2}\cos\om_1t\sin\om_2t-\sin\om_1t\cos\om_2t)=\mathrm{Im}w
\end{gather*}
Колебания маятника Фуко почти плоские,  но не плоские.

Если $l=100\text{м}$, $g=10\frac{\text{м}}{\text{с}^2}$, то
$\om_2^2\sim\frac1{10}$, $\om_2\sim\frac1{3}$, $\ph=60^\circ$,
$\om_1\sim\frac1610^{-4}$, $x_0=10{\text{ м}}$, $t=300\text{
с}\Ra\Delta y\sim5\text{ см}$.

{\bf Замечание.} Рассмотрим плоский линейный осциллятор $\ol
F=-k\r$.

Если $k=m\om_2^2$, а плоскость $Oxy$ вращается с угловой
скоростью $\w=\om_1\e_z$, то уравнения движения точки $m$
совпадают с уравнениями (2):
$$
m\dd\r=m\om_1^2\r-2m\om_1[\e_z,\d\r]-m\om_2^2\r\equiv-2m\om_1[\e_z,\d\r]-m(\om_2^2-\om_1^2)\r.
$$

%---------------------------------------------------------------------
\section{Динамика системы точек и твёрдого тела}
%---------------------------------------------------------------------

\subsection{Динамика системы точек}

\subsubsection{Основные понятия}

Мы будем рассматривать системы из $n$ точек, поэтому индексы, их нумерующие,
а равно как и суммы по ним, всегда будут от $1$ до $n$.

\begin{df}
\emph{Материальной системой} называется система точек $M_1\sco M_n$ с массами
$m_1,\dots,m_n$.
\end{df}

Введем обозначения: $\r_{ij}=\r_i-\r_j$,
$r_{ij}=|\ol{r_{ij}}|$.

Согласно принципу детерминированности,  в
абсолютном пространстве
$$m_i\dd\r_i=\F_i(\r_1\sco \r_n,\d\r_1\sco \d\r_n,t).$$
Это система дифференциальных уравнений порядка $6n$.

Разложим силы на внешние и внутренние:
$\F_i=\F_i^{(e)}+\F_i^{(i)}$, здесь $\F_i^{(e)}$ внешние силы, $\F_i^{(i)}$ внутренние.
При этом $\F_i^{(i)}=\sums{j\ne i}\F_{ij}$, где
$\F_{ij}$ сила, действующая со стороны $j$-й точки на $i$-ю.
$$
\F_{ij}=F_{ij}\frac{\r_{ij}}{r_{ij}}=-\F_{ji}=
-F_{ji}\frac{\r_{ji}}{r_{ji}}=F_{ji}\frac{\r_{ij}}{r_{ij}}\quad\Ra\quad
F_{ij}=F_{ji}.
$$
Отсюда, в частности, следует,  что $\sum\F_i^{(i)}=0$.

\begin{df}
Система отсчета называется инерциальной (ИСО),
если она движется равномерно и прямолинейно относительно абсолютной
системы отсчета (АСО).
\end{df}

\begin{df}
Материальная система называется замкнутой,  если
она не взаимодействует с внешними материальными системами,
то есть все $\F_i^{(e)}=0$.
\end{df}

\textbf{Принцип относительности Галилея:} уравнения движения замкнутой
системы одинаковы во всех ИСО.

Зафиксируем какую нибудь систему отсчёта, начало координат обозначим через $O$.

\begin{df}
Масса материальной системы: $m:=\sum m_i$.
\end{df}

\begin{df}
Точка $S$ называется \emph{центром масс} системы, если $\r_S=\frac1m \sum m_i\r_i$.
\end{df}

\begin{stm}[Корректность определения]
Центр масс не зависит от выбора точки $O$ (начала отсчёта).
\end{stm}
\begin{proof}
Пусть $O'\ne O$. Обозначим
$\r'_i=\ol{O'M_i}$.
Имеем
$$\r'_{S'}=\frac1m \sum m_i\r'_i=\frac1m\sum m_i(\ol{OO'}+\r_i)=\ol{OO'}+\frac1m\sum\r_i = \ol{OO'} + \r_S=\r'_S,$$
поэтому $S'=S$.
\end{proof}

\begin{df}
Импульс материальной системы (то~есть суммарное количество движения): $\P:=\sum\P_i=\sum m_i\d\r_i$.
\end{df}

Здесь и далее $S$ центр масс системы.

\begin{stm}
Очевидно, что $\P=m\d\r_S=m\v_S$.
\end{stm}

\begin{imp}
$\P$ не зависит от выбора точки~$O$.
\end{imp}

\begin{df}
Кинетический момент системы (то есть момент количеств движений) относительно точки $O$:
$$\K=\sum\ol K_{Oi}=\sum[\r_i,m_i\d\r_i].$$
\end{df}

\begin{stm}
$\ol K_{O'}=\ol K_O+[OO',m\d\r_S]$.
\end{stm}

\begin{proof}
Имеем
$$\ol K_{O'}=\sum[\r_i',m_i\d\r_i']=
\sum[\ol{O'O}+\r_i,\; m_i(\ub{\d{\ol{O'O}}}_0+\d\r_i)]=[\ol{O'O},m\d\r_S]+\K,$$
так как точки $O$ и $O'$ неподвижные.
\end{proof}

\begin{df}
Кинетическая энергия системы:
$$T:=\sum T_i=\sum\frac12m_iv_i^2=\sum m_i\d\r_i^2.$$
\end{df}

\subsubsection{Общие теоремы динамики}

Пусть, как и в предыдущем разделе, точка $S$ обозначает центр масс.

\begin{theorem}
$\d\P=\sum\F_i^{(e)}=:\F^{(e)}$ главный вектор сил.
\end{theorem}

\begin{proof}
$\d\P=\sum m_i\dd\r_i=\sum \br{\F_i^{(e)}+\F_i^{(i)}}=\sum \F_i^{(e)}$.
\end{proof}

\begin{imp}
$m\d\v_S=\F^{(e)}$.
В частности, если система замкнута, то центр масс движется равномерно прямолинейно.
\end{imp}

Определим моменты внешних сил: $\M_{Oi}^{(e)}:=[\r_i,\F_i^{(e)}]$.

\begin{theorem}\label{thm:KinMomDiff}
Имеем $\d{\ol K}_O=\sum\M_{Oi}^{(e)}=:\MO^{(e)}$ главный момент.
\end{theorem}
\begin{proof}
В самом деле,
\begin{multline*}
\d\K=\frac{d}{dt}\sum[\r_i,m_i\d\r_i]=
\sum\ub{[\d\r_i, m\d\r_i]}_0 + \sum[\r_i, m\dd\r_i]=
\sum[\r_i, \F_i^{(e)}+\F_i^{(i)}]=\\=
\sum[\r_i, \F_i^{(e)}]+\sum \BS{r_i,\sum_{j\ne i}\F_{ij}}=
\sum \ol M_{Oi}^{(e)}+\frac12\sums{i,\, j\ne i}\hr{[\r_i, \F_{ij}]+ [\r_j, \F_{ji}]}=\\=
M_O^{(e)}+\frac12\sums{i,\, j\ne i}\hs{\r_i-\r_j,\F_{ij}}\stackrel{!}{=}\ol M_O^{(e)},
\end{multline*}
так как сила $F_{ij}$ действует вдоль прямой, соединяющей
соответствующие точки, и все слагаемые равны нулю.
\end{proof}

\begin{note}
Вообще говоря, $\MO^{(e)}\ne \M_O(\F^{(e)})$,
то есть главный момент не имеет ничего общего с моментом
главного вектора сил относительно точки $O$.
\end{note}

\begin{imp}
Для замкнутой системы
$$\P=m\v_S=\const, \quad \K=\const.$$
\end{imp}
\begin{proof}
Достаточно применить две предыдущих теоремы.
\end{proof}

\begin{theorem}
$$\d T=\sum_i\hr{(\F_i^{(e)},\d\r_i)+(\F_i^{(i)},\d\r_i)},$$
или, что то же самое,
$$dT=\sum\hr{(\F_i^{(e)},d\r_i)+(\F_i^{(i)},d\r_i)}.$$
\end{theorem}
\begin{proof}
Следует из аналогичной теоремы для одной точки.
\end{proof}

\begin{df}
Будем говорить, что силы $\F_i$ потенциальны (и не зависят от времени),
если существует функция $V(\r_1\sco\r_n)$ (не зависящая от
времени) такая, что
$$\F_i=-\grad_{\r_i}V=-\frac{\pd V}{\pd\r_i}\quad\Lra\quad
\sum(\F_i,d\r_i)=-dV=-\sum\Br{\frac{\pd V}{\pd\r_i}, d\r_i}.$$
\end{df}

\begin{imp}
Если все связи, наложенные на систему, идеальны и не
зависят от времени, а силы потенциальны, то существует интеграл
энергии: $H=T+V=\const$.
\end{imp}

\begin{theorem}
Если внутренние силы между любыми двумя точками зависят
только от расстояний между этими точками, входящими в материальную
систему, то они потенциальны, то есть, если $F_{ij}=F_{ij}(r_{ij})$, то
$$V=\sums{i<j}\int F_{ij}\,dr_{ij}.$$
\end{theorem}
\begin{proof}
В самом деле,
\begin{multline*}
\sum(\F_i^{(i)},d\r_i)=
\sum_i\sum_{j\ne i}\hr{F_{ij}\frac{\r_{ij}}{r_{ij}}, d\r_i}=
\frac12\sum_i\sum_{j\ne i}\hs{\hr{F_{ij}\frac{\r_{ij}}{r_{ij}},d\r_i}+\hr{F_{ij}\frac{\r_{ji}}{r_{ji}}, d\r_j}}=\\=
\sum_i\sum_{j<i}\hr{F_{ij}\frac{\r_{ij}}{r_{ij}}, d\r_{ij}}=
\sum_i\sum_{j<i}F_{ij}\frac{r_{ij}dr_{ij}}{r_{ij}}=
-d\bbr{-\sum_i\sum_{j<i}\int F_{ij}dr_{ij}}=
-dV,
\end{multline*}
что и требовалось доказать.
\end{proof}

\begin{imp}
Для абсолютно твердого тела потенциальная энергия
внутренних сил есть константа.
\end{imp}

\subsubsection{Понятие о задаче $n$ тел.\\
Задача двух тел и её сведение к задаче Кеплера}

Имеется $n$ тел, взаимно притягивающихся по закону Всемирного тяготения.
Потенциал системы имеет вид
$$V=-\frac12\sums{i\ne j}\ga\frac{m_im_j}{r_{ij}}=-\ga\sums{i<j}\frac{m_im_j}{r_{ij}}.$$
Внешних сил нет. Уравнения движения
\eqn{\label{eqn:nBodiesProblem}
m_i\dd\r_i=-\frac{\pd V}{\pd \r_i}.}
образуют систему порядка $6n$. Интегралы энергии, импульса,
кинетического момента имеют вид:
\eqn{\label{eqn:nBodiesIntegrals}
H=T+V=h, \quad \P=\const, \quad \K= \const,}
(в скалярном виде $7$~интегралов).

\begin{theorem}[Брунс]
При $n\ge 3$ система~\eqref{eqn:nBodiesProblem} не имеет
алгебраических интегралов, отличных от~\eqref{eqn:nBodiesIntegrals}.
При $n\ge 3$ проинтегрировать в квадратурах систему~\eqref{eqn:nBodiesProblem} нельзя.
\end{theorem}

\textbf{Задача двух тел.}
$$
\bcase{
m_1\dd\r_1&=-\ga\frac{m_1m_2}{r_{12}^3}\r_{12},\\
m_2\dd\r_2&=\ga\frac{m_1m_2}{r_{12}^3}\r_{12},\\
\r_{12}&=\r_1-\r_2.}
$$

Пусть $S$ центр масс, тогда $\r=\ol{SM_1}=\frac{m_2}m\r_{12}$, где $m=m_1+m_2$.

Решим систему: из первого уравнения вычитаем второе:
$$\dd\r_{12}=-\ga\frac m{r_{12}^3}\r_{12}\quad \Ra \quad
\dd \r=-\ga \frac{m_2}{\frac{m^3}{m_2^3}r^3}\frac{m}{m_2}\r=
-\ga\frac{m_2^3}{m^2r^3}\r\quad\Ra\quad
m_1\dd\r= \mu\frac{m_1}{r^3}\r\text{ где }\mu=\ga\frac{m_2^3}{m^2}.$$
Это уравнение Кеплера, только вместо $\ga$ теперь $\mu$. (Отсюда уточнение законов Кеплера:
$\frac{\tau^2}{a^3}=f(\mu)$, здесь $\mu$ зависит от $m_1$ (массы
планеты), но $\frac{m_1}{m_2}\ll 1$; фокус эллипса, по которому
движется планета, не в геометрическом центре Солнца, а в центре масс
системы <<планета Солнце>>).

\subsubsection{Оси, формулы и теоремы Кёнига}

\emph{Оси Кёнига} имеют начало в точке $S$ центре масс тела и
параллельны абсолютным осям. Радиус вектор в осях Кёнига:
$\r_i'=\r_i-\r_S$.
Имеем $\w_{\text{Кёнига}}\equiv 0$, поэтому
$$\frac{d_a\ol b(t)}{dt}= \frac{d_r\ol b(t)}{dt}.$$
Штрихованные векторы это векторы в осях Кёнига.
Стало быть,
$$\sum m_i\r'_i\equiv0, \quad \sum m_i\v'_i=0.$$

\begin{theorem}[Формулы Кёнига]
$$\K=[\r_S,m\v_S]+\ol K'_S, \quad \text{где }\ol K'_S=\sum m_i[\r'_i,\d\r'_i].$$
$$T=\frac12mv_S^2+T', \quad\text{где }T'=\frac12\sum m_iv_i'^2.$$
\end{theorem}
\begin{proof}
В самом деле,
$$\K=\sum m_i[\r_i,\d\r_i]=\sum m_i[\r_S+\r'_i,\d\r_S+\d\r'_i]=m[\r_S,\v_S]+\ol K'_S+0+0.$$
$$T=\frac12\sum m_iv_i^2= \frac12\sum m_i(\d\r_S+\d\r'_i,\;\d\r_S+\d\r'_i)=
\frac12mr_s^2+\frac12\sum m_i {\d r}'^2_i+0+0,$$
что и требовалось.
\end{proof}

\begin{theorem}[Первая теорема Кёнига]
$\d\vK'_S=\ol M_S^{(e)}$.
\end{theorem}
\begin{proof}
Вспоминая теорему~\ref{thm:KinMomDiff} и первую формулу Кёнига, пишем:
$$\d\vK'_S=\d\K-m[\r_S,\d\v_S]=\MO^{(e)}-[\r_S,\F^{(e)}]=
\sum [\r_S+\r'_i,\F_i^{(e)}]-[\r_S,\sum \F_i^{(e)}]=\ol M_S^{(e)},$$
что и требовалось.
\end{proof}

\begin{theorem}[Вторая теорема Кёнига]
$$\d T'=\sum (\F_i,\d\r_i').$$
\end{theorem}
\begin{proof}
Применяя вторую формулу Кёнига, пишем:
$$\d T'=\d T-m(\v_S,\d\v_S)=\sum(\F_i,\d\r_S+\d\r'_i)-(\v_S,\F)=\sum (\F_i,\d\r'_i),$$
и всё получилось.
\end{proof}

Если система замкнута, то оси Кенига инерциальная система отсчета.

\subsection{Динамика твердого тела. Геометрия масс ТТ}

\subsubsection{Динамика твёрдого тела с неподвижной точкой}

Рассмотрим ТТ с неподвижной точкой~$O$. Как мы уже знаем, его движение это
мгновенное вращение вокруг мгновенной оси вращения $\ell$ с направляющим вектором $\e$,
причём $\w=\om\e$.

По формуле Эйлера, $\v_i=\v_O+[\w,\ol{OM_i}] = [\w,\ol{OM_i}] = [\w,\r_i]$, так как $\v_O = 0$.
Отсюда, домножая  кинетическую энергию на $2$, чтобы не тащить коэффициент, получаем
$$
2T=\sum m_iv_i^2=
\sum m_i \bm{[\w,\r_i]}^2=
\sum m_i \br{\om^2r_i^2-(\w,\r_i)^2}=
\om^2\sum m_i\br{r_i^2-(\e,\r_i)^2}=:
\om^2J_\ell,
$$
где $J_\ell:=\sum m_i\br{r_i^2-(\e,\r_i)^2}=\sum m_i\rho_i^2$ момент
инерции ТТ относительно оси $\ell$, где $\rho_i=\dist (M_i,\ell)$.
Если тело сплошное, то $J_\ell=\ints{TT}\rho^2\,dm$.

\begin{theorem}[Гюйгенс Штейнер]
$J_\ell=J_s+m\rho^2$, где $s$ прямая с направляющим вектором $\e$,
проходящая через центр масс~$S$, а $\rho=\dist(s,\ell)$.
\end{theorem}
\begin{proof}
Имеем
\begin{multline*}
J_\ell=
\sum m_i(r_i^2-(\e,\r_i)^2)=
\sum m_i\Br{(\r_S+\r'_i,\r_S+\r'_i)-(\e,\r_S+\r'_i)^2}=\\=
\sum m_i(r_S^2-(\e,\r_S)^2)+\sum m_i({r'_i}^2-(\e,\r'_i)^2)=m\rho^2+J_s,
\end{multline*}
что и требуется.
\end{proof}

\subsubsection{Оператор инерции и эллипсоид инерции}
Имеем
$$T=\frac12\sum m_i(\om^2r_i^2-(\w,\r_i)^2).$$
Легко видеть, что выражение справа это некоторая квадратичная форма
для вектора $\w$. Иначе говоря, можно записать
$$T=\frac12(\JO\w;\w),$$
где $\JO$ \emph{оператор инерции}, или \emph{тензор инерции}.
Выберем ортонормированный базис, и в нём матрицу $\JO$ можно записать так:
$$
\JO=
\rbmat{\phm J_{11}&-J_{12}&-J_{13}\\
-J_{12}& \phm J_{22}&-J_{23}\\
-J_{13}&-J_{23}&\phm J_{33}}.
$$
Здесь $J_{11}=J_1=\sum m_i(y_i^2+z_i^2)$ и так далее моменты относительно
соответствующих осей,
$J_{12}=\sum m_ix_iy_i$ и так далее центробежные моменты инерции.

Если $\w=\om \e$, то
$(\JO \e, \e)=J_\ell$. Ну а поскольку $\JO$ оператор, то  равенство
$\om^2J_\ell=\frac12(\JO\w,\w)$ справедливо в любой системе координат.
Отсюда $T=\frac12(\JO\w,\w).$

\begin{note} По хорошему, надо ввести определение
моментов относительно осей и центробежных моментов, потом показать,
что для прямой, проходящей через начало координат, с направляющими
косинусами $(\al,\be,\ga)$ момент инерции относительно нее равен
$J_x\al^2+J_y\be^2\bw+J_z\ga^2-2J_{xy}\al\be-2J_{xz}\al\ga-2J_{yz}\be\ga$).
\end{note}

 Для $J_1, J_2, J_3$ справедливо
неравенство треугольника: $J_1+J_2\ge J_3$ (аналогично для всех перестановок индексов).
Ясно, что равенство достигается тогда и только тогда, когда твердое тело <<блин>>:
$J_1+J_2=J_3\Lra z_j=0.$

\begin{df}
\emph{Эллипсоид инерции} это множество $\Sig=\hc{\r\in\R^3\cln (\JO\r,\r)=1}$.
\end{df}

Здесь мы рассматриваем невырожденные твердые тела, то~есть тела, у которых
существуют 3~точки, не лежащие на одной прямой).
Мы знаем, что всякую квадратичную форму можно привести к каноническому виду.
Именно, существуют оси $x,y,z$, такие что
$$\Sig=\hc{x,y,z\cln Ax^2+By^2+Cz^2=1}.$$
Эти оси называются главными осями инерции для точки $O$, а  $A,B,C$ главными
моментами инерции для точки $O$. В главных осях $\JO=\diag(A,B,C)$.

\subsubsection{Основные динамические характеристики ТТ}

Пусть тело имеет неподвижную точку $O$. Тогда
\begin{gather*}
\P=m\v_S; \quad T=\frac12(\JO\w,\w);\\
\K=\sum m_i[\r_i,\v_i]=\sum m_i[\r_i,[\w,\r_i]]=
\sum m_i(\w r_i^2-\r_i(\w,\r_i))=\frac{\pd T}{\pd \w}=\JO\w.
\end{gather*}

В общем случае, если $S$ центр масс, то
\begin{gather*}
\P=m\v_S; \quad T=\frac12mv_S^2+\frac12(\JS\w,\w)\text{ (по теореме Кёнига);}\\
\vK_S=\JS\w.
\end{gather*}
Всегда верно следующее: $\P=\frac{\pd T}{\pd \v}$, $\ol K=\frac{\pd T}{\pd \w}$.
Если $P$ произвольная точка ТТ, то
\begin{gather*}
\P=m(\v_P+[\w,\ol{PS}]); \quad T=\frac 12mv_P^2+m(\v_P,[\w,\ol{PS}])+\frac12(\J_P\w,\w);\\
\ol K_P=\J_P\w+m[\ol {PS},\v_P].
\end{gather*}

\subsubsection{Основные уравнения движения ТТ}

\pt1. Случай ТТ с неподвижной точкой: Имеем
$\d {\K}\bw=\MO^{(e)}$ динамическое уравнение,
$\w\bw=\w(\Ga,\d\Ga)$ кинематическое уравнение (система ОДУ 6-го порядка).

\pt2. Общий случай: ТТ свободное или при наличии связей.
В этом случае имеется система динамических и кинематических уравнений (система ОДУ 12-го порядка):
$$
\case{
m\d\v_S=\F^{(e)},\\
\d{\K}=\MO^{(e)};}\qquad
\case{\v_s=\d\r_s,\\
\w=\w(\Ga,\d\Ga).}
$$

\subsubsection{Эквивалентные системы сил, действующих на ТТ}

Пусть силы $\F_i$ приложены в точках $M_i$ ($i=1\sco n$), а силы $\F_j'$ приложены в точках
$M_j'$ ($j=1\sco m$).

\begin{df}
Система сил $\hc{F_i,M_i}$ называется эквивалентной системе сил $\hc{F_j',M_j'}$, если
$$
\sum\F_i=\sum\F_j'\quad\text{и}\quad
\sum[\r_i,\F_i]=\sum[\r_j',\F_j']\quad
\text{где } \r_i=\ol{SM}_i,\;\r_j'=\ol{SM}_j'.
$$
\end{df}

\begin{note}
Внутренние силы, действующие на ТТ, эквивалентны нулевой системе сил.
\end{note}

\begin{lemma}
Если ТТ находится в однородном поле тяжести, то силы тяжести,
действующие на тело, эквивалентны одной силе $m\ol g$, приложенной в центре
масс $S$ тела.
\end{lemma}
\begin{proof}
В самом деле, пусть $\r_i=\ol{SM}_i$. Тогда
$$\sum m_i\ol g=m\ol g, \quad \sum[\r_i,m_i\ol g]= \BS{\sum m_i\r_i,\ol g} = 0,$$
что и требуется.
\end{proof}

\begin{note}
Существуют системы сил, не приводимые только к равнодействующей.
Пример: \emph{пара сил} (это название такой системы сил)
$\F_1+\F_2=\ol0$, $\ol M=[\r_1,\F_1]+[\r_2,\F_2]\ne0$.
Линии действия двух сил параллельны, силы равны по величине и
противоположны по направлению, приложены к разным точкам: \pic{?}.
\end{note}

\begin{note}
Любая система сил, действующих на ТТ, может быть приведена к
любой точке $P$ тела в виде $\F$ и $\ol M_P$ (то есть к равнодействующей и паре).
\end{note}

\begin{note}
В центральном поле, вообще говоря, приведение системы сил к
центру масс таково: равнодействующая $\F$ и пара $\ol M_S\ne0$.

\begin{ex} (поле Ньютона): Пусть $A$ и $B$ две точки,
принадлежащие прямой $l$ с массами $m_1=1$ $m_2=2$ соответственно. А
силы $\F_A$ и $\F_B$, действующие на эти точки направлены вдоль
прямых $AO$ и $BO$ соответственно к точке $O$ и являются
центральными. Тогда равнодействующая не проходит через центр масс.
(Это простое упражнение по геометрии и оставляется на
самостоятельное решение).
\end{ex}
\end{note}

\begin{theorem}
$\d T=(\F^{(e)}, \v_S)+(\ol M_S^{(e)},\w)$.
\end{theorem}

\begin{proof}
$$T =\frac12mv^2_S+\frac12(\JS\w,\w)\Ra\d T=
(m\d\v_S,\v_S)+(\JS\d\w,\w)=(\F^{(e)}, \v_S)+(\ol M_S^{(e)},\w).$$
\end{proof}

\subsubsection{Динамика твердого тела}

I. Вращение ТТ вокруг неподвижной  оси. Пусть $l$ ось вращения,
$O\xi\eta\ze$ и $Oxyz$ неподвижная и <<вмороженная>>  в ТТ
системы координат соответственно. $O\ze=Oz$. В точках $O$ и $P$ шарниры.
Больше сил никаких нет. Пусть точка означает
дифференцирование по $t$ в подвижных осях. Имеем
$$
\v_o=0,\quad\v_p=0,\quad\w=\d\ph\e_z\equiv\d\ph\e_\ze
$$
$$
\begin{cases}
m\d\v_s+[\w,m\v_s]=\F+\ol R_o+\ol R_p,\\
\JO\d\w+[\w,\JO\w]=\ol M_o+[\ol{OP},\ol R_p],\\
\v_s=[\w,\ol{OS}],\qquad\w=\d\ph\e_z;
\end{cases}
$$

Эта система уравнений движения ТТ.
\begin{gather*}
\v_s=\d\ph[\e_z,(a\e_x+b\e_y+c\e_z)]=\d\ph(a\e_y-b\e_x),\\
\d\v_s=\dd\ph(a\e_y-b\e_x),\qquad[\w,\v_s]=\d\ph^2[\e_z,a\e_y-b\e_x]=-\d\ph^2(a\e_x+b\e_y).
\end{gather*}

Тензор инерции:
$$\JO=
\rbmat{
\phm J_x&-J_{xy}&-J_{xz}\\
-J_{xy}&\phm J_y&-J_{yz}\\
-J_{xz}&-J_{yz}&\phm J_z}
$$
\begin{gather*}
\JO\w=-\d\ph(-J_{xz}\e_x-J_{yz}\e_y+J_z\e_z),\\
[\w,\JO\w]=\d\ph^2[\e_z,-J_{xz}\e_x-J_{yz}\e_y+J_z\e_z]=\d\ph^2(-J_{xz}\e_y+J_{yz}\e_x);\\
\ol{OP}=h\e_z,\qquad[\ol{OP},\ol R_p]=h[\e_z,R_{px}\e_x+R_{py}\e_y+R_{pz}\e_z]=
h(R_{px}\e_y-R_{py}\e_x).
\end{gather*}

С учетом полученного система перепишется в виде
$$
\case{
m\d\ph(a\e_y-b\e_x)-m\d\ph^2(a\e_x+b\e_y)=\F+\ol R_o+\ol R_p,\\
\JO\dd\ph\e_z+\d\ph^2(-J_{xz}\e_y+J_{yz}\e_x)=\ol M_o+h(R_{px}\e_y-R_{py}\e_x)\\
\v_s=[\w,\ol{OS}],\qquad\w=\d\ph\e_z;}
$$

Первые два уравнения в проекциях на ось $Oz$ дают
$$0=F_z+R_{oz}+R_{pz},\qquad J_z\dd\ph=M_{oz},\quad M_{oz}(\ph,\d\ph,t)$$

Из последнего уравнения находится закон движения ТТ. Остальные 5~уравнений
служат для нахождения реакций.
$$
R_{px},\quad R_{py},\quad R_{ox},\quad R_{oy}
$$

Если в точке~$O$ находится сферический шарнир,в точке $P$ цилиндрический шарнир, то
$$
R_{ox},\quad R_{oy},\quad R_{oz},\quad R_{px},\quad R_{py},\quad R_{pz}=0.
$$

Без указания конкретной реализации закрепления точек $O,P$ реакцию определить нельзя.

\subsubsection{Физический маятник: ТТ в однородном поле тяжести}

Пусть ось $Oz$ направлена горизонтально, $\ph$ угол между
вертикалью направленной вниз (вектором $\ol g$) и $\ol{OS}$,
$a=OS$.
$$
J_{oz}\dd\ph=-mga\sin\ph,\quad\dd\ph=-\frac{mga}{J_{oz}}\sin\ph.\eqno(1,1')
$$

Это уравнение совпадает с точностью до обозначений с уравнением
движения математического маятника:

$$
\dd\ph=-\frac{g}{l}\sin\ph.\eqno(2)
$$

(1')$\Lra$(2) при $l=\frac{J_{oz}}{ma}$ приведенная
длина физического маятника.

\begin{gather*}
J_{oz}=J_{sz}+ma^2=m(a^2+\rho_s^2),\qquad J_{sz}=m\rho_s^2,\\
l=a+\frac{\rho_s^2}{a}>a \text{ при } \rho\ne0.\\
SO'=b=\frac{\rho_s^2}{a},\qquad OO'=l.
\end{gather*}
$\rho_s$ называется радиусом инерции. Период малых колебаний:
$\tau_o=\frac{2\pi}{\sqrt{l/g}}$.

\begin{theorem}[Гюйгенс]
Период малых колебаний физического
маятника не изменится, если его подвесить на оси $O'z$.
(Точки $O$ и $O'$ взаимны).
\end{theorem}

\begin{proof}
$l'$ приведенная длина физического
маятника, подвешенного на оси $O'z$. Тогда
$$l'=b+\frac{\rho_s^2}{b}=\frac{\rho_s^2}{a}+\frac{\rho_s^2}{\rho_s^2/a}=l\Ra\tau_{o'}=\tau_o.$$
\end{proof}

\subsubsection{Плоско-параллельное движение ТТ}

$Oxy$ неподвижная плоскость, $S\xi\eta$ жестко связана с
плоской фигурой (телом). $S$ центр масс.
$$
\begin{cases}
m\dd\r_s=\F,\\
J_{sz}\dd\ph=M_{sz}; \end{cases}
$$

В проекциях на оси $x,y$ и естественные оси $\tau,\nu$ для центра масс $S$ имеем
$$\leqno(1')
\begin{cases}
m\dd x=F_x,\\
m\dd y=F_y;
\end{cases}
\qquad\begin{cases} m\d v=F_\tau,\\
m\frac{v^2}{\rho}=F_\nu;
\end{cases}\eqno(1'')
$$

\subsubsection{Задача: Диск на наклонной прямой}

Пусть однородный диск массы $m$ и радиуса $r$ скатывается (или
соскальзывает) под действием силы тяжести $mg$ по наклонной прямой с
углом наклона $\al$. Направим ось $x$ вдоль прямой, а $y$ по
нормали. $y_s=r\Ra \vR=\ol N+\ol T=N\e_y-T\e_x$
$$
\begin{cases}
m\dd x_S=mg\sin\al-T,\quad(1)\\
m\dd y_S=-mg\cos\al+N,\quad(2)\\
m\frac{r^2}{2}\dd\ph=M_s=Tr,\\
y_S=r\\
?
\end{cases}
$$

Необходимо еще одно уравнение, чтобы система была разрешима, так как
неизвестных 5 штук: $x,y,T,N,\ph$. Его можно получить в одном из
следующих частных случаев:

a) $T=0$ (диск свободно скользит без трения).
$$\dd x_s=g\sin\al,\quad\dd\ph=0,\quad y_S=r,\quad N=mg\cos\al.$$

b) $\d x_S=r\d\ph$ (скольжения нет).
$$mr\dd\ph=mg\sin\al-T,\quad  mr\dd\ph=2T\quad \Ra\quad T=\frac13mg\sin\al.$$
$$\dd x_S=\frac23g\sin\al,\quad \dd\ph=\frac23\frac gr\sin\al,\quad y_S\equiv r,\quad
N=mg\cos\al,\quad T=\frac13mg\sin\al.$$

c) $T=fN$ (сухое трение).
$$m\dd x_S=mg\sin\al-fmg\cos\al(N=mg\cos\al)$$

$c_1$) $fmg\cos\al\ge\frac13mg\sin\al$ Тогда скольжения
нет и этот случай эквивалентен случаю b)

$c_2$) $f<\frac13\tg\al$ Тогда
$$\dd x_S=g\sin\al-gf\cos\al,\dd\ph=2f\frac gr\cos\al,y_S=r;$$
$$N=mg\cos\al;T=fmg\cos\al.$$

\subsubsection{Динамика ТТТ с неподвижной точкой}

\begin{gather*}
\frac{d\K}{dt}={\ol M_o}\text{  динамическое уравнение.}\\
\w=\w(\Ga,\d\Ga)\text{ кинематическое уравнение.}
\end{gather*}
$$\K=\JO\w.$$

$O\e_x\e_y\e_z$ репер, жестко связанный с телом. $Oxyz$ главные оси
инерции тела для точки $O$, $A,B,C$ главные моменты инерции для точки $O$:

$$\JO_{xyz}=
\rbmat{
A&0&0\\
0&B&0\\
0&0&C}
$$

$$\JO\d\w+[\w,\JO\w]=\ol M_O\quad (()^{\cdot}=\frac{d_r}{dt}).$$

\begin{df}
\emph{Тяжёлым твёрдым телом} (ТТТ) называется ТТ в однородном поле тяжести.
\end{df}

$$\JO\d\w+[\w,\JO\w]=[\g,\frac{\pd
V}{\pd\g}]\equiv mg[\g,\ol s]$$
$$V=mg(\ol{OS},\g)\equiv mg(\ol s,\g)$$
$\g$ --орт восходящей вертикали,
$\ol{OS}=(a\e_x+b\e_y+c\e_z)=\ol s$.
$$
\begin{cases}
\JO\d\w+[\w,\JO\w]=[\g,\frac{\pd
V}{\pd\g}]\equiv mg[\g,\ol s]\\
\d{\g}+[\w,\g]=\ol0.
\end{cases}
$$
Это уравнения Эйлера Пуассона, система ОДУ 6-го порядка. Ее первые
интегралы: $H(\w,\g)=\frac12[\JO\w,\w]+ mg(\g,\ol
s)=h$ интеграл энергии, $K_{o\ze}=K=(\K
,\g)=(\JO\w,\g)=k$ интеграл площадей,
$\Ga=(\g,\g)=1$ геометрический или тривиальный
интеграл.

Система автономна (не зависит от $t$) и сохраняет объемы поэтому для
интегрирования в квадратурах нужно 4 первых интеграла. 3 уже имеем.

Случаи интегрируемости (наличия дополнительного интеграла):

\begin{enumerate}
\item Случай Эйлера: $a=b=c=0$ ($S\ez$).
\item Случай Лагранжа: $A=B,a=b=0$ ($S\in Oz$).
\item Случай Ковалевской: $A=B=2C,c=0$ ($\Lra b=c=0$, то есть $S\in Ox$).
\end{enumerate}
В общем случае Пуанкаре доказал, что других алгебраических
интегралов нет, а В.\,В.\,Козлов что других аналитических интегралов нет.

Пусть $S$ центр масс, $xyz$ главные оси инерции.
$$
\case{
\JO\d\w+[\w,\JO\w]=mg[\g,\ol s],\\
\d\g+[\w,\g]=0;}
\eqno(1)
$$

$\JO=\diag(A,B,C)$. Если нашли решение~$(1)$:
$\w=\w(t),\g=\g(t)$, то

$$\g_1=\sin\ta\sin\ph,\g_2=\sin\ta\cos\ph,\g_3=\cos\ta,$$
откуда находим $\ta$ и $\ph$. $\w_3=\d\ph+\d\psi\cos\ta(t)$,
откуда $\d\psi=\frac{\w_3(t)-\d\ph(t)}{\cos\ta(t)}$,
$\psi=\int\d\psi\,dt$.

\subsection{Волчки и всё о них}

\subsubsection{Перманентные вращения. Вращение с
постоянной угловой скоростью вокруг постоянной в теле оси}

Это движения такого типа: $\g=\g_0;\w=\om_0\g_0$.

Подставим в~(1), получим $\om_0^2[\g_0,\JO\g_0]=-mg[\ol s,\g_0]$.
Умножим обе части скалярно на $\ol s$:

\begin{gather*}
\om_0([\g_0,\JO\g_0],\ol s)=0,\quad(\om_0\ne 0)\\
\g_0:([\g_0,\JO\g_0],\ol s)=0.
\end{gather*}

Эта поверхность конус Штауде. Пересечение конуса Штауде со сферой
Пуассона: две окружности.

Динамически допустимые $\g_0$ (достаточное условие):
$([\g_0,\JO\g_0],[\g_0,\ol s])\ge 0$.

\subsubsection{Волчок Эйлера}

Волчок Эйлера это ТТ с неподвижной точкой при отсутствии сил, или
что то же самое тяжелое ТТ с закрепленным центром масс.

$$
\frac{d\K}{dt}=\ol 0\quad\Ra\quad\K=\ol k=\const\text{(в
абсолютном пространстве)}
$$

Динамические уравнения Эйлера
$$
\JO\d\w+[\w,\JO\w]=\ol 0\qquad(\K=\JO\w)
$$
(отделяются от уравнений Пуассона; автономная система ОДУ 3-го
порядка).

Первые интегралы:
$$
2H=2T=(\JO\w,\w)=2h,\qquad K^2=(\JO\w,\JO\w)=k^2.
$$

Перепишем систему в виде
$$
\case{
A\d\om_1+(C-B)\om_2\om_3=0,\\
B\d\om_2+(A-C)\om_3\om_1=0,\\
C\d\om_3+(B-A)\om_1\om_2=0,\\
2H=A\om_1^2+B\om_2^2+C\om_3^2=2h,\\
k^2=A^2\om_1^2+B^2\om_2^2+C^2\om_3^2.}
\eqno(1^{**})
$$

Пусть $A\ne B\ne C\ne A$, тогда можно считать, что $A<B<C$. Из двух
последних уравнений системы имеем

\begin{gather*}
k^2-2AH=C(C-A)\om_3^2+B(B-A)\om_2^2,\\
2hC-k^2=B(C-B)\om_2^2+A(C-A)\om_1^2,\text{(отсюда $k^2\in[2Ah,2Ch]$.)}\\
\om_3^2=\frac{-B(B-A)\om_2^2+(k^2-2Ah)}{C(C-A)}\\
\om_1^2=\frac{(2hC-k^2)-B(C-B)\om_2^2}{A(C-A)}.
\end{gather*}

Подставим эти выражения для $\om_3^2$ и $\om_2^2$ в
($1^{**}$), получим дифференциальное уравнение на $\om_2$:
$$\d\om_2=\pm\frac{C-A}{B}\sqrt{\om_1^2\om_3^2}=\pm\sqrt{P_4(\om_2)}\Ra
t-t_0=\pm\int_{\om_{0}}^\om\frac{d\om_2}{\sqrt{P_4(\om_2)}}.$$
Это эллиптический интеграл, он обычно не берется в элементарных
функциях).

\begin{center}
{\bf Фазовый портрет}
\end{center}

Пространство 3-х мерное. Рассмотрим 2 интеграла из системы ($1^{**}$).
В осях $\om_1\om_2\om_3$ они задают 2
эллипсоида. Пусть $h$ фиксировано, $k$ меняется.
\begin{enumerate}
\item $k^2<2Ah$. Тогда первый эллипсоид лежит внутри второго и они
не пересекаются. Движение невозможно.
\item $k^2=2Ah$. Тогда $\om_1=\pm\frac kA, \om_2=\om_3=0$.
\item $k^2\in (2Ah,2Bh)$. Область возможности движения две
окружности.
\item $k^2=2Bh$. Тогда $\om_2=\pm\frac kB, \om_1=\om_3=0$.
\item $k^2\in(2Bh, 2Ch)$. Область возможности движения две
окружности.
\item $k^2=2Ch$. Тогда $\om_3=\pm\frac kC, \om_1=\om_2=0$.
\item $k^2>2Ch$. Тогда первый эллипсоид лежит вне второго и они
не пересекаются. Движение невозможно.
\end{enumerate}

\subsubsection{Перманентные вращения волчка Эйлера}

$$\om_1=\om=\const, \om_2=\om_3=0$$
$$\om_2=\om=\const, \om_1=\om_3=0$$
$$\om_3=\om=\const, \om_1=\om_2=0$$

Перманентные вращения волчка Эйлера вокруг наибольшей и наименьшей
полуосей эллипсоида инерции устойчивы, а вокруг средней оси неустойчивы.

\subsubsection{Геометрическая интерпретация Пуансо}

\begin{theorem}
Эллипсоид инерции волчка Эйлера катится без скольжения по неподвижной плоскости,
ортогональной вектору кинетического момента волчка.
\end{theorem}

\begin{proof}
Рассмотрим эллипсоид инерции.
$\Sig=\{\r\in\R^3\cln (\JO\r,\r)\}$. Рассмотрим точку $P$
пересечения мгновенной оси вращения $O\w$ с $\Sig$. Проведем через
точку $P$ плоскость $\pi$, касательную к $\Sig$. Тогда
$\v_P=\ol0$, так как $P$ принадлежит мгновенной оси вращения.
$$
\ol{OP}=\r_P=\la \w \Ra 1=(\JO\r,\r)=\la^2(\JO\w,\w)=\la^2\cdot 2h\Ra\la=\frac1{\sqrt{2h}}.
$$
$$
\pi\perp\n_p=\frac{(\grad\Sig)_P}{|(\grad\Sig)_P|}=\frac{\JO\r_P}{|\JO\r_P|}=
\frac{\JO\w}{|\JO\w|}=\frac{\K}{k}\Ra\pi\perp\K.
$$

$$\dist(O,\pi)=(\r_P,\n_P)=\la\hr{\w,\frac{\JO\w}{k}}=\frac{\la\cdot2h}{k}=\frac{\sqrt{2h}}{k}=\const$$
Поэтому $\pi$ неподвижна.
\end{proof}

\subsubsection{Регулярные прецессии динамически симметричного волчка Эйлера ($A=B\ne C$)}

Регулярная прецессия динамически симметричного ТТ это такое
движение ТТ, при котором оно вращается с постоянной угловой
скоростью вокруг оси симметрии, которая вращается с постоянной
угловой скоростью вокруг неподвижной в абсолютном пространстве оси,
причем угол между этой осью и осью симметрии постоянен.

\begin{theorem}
Общее движение динамически симметричного волчка
Эйлера представляет собой регулярную прецессию (причем $Oz=\ol K_o$).
\end{theorem}
\begin{proof}
Пусть ось $Oz$ неподвижной системы координат совпадает с $\K$. Из
уравнений Эйлера следует, что $\om_3=\om_{30}=\const$. Значит,
$$K_z=C\om_3=C\om_{30}=k\cos\ta\Ra\cos\ta=\hr{\frac{k}{C\om_{30}}}^{-1}=\const.$$
Воспользуемся кинематическими формулами Эйлера:

$$
\case{
\om_1=\d\ta\cos\ph+\d\psi\sin\ta\sin\ph,\\
\om_2=-\d\ta\sin\ph+\d\psi\sin\ta\cos\ph,\\
\om_3=\d\ph+\d\psi\cos\ta.}
\Ra
\case{
\om_1=\d\psi\ga_1,\\
\om_2=\d\psi\ga_2,\\
\om_3=\d\ph+\d\psi\ga_3.}
$$

\begin{gather*}
K_x=A\om_1=A\d\psi\sin\ta_0\sin\ph=k\ga_1=k\sin\ta_0\sin\ph\\
K_y=A\om_2=A\d\psi\sin\ta_0\cos\ph=k\ga_2=k\sin\ta_0\cos\ph\\
\d\psi=\frac
kA=\Om_2=\const(\sin\ta_0\ne0); \om_3=\om_{30}=\d\ph+\frac
kA\frac{C\om_{30}}k\Ra\d\ph=\om_{30}\hr{1-\frac CA}=\Om_1=\const.
\end{gather*}
\end{proof}

\subsubsection{Волчок Лагранжа}

Тяжелое, динамически симметричное ТТ с неподвижной точкой. В осях,
связанных с телом $\ol s=(0,0,c)$.
$$
\case{
\JO\d\w+[\w,\JO\w]=[\g,\frac{\pd V}{\pd\g}],\\
\d{\g}+[\w,\g]=0}
$$
Проекция уравнения Эйлера на $e_3$ дает $C\d\om_3=0$.
$\JO=\mathrm{diag}(A,A,C)$, $C\ne A$. $V=mg(\ol
s,\g)=mgc\ga_3$. Имеют место интегралы:
\begin{gather*}
H=\frac12(\JO\w,\w)+mgc\ga_3=h,\qquad K=(\JO\w,\g)=k,\\
\Om=\om_3=\om,\qquad \Ga=\g^2=1.
\end{gather*}

Перепишем их в углах Эйлера ($\ta,\psi,\ph$):
\begin{gather*}
H=\frac12A(\d\ta^2+\d\psi\sin^2\ta)+\frac12C(\d\ph+\d\psi\cos\ta)^2+mgc\cos\ta=h,\\
K=A\d\psi\sin^2\ta+C(\d\ph+\d\psi\cos\ta)\cos\ta=k,\qquad
\Om=\d\ph+\d\psi\cos\ta=\om.
\end{gather*}

\begin{note}
Если существует $t_0$ ($t_\pi$), такое что $\ta(t_0)=0$ ($\ta(t_\pi)=\pi$), то $k=C\om$
($k=-C\om$). Тогда если $k\ne \pm C\om$, то $\sin\ta\ne0,\fa t$.
\end{note}

1. Пусть $k\ne\pm C\om$, тогда
$\d\ph+\d\psi\cos\ta=\om$,
$A\d\psi\sin^2\ta=k-C\om\cos\ta$. Получаем
$\d\psi=\frac{k-C\om\cos\ta}{A\sin^2\ta}$. Положим
$$
H_{k,\Om}=\frac12A\d\ta^2+V_{k,\om}(\ta)=h-\frac12C\om^2,\qquad
V_{k,\om}=mgc\cos\ta+\frac12\frac{(k-c\om\cos\ta)^2}{A\sin^2\ta}.
$$

\subsubsection{Динамика твердого тела на горизонтальной плоскости}

Пусть дано выпуклое тело. $S,x_1,x_2,x_3$ центр масс и главные
оси инерции тела (подвижная система координат имеет центр в центре
масс тела). Оператор инерции $\JO$ и масса $m$ заданы. Поверхность тела задается уравнением
$\Sig\cln f(\ol x)=0$. А само тело $\Int\Sig\cln f(\ol x)\le 0$. Пусть $\r_S=\ol {OS}$, $\rho=\ol {SK}$ радиус
вектор точки касания тела с плоскостью, $\ol u=\v_K=\v+[\w,\p]$.
$\ga$ орт восходящей вертикали. Запишем уравнения:
$$
\begin{cases}
\frac d{dt}(m\v)=m\ol g+\vR;\\
\frac d{dt}(\JS\w)=[\p,\vR];\\
(\ol u,\ga)=?
\end{cases}
$$

Рассмотрим следующие модели:
\begin{enumerate}
\item $\ol R=\ol N\ga$ абсолютно гладкая плоскость
\item $\ol u=\ol0$ чистое качение (абсолютно шероховатая поверхность)
\item $\ol R= N\g-f N\frac{\ol u}{u}$ сухое трение.
\end{enumerate}

Разберем их на примере шара радиуса $a$. В этом случае
$\p=a\g\Ra(\v,\g)=0$.
$\JS=\frac25ma^2\mathbb{E}$.

\begin{enumerate}
\item $\vR=N\ga$. Тогда
$$
\frac d{dt}{m\v}=(N-mg)\g,\eqno(1)
$$
$$
\frac25ma^2\frac{d\w}{dt}=\ol0.\eqno(2)
$$
$$\frac d{dt}(m\v,\g)=m\frac
d{dt}(\v,\g)=0\Ra ((1),\ga)=0\Ra N=mg.$$
Здесь мы пользовались тем, что $\frac{d\ga}{dt}=0$. Таким
образом,
$$
\frac{d\v}{dt}=0\Ra \v=\v_0; \quad\frac{d\w}{dt}=\ol0\Ra\w=\w_0
$$
Причем начальная скорость удовлетворяет условию $(\v_0,\g)=0$.
Итак, траекторией движения центра является прямая
$\r=\r_0+\v_0t$.
\item $\ol u=0$. Тогда
$$\frac d{dt}{m\v}=-mg\g+\vR,\eqno(1)$$
$$\frac25ma^2\frac{d\w}{dt}=-a[\g,\vR].\eqno(2)$$
$u=0\Ra v=[\p;\w]=a[\w,\g]$. Поэтому (1) и (2) переписываются в виде
$$ma\hs{\frac {d\w}{dt},\ga}=\vR-mg\g,\eqno(1')$$

$$ma\frac{d\w}{dt}=-\frac52[\g,\vR].\eqno(2')$$
Подставляем $(1')$ в $(2')$ имеем
$$\frac52[[\vR,\g]\g]=\vR-mg\g\quad\Ra\quad\frac52\g(\vR,\g)-\frac52\vR=\vR-mg\g$$
Поскольку из последнего равенства следует параллельность векторов
$\vR$ и $\g$, имеем $\vR=N\g\Ra N=mg$
$$\frac{d\v}{dt}=0;\frac{d\w}{dt}=\ol0  \Ra\w=\w_0,\v=a[\w_0,\g]$$
Таким образом траектория, описываемая центром шара имеет уравнение
$\r=\r_0+a[\w_0,\g]t$.
\item $\ol R=N\g-fN\frac{\ol u}{|u|}$. Тогда

$$\frac d{dt}{m\v}=(N-mg)\g-fN\frac{\ol u}u,\eqno(1)$$
$$\frac25ma^2\frac{d\w}{dt}=-afN[\g,\frac{\ol u}u].\eqno(2)$$

Пусть $\e=\frac{\ol u}u$. Тогда из  $((1), \g)$ имеем $N=mg$ и $(\ol u, \g)=0$.

$$\frac {d\v}{dt}=-fg\e,\eqno(1')$$
$$\frac{d\w}{dt}=\frac52\frac{fg}a[\g,\e].\eqno(2')$$
$$\frac{d\ol u}{dt}=-fg\e-\frac 52fg[[\g,\e],\g]=-fg\e+\frac 52fg\g(\g,\e)-\frac 52fg\e(\g,\g)=-\frac72fg\e\Ra$$
$$\frac{du}{dt}\e+u\frac{d\e}{dt}=-\frac72fg\e \eqno(3)$$

Из того, что $(\frac{d\e}{dt},\e)=0$, имеем, c учетом   (3),
$\frac{d\e}{dt}=0, \frac{du}{dt}=-\frac72fg$ (Обе части последнего
равенства умножить скалярно на $\frac{d\e}{dt}$). Поэтому имеем

$$\e=\e_0\quad\v=\v_0-\e_0fgt;\Ra\r=\r_0+\v_0t-\e_0\frac{fgt^2}2$$
Отсюда, если $ \v_0\not\|\r_0$, траектория парабола, если
$\v_0\|\r_0$, траектория прямая.

$$\w=\w_0+\frac52\frac{fg}a[\g,\e_0]t$$
$$\v_0\|\e_0\Lra[\v_0\ol u_0]=\ol0\Lra(\v_0,\w_0)=0.$$
\end{enumerate}

\end{document}
