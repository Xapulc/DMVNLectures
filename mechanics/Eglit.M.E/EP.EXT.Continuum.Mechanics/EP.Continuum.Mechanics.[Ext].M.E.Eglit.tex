\documentclass[a4paper,11pt]{article}
\usepackage[simple,utf]{dmvn}

\title{Программа внешнего (дополнительного) экзамена\\по механике сплошных сред}
\author{Составитель~--- Маргарита Эрнестовна Эглит}
\date{Для аспирантов-математиков, 2003 г.}

\def\tm#1{т.\,#1}
\def\g#1{гл.\,#1}
\def\gp#1#2{гл.\,#1\;\textsection#2}

\begin{document}
\maketitle

\subsection*{Историческая справка}

Данный документ был найден в~бумажном виде в~библиотеке МехМата
на~14~этаже среди кипы других программ. Как гласила надпись в~углу карандашом,
программа составлена 28 марта 2003 года. Заботливо отсканировано
\texttt{dipterix@gmail.com} и~по\TeX ано DMVN на~благо всем разумным существам.
\medskip\dmvntrail

\section*{Программа экзамена}

\subsection*{Геометрические и кинематические понятия}

\begin{nums}{-2}

\item Понятие сплошной среды. Пространственные и~материальные координаты.
Эйлерово и~лагранжево описание движения. Индивидуальная производная по~времени.
Связь полей перемещений, скоростей и~ускорений при~лагранжевом и~эйлеровом описании.
(\cite{1} \tm{1} \g{I}; \gp{II}{1-3}).

\item Тензоры конечных и~малых деформаций. Геометрический смысл компонент, их выражение
через производные компонент вектора перемещения. Уравнения совместности
для компонент тензоров деформаций. (\cite{1} \tm{1} \gp{II}{5}).

\item Тензор скоростей деформации (\cite{1} \tm{1} \gp{II}{6,7}).

\item Вектор вихря, его кинематический смысл. Циркуляция скорости, связь с~вихрем.
Потенциал скорости. (\cite{1} \tm{1} \gp{II}{7}).

\subsection*{Универсальные законы сохранения\\и соответствующие дифференциальные уравнения}

\item Закон сохранения массы для конечного объёма сплошной среды. Уравнение неразрывности
в~эйлеровых и~лагранжевых переменных. (\cite{1} \tm{1} \gp{III}{1}; \gp{IV}{4}; \cite{2} \gp{2}{8})

\item Закон сохранения количества движения для конечного объема сплошной
среды. Вектор напряжений. Тензор напряжений. Механический смысл
компонент в~декартовой системе координат. Дифференциальные уравнения движения
сплошной среды. (\cite{1} \tm{1} \gp{III}{2,4}; \cite{2} \gp{2}{9,10}).

\item Закон сохранения момента количества движения. Тензор моментных напряжений.
Дифференциальные уравнения момента количества движения. Симметрия тензора напряжений.\\
(\cite{1}~\tm{1}~\gp{III}{3}; \cite{2} \gp{2}{12}).

\item Закон сохранения энергии (Первый закон термодинамики). Внутренняя энергия.
Уравнение кинетической энергии. Уравнение притока тепла. Дифференциальные уравнения
энергии и~притока тепла. Теплопроводность. Закон Фурье.
(\cite{1} \tm{1} \gp{V}{1,2,4,7,8}; \cite{2} \gp{3}{14}).

\item Второй закон термодинамики. Производство энтропии в~необратимых процессах.
Дифференциальное уравнение изменения энтропии. Производство энтропии в~процессе
теплопроводности. (\cite{1}~\tm{1}~\gp{V}{4,5,8}; \cite{2}~\gp{3}{15}).

\item Условия на~поверхностях сильного разрыва в~сплошных средах,
следующие из законов сохранения массы, количества движения, момента количества движения,
энергии и~второго закона термодинамики. Ударные волны и~тангенциальные разрывы.
(\cite{1} \tm{1} \gp{VII}{4}; \cite{2} \gp{3}{18}).

\subsection*{Классические модели сплошных сред}

\item Идеальная жидкость. Уравнения Эйлера. Интегралы Бернулли и~Коши~-- Лагранжа.
Уравнения для потенциального движения идеальной несжимаемой жидкости.\\
(\cite{1}~\tm{1}~\gp{IV}{1,7}; \tm{2}~\gp{VIII}{2,3,11,12}; \cite{3}~\g{I}; \cite{2}~\gp{5}{20,22}).

\item  Идеальная сжимаемая жидкость. Полная система уравнений. Совершенный газ.
Адиабата Пуассона. Энтропия совершенного газа. Система уравнений
газовой динамики. Полная система уравнений для идеального совершенного теплопроводного газа.
(\cite{1} \tm{1} \gp{IV}{2,7}; \tm{2} \gp{VIII}{5,11}; \cite{5} \gp{I}{7}; \cite{2} \gp{5}{25}).

\item Начальные и~граничные условия на~поверхности твердых тел и~на~свободных
поверхностях для системы уравнений идеальной жидкости и~газа.
(\cite{1} \tm{1} \gp{VII}{1}; \cite{5} \gp{1}{7} \gp{II}{2}; \cite{7}).

\item Модель вязкой жидкости. Термодинамические соотношения для вязкой жидкости.
Линейно-вязкая жидкость. Уравнения Навье~-- Стокса. Полные системы уравнений
для линейно-вязкой несжимаемой жидкости и~для линейно-вязкого совершенного
теплопроводного газа.
(\cite{1} \tm{1} \gp{IV}{2}; \gp{V}{7}; \cite{4} \gp{XV}{141},
\cite{6} \gp{II}{2}; \cite{7}; \cite{2} \gp{5}{20,23,25}).

\item Начальные и~граничные условия для системы уравнений вязкой жидкости.
(\cite{1}~\tm{1}~\gp{VII}{1}; \cite{6}~\gp{II}{2}; \cite{7}).

\item Различные приближения для системы уравнений вязкой жидкости
по~числу Рейнольдса: уравнения Эйлера, уравнения Стокса. Пограничный слой.
(\cite{1} \tm{2} \gp{VIII}{23,24,25}; \cite{4} \gp{XII}{105,106}; \cite{6} \g{V,VII,VIII}; \cite{9} \gp{II}{23}; \g{IV})

\item Турбулентность. Уравнения Рейнольдса. Полуэмпирические теории турбулентности.\\
(\cite{1} \tm{2} \gp{VIII}{22}; \cite{4} \gp{XIII}{118,119-123}; \cite{6} \g{XII}; \cite{3} \g{III}).

\item Модель упругой среды. Внутренняя энергия и~свободная энергия как
термодинамические потенциалы. Система уравнений нелинейной теории упругости в~начальной
лагранжево системе координат. Тензор, напряжений Кирхгофа~-- Пиолы. Уравнения
нелинейной теории упругости в~эйлеровой системе координат. Начальные и~граничные условия.
(\cite{1} \tm{2} \gp{IX}{2}; \cite{10}; \cite{2} \gp{6}{27,29}).

\item Изотропная линейная термоупругая среда с~малыми деформациями. Полная
система уравнений. Постановки задач в~напряжениях и~перемещениях. Уравнения Навье~-- Ламе,
уравнения Бельтрами~-- Мичелла. Принцип Сен-Венана.
(\cite{1} \tm{2} \gp{IX}{2,5}; \cite{8} \g{I}; \cite{9}; \cite{11} \gp{8}{8.2-8.5}; \gp{13}{13.1}).

\item Основные понятия теории пластичности. Поверхность нагружения,
нагружение и~разгрузка, пластические деформации. Идеально-пластические тела и~тела с~упрочнением.
(\cite{1}~\tm{2}~\gp{X}{1,2}; \cite{11}~\gp{15}{15.1}; \g{16};
\cite{12}~\tm{4}~\g{I}; \cite{13}~\g{1,2}; \cite{2}~\gp{7}{31}).

\item Определяющие соотношения в~теории пластичности. Деформационные теории и~теории течения.
Ассоциированный закон. Полная система уравнений для упруго-идеально-пластической
среды в~теории Прандтля~-- Рейсса.
(\cite{1}~\tm{2} \gp{X}{4}; \cite{11} \gp{15}{15.2}; \gp{16}{16.19,16.2,16.4};
\cite{12}~\gp{II}{1}; \cite{13}~\g{II}; \cite{2}~\gp{7}{31}).

\subsection*{Электродинамика сплошных сред}

\item Взаимодействие сплошных сред с~электромагнитным полем. Плотность заряда
и~плотность тока. Сила Лоренца. Джоулево тепло. Закон Ома.
Уравнения электродинамики и~механики сплошных сред с~учетом зарядов и токов.
(\cite{1} \tm{1} \gp{VI}{4,5,6}; \cite{14}; \cite{15}; \cite{2} \gp{8}{35}).

\item Уравнения магнитной гидродинамики, уравнения электрогидродинамики.
(\cite{1} \tm{1} \gp{VI}{6}; \cite{14}, \cite{15}; \cite{16}; \cite{2} \gp{8}{36,37}).

\end{nums}

\begin{thebibliography}{16}
\setlength\itemsep{-.5mm}
\bibitem{1} Седов\,Л.\,И. Механика сплошной среды, т.\,1, т.\,2~--- М.: Наука, 1994. 5-е изд.
\bibitem{2} Галин Г.\,Я., Голубятников А.\,Н., Каменярж Я.\,А., Карликов В.\,П.,
Куликовский А.\,Г., Петров А.\,Г., Свешникова Е.\,И., Шикина И.\,С., Эглит М.\,Э. Механика сплошных сред
в~задачах. т. 1, 2~--- М.: Московский лицей, 1996.
\bibitem{3} Ландау Л.\,Д., Лифшиц Е.\,М. Гидродинамика.~--- Л.: Наука, 1986. 3-е изд.
\bibitem{4} Лойцянский Л.\,Г. Механика жидкости и газа.~--- Н.: Наука, 1978. 5-е изд.
\bibitem{5} Черный, Г.\,Г. Газовая динамика.~--- М.: Наука, 1988.
\bibitem{6} Слезкин Н.\,А. Динамика вязкой несжимаемой жидкости. М.: Гос. изд.-во физ.-тех. лит-ры, 1955.
\bibitem{7} Антонцев С.\,Н., Кажихов А.\,В., Монахов В.\,Н.
Краевые задачи механики неоднородных жидкостей.~--- Новосибирск: Наука, 1983.
\bibitem{8} Ландау Л.\,Д., Лифшиц Е.М. Теория упругости.
\bibitem{9} Амензаде Ю.\,А. Теория упругости.~--- М.: Высшая школа, 1974
\bibitem{10} Бленд Д.\,Р. Динамическая теория упругости.~--- М.: Мир, 1972.
\bibitem{11} Работнов Ю.\,Н. Механика деформируемого твердого тела.~--- М.: Наука, 1988.
\bibitem{12} Клюшников В.\,Д. Математическая теория пластичности.~--- М.: Изд-во МГУ, 1979.
\bibitem{13} Хилл Р. Математическая теория пластичности.~--- М.: ГИТТЛ, 1956.
\bibitem{14} Ландау Л.\,Д., Лифшиц Е.М. Электродинамика сплошных сред.~--- М.: Наука, 1992, 3-е изд.
\bibitem{15} Пуликовский А.\,Г., Любимов Г.А. Магнитная гидродинамика.~--- М. Физматгиз, 1962.
\bibitem{16} Ратажин А.\,Б., Грабовский В.\,И., Лихтер В.\,А., Шульгин В.\,И.
Электродинамические течения.~--- М.: Наука, 1983.
\end{thebibliography}

\medskip\dmvntrail

\end{document}
