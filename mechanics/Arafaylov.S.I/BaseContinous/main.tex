\documentclass[10pt]{article}
\usepackage[xe]{dmvn}
\usepackage{polyglossia}
\usepackage{unicode-math}
\usepackage{fontspec}
\usepackage{makeidx}
\defaultfontfeatures{Scale=MatchLowercase, Mapping=tex-text}
\setmainfont{CMU Serif}
\setsansfont{CMU Sans Serif}
\setmonofont{CMU Typewriter Text}
\setmathfont{xits-math.otf}
\newcommand{\type}[2]{\ensuremath{\colon\!#1→#2}}
\begin{document}
\dmvntitle{Курс механики}
{ сплошной среды}
{Лектор -- Арафайлов С.И}
{IV(V) курс, 9 семестр, поток математиков, ЕНС}
{Москва, 2012г}
\section{Введение}
\begin{petit}
	Увы, обстоятельства не позволяют ходить на эту замечательную лекцию.
	Господа, подхватите знамя!
\end{petit}
\begin{df}
  Скорость звука -- скорость распространения малых деформаций.
\end{df}
\begin{df}[Критерия сплошности среды]
  Пусть $λ$ -- длина свободного пробега, $l$ -- характерная длина
  задачи.  Среду можно считать сплошной, если число Кнутся $K_n=\frac
  λl \ll 1$.  Также требуем, что бы $τ \ll T$, где $τ$ -- время выхода
  на равновесие, $T$ -- характерный период возмущений.
\end{df}
\section{Деформация, напряжение}

Рассмотрим две системы координат (картинок пока нет). Одна,
пространственная, также называемую Эйлеровой, обозначается
$x_1x_2x_3$. Другая, жестко связанная с телом, обозначается
$ξ_1ξ_2ξ_3$, называется Лагранжевой(материальной).

\begin{denote}
  Пусть есть некоторая величина $T(\bar x,t) = T(\bar x(\bar ξ, t), t)$.
  Тогда будем обозначать $\pf Tt\evn{ξ=\const}$ как $\frac{dT}{dt}$.
  Заметим, что имеет место равенство
  \begin{equation*}
    \frac{dT}{dt} = \pf Tt
  \end{equation*}
\end{denote}
\end{document}
