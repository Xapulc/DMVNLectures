\begin{petit}
  Счастье то какое! Мне не придется это доказывать самому!
\end{petit}
Вспомним, что нам оставался случай $\lliml{a→∞} P_V(a)^Φ = ±∞$.
Поймем, что этого быть не может. Число слагаемых в $E_V^Φ$ меньше, чем
$C·\hm V$. Каждое слагаемое тоже ограничено, откуда $E_V^Φ(ω) < K·\hm V$.
\begin{align*}
  \hm S^{\hm V} e^{-K\hm{V}} ≤ Z_V^Φ ≤ \hm S^{\hm V} e^{K\hm{V}}\\
  (\ln S -K) \hm V ≤ \ln Z_V^Φ ≤ \hm V(\ln S + K)\\
  (\ln S -K)  ≤ P_V^Φ ≤ (\ln S + K)
\end{align*}
И все.

\begin{df}
  Пусть $\nw Φ <∞$. Тогда $f^Φ(ω) = \suml{0∈V∈\Fc}{}\inv{\hm V}Φ_V(ω_V)$.
  Определим $\hn{f^Φ} ≝ \supl{ω}\hm{f^Φ(ω)}$.
\end{df}
\begin{denote}
  $K(ω, ω') = \min\hc{n∈ℵ| ω_{V_n} ≠ ω'_{V_n}}$, где $V_n = [-n, n] ^d$.
\end{denote}
\begin{df}
  $d(ω,ω') = 2^{-K(ω,ω')}$
\end{df}
\begin{note}
  Метрика это есть. Любое цилиндрическое множество (в фиксированных
  местах есть фиксированные спины) является открытым в этой топологии.
\end{note}
Отображение $Φ→f^Φ$ линейно для любого $ω$. В этой топологии $Ω$
компактно.


Вспомним про преобразование $τ_t\cln τ_t(ω)(x) = ω(t+x)$. При каждом
$t$ отображение непрерывно в нашей топологии на $Ω$. Более того, это
преобразования образую группу. Рассмотрим вероятностные меры на $Ω$
--- $M$ На ней $σ$ - алгебра -- это Борелевская $σ$ - алгебра (она же
цилиндрическая).  Пусть $μ∈M$, $B∈\Bc(Ω)$ и $μ(τ_t^{-1}B) =μ(B)$,
тогда $μ$ называется $τ_t$--инвариантной мерой. Если она инвариантна
для любого $τ_t$, то она $\hc{τ_t}$--инвариантна.
\begin{df}
  Энтропия -- неотрицательное число, характеризующее динамическую
  систем $h_μ = h_μ(\hc{τ_t})$.
\end{df}
\begin{denote}
  $τ = \hc{τ_t}$
\end{denote}
\begin{df}
  $f(τ)$ -- множество $τ$ -- инвариантных борелевских мер на $Ω$.
\end{df}
\begin{theorem}[Вариационный принцип]
  Если $\ns{P^Φ} < ∞$, то $\supl{μ}(h_μ - \intl{Ω}{}f^Φdμ) = P^Φ$
\end{theorem}
\begin{proof}
  \begin{petit}
    Ереси много здесь есть. Мастера Йоды мудрость требуется нам.
  \end{petit}
  Мы будем доказывать, что этот супремум достигается. Равенство
  доказывать не будет.  Пусть $(X, τ)$ -- динамическая система, $X$ --
  компактное метрическое пространство. Пусть $f\type{X}{\R}$ --
  непрервына, $M(τ)$ -- $τ$-инвариантные вероятностные борелевские меры.

  Если $τ$ -- аминабельная группа(пусть даже не коммутативная), то
  можно рассмотреть функционал $μ→h_μ(τ) - μ(τ)$. Т.е можно
  рассмотреть $\supl{μ} μ→h_μ(τ) - μ(τ) ≝ P$.
\end{proof}

%% Local Variables:
%% compile-command: "xelatex -halt-on-error -file-line-error main.tex"
%% End:
