\section{Основные состояния}
Посмотрим, что происходит при различных $β$.
\begin{df}
 Спонтанная намагниченность
 \begin{equation*}
   μ(β) ≝\uliml{N→+∞} P^{(β)}_{V_N,+}(ω(0,0) = +1)
 \end{equation*}
\end{df}

\begin{problem}
  Найти наименьшее $β_0$, что для всех $β>β_0$ выполнено $μ(β) > ½$.
\end{problem}
\begin{answer}
  Нобелевский лауреат Ларс Анзайгер нашел решение. Доказательство
  довольно сложное с использованием алгебры. Есть альтернативное
  доказательство Ландау-Фавыченко (возможно с ошибкой). В курсе
  решение задачи не рассматривается.
\end{answer}

\newcommand{\Po}{P^{(β)}_{V,\bar ω}}

Зафиксируем сосуд $V$ и
граничные условия $\bar ω$ и устремим $β→+∞$.  Тогда вероятностная
мера $\Po$ будет сходиться к равномерному распределению на
конфигурациях, имеющих минимум энергии (обозначаются $Ω^{\min}$).

\begin{df}
  $Ω^{\min}$ -- точки основного состояния (`ground state').
\end{df}
\begin{petit}
  Возможно, это и не так. Проверяйте.
\end{petit}
\begin{note}
  Точки основного состояния для $\bar ω ≡ ±1$ -- конфигурации $ω ≡ ±1$.
\end{note}
\begin{problem}
  Найти $Ω^{\min}$ для квадрата, у которого противоположные стороны
  имеют граничные условия -- константы разного знака.
\end{problem}

\begin{petit}
  Далее идет `best efford`.
\end{petit}

Можно рассмотреть альтернативную модель, в которой есть так называемое
«внешнее поле» и энергия определяется следующим образом
\begin{equation*}
  E(ω,\bar ω, h) = E(ω,\bar ω) - h\suml{t∈V}{}ω(t)
\end{equation*}
Оказывается, что при $h≠0$ вся картинка существенно меняется.
\begin{problem}
  Найти основные состояния для этой модели.
\end{problem}
\begin{answer}
  При любых граничных условиях основное состояние $ω ≡ \sgn(h)$ при достаточно
  большом объеме сосуда.
\end{answer}
\begin{problem}
  $\displaystyle P^{(β, h)}_{V, \bar ω(V)}(ω(0,0) = 1) \stackrel[N→∞]{β→∞}{\longrightarrow} Φ(h)$
\end{problem}
%% Local Variables:
%% compile-command: "xelatex -halt-on-error -file-line-error main.tex"
%% End:
