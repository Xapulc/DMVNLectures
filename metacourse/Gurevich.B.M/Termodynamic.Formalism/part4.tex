\section{Одномерная модель Изинга.}

Пусть теперь $S$ -- спиновое пространство, $\hm{S} < ∞$, $T = \Z$,
$V⊆T$, $ω\type{V}{S}$. Пусть определена функция $U\type{S×S}{R}$ и рассмотрим
формальный ряд
\begin{equation*}
  H(ω) = \suml{\mat{t, t'∈T\\ \hm{t - t'} = 1}}{} U(ω(t), ω(t'))
\end{equation*}

Задача та же, что и в предыдущем разделе. Сосуд $V_N = [-N, N]$, граничные условия
$s^- = ω(-N - 1)$, $s^+ = ω(N+1)$. Интересно, что происходит при $N→∞$.
\begin{df}
  Матрица Больцмана $B(s,s') ≝ e^{-U(s,s')}$
\end{df}
\begin{lemma}
  Статсумма $Z = B^{2N+2}(s^-,s^+)$
\end{lemma}
\begin{proof}
  Доказывается по индукции.
\end{proof}
\begin{problem}
  Пусть $A$ -- матрица, все элементы которой положительны, и пусть
  $λ = λ(A)$ -- максимальное собственное значение. Тогда существует в смысле
  поэлементной сходимости предел $ \liml{n→+∞}\frac{A^n}{λ^n}$.
\end{problem}
\begin{df}
  Будем называть матрицу положительной, если все её элементы положительны.
\end{df}
\begin{theorem}[Перрон, 1909]
  Пусть $A$ -- положительная матрица. Тогда её максимальное по модулю
  собственное значение вещественно и кратности $1$. Причем выполнено
  \begin{equation*}
    Ax = λx \quad yA = λy\quad x_i,y_i >0
  \end{equation*}
\end{theorem}
\begin{theorem}[Фробениус]
  Пусть существует $n_0$, что $A^{n_0}$ -- положительная матрица.
  Если при этом матрица $A$ -- неотрицательна, то для неё верно
  утверждение теоремы Перрона.
\end{theorem}
\begin{hint}
  Утверждение вытекает из эргодической теоремы для цепей Маркова.
\end{hint}
\begin{petit}
  Здесь пропущено примерно пол-страницы которые я не понимаю.
\end{petit}



%% Local Variables:
%% compile-command: "xelatex -halt-on-error -file-line-error main.tex"
%% End:
