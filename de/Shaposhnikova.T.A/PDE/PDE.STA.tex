\documentclass[a4paper,draft]{article}
\usepackage[xe]{dmvn}
\usepackage{dmvnadd}

\begin{document}
\tocsubsubsectionparam{3.5em}
\tocsubsectionparam{3em}
%\tocsectionparam{1.5em}
\dmvntitle{Курс лекций по}{уравнениям в частных производных}{Лектор\т Татьяна Ардолионовна Шапошникова}
{III курс, 5--6 семестр, поток математиков}{Москва, 2005 г.}

\tableofcontents

\pagebreak

\section{Лекция 1}

\subsection{Введение}
Необходимость в изучении дифференциальных уравнений с частными
производными возникла в связи с тем, что математическое описание многих
физических явлений приводит к определению решений уравнений с частными производными.

Рассмотрим некоторые примеры:

$$u_{tt}=a^2u_{xx}+f(x,t)$$
Такие уравнения описывают многие колебательные процессы.
$$u_t=k^2\De u+ f(x,t)$$
Это уравнение называется \emph{уравнением теплопроводности} и задаёт распределение температуры в однородном теле.
$$\De u =f(x)$$
Уравнение такого вида называется \emph{уравнением Пуассона} и описывает стационарное распределение температуры
при наличии тепловых источников внутри тела с плотностью $f(x)$.

Дадим несколько определений:

\begin{df}
\emph{Дифференциальными} называются уравнения, в которых неизвестными являются функции
одного или нескольких переменных, причем в уравнения входят не только сами функции, но и их производные.
Если неизвестными являются функции нескольких переменных (не менее двух), то уравнения называются
\emph{уравнениями в частных производных.}
\end{df}

В дальнейшем мы будем рассматривать только одно уравнение в частных производных с одной неизвестной функцией.

\subsection{Линейные уравнения в частных производных второго порядка}

Рассмотрим уравнение
\eqn{\label{1.1}
F\hr{x,u(x),u_{x_1}(x),\dots,u_{x_n}(x),\dots, \pf{^m u}{x_{n}^m}(x)}=0,}
где $x=(x_1,\dots,x_n)$.

Скажем, что это уравнение в частных производных порядка $m$, если $m$ является максимальным порядком
производных в этом уравнении.

\begin{df}
Уравнение \eqref{1.1} называется \emph{линейным}, если $F$, как функция переменных
$$u, u_{x_1}\sco u_{x_n}\sco \pf{^m u}{x_{n}^m}$$
является линейной функцией.
\end{df}

Будем рассматривать линейные уравнения в частных производных второго порядка. Такие уравнения имеют вид
\eqn{\label{1.2}
\suml{i,j=1}{n} a_{ij}(x)u_{x_ix_j}+\suml{i,j=1}{n} a_j(x)u_{x_j}+a(x)u=f(x),}
где $\Om \subs \R^n$, $a_{ij}(x), a_j(x), a(x), b(x)$
достаточно гладкие функции,
заданные в $\Om$.

Обозначим $A(x):=(a_{ij}(x))$.

\begin{stm}
Матрицу $A$ можно считать симметрической.
\end{stm}
\begin{proof}
Действительно, в силу гладкости функции $u$ смешанные производные второго порядка совпадают, поэтому
соответствующие элементы матрицы А можно <<перераспределить>>, введя
$a'_{ij}=\frac{a_{ij}(x)+a_{ji}(x)}{2}$, тогда $a_{ij}(x)=a'_{ij}(x)+a''_{ij}(x)$, где
$a''_{ij}=\frac{a_{ij}(x)-a_{ji}(x)}{2}$ и $a''_{ij}=-a''_{ji}$. После такого преобразования мы получим
то же самое уравнение, но с симметрической матрицей $A'(x)$.
\end{proof}

Сразу отметим то обстоятельство, что существуют уравнения вида \eqref{1.2} с бесконечно дифференцируемыми
коэффициентами, которые в любой окрестности точки $x_0$ не имеют  ни одного решения. Такие
уравнения называются \emph{локально неразрешимыми.}

Примером такого уравнения может служить
\eqn{(x_2^2-x_3^2)u_{x_1x_2}+(1+x_1^2)(u_{x_2x_2}-u_{x_3x_3})-x_1x_2u_{x_1x_3}-(x_1x_2u)_{x_1x_2}+x_1x_3u_{x_1x_3}=
f(x)-(x_1x_3u)_{x_1x_3}.}
Л.\,Хермандер доказал, что такое уравнение не имеет ни одного решения при некоторой $f(x)\in \Cb^\infty(\R^n)$
для любой области $\Om\subs\R^n$.

\subsection{Задача Коши. Характеристики}

В случае одного пространственного переменного уравнение \eqref{1.2} выглядит так:
\eqn{y''+a(x)y'+b(x)y=f(x).}
Вспомним, что задачей Коши в этом случае являлась задача о нахождении решения этого уравнения,
удовлетворяющего в точках $x_0, x_1$, где $[l_1,l_2]\subs\R$, а $x_n\in (l_1,l_2)$, начальным условиям
\eqn{\case{y(x_0)=y_0,\\
y'(x_1)=y_1,} \quad y_0, y_1 \in\R.}
Заметим, что в этом случае всегда существует решение задачи Коши.

Перейдем к постановке аналогичной задачи для уравнения в частных производных \eqref{1.2}. Возьмем некоторую
достаточно гладкую $(n-1)$\д мерную поверхность $S\subs \Om$, заданную уравнением
\eqn{F(x)=0,\quad F\in \Cb^2(\Om),}
и пусть $|\nabla F|\neq 0$ при всех $x\in S$. Пусть на $S$ задано векторное поле
$l(x)=(l_1(x),\dots,l_n(x))$, причём $|l(x)| \neq 0$, не касающееся поверхности $S$, то есть
\eqn{\pf{F}{l}(x)=\frac{(l(x),\nabla F(x))}{|l(x)|}\neq 0 \quad \forall x\in S.}
Тогда задачей Коши будет нахождение решения уравнения \eqref{1.2} с начальными условиями
\eqn{\label{1.3}u\evn{S}=u_0(x),}
\eqn{\label{1.4}\pf{u}{l}(x)\evn{S}=u_1(x),}
где $u_0(x)$ и $u_1(x)$\т функции, заданные на $S$.

Пусть $x_0\in S$, $U(x_0)$\т шар радиусом $\rho$ с центром в $x_0$, причем $\ol{U(x_0)}\subs\Om$.
Обозначим $S_0=S\cap U(x_0)$ и будем решать задачу Коши в окрестности точки $x_0$ с условиями
\eqn{\label{InitialCoditions}
\begin{aligned}
u(x)\in \Cb^2(Q(x_0)),\\
u(x)\evn{S_0}=u_0(x),\\
\pf{u}{l}(x)\evn{S_0}=u_1(x).
\end{aligned}}
Поскольку $|\nabla F(x_0)|\neq 0$, то для определенности будем считать, что $F_{x_n}(x_0)\neq 0$. Рассмотрим следующее
преобразование $y=\Fc(x)$:
\eqn{\case{y_1 &= x_1-x_{0_1} \equiv F_1(x)\\
&\ldots \\
y_{n-1}&= x_{n-1}-x_{0_{n-1}}\equiv F_{n-1}(x)\\
y_n&= F(x)\equiv F_n(x)}}
При этом $J(x)\neq 0$, где $J(x)$ имеет следующий вид
\eqn{J(x) := \mbmat{1 & 0&\ldots &0 &0\\
0 & 1&\ldots &0 &0\\
\vdots& \vdots& \ddots &\vdots &\vdots \\
0 & 0&\ldots &1 &0\\
F_{x_1} & F_{x_2}& \ldots &F_{x_{n-1}} &F_{x_n}}}
Легко заметить, что $J(x)=F_{x_n}(x)$, поэтому $J(x)\neq 0$ в
окрестности $x_0$. Таким образом, наше преобразование $y=\Fc(x)$ взаимо\д однозначно
отображает $U(x_0)\subs\R^n_x$ на $V(0)\subs\R^n_y$, то есть часть $U(x_0)$, принадлежащая
исходной поверхности, перейдет в окрестность точки $0$ на гиперплоскости $y_n=0$.
Обозначим $\Sigma=V(0)\cap\{y_n=0\}$, где $\Sigma:=\Phi(S)$, а $v(y):=u\br{x(y)}$,
и выразим частные производные получившейся функции:
\eqn{\begin{aligned}
u_{x_i} &=\sumkun v_{y_k}y_{k_{x_i}} = \sumkun v_{y_k}\pf{F_k}{x_i},
u_{x_i x_j}&=\suml{k,p=1}{n} v_{y_k}v_{y_p}\pf{F_k}{x_i}\pf{F_p}{x_j} + \sumkun v_{y_k}\pf{F^2_k}{x_i \pd x_j}.
\end{aligned}}
Тогда уравнение \eqref{1.2} запишется в следующем виде:
\eqn{\suml{i,j=1}{n} a_{ij}(x)\hr{\suml{k,p=1}{n} v_{y_k}v_{y_p}\pf{F_k}{x_i}\pf{F_k}{x_j} +
\sumkun v_{y_k}\pf{F^2_k}{x_i \pd x_j}}+\suml{i=1}{n} a_{i}(x)\sumkun v_{y_k}\pf{F_k}{x_i}+a(x(y))v=\wt{f}(y),}
или, что то же самое,
\eqn{\suml{k,p=1}{n} b_{kp}(y)v_{y_k}v_{y_p} + \sums{k=1}{n}b_{k}(y)v_{y_k} + b(y)v = \wt{f}(y),}
где
\eqn{b_{kp}(y)=\suml{i,j=1}{n}a_{ij}(x(y))\pf{F_k}{x_j}\pf{F_p}{x_i}.}

Заметим, что
\eqn{b_{nn}=\suml{i,j=1}{n} a_{ij}(x(y))F_{x_i}F_{x_j}=\br{A(x(y))\nabla F,\nabla F}, \text{ где } \nabla F \equiv \grad F.}

Посмотрим теперь, какой вид примут начальные условия \eqref{InitialCoditions}.
\textbf{Первое} условие превратится в
\eqn{u(x)\evn{S_0}=u(x(y))\evn{\Sigma}=v_0(y'), \quad y'=(y_1,y_2,\ldots,y_n),}
а \textbf{второе} условие перейдёт в
\eqn{\frac{du}{dl}(x)\evn{S_0}=\frac{1}{|l|}({\nabla}_x u,l)=\frac{1}{|l|}\suml{j=1}{n}
u_{x_j}l_j(x)=\frac{1}{|l|}\suml{j=1}{n} \sumkun v_{y_k}y_{k_{x_j}}l_j(x)=\sumkun  v_{y_k}\pf{F_k}{l}.}
Здесь мы пользуемся тем, что
\eqn{\frac{1}{|l|}\suml{j=1}{n}\pf{F_k}{x_j} l_j(x)=\pf{F_k}{l}.}
Пусть
\eqn{\la(y)=\hr{\pf{F_1}{l}\sco\pf{F_n}{l}}.}
Тогда $({\nabla}_y v,\la (y))=\sumkun  v_{y_k}\pf{F_k}{l}$.

$$({\nabla}_y v,\la (y))\evn{\Sig}=v_1(y')$$

$v_{y_k}(y')\evn{\Sigma}$ для $k=1,\ldots,n-1$ мы находим из условия $v\evn{\Sigma}=v_0(y')$
$$v_{y_n}\frac{dF_n}{dl}=v_1(y')-\suml{j=1}{n-1}v_{y_k}\pf{F_k}{l}$$
На $\pf{F_n}{l}$ можно поделить, так как $l$ не касается $S$, то есть $\pf{F_n}{l}(x_0) \neq 0$.
Итак, мы нашли все первые производные из начальных условий.

Значения $v_{y_k y_p}(y')\evn{\Sigma}$ для $k,p=1,\ldots,n-1$ мы находим из первого
условия. $\quad v_{y_k y_n}(y')\evn{\Sigma}$ для $k=1,\ldots,n-1$ мы
находим из выражения для $v_{y_n}$.
\eqn{b_{nn}v_{y_n y_n}\evn{\Sigma}=\wt{f}\evn{\Sigma}- \suml{k,p=1;p+k<2n}{n}
b_{pk}(y)v_{0_{y_k}}v_{0_{y_p}}\evn{\Sigma}-\sumkun b_k(y)v_{0_{y_k}}\evn{\Sigma}-b(y)v.}
Если $b_{nn} \neq 0$ на $\Sigma$,то мы находим $v_{y_n y_n}$. Если же $b_{nn}=0$, то получаем
условие на начальные данные задачи, и она может быть неразрешима.

\begin{df}
Поверхность $S: F(x)=0, \nabla F(x) \neq 0 \fa x \in S$ называется \emph{характеристикой}
\emph{(характеристической поверхностью)} уравнения \eqref{1.2}, если $\br{A(x(y))\nabla F,\nabla F} = 0$ для
$\fa x \in S$ (или нормаль $\nu (x)$ в каждой точке удовлетворяет равенству $(A(x)\nu (x),\nu (x))=0$).
Точки данной поверхности называются \emph{характеристическими}.
\end{df}
%------------------------------Лекция 2--------------------------------------------

\section{Лекция 2}

Если в точка $x^0\in S$ является характеристической, а данные задачи удовлетворяют
соответствующим соотношениям, то решений задачи Коши может быть бесконечно много.

Рассмотрим несколько примеров:
\subsection{Задача Коши с данными на характеристике}
\underline{\textsl{Пример 1.}}\\
Рассмотрим уравнение в некотором круге $U$ с центром в начале координат (n=2)
$$
u_{x_1x_1}-u_{x_2x_2}=f(x_1,x_2)
$$
Характеристиками этого уравнения будут прямые $x_1-x_2=C_1 \  ,x_1+x_2=C_2$
(это показывается по определению).
Рассмотрим задачу Коши на поверхности $x_1=x_2$ т.е
$$
u\evn{x_1=x_2}=\ph(x_1) \quad u_{x_1}\evn{x_1=x_2}=\psi(x_1)
$$

Такое уравнение с помощью замены переменных приводится к
уравнению $u_{x_2x_2}=0 $
у которого $x_2=0$ будет характеристикой. Легко проверить, что для существования
гладкого в $U$ решения этого уравнения, удовлетворяющего условиям

$$
u\evn{x_2=0}=\ph(x_1) \quad u_{x_2}\evn{x_2=0}=\psi(x_1)
$$
необходимо и достаточно, чтобы выполнялось условие $\frac{d\psi(x_1)}{dx_1}\equiv f(x_1,0)$.
Причем, если это условие выполнено, то решение предоставляется в виде

$$
u(x_1,x_2)= \int\limits_0^{x_1}d \xi_1 \int \limits_0^{x_2} f(\xi_1,\xi_2)d\xi_2+\ph(x_1)+g(x_2)
$$
где $g$\т произвольная дважды непрерывно дифференцируемая функция, удовлетворяющая условиям
$g(0)\bw=0, \quad \frac{dg(0)}{dx_2}=\psi(0)$. То есть решений бесконечно много.

\subsection{Задача о слабом разрыве решения вдоль некоторой кривой $y=\ph(x)$}
\underline{\textsl{Пример 2.}}\\
Пусть на плоскости есть область которая разделена на две части $\Om_1, \ \text{и}\  \Om_2$
кривой $l: y=\ph(x)$.
Рассмотрим уравнение
$$
a(x,y)u_{xx}+2b(x,y)u_{xy}+c(x,y)u_{yy}+d(x,y)u_x+e(x,y)u_y+g(x,y)u=f(x,y)
$$
Причём $u\in C^1(\Om) \cap C^2(\ol{\Om}_1) \cap C^2(\ol{\Om}_2)$
То есть на кривой $l$ рвутся вторые производные. Докажем, что $l$\т характеристика.

Пусть
$$
u(x,y)=
\begin{cases}
u^1(x,y)\ x,y \in \Om^1 \\
u^2(x,y) \ x,y \in \Om^2
\end{cases}
$$
Обозначим $[u]\evn{l} \equiv u^1(x,\ph(x))-u^2(x,\ph(x))=0$
$$
\frac{d[u]}{dx}=u^1_x(x,\ph(x))+u^1_y(x,\ph(x))\ph'(x)-u^2_x(x,\ph(x))-u^2_y(x,\ph(x))\ph'(x)=
[u_x]+[u_y]\ph'=0
$$
$$
\frac{d[u_x]}{dx}=[u_{xx}+[u_{xy}]\ph'=0
$$
$$
\frac{d[u_y]}{dx}=[u_{xy}]+[u_{yy}]\ph'=0
$$
Пусть $\la = [u_{yy}]\neq 0$, тогда $[u_{xy}]=-\la\ph',\ [u_{xx}]=\la\ph'^2$
Имеем
$$
a[u_{xx}]+2b[u_{xy}]+c[u_{yy}]=0
$$
т.е
$$
a\la \ph'^2-2\la b \ph'+c\la=0
$$
После сокращения на $\la$, несложно будет увидеть, что это, по определению, и есть уравнение
характеристики.

\subsection{Нехарактеристическая задача Коши для линейного уравнения в частных производных второго порядка}

Теперь пусть поверхность $S$ не содержит характеристических точек.
Напомним, что у нас задана $n$\д мерная область $Q$  и $(n-1)$\д мерная поверхность $S$: $F(x)=0$, причем поверхность
делит $Q$ на две непересекающиеся области. И в $Q$ задано уравнение (1.2) т.е коэффициенты и свободный член.
А на поверхности $S$ задано нигде не касающееся поверхности векторное поле $l(x)$, $|l(x)|>0$ на $S$,
и две функции $u_0(x)$, $u_1(x)$. Пусть поверхность не имеет характеристических точек уравнения

$$
(A(x)\nabla F(x),\nabla F(x))\neq 0,\qquad x\in S.
$$
Требуется найти функцию u(x), принадлежащую $C^2(Q)$,
удовлетворяющую в $Q$ уравнению (1.2) и на $S$ начальным условиям
(1.3),(1.4). Будем называть эту задачу \textit{нехарактеристической задачей Коши}.
А заданные условия - данными задачи.


Предположим, что данные задачи бесконечно дифференциируемые
на соответствующих множествах. Предположим также, что существует
бесконечно дифференциируемое в $Q$ решение $u(x)$ задачи (1.2),(1.3),(1.4).

Тогда на поверхности $S$ через данные задачи однозначно определяются все
производные любого порядка функции $u(x)$.

\textit{Примечание. Доказательство этого факта подробно описано
в книге Михайлова Дифференциальные уравнения в частных производных на стр. 15-17.}



%----------------------------Лекция 3----------------------------------------




\section{Лекция 3}

\subsection{Теорема Ковалевской}

\underline{\textit{\textbf{Теорема.}}}

1. Если данные задачи Коши(1.2) аналитичны, т.е. коэффициенты
уравнения и правая его часть аналитичны в $\Om$, а $l, u_0,
u_1$ аналитичны на $S$, где $S$\т характеристика, то $\exists
\Om' \subs \Om, S \subs \Om'$, где существует
аналитическое решение задачи Коши для уравнения (1.2).

2. В любой подокрестности $\Om''\supset S$ не существует
решения, отличного от найденного в пункте 1.

\underline{\textsl{Пример Ковалевской.}}

$$
u_t=u_{xx},\quad x\in\R^1,\quad t\ge 0,\quad
u\evn{t=0}=\frac {1} {1+x^2},\quad t=\const - \mbox{ характеристика}
$$
Аналитического решения нет ни в какой окрестности 0.\\
Докажем от противного. Пусть $u(x,t)=\suml{i,j=0}{\bes}
h_{ij}x^it^j$

$$
\frac {1} {1+x^2}=1-x^2+x^4-\ldots
$$
Имеем $h_{i0}=0$ при нечетных $i$, $h_{i0}=(-1)^{i/2}$ при четных
$i$.
$$
\suml{i,j}{\bes}  j h_{ij}x^it^{j-1}=\suml{i,j}{\bes}
i(i-1)h_{ij}x^{i-2}t^j
$$
Отсюда $h_{m,p+1}(p+1)=h_{m+2,p}(m+2)(m+1)$. Получаем
$h_{2n+1,j}=0 \quad \fa n\ge 0,j\ge 0$, а также
$$
h_{2m,p}=h_{2(m+1),p-1}\frac
{(2m+1)(2m+2)}{p}=\ldots=h_{2(m+p),0}\frac
{\prod_{k=2m+1}^{2m+2p}k}{p!}=(-1)^{m+p}\frac{(2m+2p)!}{(2m)!p!}
$$
Итак,
$u(x,t)=\suml{m,p=0}{\bes}(-1)^{m+p}\frac{(2m+2p)!}{(2m)!p!}x^{2m}t^p$.
При $x=0$ ряд имеет вид $\suml{p=0}{\bes}(-1)^p\frac
{(2p)!}{p!}t^p$. Легко показать, что радиус сходимости
$R=\liml{p\ra \bes} \left|\frac {a_p}{a_{p+1}}\right|=0$

\subsection{Классификация линейных УРЧП 2-го порядка. О приведении их к каноническому виду}

Пусть
$$
\suml{i,j=1}{n} a_{ij}(x)u_{x_i}u_{x_j}=\Phi(x,u,\nabla u),
\quad a_{ij}\equiv a_{ji}
$$
Рассмотрим форму $\suml{i,j=1}{n}
a_{ij}(x)\xi_i \xi_j$ и замену $\xi=T\eta$, тогда
$\xi_i=\sumkun  t_{ik}\eta_k$, $\xi_j=\suml{p=1}{n}
t_{jp}\eta_p$ и наша форма примет вид
$$\suml{i,j=1}{n}
a_{ij}(x)\sumkun  t_{ik}\eta_k\suml{p=1}{n}
t_{jp}\eta_p=\suml{k,p=1}{n}\left(\suml{i,j=1}{n}
a_{ij}(x)t_{ik}t_{jp}\right)\eta_k\eta_p
$$
Обозначим $b_{kp}=\suml{i,j=1}{n} t_{ik}a_{ij}t_{jp}$. Пусть
$y=T_tx$, $T_t=(\wt{t}_{kp})$, т.е. $y_p=\sumkun
\wt{t}_{pk}x_k$. Тогда
$$
u_{x_i}=\suml{p=1}{n} v_{y_p}\wt{t}_{pi},\quad
u_{x_ix_j}=\suml{k,p=1}{n} v_{y_{pk}}\wt{t}_{pi}\wt{t}_{kj}
$$
и уравнение примет вид
$$
\suml{k,p=1}{n} b_{kp}(x)v_{y_p y_k}=\wt{\Phi}(x,v,\nabla v)
$$
Приводим форму $\suml{i,j=1}{n} a_{ij}(x)\xi_i \xi_j$ к
квадратичному виду $\eta_1^2+\ldots
+\eta_{n_+}^2-\eta_{n_++1}^2-\ldots - \eta_{n_++n_-}^2$, где $n_+$
- число собственных значений, больших 0, $n_-$ - число собственных
значений, меньших 0, $n_0$ - число собственных значений, равных 0.
Тогда наше уравнение будет выглядеть следующим образом:
$$
v_{y_1 y_1}+\ldots
+v_{y_{n_+}y_{n_+}}-v_{y_{n_++1}y_{n_++1}}-\ldots -
v_{y_{n_++n_-}y_{n_++n_-}}=\wt{\Phi}(x,v,\nabla v)
$$
Возможны следующие случаи, в соответствии с которыми мы и
классифицируем уравнения:

\medskip
1) $n_+=n$ или $n_-=n$ - \textit{эллиптический тип}.  $\nabla u$=0

2) $n_+=n-1,n_-=1$ или $n_-=n-1,n_+=1$ - \textit{гиперболический
тип}.  $u_{x_nx_n}=\suml{i=1}{n-1}u_{x_ix_i}$

3) $n_0\neq 0$  - \textit{параболический тип}.
$u_{x_n}=\suml{i=1}{n-1}u_{x_ix_i}$

4) $n_+\ge 2, n_-\ge 2, n_0 = 0$ - \textit{ультрагиперболический
тип}




%---------------------------Лекция 4--------------------------




\section{Лекция 4}
\subsection*{Корректно поставленные задачи}
Чтобы полностью описать тот или иной физический
процесс, необходимо, кроме самого уравнения, описывающего этот
процесс, задать начальное состояние этого процесса
(\emph{начальные условия}) и режим на границе той области, в
которой происходит процесс (\emph{граничные условия}).

Различают три основных типа краевых задач для дифференциальных
уравнений:

\begin{enumerate}
\item Задача Коши для уравнений гиперболического и параболического
типов: задаются начальные условия, область $G$, где происходит
процесс, совпадает со всем пространством $\R^n$,
соответственно граничные условия отсутствуют.
\item Краевая задача для уравнений эллиптического типа: задаются
граничные условия на границе, начальные условия отсутствуют.
\item Смешанная задача для уравнений гиперболического и параболического
типов: задаются и начальные,и граничные условия,
$G\neq\R^n$.
\end{enumerate}

Рассмотрим вопрос \emph{корректности} постановки краевой (в
широком смысле) задачи. Обозначим через $\beta_1$ класс функций,
среди которых мы ищем решение, а через $\beta_2$ -- класс данных
задачи. Пусть $\beta_1$ и $\beta_2$ -- банаховы пространства. На
линейное дифференциальное уравнение можно смотреть как на линейный
оператор $\mathcal{A}:\beta_1\ra\beta_2$, который ставит в
соответствие решению набор начальных и краевых данных.

Чтобы решить дифференциальное уравнение, нам нужно узнать, какому
решению соответствуют наши начальные данные, то есть найти
обратный оператор.

Поскольку уравнения математической физики описывают реальные
физические процессы, то математические постановки этих задач
должны удовлетворять некоторым естественным условиям.

Итак, краевая задача называется поставленной корректно в паре
банаховых пространств $\beta_1$ и $\beta_2$, если:
\begin{enumerate}
\item  $\fa \Phi\in\beta_2\quad \exists u\in\beta_1 :
\mathcal{A}u=\Phi$.
\item Такой элемент $u$ единственный.
\item И зависит от данных задачи непрерывно.
\end{enumerate}

Рассмотрим несколько примеров:

\textbf{1.} Задача Дирихле уравнения Лапласа:
$$\De u=0,\quad x\in\Om\subs\R^n,\quad u\mid_{\pd\Om}=f $$
Здесь $\beta_1=C(\ol\Om)$,
$\beta_2=C(\pd\Om)$,$\mathcal{A}:u(x)\ra f(x)$\\
Существование и единственность доказывались в курсе комплексного
анализа, а непрерывная зависимость от $u$ (относительно
равномерной нормы) следует из принципа максимума для гармонических
функций:
$$
 \maxl{\ol\Om} |u|\le \maxl{\pd\Om} |f|
$$
Вывод: краевая задача поставлена корректно.

\textbf{2.} Пример Адамара: Рассмотрим задачу Коши для уравнения
Лапласа:
\eqn{\case{u_{yy}=-u_{xx},\\
u\evn{y=0}=\frac{\cos nx}{n},\\
u_y\evn{y=0}=0.}
}

Будем искать решение в области
\eqn{\Om=\left\{(x,y) \Bigl|\, |y|<\delta\right\},\de>0.}

 в классах непрерывных ограниченных функций.
Решением будет функция $ u(x,y)=\frac{\ch ny \cos nx}n$.\\
Покажем, что условие 3 для такой задачи не выполняется. В самом
деле,
$$
\hn{u}_{\beta_1}= \supl{\Om} |u(x,y)|=\frac{\ch n\de}n
\ra \bes \quad n\ra\bes
$$
$$
\hn{u(x,0)}_{\beta_2}= \supl{\R} \left|\frac{\cos
nx}n\right|=\frac1n\ra0\quad n\ra\bes
$$
Что и доказывает отсутствие непрерывности. Вывод: задача Коши
поставлена некорректно.

\subsection {Задача Коши для волнового уравнения.}

Задачей Коши для волнового уравнения называется задача:

$$
 \left\{
\begin{aligned}
&u_{tt}=\De u +f(x,t),\quad x\in\R^n,\, n\le3\\
&u|_{t=0}=u_0(x)\\
&u_t|_{t=0}=u_1(x)\\
\end{aligned}
\right.
$$

Функция u(x,t) называется \emph{классическим решением} задачи
Коши, если она удовлетворяет приведенной системе и принадлежит
классу $\Cb^2(t>0) \cap \Cb^1(t\ge0)$. Здесь под множеством
$\{\,t>0\}$ понимается $\left\{(x,t)\,\bigl|\,
x\in\R^n,\,t>0\right\}$, и аналогично мы понимаем
множество $\{\,t\ge0\}$.

Покажем, что решением задачи Коши для однородного волнового
уравнения в пространстве трех измерений является функция Кирхгофа:
$$
u(x,t)=\frac1 {4\pi t}
\int\limits_{|x-\xi|=t}u_1(\xi)\,dS_\xi+\frac\pd{\pd
t}\Biggl(\frac1 {4\pi
t}\int\limits_{|x-\xi|=t}u_0(\xi)\,dS_\xi\Biggr)
$$

Для этого введем функцию
$$
u_g(x,t,\tau)=\frac 1{4\pi
t}\int\limits_{|x-\xi|=t}g(\xi,\tau)\,dS_\xi
$$

С её помощью функция Кирхгофа запишется следующим образом
(аргумент $\tau$ нам не пригождается):
$$
u(x,t)=u_{u_0}(x,t)+\frac\pd{\pd t}u_{u_1}(x,t)
$$

Изучим свойства функции $u_g(x,t,\tau)$. Итак,

\begin{lemma}
Пусть функция $g(x,\tau)$ имеет при $\tau\ge0$ непрерывные
производные по $x_1,x_2,x_3$ до порядка $k$ включительно.

Тогда $u_g(x,t,\tau)$ при $\tau,t\ge0$ имеет непрерывные
производные по $x_1,x_2,x_3,t$ до порядка $k$ включительно, и если
$k\ge2$, то $u_g(x,t,\tau)$ удовлетворяет соотношениям:

$$
 \left\{
\begin{aligned}
&\frac{\pd^2 u_g}{\pd t^2}=\De u_g, \quad
\tau\ge0,t\ge0 \\
&u_g|_{t=0}=0,\quad x\in\R^3 \\
&\frac{\pd u_g}{\pd t}|_{t=0}=g(x,\tau)\\
\end{aligned}
\right.
$$

Кроме того, $\De u_g|_{t=0}=0$.
\end{lemma}


%---------------------------Лекция 5---------------------------

\section{Лекция 5}
\textit{Доказательство леммы.}\\
Прежде всего в интеграле
$$
u_g(x,t,\tau)= \frac{1}{4\pi t}\int \limits_{|x-\xi|=t}g(\xi,\tau)dS_\xi
$$
Сделаем замену переменной $\frac{\xi - x}{t}=\eta, \ dS_\xi=t^2dS_\eta$.
Получим
$$
u_g(x,t,\tau)= \frac{t}{4\pi }\int \limits_{|\eta|=1}g(t\eta+x,\tau)dS_\eta \eqno (5.1)
$$
Из формулы (5.1) вытекает первое утверждение леммы. Кроме того, $u_g|_{t=0}=0$, поскольку при
$k\ge2$
$$
\De u_g(x,t,\tau)= \frac{t}{4\pi }\int \limits_{|\eta|=1}\De g(t\eta+x,\tau)dS_\eta \eqno (5.2)
$$
то $\De u_g|_{t=0}=0$.

Дифференцируя (5.1) по $t$ получим
$$
\frac{du_g}{dt}=
\frac{1}{4\pi }\int \limits_{|\eta|=1}g(t\eta+x,\tau)dS_\eta +
\frac{t}{4\pi }\int \limits_{|\eta|=1}(\nabla g(t\eta+x,\tau), \eta)dS_\eta \eqno(5.3)
$$
откуда
$$
u_{gt}|_{t=0}=\liml{t \ra 0} \frac{dg_u}{dt}=\frac{1}{4 \pi} \int \limits_{|\eta|=1}g(x,\tau)dS_{\eta}
=g(x,\tau).
$$
Поскольку
$$
\frac{t}{4\pi }\int \limits_{|\eta|=1}\nabla g(t\eta+x,\tau), \eta)dS_\eta
=\frac{t}{4\pi }\int \limits_{|\eta|=1}\frac{dg(t\eta+x,\tau)}{dn_\eta}dS_\eta=
\frac{1}{4\pi t}\int \limits_{|x-\xi|=1}(\frac{dg(\xi,\tau)}{dn}dS_\xi=
$$
$$
=\frac{1}{4\pi t}\int \limits_{|x-\xi|<1}\De g(\xi,\tau)d\xi=\frac{I(x,t,\tau)}{4\pi t}
$$
где $I(x,t,\tau)=\int \limits_{|x-\xi|<1}\De g(\xi,\tau)d\xi $ то (5.3) можно представить в виде
$$
\frac{du_g}{t}=\frac{1}{t}u_g+\frac{I}{4\pi t},
$$
откуда
$$
\frac{d^2u}{dt^2}=
-\frac{1}{t^2}+\frac{1}{t} \frac{du_g}{dt}+\frac{1}{4\pi t} \frac{dI}{dt} -\frac{1}{4\pi t^2}=
-\frac{1}{t^2}+\frac{1}{t} (\frac{u_g}{t}+\frac{I}{4\pi t})+\frac{1}{4\pi t} \frac{dI}{dt}-\frac{1}{4\pi t^2}=
$$
$$
=\frac{1}{4\pi t}\frac{dI}{dt}
=\frac{1}{4\pi t}\int \limits_{|x-\xi|=t}\De g(\xi,\tau)dS_\xi
=\frac{t}{4\pi }\int \limits_{|\eta|=1}\De g(x+t\eta,\tau)dS_\eta \eqno (5.4)
$$
Из (5.4) и (5.2) следует, что $u_{gtt}=\De u_g$. Лемма доказана.

Далее, в силу линейности, мы можем задачу Коши разбить в сумму трех:
$$
\begin{cases}
u_{tt}=\De u,\ x\in \R^3,\ t>0\\
u|_{t=0}=0, u_t|_{t=0}= u_1(x)
\end{cases} \eqno (a)
$$

$$
\begin{cases}
u_{tt}=\De u,\ x\in \R^3,\ t>0\\
u|_{t=0}=u_0(x), u_t|_{t=0}=0
\end{cases} \eqno (b)
$$

$$
\begin{cases}
u_{tt}=\De u + f(x,t),\ x\in \R^3,\ t>0\\
u|_{t=0}=0, u_t|_{t=0}=0
\end{cases} \eqno (c)
$$

Итак,

1) $u_{u_1}(x,t)$  решение (а), в предположении $u_1\in C^2$;

2) Пусть $u_0 \in C^3,\ \om(x,t)=\frac{d}{dt}u_{u_0}(x,t),\ \om(x,t)\in C^2$;

Покажем, что $\om (x,t)$\т решение (b).

$$
\om(x,t)|_{t=0}=u_0(x)
$$
$$
\om_t(x,0)= (u_{u_0})_{tt}|_{t=0}=(\De u_{u_0})|_{t=0}=0
$$
$$
((u_{u_0})_t)_{tt}=((u_{u_0})_tt)_{t}=(\De u_{u_0})_t)=\De(\frac{d}{dt}u_{u_0})
\Rightarrow ((u_{u_0})_t)_{tt}=\De ((u_{u_0})_t)
$$
т.е $\om(x,t)$ действительно решение (b).

3)
$$
u_f(x,t,\tau)=\frac{1}{4\pi t}\int \limits_{|x-\xi|=t} f(\xi,\tau )dS_\xi
$$
$f(x,t)$ вместе со своими производными по $x_1, x_2, x_3$ включительно непрерывна при $t\ge 0$.
Тогда
$$
h(x,t)=\int \limits_0^t u_f(x,t-\tau,\tau)d\tau \qquad \text{- решение (c)}
$$
Покажем это
$$
h|_{t=0}=0
$$

$$
h_t=u_f(x,0,t)+\int \limits_0^t(u_f)_t(x,t-\tau,\tau)d\tau
$$

$h_t|_{t=0}=u_f(x,0,0)$, но $u_f(x,0,\tau)=0$ тогда

$$
h_t=\int \limits_0^t(u_f)_t(x,t-\tau,\tau)d\tau \Rightarrow h_t|_{t=0}=0
$$

$$
h_{tt}=(\frac{d}{dt}u_f(x,t-\tau,\tau))|_{t=\tau}+
\int \limits_0^t \frac{d^2}{dt^2}(u_f)(x,t-\tau,\tau)d\tau=$$$$
=\frac{d}{dt}u_f(x,0,t)+\int \limits_0^t \De u_f(x,t-\tau,\tau)d\tau
=f(x,t)+\De h
$$


Таким образом получаем, что решение задачи Коши
$$
\begin{cases}
u_{tt}=\De u+f(x,t),\ x \in \R^3,\ t>0\\
u|_{t=0}=u_0(x), \ u_t|_{t=0}=u_1(x), x \in \R^3
\end{cases}
$$
при следующих предположениях:
$u_0(x)\in C^3( \R^3), \ u_1(x)\in C^2(\R^3)$;
функция $f(x,t)$ непрерывна при $t\ge 0$ вместе со всеми своими производными
по $x_1, x_2, x_3$ до второго порядка\\
задается формулой Кирхгофа
$$
u(x,t)=u_{u_1}(x,t)+\frac{d}{dt} u_{u_0}(x,t)+ \int \limits_0^t u_f(x,t-\tau,\tau)d\tau
$$
или
$$
u(x,t)=\frac{1}{4\pi t} \int \limits_{|x-\xi|=t} u_1(\xi) dS_\xi+
\frac{d}{dt}(\frac{1}{4\pi t} \int \limits_{|x-\xi|=t} u_0(\xi) dS_\xi)+
\int \limits_0^t \frac{1}{4\pi (t-\tau)} \int \limits_{|x-\xi|=t-\tau} f(\xi,\tau) dS_\xi d\tau
$$

Исследуем единственность. Для этого достаточно показать, что
$$
\begin{cases}
u_{tt}=\De u\\
u|_{t=0},\  u_t|_{t=0}=0
\end{cases}
\Rightarrow u\equiv 0
\eqno (d)
$$

\subsection{Энергетическое неравенство. Единственность решения}
$|x-x_0|=T-t,\ 0<t<T$ - характеристики.\\
$\Ga_{t_1,t_2} = \left\{(x,t)|,|x-x_0|=T-t,\ t_1 \le t \le t_2 \right\}, \quad 0<t_1\le t_2<T$\\
$\Om_\tau = \left\{(x,t)|,|x-x_0|\le T-\tau,t=\tau \right \}$\\
\textit{Энергия системы} определяется так:
$$
E(t)=\int \limits_{\Om_t}(u_t^2+|\nabla u|^2)dx \quad \text{с точностью до некоторого множителя.}
$$
К слову первое слагаемое в этой сумме имеет физический смысл
кинетической энергии колебаний, а второе - потенциальный.\\
\underline {\textit{Энергетическое неравенство:}}
$$
E(t_1)\le E(t_2),\qquad \fa 0\le t_1 \le t_2 \le T.
$$
Сразу отметим, что в частности $E(t)\le E(0)$,
Соткуда следует единственность решения системы (d),
т,к $E(0)=0 \Rightarrow E(t)=0$\\
\textit{Идея доказательства:}
$$
u_{tt}=\De u \Rightarrow u_tu_{tt}=u_t\De u
$$
$$
dK_{t_1,t_2}=\Ga_{t_1,t_2}\cup \Om_{t_1} \cup \Om_{t_2}
$$
$$
\int\int\int \limits_{K_{t_1,t_2}}u_tu_{tt}=\int\int\int \limits_{K_{t_1,t_2}}u_t \De u,
\quad u_tu_{tt}=\frac12 \frac{d}{dt}(u_t)^2
$$
Теперь надо воспользоваться формулой Гаусса- Остроградского и
свести \\равенство интегралов к виду
$$
\frac12 \int \limits_{dK_{t_1,t_2}}\frac{d}{dt} (u_t)^2 dxdt=
\frac12 \int \limits_{dK_{t_1,t_2}} u_t \De u dxdt
$$
Далее нужно аккуратно расписать данные интегралы и все получится.





%---------------------Лекция 6---------------------------------




\section{Лекция 6}
Зная формулу Кирхгофа решения классической задачи Коши для
уравнения $u_{tt}-\De u=f(x,t)$, мы можем легко получить
решение более общего уравнения $u_{tt}-a^2 \De u=f(x,t)$,
сделав замену $\tau=at$. В этих обозначениях задача Коши примет
вид:
$$
 \left\{
\begin{aligned}
&v_{tt}=\De v +\wt f(x,t)\\
&v|_{\tau=0}=u_0(x)\\
&u_\tau|_{\tau=0}=\frac 1a u_1(x)\\
\end{aligned}
\right.
$$
Здесь $\wt f(x,\tau)=\frac 1 {a^2}f(x,\frac\tau a)$, а
$v(x,\tau)=u(x,\frac\tau a)$. Подставляя в формулу Кирхгофа,
получаем:
$$
u(x,t)=\frac1 {4\pi a^2 t}
\int\limits_{|x-\xi|=at}u_1(\xi)\,dS_\xi+\frac\pd{\pd
t}\Biggl(\frac1 {4\pi a^2
t}\int\limits_{|x-\xi|=at}u_0(\xi)\,dS_\xi\Biggr)+\frac1 {4 \pi
a^2} \int\limits_{|x-\xi|<at}
\frac{f(\xi,t-\frac{|x-\xi|}a)}{|x-\xi|} d\xi
$$
Эта формула тоже называется формулой Кирхгофа.

В дальнейшем всюду будем полагать $a=1$, поскольку переходить к
случаю $a\neq 1$ мы уже умеем, а вычисления упростятся.

\subsection{Единственность классического решения задачи Коши}
Исследование единственности начнем с вывода утверждения, имеющего,
впрочем, и самостоятельную ценность.
\subsubsection{Энергетическое неравенство}
 Итак, пусть $u(x,t)\in C^2(t>0)\,\bigcap\, C^1(t\ge0)$ -
 Классическое решение задачи Коши для однородного волнового
 уравнения.
Рассмотрим конус (называемый характеристическим)
 $$
 K_{{x_0},T}=\left\{\,(x,t)\,\Bigl|\,|x-x_0|<T-t, 0<t<T,\,
 x\in\R^n \right\}
 $$
Обозначим через $\Om_\tau$ поперечное сечение нашего конуса
плоскостью $T=\tau$:
$$
\Om_\tau=K_{{x_0},T}\bigcap \{\,T=\tau\}
$$
А через $\Ga_{t_1 t_2}$ обозначим часть конической поверхности
между двумя сечениями (соответственно плоскостями $\{\,T=t_1\}$ и
$\{\,T=t_2\}$:
$$
\Ga_{t_1 t_2}=\left\{\,(x,t)\,\Bigl|\,|x-x_0|=T-t, \, t_1\le
t\le t_2\, \right\}
$$
Выражение
$$
E(t)=\int\limits_{\Om_t} (u_t^2+|\nabla u|^2)\,dx
$$
называется функционалом энергии.


\begin{theorem}[Энергетическое неравенство]
Для любых $t_1$, $t_2$, таких что $0\le t_1\le t_2 <T$,
выполняется неравенство $E(t_2)\le E(t_1)$.
\end{theorem}

\textbf{Доказательство:}Найдем вектор единичной внешней нормали к
поверхности $\Ga_{t_1 t_2}$: поскольку поверхность задается
уравнением $ F(x,t)=|x-x_0|-T+t=0 $, то компонентами нормали (пока
не единичной) будут частные производные:
$F_t=1,\,F_{x_j}=\frac{x_j-x_{0j}}{|x-x_0|}$.

Получается, что $|\nabla F|^2=2$, и таким образом находим
единичную внешнюю нормаль $\nu=(\nu_1,\ldots,\nu_n,\nu_t)$:
$$
\nu_j=\frac 1{\sqrt 2} \frac{x_j-x_{0j}}{|x-x_0|},\quad
j=1,\ldots,n,\quad \nu_t=\frac 1{\sqrt 2}
$$
Умножим уравнение $u_{tt}-\De u=0$ на $u_t$ и проинтегрируем по
области
$$
K_{t_1 t_2}=\left\{\,(x,t)\,\Bigl|\,|x-x_0|<T-t, \, t_1\le t\le
t_2\, \right\}
$$
Имеем:
$$
0=\int\limits_{K_{t_1 t_2}} (u_{tt}-\De u)u_t\,dt\,dx
$$
А теперь воспользуемся соотношениями:
\begin{align*}
&u_{tt}=\frac 12 \frac \pd{\pd t} (u_t)^2\\
&u_{x_k x_k}u_t=\frac \pd{\pd x_k} (u_t u_{x_k})-u_{t
x_k}u_{x_k}=\frac \pd{\pd x_k} (u_t u_{x_k})-\frac 12
\frac \pd{\pd t} (u_{x_k})^2
\end{align*}
И формулой Стокса (сведя таким образом интегрирование по области к
интегрированию по границе). Заметив, что на верхней крышке области
внешняя нормаль имеет вид $(0,\ldots,0,1)$, а на нижней,
соответственно, $(0,\ldots,0,-1)$, получаем

$$
0=\int\limits_{K_{t_1 t_2}} u_t^2\, dx\, dt -\frac
12\int\limits_{K_{t_1 t_2}} \frac
\pd{\pd t} |\nabla u|^2 \, dx\, dt + \sumkun
\,\int\limits_{K_{t_1 t_2}} \frac
\pd{\pd x_k} (u_t u_{x_k})\, dx\, dt
$$
$$
= \frac 12 \int\limits_{\Om_{t_2}} u_t^2\, dx - \frac 12
\int\limits_{\Om_{t_1}} u_t^2\, dx + \frac 12
\int\limits_{\Ga_{t_1 t_2}} u_t^2 \nu_t\, dS+\frac 12
\int\limits_{\Om_{t_2}} |\nabla u|^2\, dx - \frac 12
\int\limits_{\Om_{t_1}} |\nabla u|^2\, dx + \frac 12
\int\limits_{\Ga_{t_1 t_2}} |\nabla u|^2 \nu_t\, dS
$$
$$
-\sumkun  \,\int\limits_{\pd K_{t_1 t_2}} u_t u_{x_j}
\nu_j \, dS
$$
Заметим, что в последнем слагаемом интегралы по $\Om_{t_1}$ и
$\Om_{t_2}$ равны нулю, так как обращается в ноль
соответствующая компонента вектора нормали. Таким образом, собирая
подобные члены, получаем
$$
0=\frac 12 E(t_2)-\frac 12 E(t_1)+ \frac 1 {2\sqrt 2}
\int\limits_{\Ga_{t_1 t_2}} (u_t^2+|\nabla u|^2 - 2u_t(\nabla
u,\wt{\nu}))\,dS
$$
Здесь за $\wt{\nu}$ мы обозначили вектор  $\sqrt 2\,
(\nu_1,\ldots,\nu_n)$,  длина которого (см. начало лекции) равна
1. Поэтому для скалярного произведения справедлива оценка
$$
(\nabla u,\wt{\nu}) \le |\nabla u| |\wt{\nu}|=|\nabla u|
$$
Теперь мы можем оценить подынтегральное выражение:
$$
(u_t^2+|\nabla u|^2 - 2u_t(\nabla u,\wt{\nu}))\ge
(u_t^2+|\nabla u|^2 - 2|u_t| |\nabla u|)=(|u_t|-|\nabla u|^2)\ge 0
$$
Следовательно,неотрицателен и сам интеграл и верно неравенство
$$
E(t_2)\le E(t_1)
$$
что и требовалось доказать.

\begin{theorem}
Если начальные функции двух задач Коши для одного и того же
уравнения совпадают в шаре $|x-x_0|<t_0,\quad x\in \R^n$,
то решения этих задач Коши совпадают  в характеристическом конусе
с вершиной в $(x_0,t_0)$ и на его границе.
\end{theorem}
\textbf{Доказательство:} Пусть $u(x,t)$ и $v(x,t)$ - решения задач
Коши. Рассмотрим $w(x,t)=u(x,t)-v(x,t)$. Из условия (и линейности
операции дифференцирования) сразу следует, что $w$ является
решением однородного волнового уравнения $ w_{tt}=\De w$, и
кроме того, $w|_{t=0}=w_t|_{t=0}=0$ в шаре $|x-x_0|\le t_0$.

Воспользуемся энергетическим неравенством в характеристическом
конусе с вершиной в $(x_0,t_0)$. Понятно,что
$$
E(0)=\int\limits_{|x-x_0|\le t_0} (u_t^2+|\nabla u|^2)\,dx=0
$$
А из энергетического неравенства следует, что $E(t)\le E(0)=0$
при $t<t_0$. Однако функционал энергии неотрицателен, отсюда
$E(t)\equiv 0$, поэтому равна нулю (так как неотрицательна) и
подынтегральная функция, а значит, и $w(x,t)=0$ в
характеристическом конусе (и по непрерывности - на границе). Но
$w(x,t)$ есть разность решений задач Коши, следовательно решения в
$\ol K_{x_0,t_0}$ совпадают. А это мы и хотели доказать.

\begin{theorem}[Единственности классического решения]
Если задача Коши для волнового уравнения имеет классическое
решение,то это решение единственно.
\end{theorem}
\textbf{Доказательство:} Предположим противное. Пусть есть два
различных решения, тогда существует точка, в которой эти решения
 различны. Возьмем произвольный характеристический конус,
содержащий нашу точку. По предыдущей теореме решения внутри конуса
должны совпадать. Противоречие.


Объединим все вышесказанное
\begin{theorem}[Существования и единственности]
Пусть $u_0(x)\in C^3(\R^3),\,u_1(x)\in C^2(\R^3),$
$D_x^\alpha f\in C(t\ge 0),\,|\alpha|\le 2$.Тогда задача Коши:
$$
 \left\{
\begin{aligned}
&u_{tt}=a^2 \De u +f(x,t),\quad x\in\R^3,\, t>0\\
&u|_{t=0}=u_0(x)\\
&u_t|_{t=0}=u_1(x)\\
\end{aligned}
\right.
$$
\end{theorem}
имеет единственное классическое решение, и это решение задается
формулой Кирхгофа:
$$
u(x,t)=\frac1 {4\pi a^2 t}
\int\limits_{|x-\xi|=at}u_1(\xi)\,dS_\xi+\frac\pd{\pd
t}\Biggl(\frac1 {4\pi a^2
t}\int\limits_{|x-\xi|=at}u_0(\xi)\,dS_\xi\Biggr)+\frac1 {4 \pi
a^2} \int\limits_{|x-\xi|<at}
\frac{f(\xi,t-\frac{|x-\xi|}a)}{|x-\xi|} d\xi
$$

\subsection{Решение задачи Коши для случая $n=2$}
Двумерная задача Коши хорошо вкладывается в трехмерную постановку.
Рассмотрим трехмерную задачу Коши, и положим, что ни данные
задачи, ни решение не зависят от $x_3$.Тем самым задачу мы
произвели редукцию к двумерному случаю - этот прием называется
\emph{методом спуска}. Так что мы можем воспользоваться формулой
Кирхгофа, сделав в ней, впрочем, некоторые упрощения.

Сфера радиуса $R$ с центром в точке $(x_1,x_2,x_3)$ проецируется в
круг того же радиуса с центром в $(x_1,x_2)$. Элемент площади на
сфере $dS$ и элемент площади на круге $d\xi$ связаны соотношением
$d\xi=dS \cos\ga$, здесь $\ga$ - угол между
плоскостью,касательной к сфере, и плоскостью $(x_1,x_2)$. Из
геометрических соображений найдем
$$
\cos\ga=\frac{\sqrt{R^2-|\xi-x|^2}}R
$$
Учтем, что $R=t$ и что в каждую точку круга проецируются 2 точки
сферы - с верхней и с нижней половинки. Поэтому
$$
\frac 1{4\pi t}\int\limits_{|x-\xi|=t}
u_1(\xi_1,\xi_2)\,dS_\xi=\frac {2t}{4 \pi
t}\int\limits_{|x-\xi|\le t}\frac {u_1(\xi_1,\xi_2)\,d\xi_1\,
d\xi_2}{\sqrt{t^2-|x-\xi|^2}}
$$
И преобразуя аналогично всю формулу Кирхгофа, получаем
\emph{формулу Пуассона}:
\begin{multline*}
u(x_1,x_2,t)=\frac 1{2 \pi}\int\limits_{|x-\xi|\le t}\frac
{u_1(\xi_1,\xi_2)\,d\xi_1\, d\xi_2}{\sqrt{t^2-|x-\xi|^2}}+\frac
1{2 \pi} \frac\pd{\pd t}\Bigl(\int\limits_{|x-\xi|\le
t}\frac {u_0(\xi_1,\xi_2)\,d\xi_1\,
d\xi_2}{\sqrt{t^2-|x-\xi|^2}}\Bigl)+\\
\frac 1{2
\pi}\int\limits_0^t\int\limits_{|x-\xi|\le t}
f(\xi_1,\xi_2,\tau)\,d\xi_1\,d\xi_2\,d\tau
\end{multline*}





%------------------------------Лекция 7---------------------------




\section{Лекция 7}
\subsection{Формула Пуассона}
Пусть $n=2,\quad u_1(\xi_1,\xi_2)\in C^2(\R^2),\quad
u_0(\xi_1,\xi_2)\in
C^3(\R^2),\quad\Dc_k^{\alpha}f(x,t)\in C(t\ge 0),
|\alpha|\le 2$\\
Тогда задача Коши с соответствующими условиями имеет единственное
классическое решение, и это решение задается формулой Пуассона.\\
Единственность доказана. Заметим, что правая часть формулы
Пуассона совпадает с правой частью формулы Кирхгофа, при условии,
что функции $u_0, u_1, f$ не зависят от третьего аргумента, мы
получим решение волновой задачи и в $\R^3$.
$$
u_{x_1x_1}+u_{x_2x_2}+u_{x_3x_3}+f(x_1,x_2,t)=u_{tt},\quad
u|_{t=0}=u_0(x_1,x_2), u_t|_{t=0}=u_1(x_1,x_2)
$$
$u_{x_3x_3}=0$. Ч.т.д.

\textit{Задача 1.} Использована ли полностью гладкость начальных
условий в теореме о существовании и единственности?

\textit{Задача 2.} Получить формулу Даламбера методом спуска.
$$
u(x,t)=\frac
{u_0(x+t)+u_0(x-t)}{2}+\frac{1}{2}\intl{x-t}{x+t} u_1(\xi)d\xi +
\frac{1}{2}\intl{0}{t}\intl{x-\tau}{x+\tau}f(\xi,\tau)d\xi d\tau
$$
\subsection{Область зависимости решений от начальных данных}
Решение в точке $(x,t)$ зависит от значений $u_0, u_1$ в области
$|x-\xi|\le t$, а также от значений $f$ на конической поверхности
$\Ga_{0,t}$ характеристического конуса.\\
Рассмотрим случай $n=3$\\
Пусть $f\equiv 0, \supp u_0\subs
\ol{\Om}, \supp u_1\subs \ol{\Om}$.
Фиксируем $x_0$. Обозначим
$$
\rho_0=\minl{\xi\in\ol{\Om}} |x_0-\xi|,\quad
\rho_1=\maxl{\xi\in\ol{\Om}} |x_0-\xi|
$$
В момент времени $\rho-0$ в точку $x_0$ приходит волна
(возмущение).При $\rho_0\le t \le \rho_1$ возмущение все еще
идет, при $t=\rho_1$ волна проходит через $x_0$, при $t>\rho_1$
имеем состояние покоя. Т.е. в момент $\rho_0$ через $x_0$ прошел
передний фронт волны, в момент $\rho_1$ - задний фронт волны.
Фиксируем теперь $t=t_0$. $\Om - \mbox{шар } (0,R_0)$. Передним
фронтом будет сфера $(0,R_0+t)$, задним фронтом при $t>R_0$- сфера
$(0,t-R_0)$.\\
Рассмотрим случай $n=2$.\\
Областью зависимости от $u_0, u_1$ будет круг $|x-\xi|<t$, от $f$
- весь характеристический конус $K_{x,t}$. При $t<\rho_0$ имеем
$u=0$, при $t<\rho_0$ $x_0$ находится в состоянии возмущения.

\subsection{Обобщенные решения волнового уравнения}

\underline{\textit{Определение 1.}} Функция $u(x,t)\in
L_{2,loc}(\R^{n+1})$ называется обобщенным решением
волнового уравнения $u_{tt}-\De u=0$, если существует
последовательность классических решений $u_k(x,t)$ данного
волнового уравнения, $\fa\Om\subs(\R^{n+1})\quad
\|u_k-u\|_{L_2(\Om)}\ra 0, k\ra\bes$.\\
\underline{\textit{Определение 2.}} Функция $u(x,t)\in
L_{2,loc}(\R^{n+1})$ называется обобщенным решением
волнового уравнения $u_{tt}-\nabla u=0$, если $\fa \ph
(x,t)\in C_0^{\bes}(\R^{n+1})$ имеет место интегральное
равенство
$$
\int_{\R^{n+1}}u(x,t)(\ph_{tt}(x,t)-\De
\ph(x,t)) dxdt=0
$$
Эти определения эквивалентны. Докажем $(1)\Rightarrow(2)$. Пусть
$u$ - классическое решение уравнения $u_{tt}-\De u=0$.
$$
\int_{\R^{n+1}}\ph(x,t)(u_{tt}-\De u) dxdt=0
$$
$$
\int_{\Om}u\frac{\pd v}{\pd x_j}dx = -
\int_{\Om}v\frac{\pd u}{\pd x_j}dx +
\int_{\pd\Om}uv\nu_jdS, \quad u,v\in
C^1(\ol{\Om})
$$
Но $\ph$ равна 0 вместе со всеми производными вне $\mbox{supp
} \ph$, который конечен. Тогда
$$
0=\int_{\R^{n+1}}\ph(x,t)(u_{tt}-\De u) dxdt=-
\int_{\R^{n+1}}\ph_tu_tdxdt +
\int_{\R^{n+1}}(\nabla \ph,\nabla u)dxdt=
$$
$$
=\int_{\R^{n+1}}\ph_{tt}udxdt -
\int_{\R^{n+1}}u\De \ph
dxdt=\int_{\R^{n+1}}(\ph_{tt}-\De \ph)u dxdt
$$
Докажем $(2)\Rightarrow(1)$. Рассмотрим функцию $\om (t)\in
C^{\bes}, \om (t)\ge 0, \om (t)=0$ при $t\ge 1$, $\om
(t)=\om (-t)$, $\int_{\R^n}\om (|x|)dx=1$. В
качестве примера можно взять функцию $\om (t), t\in
\R$, равную $Ce^{\frac {1}{1-t^2}}$ при $|t|<1$, и равную
0 в остальных случаях. Пусть $x\in \R^n$, тогда
$$
\om_h (x)=h^{-n}\om \left(\frac{|x|}{h}\right) \mbox{ - ядро
усреднения}
$$
Очевидно, $\om_h (t)\in C^{\bes}, \om_h (t)\ge 0,
\om_h (t)=0$ при $t\ge 1$, $\om_h (t)=\om_h (-t)$,
$\int_{\R^n}\om_h (|x|)dx=1$. Пусть $u(x)\in
L_{2,loc}(\R^{n})$. Тогда
$$
u_h(x)=\int_{\R^{n}}\om_h(x-y)u(y)dy \mbox{ -
усреднение } u \mbox{ с радиусом } h
$$
Получаем $u_h(x)\in C^{\bes}(\R^n)$ и
$u_h^{(n)}=\int_{\R^{n}}\om_h^{(n)}(x-y)u(y)dy$
равномерно сходится.

\underline{\textit{\textbf{Лемма.}}} Пусть $u(x)\in L_2(\Om),
\Om\subs\R^n$ - ограниченная область, $u\equiv 0$ в
$\R^n\setminus\Om$. Тогда\\
1. $\|u_h\|_{L_2}\le \|u\|_{L_2}$.\\
2. $\|u_h - u\|_{L_2}\ra 0$ при $h\ra 0$.

Доказательство.
$$
|u_h(x)|^2=\left|\int_{\R^{n}}\sqrt{\om_h(x-y)}\sqrt{\om_h(x-y)}u(y)dy\right|^2\le
\int_{\R^{n}}\om_h(x-y)dy
\int_{\R^{n}}\om_h(x-y)u^2(y)dy=
$$
$$
=\int_{\R^{n}}\om_h(x-y)u^2(y)dy
$$
Тогда
$$
\int_{\Om}u_h^2(x)dx\le
\int_{\Om}\int_{\R^{n}}\om_h(x-y)u^2(y)dydx=\int_{\R^{n}}\int_{\Om}\om_h(x-y)u^2(y)dxdy\le
\int_{\R^{n}}u^2(y)dy
$$




%------------------------------Лекция 8------------------------------




\section{Лекция 8}
для $u\in L_{2,loc}(\R^n),\ \Om \subs\R^n$ - ограниченное множество
$$
\int\limits_{\Om}(u_h(x))^2 \le
\int\limits_{\R^n} \int\limits_{\Om}\om_h(x-y)u^2(y)dxdy=
\int\limits_{|z|<h}\om_h(z)\int\limits_{\Om}u^2(x+z)dxdz\le
$$
$$
\int\limits_{|z|<h}\om_h(z)dz \int\limits_{\Om_h}u^2(x)dx=
\int\limits_{\Om_h}u^2(x)dx
$$


2. $u\in L_2(\Om). \fa \epsilon >0 \ \exists v \in C_0^\bes (\Om) \ \hn{u-v}_{L_2(\Om)}<\ep$\\
Тогда
$$
\hn{u_h-v_h}_{L_2(\Om)}=\hn{(u-v)_h}_{L_2(\Om)} \le \hn{u-v}_{L_2(\Om)}<\ep
$$
$$
\hn{u_h-v}_{L_2(\Om)}\le \hn{u_h-v_h}_{L_2(\Om)}+ \hn{v_h-v}_{L_2(\Om)}+ \hn{v-u}_{L_2(\Om)} \le
2\ep+ \hn{v_h-v}_{L_2(\Om)}
$$

Достаточно доказать для $C_0^\bes(\ol{\Om})$
$$
|v_h(x)-v(x)|=|\int\limits_{\R^n}\om_h(x-y)(v(y)-v(x))dy|\le \maxl{x-y<h}|v(y)-v(x)|
$$
$$
||v_h-v||_{L_2(\Om)}=\int\limits_{\Om}(v_h-v)^2dx
\le (\maxl{x-y<h}|v(y)-v(x)| \mes(\Om)\ra 0, h \ra 0
$$
- в силу равномерной непрерывности функции.

Осталось все рассуждения собрать вместе и 2 будет доказано,\\
 итак
для $L_{2,loc}$:\\
$
\Om, \Om_1 \quad \ol{\Om}\subs\Om_1,
\quad v\in C_0^\bes(\Om_1),\  ||u-v||<\ep,\ \Om_h
\in \Om_1$ для $h\ll1,\\
||u_h-v_h||_{L_2(\Om)}\le ||u-v||_{L_2(\Om)}<\ep.
$


Теперь из 2 докажем 1:

$u\in L_{2,loc}(\R^{n+1})$
$$
\int\limits_{\R^{n+1}}u(x,t)(\ph_{tt}-\De\ph)dxdt=0 \quad
\fa \ph \in C_0^\bes(\R^{n+1})
$$
$$
u_h(x,t)=\int\limits_{\R^{n+1}}\om_h(x-y,t-\tau)u(y,\tau)dyd\tau \text{- классическое решение.}
$$
$$
\diamond_{x,t}u_h(x,t)=
\int\limits_{\R_{y,\tau}^{n+1}}\diamond_{x,t}(\om_h(x-y,t-\tau))u(y,\tau)dyd\tau
=\int\limits_{\R_{y,\tau}^{n+1}}\diamond_{y,\tau}(\om_h(x-y,t-\tau))u(y,\tau)dyd\tau=0
$$

Отсюда как-то все следует, учитывая что $\diamond_{y,\tau}(\om_h(x-y,t-\tau))\in C_0^\bes(\R^n)$

\underline{\textit{Задача.}}

$$
u_{tt}=u_{xx},\quad f,g\in L_{2,loc}(\R^1)\Rightarrow f(x+t)+g(x-t)
\text{- обобщенное решение волнового уравнения}
$$


\subsection{Обобщенные производные. Пространства Соболева}
Пусть
$$
u\in L_{2,loc}(\Om),\ \Om \ \text{ограничено.}
$$
Тогда $\om_\alpha(x)\in L_{2,loc}(\Om)$ - обобщенная производная
вида $D^\alpha$ от $u$ в смысле Соболева, если выполнено равенство
$$
\int\limits_\Om \om_\alpha(x)g(x)dx=
(-1)^{|\alpha|}\int\limits_\Om u(x)D^\alpha gdx \qquad \fa g\in C_0^\bes(\Om)
$$
Рассмотрим некоторые примеры:\\
1. Пример функции, у которой существуют производные почти всюду,
но не существует производной в смысле Соболева.
$$
\Om=[0,1]\times[0,1],\ u=I_{\left\{y>1/2\right\}}
$$
У этой функции производная почти всюду равна 0, но не
существует производной в смысле Соболева, что непосредственно вытекает
из
$$
0=\int\limits_0^1\int\limits_0^1\frac{\pd u}{\pd y}\ph(x,y)dxdy
=-\int\limits_0^1\int\limits_0^1 u \frac{\pd \ph}{\pd y}dxdy
=-\int\limits_0^1\int\limits_{1/2}^1 \frac{\pd \ph}{\pd y}dxdy
=-\int\limits_0^1\ph(x,1/2)dx
$$
-противоречие.

2. Пример, когда из существования производной порядка 2 в смысле Соболева
не следует существование производной первого порядка в смысле Соболева.
Пусть
$$
u(x,y)=f(x)+g(y),\ \text{где} \ f,g\  \text{не имеют обобщенной производной первого порядка.}
$$
Несложно показать (просто по определннию),
что у функции $u$ существует обобщенная производная второго порядка,
но не существует обобщенной производной первого порядка в смысле Соболева.


%--------------------------Лекция 9-------------------------------

\section{Лекция 9}

\textbf{Задача}: Пусть $u(x,t)$ - обобщенное решение волнового
уравнения
$$
u_{tt}=u_{xx}
$$
Пусть $ \ga : \, t=f(x)$ -гладкая кривая, а области, лежащие по
разные стороны кривой, обозначим $\Om^1$ и $\Om^2$.
Известно, что $u\in C^2(\Om^1)\bigcap C^2(\Om^2)\bigcap
C(\R ^2)$ и что $[u_t]^2+[u_x]^2\neq 0$, где за $[v]$ мы
обозначаем скачок функции $v$ на $ \ga $.

Доказать, что $\ga$\т характеристика.


%\newtheorem*{theorem}{Теорема}

\begin{theorem}
Не может быть двух различных производных вида $D^\alpha u$ у
функции $u\in L_{2\,loc}(\Om)$
\end{theorem}

\textbf{Доказательство:} Предположим противное - пусть существуют
2 обобщенные производные $w_\alpha^1$ и $w_\alpha^2$. Это
означает, что
$$
\int\limits_\Om (w_\alpha^1-w_\alpha^2)f(x)\,dx=0 \quad \fa
f\in C_0^\bes (\Om)
$$

Рассмотрим область $\wt{\Om}\subseteq\Om$, в ней,
очевидно,
$$
\int\limits_{\wt{\Om}} (w_\alpha^1-w_\alpha^2)f(x)\,dx=0
\quad \fa f\in C_0^\bes (\wt{\Om})
$$
Но $w_\alpha^1-w_\alpha^2\in L_2(\wt{\Om})$, а в
$L_2(\wt{\Om})$ множество $C_0^\bes (\wt{\Om})$
всюду плотно (по норме $L_2$), поэтому существует
последовательность $f_k(x)\in C_0^\bes (\wt{\Om})$,
сходящаяся к $(w_\alpha^1-w_\alpha^2)$:
$$
||(w_\alpha^1-w_\alpha^2)-f_k
(x)||_{L_2(\wt{\Om})}\ra 0,\quad k\ra\bes
$$
Вернемся к интегральному тождеству:
$$
\int\limits_{\wt{\Om}} (w_\alpha^1-w_\alpha^2)f_k(x)\,dx=0
$$
И перейдем к пределу при $k\ra\bes$:
$$
\int\limits_{\wt{\Om}} (w_\alpha^1-w_\alpha^2)^2)\,dx=0
$$
А значит, $w_\alpha^1-w_\alpha^2=0$, что и требовалось.

\subsection {Пространство Соболева}
Пусть $\Om$ - некоторая область в $\R^n$.\\
Обозначим
$$
H_{1\,loc}(\Om)=\Bigl\{u\in L_{2\,loc}(\Om)\, |\,\fa
j=1,\ldots , n\quad  \exists u_{x_j}\in L_{2\,loc}(\Om)\Bigl\}
$$
$$
H_{1}(\Om)=\Bigl\{u\in L_{2}(\Om)\, |\,\fa j=1,\ldots ,
n\quad  \exists u_{x_j}\in L_{2}(\Om)\Bigl\}
$$
Где производная понимается обобщенная в смысле Соболева.

 В
пространстве $H_{1}(\Om)$ можно ввести скалярное произведение:
$$
(u,v)_1=\int\limits_\Om (uv+\nabla u \nabla v)\,dx
$$
Которое порождает норму
$$
||u||_1^2=\int\limits_\Om (u^2+|\nabla u|^2)\,dx
$$
Докажем, что относительно введенной нормы пространство
$H_{1}(\Om)$ полно.

\textbf{Теорема 2.}
$$
H_1(\Om)\qquad \text{-  полное (т.е и гильбертово)}
$$
\textit{Доказательство теоремы.}
$$
\left\{u_k(x)\right\},\ u_k\in(\Om),\ ||u_k-u_m||_1\ra 0, \ k,m\ra \bes
$$

$$
||u_k-u_m||_1=\int\limits_\Om(|u_k-u_m|^2)dx+\int\limits_\Om|\nabla(u_k-u_m)|dx \ra 0
\Rightarrow
$$
$$
\Rightarrow
\int\limits_\Om(|u_k-u_m|^2)dx \ra 0
\Rightarrow
||u_k-u_m||_{L_2(\Om)} \ra 0
$$
Т.к пространство $L_2(\Om)$ полное, то
$$
\exists u\in L_2(\Om): ||u-u_k||_{L_2(\Om)}\ra 0, k \ra \bes.
$$
Далее
$$
\int\limits_\Om|\frac{\pd u_k}{\pd x_j}-\frac{\pd u_m}{\pd x_j}|dx\ra  0
\Rightarrow
\left\{\frac{\pd u_k}{\pd x_j}\right\} \ \text{ фундаментальная в} L_2(\Om)
$$
Тогда
$$
\exists \om_j(x)\in L_2(\Om):||\frac{\pd u_k}{\pd x_j}-\om_j||_{L_2(\Om)}
\ra 0,\ k \ra \bes.
$$

Тем самым осталось доказать, что
$$
\int\limits_\Om \om_j(x)\ph(x)dx=
-\int\limits_\Om u \frac{\pd\ph}{\pd x_j}dx, \ \fa\ph\in C_0^\bes
$$
$$
\int\limits_\Om\frac{\pd u_k}{\pd x_j}\ph(x)dx
=-\int\limits_\Om u_k  \frac{\pd\ph}{\pd x_j}   \ra (k\ra\bes)
\int\limits_\Om\om_j(x)\ph(x)dx=-\int\limits_\Om u\frac{\pd\ph}{\pd x_j}dx
$$
Доказательство закончено.

\textit{Пример.}\\
$\Om=\left\{|x|<1\right\},\ x\in \mathbb(R)^n,\  u(x)=|x|^\ep \in H_1(\Om)$(?)\\
Посмотрим, когда $u\in L_2(\Om)$
$$
\int\limits_{|x|<1}|x|^{2\ep}dx=
C\int\limits_0^1 r^{2\ep}r^{n-1}dr=
C\int\limits_0^1 r^{2\ep+n-1}<\bes
\ \text{если } 2\ep+n>0
$$
т.е
$$
\ep>-\frac{n}{2} \Rightarrow u\in L_2(\Om)
$$
$$
x\neq0,:\ \frac{\pd u}{\pd x_j}=\frac{\pd }{\pd x_j}|x|^\ep
=\ep |x|^{\ep-1}\frac{x_j}{x}
$$
Отсюда
$|\nabla u|\in L_2(\Om)$ если

$$
\ep^2\int\limits_{|x|<1} |x|^{2\ep-2}dx<\bes \Rightarrow
2\ep-2+n>0 \Rightarrow \ep > 1-\frac{n}{2}
$$
Убедимся, что $\om_j$ обобщенная производная, т.е
$$
\ep\int\limits_{|x|<1}|x|^{\ep-1} \frac{x_j}{|x|} \ph (x)dx
\ep\int\limits_{|x|<1}|x|^\ep \frac{\pd \ph}{\pd x_j}dx
$$
Вырезаем шарик $O(0,\de).\ \Om_\de=\left\{\de<|x|<1\right\},\ S_\de=
\left\{|x|=\de\right\}$
$$
\int\limits_{\Om_\de}|x|^\ep \frac{\pd \ph}{\pd x_j}dx=
-\ep \int\limits_{\Om_\de}|x|^(\ep-1) \frac{x_j}{|x|}\ph(x)dx+
\int\limits_{S_\de}|x|^\ep \ph \nu_j dS
$$
Пусть $I=\int\limits_{S_\de}|x|^\ep \ph \nu_j dS$, покажем, что $I\ra 0,\
\de \ra 0
$\\
Действительно, т.к $\ep n-1>0$ то
$$
|I|=\de^\ep|\int\limits_{S_\de}\ph \nu dS|\le(|\nu_j|\le1)\le
\maxl{|x|\le 1}|\ph|\de^\ep|S_\de|=K(\ph,n)\de^{\ep+n-1}
\ra 0,\ \de \ra 0.
$$

$u\in
L_2(\Om).\quad$$u_h(x)=\int_{\R^{n}}\om_h(x-y)u(y)dy$.
Пусть существует обобщенная производная вида
$\Dc^{\alpha}u\in L_2(\Om)$, тогда:\\
1.$\Dc^{\alpha}u_h(x)=(\Dc^{\alpha}u)_h(x) \quad
\fa x\in\Om \quad\fa h\le H(x)$.\\
2.$\fa \Om'\subs\subs\Om\quad
\|(\Dc^{\alpha}u)_h-\Dc^{\alpha}u\|_{L_2(\Om')}$
при $h\ra 0$.\\
Доказательство\\
1. $u_h\in C^{\bes}(\Om)$.
$$
\Dc^{\alpha}u_h(x)=\int_{\Om}\Dc_x^{\alpha}\om_h(x-y)u(y)dy=
(-1)^{|\alpha|}\int_{\Om}\Dc_y^{\alpha}\om_h(x-y)u(y)dy=
$$
$$
=\int_{\Om}\om_h(x-y)\Dc_x^{\alpha}u(y)dy=(\Dc^{\alpha}u)_h(x)
$$
2.$\|(\Dc^{\alpha}u)_h-\Dc^{\alpha}u\|_{L_2(\Om)}$
при $h\ra 0$ - уже доказано (на $\Om'$ имеем $\Dc^{\alpha}u_h(x)=(\Dc^{\alpha}u)_h(x)$).\\
Примечание. Выше мы рассматривали $h$:
$h<\rho(x,\pd\Om)$.\\
Ч.т.д.\\
\textit{\underline{Следствие.}} $u\in H_1(\Om)$.\\
$u_h(x)\ra u(x)$ при $h\ra 0$ по норме
$H_1(\Om')$, и это $\fa \Om'\subs\subs\Om$.
Рассмотрим $H_1(\Pi_a), a=(a_1,\ldots,a_n), a_j>0,
\Pi_a=\{x\in\R^n:|x_j|<a_j\}$.\\
Пространство $C^{\bes}(\ol{\Pi}_a)$ всюду плотно в
$H_1(\Pi_a)$.\\
Доказательство.\\
Пусть $u\in H_1(\Pi_a),\quad\Dc^{\alpha}u,\quad 0<|\alpha|<1$.\\
$\fa\ep>0\quad\exists\ph_{\alpha}(x)\in C(\Pi_a):
\|(\Dc^{\alpha}u)_h-\ph_{\alpha}\|_{L_2(\Pi_a)}<\ep$\\
$F_{\si}(x)=u\left(\frac{x}{\si}\right), \si>1.\quad
F_{\si}(x)\in H_1(\Pi_{\si a}).\quad
\ol{\Pi}_a\subs\Pi_{\si a} \quad H_1(\Pi_a)\subs
H_1(\Pi_{\si a})$. Нужно доказать
$\fa\ep>0\quad\exists\si_0>1:
\|\Dc^{\alpha}F_{\si_0}-\ph_{\alpha}\|_{L_2(\Pi_a)}<\ep
C$\\
Если это доказано, то
$(\Dc^{\alpha}F_{\si_0})_h=\Dc^{\alpha}(F_{\si_0})_h\ra\Dc^{\alpha}F_{\si_0}$
в $L_2(\Pi_a)$.\\
Заметим, что $F_{\si_0}\in
H_1(\Pi_{a\si_0}),\Pi_a\subs\Pi_{\si_0
a},(F_{\si_0})_h\in C^{\bes}(\ol{\Pi}_a)$.\\
Кроме того,
$$
\|(\Dc^{\alpha}F_{\si_0})_h-\Dc^{\alpha}u\|_{L_2(\Pi_a)}\le
$$
$$
\le\|(\Dc^{\alpha}F_{\si_0})_h-\Dc^{\alpha}F_{\si_0}\|_{L_2(\Pi_a)}+
\|\Dc^{\alpha}F_{\si_0}-\ph_{\alpha}\|_{L_2(\Pi_a)}+
\|\Dc^{\alpha}u-\ph_{\alpha}\|_{L_2(\Pi_a)},
$$
откуда мы в итоге и получим требуемое утверждение.

%--------------------------Лекция 10-------------------------------

\section{Лекция 10}
\subsection{Строго липшицева область}
$\Om$ - ограниченная область в $\R^n$.
$C_{L,R}=\{x\in\R^n: \suml{j=1}{n-1}x_j^2<R^2,
-2LR<x_n<2LR\}$\\
\underline{\textit{Определение.}} $\Om$ - строго липшицева,
если $\fa x_0\in\pd\Om\quad \exists C=(c_{kl})$ -
ортогональная : $y_k= c_{kl}(x_l-x_{0l})$ $\Rightarrow$
$\pd\Om\cap C_{L,R}(y)$ задается уравнением $y_n=\ph
(y'), y'=(y_1,\ldots,y_{n-1})$ и $\ph$ - липшицева с
постоянной Липшица $L$.\\
Например, если $\pd\Om\in C^1$, то $\Om$ строго
липшицева. Выпуклый многогранник также строго липшицев.\\
Пусть $\Om$ строго липшицева. Тогда
$C^{\bes}(\ol{\Om})$ всюду плотно в $H_1(\Om)$.
Доказываем в частном случае $\Om=\Pi_a$ (см. предыдущую
лекцию).\\
Выберем $\si_0$.
$$
\|\Dc^{\alpha}F_{\si}(x)-\ph_{\alpha}(x)\|_{L_2(\Pi_a)}\le
\|\Dc^{\alpha}F_{\si}(x)-\ph_{\alpha}\left(\frac{x}{\si}\right)\|_{L_2(\Pi_{\si^a})}+
\|\ph_{\alpha}\left(\frac{x}{\si}\right)-\ph_{\alpha}(x)\|_{L_2(\Pi_a)}
$$
$$
\si\ra1+0,\quad \ph_{\alpha}\in
C(\ol{\Pi}_{\alpha})\Rightarrow\|\ph_{\alpha}\left(\frac{x}{\si}\right)-\ph_{\alpha}(x)\|_{L_2(\Pi_a)}\ra
0
$$
Значит, $\fa \ep > 0 \quad \exists \si_1 > 1 :
\fa\si : 1<\si<\si_1\quad
\|\ph_{\alpha}\left(\frac{x}{\si}\right)-\ph_{\alpha}(x)\|_{L_2(\Pi_a)}<\ep$.\\
Но $\Dc^{\alpha}F_{\si}(x) = \frac
{1}{\si^{|\alpha|}}\Dc^{\alpha}u\left(\frac{x}{\si}\right)$.
Тогда
$$
\|\Dc^{\alpha}F_{\si}(x)-\ph_{\alpha}\left(\frac{x}{\si}\right)\|_{L_2(\Pi_{\si^a})}=
\|\frac
{1}{\si^{|\alpha|}}\Dc^{\alpha}u\left(\frac{x}{\si}\right)-\ph_{\alpha}\left(\frac{x}{\si}\right)\|_{L_2(\Pi_{\si^a})}\le
$$
$$
\le\|\Dc^{\alpha}u\left(\frac{x}{\si}\right)-\ph_{\alpha}\left(\frac{x}{\si}\right)\|_{L_2(\Pi_{\si^a})}+
\left(1-\frac{1}{\si^{\alpha}}\right)\|\Dc^{\alpha}u\left(\frac{x}{\si}\right)\|_{L_2(\Pi_{\si^a})}\le
$$
$$
\le\si^{n/2}\|\Dc^{\alpha}u(x)-\ph_{\alpha}(x)\|_{L_2(\Pi_a)}+\left(1-\frac{1}{\si^{\alpha}}\right)\si^{n/2}\|\Dc^{\alpha}u\|
_{L_2(\Pi_a)}
$$
, что стремится к 0 при $\si\ra 1+0$. Множители вида
$\si^{n/2}$ возникают за счет замены переменных
$\frac{x}{\si}=y$ в выражении для нормы.\\
Ч.т.д.\\
\underline{\textit{Упражнение.}} Доказать,что если $\Om=Q_R^0$,
то $C^{\bes}(\ol{\Om})$ всюду плотно в $H_1(\Om)$.\\
$\Pi_l=\{x: 0<x_j<l_j, j=1,\ldots,n\}$.
\subsection{Неравенство Пуанкаре}
$$
u\in H_1(\Pi_l)\quad\Rightarrow\quad \int_{\Pi_l}u^2dx\le
\frac{1}{|\Pi_l|}\left(\int_{\Pi_l}udx\right)^2+\frac{n}{2}\suml{j=1}{n}l_j^2\int_{\Pi_l}u_{x_j}^2dx
$$
Доказательство. Сделаем замену $y_j=\frac{x_j}{l_j}$, тогда
$dx=|\Pi_l|dy$ и $u_{x_j}=\frac{1}{l_j}u_{y_j}$. Получаем, что
достаточно доказать следующее неравенство:
$$
\int_{\Pi_1}u^2dy\le\left(\int_{\Pi_1}udy\right)^2+\frac{n}{2}\int_{\Pi_1}|\nabla_yu|^2dy
$$
В частности, если $\int_{\Pi_1}udy=0$, то $\|u\|_{L_2(\Pi_1)}\le
\sqrt{\frac{n}{2}}\|\nabla u\|_{L_2(\Pi_1)}$ Докажем теперь наше
неравенство.\\
$C^{\bes}(\ol{\Pi}_1)$ всюду плотно в $H_1(\Pi_1)$. Рассмотрим

$$P_0=(y_1^0,\ldots,y_n^0), \quad P_1=(y_1,\ldots,y_n) \quad P_0\in\Pi_1, P_1\in\Pi_1$$
$$A_1=(y_1,y_2^0,\ldots,y_n^0), \quad A_2=(y_1,y_2,\ldots,y_n^0), \quad\ldots ,\quad A_n=(y_1,\ldots,y_n)=P_1$$

$$
u(y_1,\ldots,y_n)-u(y_1^0,\ldots,y_n^0)=\int_{y_1^0}^{y_1}u_{x_1}(x_1,y_2^0,\ldots,y_n^0)dx_1+
\int_{y_2^0}^{y_2}u_{x_1}(y_1,x_2,\ldots,y_n^0)dx_2+\ldots+
$$
$$
+\int_{y_n^0}^{y_n}u_{x_n}(y_1,y_2,\ldots,x_n)dx_n
$$
Воспользуемся тем, что $\left(\sumkun  a_k\right)^2\le
n\sumkun a_k^2$, тогда
$$
(u(P_1)-u(P_0))^2\le
n\left(\left(\int_{y_1^0}^{y_1}u_{x_1}(x_1,y_2^0,\ldots,y_n^0)dx_1\right)^2+\ldots+\left(\int_{y_n^0}^{y_n}u_{x_n}(y_1,y_2,\ldots,x_n)dx_n\right)^2\right)\le
$$
$$
\le
n\left(\int_0^1u_{x_1}^2dx_1+\ldots+\int_0^1u_{x_n}^2dx_n\right)
$$
Интегрируя обе части неравенства по $P_0$ и $P_1$, получим
$$
2\int_{\Pi_1}u^2(y)dy-2\left(\int_{\Pi_1}u(y)dy\right)^2\le
n\int_{\Pi_1}|\nabla_yu|dy
$$
Ч.т.д.\\
$\Om$ - ограниченная область. $H_1^0(\Om)$ - замыкание
$C_0^{\bes}(\Om)$ по норме $H_1(\Om)$.\\
$H_1^0(\Om)\subs H_1(\Om)$, $H_1^0(\Om)$ - гильбертово.
$\Om'$ - ограниченная область, $\Om\subs \Om'$, $u\in
H_1(\Om)$, тогда $\wt{u}(x)$ равна $u(x)$ при $x\in\Om$,
и равна 0 при $x\in \Om'\wo\Om$. $\wt{u}\in
H_1^0(\Om)$.

\subsection{Теорема Реллиха}
Ограниченное множество в $H_1^0(\Om)$ компактно в
$L_2(\Om)$.\\
$u_m\in H_1^0(\Om),\quad \|u_m\|_{H_1^0(\Om)}\le
C=\const\quad \Rightarrow $ существует сходящаяся
подпоследовательность в норме $L_2(\Om)$.\\
Доказательство.\\
Можно считать, что $\Om\subs\Pi_l$ (с точностью до сдвига).
Тогда продолжаем $u$ тождественным 0 на
$\Pi_l\wo\ol{\Om}$. Разобьем $\Pi_l$ на $N^n$
элементарных параллелепипедов $\om_i$. $u\in H_1(\om_i)$,
тогда
$$
\int_{\om_i}u^2dx\le
\frac{1}{|\om_i|}\left(\int_{\om_i}udx\right)^2+\frac{n}{2N^2}\suml{j=1}{n}l_j^2\int_{\om_i}u_{x_j}^2dx
$$
$$
\int_{\Pi_l}u^2dx\le
\suml{i=1}{N^n}\frac{1}{|\om_i|}\left(\int_{\om_i}udx\right)^2+\frac{n}{2N^2}\suml{j=1}{n}l_j^2\int_{\Pi_l}u_{x_j}^2dx
$$
$$
\int_{\Pi_l}(u_m-u_s)^2dx\le
\suml{i=1}{N^n}\frac{1}{|\om_i|}\left(\int_{\om_i}(u_m-u_s)dx\right)^2+\frac{n}{2N^2}\suml{j=1}{n}l_j^2\int_{\Pi_l}(u_m-u_s)_{x_j}^2dx
$$
$\|u_m\|_{L_2(\Pi_l)}\le C$. Существует подпоследовательность
$u_m$, слабо сходящаяся к $u$ в $L_2(\Pi_l)$. Т.е. $\fa v\in
L_2(\Pi_l)$ имеем $\int_{\Pi_l}(u_m-u_s)vdx\ra 0$ при
$m,s\ra\bes$. Возьмем $v=\chi\om_j$, получим
$\int_{\om_i}(u_m-u_s)dx\ra 0$ при
$m,s\ra\bes$.\\
$\|u_m\|_{L_2(\Pi_l)}\le C$. Выбираем $N_0$ такое, чтобы второе
слагаемое в нашем неравенстве было меньше $\ep$. Выбираем
$m,s$ такие, чтобы выполнялось
$\int_{\Pi_l}(u_m-u_s)vdx<\sqrt{\frac{\ep}{N_0^n}|\om_j|}$.
Тогда $\int_{\Pi_l}(u_m-u_s)^2vdx<2\ep \quad
\Rightarrow\quad \{u_m\}$ фундаментальна в $L_2(\Pi_l)$, откуда и
получаем требуемую сходимость.\\
Ч.т.д.


%-------------------------Лекция 11-----------------------------

\section{Лекция 11}

\underline{Задача}\\
Пусть $\Om_\de$ - область <<вида цоколя>>.
$$
\Om_\de \left\{x\in \R^n|-\de<x_n<f(x'), x'=(x_1,\dots,x_{n-1})\in D \right\}
,\ \de>0,\ f\in C(\ol{D}),\ f(x')\ge 0.
$$

Тогда
$$
\left\{
\exists \ \text{обобщенной производной вида} \frac{\pd u}{\pd x_n} \in L_2(\Om_\de)
\right\}=
\ol{C^\bes(\ol{\Om_\de})}
$$

\subsection{Неравенство Фридрихса}

$$
\fa u\in H_0^1(\Om),\ \Om\subs\R^n \text{огр.}\
||u||_{L_2(\Om)}\le C(\Om) ||\nabla u||_{L_2(\Om)} \eqno(11.1)
$$
\textit{Доказательство.}\\
Можно считать, что\\
$
\ol{\Om}\subs \Pi_l:=
\left\{x\in \R^n:0\le x_j\le l_j\right\},\ l_1\le l_j,\ j=\ol{2,n}
$
Тогда (11.1) достаточно доказать $\fa u\in C_0^\bes$.\\
Итак,
$$
u\in C_0^\bes,\ x'=(x_2,\dots,x_n)
$$
$$
u(x_1,x')-u(0,x')=\int\limits_0^{x_1}\frac{\pd }{\pd y_1} u(y_1,x')dy_1
$$
$$
\int\limits_{\Pi_l}u^2(x)dx=
\int\limits_{\Pi_l}(\int\limits_0^{x_1}\frac{\pd }{\pd y_1} u(y_1,x')dy_1)^2dx
$$

По неравенству Коши-Буняковского
$$
\int\limits_{\Pi_l}u^2(x)dx\le
\int\limits_{\Pi_l}x_1\int\limits_0^{l_1}\left|\frac{\pd u}{\pd x_1}\right|^2dx_1dx=
\int\limits_0^{l_1}x_1\int\limits_{\Pi_l}\left|\frac{\pd u}{\pd x_1}\right|^2dxdx_1=
\frac{l_1^2}{2}\int\limits_{\Pi_l}\left|u_{x_1}\right|^2dx=
\frac{l_1^2}{2}\int\limits_{\Om}\left|u_{x_1}\right|^2dx
$$
$$
C(\Om)=\frac{l_1^2}{2}
$$
Доказательство закончено.


На $H_0^1$ можно ввести норму
$$
||u||_{H_1(\Om)}^2=
\int\limits_\Om |\nabla u|^2dx
$$
Она будет эквивалентна норме $H_1(\Om)$.\\
$H_1^0(\Om)\neq H_1(\Om$ т.к $u\equiv 1$
не удовлетворяет неравенству Фридрихса.\\
\textit{Формула интегрирования по частям для} $u\in H_0^1(\Om),\ v\in H_1(\Om)$
$$
\int\limits_{\Om}u_{x_j}vdx=
-\int\limits_{\Om}uv_{x_j}dx
$$
Доказательство очень простое, достаточно вспомнить, что
$$
\exists u_m\in C_0^\bes(\Om): ||u-u_m||_{H_1}\ra 0
$$
и перейти к пределу в равенстве
$$
\int\limits_{\Om}u_{mx_j}vdx=
-\int\limits_{\Om}u_mv_{x_j}dx
$$

\subsection{След функции}

К понятию следа функции мы придем не сразу.
Пусть $u\in H_1(\Om)$, где $\Om$, ограниченное область в $\R^n$,
причем $\pd \Om\in C^1,\ \pd\Om = \bigcup\limits_{j=1}^N S_j$\\
$S_1: x_n=\ph\ (x'),\  x'=(x,\dots,x_{n-1}),\ \ph(x')\in C^1(D)$
Пусть теперь
$$
S_1\subs \left\{x_n=0\right\}
$$
$$
\exists\de >0:Q_\de=\left\{x|0<x_n<\de,\ x'\in S_1\right\} \subs\Om
$$
$Q_\de$ - область <<вида цоколя>>, $C^\bes(\ol{Q})_\de$ всюду плотно в
$H_1(Q_\de)\ u\in C^\bes(\ol{Q}_delta)$.
$$
u(x',x_n)-u(x',0)=\int\limits_0^{x_n}\frac{\pd u}{\pd y_n}(x',y_n) dy_n
$$
$$
u(x',0)=u(x',x_n)-\int\limits_0^{x_n}\frac{\pd u}{\pd y_n}(x',y_n) dy_n
$$
$$
u^2(x',0)\le 2u^2(x',x_n)+2(\int\limits_0^{x_n}\frac{\pd u}{\pd y_n}(x',y_n) dy_n)^2
$$
Интегрируем это выражение по $Q_\de$:
$$
\de\int\limits_{S_1}u^2(x',0)dx'\le
2\int\limits_{Q_\de}u^2(x',x_n)dx+2\int\limits_{Q_\de}(\int\limits_0^{x_n}\frac{\pd u}{\pd y_n}(x',y_n) dy_n)^2dx\le
$$
$$
\le 2\int\limits_{Q_\de}u^2(x',x_n)dx+\de^2\int\limits_{Q_\de}|u_{x_n}|^2dx
$$
Т.е получаем, что
$$
\int\limits_{S_1}u^2(x',0)dx'\le
\frac{2}{\de}\int\limits_{Q_\de}u^2(x',x_n)dx+\de\int\limits_{Q_\de}|u_{x_n}|^2dx
$$
Если $S$ не плоская, то распрямляем диффеоморфизмом.
$$
\int\limits_{S_1}u^2dS\le C_1(\int\limits_\Om u^2dx+\int\limits_\Om |\nabla u|dx)
$$
Так как $u\in H_1(\Om)$, то $u\in H_1(Q_\de),\ \exists u_m \ra u$, пусть $u=u_m-u_k$.\\
Подставляя в предыдущее выражение, получим
$$
\int\limits_{S_1}(u_m-u_k)^2dS\le
\frac{2}{\de}\int\limits_\Om (u_m-u_k)^2dx+\de\int\limits_\Om |\nabla (u_m-u_k)|^2dx
\ra 0,\ m,k \ra \bes.
$$
$\Rightarrow \left\{u_m\right\}$ фундаментальна в
$L_2(S_1)$, т.е $\exists u|_{S_1}= \liml{||.||_{L_2(S_1)}}u_m$

\textit{\underline{Определение}} $u|_{S_1}$ называется следом.


Оператор следа $T:H_1(\Om)\ra L_2(\pd \Om)(L_2(S_1)),\  Tu=u|_{S_1}$.

Покажем, что такое определение корректно.
Пусть
$$
\left\{u_m\right\},\ \left\{u'_k\right\}\ra u\  \text{по норме}\  H_1
\qquad u|_{S_1},u'|_{S_1}\in L_2(S_1)
$$
$$
||u|_{S_1}-u'|_{S_1}||_{L_2(S_1)}\le
||u|_{S_1}-u_m|_{S_1}||+
||u_m|_{S_1}-u'_k|_{S_1}||+
||u'|_{S_1}-u'_k|_{S_1}||
$$
$$
\int\limits_{S_1}(u|_{S_1}-u_m)^2dx'\le
\frac{2}{\de}\int\limits_\Om (u-u_m)^2dx+\de\int\limits_\Om |\nabla (u-u_m)|^2dx
$$
$$
||u|_{S_1}-u_m|_{S_1}||_{L_2(S_1)}\le C(\de)
||u-u_m||_{H_1(Q_\de)}
$$
$$
||u|_{S_1}-u'_m|_{S_1}||_{L_2(S_1)}\le
C(\de)
(
||u-u_m||_{H_1(Q_\de)}
||u_m-u'_k||_{H_1(Q_\de)}
||u'_k-u||_{H_1(Q_\de)}
)
\ra 0,\ k,m \ra \bes
$$
$$
\Rightarrow
||u|_{S_1}-u'|_{S_1}||_{L_2(S_1)}=0
$$
Что и требовалось.\\
\textit{Задача.} $u\in C(\ol{\Om})\Rightarrow Tu=u|_{\pd \Om}$\\
\textit{Упражнение.} $u\in C(\ol{\Om})\Rightarrow u_h$ сходится равномерно к $u(x)$
на $\Om'\subs\subs \Om$ при $h\ra 0$\\
\textit{Следствие.} $u\in H_1^0(\Om) \Rightarrow Tu|_{\pd\Om}=0$\\
Доказательство следствия.
$$
u_m\in C_0^\bes(\Om)
||u-u_m||_{H_1(\Om)}\ra 0, m\ra\bes
$$
$$
||u|_{\pd\Om}-u_m|_{\pd\Om}||\le C
||u-u_m||_{H_1(\Om)}\Rightarrow
u|_{\pd\Om}=0
$$

\subsection{Метод Фурье (метод разделения переменных)}
Пусть $\Om\subs\R^n$ ограничено, $\pd\Om\in C^1$\\
$Q_T=\left\{(x,t)|x\in \Om,0<t<T\right\}$\\
$\Ga_T=\pd\Om\times[0,T]$

$$
u_{tt}=\De u+f(x,t),\ (x,t)\in Q_T \eqno(1)
$$
$$
(3) u|_{\Ga_T}=0,\ u|_{t=o}=u_0(x),\ u_t|_{t=0}=u_1(x),\ x\in\Om (2)
$$
$$
u|_{\Ga_T}=0,\ \frac{\pd u}{\pd \nu}+\alpha(x,t)u=\beta(x,t)u,\ \nu\
\text{единичная нормаль к }\Ga_T
$$
$\alpha\equiv 0$ - вторая нормаль.\\
$\alpha\neq 0$ - третья нормаль.


%---------------Лекция 12-----------------------

\section{Лекция 12}
\subsection{Обобщенное решение первой начально-краевой задачи}
$\Om_{\tau}=Q_T\cap\{t=\tau\}$\\
Пусть $u(x,t)$ - решение задачи (1)-(3), $v(x,t)\in
C^1(\ol{Q}_T), \quad v|_{t=T}=v|_{\Ga_T}=0$.
$$
\int_0^T\int_{\Om}(u_{tt}-\De
u)vdxdt=\int_0^T\int_{\Om}f(x,t)v(x,t)dxdt
$$
$$
\int_0^T\int_{\Om}(-u_tv_t+\nabla u\nabla v)dxdt+\int_{\pd
Q_T}u_tv\nu_tds-\suml{j=1}{n} \int_{\pd
Q_T}u_{x_j}v\nu_{x_j}ds=\int_{Q_T}f(x,t)v(x,t)dxdt
$$
$$
\nu=(\nu_{x_1},\ldots,\nu_{x_n},\nu_t)
$$
На верхнем основании $\nu=(0,\ldots,0,1)$, на нижнем основании
$\nu=(0,\ldots,0,-1)$
$$
\int_{\pd
Q_T}u_tv\nu_tds=-\int_{\Om_0}u_t(x,0)v(x,0)dx=-\int_{\Om}\psi(x)v(x,0)dx
$$
$$
\int_{\pd Q_T}u_{x_j}v\nu_{x_j}ds=0
$$
$$
\int_0^T\int_{\Om}(-u_tv_t+\nabla u\nabla
v)vdxdt=\int_{\Om}\psi(x)v(x,0)dx+\int_{Q_T}f(x,t)v(x,t)dxdt\quad\quad(4)
$$
\underline{\textit{Определение.}} Функция $u(x,t)\in H_1(Q_T)$
называется обобщенным решением первой начально-краевой задачи
(1)-(3), если $u(x,t)=0$ на $\Ga_T$, $u(x,0)=u_0(x)$ в смысле
следа функции и $\fa v\in H_1(Q_T): v|_{\Ga_T}=0,
v|_{t=T,x\in\Om}$ выполнено интегральное тождество (4).

\subsection{Единственность решения смешанной задачи (1)-(3)}
Если задача (1)-(3) имеет обобщенное решение, то оно
единственно.\\
Доказательство.\\
Пусть $u_1(x,t),u_2(x,t)$ - два
решения.$u(x,t):=u_1(x,t)-u_2(x,t)$.
$$
\int_0^T\int_{\Om}(-u_tv_t+\nabla u\nabla v)vdxdt=0.
$$
Возьмем $v(x,t)$, определенную следующим образом:
$$
\left\{
\begin{aligned}
\int_t^{\tau}u(x,s)ds,\quad 0\le t\le\tau\\
0,\quad \tau\le t\le T\\
\end{aligned}
\right.
$$
Тогда $v_t(x,t)$ будет выглядеть следующим образом:
$$
\left\{
\begin{aligned}
-u(x,t),\quad 0\le t\le\tau\\
0,\quad \tau\le t\le T\\
\end{aligned}
\right.
$$
а $v_{x_j}(x,t)$ - так:
$$
\left\{
\begin{aligned}
\int_t^{\tau}u_{x_j}(x,s)ds,\quad 0\le t\le\tau\\
0,\quad \tau\le t\le T\\
\end{aligned}
\right.
$$
Имеем
$$
\int_0^T\int_{\Om}u_tudxdt + \int_0^{\tau}\int_{\Om}\nabla
u(x,t)\int_t^{\tau}\nabla u(x,s)dsdxdt=0
$$
Обозначим $\eta(x,t)=\int_t^{\tau}\nabla u(x,s)ds$ и воспользуемся
тем, что $\nabla u=-\eta_t$ и $u_tu=\frac{1}{2}(u^2)_t$, тогда
$$
\frac{1}{2}\int_{\Om}u^2(x,\tau)dx + \frac{1}{2}\int_{\Om}(\eta)^2(x,0)dx=0,
$$
откуда получаем, что $u(x,\tau)=0$ п.в.\\
Ч.т.д.\\
\textit{Следствие.} Если задача (1)-(3) имеет классическое
решение, то оно единственно.

\subsection{Задача Дирихле}
$\De u=\la u,\quad x\in\Om;\quad u|_{\pd\Om}=0$
- спектральная задача Дирихле для оператора Лапласа.\\
\underline{\textit{Определение.}} Функция $u$, не являющаяся
тождественно равной 0 в $\Om$, называется собственной функцией
задачи Дирихле для оператора Лапласа, если $\exists\la:$
$\fa v\in H_1^0(\Om)$ выполняется интегральное тождество
$$
\int_{\Om}\nabla u\nabla vdx=-\la\int_{\Om}uvdx
$$
Если $u(x,t)$ - собственная функция, отвечающая собственному
значению $\la$, то $Cu(x,t)$ тоже будет собственной функцией c
тем же собственным значением. Поэтому мы можем считать,что
$\|u\|_{L_2(\Om)}=1$.
$$
(u,v)_{H_1^0(\Om)}=\int_{\Om}\nabla u\nabla vdx
$$
\textbf{\underline{\textit{Теорема.}}} Существует линейный
ограниченный оператор A:$L_2(\Om)\ra H_1^0(\Om) :$\\
1)$\fa u\in L_2(\Om)\fa v\in H_1^0(\Om)\quad
(u,v)_{L_2(\Om)}=(Au,v)_{H_1^0(\Om)}$\\
2)$\exists A^{-1}$\\
3)$A|_{H_1^0(\Om)}$ - самосопряженный, положительный и вполне
непрерывный (компактный).\\
(напомним, что $A:H\ra H$ вполне непрерывен, если любое
ограниченное множество он переводит в компактное).\\
Доказательство.\\
Фиксируем $u\in L_2(\Om)$. $l_u: H_1^0(\Om)\ra
\R$, $l_u(v)=(u,v)_{H_1^0(\Om)}$.\\
Оператор $l_u$ непрерывен. В самом деле,
$|l_u(v)\le\|u\|_{L_2(\Om)}\|v\|_{L_2(\Om)}\le\|u\|_{L_2(\Om)}C\|v\|_{H_1^0(\Om)}$,
т.е. $\|l_u\|\le C\|u\|_{L_2(\Om)}$\\
По теореме Рисса $\exists ! V\in H_1^0(\Om):\quad \fa v\in
H_1^0(\Om) l_u(v)=(V,v)_{H_1^0(\Om)}$. Положим $Au:=v$.
Оператор А линеен и ограничен:
$$
\|Au\|^2_{H_1^0(\Om)}=(Au,Au)_{H_1^0(\Om)}=(u,Au)_{L_2(\Om)}\le\|u\|_{L_2(\Om)}\|Au\|_{L_2(\Om)}\le
\|u\|_{L_2(\Om)}C\|Au\|_{H_1^0(\Om)}
$$
Значит, $\|A\|\le C$.\\
Покажем $\mbox{Ker A} = \{0\}$. Действительно, $Au=0\Rightarrow
\fa v\in H_1^0(\Om) \quad (u,v)_{L_2(\Om)}=0\Rightarrow
u\perp H_1^0(\Om)$.\\
$C_0^{\bes}(\Om)$ всюду плотно в
$L_2(\Om)\quad\Rightarrow\quad H_1^0(\Om)$ всюду плотно в
$L_2(\Om)\quad\Rightarrow\quad u=0$.\\
Из того,что $\fa u\in H_1^0(\Om) \quad
(Au,u)_{H_1^0(\Om)}\ge 0$ и
$(Au,u)_{H_1^0(\Om)}=0\Leftrightarrow u=0$ следует
положительность оператора А.\\
Покажем самосопряженность. $u,v\in H_1^0(\Om)$:
$$
(Au,v)_{H_1^0(\Om)}=(u,v)_{L_2(\Om)}=(v,u)_{L_2(\Om)}=(Av,u)_{H_1^0(\Om)}=(u,Av)_{H_1^0(\Om)}
$$
Покажем компактность. $\{u_k\},\quad u_k\in H_1^0(\Om),\quad
\|u_k\|_{H_1^0(\Om)}\le M$.\\
По теореме Реллиха существует подпоследовательность $\{u_{i_k}\}:
\|u_{i_m}-u_{i_n}\|_{L_2(\Om)}\ra 0$ при
$m,n\ra\bes$. Тогда $\{Au_{i_k}\}$ фундаментальна в
$H_1^0(\Om)$, т.к.
$$
\|Au_{i_m}-Au_{i_n}\|_{H_1^0(\Om)}=\|A(u_{i_m}-u_{i_n})\|_{H_1^0(\Om)}\le
C\|u_{i_m}-u_{i_n}\|_{L_2(\Om)}\ra 0 \mbox{ при }
m,n\ra\bes.
$$
Ч.т.д.\\
Если $u$ - собственная функция с собственным значением $\la$,
то $$(u,v)_{H_1^0(\Om)}=-\la (u,v)_{L_2(\Om)}=-\la
(Au,v)_{H_1^0(\Om)}$$
,т.е. $u=-\la Au$.





%--------------------Лекция 13-------------------------






\section[Лекция 13.]{Лекция 13.}
\subsection {Базис в пространстве $H_1^0(\Om)$}
Итак, пусть $u$ -собственная функция задачи Дирихле, отвечающая
собственному значению $\la$. Тогда имеем
$$
(u,v)_{H_1^0(\Om)}=-\la(u,v)_{L_2(\Om)}=-\la(Au,v)_{H_1^0(\Om)}\quad
\fa u\in H_1^0(\Om)
$$
И, значит, $u=-\la Au$, то есть при $\la\neq 0$ $u$-
собственная функция оператора $A$, отвечающая собственному
значению $-\frac 1 \la$.

Обозначим $\mu = - \la$, тогда
$$
u=\mu Au,\quad A:H_1^0(\Om)\ra H_1^0(\Om)
$$

Тогда $\mu$ называется характеристическим числом оператора $A$.
Вспоминая свойства оператора $A$, получаем, что множество
характеристических чисел - счетно (из компактности) с предельной
точкой $\bes$, причем характеристические числа- вещественны (из
самосопряженности) и положительны (из положительности оператора).
Кроме того, собственные функции образуют ортогональный базис в
$H_1^0(\Om)$.

Переводя эти утверждения на язык $\la$, получаем, что задача
(1) имеет счетное множество собственных значений конечной
кратности, вещественных, отрицательных и имеющих предельную точку
в $-\bes$.Собственные функции $v_1(x),v_2(x),\ldots$ образуют
ортогональный базис в $H_1^0(\Om)$.

Мы нормировали собственные функции условием
$||u||_{L_2(\Om)}=1$, поэтому
$$
(v_k,v_k)_{H_1^0(\Om)}=||v_k||_{H_1^0(\Om)}=-\la_k
||v_k||_{L_2(\Om)}=-\la_k
$$
Поэтому функции
$$
\frac {v_1(x)}{\sqrt{-\la_1}}, \frac
{v_2(x)}{\sqrt{-\la_2}} ,\ldots
$$
образуют ортонормированный базис в $H_1^0(\Om)$. Заметим, что
функции $v_1,v_2,\ldots$ образуют ортонормированный базис в
$L_2(\Om)$. Следовательно, любую функцию $g\in H_1^0(\Om)$
можно представить в виде ряда:
$$
g(x)=\sumkui \wt{g_k} \frac
{v_k(x)}{\sqrt{-\la_k}}
$$
И в виде ряда
$$
g(x)=\sumkui  g_k v_k
$$
Где $g_k=(g,v_k)_{L_2(\Om)}$ и $g_k = \wt{g_k} \frac
1{\sqrt{-\la_k}}$

В силу равенства Парсеваля,
$$
||g(x)||^2_{H_1^0(\Om)}=\sumkui  |\wt{g_k}|^2 =
\sumkui  |\la_k|\, g_k^2
$$
Следовательно,$ \fa g\in H_1^0(\Om)\,: \, \sumkui
|\la_k|\, g_k^2 < \bes$.


\subsection {Обобщенное решение первой начально-краевой задачи}
Вернемся к нашей задаче (3):
$$
 \left\{
\begin{aligned}
&u_{tt}-\De u =f(x,t),\quad (x,t)\in Q_T=\Om \times (0,T)\\
&u|_{\Ga_T=\pd Q_T}= 0\\
&u|_{t=0}=\ph(x)\\
&u_t|_{t=0}=\psi(x)\\
\end{aligned}
\right.
$$

\textbf{Определение:} функция $u\in H_1^0(Q_T),\,u|_{\Ga_T}=
0,\, u|_{t=0}=\ph(x) $ называется обобщенным решение задачи
(3), если она удовлетворяет интегральному тождеству
$$
\int \limits_{Q_T)} (-u_t h_t+\nabla u \nabla h)\, dx\, dt = \int
\limits_{Q_T)} f(x,t) h(x,t)\,dx\,dt+ \int \limits_\Om \psi(x)
h(x,0)\,dx \quad \fa h\in H_1(Q_T):\,h|_{\Ga_T}= 0,\,
h|_{t=T}=0
$$
Далее предполагаем, $f\in L_2(Q_T), f\in H_1^0(\Om), \psi \in
L_2(\Om) $. Так как $f(x,t) \in L_2(Q_T)$,то $ f(x,t) \in
L_2(\Om) $ для почти всех $t \in [0,T]$. Разложим ее в ряд
Фурье:
$$
f(x,t)=\sumkui  f_k(t) v_k(x),\quad f_k (t)= (f(x,t) v_k
(x)
$$
Напишем неравенство
$$
f_k^2(t)=\Bigl( \int\limits_\Om f(x,t) v_k(x)\, dx \Bigr)^2\le
\int\limits_\Om f^2(x,t) \, dx\, ||v_k||^2_{L_2(\Om)}=
\int\limits_\Om f^2(x,t) \, dx
$$
И проинтегрируем его по отрезку $[0,T]$:
$$
\int\limits_0^T f_k^2(t) \, dt\le ||f||^2_{L_2(Q_T)}
$$
Отсюда, в частности, $f_k(t)\in L_2[0,T] $.

\noindent Рассмотрим вспомогательную задачу: Разложим функции
$\ph$ и $\psi$ в ряд Фурье
$$
\ph(x)=\sumkui  \ph_k \,v_k(x),\quad \ph_k
(t)= (\ph,v_k)_{L_2(\Om)}
$$
$$
\psi(x)=\sumkui  \psi_k \,v_k(x),\quad \psi_k (t)=
(\psi,v_k)_{L_2(\Om)}
$$
И будем решать задачу
$$
 \left\{
\begin{aligned}
&u_{k_{tt}}-\De u_k =f_k(t) v(k(x) ,\quad (x,t)\in Q_T\\
&u_k|_{\Ga_T}= 0\\
&u|_{t=0}=\ph_k v_k(x)\\
&u_{k_{t}}|_{t=0}=\psi_k v_k(x)\\
\end{aligned}
\right.
$$

Обобщенное решение этой задачи ищем в виде $u_k(x,t)=V_k(t)
v_k(x)$. В этом случае, $V_k(t)$ - решение задачи Коши

$$
 \left\{
\begin{aligned}
&V_k''-\la_k V_k=f_k(t)\\
&V_k(0)=\ph_k\\
&V_k'(0)=\psi_k\\
\end{aligned}
\right.
$$
То есть
$$
V_k(t)=C_1^k \cos \sqrt {-\la_k} t + C_2^k \sin \sqrt
{-\la_k} t + \frac 1{\sqrt {-\la_k}} \int\limits_0^t
f_k(\tau) \sin \sqrt {-\la_k}(t-\tau)\,d\tau
$$

Покажем, что $u_k(x,t)$ действительно обобщенное решение. Краевые
условия очевидным образом выполняются, поэтому осталось проверить
интегральное тождество:
$$
\int \limits_{Q_T} (-u_{k_t} h_t+\nabla u_k \nabla h)\, dx\, dt =
\int \limits_{Q_T} f_k(t) v_k(x) h(x,t)\,dx\,dt+ \int
\limits_\Om \psi_k v_k(x) h(x,0)\,dx
$$
Подставим вместо $u_k$ наше предполагаемое решение:
$$
\int \limits_{Q_T} (-V_k'(t)v_k(x) h_t+ V_k(t)\nabla v_k(x) \nabla
h)\, dx\, dt = \int \limits_{Q_T} f_k(t) v_k(x) h(x,t)\,dx\,dt+
\int \limits_\Om \psi_k v_k(x) h(x,0)\,dx \quad
$$
Интегрируя , находим:
$$
\int \limits_{Q_T} -V_k'(t)v_k(x) h_t\, dx\, dt = \int
\limits_{Q_T} V_k(t) v_k(x) h(x,t)\, dx\, dt + \int \limits_\Om
\psi_k v_k(x) h(x,0)\,dx
$$

 Вспомним, что $v_k$ - собственная функция, то есть
$\int_\Om \nabla v_k \nabla g\, dx= -\la_k \int_\Om v_k
g\, dx$, $\fa g\in H_1^0(\Om)$, следовательно
$$
\int\limits_0^T V_k(T) \Bigl( \int_\Om \nabla v_k \nabla h\,
dx\Bigr)\, dt= -\la_k\int_{Q_T}V_k(t)v_k(x)h(x,t)\, dx\,dt
$$
И подставляя эти результаты, преобразуем левую часть к виду
$$
\int \limits_{Q_T)} (V_k''-\la_k V_k) v_k(x) h(x,t)\,dx\,dt+
\int \limits_\Om \psi_k v_k(x) h(x,0)\,dx
$$
Что, как легко видеть, равно правой части. Итак, $u_k(x,t)=V_K(t)
v_k (x)$ - обобщенное решение.

В силу линейности нашей задачи, функция $S_N(x,t)= \suml{k=1}{N}
V_k(t) v_k(x)$ является решением задачи:
$$
 \left\{
\begin{aligned}
&S_{N_{tt}}-\De S_N =\suml{k=1}{N} f_k(t)v_k(x) ,\quad (x,t)\in Q_T=\Om \times (0,T)\\
&S_N|_{\Ga_T}= 0\\
&S_N|_{t=0}=\suml{k=1}{N} \ph_k v_k(x)\\
&S_{N_t}|_{t=0}=\suml{k=1}{N} \psi_k v_k(x)\\
\end{aligned}
\right.
$$

Покажем, что последовательность $S_N$ сходится по норме, для этого
достаточно доказать фундаментальность. Имеем:
$$
||S_N-S_M||_{H_1^0(\Om)}=||\suml{k=N+1}{M} V_k(t)
v_k(x)||^2_{H_1^0(\Om)} = \suml{k=N+1}{M} V_k(t)
V_k^2(t)||v_k||^2_{H_1^0(\Om)}=\suml{k=N+1}{M} V_k(t) V_k^2(t)
|\la_k|
$$
Оценим $V_k^2(t)$ (используя тождество $(a+b+c)^2\le
3(a^2+b^2+c^2)$:
$$
V_k^2(t)\le 3f_k^2+\frac{3\psi_k^2}{|\la_k|} + \frac
{3T}{|\la_k} \int_0^T f_k^2(t)\,dt
$$
Отсюда
$$
\suml{k=N+1}{M} V_k(t) V_k^2(t) |\la_k|\le C_0\Bigl(
\suml{k=N+1}{M} f_k^2|\la_k|+\suml{k=N+1}{M} \psi_k^2
+\suml{k=N+1}{M} \int_0^T f_k^2(t)\,dt \Bigr)
$$
Каждое из слагаемых при $$M>N\ra\bes$$ стремится к нулю
ввиду сходимости соответствующего ряда ( что было доказано чуть
выше), поэтому и сумма стремится к нулю и в силу приведенной
оценки последовательность $S_N$ фундаментальна по норме. А из
полноты пространства $H_1^0(\Om)$ следует сходимость.


%----------------ЛЕКЦИЯ 14----------------------


\section{Лекция 14}
$S_N\ra S$ при $N\ra\bes$.
$$\int_{Q_T}(-{S_N}_th_t + \De S_N\De
h)dxdt=\int_{Q_T}\suml{k=1}{N} f_k(t)v_k(x)dxdt+
\int_{\Om}\suml{k=1}{N} \psi_kv_k(x)h(x,0)dx
$$
Переходя к пределу при $N\ra\bes$, получаем выполнение
интегрального тождества для $S$.
$$
\|\suml{k=M+1}{N} v'_k(t)v_k(x)\|^2_{L_2(\Om_T)}=\suml{k=M+1}{N}\|{v'}_k(t)\|^2
$$
$$
{v'}_k(t)=-\sqrt{-\la_k}\ph_k\sin\sqrt{-\la_k}t+\psi_k\cos\sqrt{-\la_k}t+
\int_0^tf_k(\tau)\cos\sqrt{-\la_k}(t-\tau)d\tau
$$
$$
|{v'}_k(t)|^2\le
C_1(T)(\ph_k^2|\la_k|)+\psi_k^2+\int_0^Tf_k^2(t)dt
$$
$$
\|{S_N}_t(x,t)-{S_M}_t(x,t)\|^2_{L_2(\Om_t)}\le
C_1(T)\left(\suml{k=M+1}{N}\ph_k^2|\la_k|+\suml{k=M+1}{N}\psi_k^2+\suml{k=M+1}{N}\int_0^Tf_k^2(t)dt\right)\ra
0
$$
при $M,N\ra\bes$.
$$
\int_0^T\|S_N-S_M\|_{H_1(\Om_t)}dt\le
TC_0(T)\left(\suml{k=M+1}{N}\ph_k^2|\la_k|+\suml{k=M+1}{N}\psi_k^2+\suml{k=M+1}{N}\int_0^Tf_k^2(t)dt\right)\ra
0
$$
при $M,N\ra\bes$. Аналогичное равенство имеется для $\{{S_N}_t\}$\\
Итак, $\{S_N\}$ фундаментальна в $H_1^0(\Om)$, значит,
$\{S_N\}$ сходится; $S_N\ra S, N\ra\bes \Rightarrow S$ - решение
нашей задачи (1).\\
Ч.т.д.

$$
\|S_N(x,t)\|^2_{H_1^0(\Om_t)}=\suml{k=1}{N} v_k^2(t)|\la_k|\le
C_2(T)\left(\suml{k=1}{N}\ph_k^2|\la_k|+\suml{k=1}{N}\psi_k^2+\suml{k=1}{N}\int_0^Tf_k^2(t)dt\right)
$$
$$
\|{S_N}_t(x,t)\|^2_{L_2(\Om_t)}\le
C_3(T)\left(\suml{k=1}{N}\ph_k^2|\la_k|+\suml{k=1}{N}\psi_k^2+\suml{k=1}{N}\int_0^Tf_k^2(t)dt\right)
$$
$$
\|S_N(x,t)\|^2_{H_1^0(Q_t)}\le
C_4(T)(\|\ph\|^2_{H_1^0(\Om)}+\|\psi\|^2_{L_2(\Om)}+\|f\|^2_{L_2(Q_T)})
$$
В последнем неравенстве можно перейти к пределу при $N\ra\bes$ и
получить аналогичное неравенство для $\|S(x,t)\|^2_{H_1^0(Q_t)}$.

Таким образом, доказана теорема, которую мы и сформулируем в
следующем разделе.

\subsection{Теорема о существовании обобщенного решения задачи (1)}
\textbf{\underline{\textit{Теорема.}}}\\
Пусть $f\in L_2(Q_T),\quad \ph\in H_1^0(\Om),\quad\psi\in
L_2(\Om)$. Тогда задача (1) имеет обобщенное решение $u(x,t)$ и
это решение представимо в виде сходящегося в $H_1(Q_T)$ ряда
$$
u(x,t)=\suml{k=1}{\bes}V_k(x)v_k(t)
$$
,где $V_k(t)$ - решение задачи Коши, $v_k(x)$ - собственные
функции задачи Дирихле для оператора Лапласа в области $\Om$.
Кроме того, для $u(x,t)$ имеет место оценка
$$
\|u\|^2_{H_1^0(Q_t)}\le
C(T)(\|\ph\|^2_{H_1^0(\Om)}+\|\psi\|^2_{L_2(\Om)}+\|f\|^2_{L_2(Q_T)})
$$
\textit{Упражнение.} Рассмотрим уравнение $u_{tt}-u_{xx}=0,x\in
[0,l],t>0$\\
$u|_{x=0}=u|_{x=l}=0,\quad u|_{t=0}=\ph(x),\quad
u_t|_{t=0}=\psi(x)$.\\
Доказать, что если $\ph\in C^3[0,l]$,
$\ph(0)=\ph(l)=0, \ph''(0)=\ph''(l)=0$, а
также\\
$\psi\in C^2[0,l],$ $\psi'(0)=\psi'(l)=0$, то $u(x,t)$ является
классическим решением данной задачи.


%------------------------------Следующий семестр--------------------------------------


%----------------------------Лекция 15---------------------------------------------


\section{Лекция 15}
\subsection{Гармонические функции, их свойства}

Пусть $u\in C^2(\Om),\ \Om\subs\mathbb(R)^n$ - область.\\
\textit{Определение.} Функция $u$ называется гармонической в $\Om$, если $\fa x\in\Om\ \De u =0$\\
Примеры гармонических функций:\\
1. $u=\sum\limits_{i=1}^n a_ix_i+b$ - гармоническая в $\R^n$\\
2. $n=3:\ u=\frac{1}{r},\ r=|x-x_0|$ - гармоническая при $x\neq x_0$\\
3. $n=2:\ \ln{r},\ r^{\pm m} \cos{m\ph},\ r^{\pm m} \sin{m\ph}$ -  гармонические.


Пусть теперь $u,v\in C^2(\ol{\Om}),\ \Om \ \text{ограниченная область в}\ \
\R^n,\ \pd\Om\in C^1$ тогда вспомним, что

$$
\int\limits_{\Om}uv_{x_jx_j}dx=
\int\limits_{\Om}\frac{\pd u}{\pd x_j}\frac{\pd v}{\pd x_j}dx+
\int\limits_{\pd \Om} u\frac{\pd v}{\pd x_j} \nu_j dS
$$
$$
\int\limits_{\Om} divA dx= \int\limits_{\pd \Om}(A,\nu)dS :\
A=(0,\dots,u\frac{\pd v}{\pd x_j},\dots,0)
$$
\subsection{Формулы Грина}
\textit{Первая формула Грина}
$$
\int\limits_{\Om} u\De v dx=-
\int\limits_{\Om}(\nabla u,\nabla v) dx+
\int\limits_{\pd \Om} u \frac{\pd v}{\pd \nu}dS
$$

$$
\int\limits_\Om v\De u dx=
-\int\limits_\Om(\nabla u,\nabla v) dx+
\int\limits_{\pd \Om} v \frac{\pd u}{\pd \nu}dS
$$
$\nu$ - единичная внешняя нормаль к границе.


\textit{Вторая формула Грина}
$$
\int\limits_{\Om} (u\De v -v\De u) dx=
\int\limits_{\pd \Om}( u \frac{\pd v}{\pd \nu} - v \frac{\pd u}{\pd \nu})dS
$$


Рассмотрим гармонические функции, зависящие только от расстояния до точки, т.е
гармонические функции вида:\\
$v(|x-x_0|),\ x\neq x_0,\ \De v=0$, тогда
$$
\int\limits_{\R^n}v(|x-x_0|)\De\ph(x)dx=
\ph(x_0),\ \fa\ph\in C_0^\bes(\Om) \eqno(15.1)
$$
Оператор Лапласа для таких функций:
$$
\De v(r)=v''+\frac{n-1}{2}v'=0
$$
$$
v(|x-x_0|)=C_1|x-x_0|^{2-n}+C_2,\ n\ge 3
$$
$$
v(|x-x_0|)=C_1\ln{|x-x_0|}+C_2,\ n=2
$$
где $v\in L_{1,loc}$
Рассмотрим левую часть (15.1), она равна
$$
\liml{\ep\ra 0}
\int\limits_{\R^n/\ol{T}_\ep^{x_0}}
(C_1|x-x_0|^{2-n}+C_2)\De \ph dx=
\liml{\ep\ra 0}
\int\limits_{\R^n/\ol{T}_\ep^{x_0}}
(v(|x-x_0|)\De\ph - \ph\De v(|x-x_0|))dx=
$$
Тут применим 2-ю формулу Грина.
$$
=
\liml{\ep\ra 0}
\int\limits_{S_\ep^{x_0}}
[(C_1|x-x_0|^{2-n}+C_2)\frac{\pd \ph}{\pd \nu}
-\ph (x)\frac{\pd}{\pd \nu} C_1 |x-x_0|^{2-n}]dS
\eqno (15.2)
$$
$$
\liml{\ep\ra 0}
(C_1\ep^{2-n}+C_2)
\int\limits_{S_\ep^{x_0}}
\frac{\pd \ph}{\pd \nu}dS=0
$$
Т.к
$$
|\int\limits_{S_\ep^{x_0}}
\frac{\pd \ph}{\pd \nu}dS|
\le
\maxl{\R^n}|\nabla\ph|\om_n\ep^{n-1}
$$
где $\om_n$ площадь поверхности единичной сферы в $\R^n$.\\
Далее заметим, что
$$
C_1\frac{\pd}{\pd\nu}r^{2-n}=-C_1\frac{\pd}{\pd r}r^{2-n}=
-(2-n)C_1r^{1-n}
$$
Продолжаем цепочку (15.2)
$$
=
-C_1(n-2)
\liml{\ep\ra 0}
\int\limits_{S_\ep^{x_0}}
\ph (x) dS=
-C_1(n-2)
\liml{\ep\ra 0}
\ph(x_\epsilon)\om_n\ep^{n-1}\ep^{1-n}
=-C_1(n-2)\om_n\ph(x_0)
$$
Тогда из (15.1) имеем:\\
$C_1=\frac{-1}{\om_n(n-2)},\quad n\ge 3$\\
$C_1=\frac{1}{2\pi},\quad n=2$


\textit{Фундаментальное решение оператора Лапласа.}
$$
E(x,x_0)=
\begin{cases}
-\frac{|x-x_0|^{2-n}}{\om_n(n-2)},\ n\ge 3\\
\frac{1}{2\pi}\ln{(x-x_0)},\ n=2
\end{cases}
$$
Основное характерное свойство фундаментального решения - это формула (15.1) т.е
$$
\int\limits_{\R^n}E(x,x_0)\De\ph(x)dx=
\ph(x_0),\ \fa\ph\in C_0^\bes(\Om)
$$
На языке обобщенных функций это выглядит так
$$
(\De E(x,x_0),\ph (x))=\ph(x_0)=(\de(x-x_0),\ph)
$$
т.е
$$
\De E(x,x_0)=\de(x-x_0)\ \text{- дельта функция}
$$
Покажем, что $\fa$ функции из $C^2(\Om),\ \pd\Om\in C^2
\ \Om$ ограниченная область верно некоторое свойство.\\
Пусть $\Om_\ep=\Om/\ol{T}_\ep^{x_0}$
$$
\int\limits_{\Om_\ep}(u\De E(x,x_0)-E(x,x_0)\De u)dx=
-\int\limits_{\Om_\ep}E(x,x_0)\De udx=
$$
Применяем вторую формулу Грина
$$
\int\limits_{\Om_\ep}(u\frac{\pd E(x,x_0)}{\pd \nu}
-E(x,x_0)\frac{\pd u}{\pd \nu})dS
=
\int\limits_{\Om}(u\frac{\pd E}{\pd \nu}
-E\frac{\pd u}{\pd \nu})dS+
\int\limits_{S_\ep^{x_0}}(u\frac{\pd E(x,x_0)}{\pd \nu}
-E\frac{\pd u}{\pd \nu})dS
$$
Перейдем к пределу при $\ep\ra 0$
$$
-\int\limits_{\Om}E\De udx=
\int\limits_{\pd\Om}u\frac{\pd E(x,x_0)}{\pd \nu}dS-
\int\limits_{\pd\Om}E\frac{\pd u}{\pd \nu}dS
-u(x_0)
$$
т.к
$$
\int\limits_{S_\ep^{x_0}}E\frac{\pd u}{\pd \nu}dS
\ra 0,\ \ep \ra +0
$$
$$
\int\limits_{S_\ep^{x_0}}u(x)\frac{\pd E(x,x_0)}{\pd \nu}dS
\ra - u(x_0)
$$

$$
\left(
\frac{\pd E}{\pd \nu}|_{S_\ep^{x_0}}=-\frac{\pd E}{\pd r}|_{r=\ep}
=-\frac{\pd}{\pd r}\frac{-r^{2-n}}{\om_n(n-2)}|_{r=\ep}=
-\frac{\ep^{1-n}}{\om_n}
\right)
$$

Таким образом, получаем, что
$$
\fa x\in \Om\quad
u(x_0)=
\int\limits_{\Om}E(x,x_0)\De udx+
\int\limits_{\pd   \Om}u\frac{\pd E(x,x_0)}{\pd \nu}dS-
\int\limits_{\pd\Om}E(x,x_0)\frac{\pd u}{\pd \nu}dS
$$
Где
$$
\int\limits_{\Om}E(x,x_0)\De udx\  \text{- объемный потенциал.}
$$
$$
\int\limits_{\pd   \Om}u\frac{\pd E(x,x_0)}{\pd \nu}dS\ \text{- потенциал двойного слоя.}
$$
$$
\int\limits_{\pd\Om}E(x,x_0)\frac{\pd u}{\pd \nu}dS\ \text{- потенциал простого слоя.}
$$
Если $\De u =0,\ u\in C^2(\ol{\Om}), x_0\in \Om$, то
$$
u(x_0)=
\int\limits_{\pd   \Om}u\frac{\pd E(x,x_0)}{\pd \nu}dS-
\int\limits_{\pd\Om}E(x,x_0)\frac{\pd u}{\pd \nu}dS
$$

\textbf{Лемма (о потоке через границу гармонической функции)}\\
Пусть $u \in C^2(\ol{\Om}),\ \De u =0$ в $\Om,\ \Om$
ограниченная область, $\pd\Om\in C^1$.
Тогда
$$
\int\limits_{\pd\Om}\frac{\pd u}{\pd \nu}dS=0
$$
Для доказательства надо использовать вторую формулу Грина
для $v\equiv 1$

\textbf{Теорема 1 (О среднем по сфере).}\\
Пусть $u\in C^2(T_R^{x_0})\cap C(\ol{T}_R^{x_0})$, предположим, что $\De u=0$ в $T_R^{x_0}$
тогда
$$
u(x_0)=\frac{1}{|S_R^{x_0}|}\int\limits_{S_R^{x_0}}
u(x)dS
$$
\textit{Доказательство}\\
Для $T_\rho^{x_0}$ можно записать
$$
u(x_0)=
\int\limits_{S_\rho^{x_0}}u\frac{\pd E(x,x_0)}{\pd \nu}dS-
\int\limits_{S_\rho^{x_0}}E(x,x_0)\frac{\pd u}{\pd \nu}dS
=
\frac{1}{\om_n\rho^{n-1}}\int\limits_{S_\rho^{x_0}} u(x)dS
$$
Осталось перейти к пределу и получить то, что нужно. Второе
слагаемое обнулилось по предыдущей лемме.

\textbf{Теорема 2 (О среднем по шару)}

Пусть $u\in C^2(T_R^{x_0})\cap C(\ol{T}_R^{x_0})$, предположим, что $\De u=0$ в $T_R^{x_0}$
тогда
$$
u(x_0)=\frac{1}{|T_R^{x_0}|}\int\limits_{T_R^{x_0}}
u(x)dS
$$

\textit{Доказательство}

Из теоремы 1 следует, что
$$
\om_n\rho^{n-1}u(x_0)=\int\limits_{S_R^{x_0}}u(x)dS
$$
Осталось проинтегрировать это равенство по $\rho$ от 0 до R.
$$
\frac{\om_n}{n}R^n u(x_0)=\int\limits_{T_R^{x_0}}u(x)dS
$$

\textbf{Теорема 3 (Строгий принцип максимума).}

Пусть $u\in C^2(\Om)\cap C(\ol{\Om})$, предположим, что $\De u=0$ в $\Om$.
$\Om$ ограниченная область с гладкой границей. Положим
$$
\maxl{\ol{\Om}}u(x)=M,\ (\min\limits_{\ol{\Om}}u(x)=m)
$$
Тогда, если $u(x_0)=M(m)$, где $x_0\in \Om$
$$
\Rightarrow u(x)\equiv \const (m,M)\  \text{в}\  \Om
$$
\textit{Доказательство}\\
$T_R^{x_0},\ \ol{T_R^{x_0}}\in\Om$\\
Предположим, что в $T_R^{x_0},\ u(x)\equiv M,\ u(x_0)=M$. Предположим противное, т.е
$\exists x'\in T_R^{x_0},\ u(x')<M$, следовательно в силу непрерывности\\
$\exists\ep,\de>0\ :\ u(x)<M-\ep$  в шаре $T_\de^{x'}$
Тогда по предыдущей теореме имеем
$$
u(x_0)=M=\frac{1}{|T_R^{x_0}|}\int\limits_{T_R^{x_0}}u(x)dS
=\frac{1}{|T_R^{x_0}|}
\left\{
\int\limits_{T_R^{x_0}/ \ol{T_\de^{x'}}}u(x)dS
+
\int\limits_{ \ol{T_\de^{x'}}}u(x)dS
\right\}\le
$$
$$
\le\frac{1}{|T_R^{x_0}|}
\left\{
M(|T_R^{x_0}|-|T_\de^{x'}|)+|T_\de^{x'}|(M-\ep)
\right\}
=M-\frac{\ep T_R^{x_0}}{T_\de^{x'}}<M
$$
Противоречие.\\
Пусть теперь $u(x_1)<m$, соединим эти две точки области кривой (в силу односвязности)
покрываем эту кривую шарами. Для каждого шара проводим такие же рассуждения, тогда
и $u(x_1)=M$. Тем самым теорема доказана.

\textbf{Теорема 4 (О бесконечной дифференцируемости).}

Пусть $\Om$ область в $\R^n.\ u(x)\ $ гармоническая в $\Om$
тогда $u\in C^\bes(\Om)$

\textit{Доказательство}

$$
u_h(x)=\int\limits_{\R_y^n}\om_h(|x-y|)u(y)dy=
\int\limits_{|x-y|<h}\om_h(|x-y|)u(y)dy=
\int\limits_0^h\om_h(\rho)\int\limits_{S_\rho^x}u(y)dSd\rho=
$$
$$
=u(x)\int\limits_{|x|<h}\om_h(|x|)dx=u(x)
$$
-в силу свойства ядра усреднения и теоремы о среднем.\\
А в силу свойства ядра усреднения
$$
u_h\in C^\bes(\R^n)\Rightarrow u(x)\in C^\bes(\Om)
$$

\textbf{Теорема 5 (О знаке нормальной производной гармонической функции в точке минимума (максимума)).}
Пусть $u\in C^2(T_R^{x_0})\cap C(\ol{T}_R^{x_0})$, предположим, что $\De u=0$ в $T_R^{x_0}$
$u\neq \const$ в $T_R^{x_0}$

Также предположим, что в точке $x'\in S_R^{x_0},\ \min\limits_{\ol{T}_R^{x_0}}u(x)=u(x')$.
Тогда, если существует нормальная производная $\frac{\pd u}{\pd \nu}(x'),\ \nu$\т внешняя единичная
нормаль к границе в точке $x'$, то
$$\frac{\pd u}{\pd \nu}(x')<0.$$


%-------------------------Лекция 16--------------------------------------------------

\section{Лекция 16}
\subsection{Лемма о знаке нормальной производной гармонической функции в точке максимума}
Пусть $u(x)\neq \const$ - гармоническая в $\Om$,
$x_0\in\pd\Om$ - точка максимума $u(x)$, $\exists
B_{\rho}^{x'}\subs\Om: S_{\rho}^{x'}\cap \pd\Om =
\{x_0\}$, $\exists \frac {\pd u}{\pd\nu}(x_0)=\liml{s\ra
+0}\frac{u(x_0)-u(x_0-s\nu)}{s}$. Тогда $\frac {\pd
u}{\pd\nu}(x_0)>0$.\\
Доказательство.\\
Можно считать, что $u(x_0)=0, u(x)<0$ в $\Om$. Вспомним
следствие принципа максимума: если $\De u=\De v=0$ в
$\Om$, $u(x)\le v(x)$ на $\pd\Om$, то $u(x)\le v(x)$
в $\Om$.\\
Рассмотрим $w(x)=-(|x-x'|^{2-n}-\rho^{2-n})$. При $n\ge 3$ $w(x)$
будет гармонической в $\R^n\wo\{x'\}$. Рассмотрим
шаровой слой $K=B_{\rho}^{x'}\wo \ol{B}_{\rho/2}^{x'}$
и функции $u(x), \ep w(x)$ на $K$.\\
Внешняя граница: $|x-x'|=\rho$, и на ней $w(x)=0,u(x)\le
0$.\\
Внутренняя граница:
 $|x-x'|=\rho/2$, и на ней
 $u(x)<-c<0,w(x)=-(\left(\frac{\rho}{2}\right)^{2-n}-\rho^{2-n})=-\frac{2^{n-2}-1}{\rho^{n-2}}$.\\
 $\exists \ep>0: \ep w(x)\ge u(x)$ на $\pd
 K$, $\ep = c\frac {\rho^{n-2}}{2^{n-2}-1}$. Значит, $\ep w(x)\ge
 u(x)$в $K$.\\
 $\ep w(x_0) - \ep w(x_0-s\nu)\le
 u(x_0)-u(x_0-s\nu)\Rightarrow 0<\ep\frac {\pd
w}{\pd\nu}(x_0)\le\frac {\pd
u}{\pd\nu}(x_0)$\\
Ч.т.д.\\
\textit{Примечание.} В случае $n=2$ достаточно рассмотреть функцию
$w(x)=\ln|x-x_0|-\ln\rho$.
\subsection{Основные краевые задачи для уравнения Лапласа и единственность решения этих задач}
\subsubsection{Задача Дирихле}
$$
\left\{
\begin{aligned}
\De u=0\quad \mbox{в ограниченной области } \Om,\quad u\in C^2(\Om)\cap C^0(\ol{\Om})\\
u|_{\pd\Om}=\ph\quad \ph\in C(\pd\Om)\\
\end{aligned}
\right.
$$
Решение задачи Дирихле единственно. Если мы рассмотрим $w(x)$\т
разность двух решений, то имеем $\De w=0,
w|_{\pd\Om}=0$, тогда по принципу максимума/минимума
$w\equiv 0$.
\subsubsection{Задача Неймана}
$$
\left\{
\begin{aligned}
\De u=0\quad \mbox{в ограниченной области } \Om,\quad u\in C^2(\Om)\cap C^1(\ol{\Om})\\
\frac{\pd u}{\pd \nu}|_{\pd\Om}=\psi\quad \psi\in C(\pd\Om)\\
\end{aligned}
\right.
$$
Условие разрешимости: $\int_{\pd\Om}\psi ds=0$.\\
Решение задачи Неймана определено с точностью до константы. Если
мы рассмотрим $w(x)$ - разность двух решений, то имеем $\De
w=0,\frac{\pd w}{\pd \nu}|_{\pd\Om}=0$, тогда по
лемме о нормальной производной $w=\const$.

\subsection{Оценки производных гармонической функции}

$$
u(x_0)=\frac{1}{\om_nR^n}\int_{|x-x_0|\le R}u(x)dx
$$
$u(x)$ - гармоническая $\Rightarrow$ $\frac{\pd u}{\pd
x_k}$ - гармоническая.
$$
\left|\frac{\pd u}{\pd x_k}(x_0)\right|=\frac{1}{\om_nR^n}\int_{|x-x_0|\le
R}\frac{\pd u}{\pd
x_k}dx=\frac{1}{\om_nR^n}\int_{|x-x_0|\le R}u\cos(\nu,x)ds
$$
$$
\left|\frac{\pd u}{\pd x_k}(x_0)\right|\le
\frac{\si_n R^{n-1}}{\om_nR^n} \maxl{|x-x_0=R|}|u|,
$$
где $\si_n$ - площадь единичной сферы, $\om_n$ - объем
единичного шара в $\R^n$.
$$
\left|\frac{\pd u}{\pd x_k}(x_0)\right|\le \frac{n}{R}
\maxl{|x-x_0=R|}|u|
$$
Пусть $\ol{\Om}_1\subs\Om_0$ - ограниченная область
в $\R^n$ и $dist(\pd\Om_1,\pd\Om_0)\ge
d>0$. Пусть $u$ - гармоническая в $\Om_0$,$\quad u\in
C^2(\Om)\cap C(\ol{\Om})$. Тогда
$$
\fa x\in\Om_1\quad\left|\frac{\pd u}{\pd
x_k}(x_0)\right|\le \frac{n}{d} \maxl{\ol{\Om}_0}|u|
$$
Аналогично (методом математической индукции) доказывается
неравенство
$$
|\Dc^{\alpha}u(x)|\le\left(\frac{nm}{\si}\right)^mmax_{\ol{\Om}_0}|u|,
$$
где $x\in\Om_0,\quad dist(x,\pd\Om_0)=\si>0$.
Действительно, пусть оценка доказана для всех $\alpha:|\alpha|\le
k-1$. Возьмем два шара $B^x_{\si'}$ и $B^x_{\si'/k}$, где
$\si'$ - любое положительное число, меньшее $\si$. По
предположению индукции для любой точки $\xi$ из шара
$B^x_{\si'/k}$ и любого $\beta$, $|\beta|=k-1$, имеет место
неравенство
$$
|\Dc^{\beta}u(\xi)|\le\left(\frac{n(k-1)}{\si'-\si'/k}\right)^{k-1}\maxl{\ol{\Om}_0}|u|=
\left(\frac{nk}{\si'}\right)^{k-1}\maxl{\ol{\Om}_0}|u|
$$
Таким образом, для любого $\beta$, $|\beta|=k-1$, гармоническая
функция $|\Dc^{\beta}u(\xi)|$ ограничена в шаре
$B^x_{\si'/k}$ постоянной
$\left(\frac{nk}{\si'}\right)^{k-1}\maxl{\ol{\Om}_0}|u|$.
Тогда для первых производных этой функции по уже доказанному имеем
$$
|\Dc^{\beta}u(\xi)_{\xi_l}|\le\left(\frac{nk}{\si'}\right)^kmax_{\ol{\Om}_0}|u|
$$
Переходя в этом неравенстве к пределу при $\si'\ra\si-0$,
получаем требуемое неравенство. Ч.т.д.

\subsection{Аналитичность гармонических функций}
\textit{\textbf{\underline{Теорема.}}} Гармоническая в области
$\Om$ функция $u(x)$ является аналитической в $\Om$.
$$
u(x)=\sums{|\alpha|<m}\frac{\Dc^{\alpha}u(x_0)}{\alpha!}(x-x_0)^{\alpha}+
\sums{|\alpha|=m}\frac{\Dc^{\alpha}u(\wt{x})}{\alpha!}(x-x_0)^{\alpha}
,$$
где
$\alpha!=\alpha_1!\ldots\alpha_n!$,$\quad(x-x_0)^{\alpha}=(x_1-{x_0}_1)^{\alpha_1}\ldots(x_n-{x_0}_n)^{\alpha_n}.$\\
Обозначим
$$
\ga_m(x_0,x,\wt{x})=\sums{|\alpha|=m}\frac{\Dc^{\alpha}u(\wt{x})}{\alpha!}(x-x_0)^{\alpha}
$$
Пусть $|x-x_0|<\ep,\quad x,\wt{x}\in
B_{\rho}^{x_0},\quad$ $u$ - гармоническая в $B_{2\rho}^{x_0}$.
Тогда
$$
|\ga_m(x_0,x,\wt{x})|\le\ep^m\left(\frac{nm}{\rho}\right)^mmax_{B_{2\rho}^{x_0}}|u|\sums{|\alpha|=m}\frac{1}{\alpha!}
$$
Но
$$
\sums{|\alpha|=m}\frac{1}{\alpha!}=\frac{1}{m!}\sums{|\alpha|=m}\frac{m!}{\alpha!}=\frac{n^m}{m!}
$$
Получаем
$$
|\ga_m(x_0,x,\wt{x})|\le\left(\frac{\ep
n^2m}{\rho}\right)^m\frac{1}{m!}\maxl{B_{2\rho}^{x_0}}|u|
$$
Согласно формуле Стирлинга, $m!\sim \sqrt{2\pi
m}\left(\frac{m}{e}\right)^m$,тогда $|\ga_m(x_0,x,\wt{x})|$
оценивается сверху величиной, эквивалентной
$$
\frac{c}{\sqrt{m}}\left(\frac{\ep e n^2}{\rho}\right)^m\ra
0 \mbox{ при } m\ra\bes, \ep\ll1
$$







%---------------------Лекция 17------------------------------










\section[Лекция 17.]{Лекция 17.}
\subsection {Функция Грина. Задача Дирихле для уравнения Лапласа.}
%\newtheorem*{theorem}{Теорема}

\begin{theorem}[Лиувилль]
Пусть $u(x)$ - гармоническая в $\R^n$, неотрицательная
функция. Тогда $u=\const$.
\end{theorem}

\textbf{Доказательство:} Зафиксируем точку $x_0$ и шар $Q_R^{x_0}$
радиуса $R$ с центром в нашей точке.Поскольку производная
гармонической функции -- также функция гармоническая, по теореме о
среднем имеем:
$$
\frac {\pd u}{\pd x_j}(x_0)=\frac 1{|Q_R^{x_0}|}
\int_{Q_R^{x_0}} \frac {\pd u}{\pd x_j}(x)\, dx= \frac
1{|Q_R^{x_0}|} \int_{S_R^{x_0}} u(x) \nu_j(x) dS
$$
Мы использовали формулу Стокса, чтобы перейти к интегрированию по
границе шара. А теперь используем еще одну теорему о среднем, на
этот раз из курса математического анализа:
$$
=\frac 1{|_R^{x_0}|} \nu_j(\wt{x}) \int_{S_R^{x_0}} u(x)
 dS=\frac{|S_R^{x_0}|}{|Q_R^{x_0}|} \nu_j(\wt{x}) |u(x_0)|
$$
Здесь $|S_R^{x_0}|=w_n R^{n-1}$,а $|Q_R^{x_0}|=\frac {w_n} n R^n$.
Таким образом,
$$
|\frac {\pd u}{\pd x_j}(x_0)|\le \frac n R
|u(x_0)|\ra 0 \quad R\ra\bes
$$
Ничего не мешает нам выбрать радиус шара сколь угодно большим, а
значит, $|\frac {\pd u}{\pd x_j}(x_0)|=0\quad \fa
j=1,\ldots,n$. Следовательно, $u=\const$, что и требовалось.

\textbf{Задача:} Пусть $u(x)$ - гармоническая в $\R^n$ и
$u(x)\ge -C(1+|x|^m)$, где $c,m>0$ $u(x)$. Показать, что в этом
случае $u(x)$ есть полином степени не выше $[m]$.

Рассмотрим задачу Дирихле для оператора Лапласа:
$$
 \left\{
\begin{aligned}
&\De u(x)=0\quad x\in\Om\\
&u(x)=\ph(x)\quad x\in \pd\Om,\,f\in C(\pd\Om)\\
\end{aligned}
\right.
$$
Под классическим решением понимается решение из класса
$C^2(\Om)\bigcap C(\ol \Om)$.

Для такого решения ранее была получена формула
$$
u(x_0)=\int_\Om E(|x-x_0|) \De u(x)\, dx
+\int_{\pd\Om}u(x) \frac {\pd
E(|x-x_0|)}{\pd\nu}\, dS-\int_{\pd\Om} E(|x-x_0|)
\frac {\pd u}{\pd \nu}(x)\, dS
$$

Пусть существует функция $g(x,x_0)$ со следующими свойствами:\\
$g(x,x_0)\in C^2_x(\ol \Om)\,$,  $\De_x
g(x,x_0)=0\,\fa x_0 \in \Om$ и при этом $g(x,x_0)|_{x\in
\pd\Om}=-E(|x-x_0|)|_{x\in
\pd\Om},\,\fa x_0 \in \Om$

Запишем вторую формулу Грина для $u(x),g(x,x_0)$:
$$
\int_\Om (u\De g-g\De u )\, dx = \int_{\pd\Om} u
\frac{\pd g}{\pd \nu}-g \frac{\pd u}{\pd \nu}
$$
Поскольку функции $u$ и $g$ -- гармонические в области, то левая
часть формулы обращается в ноль. Теперь прибавим к левой и правой
частям уже упоминавшееся соотношение:
$$
u(x_0)=\int_{\pd\Om}u(x) \frac {\pd
E(|x-x_0|)}{\pd\nu}\, dS-\int_{\pd\Om} E(|x-x_0|)
\frac {\pd u}{\pd \nu}(x)\, dS
$$
И поскольку $g(x,x_0)+E(|x-x_0|)=0$ на $\pd\Om$, получаем:
$$
u(x_0)=\int_{\pd\Om}u(x)
\frac{\pd(E(|x-x_0|)+g(x,x_0))}{\pd\nu}\, dS
$$

\textbf{Определение:} \emph{Функцией Грина} называется функция
$G(x,x_0)=E(|x-x_0|)+g(x,x_0)$, где функция $g(x,x_0)$ введена
ранее. Точка $x_0$ называется \emph{полюсом}.

Итак, если $u\in C^2(\ol\Om)$ -- решение задачи Дирихле,
то:
$$
u(x_0)=u(x_0)=\int_{\pd\Om}u(x) \frac{\pd
G(x,x_0)}{\pd\nu}\, dS
$$

%\newtheorem*{lemma}{Лемма}
\begin{lemma}
$\fa x_1,x_0 \in \Om \quad G(x_1,x_0)=G(x_0,x_1)$
\end{lemma}
\textbf{Доказательство:} Исходя из определения функции Грина,
$\De_x G(x,x_0)=0$ при $\neq x_0$. Вырежем вокруг точек $x$ и
$x_0$ шарики маленького радиуса $\ep$, и то что осталось
обозначим за $\Om_\ep$:
$$
\Om_\ep=\Om \\(\ol{Q_\ep^{x_0}\cup
Q_\ep^{x_1}})
$$
Введем функции $u(x)=G(x,x_0)$, и $v(x)=G(x,x_1)$. В области
$\Om_\ep$ они гармонические, поэтому по второй формуле
Грина:
$$
0=\int_{\Om_\ep} (u\De v- v\De u)\,
dx=\int_{\pd \Om_\ep} \Bigl(u \frac {\pd
v}{\pd \nu}-v \frac {\pd u}{\pd \nu}\Bigr) \, dS
$$
Так как функция $g$ определялась условием
$$g(x,x_0)\evn{x\in \pd\Om}=-E(|x-x_0|)\evn{x\in \pd\Om}, \fa x_0 \in \Om,$$
то $G(x,x_0)|_{\pd\Om}=0$ и $u(x)|_{\pd\Om}=v(x)|_{\pd\Om}=0$. Поэтому
$$
0=\int_{S_\ep^{x_0}}G(x,x_0)\frac{\pd
G(x,x_1)}{\pd\nu}\,dS+\int_{S_\ep^{x_1}}G(x,x_0)\frac{\pd
G(x,x_1)}{\pd\nu}\,dS
$$
Перейдем к пределу при $\ep\ra 0$,
воспользовавшись тем, что
$$
\frac{\pd E(|x-x_1|)}{\pd\nu}= -\frac{\pd
E}{\pd r}(r)=-\frac 1{w_n r^{n-1}}
$$
Получаем,что
$$
\int_{S_\ep^{x_1}}G(x,x_0)\frac{\pd
G(x,x_1)}{\pd\nu}\,dS=-\frac 1{w_n \ep^{n-1}}
\int_{S_\ep^{x_1}} G(x,x_0)\, dS\ra - G (x_1,x_0)
$$
Аналогично переходя к пределу в другом слагаемом, имеем
$$
0=-G(x_1,x_0)+G(x_0,x_1)
$$
что и требовалось доказать.











%-----------------------------ЛЕКЦИЯ 18-------------------------------





\section{Лекция 18}
\subsection{Интеграл Пуассона}
$u(x)\in C^2(\ol{\Om}),\quad$ $\Om$ - ограниченная
область в $\R^n$,$\quad\pd\Om\in C^1$.\\
Будем решать следующую задачу (1):
$$
\left\{
\begin{aligned}
\De u=f(x)\quad \mbox{в } \Om\\
u|_{\pd\Om}=\ph(x)\\
\end{aligned}
\right.
$$
$$
u(x_0)=\int_{\Om}f(x)G(x,x_0)dx+\int_{\pd\Om}\ph(x)\frac{\pd
G}{\pd \nu_x}(x,x_0)ds
$$
$G(x,x_0)$ - функция Грина, $G$ не более чем единственна.\\
$\Om=Q^0_R\subs\R^n,\quad
G(x,x_0)=E(|x-x_0|)-E\left(\frac{\rho}{R}|x-x_*|\right)$\\
Если
$$
\left\{
\begin{aligned}
\De u=f(x), x\in  \Om,\quad u\in C^2(\ol{Q_R^0})\\
u|_{S^0_R}=\ph(x)\\
\end{aligned}
\right.
$$
,то $u(x_0)=\int_{S^0_R}\ph(x)\frac{\pd
G(x,x_0)}{\pd \nu_x}ds_x$.
$$
\frac{\pd G(x,x_0)}{\pd \nu}|_{x\in S_R^0,x_0\in
Q_R^0}=E'(r)\frac{\pd |x-x_0|}{\pd
\nu}-\frac{\rho}{R}E'(\frac{\rho}{R}r_1)\frac{\pd
|x-x_*|}{\pd \nu}
,$$
где $r=|x-x_0|,r_1=|x-x_*|$.\\
$\frac{\pd |x-x_0|}{\pd
\nu}=\suml{j=1}{n} \frac{x_j-{x_0}_j}{|x-x_0|}=\cos\ga$, где
$\ga$ - угол между $\nu$ и $x-x_0$. Аналогично $\frac{\pd
|x-x_*|}{\pd \nu}=\cos\beta$.\\
По теореме косинусов $\rho^2=R^2+r^2-2Rr\cos\ga\Rightarrow
\cos\ga=\frac{R^2+r^2-\rho^2}{2Rr}$.\\
Аналогично
$$
\cos\beta=\frac{R^2+r_1^2-\rho_1^2}{2Rr_1}=\frac{R^2+\frac{R^2r^2}{\rho^2}-\frac{r^4}{\rho^2}}{2R\frac{rR}{\rho}}=
\frac{1}{\rho}\frac{\rho^2+r^2-R^2}{2r}
$$
Т.к. $E(r)=-\frac{r^{2-n}}{\om_n(n-2)}$, то
$E'(r)=\frac{1}{\om_nr^{n-1}}$.\\
Тогда
$$
\frac{\pd G(x,x_0)}{\pd \nu}|_{x\in S_R^0,x_0\in
Q_R^0}=E'(r)(\cos\ga-\frac{\rho}{R}\cos\beta)=E'(r)\frac{R^2-\rho^2}{Rr}=\frac{R^2-|x_0|^2}{\om_nR}\frac{1}{|x-x_0|^n}
$$
Следовательно,
$$
u(x_0)=\frac{R^2-|x_0|^2}{\om_nR}\int_{S_R^0}\frac{\ph(x)}{|x-x_0|^n}ds_x\quad
\mbox{- интеграл Пуассона}
$$
$r^2=R^2+\rho^2-2R\rho\cos\alpha$, тогда
$$
u(x_0)=\frac{R^2-|x_0|^2}{\om_nR}\int_{S_R^0}\frac{\ph(x)ds_x}{(R^2+|x_0|^2-2R|x_0|\cos\alpha)^{n/2}}
$$
Пусть $\ph\in C(S^0_R)$. Тогда функция $u(x)$, задаваемая
интегралом Пуассона, есть решение задачи (1). Необходимо проверить
2 условия:\\
1) $u(x_0)=\int_{\pd\Om}\ph(x)\frac{\pd
G(x,x_0)}{\pd \nu_x}ds_x$- гармоническая по $x_0$,$\quad x_0\in Q^0_R$.\\
2) $\fa \hat{x}\in S^0_R\quad\exists \liml{x_0\ra
\hat{x}}u(x_0)=\ph(\hat{x})$\\
Докажем это.\\
1) При $x\neq x_0$ $\De_xG(x,x_0)=0\quad\Rightarrow\quad
\De_{x_0}G(x_0,x)=0\quad\Rightarrow\quad
\De_{x_0}G(x,x_0)=0$ в силу симметричности $G$.\\
$x_0\in\Om\subs\subs Q^0_R\quad\Rightarrow\quad
\De_{x_0}u(x_0)=\int_{\pd\Om}\ph(x)\frac{\pd}{\pd
\nu_x}\De_{x_0}G(x,x_0)ds_x=0$\\
2) В силу решения задачи Дирихле задача
$$
\left\{
\begin{aligned}
\De u=0, x\in  Q_R^0\\
u|_{S^0_R}=1\\
\end{aligned}
\right.
$$
имеет единственное решение $u\equiv 1$, тогда
$$
1=\frac{R^2-|x_0|^2}{\om_nR}\int_{S_R^0}\frac{ds_x}{r^n}
$$
$$
\ph(\hat{x})=\frac{R^2-|x_0|^2}{\om_nR}\int_{S_R^0}\frac{\ph(\hat{x})}{r^n}ds_x
$$
$$
|u(x_0)-\ph(\hat{x})|=\left|\frac{R^2-|x_0|^2}{\om_nR}\int_{S_R^0}\frac{\ph(x)-\ph(\hat{x})}{r^n}ds_x\right|\le
\frac{R^2-|x_0|^2}{\om_nR}\int_{S_R^0}\frac{|\ph(x)-\ph(\hat{x})|}{r^n}ds_x=
$$
$$
=\frac{R^2-|x_0|^2}{\om_nR}\int_{\si_{\de(\ep)}}\frac{|\ph(x)-\ph(\hat{x})|}{r^n}ds_x+
\frac{R^2-|x_0|^2}{\om_nR}\int_{S_R^0\wo\si_{\de(\ep)}}\frac{|\ph(x)-\ph(\hat{x})|}{r^n}ds_x
$$
Первое слагаемое обозначим за $I_1$, второе за $I_2$.\\
Здесь мы пользовались тем, что $\fa \ep \quad\exists
\de>0:\quad x\in\si_{\de}=Q_{\de}^{\hat{x}}\cap S_R^0
\Rightarrow |\ph(x)-\ph(\hat{x})|<\frac{\ep}{2}$.
$$
I_1\le\frac{\ep}{2}\frac{R^2-|x_0|^2}{\om_nR}\int_{S_R^0}\frac{ds_x}{r^n}=\frac{\ep}{2}
$$
Что можно сказать об $I_2$? При $x\in
S^0_R\wo\si_{\ep}$ имеем $|x-x_0|\ge a>0$, как
только $|x_0-x|<\de_1$. Тогда
$$
I_2\le
2\maxl{S_R^0}|\ph(x)|a^{-n}\frac{\om_nR^{n-1}}{\om_nR}(R^2-|x_0|^2)=c_1(R^2-|x_0|^2)
$$
С другой стороны, $\exists \wt{\de}>0:
|x_0-\hat{x}|<\wt{\de} \Rightarrow
R^2-|x_0|^2<\frac{\ep}{c_1}$. Тогда
$I_2<\frac{\ep}{2}$, а это и требовалось.

\subsection{Неравенство Харнака}

\begin{theorem}[Неравенство Харнака]
Пусть $u(x)$\т гармоническая в $Q^0_R$ функция, непрерывная вплоть
до границы; $u(x)\ge 0$. Тогда $\fa x\in Q^0_R$
$$
\frac{R^{n-2}(R-\rho)}{(R+\rho)^{n-1})}\le u(x_0)\le
\frac{R^{n-2}(R+\rho)}{(R-\rho)^{n-1})},
$$
где $\rho=|x_0|$.
\end{theorem}
\begin{proof}
$$
u(x_0)=\frac{R^2-|x_0|^2}{\om_nR}\int_{S_R^0}\frac{u(x)}{r^n}ds_x;\quad
R-\rho\le r\le R+\rho
$$
$$
\frac{R^2-|x_0|^2}{\om_nR}\frac{1}{(R+\rho)^n}\int_{S_R^0}u(x)ds_x\le
u(x_0)\le\frac{R^2-|x_0|^2}{\om_nR}\frac{1}{(R-\rho)^n}\int_{S_R^0}u(x)ds_x
$$
Но $\int_{S_R^0}u(x)ds_x=u(0)\om_nR^{n-1}$, откуда и получаем
нужное нам
утверждение.
\end{proof}

\subsection{Обратная теорема о среднем}
Пусть $u\in C(\Om)$ и для любого шара
$\ol{Q_r^{x_0}}\subs\Om$ для $u$ справедлива теорема о
среднем по сфере. Тогда $u(x)$ - гармоническая в $\Om$
функция.\\
Доказательство.\\
Задача
$$
\left\{
\begin{aligned}
\De v=0, x\in  Q_R^{x_0},\ol{Q_r^{x_0}}\subs\Om\\
v|_{S^{x_0}_R}=u|_{S^{x_0}_R}\\
\end{aligned}
\right.
$$
имеет единственное решение, которое задается интегралом
Пуассона.\\
Тогда имеем $v(x)\equiv u(x)$ в $Q_R^{x_0}$, что следует из
равенства $u(x)$ и $v(x)$ на $\pd Q_R^{x_0}$, выполнения
теоремы о среднем и принципа максимума.
\subsection{Теорема об устранимой особенности}
Пусть $u(x)$ - гармоническая в $\Om\wo\{x_0\}$ функция,
$m(\rho)=\supl{\Om\wo Q_{\rho}^{x_0}}|u(x)|$.\\
Пусть $m(\rho)\le a(\rho)E(\rho)$, где $a(\rho)\ra 0$ при
$\rho\ra 0$.\\
Тогда $u(x)$ можно доопределить в $x_0$ таким образом, что
полученная функция будет гармонична в $\Om$.







%---------------------------Лекция 19 (5)--------------------------







\section{Лекция 19}
\textit{Доказательство теоремы об устранимой особенности}\\
Рассмотрим $Q_{\rho_1}^{x_0}: \ \ol{Q_{\rho_1}^{x_0}}\in\Om $\\
Найдем:
$$
\begin{cases}
\De v(x)=0,\ x\in Q_{\rho_1}^{x_0}\\
v(x)=u(x),\ x\in Q_{\rho_1}^{x_0}
\end{cases}
$$
Наша цель показать, что $u(x)=v(x)$ в шаре всюду, кроме его центра $x_0$.
Рассмотрим $\om(x)=v(x)-u(x), x\in Q_{\rho_1}^{x_0}/\ol{Q_{\rho_1}^{x_0}}$.
$\De \om=0,\ x\in Q_{\rho_1}^{x_0}/\ol{Q_{\rho_1}^{x_0}}$
$\om=0$ на $S_{\rho_1}^{x_0}$
$$
\maxl{\ol{Q_{\rho_1}^{x_0}}}|v(x)|\le \maxl{S_{\rho_1}^{x_0}}|u(x)|=M
$$
Возьмем произвольное $\ep>0,\ Q_{\rho_1}^{x_0}/\left\{x_0\right\},\ |\om(x)|\le\ep E(|x-x_0|)$
$$
\maxl{S_{\rho_1}^{x_0}}|\om(x)|\le
\maxl{S_{\rho_1}^{x_0}}|u(x)|+
\maxl{S_{\rho_1}^{x_0}}|v(x)|\le
\eqno(19.1)
$$
$\fa \ep>0 \exists\rho\le\rho_0:\ M\le\frac{\ep}{2}|E(\rho)|$\\
С другой стороны по условию теоремы
$$
\maxl{S_\rho^{x_0}}|u(x)|\le a(\rho)|E(\rho)|\
\fa\ep>0\exists \widetilde{\rho}_0:
\fa\rho\le\widetilde{\rho}_0,\ a(\rho)\le\frac{\ep}{2}
$$
$$
\Rightarrow \maxl{S_\rho^{x_0}}|u(x)|\le \frac{\ep}{2}|E(\rho)|
$$
Тогда (19.1) можно продолжить
$$
\le
\ep|E(\rho)|,\ \fa\rho\le\widehat{\rho}
$$
По принципу максимума:
$$
|\om(x)|\le\ep|E(|x-x_0|)|,\ \fa x\in Q_{\rho_1}^{x_0}/\left\{x_0\right\}
$$
Следовательно, полагая $u(x_0):=v(x_0)$, получаем гармоническую функцию во всем шаре

\subsection{Теория потенциала}
$$
u(x)\in C^2({\ol{\Om}}),\ \pd\Om\in C^1
$$
Вспомним, что
$$
u(x_0)=
\int\limits_{\Om}E(x,x_0)\De udx+
\int\limits_{\pd   \Om}u\frac{\pd E(x,x_0)}{\pd \nu}dS-
\int\limits_{\pd\Om}E(x,x_0)\frac{\pd u}{\pd \nu}dS
$$
- представляется в виде трех потенциалов.


Введем для каждого из потенциалов обозначения:
$$
P_0(x)=\int\limits_{\Om}\rho(\xi)\frac{1}{|x-\xi|^{n-2}}d\xi,\ n\ge 3\ \text{- объёмный потенциал.}
$$
$$
P_1(x)=\int\limits_{\Ga}\mu(\xi)\frac{1}{|x-\xi|^{n-2}}dS_\xi \ \text{- потенциал простого слоя.}
$$
$$
P_2(x)=\int\limits_{\Ga}\si(\xi)\frac{\pd}{\pd\nu_\xi}
\frac{1}{|x-\xi|^{n-2}}dS_\xi \ \text{- потенциал двойного слоя.}
$$
Здесь $E|x-\xi|=c_n \frac{1}{|x-\xi|^{n-2}}$.

\textbf{Теорема 2}\\
Пусть $\mu(x),\ \si(x)\in L_1(\Ga)$. Тогда $P_1(x),\ P_2(x)$ - гармонические функции
в $R^n/\Ga$\\
\textit{Доказательство}\\
Можно дифференцировать 2 раза $P_1,\ P_2$ под знаком интеграла.
$$
\fa x\in\Om_1;\ \Om,\ \pd \Om=\Ga,\ \Om_0=R^n/\ol{\Om}\
\Om_1\subs\subs\Om
$$
Возьмем $\Om_1,\ \rho(\Om_1,\Ga)\ge\alpha>0$.
В $P_1(x)$ можно дифференцировать под знаком интеграла. То же самое и для $\Om_0$
$$
\De_{x}P_1(x)=\int\limits_{\Ga}\mu(\xi)\De_x\frac{1}{|x-\xi|^{n-2}}dS_\xi=0
$$

Теперь посмотрим на потенциал двойного слоя
$$
\frac{\pd}{\pd\nu_\xi}\frac{1}{|x-\xi|^{n-2}}=
-(n-2)\frac{1}{|x-\xi|^{n-1}}\sum\limits_{k=1}^n\frac{\pd |x-\xi|}{\pd\xi_k}\nu_\xi^k=
-(n-2)\frac{1}{|x-\xi|^{n-1}}\sum\limits_{k=1}^n\frac{\xi_k-x_k}{|x-\xi|}\nu_\xi^k=
$$
$$
=-(n-2)\frac{\cos{(r,\nu_\xi)}}{r^{n-1}}
$$
Тогда
$$
P_2(x)=\int\limits_\Ga\si(\xi)\frac{\pd}{\pd\nu_\xi}\frac{1}{|x-\xi|^{n-2}}dS_\xi=
\int\limits_\Ga\si(\xi)
\sum\limits_{k=1}^n
\frac{\pd}{\pd\xi_k}\frac{1}{|x-\xi|^{n-2}}\nu_k^\xi dS_\xi
$$
Теперь по тем же соображениям можно дифференцировать и потенциал двойного слоя.


Посмотрим, как ведут себя $P_1,\ P_2$ на бесконечности.
Если $\xi\in\Ga,\ |x|>>1,\ \Rightarrow |x-\xi|\ge |x|-|\xi|\ge\frac{|x|}{2}$
Тогда для $P_1$:
$$
|P_1(x)|\le\frac{2^{n-2}}{|x|^{n-2}}\int\limits_\Ga |\mu(\xi)|dS_\xi=\frac{M_0}{|x|^{n-2}}
$$
Следовательно для $n\ge 3,\ P_1(x)\ra 0,\ x\ra\bes.$\\
По тем же соображениям
$$
|P_2(x)|\le\frac{M_1}{|x|^{n-2}}
$$
$$
|P_0(x)|\le\frac{M_2}{|x|^{n-2}}
$$

\begin{theorem}[3]
Пусть $\rho(x)\in C(\ol{\Om})$. Тогда $P_0(x)\in C^1(\R^n)$ и
$P_0(x)$ гармоническая функция при $x\in\R^n/\ol{\Om}$.
\end{theorem}
\begin{proof}
$x\in\Om\subs\subs\R^n/\ol{\Om}\Rightarrow \De_x P_0(x)=
\int\limits_\Om\rho(\xi)\De_x\frac{1}{|x-\xi|^{n-2}}d\xi=0$\\
$x\in\ol{\Om},\ Q_\ep^x$ - шар.
$$
P_0(x)\le
\int\limits_{\ol{\Om}\cap Q_\ep^x} |\rho(\xi)|\frac{1}{|x-\xi|^{n-2}}d\xi
+
\int\limits_{\ol{\Om}/\ol{Q}_\ep^x} |\rho(\xi)|\frac{1}{|x-\xi|^{n-2}}d\xi
$$
$$
\int\limits_{\ol{\Om}\cap Q_\ep^x} |\rho(\xi)|\frac{1}{|x-\xi|^{n-2}}d\xi
\le
\maxl{\ol{\Om}}|\rho(x)|\int\limits_{Q_\ep^x}\frac{1}{|x-\xi|^{n-2}}d\xi
$$
$$
\int\limits_{\ol{\Om}/Q_\ep^x} |\rho(\xi)|\frac{1}{|x-\xi|^{n-2}}d\xi
\le M_{1,\ep}
$$
$x\in\R^n/\ol{\Om}\Rightarrow |P_0(x)|\le M$\\
$\fa x_0\in\R^n$
$$
|P_0(x_0+h)-P_0(x_0)|\le
\int\limits_{\ol{\Om}\cap Q_\de^x} |\rho(\xi)|\frac{1}{|x_0+h-\xi|^{n-2}}d\xi
+
\int\limits_{\ol{\Om}\cap Q_\de^x} |\rho(\xi)|\frac{1}{|x_0-\xi|^{n-2}}d\xi
+
$$
$$
+
\int\limits_{\ol{\Om}/\ol{Q}_\de^x} |\rho(\xi)|
\left| \frac{1}{|x_0+h-\xi|^{n-2}}-\frac{1}{|x_0-\xi|^{n-2}}\right|d\xi
$$
$$
\int\limits_{\ol{\Om}/\ol{Q}_\de^x} |\rho(\xi)|
\frac{1}{|x_0-\xi|^{n-2}}d\xi\  \text{гладкая поверхность}.
$$
$\fa\ep>0 \ \exists h_0>0,\ \fa |h|<h_0$
$$
\int\limits_{\ol{\Om}/\ol{Q}_\de^x} |\rho(\xi)|
\left| \frac{1}{|x_0+h-\xi|^{n-2}}-\frac{1}{|x_0-\xi|^{n-2}}\right|d\xi
\le\frac{\ep}{3}
$$

$$
\int\limits_{\ol{\Om}\cap Q_\de^x} |\rho(\xi)|\frac{1}{|x_0-\xi|^{n-2}}d\xi
\le
\maxl{\ol{\Om}}|\rho(x)|\int\limits_{Q_\de^{x_0}}\frac{1}{|x-\xi|^{n-2}}d\xi=
C_1\int\limits_0^\de rdr=C_2\de^2<\frac{\ep}{3}
\ \text{при}\
\de\ll1
$$
$$
|\int\limits_{\ol{\Om}\cap Q_\de^x} |\rho(\xi)|\frac{1}{|x_0+h-\xi|^{n-2}}d\xi|
(|h|<\de/2)(|x_0+h-\xi|\le |x_0-\xi|+|h|<3\de/2)\le
$$
$$
\le\maxl{\ol{\Om}}|\rho(x)|C_2\int\limits_0^{3\de/2}rdr=C_3\de^2\le\ep/3
\ \text{при}\
\de\ll1
$$
Окончательно получаем, что при малых $h$
$$
|P_0(x_0+h)-P(x_0)|<\ep \ra P_0\in C(\R^n)
$$
\end{proof}

\begin{problem}
$$
\frac{\pd P_0}{\pd x_k}\in C(\R^n)
$$
$\De P_0(x)=0,\ x\in\R^n/\ol{\Om}$\\
$\De P_0(x)=-(n-2)\om_n\rho(x),\ x\in\Om',\ \om_n=|S_1^0|$\\
$(n=2:\ \De P_0=-2\pi\rho(x))$
\end{problem}

\begin{theorem}[4]
Пусть $\rho\in C^1(\ol(\Om))$. Тогда $P_0(x)\in C^2(\Om)$, при $x\in\Om$
$$
\De P_0(x)=
\begin{cases}
-(n-2)\om_n\rho(x),\ n\ge 3\\
-2\pi\rho(x),\ n=2
\end{cases}
$$
\end{theorem}
\begin{proof}
$n\ge 3$, вычислим
$$
\frac{\pd P_0(x)}{\pd x_k}=
\int\limits_\Om \rho(\xi)\frac{\pd}{\pd x_k}\frac{1}{|x-\xi|^{n-2}}d\xi=
-\int\limits_\Om \rho(\xi)\frac{\pd}{\pd \xi_k}\frac{1}{|x-\xi|^{n-2}}d\xi=
$$
$$
=
\liml{\ep\ra 0}
-\int\limits_{\Om/\Om_\ep^x}
\rho(\xi)\frac{\pd}{\pd \xi_k}\frac{1}{|x-\xi|^{n-2}}d\xi
$$
И получим то, что требуется доказать в условии.
\end{proof}


% ------------------------Лекция 20 (6)---------------------------


\section{Лекция 20}
Объёмный потенциал:       $P_0(x)=\int_\Om\rho(\xi)\frac{1}{|x-\xi|^{n-2}}d\xi$\\
Потенциал простого слоя:  $P_1(x)=\int_\Om\mu(\xi)\frac{1}{|x-\xi|^{n-2}}d\xi$\\
Потенциал двойного слоя:  $P_2(x)=\int_\Om\si(\xi)\frac{1}{|x-\xi|^{n-2}}d\xi$\\
$\De P_i(x)=0,\ x\in \R^n/\Ga,\ i=1,2;$
\subsection{Объёмный потенциал}
$\ \De P_0(x)=0,\ x\in \R^n/\ol{\Om},\
P_0(x)\in C^1(\R^n),\ \rho\in C(\ol{\Om})$
$$
\text{Уравнение Пуассона}\
\begin{cases}
\De P_0= -(n-2)\om_n\rho(x),\ x\in \Om,\ n\ge3;\\
\De P_0= -2\pi\rho(x),\ x\in \Om,\ n=2
\end{cases}
\eqno (20.1)
$$

\begin{theorem}[1]
Пусть $\rho\in C^1(\ol{\Om})$. Тогда $P_0\in C^2(\Om),\ P_0$ удовлетворяет уравнению (20.1)
\end{theorem}
\begin{proof}
$$
\frac{\pd P_0}{\pd x_k}=
\int_\Om\rho(\xi)\frac{\pd}{\pd x_k}\frac{1}{|x-\xi|^{n-2}}d\xi=
-\int_\Om\rho(\xi)\frac{\pd}{\pd \xi_k}\frac{1}{|x-\xi|^{n-2}}d\xi=
$$
$$
=\liml{\ep\ra 0}
\int\limits_{\Om/\ol{Q}_\ep^x}
\rho(\xi)\frac{\pd}{\pd \xi_k}\frac{1}{|x-\xi|^{n-2}}d\xi=
$$
$$
=
\liml{\ep\ra 0}
\left(
\int\limits_{\Om/\ol{Q}_\ep^x}
\frac{\pd\rho}{\pd \xi_k}\frac{1}{|x-\xi|^{n-2}}d\xi
-
\int\limits_{\pd(\Om/\ol{Q}_\ep^x)}
\rho(\xi)\frac{1}{|x-\xi|^{n-2}}\nu_\xi^k dS_\xi
\right)=
$$
$$
=
\int_\Om\frac{\pd\rho}{\pd \xi_k}\frac{1}{|x-\xi|^{n-2}}d\xi
-\int\limits_{\pd(\Om)}
\rho(\xi)\frac{1}{|x-\xi|^{n-2}}\nu_\xi^k dS_\xi
-\liml{\ep\ra 0}
\int\limits_{S_\ep^x}\rho(\xi)\ep^{2-n}\nu_\xi^k dS_\xi
$$
$$
|\int\limits_{S_\ep^x}\rho(\xi)\ep^{2-n}\nu_\xi^k dS_\xi|\le
M\ep^{2-n}\ep^{n-1}=M\ep
$$
$$
\Rightarrow -\liml{\ep\ra 0}
\int\limits_{S_\ep^x}\rho(\xi)\ep^{2-n}\nu_\xi^k dS_\xi=0
$$
Т.е получаем что $\frac{\pd P_0}{\pd x_k}\in C^1(\Om)$, поскольку первое слагаемое есть
объёмный потенциал с плотностью
$\frac{\pd \rho}{\pd \xi_k}\in C(\Om)$,
а второе\т потенциал простого слоя с $\rho(\xi)\nu_\xi^k\in L_1(\Ga)$
$$
\frac{\pd^2 P_0}{\pd x_k^2}(x)=
-\int_\Om
\frac{\pd \rho(\xi)}{\pd \xi_k}\frac{\pd}{\pd \xi_k}\frac{1}{|x-\xi|^{n-2}}d\xi
-\int_{\pd\Om}
\rho(\xi)\frac{\pd}{\pd x_k}\frac{1}{|x-\xi|^{n-2}}\nu_\xi^k dS-\xi=
$$
$$
=
-\liml{\ep\ra 0}
\int\limits_{\Om/\ol{Q}_\ep^x}
\frac{\pd\rho(\xi)}{\pd\xi_k}\frac{\pd}{\pd \xi_k}\frac{1}{|x-\xi|^{n-2}}d\xi
-\int_{\pd\Om}
\rho(\xi)\frac{\pd}{\pd x_k}\frac{1}{|x-\xi|^{n-2}}\nu_\xi^k dS_\xi=
$$
$$
=
\liml{\ep\ra 0}
\left\{
\int\limits_{\Om/\ol{Q}_\ep^x}
\rho(\xi)\frac{\pd^2}{\pd \xi_k^2}\frac{1}{|x-\xi|^{n-2}}d\xi
-
\int\limits_{\pd(\Om/\ol{Q}_\ep^x)}
\rho(\xi)\frac{\pd}{\pd \xi_k}\frac{1}{|x-\xi|^{n-2}}\nu_\xi^kdS_\xi
\right\}-
$$
$$
-\int_{\pd\Om}
\rho(\xi)\frac{\pd}{\pd x_k}\frac{1}{|x-\xi|^{n-2}}\nu_\xi^k dS_\xi
$$

Записываем эти равенства для всех $k=1,\dots,n$ и складываем
$$
\De P_0(x)=
\liml{\ep\ra 0}
\int\limits_{\Om/\ol{Q}_\ep^x}
\rho(\xi)\De_\xi\frac{1}{|x-\xi|^{n-2}}d\xi-
\sum\limits_{k=1}^n
\int\limits_{\pd\Om}
\rho(\xi)\frac{\pd}{\pd \xi_k}\frac{1}{|x-\xi|^{n-2}}\nu_\xi^kdS_\xi-
$$
$$
\liml{\ep\ra 0}
\int\limits_{S_\ep^x}
\rho(\xi)\frac{\pd}{\pd\nu_\xi}\frac{1}{|x-\xi|^{n-2}}dS_\xi
-\sum\limits_{k=1}^n
\int\limits_{\pd\Om}
\rho(\xi)\frac{\pd}{\pd x_k}\frac{1}{|x-\xi|^{n-2}}\nu_\xi^kdS_\xi-
$$
В $\Om/\ol{Q}_\ep^x,\ \frac{1}{|x-\xi|^{n-2}}$ гармоническая, тогда
первое слагаемое равно 0.\\
Второе и четвертое слагаемое отличаются знаком, т.к
$\frac{\pd}{x_k}\frac{1}{|x-\xi|^{n-2}}=-\frac{\pd}{\xi_k}\frac{1}{|x-\xi|^{n-2}}$
$$
\int\limits_{S_\ep^x}
\rho(\xi)\frac{\pd}{\pd\nu_\xi}\frac{1}{|x-\xi|^{n-2}}dS_\xi=
-\int\limits_{S_\ep^x}
\rho(\xi)\frac{\pd}{\pd r}\frac{1}{r^{n-2}}dS_\xi=
$$
$$
=(n-2)\ep^{1-n}
\int\limits_{S_\ep^x} \rho(\xi)dS_\xi=
(n-2)\om_n\frac{1}{\om_n\ep^{n-1}}
\int\limits_{S_\ep^x} \rho(\xi)dS_\xi
\ra(n-2)\om_n\rho(x),\ \ep\ra 0.
$$
Получаем (20.1).

Теорема доказана.
\end{proof}
\begin{problem}
Самостоятельно провести доказательство для $n=2$.
($P_0(x)=\int_\Om\rho(\xi)\ln{|x-\xi|}d\xi$)
\end{problem}

Подведем итоги.

Свойства $P_0(x)=\int_\Om\rho(\xi)\frac{1}{|x-\xi|^{n-2}}d\xi$\\
Пусть $\rho\in C(\bar{\Om}),$ тогда\\
а) $P_0(x)\in C^1(\R^n)$\\
б) $|P_0(x)|\le\frac{c}{|x|^{n-2}},\ |x|\ra\bes$\\
в) $\De P_0(x)=0,\ x\in\R^n/\bar{\Om}$\\
Пусть $\rho\in C^1(\bar{\Om}),$ тогда\\
г) $\De P_0(x)=-(n-2)\om_n\rho(x),\ x\in\Om$\\

\textit{Замечание.}
Используя эти свойства, можно вычислить $P_0$ не через $\int_\Om$,
а как решение г) со свойствами а)-в)

\subsection{Потенциал двойного слоя}

\begin{df}
Замкнутая поверхность $\Ga\subs\R^n$ называется \underline{поверхностью Ляпунова}, если:
а)$\fa x\in\Ga\ \exists$ нормаль $\nu_x$ к $\Ga$ в точке x. ($\nu_x$ -внешняя)\\
б)$\exists a>0,\ \alpha >0\ \fa x,\xi\in\Ga,\ \nu_x,\nu_\xi$ - нормали;
 $\theta$ - угол между ними $\Rightarrow \theta\le a|x-\xi|^\alpha$ \\
Отметим некоторые очевидные свойства:\\
1. $\Ga\in C^2\Rightarrow \Ga$ - поверхность Ляпунова. \\
2. $\Ga$ - поверхность Ляпунова $\Rightarrow \Ga \in C^2$\\
\end{df}

\textit{Упражнение.}

Доказать, что из условия б) следует условие Гёльдера для нормали:
$ |\nu_x-\nu_\xi|\le a|x-\xi|^\alpha$.

\textbf{Теорема 2.}\\
Пусть $\Ga$ - поверхность Ляпунова. Тогда $\exists d>0:\ \fa x\in\Ga$
любая прямая, параллельная $\nu_x$, пересекает $\Ga$ внутри $Q_d^x$ не более чем в одной точке.\\
\textit{Доказательство.}\\
Берем $d$ таким, чтобы $ad^\alpha<1$, предположим противное, т.е
пусть в точке $\xi_1$ прямая $l$ вышла из $\Om$ , а в $\xi_2$ - вошла.
Проведем касательную плоскость $\Pi$ в $\xi_2$. Прямая l
и внешняя нормаль $\nu_{\xi_2}$ будут лежать по разные стороны от $\Pi,\ \nu_{\xi_2}\bot\Pi$
$$
\widehat{(l,\nu_{\xi_2})}\Rightarrow
\widehat{(\nu_x,\nu_{\xi_2})}\ge \pi/2;\ |x-\xi_2|\le d,\ \text{т.к}\ \xi_2\in Q_d^x
$$
Тогда $\pi/2\le ad^\alpha$. Противоречие.\\

$\Ga'=\Ga\cap Q_d^x$ однозначно проектируется на касательную плоскость в x
$\Rightarrow \Ga'$ можно \\рассматривать в некоторой системе координат, как
график функции.


Фиксируем $x\in\Ga.\ S_d^x$ называется \underline{сферой Ляпунова.}\\
$\Ga':\ \xi_n=f(\xi_n,\dots,\xi_{n-1}),\ f$ задана на проекции $\Ga'$
на касательную плоскость.\\
При этом $f(0,\dots,0)=0;\ \nu_x=(0,\dots,0,1)$\\
Теперь будет некоторая муторная работа, целью которой будет следующее:\\
$\fa x,\xi\in\Ga\ |\cos{(r,\nu_\xi)}|\le cr^\alpha,\ \alpha$ из
определения поверхности Ляпунова.\\
\eqn{(P_2(x)=-(n-2)\int_\Ga\si(\xi)\frac{\cos{(r,\nu_\xi)}}{|x-\xi|^{n-1}}dS_\xi)}

$$
\nu_x=
\left(
-\frac{f_{\xi_1}(0)}{\sqrt{1+|\nabla_{\xi'} f|^2}},
-\frac{f_{\xi_2}(0)}{\sqrt{1+|\nabla_{\xi'} f|^2}},
\dots,
-\frac{f_{\xi_{n-1}}(0)}{\sqrt{1+|\nabla_{\xi'} f|^2}},
-\frac{1}{\sqrt{1+|\nabla_{\xi'} f|^2}}
\right)
$$
Здесь и далее $\xi'=(\xi_1,\dots,\xi_n)$.\\
Сравнивая это выражение с выражением раньше, получаем $f_{\xi_j}=0,\ j=1,\dots,n-1$\\
\textit{Упражнение.}\\
$$
\cos{(r,\nu_\xi)}=\sum\limits_{k=1}^n \cos{(r,\xi_k)}\cos{(\nu_\xi,\xi_k)}
$$
Выделим в этой сумме последнее слагаемое.\\
$$
\cos{(r,\xi_n)}\cos{(\nu_\xi,\xi_n)}+
\sum\limits_{k=1}^{n-1} \cos{(r,\xi_k)}\cos{(\nu_\xi,\xi_k)} \eqno (*)
$$
Хочется оценить $|\xi_n|=|f|$ и $|\cos{(\nu_\xi,\xi_k)}|,\ k=1,\dots,n-1$
$$
\nu_\xi=
\left(
-\frac{f_{\xi_1}(\xi)}{\sqrt{1+|\nabla_{\xi'} f|^2}},
-\frac{f_{\xi_2}(\xi)}{\sqrt{1+|\nabla_{\xi'} f|^2}},
\dots,
-\frac{f_{\xi_{n-1}}(\xi)}{\sqrt{1+|\nabla_{\xi'} f|^2}},
-\frac{1}{\sqrt{1+|\nabla_{\xi'} f|^2}}
\right)
$$
$$
\cos{(\nu_\xi,\xi_k)}=
-\frac{f_{\xi_k}(\xi)}{\sqrt{1+|\nabla_{\xi'} f|^2}}
,\ k=1,\dots,n-1;
$$
$$
\cos\theta=
\cos{(\nu_\xi,\nu_x)}=
\cos{(\nu_\xi,\xi_n)}=
-\frac{1}{\sqrt{1+|\nabla_{\xi'} f|^2}}
$$
$\theta$ из определения поверхности Ляпунова.
$\cos{\theta}\ge 1-\frac{\theta^2}{2}$ (из математического анализа),
$\theta\le ar^\alpha\le ad^\alpha < 1\Rightarrow \cos\theta\ge 1/2$
$$
-\frac{1}{\sqrt{1+|\nabla_{\xi'} f|^2}}
\ge
1-\frac{a^2r^{2\alpha}}{2}
$$
$$
\sqrt{1+|\nabla_{\xi'} f|^2}\le
\frac{1}{1-\frac{a^2r^{2\alpha}}{2}}=
1+\frac{a^2r^{2\alpha}}{2-a^2r^{2\alpha}}
\ge(a^2r^{2\alpha}<1)1+a^2r^{2\alpha}
$$
$$
|\nabla_{\xi'} f|^2\le
2a^2r^{2\alpha}+a^4r^{4\alpha}
\le
3a^2r^{2\alpha}
$$
Окончательно получаем, что
$$
|\nabla_{\xi'} f|\le
\sqrt3ar^{\alpha}
$$
$$
|f_{\xi_k}(\xi)|\le
|\nabla_{\xi'} f|\le
\sqrt3ar^{\alpha},\ k=1,...,n-1
$$
В (*) последнее слагаемые оцениваются
$$
|\cos{(r,\xi_k)}\cos{(\nu_\xi,\xi_k)}|
\le|\cos{(\nu_\xi,\xi_k)}|\le\sqrt3ar^{\alpha}
$$
$r^2=\xi_n^2+\rho^2,\ \rho^2=\sum\limits_{k=1}^{n-1}\xi_k^2$
$$
\left|
\frac{\pd f}{\pd\rho}
\right|
=
\left|
\sum\limits_{k=1}^{n-1}\frac{\pd f}{\pd\xi_k}\frac{\xi_k}{\rho}
\right|
\le \sum\limits_{k=1}^{n-1}
\left|\frac{\pd f}{\pd\xi_k}\right|\le\sqrt3anr^\alpha\le\sqrt3and^\alpha=C_1
$$
$$
f(0)=0\Rightarrow |f(\xi)|\le
\int\limits_0^\rho\left|\frac{\pd f}{\pd\rho'}\right|d\rho'
\le C_1\rho\Rightarrow
|\xi_n|=|f(\xi)|\le C_1\rho
$$
Отсюда
$$
\rho^2<r^2\le C_1\rho^2+\rho^2=C_2\rho^2 \Rightarrow
\left|\frac{\pd f}{\pd\rho}\right|\le C_3\rho^\alpha
$$
$$
|\xi_n|=|f(\xi)|\le C_4\rho^{\alpha+1}\Rightarrow
|\xi_n|=|f(\xi)|\le C_4r^{\alpha+1}
$$
Из (*):
$$
|\cos{r,\nu_\xi}|\le\frac{|\xi_n|}{2}+C_0r^\alpha\le
\bar{C}r^\alpha
$$
что и требовалось.







%----------------------Лекция 21--------------------------------





 \section{Лекция 21}
 \underline{\textit{\textbf{Теорема 1.}}} Пусть $\Ga$ -
 замкнутая поверхность Ляпунова, $\si(\xi)\equiv 1$. Тогда
 $\fa x\in\R^n,n\ge 3$
 $$
 \int_{\Ga}\left|\frac{\pd}{\pd
 \xi}\frac{1}{r^{n-2}}\right|ds_{\xi}=(n-2)\int_{\Ga}\frac{|\cos(r,\nu_{\xi})|}{r^{n-1}}ds_{\xi}\le
 M<\bes
 $$
 Доказательство.\\
 1. Пусть $\rho(x,\Ga)\ge \frac{d}{2}$. Тогда
 $r=|x-\xi|\ge\frac{d}{2}$, откуда
 $$
 \int_{\Ga}\frac{|\cos(r,\nu_{\xi})|}{r^{n-1}}ds_{\xi}\le\frac{2^{n-1}}{d^{n-1}}|\Ga|\le\bes
 $$
 2. Пусть $\rho(x,\Ga)< \frac{d}{2}$.\\
 2a. Пусть $x\in \Ga$, тогда
 $$
\int_{\Ga}\frac{|\cos(r,\nu_{\xi})|}{r^{n-1}}ds_{\xi}=\int_{\Ga'}\frac{|\cos(r,\nu_{\xi})|}{r^{n-1}}ds_{\xi}+
\int_{\Ga''}\frac{|\cos(r,\nu_{\xi})|}{r^{n-1}}ds_{\xi}
 $$
,где $\Ga'=\Ga\cap Q^x_d$,
$\Ga''=\Ga\wo\Ga'$.
$$
\int_{\Ga''}\frac{|\cos(r,\nu_{\xi})|}{r^{n-1}}ds_{\xi}\le
\frac{1}{d^{n-1}}|\Ga|
$$
$$
\int_{\Ga'}\frac{|\cos(r,\nu_{\xi})|}{r^{n-1}}ds_{\xi}\le\int_{\Ga'}\frac{cr^{\alpha}}{r^{n-1}}ds_{\xi}\le
c_2\int_{\Dc}\frac{\rho^{\alpha}}{\rho^{n-1}}d\xi_1\ldots
d\xi_{n-1}=c_3\int_0^{d_1}\rho^{\alpha+1-n}\rho^{n-2}d\rho=
$$
$$
=c_3\int_0^{d_1}\rho^{\alpha-1}d\rho\le c_4<\bes
$$
К последним неравенствам мы переходили, заменяя интегрирование по
$\Ga'$ интегрированием по проекции $\Ga'$ на касательную
плоскость, $\cos(r,\nu_{\xi})ds_{\xi}=d\xi_1\ldots
d\xi_{n-1};\quad \rho^2=\suml{k=1}{n-1}\xi^2_k;\quad$ $\rho^2\le
r^2=\rho^2+\xi_n^2;\quad
|\xi_n|\le c_1\rho^{\alpha+1}$\\
2б. $x$ не принадлежит $\Ga$, $|x-x_0|<\frac{d}{2}$.\\
В $\{\xi_i\}\quad$ $x$ имеет координаты
$(0,\ldots,0,\pm\de),\de>0$.
$$
\cos(r,\nu_{\xi})=\sumkun \cos(r,\xi_k)\cos(\nu_{\xi},\xi_k)\quad
\mbox{- это упражнение из предыдущей лекции}
$$
$$
\cos(r,\nu_{\xi})=\suml{k=1}{n-1}\cos(r,\xi_k)\cos(\nu_{\xi},\xi_k)+\cos(r,\xi_n)\cos(\nu_{\xi},\xi_n)
$$
$$
|\cos(\nu_{\xi},\xi_k)|\le Cr_0^{\alpha},\quad r_0=|x_0-\xi|,
\quad k=1,\ldots,n-1,\quad r_0^2=\rho^2+\xi_n^2
$$
$$
|\cos(r,\xi_n)|\le\frac{|\xi_n\pm\de|}{r}
$$
$$
\cos(r,\nu_{\xi})\le Cr_0^{\alpha}+\frac{|\xi_n\pm\de|}{r}
$$
Теперь разбиваем наш интеграл на 2 других интеграла:
$$
\int_{\Ga}\frac{|\cos(r,\nu_{\xi})|}{r^{n-1}}ds_{\xi}=\int_{\Ga'}\frac{|\cos(r,\nu_{\xi})|}{r^{n-1}}ds_{\xi}+
\int_{\Ga''}\frac{|\cos(r,\nu_{\xi})|}{r^{n-1}}ds_{\xi}
$$
$\int_{\Ga''}\frac{|\cos(r,\nu_{\xi})|}{r^{n-1}}ds_{\xi}$
оценивается так же, как в пункте 2а. Оценим
$\int_{\Ga'}\frac{|\cos(r,\nu_{\xi})|}{r^{n-1}}ds_{\xi}$.
$$
r^2=\sumkun (x_k-\xi_k)^2=\suml{k=1}{n-1}\xi_k^2+(\xi_n\pm\de)^2=\rho^2+(\xi_n\pm\de)^2
$$
$$
\pm\xi_n\de\ge-\frac{\de^2}{2}-2\xi^2\quad
r^2\ge\rho^2-\xi_n^2+\frac{\de^2}{2}, \mbox{ и еще }
|\xi_n|\le c_1\rho^{\alpha+1}\le c_1d^{\alpha}\rho
$$
Уменьшим, если надо, $d$, чтобы
$c_1d^{\alpha}\le\frac{1}{\sqrt{2}}$, тогда
$|\xi_n|\le\frac{\rho}{\sqrt{2}}$. Итак,
$r^2\ge\frac{\rho^2+\de^2}{2}$.
$$
\int_{\Ga'}\frac{|\cos(r,\nu_{\xi})|}{r^{n-1}}ds_{\xi}\le
C_0\left(\int_{\Ga'}\frac{r_0^{\alpha}}{(\rho^2+\de^2)^{\frac{n-1}{2}}}ds_{\xi}+
\int_{\Ga'}\frac{\rho^{\alpha+1}}{(\rho^2+\de^2)^{\frac{n}{2}}}ds_{\xi}
+\de\int_{\Ga'}\frac{ds_{\xi}}{(\rho^2+\de^2)^{\frac{n}{2}}}\right)
$$
Первый интеграл обозначим $I_1$, второй $I_2$, третий $I_3.\quad$$
r_0^2=\rho^2+\xi_n^2\le \wt{c}\rho^2$.
$$
I_1\le
K_1\int_{\Dc}\frac{\rho^{\alpha}}{(\rho^2+\de^2)^{\frac{n-1}{2}}}d\xi_1\ldots
d\xi_{n-1}\le
K_2\int_0^{d_1}\rho^{\alpha+1-n}\rho^{n-2}d\rho<\bes
$$
Аналогично оценивается $I_2$.
$$
I_3\le
K\de\int_0^{d_1}\frac{\rho^{n-2}d\rho}{(\rho^2+\de^2)^{\frac{n}{2}}}=
K\de\int_0^{d_1}\frac{\rho^{n-2}d\rho}{\rho^n(1+\frac{\de^2}{\rho^2})^{\frac{n}{2}}}=
K\int_0^{d_1}\frac{-d\frac{\de}{\rho}}{(1+\frac{\de^2}{\rho^2})^{\frac{n}{2}}}=
K\int_{\de/d_1}^{\bes}\frac{dt}{(1+t^2)^{\frac{n}{2}}}\le\bes
$$
Ч.т.д.\\
$\Sigma$\т часть поверхности, на которой задано положительное
направление нормали, $x$ не принадлежит $\Sigma$. Предполагаем
$\xi\in\Sigma \quad\cos(\overrightarrow{x\xi},\nu_{\xi})\ge 0$.
Соединим теперь $x$  с каждой точкой $\Sigma$. Полученную
коническую границу обозначим $K$. $\widetilde{\pd
K}=K\cup\Sigma$. $Q^x_R\cap\Sigma=\emptyset$. $K$ высечет на
$S^x_R$ некоторую поверхность, обозначим ее $\si_R\subs
S^x_R$.
$$
\om_x(\Sigma)=\frac{|\si_R|}{R^{n-1}}=|\si_1|
$$
В случае $cos(\overrightarrow{x\xi},\nu_{\xi})<0$ считаем
$\om_x(\Sigma)=-\frac{|\si_R|}{R^{n-1}}$.\\
В общем случае мы разбиваем $\Sigma$ на соответствующие части.\\
\underline{\textit{\textbf{Теорема 2.}}}
$$
\om_x(\Sigma)=-\frac{1}{n-2}\int_{\Sigma}\frac{\pd}{\pd\nu_{\xi}}\frac{1}{|x-\xi|^{n-2}}ds_{\xi}\quad
(n\ge 3)
$$
(т.е. $\om_x$ - потенциал двойного слоя)\\
Доказательство.\\
$\Om_{\ep}=\widetilde{K}\wo
\ol{Q^x_{\ep}};\quad$ $K_{\ep}=K\wo
\ol{Q^x_{\ep}}$.\\
В $\Om_{\ep}\quad$ $\frac{1}{|x-\xi|^{n-2}}$ -
гармоническая, тогда запишем
$\De_{\xi}\frac{1}{|x-\xi|^{n-2}}=0$ в $\Om_{\ep}$.
$$
0=\int_{\Om_{\ep}}\De_{\xi}\frac{1}{|x-\xi|^{n-2}}d\xi=
int_{\pd\Om_{\ep}}\frac{\pd}{\pd\nu_{\xi}}\frac{1}{|x-\xi|^{n-2}}ds_{\xi}=
$$
$$
=\int_{\Sigma}\frac{\pd}{\pd\nu_{\xi}}\frac{1}{|x-\xi|^{n-2}}ds_{\xi}+
\int_{K_{\ep}}\frac{\pd}{\pd\nu_{\xi}}\frac{1}{|x-\xi|^{n-2}}ds_{\xi}+
\int_{\si_{\ep}}\frac{\pd}{\pd\nu_{\xi}}\frac{1}{|x-\xi|^{n-2}}ds_{\xi}
$$
$$
\frac{\pd}{\pd\nu_{\xi}}\frac{1}{|x-\xi|^{n-2}}|_{\xi\in\si_{\ep}}=
-\frac{\pd}{\pd
r}\frac{1}{|x-\xi|^{n-2}}|_{\xi\in\si_{\ep}}=
=(n-2)\ep^{1-n}
$$
$$
\int_{\si_{\ep}}\frac{\pd}{\pd\nu_{\xi}}\frac{1}{|x-\xi|^{n-2}}ds_{\xi}=
=(n-2)\ep^{1-n}|\si_{\ep}|=(n-2)\om_x(\Sigma)
$$
$$
\int_{K_{\ep}}\frac{\pd}{\pd\nu_{\xi}}\frac{1}{|x-\xi|^{n-2}}ds_{\xi}=
-(n-2)\int_{K_{\ep}}\frac{\cos(r,\nu_{\xi})}{r^{n-1}}ds_{\xi}=0,\quad
\mbox{ т.к. } \cos(r,\nu_{\xi})=0
$$
Итак,
$$
0=\int_{\Sigma}\frac{\pd}{\pd\nu_{\xi}}\frac{1}{|x-\xi|^{n-2}}ds_{\xi}+(n-2)\om_x(\Sigma)
$$
Ч.т.д.\\
\textit{Следствие.} $\Ga$ - замкнутая поверхность Ляпунова,
ограничивающая область $\Om$.\\
Тогда
$\int_{\Ga}\frac{\pd}{\pd\nu_{\xi}}\frac{1}{|x-\xi|^{n-2}}ds_{\xi}$
может принимать следующие значения:
$$
\left\{
\begin{aligned}
-\om_n(n-2),\quad x\in\Om\\
0,\quad x\in\R^n\wo\ol{\Om}\\
-\frac{\om_n(n-2)}{2},\quad x\in\Ga\\
\end{aligned}
\right.
$$
, где $n\ge 3$, $\om_n=|S_1|$.\\
Доказательство.\\
1)
$$ x\in\Om\Rightarrow\om_x(\Ga)=\om_n\Rightarrow
\int_{\Ga}\frac{\pd}{\pd\nu_{\xi}}\frac{1}{|x-\xi|^{n-2}}ds_{\xi}=-\om_n(n-2)
$$
2)
$$
x\in\R^n\wo\ol{\Om}\Rightarrow\frac{1}{|x-\xi|^{n-2}}\mbox{
- гармоническая по }\xi\in\Om
$$
3)\\
$x\in\Ga.\quad$ $\pi_x$ - касательная плоскость к $\Ga$ в
точке $x$. Рассмотрим $Q^x_{\ep}, \ep\ll d$.
$\Ga_{\ep}=\Ga\cap Q^x_{\ep}$.
$\widetilde{S^x_{\ep}}=S^x_{\ep}\cap
\Om.\quad$$\widehat{S^x_{\ep}}$ - полусфера.
$\widehat{S^x_{\ep}}=\widetilde{S^x_{\ep}}+B_{\ep}.\quad$
$\Om_{\ep}=\Om\wo
\ol{Q^x_{\ep}}$.\\
$\frac{1}{|x-\xi|^{n-2}}$
гармоническая по $\xi$ в $\Om$.
$$
\int_{\Ga\wo\Ga_{\ep}}\frac{\pd}{\pd\nu_{\xi}}\frac{1}{|x-\xi|^{n-2}}ds_{\xi}=
-\int_{\widetilde{S^x_{\ep}}}\frac{\pd}{\pd\nu_{\xi}}\frac{1}{|x-\xi|^{n-2}}ds_{\xi}=
$$
$$
=-\int_{\widehat{S^x_{\ep}}}\frac{\pd}{\pd\nu_{\xi}}\frac{1}{|x-\xi|^{n-2}}ds_{\xi}+
\int_{B_{\ep}}\frac{\pd}{\pd\nu_{\xi}}\frac{1}{|x-\xi|^{n-2}}ds_{\xi}
$$
$$
\int_{B_{\ep}}\frac{\pd}{\pd\nu_{\xi}}\frac{1}{|x-\xi|^{n-2}}ds_{\xi}=
-(n-2)\int_{B_{\ep}}\frac{\cos(r,\nu_{\xi})}{r^{n-1}}ds_{\xi}
$$
$$
\int_{B_{\ep}}\frac{|\cos(r,\nu_{\xi})|}{r^{n-1}}ds_{\xi}\le
K_0\ep^{\alpha+1-n}|B_{\ep}|\ra 0 \mbox{ при }
\ep\ra 0
$$
Кроме того,
$$
\int_{\Ga\wo\Ga_{\ep}}\frac{\pd}{\pd\nu_{\xi}}\frac{1}{|x-\xi|^{n-2}}ds_{\xi}\ra
\int_{\Ga}\frac{\pd}{\pd\nu_{\xi}}\frac{1}{|x-\xi|^{n-2}}ds_{\xi}
\mbox{ при } \ep\ra 0
$$
и
$$
-\int_{\widehat{S^x_{\ep}}}\frac{\pd}{\pd\nu_{\xi}}\frac{1}{|x-\xi|^{n-2}}ds_{\xi}\ra
-\frac{\om_n(n-2)}{2} \mbox{ при } \ep\ra 0
$$
Ч.т.д.\\
\subsection{Теорема о скачке потенциала двойного слоя}
$x_0\in\Ga\quad$ $P_2^+(x_0)=\liml{x\ra
x_0,x\in\Om}P_2(x),\quad$ $P_2^-(x_0)=\liml{x\ra
x_0,x\in\R^n\wo\Om}P_2(x),\quad$
$\ol{P_2(x_0)}$ - прямое значение.\\
Пусть $\Ga$ - замкнутая поверхность Ляпунова,
$x_0\in\Ga,\quad$ $\si(x)\in C(\Ga)$.\\
Тогда
$$
P_2^+(x_0)=-\frac{\om_n(n-2)}{2}\si(x_0)+\ol{P_2(x_0)}
$$
$$
P_2^-(x_0)=\frac{\om_n(n-2)}{2}\si(x_0)+\ol{P_2(x_0)}
$$

%------------------------------------- Лекция 22 ----------------------------------------------

\section{Лекция 22}
Пусть $\Ga$ -- замкнутая поверхность Ляпунова, тогда для
потенциала двойного слоя с $\si(x) \equiv 1$ мы получили
результат
$$
P(x)=\begin{cases}
0,&\text{$x\in\R^n\backslash\bar\Om$}\\
-\frac{(n-2)w_n}2,&\text{$x\in \Ga$}\\
-(n-2)w_n,&\text{$x\in \Om$}\\
\end {cases}
$$


\begin{theorem}
Пусть $\Ga$ -- замкнутая поверхность Ляпунова,$\si(x)\in
C(\Ga)$, тогда $\fa x_0 \in \Ga$:
$$
P_2^+(x_0)=\liml{x\ra x_0,x \in \Om}
P_2(x)=-\frac{(n-2)w_n}2 \si (x_0)+\ol P_2(x_0)
$$
$$
P_2^-(x_0)=\liml{x\ra x_0,x \in \R^n\backslash \bar
\Om} P_2(x) =\frac{(n-2)w_n}2 \si (x_0)+\ol P_2(x_0)
$$
Где $\ol P_2(x_0)$ -- прямое значение в точке $x_0$.
\end{theorem}


\textbf{Доказательство:}
$$
P_2(x)=\int_\Ga \si (\xi)
\frac\pd{\pd\nu_\xi}\frac
1{|x-\xi|^{n-2}}\,dS_\xi=\int_\Ga (\si (\xi)-\si
(x_0))\frac\pd{\pd\nu_\xi}\frac
1{|x-\xi|^{n-2}}\,dS_\xi+ \si (x_0) \int_\Ga
\frac\pd{\pd\nu_\xi}\frac 1{|x-\xi|^{n-2}}\,dS_\xi
$$

Первое слагаемое назовем $W_0(x)$,а второе $W(x)$. Если мы докажем
непрерывность $W_0(x)$ в точке $x_0$, то тем самым мы докажем
теорему.В самом деле, если $W_0^+(x_0)=W_0^-(x_0)=\bar W_0(x_0)$,
то $\bar P_2(x_0)=\bar W_0(x_0)-\si(x_0)\frac{(n-2)w_n}2$,
тогда $P_2^+(x_0)=\bar W_0(x_0)-\si(x_0)(n-2)w_n = \bar
P_2(x_0)-\si(x_0)\frac{(n-2)w_n}2$  и аналогично для
$P_2^-(x_0)$.

Докажем непрерывность. Выбросим из $\Ga$  маленькую шаровую
окрестность $\Ga '$, тогда $\Ga=\Ga'\cup\Ga''$.
$$
|W_0(x)-W_0(x_0)|=|W'_0(x)+W''_0(x)-W'_0(x_0)-W''_0(x_0)|\le
|W'_0(x)|+|W'_0(x_0)|+|W''_0(x)-W''_0(x_0)|
$$
Здесь $W'_0(x)$ - интеграл по $\Ga '$, а $W''_0(x)$ - интеграл
по $\Ga ''$.

$W'_0(x)$ в точке $x_0$ дифференцируема, поэтому
$|W''_0(x)-W''_0(x_0)|\ra 0$ при $x\ra x_0$.


Функция $\si(x)$ непрерывна, поэтому $\fa\ep>0\quad
\exists \eta_0\, \fa \eta<\eta_0\,
\fa\xi\in\Ga:\,|\xi-x_0|<\eta\quad
|\si(\xi)-\si(x_0)|<\ep$ А значит можно оценить
$$
|W_0(x)|\le \ep \int_\Ga
\frac\pd{\pd\nu_\xi}\frac
1{|x-\xi|^{n-2}}\,dS\le\ep\cdot Const
$$
И уменьшая $\ep$, можно и первые слагаемые сделать сколь
угодно малыми. Непрерывность доказана.\\

\subsection{Потенциал простого слоя}
Напоминаем: Это
$$
P_1(x)=\int_\Ga \mu(\xi) \frac 1{|x-\xi|^{n-2}}\,dS_\xi
$$
Где $\mu(x)\in C(\Ga)$, $\Ga$ -- замкнутая поверхность
Ляпунова.

\begin{theorem}
Потенциал простого слоя непрерывен в $\R^n$.
\end{theorem}

\textbf{Доказательство:} Непрерывность может нарушаться только при
переходе через $\Ga$. Вначале проверим,что потенциал определен
в точках поверхности. Заметим, что $\fa x_0\in\Ga$:
$$
|P_1(x_0)|\le\maxl\Ga |\mu(x)| \int_\Ga \frac
1{|x_0-\xi|^{n-2}}\,dS_\xi = M \Bigl( \int_{\Ga'}\frac
1{|x_0-\xi|^{n-2}}\,dS_\xi+\int_{\Ga''}\frac
1{|x_0-\xi|^{n-2}}\,dS_\xi\Bigr)
$$
Где $M=\maxl\Ga |\mu(x)|$, $\Ga'$-пересечение $\Ga$ с
маленьким шариком с центром в $x_0$, а $\Ga''$ -- оставшаяся
часть $\Ga$. Второй интеграл конечен,а конечность первого
проверяется переходом к системе координат $r^2=\rho^2+\xi_n^2$:
$$
\int_{\Ga'}\frac 1{|x_0-\xi|^{n-2}}\,dS_\xi=\int_{\Ga'}\frac
{dS_\xi}{r^{n-2}}=\int_{D(x_0)}\frac{d\xi_1\ldots
d\xi_{n-1}}{\rho^{n-2}\cos (\nu_\xi,\xi_n)}\le C\int_0^d
\rho^{2-n} \rho^{n-2}\,d\rho=Cd
$$


Непрерывность доказывается аналогично предыдущей теореме:
$$
 |P_1(x)-P_1(y)|\le |P'_1(x)|+|P'_1(y)|+|P''_1(x)-P''_1(y)|
$$
Где, как обычно, $P'_1(x)$ и $P''_1(x)$ -- соответственно
интегралы по $\Ga'=\Ga\cap Q^x_\eta, \quad \eta\ll 1$ и по
$\Ga''=\Ga\backslash\Ga'$. Последнее слагаемое стремится
к нулю ввиду дифференцируемости $P''$,а первые в силу оценки
$$
|P'_1(x)|\le M\int_{\Ga'} \frac{dS_\xi}{|x-\xi|^{n-2}} \le
M\eta
$$
и выбора достаточно маленького $\eta$\\

Покажем, что $\fa x\in \R^n$ определена (конечна)
нормальная производная
$$
\frac{\pd P_1}{\pd \nu_x}=\int_\Ga \mu(\xi)
\frac\pd{\pd\nu_\xi}\frac 1{|x-\xi|^{n-2}}\,dS_\xi
$$
Достаточно проверить существование для $x\in\Ga$. Аналогично
выкладкам для потенциала двойного слоя,
$$\frac\pd{\pd\nu_\xi}\frac 1{|x-\xi|^{n-2}}=-\frac{|\cos (r,\nu_\xi)|}{r^{n-1}},$$
поэтому
$$
\bigl| \int_\Ga \mu(\xi) \frac\pd{\pd\nu_\xi}\frac
1{|x-\xi|^{n-2}}\,dS_\xi\Bigr|\le M\Bigl( \int_{\Ga'} \frac
{|\cos(r,\nu_x)|}{r^{n-1}}\, dS_\xi+\int_{\Ga''} \frac
{|\cos(r,\nu_x)|}{r^{n-1}}\, dS_\xi \Bigl)
$$
Второй интеграл конечен, а первый, в силу сделанной ранее оценки,
$$
\int_{\Ga'} \frac {|\cos(r,\nu_x)|}{r^{n-1}}\, dS_\xi\le C_1
\int_{D(x)} \frac {r^\alpha}{r^{n-1}}\,d\xi_1\ldots d\xi_n\le C_2
\int_0^d \rho^{\alpha-n+1}\rho^{n-2}\,d\rho=C_3 d^\alpha
$$
следовательно, нормальная производная определена $\fa x\in
\R^n$.


%-------------------------------------------- Лекция 23 ---------------------------------------------

\section {Лекция 23.}
\begin{theorem}[О скачке нормальной производной потенциала простого
слоя]
 Пусть $\Ga$ -- замкнутая поверхность
Ляпунова,$\mu(x)\in C(\Ga)$, тогда $\fa x_0 \in \Ga$:
$$
\Bigl(\frac{\pd P_1}{\pd
\nu_x}\Bigr)^+(x_0)=\liml{\Om \ni x\ra x_0,x\in \nu_x}
\Bigl(\frac{\pd P_1}{\pd \nu_x}\Bigr)=\frac{(n-2)w_n}2
\mu (x_0)+\ol{\Bigl(\frac{\pd P_1}{\pd
\nu_x}\Bigr)}
$$
$$
\Bigl(\frac{\pd P_1}{\pd
\nu_x}\Bigr)^-(x_0)=\liml{\R^n\backslash\bar\Om \ni
x\ra x_0,x\in \nu_x} \Bigl(\frac{\pd P_1}{\pd
\nu_x}\Bigr)=-\frac{(n-2)w_n}2 \mu
(x_0)+\ol{\Bigl(\frac{\pd P_1}{\pd \nu_x}\Bigr)}
$$
Где $\ol{\Bigl(\frac{\pd P_1}{\pd \nu_x}\Bigr)}$
-- прямое значение в точке $x_0$.
\end{theorem}

\textbf{Доказательство:} \\

$$
\frac{\pd P_1}{\pd \nu_x}(x)=\int_\Ga \Bigl[\mu(\xi)
\frac\pd{\pd\nu_x}\frac
1{|x-\xi|^{n-2}}+\frac\pd{\pd\nu_\xi}\frac
1{|x-\xi|^{n-2}}\Bigr]\,dS_\xi-\int_\Ga \mu(\xi)
\frac\pd{\pd\nu_\xi}\frac 1{|x-\xi|^{n-2}}\,dS_\xi =
W_0(x)+W(x)
$$
Достаточно доказать непрерывность $W_0(x)$.В самом деле, пусть
$W_0^+(x)=W_0^-(x)=\ol{ W_0(x)}$. функция $W(x)$ уже была
фактически вычислена в предыдущей теореме, поэтому
$$
\ol{\frac{\pd P_1}{\pd \nu_x}}(x_0)=\ol{
W_0(x_0)}-\ol{P_2(x_0)}
$$
$$
\Bigl(\frac{\pd P_1}{\pd \nu_x}\Bigr)^+(x_0)=\ol{
W_0(x_0)}+\mu(x_0)\frac{(n-2)w_n}2-\ol
{P_2(x_0)}=\ol{\frac{\pd P_1}{\pd
\nu_x}}(x_0)+\mu(x_0)\frac{(n-2)w_n}2
$$
Аналогично,
$$
\Bigl(\frac{\pd P_1}{\pd \nu_x}\Bigr)^-(x_0)=\ol{
W_0(x_0)}-\mu(x_0)\frac{(n-2)w_n}2-\ol
{P_2(x_0)}=\ol{\frac{\pd P_1}{\pd
\nu_x}}(x_0)-\mu(x_0)\frac{(n-2)w_n}2
$$

Докажем непрерывность $W_0(x)$.Опять рассмотрим разбиение
$\Ga=\Ga'\cup\Ga''$,где $\Ga'=\Ga\cap Q^x_\eta,
\quad \eta\ll d$, $d$ -- радиус Ляпунова нашей поверхности.
Интеграл распадется в сумму двух, причем интеграл по $\Ga''$
непрерывен в точке $x_0$ и,значит, стремится к нулю при
$x\ra x_0$.Оценим интеграл по $\Ga'$.

 Поскольку
$$
\frac\pd{\pd\nu_x}\frac 1{|x-\xi|^{n-2}}=(n-1)
\frac{\cos (r,\nu_x)}{|x-\xi|^{n-1}}
$$

$$
\frac\pd{\pd\nu_x}\frac 1{|x-\xi|^{n-2}}=(n-1)
\frac{\cos (r,\nu_\xi)}{|x-\xi|^{n-1}}
$$
То справедливо равенство :
$$
\Bigl|\frac\pd{\pd\nu_x}\frac
1{|x-\xi|^{n-2}}+\frac\pd{\pd\nu_x}\frac
1{|x-\xi|^{n-2}}\Bigr|=\frac{(n-2)}{r^{n-1}}\Bigl|\cos
(r,\nu_x)-\cos (r,\nu_\xi)\Bigl|
$$

Введем локальную систему координат $\xi_1,\ldots \xi_n$ с центром
в $x_0$. Тогда
$$
\cos (r,\nu_x)=\cos (r,\xi_n)
$$
$$
\cos(r,\nu_\xi)=\suml{k=1}{n-1} \cos(r,\xi_k)
\cos(\nu_\xi,\xi_k)+\cos(r,\xi_n) \cos(\nu_\xi,\xi_n)
$$
Поэтому
$$
\frac{(n-2)}{r^{n-1}}\Bigl|\cos (r,\nu_x)-\cos
(r,\nu_\xi)\Bigl|\le \frac{(n-2)}{r^{n-1}}\Bigl (\suml{k=1}{n-1}
| \cos (\nu_\xi,\xi_k)|+|1-\cos(\nu_\xi,\xi_n)|\Bigr)
$$
Из неравенства $|1-cos \theta|\le \frac {\theta^2}2$ и свойства
поверхности Ляпунова $|\cos (\nu_\xi,\xi_k)|\le c r_0^\alpha$:
$$
\Bigl | \int_{\Ga'_\eta} \mu(\xi)
\Bigl[\frac\pd{\pd\nu_x}\frac
1{|x-\xi|^{n-2}}+\frac\pd{\pd\nu_\xi}\frac
1{|x-\xi|^{n-2}}\Bigr]\,dS_\xi\Bigr |\le M_1 \int_{\Ga'_\eta}
\frac{r_0^\alpha}{r^{n-1}}\,dS_\xi
$$
Поскольку  $r_0^2=\sumkun \xi_k^2=\rho^2+\xi_n^2\le C_1
\rho^2$ (Так как $|\xi_n|\le C\rho^{\alpha+1}$), то
$$
M_1 \int_{\Ga'_\eta} \frac{r_0^\alpha}{r^{n-1}}\,dS_\xi\le M_2
\int_{\Ga'_\eta} \rho ^{\alpha-n+1}\,dS_\xi=M_3\int_0^\eta \rho
^{\alpha-n+1} \rho ^{n-2}\,d\rho=M_4 \eta^\alpha
$$
Так что выбрав подходящее $\eta$, можно сделать сколь угодно малым
и этот интеграл. Это доказывает непрерывность, а значит и всю
теорему.\\

\subsection{Постановка краевых задач}
\textbf{$D_i$ (внутренняя задача Дирихле):}\\
Пусть $\Om$ - ограниченная область,$\pd\Om=\Ga$ -
поверхность Ляпунова.
$$
\De u=0,\quad x\in\Om \quad u\mid_{\Ga}=f(x) \quad f\in
C(\Ga)
$$
Решение ищем в классе функций $C^2(\Om)\cap C(\ol
\Om)$\\.

\noindent \textbf{$D_e$ (внешняя задача Дирихле):}\\
$$
\De u=0,\quad x\in \R^n\backslash\ol\Om\quad
u\mid_{\Ga}=f(x) \quad f\in C(\Ga)\quad u\ra 0\,
|x|\ra \bes
$$
Решение также ищем в классе $C^2(\Om)\cap C(\ol
\Om)$\\.

Пусть $u \in C^2(\Om)\cap C(\ol \Om)$,
$x_0\in\Ga=\pd\Om$

\textbf{Определение:} правильная нормальная производная функции
$u$ на поверхности $\Ga$ -- равномерный по $x_0$ предел (если
таковой существует и непрерывен)
$$
\lim{\Om\ni x\ra x_0,x\in \nu_x} \frac{\pd
u}{\pd \nu_x}
$$\\

\textbf{Определение:} $\Ga$ -- регулярная поверхность, если:
\begin{enumerate}
\item $\fa x_0\in\Ga \quad \nu_{x_0}$
\item В локальной системе координат $\xi_1,\ldots,\xi_n$ с началом
координат $x_0$,такой что  $\xi_n$ направлено по $\nu_x$, может
быть записано $\xi_n=f(\xi_1,\ldots,\xi_{n-1}$ в некоторой
окрестности $x_0$.
\item $f\in C^2(D)$, где $D$ -- проекция окрестности $x_0$ на
касательную плоскость.
\end{enumerate}





%--------------------------Лекция 24 (10)----------------------------






\section{Лекция 24}
\subsection{Решение внутренней задачи Дирихле и внешней задачи Неймана в виде потенциала}
$\Om$ - ограниченная область, $\pd\Om=\Ga$ -
регулярная поверхность.\\
Будем искать решение внутренней задачи Дирихле в виде потенциала
двойного слоя с неизвестной плотностью.
$$
u(x)=\int_{\Ga}\si(\xi)\frac{\pd}{\pd\nu_{\xi}}\frac{1}{|x-\xi|^{n-2}}ds_{\xi}
$$
$$
\si(x)-\frac{2}{\om_n(n-2)}\int_{\Ga}\si(\xi)\frac{\pd}{\pd\nu_{\xi}}\frac{1}{|x-\xi|^{n-2}}ds_{\xi}=
-\frac{2}{\om_n(n-2)}\ph(x)\quad (*)
$$
Будем искать решение внешней задачи Неймана в виде потенциала
простого слоя.
$$
u(x)=\int_{\Ga}\mu(\xi)\frac{\pd}{\pd\nu_{\xi}}\frac{1}{|x-\xi|^{n-2}}ds_{\xi}
$$
$$
\mu(x)-\frac{2}{\om_n(n-2)}\int_{\Ga}\mu(\xi)\frac{\pd}{\pd\nu_x}\frac{1}{|x-\xi|^{n-2}}ds_{\xi}=
-\frac{2}{\om_n(n-2)}\psi(x)\quad(**)
$$
В обоих случаях интегральное уравнение имеет вид
$\si(x)-\la T\si=f(x)$.\\
Пусть $X$ - банахово пространство, $T:X\ra X$ - вполне непрерывен
(компактен).\\
$T^*:X^*\ra X^*$.$\quad\la\in\mathbb{C}$.
$$
u-\la Tu=f,\quad f\in X,u\in X\quad\quad (1)
$$
$$
v-\bar{\la} T^*v=g,\quad g\in X^*,v\in X^*\quad\quad (2)
$$
$$
u-\la Tu=0,\quad u\in X\quad\quad (3)
$$
$$
v-\bar{\la} T^*v=0,\quad v\in X^*\quad\quad (4)
$$
\textit{\underline{Определение.}} $\la$ -
\textit{характеристическое число} $T$, если существует
нетривиальное решение (3); это решение будем называть
\textit{собственным элементом} $T$. Размерность пространства
решений (3), отвечающих данному характеристическому числу
$\la$, назовем \textit{рангом} $\la$.
\subsubsection{Теоремы Фредгольма}
\underline{\textit{\textbf{Теорема 1.}}} Пусть $T$ - вполне
непрерывный оператор в банаховом пространстве $X$. Тогда любое
характеристическое число $T$ имеет конечный ранг.\\
\underline{\textit{\textbf{Теорема 2.}}} Пусть $T$ - вполне
непрерывный оператор в банаховом пространстве $X$. Тогда если
$\la$ - характеристическое число $T$, то $\bar{\la}$ -
характеристическое число $T^*$ того же ранга.\\
\underline{\textit{\textbf{Теорема 3.}}} Пусть $T$ - вполне
непрерывный оператор в банаховом пространстве $X$. Тогда множество
характеристических чисел $T$ либо конечно, либо счетно, причем
если оно счетно,то имеет единственную предельную точку на
бесконечности.\\
\underline{\textit{\textbf{Теорема 4.}}} Пусть $T$ - вполне
непрерывный оператор в банаховом пространстве $X$. Тогда (1)
разрешимо $\Leftrightarrow$ $f\perp v$, где $v$ - произвольное решение (4).\\
\underline{\textit{\textbf{Теорема 5 (альтернатива Фредгольма)}}}
Пусть $T$ - вполне непрерывный оператор в гильбертовом
пространстве $H$. Тогда либо (3) имеет только тривиальное решение,
и тогда (1) имеет единственное решение $\fa f\in H$, либо (3)
имеет нетривиальное решение, и тогда (1) или не имеет решений, или
имеет бесконечно много решений (это зависит от $f$).\\
Рассмотрим оператор $T:L_2(\Ga)\ra L_2(\Ga)$.
$$
Tu=\int_{\Ga}u(\xi)\frac{\pd}{\pd\nu_{\xi}}\frac{1}{|x-\xi|^{n-2}}ds_{\xi}
$$
Тот факт, что $Tu\in L_2(\Ga)$, будет доказан позднее. Тогда
$$
T^*v=\int_{\Ga}v(\xi)\frac{\pd}{\pd\nu_x}\frac{1}{|x-\xi|^{n-2}}ds_{\xi}
$$
Обозначим
$$K(x,\xi)=\frac{\pd}{\pd\nu_{\xi}}\frac{1}{|x-\xi|^{n-2}}\quad
K_1(x,\xi)=\frac{\pd}{\pd\nu_x}\frac{1}{|x-\xi|^{n-2}}
$$
Легко заметить, что $K_1(\xi,x)=K(x,\xi)$.\\
покажем, что из разрешимости соответствующих интегральных
уравнений следует разрешимость внутренней задачи Дирихле и внешней
задачи Неймана.\\
Рассмотрим сопряженное однородное уравнение
$$
\mu(x)-\frac{2}{\om_n(n-2)}\int_{\Ga}\mu(\xi)\frac{\pd}{\pd\nu_x}\frac{1}{|x-\xi|^{n-2}}ds_{\xi}=0
$$
и докажем, что у него есть только тривиальное решение.\\
Пусть $\mu_0(x)\in L_2(\Ga)$ - решение нашего уравнения (то,что
$\mu_0\in C(\Ga)$, мы докажем позже). Можно построить потенциал
простого слоя с плотностью $\mu_0(x)$:
$$
P_1(x)=\int_{\Ga}\mu_0(\xi)\frac{1}{|x-\xi|^{n-2}}d\xi
$$
По свойствам потенциала простого слоя:
$$
\De P_1(x)=0,\quad x\in\R^n\wo\ol{\Om}
$$
$$
P_1(x)\ra 0,\quad |x|\ra\bes
$$
Поскольку $\mu_0$ - решение нашего уравнения, то $\liml{x\ra
x_0\in\Ga}\frac{\pd P_1}{\pd\nu_x}=0$, т.е. $P_1$ -
решение внешней задачи Неймана. В силу единственности решения
такой задачи получаем $P_1\equiv 0$ в
$\R^n\wo\ol{\Om}$. По свойствам потенциала
простого слоя $P_1\in C(\R^n)$, откуда $P_1\equiv 0$ на
$\Ga$.\\
Внутри области $\Om$ имеем $\De P_1(x)=0,\quad x\in\Om$,
$P_1(x)=0, x\in\Ga$. В силу единственности решения внутренней
задачи Дирихле получаем $P_1\equiv 0$ на $\Om$.\\
Итак, $P_1\equiv 0$ в $\R^n$. По теореме о скачке
нормальной производной потенциала простого слоя $\mu_0(x)\equiv 0$
на $\Ga$, что и требовалось доказать.\\
По теореме 5 уравнение (**) имеет единственное решение $\fa
\psi \in L_2(\Ga)$, а,значит, и для $\fa \psi \in
C(\Ga)$, т.е. внешняя задача Неймана разрешима. Но тогда
$\la=\frac{2}{\om_n(n-2)}$ не является характеристическим
числом $T^*$, следовательно,
$\bar{\la}=\frac{2}{\om_n(n-2)}$ не является
характеристическим числом $T$. Отсюда получаем, что внутренняя
задача Дирихле разрешима $\fa \ph \in L_2(\Ga)$,
а,значит, и для $\fa \ph \in C(\Ga)$, что и
требовалось.\\
Теперь можно окончательно сформулировать доказанные теоремы.\\
\underline{\textit{\textbf{Теорема.}}} Внутренняя задача Дирихле
$(\Dc_i)$ имеет единственное классическое решение для
любой непрерывной граничной функции $\ph$, и это решение
представимо в виде потенциала двойного слоя.\\
\underline{\textit{\textbf{Теорема.}}} Внешняя задача Неймана
$(\mathcal{N}_e)$ имеет единственное классическое решение для
любой непрерывной граничной функции $\psi$, и это решение
представимо в виде потенциала простого слоя.
\subsection{Решение внешней задачи Дирихле и внутренней задачи Неймана в виде потенциала}
Будем искать решение наших задач в таком же виде, что и в
предыдущем разделе, где
$$
\si(x)+\frac{2}{\om_n(n-2)}\int_{\Ga}\si(\xi)\frac{\pd}{\pd\nu_{\xi}}\frac{1}{|x-\xi|^{n-2}}ds_{\xi}=
\frac{2}{\om_n(n-2)}\ph(x)\quad (*)
$$
$$
\mu(x)+\frac{2}{\om_n(n-2)}\int_{\Ga}\mu(\xi)\frac{\pd}{\pd\nu_x}\frac{1}{|x-\xi|^{n-2}}ds_{\xi}=
\frac{2}{\om_n(n-2)}\psi(x)\quad(**)
$$
%---------------------------------------------------------------
Однородное уравнение  $D_l$:
$$
\si(x)+\frac{2}{(n-2)\om_n}\int\limits_\Ga \si(\xi)\frac{\pd}{\pd \nu_\xi}
\frac{1}{|x-\xi|^{n-2}}dS_\xi = 0,\ x\in\Ga.
$$

Есть нетривиальное решение $\si(x)\equiv 1$:
$$
\int\limits_\Ga \si(\xi)\frac{\pd}{\pd \nu_\xi}
\frac{1}{|x-\xi|^{n-2}}dS_\xi =
-\frac{(n-2)\om_n}{2},\ x\in \Ga
$$
Ранг  $\la=-\frac{2}{(n-2)\om_n}\ge 1 \ \Rightarrow $ у сопряженного однородного
уравнения

$$
\mu(x)+\frac{2}{(n-2)\om_n}\int\limits_\Ga \mu(\xi)\frac{\pd}{\pd \nu_\xi}
\frac{1}{|x-\xi|^{n-2}}dS_\xi = 0,\ x\in\Ga.
$$

Есть нетривиальное решение $\mu_(x)$.\\
Покажем, что любое нетривиальное решение линейно выражается через это
тогда и только тогда, когда $rank \ \la = 1$\\
Пусть $\mu_1(x)$ - другое нетривиальное решение. Построим два потенциала простого слоя:
$$
P_1^0(x)=\int\limits_\Ga \mu_0(\xi)\frac{1}{|x-\xi|^{n-2}}dS_\xi
P_1^1(x)=\int\limits_\Ga \mu_1(\xi)\frac{1}{|x-\xi|^{n-2}}dS_\xi
$$
$$
\begin{cases}
\De P_1^0(x)=0,\ x\in\Om\\
\liml{x\ra x_0\in\Ga}\frac{\pd P_1^0}{\pd\nu_x}=0
\end{cases}
$$
Т.е $P_1^0$ есть решение $N_i \Rightarrow P_1^0\equiv c_0=\const,\ x\in\Om$\\
То же самое относится к $P_1^1\Rightarrow P_1^1\equiv c_1=\const,\ x\in\Om$\\
Возьмем плотность $\wt{\mu}=c_1\mu_0(x)-c_0\mu_1(x)$
и рассмотрим потенциал простого слоя
$$
P_1(x)=\int\limits_\Ga \wt{\mu}(\xi)\frac{1}{|x-\xi|^{n-2}}dS_\xi=
c_1c_0-c_0c_1=0,\ x\in\Om
$$
$P_1\equiv 1,\ x\in \Om\Rightarrow P_1\equiv 0,\ x\in\Ga$ ;
$\De P_1=0,\ x\in\R^n/\ol{\Om},\ P_1\ra 0,\ |x|\ra\bes$\\
В силу единственности решения $D_l,\ P_1\equiv 0,\ x\in\R^n\Rightarrow\wt{\mu}
,\ x\in\Ga\Rightarrow\mu_0,\mu_1$ линейно независимы.\\
Согласно теореме Фредгольма (**) разрешимо $\Leftrightarrow \psi$ ортогональна константе:
$$
(\psi,1)_{L_2(\Ga)}=\int_\Ga \psi dS=0\ \text{- условие разрешимости необходимое и достаточное}
$$

\textbf{Теорема 3}\\
Внутренняя задача Неймана ($N_i$) имеет (классическое) решение для тех
и только для тех непрерывных ограниченных функций $\psi(x)$, для которых
$\int_\Ga\psi(x)dS=0$. Если есть решение, то оно единственное, с точностью
до прибавления постоянной.\\
$D_l:$ согласно теореме Фредгольма (*) разрешимо
$\Leftrightarrow\\
(\ph,\mu_0)_{L_2(\Ga)},\
\mu_0(x)+\frac{2}{(n-2)\om_n}\int\limits_\Ga \mu_0(\xi)\frac{\pd}{\pd \nu_\xi}
\frac{1}{|x-\xi|^{n-2}}dS_\xi = 0$\\
Решение $D_l\ \exists\ \fa\ \ph\in C(\Ga)$, но в общем случае его нельзя найти в
виде потенциала двойного слоя.\\
Для $(\ph,\mu_0)$ все хорошо.\\
В общем случае ищем решение в виде (считаем, что $0\in\Om$)
$$
u(x)=
\int\limits_\Ga \si(\xi)\frac{\pd}{\pd \nu_\xi}
\frac{1}{|x-\xi|^{n-2}}dS_\xi
+
\frac{1}{|x|^{n-2}}\int_\Ga\si(\xi)dS_\xi
$$
Тогда на границе
$$
\ph(x)-\frac{1}{|x|^{n-2}}\int_\Ga\si(\xi)dS_\xi=
\text{(теорема о скачке)}=
$$
$$
=
\frac{(n-2)\om_n}{2}\si(x)+
\int\limits_\Ga \si(\xi)\frac{\pd}{\pd \nu_\xi}
\frac{1}{|x-\xi|^{n-2}}dS_\xi,\ x\in\Ga
$$
$$
\si(x)+\frac{2}{(n-2)\om_n}
\int\limits_\Ga \si(\xi)
(
\frac{\pd}{\pd \nu_\xi}\frac{1}{|x-\xi|^{n-2}}+\frac{1}{|x|^{n-2}}
)dS_\xi
=\frac{2}{(n-2)\om_n}\ph(x),\ x\in\Ga
$$
Рассмотрим соответствующее однородное уравнение:
$$
\si(x)+\frac{2}{(n-2)\om_n}
\int\limits_\Ga \si(\xi)
(
\frac{\pd}{\pd \nu_\xi}\frac{1}{|x-\xi|^{n-2}}+\frac{1}{|x|^{n-2}}
)dS_\xi=0
$$
Докажем, что $\si\equiv 0$.\\
Пусть существует решение $\si_0\in L_2(\Ga)(\Rightarrow C(\Ga)-\text{ докажем дальше})$:
$$
u_0(x)=
\int\limits_\Ga \si_0(\xi)\frac{\pd}{\pd \nu_\xi}
\frac{1}{|x-\xi|^{n-2}}dS_\xi
+
\frac{1}{|x|^{n-2}}\int_\Ga\si_0(\xi)dS_\xi
$$
$\De u_0=0$ вне $\ol{\Om};\ u_0(x)\ra 0,\ x\ra\bes;\ u_0|_\Ga =0$.\\
В силу единственности решения $D_l:\ u_0\equiv 0,\ x\in \R^n/\ol{\Om}$ имеем
$$
\int\limits_\Ga \si_0(\xi)\frac{\pd}{\pd \nu_\xi}
\frac{1}{|x-\xi|^{n-2}}dS_\xi
+
\frac{1}{|x|^{n-2}}\int_\Ga\si_0(\xi)dS_\xi
=0
$$
Умножим обе части уравнения на $|x|^{n-2}$, устремим $x\ra\bes$
и воспользуемся тем, что $|P_2(x)|\le \frac{c}{|x|^{n-1}}$ получаем, что
$$
\int_\Ga \si_0(\xi)dS_\xi=0
$$
Вспоминая уравнение для $\si_0$ получаем
(т.к $\int_\Ga \si_0(\xi)\frac{1}{|x|^{n-2}}dS_\xi=0$)
$$
\si_0(x)+\frac{2}{(n-2)\om_n}
\int\limits_\Ga \si_0(\xi)
\frac{\pd}{\pd \nu_\xi}\frac{1}{|x-\xi|^{n-2}}dS_\xi=0
$$
Как мы уже доказали, в таком случае $\si_0(\xi)\equiv c=\const$. Подставляя,
получаем $\int_\Ga cdS_\xi=0\Rightarrow c=0\Rightarrow\si_0\equiv 0$, что и требовалось.\\
Теперь по теореме Фредгольма неоднородное уравнение разрешимо $\fa\ph\in C(\Ga)$.

\textbf{Теорема 4.}\\
Внешняя задача Дирихле $D_l$ имеет единственное (классическое) решение
для любой непрерывной граничной функции $\ph$ и это решение представляется в
виде (выше).






%--------------------Лекция 25(11)---------------------------------







\section{Лекция 25}
$$
T:L_2(\Ga)\ra L_2(\Ga),\
Tu=\int\limits_{\Ga}u(\xi)\frac{\pd}{\pd\nu_\xi}\frac{1}{|x-\xi|^{n-2}}dS_\xi
$$
$$
\frac{\pd}{\pd\nu_\xi}\frac{1}{|x-\xi|^{n-2}}
=K(\xi,\ep)=
-(n-2)\frac{\cos{(r,\nu_\xi)}}{r^{n-1}}
$$
$|\cos(r,\nu_\xi)|\le|x-\xi|^\alpha$, $\Ga$ - поверхность Ляпунова с показателем $\alpha$.
$$
K(\xi,\ep)=
-(n-2)\frac{\cos{(r,\nu_\xi)}r^{-\alpha/2}}{r^{n-1-\alpha/2}}=
\frac{A(x,\xi)}{r^{n-1-\alpha/2}}
$$
$$
Tu\equiv\int_\Ga \frac{A(x,\xi)}{r^{n-1-\alpha/2}}u(\xi)dS_\xi
\ \text{- операторы со слабой особенностью}
$$
$A(x,\xi)\in C(\Ga\times\Ga)$- продолжение по непрерывности.\\
Пусть $u\in L_2(\Ga),\ \beta=\alpha/2$, тогда
$$
||Tu||_{L_2(\Ga)}^2=
\int_\Ga(\int_\Ga \frac{A(x,\xi)}{r^{n-1-\beta}}u(\xi)dS_\xi)^2dS_x
$$
$$
\left(
\int_\Ga \frac{A(x,\xi)}{r^{\frac{n-1-\beta}{2}}}
\frac{u(\xi)}{r^{\frac{n-1-\beta}{2}}}dS_\xi
\right)^2\le
\int_\Ga \left(\frac{A(x,\xi)}{r^{\frac{n-1-\beta}{2}}}\right)^2dS_\xi
\int_\Ga \left(\frac{u(\xi)}{r^{\frac{n-1-\beta}{2}}}\right)^2dS_\xi
$$
$|A(x,\xi)|\le M_0,\ x,\xi\in\Ga\ \eta\le d$- радиус сферы.
$$
\int_\Ga\frac{A(x,\xi)}{r^{n-1-\beta}}dS_\xi\le
M_0^2
\left(
\int_{\Ga'_\eta} \frac{dS_\xi}{r^{n-1-\beta}}
+\int_{\Ga''_\eta} \frac{dS_\xi}{r^{n-1-\beta}}
\right)
$$
$$
\int_{\Ga''_\eta} \frac{dS_\xi}{r^{n-1-\beta}}\le\bes
$$
$$
\int_{\Ga'_\eta} \frac{dS_\xi}{r^{n-1-\beta}}
\le C_0\int_0^\eta\rho^{1+\beta-n}\rho^{n-2}d\rho\le\bes\Rightarrow
\int_\Ga\frac{A(x,\xi)}{r^{n-1-\beta}}dS_\xi\le\bes
$$
$$
||Tu||_{L_2(\Ga)}^2\le
M_1\int_\Ga\int_\Ga\frac{u^2(\xi)}{r^{n-1-\beta}}
dS_\xi dS_x
=M_1\int_\Ga u^2(\xi)
\left(
\int_\Ga\frac{dS_x}{r^{n-1-\beta}}dS_\xi
\right)\le
$$
$$
\le M_1M_2\int_\Ga u^2(\xi)dS_\xi= M_1M_2||u||_{L_2{\Ga}}^2
\Rightarrow
Tu\in L_2(\Ga),T\ -\ \text{ограничен.}
$$

\textbf{Теорема 4.}\\
 $T$\т вполне непрерывен.\\
\textit{Доказательство.}\\
$\{T_n\}, \ T_n: X\ra X\ $(банахово), $T_n$ компактны, $T_n\xrightarrow{||.||} T,\
T:X\ra X\Rightarrow T$ компактен.\\
$T=T_\ep^1+T_\ep^2$
$$
K_\ep^1(x,\xi)=
\begin{cases}
K(x,\xi),\ |x-\xi|<\ep\\
0,|x-\xi|\ge\ep
\end{cases}
K_\ep^2(x,\xi)=
\begin{cases}
K(x,\xi),\ |x-\xi|\ge\ep\\
0,|x-\xi|<\ep
\end{cases}
$$
При фиксированном $\ep,\ T_\ep^2$ - фредгольмов, т.к $|K_\ep^2|\le K_0$
Осталось доказать, что \\ $||T_\ep^1||\ra 0,\ \ep\ra +0$.
$$
||T_\ep^1u||_{L_2(\Ga)}^2=
\int_\Ga
\left(
\int_{\Ga'_\ep}K(x,\xi)u(\xi)dS_\xi
\right)^2dS_x
$$
$$
\left(
\int_{\Ga'_\ep} \frac{A(x,\xi)}{r^{\frac{n-1-\beta}{2}}}
\frac{u(\xi)}{r^{\frac{n-1-\beta}{2}}}dS_\xi
\right)^2\le
\int_{\Ga'_\ep} \left(\frac{A(x,\xi)}{r^{\frac{n-1-\beta}{2}}}\right)^2dS_\xi
\int_{\Ga'_\ep} \left(\frac{u(\xi)}{r^{\frac{n-1-\beta}{2}}}\right)^2dS_\xi
$$
$$
\int_{\Ga'_\ep}\frac{A(x,\xi)}{r^{n-1-\beta}}dS_\xi\le
M\int_{\Ga'_\ep} \frac{dS_\xi}{r^{n-1-\beta}}
\le M\int_0^\ep \rho^{1+\beta-n}\rho^{n-2}d\rho=M_2\ep^\beta
$$
$$
||T_\ep^1u||_{L_2(\Ga)}^2\le
M_2\ep^\beta
\int_\Ga\int_\Ga\frac{u^2(\xi)}{r^{n-1-\beta}}dS_\xi dS_x=
M_2\ep^\beta
\int_\Ga
\left(\int_\Ga\frac{dS_\xi}{r^{n-1-\beta}}\right) u^2(\xi)dS_x\le
M_3\ep^\beta||u||_{L_2(\Ga)}^2
$$

\textbf{Теорема 2}\\
Пусть $u(x)\in L_2(\Ga)$ решение интегрального уравнения со слабой особенностью:
$$
u(x)-\int_\Ga\frac{A(x,\xi)}{r^{n-1-\beta}}u(\xi)dS_\xi=\ph(x),\
A(x,\xi)\in C(\Ga\times\Ga),\ \ph(x)\in C(\Ga),\ \beta>0
$$
Тогда $u\in C(\Ga)$.
\textit{Доказательство.}\\
$T=T_\ep^1+T_\ep^2,\ K_\ep^1(x,\xi)=K(x,\xi)\eta(|x-\xi|),\
K_\ep^2(x,\xi)=K(x,\xi)(1-\eta(|x-\xi|)),\\ \eta$\т <<шапочка>> до $\ep$.
$u(x)-T_\ep^1=T_\ep^2+\ph=:g(x),\ ||T_\ep^1||
<M\ep^{\beta/2},\ \ep\ll1$\\
$T_\ep^2u $ непрерывен, т.к $K_\ep^2\in C^\bes
\Rightarrow g(x)\in C(\Ga)$\\
$(Id-T_\ep^1)u=g(x)\in C(\Ga),||T_\ep^1||<1$.

\textbf{Теорема Банаха.}\\
A - линейный ограниченный оператор на банаховых пространствах
и норма А меньше 1, тогда существует ограниченный оператор
$(Id-A)^{-1}$  и при этом $(Id-A)^{-1}=\sum\limits_{n=0}^\bes A^n$\\
продолжим доказательство\\
$u(x)=\sum\limits_{n=0}^\bes (T_\ep^1)^n g(x);\ (T_\ep^1)^n g(x)\in C(\Ga)$
$$
|T_\ep^1 g|\le
\int_{\Ga'_\ep}\frac{|A(x,\xi)|}{r^{n-1-\beta}}|g(\xi)|dS_\xi,\
|g(\xi)|\le M_0,\ |A(x,\xi)|\le M_1\Rightarrow
$$
$$
\Rightarrow
|T_\ep^1 g|\le
M_0M_1\int_{\Ga'_\ep}\frac{dS_\xi}{r^{n-1-\beta}},
\int_{\Ga'_\ep}\frac{dS_\xi}{r^{n-1-\beta}}\le M_2\ep^\beta
\Rightarrow
$$
$$
\Rightarrow
\maxl{x}|T_\ep^1 g|\le
M_0 M_1 M_2 \ep^\beta
$$
$$
|(T_\ep^1)^2 g|=|T_\ep^1(T_\ep^1 g))|\le (M_0M_1M_2\ep^\beta)^2
$$
$$
|(T_\ep^1)^n g|\le (M_0M_1M_2\ep^\beta)^n
$$
Берем $\ep:\ M_0M_1M_2\ep^\beta=q<1$ тогда исходный ряд будет сходится равномерно
$\Rightarrow u\in C(\Ga)$

\subsection{Вариационный метод решения задачи Дирихле}
$\Om$ - ограниченная область в $\R^n,\ H'\subs H_1(\Om)$
произвольное подпространство - линейное подпространство,
на котором задано (.,.), и оно эквивалентно (.,.) на $H_1(\Om)$,
причем Н' полно относительно (.,.) $c||u||_{H_1}\le||u||_{H'}\le C||u||_{H_1}$.\\
Рассмотрим:\\
$E:H'\ra\R,\ Eu=\hn{u}_{H'}^2+2(f,u)_{L_2},\ f\in L_2(\Om)$ фиксирована.
$$
|(f,u)_{L_2(\Om)}|\le||f||_{L_2(\Om)} \hn{u}_{L_2(\Om)}\le \hn{f}_{L_2(\Om)} \hn{u}_{H_1(\Om)}\le
c \hn{f}_{L_2(\Om)} \hn{u}_{H'}
$$
$$
Eu=\hn{u}_{H'}^2+2(f,u)_{L_2}\ge \hn{u}_{H'}^2-2c \hn{f}_{L_2(\Om)}\hn{u}_{H'}
=(|u||_{H'}-\hn{f}_{L_2})^2-c^2\hn{f}_{L_2}\ge -c^2\hn{f}_{L_2}
$$
$\Rightarrow \inf\limits_{H'}Eu<-\bes,\ \exists v_n:\ \liml{m\ra\bes}E(v_m)=d$
- минимизирующая последовательность.\\
$u\in H'$ называется элементом, реализующим $min E$ на H', если $Eu=d$

\textbf{\textit{Лемма.}}\\
Для любого подпространства $H'$ пространства $H(\Om)\ \exists !$
элемент $u\in H'$, реализующий минимум функционала Е на H'.\\
Любая минимизирующая последовательность сходится к этому элементу.\\
\textit{Доказательство леммы}\\
$\{v_m\}$ минимизирующая последовательность. $\fa\ep>0\
\exists M(\ep)\ \fa m>M\ d\le Ev_m<d+\ep$
$$
\hn{\frac{v_m\pm v_n}{2}}_{H'}^2= \frac14 \hn{v_m}_{H'}^2+\frac14 \hn{v_n}_{H'}^2 \pm\frac12(v_m,v_n)_{H'}
$$
$$\hn{\frac{v_m- v_n}{2}}_{H'}^2= \frac12 \hn{v_m}_{H'}^2+\frac12 \hn{v_n}_{H'}^2- \hn{\frac{v_m+ v_n}{2}}_{H'}^2
=\frac12E(v_n)+\frac12E(v_n)-E(\frac{v_n+v_m}{2})
$$
$
m,n\ge M,\Rightarrow
||\frac{v_m- v_n}{2}||_{H'}^2<\frac12(d+\ep)+\frac12(d-\ep)
-d=\ep\Rightarrow\{v_m\}
$
фундаментальная, тогда сходится по норме в H', т.к они эквивалентны\\
$v_m\xrightarrow{||.||_{H'}}\Rightarrow Eu=d$. Единственность очевидна.


%--------------------------------------- Лекция 26 ---------------------------------------------

\section{Лекция 26}
\subsection {Метод Ритца}
Возьмем в $H'$  произвольную линейно независимую систему функций
$\ph_1,\ldots,\ph_k,...$, линейная оболочка которой плотна в
$H'$. обозначим через $R_k$ линейную оболочку первых $k$ функций
из этой системы. Мы знаем, что $\exists! v_k \in R_k:\ \minl{R_k} E(u)=E(v_k)$.
Ищем $v_k=\suml{j=1}{k} c_j^k \ph_j$.ВВедем функцию
$$
F(c_1,\ldots,c_k)=E(\suml{j=1}{k} c_j^k \ph_j)
$$
В точке минимума должны выполняться условия $\frac{\pd
F}{\pd c_j}=0\ \fa j=1,\ldots,k$, что эквивалентно
системе линейных уравнений
$$
\suml{i=1}{k} c_i
(\ph_i,\ph_j)_{H'}+(f,\ph_j)_{L_2(\Om)}\quad j=1,\ldots,k
$$
Определитель системы представляет собой определитель Грамма для
системы $\ph_1,\ldots,\ph_k$ и не равен нулю в силу их линейной
независимости. Поэтому существует решение $c_1^k,\ldots,c^k_k$, и
элемент $v_k=\suml{j=1}{k} c_j^k \ph_j$, реализующий минимум на
$R_k$.

Последовательность $\{v_k\}$ называется \emph{последовательностью
Ритца}.
\begin{theorem}
Последовательность Ритца является минимизирующей функционал
$E(\cdot)$ на подпространстве $H'$ последовательностью.
\end{theorem}


\noindent\textbf{Доказательство:}\\
Имеем
$$
R_1\subs R_2\subs R_3\subs \ldots
$$
$$
E(v_1)\ge E(v_2)\ge E(v_3) \ge \ldots \ge d
$$

Так как система $\{\ph_k\}$ полна, то
$$
\fa\ep>0\  \exists K(\ep)\  \exists
u_\ep(x)=C_1(\ep)\ph_1+\ldots+C_{K(\ep)}\ph_{K(\ep)}
\in R_{K(\ep)}\,:\ ||u-u_\ep||_{H'}<\ep
$$
Где $E(u)=d$.Рассмотрим
$$
E(u_\ep)=||u_\ep||^2_{H'}+2
(f,u_\ep)_{L_2}=||u_\ep-u+u||^2_{H'}+2
(f,u_\ep-u)_{L_2}+2(f,u)_{L_2}=
$$
$$
=E(u)+E(u_\ep-u)+2(u_\ep-u,u)_{H'} \le d+
||u_\ep-u||^2_{H'}+2(f,u_\ep-u)_{L_2}+2(u_\ep-u,u)_{H'}\le
d+\ep^2+C_0\ep \le
$$
$$
\le d+C_1\ep
$$
Получили $E(u_\ep)\le d+C_1\ep$ , Но $d\le
E(v_{K_\ep})\le E(u_\ep) \le d+C_1\ep$,
поэтому $\fa\ep>0\  \exists K(\ep)\ \fa
s>K(\ep) \quad d\le E(v_s)\le d+C_1 \ep$,
 откуда $\liml{s\ra\bes} E(v_s) =d$. Что и требовалось
 доказать.\\

Итак, пусть $E(u)=d$ -- минимум функционала в $H'$. Рассмотрим
 функцию $w(t)=u+tw$, где $t\in\R,\ w\in H'$, и многочлен
$$
P(t)=E(u+wt)=||u+tw||^2_{H'}+2(f,u+tw)_{L_2}=
$$
$$
=||u||^2_{H'}+2(f,u)_{L_2}+2t(u,w)_{H'}+t^2||w||_{H'}+2t(f,w)_{L_2}\ge
d \quad \fa t
$$
Кроме того, $P(0)=E(u)=d$, значит
$P'(0)=2(u,w)_{H'}+(f,w)_{L_2}=0\ \fa w\in H'$.

Рассмотрим $H'=H_1^0(\Om)$, тогда это соотношение примет вид
$$
(u,w)_{H'}=(u,w)_{H_1^0(\Om)}=\int_\Om \nabla u \nabla w\,dx
= -\int_\Om fw\,dx \quad \fa w \in H_1^0(\Om)
$$
Тогда $u\in H_1^0(\Om)$ есть обобщенное решение задачи
Дирихле.\\

Подведем итог.
\begin{theorem}
Существует единственная функция ,реализующая минимум функционала
на. Если скалярное произведение на задается формулой ,то эта
функция является обобщенным решением задачи Дирихле
$$
 \left\{
\begin{aligned}
&\De u=f,\ x\in\Om\\
&u|_{\pd\Om}=0\\
&u\in H_1^0(\Om),\quad f\in L_2(\Om)\\
\end{aligned}
\right.
$$
Последовательность Ритца может быть рассмотрена как
последовательность, приближающая решение.
\end{theorem}






\subsection {Уравнение теплопроводности}
Рассмотрим дифференциальный оператор (теплопроводности):
$$
Tu=u_t-\De_x u
$$
тогда сопряженный оператор
$$
T^*v=-v_t-\De_x v
$$
Уравнение теплопроводности имеет вид $Tu=f(x,t)$, где $x\in\Om$
-- ограниченная область, $t\ge 0$.

Для уравнения теплопроводности имеет место аналог формулы Грина:
пусть
$$
u,v\in C^{2,1}(\wt{\om}_\tau)\cap
C^1(\ol{\om}_\tau),
$$
где
$$
\wt{\om}_\tau=\bigl\{(x,t)\,\bigl|\,x\in\Om,\,0<t\le\tau\,\bigr\}
$$
$$
\om_\tau=\bigl\{(x,t)\,\bigl|\,x\in\Om,\,0<t<\tau\,\bigr\}
$$
здесь $C^{2,1}$ -- пространство функций, дважды дифференцируемых
по х и один раз по t. Тогда
$$
\int_{\om_\tau}(vTu-\,uT^*v)\,dxdt=\int_{\om_\tau}(vu_t+uv_t-v\De
u+u\De v )\,dxdt=
$$
$$
=\int_\Om u(x,\tau) v(x,\tau)\,dx-\int_\Om u(x,0)
v(x,0)\,dx+\int_{S_\tau} \bigl(u \frac{\pd v}{\pd \nu}+v
\frac{\pd u}{\pd \nu}\bigr)\,dS
$$\\


\noindent Проверим, что фундаментальным решение для оператора
теплопроводности является
$$
\Ga(x,x_0,t,t_0)=\frac {\theta(t-t_0)}{(2\sqrt{\pi(t-t_0)})^n}
e^{-\frac{|x-x_0|^2}{4(t-t_0)}}
$$
То есть нужно проверить
$$
\bigr(T_{x,\,t}
\Ga(x,x_0,t,t_0),f(x,t)\bigl)=\bigr(\Ga(x,x_0,t,t_0),T^*f(x,t)\bigl)=f(x_0,t_0)
$$
Убедимся, что $\Ga\in L_{1,\, loc}(\R^{n+1})$:

$$
\frac 1{2^n \pi ^{n/2}}
\int\limits_{t_0}^{t_0+R}\int\limits_{|x-x_0|<\,C} \frac
1{|t-t_0|^{n/2}}
e^{-\frac{|x-x_0|^2}{4(t-t_0)}}\,dxdt\Biggl|_{\xi=\frac{x-x_0}{2\sqrt{t-t_0}}}=\frac
1{\pi^{n/2}} \int\limits_{t_0}^{t_0+R}\int\limits_{|\xi|<\frac C
{2\sqrt{t-t_0}}} e^{-|\xi|^2}\,d\xi dt\le
$$
$$
\le \frac 1{\pi^{n/2}}
\int\limits_{t_0}^{t_0+R}\int\limits_{\R^n}
e^{-|\xi|^2}\,d\xi dt=R
$$
То есть
$$
\bigr(\Ga(x,x_0,t,t_0),T^*f(x,t)\bigl)=\int_{\R^{n+1}}
\Ga(x,x_0,t,t_0)T^*f(x,t)\,dx dt=
$$
$$
\liml{\ep\ra 0}
\int\limits_{t_0+\ep}^{\bes}\int\limits_{\R^n}\Ga(x,x_0,t,t_0)T^*f(x,t)\,dx dt=
$$
Пользуясь формулой Грина,
$$
=\liml{\ep\ra 0}\Bigl(
\int\limits_{t_0+\ep}^{\bes}\int\limits_{\R^n}T\Ga(x,x_0,t,t_0)T f(x,t)\,dx dt+ \int\limits_{\R^n}\Ga(x,x_0,t_0+\ep,t_0)f(x,t_0+\ep)\,dx\Bigl)
$$
\textbf{Упражнение:} Для любого $t>t_0$ $T\Ga=0$.\\
Используя этот факт, пишем:
$$
=\liml{\ep\ra 0} \int\limits_{\R^n}\Ga(x,x_0,t_0+\ep,t_0)f(x,t_0+\ep)\,dx=\liml{\ep\ra
0} \frac 1{(2\sqrt{\pi\ep})^n} \int\limits_{\R^n}
e^{-\frac{|x-x_0|^2}{4\ep}}
f(x,t_0+\ep)\,dx\Biggl|_{\xi=\frac{x-x_0}{2\sqrt\ep}}=
$$

$$
=\liml{\ep\ra 0} \frac
1{\pi^{n/2}}\int\limits_{\R^n}e^{-|\xi|^2}f(x_0+2\sqrt\ep\xi,t_0+\ep)\,d\xi=
$$
$$
=\liml{\ep\ra 0} \Bigl\{ \frac
1{\pi^{n/2}}\int\limits_{|\xi|>N}e^{-|\xi|^2}(f(x_0+2\sqrt\ep\xi,t_0+\ep)-f(x_0,t_0))\,d\xi+\frac
1{\pi^{n/2}}
\int\limits_{|\xi|<N}e^{-|\xi|^2}(f(x_0+2\sqrt\ep\xi,t_0+\ep)-f(x_0,t_0))\,d\xi
\Bigr\}
$$
$$
+f(x_0,t_0)
$$
Покажем, что интегралы стремятся к нулю:
$$
M=\supl{\R^n} |f(x,t)|<\bes \quad
\fa\ep>0\quad \exists N \quad \frac
M{\pi^{n/2}}\int\limits_{|\xi|>N}e^{-|\xi|^2}\,d\xi<\ep/2
$$
$$
\fa\ep>0\quad \de>0\quad \fa x,t:
|x-x_0|<\de,|t-t_0|<\de\quad |f(x,t)-f(x_0,t_0)|<\ep
$$
Возьмем $2\sqrt\ep N<\de,\ep<\de$, тогда и
второй интеграл меньше $\ep$,что мы и хотели показать.
Итак, мы доказали, что $\Ga(x,x_0,t,t_0)$ является
фундаментальным решением.

\begin{theorem}[Принцип максимума в ограниченной области]
Пусть $u(x,t)$ - решение уравнения $Tu=0$ в слое $\wt{w}_\tau$,
принадлежащее классу $C^{2,1}(\wt{w}_\tau)\cap C(\ol{
w_\tau})$.  Тогда $\fa (x,t)\in \wt{w}_\tau$:
$$
\minl{\si_\tau} u(x,t)\le u(x,t)\le \maxl{\si_\tau} u(x,t)
$$
Где $\si_\tau=\Om\cup S_\tau$, $S_\tau$ -- боковая
поверхность цилиндра.
\end{theorem}


\noindent\textbf{Доказательство:}\\
Достаточно доказать, что $\minl{\si_\tau} u(x,t)\le u(x,t)$,
потому что применив это рассуждение к функции $v=-u$, получим
утверждение для максимума.

Из непрерывности $u$ $\exists M>0:\, |u(x,t)|<M$ в $\ol{
w_\tau}$. Выберем $M_1$ такое, что
$$
v(x,t)=u(x,t)-M_1<0 \quad \fa (x,t)\in \ol{ w_\tau}
$$
Тогда $Tv=0$ в $\wt{w}_\tau$ и $v<0$ в $\ol{ w_\tau}$.
 Положим $v=e^{\ga t}w,\,\ga=\const>0$. Покажем, что минимум отрицательного значения $w$ может достигаться
лишь на $\si_\tau$. Предположим противное: $\exists
(x_0,t_0)\in \wt{w}_\tau \,: \, 0>w(x_0,t_0)=\minl{\ol{w_\tau}} w(x,t)$. Ясно, что $w(x,t)<0$ в $\ol{ w_\tau}$,
$\frac{\pd w}{\pd x_j}(x_0,t_0)=0,\ \frac{\pd
w}{\pd t}(x_0,t_0)\le 0$, а $\frac{\pd^2 w}{\pd
x_j^2}(x_0,t_0)\ge 0$. Но тогда
$$
Tv\Bigl|_{(x_0,t_0)}=\Bigl(\ga e^{\ga t}w(x,t)+e^{\ga
t}\frac{\pd w}{\pd t}(x,t)-e^{\ga t}\De
w(x,t)\Bigr)\Bigl|_{(x_0,t_0)}<0
$$
Что противоречит равенству $Tv=0$.

Следовательно
$$
w(x,t)\ge \minl{\si_\tau} w(x,t)
$$
Или, что то же самое,
$$
e^{-\ga t} v(x,t)\ge \minl{\si_\tau} e^{-\ga t} v(x,t)
$$
Переходя к пределу при $\ga\ra 0$  получаем $v(x,t)\ge
\minl{\si_\tau} v(x,t)$, а отсюда и требуемое неравенство для
$u(x,t)$.



%-------------------------------------- Лекция 27 ---------------------------------------------------

\section{Лекция 27}
\subsection{Принципы максимума}


\begin{theorem}[Принцип максимума в неограниченной области]
Пусть $u(x,t)$ - решение уравнения $Tu=0$ в слое
$G_\tau=\left\{\,(x,t)\,\Bigl|\,x\in \R^n,\,0<t\le \tau\,
\right\}$, принадлежащее классу $C^{2,1}(G_\tau)\cap C(\ol{
G_\tau})$. Предположим, что $\fa (x,t)\in G_\tau\quad
|u(x,t)|\le M$. Тогда $\fa (x,t)\in G_\tau$:
$$
\inf_{x \in \R^n} u(x,0)\le u(x,t)\le \supl{x \in \R^n} u(x,0)
$$
\end{theorem}


\noindent\textbf{Доказательство:}\\
Обозначим
$$
M_0=\supl{x \in \R^n} u(x,0),\,m_0=\inf_{x \in \R^n}
u(x,0)
$$
Введем вспомогательную функцию $v(x,t)=|x|^2+2nt$. Легко видеть,
что в $G_\tau$ $Tv=v_t-\De |x|^2=2n-2n=0$. Введем еще функции
$$
W_1(x,t)=M_0-u(x,t)+\ep v(x,t)
$$
$$
W_2(x,t)=M_0-u(x,t)-\ep v(x,t)
$$
Заметим, что $TW_1=TW_2=0$ в $G_\tau$, и кроме того
$$
W_1(x,0)=M_0-u(x,0)+\ep |x|^2\ge 0
$$
$$
W_2(x,0)=m_0-u(x,0)-\ep |x|^2\le 0
$$
Далее, $\fa \ep>0\quad \exists R(\ep)\quad
\fa x:\, |x|\ge R(\ep)\quad W_1(x,t)\ge
0,\,W_2(x,t)\le 0 $ Теперь применим принцип максимума для
ограниченной области  к функциям $W_1$ и $W_2$ в области $G_\tau$:
$$
W_1(x,t)\ge 0,\quad W_2(x,t)\le 0\qquad (x,t)\in G_\tau
$$
Что равносильно
$$
m_0-\ep v(x,t)\le u(x,t) \le M_0+\ep v(x,t)
\qquad \fa(x,t)\in G_\tau
$$
Переходя к пределу при $\ep\ra 0$, получим
утверждение теоремы.\\

\begin{theorem}[строгий принцип максимума]
Пусть функция $u(x,t)$ в цилиндре
\eqn{\wt w_\tau :=\hc{(x,t) \bvl x\in \Om,\, 0<t\le \tau}}
удовлетворяет уравнению $Tu=0$, принадлежит классу $C^{2,1}(\wt{w}_\tau)\cap C(\ol w_\tau)$  и
принимает в точке $(x_0,t_0)\in \wt{w}_\tau$ наибольшее значение, то $u(x,t)\equiv u(x_0,t_0)=\const$ в цилиндре
$\wt{w}_{\tau_0}=\wt{w}_\tau\cap\{t\le\tau_0\}$.
\end{theorem}
\begin{proof}
Предположим противное. Пусть $u(x_1,t_1)<u(x_0,t_0):=M$, где $t_1<t_0$. Соединим точки $(x_0,t_0)$ и $(x_1,t_1)$ ломаной,
содержащейся в $\wt{w}_\tau$ с вершинами в точках $t_1,\ldots,t_n,t_{n+1}=t_0$, причем $t_1<t_2<\ldots<t_{n+1}=t_0$.
Если мы докажем, что из неравенства $u(x_s,t_s)<M$ следует $u(x_{s+1},t_{s+1})<M$, то, двигаясь по ломаной, получим
$u(x_0,t_0)<M$ -- противоречие, и теорема будет доказана.

Пусть
точки $(x_s,t_s)$ и $(x_{s+1},t_{s+1})$ лежат на прямой
$$
x_j=k_j t+a_j,\quad j=1,\ldots,n
$$
Рассмотрим цилиндр $P(x,t)<\rho^2$, где $P(x,t)=\suml{j=1}{n}
(x_j-k_j t-a_j)^2$. Выберем $\rho>0: u(x,t_s)<M-\ep_1\quad
\fa x: P(x,t_s)<\rho^2$. Построим вспомогательную функцию
$$
v(x,t)=e^{-\ga t}(\rho^2-P(x,t))^2,\quad \ga>0
$$
Тогда
$$
Tv=e^{-\ga t}(-\ga(\rho^2-P)^2-2(\rho^2-P)\frac{\pd
P}{\pd t}+4n(\rho^2-8P)-8P
$$
Покажем, что можно $\ga$ выбрать настолько большим, что $Tv<0$
внутри цилиндра. Действительно, на боковой поверхности цилиндра
$Tv=-8e^{-\ga t}\rho^2<0$, поэтому неравенство справедливо и в
некоторой $\de$-окрестности поверхности. Если же
$P(x,t)<\rho^2-\de$, то $\rho^2-P(x,t)>\de$ и при достаточно
большом $\ga$ первый член в скобках будет больше по модулю, чем
сумма модулей остальных членов, то есть $Tv<0$ внутри цилиндра.
Рассмотрим функцию
$$
W(x,t)=M-u(x,t)-\alpha v(x,t)
$$
Ясно, что $TW=-\alpha Tv>0$ в цилиндре. Тогда по принципу
максимума для ограниченной области,
$$
W(x,t)\ge \minl{\si} W(x,t)
$$
Где $\si$ -- <<наклонный стакан>> (боковая поверхность плюс нижнее
основание нашего цилиндра). На боковой поверхности цилиндра
$W(x,t)=M-u(x,t)>0$, а на нижнем основании
$W(x,t_s)=M-u(x,t_s)-\alpha v(x,t_s)\ge \ep_1-\alpha
v(x_t,s)>0$, при достаточно маленьком $\alpha$. Значит, и во всем
цилиндре $W(x,t)\ge 0$, в частности $W(x_{s+1},t_{s+1})\ge 0$,
то есть
$$
u(x_{s+1},t_{s+1})\le M-\alpha v(x_s,t_s)<M
$$
Что и хотели доказать.
\end{proof}

\begin{theorem}[о стабилизации]
Пусть $u(x,t)\in C^{2,1}(w_\bes)\cap C(\ol w_\bes)$ --
решение уравнения $Tu=0$,
$$
w_\bes=\left\{\,(x,t)\,\Bigl|\,x\in \Om,\,0<t<\bes,
\right\}
$$
$$
S_\bes=\left\{\,(x,t)\,\Bigl|\,x\in \pd\Om,\,0<t<\bes,
\right\}
$$
Пусть $u|_{S_\bes}=0$, тогда $u(x,t)\ra0\quad
t\ra \bes$ равномерно по всем $x \in \ol\Om$
\end{theorem}


\noindent\textbf{Доказательство:}\\
Для удобства положим $0\in\Om$. Рассмотрим функцию
$$
v(x,t)=e^{-at}\prod_{j=1}^n \cos b x_j,\quad a>0
$$
Поскольку $v_t=-av$ и $v_{x_j x_j}=-b^2 v$, то $Tv=(nb^2-a)v$.
Следовательно при $a=nb^2$ $v(x,t)$ -- решение уравнения
теплопроводности. Выберем $b$ настолько малым, что
$$
\Om\subs \left\{\,|x_j|<\frac \pi{4b},\quad j=1,\ldots,n
\right\}
$$
Внутри этого параллелепипеда $v(x,0)>0$, поэтому (из непрерывности
$u$ и $v$) $\exists M>0:\, |u(x,0)|<v(x,0),\quad x\in\Om$.
Кроме того, $v|_{S_\bes}>0$. Рассмотрим функции
$$
W_1(x,t)=Mv(x,t)-u(x,t)
$$
$$
W_2(x,t)=Mv(x,t)+u(x,t)
$$
Очевидно, $TW_1=TW_2=0$, при этом $M$ мы выбрали так, что
$W_1(x,0)=Mv(x,0)+u(x,0)\ge 0$ и кроме того, $W_1|_{S_\bes}>0$,
отсюда по принципу максимума $W_1(x,t)\ge 0,\quad x\in \ol
w_\bes$, то есть $u(x,t)\le Mv(x,t)$. Аналогично применяя
принцип максимума к $W_2$, получаем $-u(x,t)\le Mv(x,t)$. Итак,
$$
|u(x,t)| \le Mv(x,t)\quad x\in \ol w_\bes
$$
Но $v(x,t)$ убывает к нулю равномерно по $x\in\ol\Om$,
так что теорема доказана.\\

\subsection {Начально-краевые задачи}
\textbf{1) Первая начально-краевая задача:}\\
$C^{2,1}(\wt{w}_\tau)\cap C(\ol w_\tau)$\\
$Tu=f(x,t)\quad (x,t)\in \wt{w}_\tau$\\
$u|_{S_\tau}=\psi(x,t),\quad
S_\tau=\pd\Om\times(0,\tau)$\\
$u|_{t=0}=\ph(x)$\\
Здесь $f,\psi,\ph$ -- заданные непрерывные функции.\\

\textbf{2) Вторая начально-краевая задача:}\\
$C^{2,1}(\wt{w}_\tau)\cap C(\ol w_\tau)$\\
$Tu=f(x,t)\quad (x,t)\in \wt{w}_\tau$\\
$\frac{\pd u}{\pd \nu}|_{S_\tau}=\psi(x,t)$\\
$u|_{t=0}=\ph(x)$\\
Здесь $f,\psi,\ph$ -- заданные непрерывные функции, а
поверхность $\pd\Om$ -- регулярна.\\

\textbf{3) Третья начально-краевая задача:}\\
$C^{2,1}(\wt{w}_\tau)\cap C(\ol w_\tau)$\\
$Tu=f(x,t)\quad (x,t)\in \wt{w}_\tau$\\
$\Bigl( \frac{\pd u}{\pd \nu}+a(x,t)u\Bigr)|_{S_\tau}=\psi(x,t)$\\
$u|_{t=0}=\ph(x)$\\
Здесь $f,a,\psi,\ph$ -- заданные непрерывные функции, а
поверхность $\pd\Om$ -- регулярна.\\

\textbf{4) Задача Коши:}\\
$C^{2,1}(G_\tau)\cap C(\ol G_\tau)$\\
$Tu=f(x,t)\quad (x,t)\in G_\tau=\R^n \times(0,\tau] $\\
$u|_{t=0}=\ph(x),\quad x\in \R^n$\\
Здесь $f,\ph$ -- заданные непрерывные функции.

\subsection{Теоремы единственности}

\begin{theorem}
Первая краевая задача для оператора теплопроводности имеет
единственное решение.
\end{theorem}
\begin{proof}
Как обычно, рассмотрим разность $v=u_1-u_2$ двух решений этой
задачи. Тогда $Tv=0,\quad v|_{S_\tau}=0,\quad v|_{t=0}=0$.
Согласно принципу максимума для ограниченных областей,
$$
0=\minl{\si_\tau} v(x,t)\le v(x,t)\le \maxl{\si_\tau}
v(x,t)=0
$$
Поэтому $v(x,t)=0$ в $\wt{w}_\tau$ и единственность доказана.
\end{proof}

\begin{theorem}
Задача Коши для уравнения теплопроводности в классе ограниченных функций имеет единственное решение.
\end{theorem}
\begin{proof}
Рассмотрим разность $v=u_1-u_2$ двух решений этой задачи. Тогда
$Tv=0, \quad v|_{t=0}=0, |v|\le M$ для $v\in G_\tau$. Согласно
принципу максимума для неограниченных областей,
$$
0=\minl{\R^n} v(x,0)\le v(x,t)\le \maxl{\R^n}
v(x,0)=0
$$
Поэтому $v(x,t)=0$ в $G_\tau$ и единственность доказана.
\end{proof}

\end{document}
