\input amstex
\documentstyle{amsppt}

\NoBlackBoxes \TagsOnRight \magnification = \magstep1 \topmatter
\title       Примерный план занятий 11-12
\endtitle
\endtopmatter


\document
\subhead{Устойчивость положений равновесия}
\endsubhead

0. 2-4 примера из NN 899-906; 907-911; 915-922.

NB. Когда исследуются положения равновесия поля $v(x),$ удобно
вычислить $\frac{\partial v}{\partial x}(x)$  и
 подставлять вместо $x$ все особые точки. Это лучше, чем следовать
 старой традиции и переносить особую точку в $0$.

 Исследовать на устойчивость неподвижную точку $0$ для отображений

1. $x\mapsto \sin \frac{x}{2}$

2. $x\mapsto \sin ax $

$\qquad$ a)$a\ne 1$

$\qquad$ b)$a=1$

3. $(x,y)\mapsto (\frac x2 -2y+x^2, \ 2x+\frac y2 -y^2)$

4.** $x\mapsto f(x)$, $\ f(0)=0$, $f$ аналитично, $x\in \Bbb R^1$.

\remark{Указание} Случай $f'(0)\ne \pm 1$  исследуется по теореме
Ляпунова.

Случай $f'(0)=1$  прост: ответ зависит от младшего нелинейного
члена.

Случай $f'(0)=-1$  лучше исследовать не по Тейлоровскому
разложению $f$, а по разложению $f^2$: задача сводится к
предыдущей.
\endremark

5. $(x,y)\mapsto (y+\sin \frac x2, x)$

6. Найти все неподвижные точки отображения $(x,y)\mapsto
(x+\frac{\sin (x+y)}{2}, y+\frac {\sin (x-y)}{2})$
 и исследовать их на устойчивость.

7. То же для отображений:
$$
z\mapsto \frac{2z-1}{2-z},\ \ z\mapsto z^2-2,\ z\in \Bbb C^1.
$$

8.* Пусть $P(z)$  -- многочлен, $P(a)=a$. Пусть $|P'(a)|\ne 1; 0$.
Доказать, что неподвижная точка $a$ устойчива либо для отображения
$P$, рассматриваемого в малой окрестности $a$, либо для обратного
отображения, определенного в окрестности $a$.


\enddocument
