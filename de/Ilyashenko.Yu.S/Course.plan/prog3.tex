\input amstex
\documentstyle{amsppt}

\NoBlackBoxes \TagsOnRight \magnification = \magstep1 \topmatter
\title       Примерный план занятий 6-7 (список избыточный)
\endtitle
\endtopmatter


\document

%1. Найти фазовый поток на прямой: $\dot x = \sqrt {x^2 + 1}.$

1. Выпрямить векторное поле $v(x)$ на прямой в окрестности точки
$x_0 =1: v(x) = x; \ x^2; \ x^n; \ \sin x; \  \sqrt {x^2 + 1}.$

2. Выпрямить векторное поле $v(x)$ на плоскости в окрестности
точки $x_0 = (1,1): \ v(x) = 2x; \ (-x_1,x_2); \ (x_1,2x_2); \
(x_2,-x_1); \ (x_1,\sin x_2); \ (x_2,x_1-x_1^2); \ (x_2,-\sin
x_1)$ (в последних двух задачах интегралов можно не брать).

3. Найти производные по начальному условию $x(0) = a$ решения
уравнения
$$
\dot x = v(t,x) \text {при } a = 0:
$$
$$
v(t,x) = x\sin t + x^2, \ x \in \Bbb R^1;
$$
$$
v(t,x) = Ax + (x,x)t^2, \ x\in \Bbb R^2; \ A = \pmatrix 0 & 1\\ 1
& 0
\endpmatrix ;
\ \pmatrix 0 & -1\\ 1 & 0 \endpmatrix
$$

4.  Вывести уравнение в вариациях по параметру (на лекции не было
вывода!)

5. По 2-3 задачи из NN 1064-1073 и 1079-1083.

6. При каком  достаточном условии на функцию  $a(t)$ нулевое решение уравнения
$$
\dot x = xa(t) + x^2f(t,x)
$$
с $2\pi $-периодической правой частью

а) является изолированным  $2\pi $-периодическим решением?

б) сохраняется при малом возмущении (т.е. близкое уравнение с
$2\pi $-периодической правой частью имеет близкое $2\pi
$-периодическое решение)?

7. Найти производную по параметру $\varepsilon $ при $\varepsilon
= 0$ близкого к нулю $2\pi $-периодического решения уравнения
$$
\dot x = x + \varepsilon \sin t.
$$

8. Задачи 16-18 из Арнольда, 3-е изд. с. 264.

\enddocument
