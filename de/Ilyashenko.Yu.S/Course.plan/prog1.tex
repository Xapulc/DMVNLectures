\input amstex
\documentstyle{amsppt}
\magnification = \magstep1
\usepackage[cp1251]{inputenc}
\usepackage[russian]{babel}

\document
\head $\bold {ПРОГРАММА }$ $\bold {ОДУ-2003/04} $
\endhead

\head Лекторы В.А.Кондратьев, Ю.С.Ильяшенко
\endhead

\subhead   1 семестр
\endsubhead

\subhead  Введение. Примеры
\endsubhead

\subhead  Элементарные методы интегрирования
\endsubhead

1. Уравнения с разделяющимися переменными. Декартовы произведения
двух систем.

2. Однородные уравнения. Их группа симметрий.

3. Линейные уравнения первого порядка. Преобразования монодромии и
периодические решения линейных уравнений с периодическими
коэффициентами.

4. Уравнения в полных дифференциалах. Гамильтоновы уравнения с
одной степенью свободы. Маятник.

\subhead  Теорема существования
\endsubhead

5. Принцип сжатых отображений.

6. Теорема существования, единственности и непрерывной зависимости
решений от начальных условий. Метод Пикара.

7. Сходимость Пикаровских приближений к решению (будет
использована во втором семестре при доказательстве: гладкой
зависимости решения от начального условия; теоремы о выпрямлении).

8. Теорема о продолжении интегральных и фазовых кривых. Ее
применение к линейным неавтономным системам.

9. Формула Лиувилля - Ос\-тро\-град\-ско\-го.

\subhead  Линейные уравнения любого порядка с постоянными
коэффициентами
\endsubhead

10. Однородные уравнения и уравнения со специальной правой частью

11. Резонансы. Метод комплексных амплитуд.


\subhead 2 семестр
\endsubhead


\subhead    Линейные системы
\endsubhead

1. Фазовые потоки. Экспонента линейного оператора.

2. Комплексификация и овеществление. Вычисление экспоненты.

3. Экспонента комплексного числа

4. Экспонента  жордановой клетки.

\subhead Теорема о выпрямлении и ее следствия
\endsubhead

5. Теорема существования и единственности (напоминание).
Пикаровские приближения.

6. Производное отображение. Уравнение в вариациях по начальным
условиям и параметрам. Гладкая зависимость решений от начальных
условий и параметров.

7. Теорема о выпрямлении и ее следствия. Полная система первых
интегралов.

8. Задача Коши для линейных и квазилинейных уравнений.

9. Искажение фазового объема.


\subhead Устойчивость. Фазовая плоскость
\endsubhead

10. Устойчивость особых точек дифференциальных уравнений и
неподвижных точек отображений.

11. Фазовая плоскость. Топология фазовых кривых. Отображение
Пуанкаре. Предельные циклы. Теорема Флоке.


\subhead Детерминизм и хаос
\endsubhead

12. Малые колебания. Плотные обмотки тора. Равенство
пространственных и временных средних для иррационального поворота
окружности.

13. Подкова Смейла. Элементы символической динамики.


\enddocument
