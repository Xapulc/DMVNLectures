\documentclass[a4paper]{article}
\usepackage[utf,simple]{dmvn}

\author{Ю.\,С.\,Ильяшенко}
\title{Занятие последнее. Подкова Смейла}
\begin{document}
\maketitle

\section{Теория}

\subsection{Отображение подковы}

  Разделим квадрат на пять равных горизонтальных прямоугольников и пять равных
вертикальных.
 Второй и четвертый снизу прямоугольники обозначим через $D_0$ и $D_1$. Второй и
четвертый слева прямоугольники обозначим через  $D_0'$ и $D_1'$.

{\it Отображение подковы} $f:D=D_0\cup D_1\to D'=D_0'\cup D_1'$
определяется следующим образом. На $D_0$ отображение $f$ аффинно и
является композицией трех отображений: растяжения в 5 раз по
вертикали, сжатия в 5 раз по горизонтали и такого параллельного
переноса, что $f(D_0)=D_0'$. Аналогично определяется $f$ на $D_1$,
$f(D_1)=D_1'$.

Рассмотрим множество $\Lambda\subset D$, состоящее из всех точек
$x$, для которых определены <<полные орбиты>>
$$
O_x=\{f^nx|\ n\in \Zb \}.
$$

\begin{def}
Судьба точки $x\in \Lambda$ -- это последовательность $\omega =(\omega_j)$,
$j\in \Zb$, из нулей и единиц, определяемая следующим образом:
$$
\omega =(\omega_j),\; \omega_j=
\begin{cases}
0 \text{ при } f^jx\in D_0 \\
1 \mbox{ при } f^jx\in D_1.
\end{cases}
$$
\end{def}

\begin{proclaim}[Теорема 1] Множество $\Lambda$ является прямым
произведением двух канторовских множеств,
 получаемых делением отрезка на 5 частей с удалением 1, 3 и 5 части.
\end{proclaim}

\begin{proclaim}[Теорема 2] Каждая последовательность нулей и единиц
реализуется как судьба одной и только одной точки под действием
отображения подковы.

\begin{proclaim}[Теорема 3] Точка является периодической для отображения
подковы, если и только если ее судьба -- периодическая
последовательность.
\end{proclaim}

\section{Задачи}

1. Построить аналог отображения подковы, для которого судьба
точек, имеющих полную орбиту,
 является последовательностью из трех символов, и любая такая последовательность
реализуется как судьба одной и только одной точки.

Пусть $f$ -- отображение подковы, $g$ -- его аналог из задачи 1.

2.  Найти число неподвижных точек для отображений $g$ и $f$.

3. Найти число периодических точек периода $n$ для отображений $f$
и $g$.

4. Сколько точек имеют орбиты, для которых $f^n(x)\to a$ при $n\to
+\infty$, $f^n(x)\to b$ при $n\to -\infty$, где $a$ и $b$ --
различные неподвижные точки отображения $f$?

5*. Построить отображение,определенное на прямоугольниках $ D_0,
D_1 $ из раздела ``теория'', на плоскость, для которого судьба
точек, имеющих полную орбиту, является последовательностью из 0 и
1, в которой две единицы не стоят подряд.

6**.  Построить отображение из задачи 5, для которого каждая
последовательность без двух единиц подряд реализуется как судьба
одной и только одной точки. Обосновать ответ.

7. Построить последовательность $P_n$ периодических орбит
отображения подковы $f$ так, чтобы существовали две
последовательности точек $x_n\in P_n,\ y_n\in P_n$, сходящиеся к
разным неподвижным точкам отображения $f$.

8. Найти диаметр множества точек, судьба которых под действием
отображения $f$ (отображения $g$) содержит один и тот же отрезок
$\omega_{-n}\ldots \omega_0\ldots \omega_{n-1}$.

9. Модифицируем отображение $f$, считая, что  квадрат разделен на
25 равных горизонтальных и 25 равных вертикальных прямоугольников,
$D_0, D_1, D_0'$ и $D_1'$ -- по-прежнему второй и четвертый слева
(соответственно, снизу) прямоугольники. Отображение $h:D_0\cup
D_1\to D_0'\cup D_1'$ определяется так же, как и отображение $f$,
только растяжение и сжатие производятся в 25 раз.

   а) Описать инвариантное множество $\Lambda_1$ отображения $h$;

   б) Построить гомеоморфизм $H:\Lambda\to \Lambda_1$, сопрягающий отображения
$f$ и $h$. Указание: $H$ и $H^{-1}$ переводят друг в друга точки с
одинаковой судьбой.

    в) Доказать, что гомеоморфизм $H$ удовлетворяет условию Гельдера и найти
показатель Гельдера.

    г) Существует ли непостоянная непрерывная
       функция на отрезке с показателем Гельдера больше 1?

    д) Тот же вопрос для функции на Канторовом множестве.

 10. Проверить, что на пространстве $\Sigma^2$ двусторонних последовательностей
из нулей и единиц может быть задана метрика следующим образом:
$$
  d(\omega,\omega')=\cases 0, \ если \ \omega=\omega'   \\
2^{n}, \ где\  n=\min\{k\in \mathZb^+|\ \omega_k\ne \omega_k'\ или
\ \omega_{-k}\ne \omega_{-k}'\}, \ если \ \omega\ne \omega'.
\endcases
$$

  11. Доказать, что сдвиг Бернулли $\sigma:\Sigma^2\to \Sigma^2$ является
гомеоморфизмом.

  12. Доказать, что отображение $f$ имеет плотную в $\Lambda$ орбиту.

  13. (Tent map) Рассмотрим отображение $F:\Bbb R\to \Bbb R$, $F(x)= 2-|4x-2|$.
Пусть $\Delta=\{x\in \Bbb R|\  F^n(x)\ge 0 \ для \ любого\ n\in
\Zb^+\}$. Докажите, что

    а) $\Delta$ -- Канторовское множество;

    б) множество периодических орбит отображения $F$ плотно в $\Delta$;

    в) отображение $F$ имеет плотную в $\Delta$ орбиту.
\end{document}
