\input amstex
\documentstyle{amsppt}

%\NoBlackBotes
%\TagsOnRight
\magnification= \magstep1

\document

\head $\bold {Примерный \ план \  занятий\  1--5 (6)}
\endhead

Кроме стандартных задач на вычисление экспоненты, рисование орбит
и образов единичного квадрата, рекомендуется включить следующие
задачи.


\head $\bold {Рекуррентные\  последовательности.}$
\endhead

1.  Формула $n$-ого члена последовательности Фибоначчи.

2.  Формула $n$-ого члена последовательности
$x_n=x_{n-1}-x_{n-2}$.

3. Бактерии первого вида делятся на две каждую четную минуту.
Бактерия второго вида каждую нечетную минуту пожирает одну
бактерию первого вида (если есть) и немедленно делится на две.
Если есть нечего, то бактерия второго вида умирает. Выживет или
вымрет колония бактерий, в которой в начальный момент имеется
миллиард бактерий первого вида и одна второго. Если вымрет, то за
сколько часов?

4.  $x_n=3x_{n-1}-2x_{n-2},$ $\ \ x_0=0, \ x_1=1.$

5.  $x_n=3x_{n-1}-2x_{n-2}+x_{n-3}+1,$ $\ \ x_0= x_1=x_2=0.$

И т.д.

\head $\bold {Фазовые \ потоки}
\endhead

1. Найти генераторы однопараметрической группы в примерах из
лекций:

$x \mapsto x + \omega t, \ x \in \Bbb R^n;$

$x \mapsto e^tx, \  x \in \Bbb R^n;$

$x \mapsto \pmatrix \cos t & -\sin t \\ \sin t & \cos t
\endpmatrix x, \ x \in \Bbb R^2.$

2$^*.$  Проверить, что $g^t: x \mapsto \frac {x}{1-tx}$ есть
фазовый поток на проективной прямой и найти его генератор.

3. Найти преобразование фазового потока на прямой:

а) $\dot x = 2x$

b) $\dot x = \frac {f(x)}{f'(x)},$

где $f$---диффеоморфизм прямой $g^tx = f^{-1}(e^tf(x))$

c) $\dot x = \sin x$

4.  При каком значении параметра $a$ система $\cases
     \dot x=2x+y+1\\
     \dot y=-4x-2y+a
      \endcases$
      имеет ограниченные решения? При этом $a$ найти преобразование
фазового потока системы.

\vskip 2pc

Очень важно (если это не было сделано в прошлом семестре),
проверить, что студенты владеют комплексными числами, и в
частности, умеют извлекать корни.

\enddocument
