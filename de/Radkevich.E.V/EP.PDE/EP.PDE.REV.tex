\documentclass[a4paper]{article}
\usepackage[simple,thmnormal]{dmvn}
\usepackage{dmvnadd}
\usepackage{polyglossia}
\usepackage{unicode-math}
\usepackage{fontspec}
\defaultfontfeatures{Scale=MatchLowercase, Mapping=tex-text}
\setmainfont{CMU Serif}
\setsansfont{CMU Sans Serif}
\setmonofont{CMU Typewriter Text}
\setmathfont{xits-math.otf}

\title{Программа экзамена по УрЧП}
\author{Лектор\т Е.\,В.\,Радкевич}
\date{V--VI семестры, 2004--2005 г.}

\def\internalskip{-2}
\def\s#1{~{\textbf{[#1]}}}
\def\nul{{\vrule width 0pt height 1pt}}
\begin{document}
\maketitle

\section{Программа экзамена}

\vbox to .77\textheight{%
\begin{items}{-7}
  \item Гиперболические системы. Линейный анализ
  \vss

  \begin{nums}{\internalskip}
    \item Задача Коши.
    \item Теорема Ковалевской.
    \item Симметризуемые системы. Принцип Годунова.
    \item Решение задачи Коши для симметричной системы.
    \item Обобщённые функции. Основные свойства.
    \item Фундаментальные решения.
    \item Теоремы вложения С.\,Л.\,Соболева.
    \item Обобщённое решение задачи Коши. Априорная оценка.
    \item Классическое решение задачи Коши для волнового уравнения.
    \item Формула Кирхгофа.
    \item Метод спуска. Решение задачи Коши в~случае~${n=2}$. Формула Пуассона.
    \item Качественное исследование формул Кирхгофа, Пуассона, Даламбера. Распространение волн
          в~пространствах разной размерности.
    \item Неоднородное уравнение. Принцип Дюамеля.
  \end{nums}

  \item Некоторые свойства гармонических функций
  \vss

  \begin{nums}{\internalskip}
    \item Субэллиптические и~суперэллиптические функции.
    \item Лемма Олейник о~нормальной производной.
    \item Принцип максимума.
    \item Параболический потенциал.
    \item Задача Неймана.
    \item Вариационные принципы.
  \end{nums}

  \item Уравнение диффузии (параболическое уравнение)
  \vss

  \begin{nums}{\internalskip}
    \item Сравнение диффузионных и~волновых процессов.
    \item Суперпараболические и~субпараболические функции типа потенциала. Принцип максимума.
    \item Единственность решения задачи Коши (принцип Гольмгрена). Стабилизация задачи Коши при $t\ra\bes$.
    \item Параболическая $s,\beta$\д ёмкость.
    \item Поведение решения в~окрестности граничной точки.
    \item Классическое поведение первой краевой задачи (уравнение типа Кордеса).
    \item Оценка производных решения, теорема компактности семейства решений.
    \item Обобщённое решение первой краевой задачи в~произвольно ограниченно области $\R^{n+1}$.
    \item Поведение обобщённого решения в~граничных точках.
  \end{nums}

  \item Гиперболичность. Нелинейный анализ
  \vss

  \begin{nums}{\internalskip}
    \item Уравнения в~частных производных первого порядка: аналитическая и~геометрическая теория.
          Элементы теории катастроф.
    \item Волновые фронты
  \end{nums}

  \item Метод введения параметра и~теорема о~трёх шарах
  \vss

  \begin{nums}{\internalskip}
    \item Неравенство Харнака.
    \item Классы единственности. Теорема Лиувилля. Теорема Фрагмена\ч Линделёфа.
  \end{nums}

  \item Элементы теории потенциалов
  \vss

  \begin{nums}{\internalskip}
    \item Скачки потенциалов (теоремы Сохоцкого).
    \item Вычисление потенциалов.
    \item Ёмкость. Основные свойства. Теоремы об устранимой особенности.
  \end{nums}

\end{items}%
}%\vbox


\section{Ключевые слова из лекций}

Данный список был составлен DMVN Corporation на основе
лекций, прочитанных в~2004--2005 г. Он не претендует на полноту и~абсолютную точность,
а~формулировки не всегда страдают понятностью. Однако по нему можно понять, что именно
было прочитано в~курсе.

\begin{items}{-3}
\item Теорема Коши для ОДУ (существование, единственность, корректность).
\item Мажоранты и~их свойства
\item Теорема Коши\ч Ковалевской. Условие Коши\ч Ковалевской. Доказательство
      для линейных уравнений с~отделяющейся производной по времени.
\item Понятие характеристики. Обобщённая задача коши.
\item Общий вид системы Ковалевской и~приведение её к~нормальному виду.
      Параболический, гиперболический и~эллиптический случай.
      Пример: уравнение Трикоми.
\item Конечность скорости распространения.
\item Теорема единственности.
\item Анализ задачи коши. Энтропия и~поток. Симметризуемые системы.
      принцип максимума Годунова. Примеры.
\item Примеры, показывающие, что на бесконечных временах всё плохо.
\item Корректность по Адамару.
\item Обобщённая производная по Шварцу. Обобщённые функции.
\item Преобразование Фурье обобщённых функций. Пример: дельта\д функция Дирака.
\item Пространства Соболева. Два определения.
\item Носители обобщённых функций. Куча примеров уравнений в~обобщённых функциях.
\item Фундаментальные решения для оператора Лапласа и~для ОДУ.
\item Обобщённые решения линейных уравнений (задача Коши).
\item Теоремы вложения Соболева. Плотное вложение $\Sc \inj \Sc'$.
      Теорема о~следе. Теорема о~компактности.
\item Априорная оценка для нормы решения.
\item Принцип Дюамеля.
\item Теорема единственности для обобщённого решения.
\item Формула Кирхгофа: пример и~доказательство для $n=3$.
\item Метод спуска.
\item Гармонические функции. Формулы Грина.
\item Напоминание про фундаментальное решение для оператора Лапласа.
\item Потенциал Ньютона.
\item Теоремы о~среднем. Принцип максимума.
\item Априорная оценка для производных. Гипоэллиптичность оператора Лапласа.
\item Принцип Бернштейна. Поточечные оценки.
\item Поверхности Ляпунова. Потенциал двойного слоя. Теорема о~скачке.
\item Лемма об аналитическом продолжении.
\item Ещё раз о~максимумах и~минимумах.
\item Обобщенные решения для эллиптических задач Неймана и~Дирихле.
\item Лемма Фридрихса. Барьерные функции. Контрпример к~теореме единственности.
\item Оценка Фридрихса.
\item Неравенство Пуанкаре. Вариационный принцип.
\item Гладкость гармонических функций, теоремы о~среднем, теоремы
      об устранимых особенностях.
\item Лемма Олейник.
\item Параболические уравнения: $\mathrm{sub}$\д решения и~$\mathrm{sup}$\д решения.
      Усиленный принцип максимума.
\item Задача Коши для параболического уравнения. Класс Тихонова.
\item Стабилизация (релаксация) в~простейшем случае.
\end{items}

\begin{thebibliography}{5}
\setlength{\itemsep}{-3pt}
\bibitem{oleinik} О.\,А.\,Олейник. \emph{Лекции об уравнениях с частными производными.}\т М.: БИНОМ, 2005.
\bibitem{shubin} М.\,А.\,Шубин. \emph{Уравнения математической физики.}\т М.: МЦНМО, 2005.
\bibitem{petrovsky} И.\,Г.\,Петровский. \emph{Лекции об уравнениях с частными производными.}\т М.: Физматгиз, 1961.
\bibitem{vladimirov} В.\,С.\,Владимиров. \emph{Уравнения математической физики.}\т М.: Наука, 1988.
\bibitem{landis} Е.\,М.\,Ландис. \emph{Уравнения математической физики.}\т М.: ?Наука, 19??.
\end{thebibliography}


\section{Задачи с экзаменов}

\subsection{Основной экзамен 3 июня 2005 г.}

Правила игры: время\т 2 астрономических часа, разделённых пополам. Первоначальные критерии:
\ctab{|c|c|}{%
\hline \textbf{Баллы} & \textbf{Оценка}\\
\hline $\le 6$ & 2\\
\hline 7--9 & 3\\
\hline 10--12 & 4\\
\hline 13--14 & 5\\
\hline}

Мы гарантируем, что эти условия переписаны без ошибок с~листочков, на которых их раздавали
на экзамене. Мы \emph{не} гарантируем, что люди, делавшие эти листочки, не совершили ни одной ошибки.
Более того, достоверно известно, что в~процессе экзамена обнаружилось, что ошибки там есть.

\subsubsection{Вариант 1}
\setcounter{problem}{0}

\begin{problem}

а)\s1 Дать определение функции Грина для оператора Лапласа в~области $\Om \subs \R^n$.

б)\s2 Даны две области в~$\R^n$: $\ol\Om_1 \subs \Om_2$. Пусть $G_i(x,x_0)$\т функция Грина для области $\Om_i$,
а~$x_0 \in \Om_1$. Доказать, что $G_2(x,x_0) < G_1(x,x_0)$ при всех $x \in \Om_1 \wo \hc{x_0}$.
\end{problem}

\begin{problem}
Дана смешанная задача
\begin{gather*}
u_{tt}-9 u_{xx} = \sin 2x \sin \om t,\\
u\evn{x=0} = u\evn{x=\pi} = 0, \quad u\evn{t=0} = \ph(x), \quad u_t\evn{t=0} = \psi(x).
\end{gather*}

а)\s1 Когда существует периодическое решение этой задачи?

б)\s1 Когда все решения этой задачи периодичны?
\end{problem}

\begin{problem}

а)\s1 Сформулировать теорему о~стабилизации решений задачи Коши для уравнения теплопроводности

б)\s2 Дана начально\д краевая задача для уравнения теплопроводности в~полуполосе:
\begin{gather*}
u_t - u_{xx} = 4,\\
u\evn{x=0} = 3, \quad u\evn{x=1} = 3, \quad u\evn{t=0} = \ph(x).
\end{gather*}
Найти $\liml{t\ra+\bes} u(t,x)$.
\end{problem}

\begin{problem}

\nul\phantom{б)}\s2 Пусть $B := \hc{ |x| < R}$. Пусть существует классическое
решение $u \in \Cb^2(B) \cap \Cb^1(\ol B)$ задачи Неймана
\equ{\De u = -1, \quad u_r\evn{r=R} = \psi.}
Доказать, что функция $\psi$ не может быть положительной во всех точках границы $\pd B$.
\end{problem}


\begin{problem}
Пусть $R := \hc{|x| \le 1, |t| < 2}$.
Про решение уравнения
\equ{u_{tt} - 4 u_{xx}=0, \quad (x,t) \in \R^2}
известно, что

а)\s1 оно равно нулю в~прямоугольнике $R$. Где ещё решение определено однозначно?

б)\s1 оно равно нулю вне прямоугольника $R$. Где оно обязательно равно нулю?
\end{problem}


\begin{problem}

а)\s1 Принадлежит функция $u \equiv 1$ пространству $\os{\circ}{H}{}^1(-1,1)$? Ответ обосновать.

б)\s1 Привести пример непрерывной функции, не принадлежащей пространству $H^1(-1,1)$?
\end{problem}


\subsubsection{Вариант 2}

\setcounter{problem}{0}

\begin{problem}

\nul\phantom{б)}\s2 Пусть $Q$\т ограниченная область с~гладкой границей. Существует ли строго положительное решение
задачи Неймана
\equ{\De u - u = 1, \quad \pf{u}{\nu}\evu{\pd Q}{12pt}{12pt} = 0?}
\end{problem}

\begin{problem}
Дана смешанная задача
\begin{gather*}
u_{tt}-4 u_{xx} = 0,\\
u\evn{x=0} = u\evn{x=1} = 0, \quad u\evn{t=0} = \ph(x), \quad u_t\evn{t=0} = \psi(x).
\end{gather*}

а)\s1 Найти период периодического решения.

б)\s1 Может ли период при некоторых $\ph$ и~$\psi$ быть меньше $\frac1{2005}$?
\end{problem}


\begin{problem}
Дано уравнение
\equ{u_{xx} - 6 u_{xy} + 5 u_{yy} = 2.}

а)\s1 Найти общее решение этого уравнения.

б)\s1 Найти решение задачи Коши для этого уравнения с~начальными условиями
\equ{u\evn{y=0} = 2x^2 + 5x, \quad u_y\evn{y=0} = 2x+1.}
\end{problem}

\begin{problem}

\nul\phantom{б)}\s2 Пусть $Q = \hc{1 < x^2 + y^2 < 2}$. Пусть $u \in \Cb^2(\ol Q)$\т гармоническая функция,
удовлетворяющая граничными условиям
\equ{u\evn{x^2+y^2=2} = x+ y, \quad \hr{\pf{u}{\nu} + (1-y)u}\evu{x^2+y^2=1}{12pt}{12pt} = 0.}
Найти $\maxl{\ol Q} |u|$.
\end{problem}

\begin{problem}
Дана задача Коши
\equ{u_{tt} = \De u, \quad x \in \R^2, \quad t > 0,\quad u\evn{t=0} = \ph(x), \quad u_t\evn{t=0} = \psi(x).}
Пусть $\ph = \psi = 0$ вне кольца $\hc{1 \le |x| \le 2}$.

а)\s1 Где решение заведомо равно нулю при $t = \frac14$ независимо от выбора~$\ph$ и~$\psi$,
удовлетворяющих условиям задачи?

б)\s1 При каких~$t$ существует окрестность точки $x = 0$, в~которой $u = 0$ независимо от выбора
функций $\ph$ и~$\psi$, удовлетворяющих условиям задачи?
\end{problem}

\begin{problem}

а)\s1 Сформулировать теорему о~стабилизации решения задачи Коши для параболического уравнения.

б)\s1 Доказать, что если $u$\т ограниченное решение задачи Коши
\equ{u_t = u_{xx}, \quad u\evn{t=0} = \ph(x), \quad \ph(x) \in \Cb(\R) \cap \Bb(\R),}
причём
\equ{\liml{x \ra+\bes} \frac1x\intl{-x}{x} \ph(t)\,dt = A,}
то $u(0,t) \ra \frac{A}{2}$ при $t \ra+\bes$.

в)\s1 Что можно сказать о~пределе $\liml{t\ra+\bes} u(x,t)$ при $x \neq 0$?
\end{problem}


\subsection{Пересдача 25 июня 2005 г.}

\subsubsection{Вариант 2}
\setcounter{problem}{0}

\begin{problem}
Решить краевую задачу:
\equ{u_{tt} = u_{xx}, \quad x > 0, \quad t > 0, \quad u\evn{x=0,t>0} = t, \quad u\evn{t=0, x\ge 0} = 0, \quad
u_t\evn{t = 0, x \ge 0} = \cos x.}
\end{problem}

\begin{problem}
Решить краевую задачу:
\equ{u_t = u_{xx} + 1, \quad x \in (0,1), \quad t > 0, \quad u(0,t) = 0, \quad u(1,t) = 1,
\quad u(x,0) = x^2 \text{ при } x \in [0,1].}
\end{problem}

\begin{problem}
Решите уравнение $\De u = 1$ в области $\Om$ с граничными условиями, показанными на рисунке:
$$\epsfbox{pictures.1}$$
\end{problem}

\begin{problem}

а) Определение $\os{\circ}{H}{}^1(B_1)$, неравенство Фридрихса в $B_1$. Эквивалентные нормы (формулировки).
Здесь и далее $B_1$\т единичный шар в $\R^n$.

б) При каких $\al \in \R$ функция $|x|^\al \cos\frac{1}{|x|^2}$ лежит в $W_2^1(B_1)$?
\end{problem}

\begin{problem}
Функция $u \in \Cb^\bes[0,\pi)\times (0,\bes)$ является решением уравнения $u_t = u_{xx} - \sin x$
в области $Q = (0,\pi) \times (0,\bes)$ при условиях $u(0,t) = 0$ и $u(\pi,t) = 1$.
Чему равен предел $\liml{t \ra \bes} u(x,t)$, если он существует?
\end{problem}

\begin{problem}
Существует ли решение задачи Неймана $\De u = 1$ в области $\Om$ с граничным условием $\pf{u}{\nu} = 1$?
Ответ обосновать.
\end{problem}


\medskip\dmvntrail
\end{document}
