\documentclass[unicode,12pt,draft]{article}
\usepackage[xe]{dmvn}
\usepackage{dmvnadd}

\tocsubsectionparam{3em}

\begin{document}
%title

\begin{titlepage}
\vskip40pt \centerline{\Large МОСКОВСКИЙ ГОСУДАРСТВЕННЫЙ
УНИВЕРСИТЕТ}
\medskip
\centerline{\Large имени М.\,В.\,ЛОМОНОСОВА}
\medskip
\centerline{\Large Механико--математический факультет}

\par\vskip40pt

\begin{figure}[htbp]
    \centering
    \centerline{\epsfbox{mmlogo.2}}
    %\caption{Подпись к рисунку}
\end{figure}

\par\vskip40pt

\centerline{\Huge\textbf{\textsf{Курс лекций по}}}
\medskip

\centerline{\Huge\textbf{\textsf{уравнениям математической
физики}}}
\medskip
\par\vskip30pt

\centerline{\Large\textbf{\textsf{Лектор --- Андрей Александрович
Коньков}}}
\medskip

\par\vskip70pt

\centerline{III курс, 5--6 семестр, поток механиков}

\par\vskip240pt

\centerline{Москва, 2008 г.}

\end{titlepage}

\newpage

\tableofcontents

\newpage

\begin{center}
\textbf{Предисловие.}\end{center} Этот конспект был набран по
курсу лекций, прочитанному на потоке механиков в 2007--2008
учебному году. Главный \TeX ник --- Анастасия Янгирова, студентка
кафедры Вычислительной механики. Автор конспекта выражает
благодарность Евгению Кудашеву за исправление опечаток и
консультацию по различным аспектам издательской системы \LaTeX.

%lect1

\section*{Введение}

Наши лекции будут посвящены уравнениям математической физики или, как их еще называют,
уравнениям в частных производных.
Мы познакомимся с классификации уравнений второго порядка,
данной Иваном Георгиевичем Петровским.
Далее нам предстоит исследовать задачу Коши для уравнений в частных производных.
Мы узнаем доказательство теоремы Софьи Ковалевской существования и единственности
решения этой задачи.
Для целого ряда уравнений, возникающих в физике и других приложениях,
будут получены точные формулы решений.
Все наши рассуждения будут основаны на едином подходе, использующим теорию обобщенных функций.

Во втором семестре мы познакомимся с пространствами С.\,Л.\,Соболева и научимся с их помощью
решать краевые задачи для эллиптических уравнений.

Мы также подробно остановимся на такой важной области математической физики, как теория гармонических
функций. Докажем теоремы о среднем, принцип максимума, теоремы Харнака и Лиувилля.

Несколько лекций нашего курса мы посвятим параболическим уравнениям.
На этих лекциях вы узнаете, что такое собственная (или параболическая) граница
множества, научитесь доказывать принцип максимума и априорные оценки
для уравнения теплопроводности.

\newpage

\section{Классификация линейных уравнений второго порядка по И.\,Г.\,Петровскому}

В математической физике традиционно изучаются три основных типа уравнений.

\medskip

1) {\itshape Гиперболический тип}. Уравнения этого типа описывают колебательные процессы.
Типичный представитель --- волновое уравнение
$$ \frac{\partial^2 u}{\partial t^2}=k^2 \triangle u+f(x,t), x\in
\mathbb R^n, t>0, \,\mbox {где } \triangle u=\frac{\partial^2
u}{\partial x_1^2}+\ldots+\frac{\partial^2 u}{\partial x_n^2}.$$

2) {\itshape Параболический тип}. Уравнения этого типа описывают явления переноса.
Типичный представитель --- уравнение теплопроводности
$$ \frac{\partial x}{\partial t}=k^2 \triangle u+f(x,t).$$

3) {\itshape Эллиптический тип}. Уравнения этого типа описывают стационарные процессы.
Типичный представитель --- уравнение Пуассона
$$\triangle u=f(x).$$

\medskip

Рассмотрим линейное уравнение второго порядка

$$
    \sum_{i,j=1}^n
    a_{ij}(x)
    \frac{
        \partial^2 u
    }{
        \partial x^i \partial x^j
    }
    +
    \sum_{i=1}^n
    b_{i}(x)
    \frac{
        \partial u
    }{
        \partial x^i
    }
    +
    c(x) u
    =
    f(x),
    \quad
    x = (x^1, \ldots, x^n)
    \in
    \Omega
    \subset
    \mathbb R^n.
    \eqno (1)
$$
В обозначения, принятых в дифференциальной геометрии, уравнение~(1) можно записать в виде
$$
    a^{ij}
    \frac{
        \partial^2 u
    }{
        \partial x^i \partial x^j
    }
    +
    b^{i}
    \frac{
        \partial u
    }{
        \partial x^i
    }
    +
    c u
    =
    f.
$$
При этом мы опускаем знак суммы, подразумевая, что суммирование по одноименным верхним и нижним
индексам происходит автоматически.

Воспользуемся заменой координат $x'=x'(x)$. Очевидно, получим

$$\frac{\partial u}{\partial x^j}=\frac{\partial x^{i'}}{\partial x^j}\frac{\partial u}{\partial x^{i'}},$$

$$\frac{\partial^2 u}{\partial x^i \partial x^j}
=\frac{\partial}{\partial x^i} \left(\frac{\partial u}{\partial
x^j}\right) =\frac{\partial}{\partial x^i}\left(\frac{\partial
{x^{i}}'}{\partial x^j}\frac{\partial u}{\partial {x^{i}}'}\right)
=\frac{\partial^2 {x^{i}}'}{\partial x^i \partial
x^j}\frac{\partial u}{\partial {x^{i}}'}+\frac{\partial
{x^{i}}'}{\partial x^j}\frac{\partial }{\partial
x^i}\left(\frac{\partial u}{\partial {x^i}'}\right)=$$

$$=\frac{\partial^2 x^{j'}}{\partial x^i \partial
x^j}\frac{\partial u}{\partial {x^{j}}'}+\frac{\partial
{x^{j}}'}{\partial x^j}\frac{\partial {x^{i}}'}{\partial
x^i}\frac{\partial^2 u}{\partial {x^{i}}' \partial {x^{j}}'}.$$
Таким образом, уравнение~(1) преобразуется к виду
$$
    a^{ij}
    \frac{
        \partial {x^{i}}'
    }{
        \partial x^i
    }
    \frac{
        \partial {x^{j}}'
    }{
        \partial x^j
    }
    \frac{
        \partial^2 u
    }{
        \partial {x^{i}}' \partial {x^{j}}'
    }
    +
    a^{ij}
    \frac{
        \partial^2 {x^{j}}'
    }{
        \partial x^i \partial x^j
    }
    \frac{
        \partial u
    }{
        \partial {x^{j}}'
    }
    +
    b^i
    \frac{
        \partial {x^{i}}'
    }{
        \partial x^i
    }
    \frac{
        \partial u
    }{
        \partial {x^{i}}'
    }
    +
    c u
    =
    f.
$$
Обозначая
$$
    a^{i'j'}
    =
    a^{ij}
    \frac{
        \partial {x^{i}}'
    }{
        \partial x^i
    }
    \frac{
        \partial {x^{j}}'
    }{
        \partial x^j}
$$
и
$$
    {b^{i}}'
    =
    \frac{
        \partial {x^{i}}'
    }{
        \partial x^i
    }
    +
    a^{ij}
    \frac{
        \partial^2 {x^{j}}'
    }{
        \partial x^i \partial x^j
    },
$$
запишем это уравнение как
$$
    a^{i'j'}
    \frac{
        \partial^2 u
    }{
        \partial {x^{i}}' \partial {x^{j}}'
    }
    +
    {b^{i}}'
    \frac{
        \partial u
    }{
        \partial {x^{i}}'
    }
    +
    c(x) u
    = f(x).
$$

В случае, когда замена координат $x'=x'(x)$ является линейной, мы будем иметь
$$
    {x^{i}}'
    =
    \alpha_i^{i'}
    x^i,
$$
где
$
    \alpha_i^{i'}
    \in
    \mathbb R,
$
$
    \det  \| \alpha_i^{i'} \|
    \ne
    0.
$
Тем самым,
$$
    a^{i'j'}
    =
    \alpha_i^{i'}
    \alpha_j^{j'}
    a^{ij}
$$
и
$$
    b^{i'}
    =
    \alpha_i^{i'} b^i.
$$
Рассмотрим квадратичную форму
$$
    Q (\xi) =
    a^{ij} \xi_i \xi_j,
$$
где $\xi=(\xi_1,\ldots,\xi_n)\in {\mathbb R^n}^*$ ---  ковектор.
Из курса линейной алгебры известно, что вещественные числа
$\alpha_i^{i'}$ можно подобрать так, что замена координат
$$\xi_i=\alpha_i^{i'}\xi_{i'},$$
приводит нашу квадратичную форму к диагональному виду
$$a^{ij}\xi_i\xi_j=\alpha_i^{i'}\alpha_j^{j'}a^{ij}\xi_{i'}\xi_{j'}=a^{i'j'}\xi_{i'}\xi_{j'}=\sum_{i=1}^m
\pm (\xi_{i'})^2,$$
где $m$ --- ранг матрицы квадратичной формы ($m \le n$).

****Рассмотрим $$\sum_{i=1}^m\frac{\partial^2 u}{\partial
{x^i}^2}+\mbox {младшие члены}=f(x).$$

1) {\itshape эллиптический тип}: все коэффициенты
$\sum\limits_{i=1}^m\frac{\partial^2 u}{\partial {x^i}^2}$ имеют
один знак и $m=n$, тогда $\pm \triangle u+\mbox {младшие
члены}=f(x);$

2) {\itshape гиперболический тип}: все коэффициенты имеют один
знак, кроме одного, имеющего противоположный:
$$\frac{\partial^2 u}{\partial {x^n}^2}-\sum_{i=1}^{n-1}\frac{\partial^2 u}{\partial {x^i}^2}+\mbox {младшие
члены}=f(x);$$

3) {\itshape ультрагиперболический тип}: $m=n$ и при этом более
одного члена со знаком $+$ и более одного со знаком $-$,
например:$$\frac{\partial^2 u}{\partial {x^n}^2}+\frac{\partial^2
u}{\partial {x^{n-1}}^2}-\sum_{i=1}^{n-2}\frac{\partial^2
u}{\partial {x^i}^2}+\mbox {младшие члены}=f(x);$$

4) {\itshape параболический тип (в широком смысле)}: $m<n$;

5) {\itshape параболический тип (в узком смысле)}: $m=n-1$,  все
коэффициенты имеют один знак при $i=\overline{1,n-1}$, а при
производной первого порядка коэффициент не равен
нулю:$$\pm\sum_{i=1}^{n-1}\frac{\partial^2 u}{\partial
{x^i}^2}+b^n\frac{\partial u}{\partial
x^n}+\sum_{i=1}^{n-1}b^i\frac{\partial u}{\partial x^i}+cu=f(x),
b^n\ne0.$$

\textbf{Пример.} Уравнение теплопроводности:$$\frac{\partial
u}{\partial t}=\triangle u+f(x,t), \triangle u=\frac{\partial^2
u}{\partial x_1^2}+\ldots+\frac{\partial^2 u}{\partial x_n^2}.$$

Приведем примеры задач Коши.

\textbf{Пример 1}. Гиперболическое уравнение.

Уравнение колебаний натянутой струны со следующими начальными
условиями:$$ \frac{\partial^2 u}{\partial t^2}=k^2 \triangle
u+f(x,t),$$

$$u(x,0)=u_o(x);$$

$$\frac{\partial u}{\partial t}(x,0)=u_1(x).$$

\textbf{Пример 2}. Параболическое уравнение.

Уравнение теплопроводности со следующим начальным условием (задана
температура в начальный момент): $$ \frac{\partial x}{\partial
t}=k^2 \triangle u+f(x,t);$$
$$u(x,0)=u_o(x).$$

%lect2

$$\sum_{i,j=1}^n a_{ij}(x)\frac{\partial^2 u}{\partial x^i \partial x^j}+\sum_{i=1}^n b_{i}(x)\frac{\partial u}{\partial x^i}+c(x)u=f(x), x=(x_1,\ldots,x_n),\eqno (1)$$
С двумя условиями: $$u|_S=\varphi_o,$$ $$\left.\frac{\partial
u}{\partial \nu}\right|_S=\varphi_1,$$ где $S$ ---
гиперповерхность в $\mathbb R^n$, $\nu$ --- нормаль к поверхности.

\textbf{Определение.} Поверхность $S$ называется
\textbf{характеристической} по отношению к уравнению~$(1)$ в точке
$x_0\in S$, если $\sum a_{ij}(x_0)\nu_i\nu_j=0,$ где
$\overline{\nu}=(\nu_1,\ldots\,\nu_n)$ --- вектор нормали к $S$ в
точке $x_0$. Поверхность $S$ называется \textbf{характеристической}
для уравнения~(1), или характеристикой, если она является
характеристической по отношению к уравнению (1) в каждой своей
точке.

\textbf{Пример.} $u_{tt}=u_{xx}.$ Будет ли поверхность $t=0$
характеристикой?

$u_{tt}-u_{xx}=0.$

Должно быть $\sum a_{ij}(x_0)\nu_i\nu_j=0$. Проверим: $\nu_1=0,
\nu_2=1$, тогда $-\nu_1^2+\nu_2^2=1\ne0,$ значит $t=0$
--- свободная поверхность (не~ характеристика). Найдем
характеристики: для вектора
$\overline{\nu}=(-\frac{\sqrt2}2,\frac{\sqrt2}2)$ получим
$-\nu_1^2+\nu_2^2=-\frac12+\frac12=0 \Rightarrow t=x+C$ являются
характеристиками. Аналогично для
$\overline{\nu}=(\frac{\sqrt2}2,-\frac{\sqrt2}2)$ характеристиками
служат $t=-x+C.$

Сделаем замену координат и докажем, что $a^{n'n'}\ne0.$
$$\frac{\partial^2 u}{\partial x^i \partial x^j}=\frac{\partial u}{\partial x^i}\left(\frac{\partial
{x^{j}}'}{\partial x^j}\frac{\partial u}{\partial
{x^{j}}'}\right)=\frac{\partial^2 x^{j'}}{\partial x^i \partial
x^j}\frac{\partial u}{\partial {x^{j}}'}+\frac{\partial
{x^{j}}'}{\partial x^j}\frac{\partial {x^{i}}'}{\partial
x^i}\frac{\partial^2 u}{\partial {x^{i}}' \partial {x^{j}}'}$$
$$a^{ij}\frac{\partial^2 u}{\partial x^i \partial x^j}=\underbrace {a^{ij}\frac{\partial {x^{i}}'}{\partial x^i}\frac{\partial {x^{j}}'}{\partial x^j}}_{a^{i'j'}}\frac{\partial^2 u}{\partial {x^i}' \partial {x^j}'}+\mbox {младшие члены}.$$

Докажем, что $a^{n'n'}\ne0.$

$$\overline r=(x^1(x'),\ldots,x^n(x')), \frac{\partial \overline r}{\partial
{x^n}'}=\left(\frac{\partial x^1}{\partial
{x^n}'},\ldots,\frac{\partial x^n}{\partial {x^n}'}\right).$$

$$a^{n'n'}=a_{ij}\frac{\partial x{^n}'}{\partial x^i}\frac{\partial x{^n}'}{\partial x^j} \eqno(2)$$
$$\sum_{i,j=1}^n a_{ij}(x_0)\nu_i\nu_j\ne0,$$ т.е. $$\sum_{i,j=1}^n
a_{ij}\frac{\partial x^i}{\partial {x^n}'}\frac{\partial
x^j}{\partial {x^n}'}\ne0.$$ Последнее равенство не равносильно
(2).

Векторы перпендикулярны, следовательно $g_{ij}=\delta_{ij}$ ---
метрический тензор.

$$a_{k'p'}=g_{k'i'}g_{p'j'}a^{i'j'},\,\mbox{где } a^{i'j'}=\frac{\partial {x^i}'}{\partial x^i}\frac{\partial {x^j}'}{\partial
x^j}a^{ij},\, g_{p'j'}=\frac{\partial x^p}{\partial
{x^p}'}\frac{\partial x^j}{\partial {x^j}'}g_{pj},\,
g_{k'i'}=\frac{\partial x^k}{\partial {x^k}'}\frac{\partial
x^i}{\partial {x^i}'}g_{ki}.$$ Тогда получим:
$$a_{k'p'}=\frac{\partial x^k}{\partial {x^k}'}\underline{\frac{\partial x^i}{\partial {x^i}'}}g_{ki}\frac{\partial x^p}{\partial {x^p}'}\underline{\underline{\frac{\partial x^j}{\partial {x^j}'}}}g_{pj}\underline{\frac{\partial {x^i}'}{\partial x^i}}\underline{\underline{\frac{\partial {x^j}'}{\partial x^j}}}a^{ij}.$$
$$\mbox{Подчеркнутые множители дают 1, т.к. при фиксированном $i$: } \sum_{i'=1}^n\frac{\partial
x^i}{\partial {x^i}'}\frac{\partial {x^i}'}{\partial
x^i}=\frac{\partial x^i}{\partial x^i}=1.$$ $$\mbox{Аналогично,
если фиксирована $j$}: \sum_{j'=1}^n\frac{\partial x^j}{\partial
{x^j}'}\frac{\partial {x^j}'}{\partial x^j}=\frac{\partial
x^j}{\partial x^j}=1.$$

Таким образом, получим
$$a_{k'p'}=\cfrac{\partial
x^k}{\partial {x^k}'}g_{ki}\cfrac{\partial x^p}{\partial
{x^p}'}g_{pj}a^{ij}=\cfrac{\partial x^k}{\partial
{x^k}'}\cfrac{\partial x^p}{\partial {x^p}'}a_{kp},$$
где
$a_{kp}=g_{ki}g_{pj}a^{ij},\,g_{ki}=\delta_{ki},\,a_{kp}=a^{kp}$, $g_{k'i'}=\delta_{k'i'},\,a_{n'n'}=a^{n'n'}
\Rightarrow a_{n'n'}=\cfrac{\partial x^k}{\partial
{x^n}'}\cfrac{\partial x^p}{\partial {x^n}'}a_{kp}\ne0$
$$a^{i'j'}\frac{\partial^2 u}{\partial {x^{i}}' \partial {x^{j}}'}+{b^{i}}'(x)\frac{\partial u}{\partial
{x^{i}}'}+c(x)u=f(x'),$$
$$a^{n'n'}\ne0,\,\cfrac{\partial^2 u}{\partial {x^{n'}}^2}=\sum_{i',j'=1}^{n-1} \tilde a_{i'j'}\frac{\partial^2 u}{\partial {x^i}' \partial {x^j}'}+\sum_{i'=1}^{n-1} \tilde a_{n'i'}\frac{\partial^2 u}{\partial {x^n}' \partial {x^i}'}+\sum_{i'=1}^n \tilde{b_i}'\frac{\partial u}{\partial {x^i}'}+cu+f$$
Поверхность $S$ в координатах $x'$ будет определяться уравнением
$x^{n'}=0$.

$u({x^1}',\ldots,{x^{(n-1)}}',0)=\varphi_0({x^1}',\ldots,{x^{(n-1)}}'),\,\cfrac{\partial
u}{\partial
{x^n}'}({x^1}',\ldots,{x^{(n-1)}}',0)=\varphi_1({x^1}',\ldots,{x^{(n-1)}}').$

Т.о., без ограничения общности, можно считать, что задача Коши
имеет следующий вид:

$${\begin{aligned}
\left\{%
\begin{array}{ll}
u_{tt}&=\sum\limits_{i,j=1}^n a_{ij}\cfrac{\partial^2 u}{\partial
x^i
\partial x^j}+\sum\limits_{i=1}^n a_{0i}\cfrac{\partial^2 u}{\partial t
\partial x^i}+\sum\limits_{i=1}^n b_i\cfrac{\partial u}{\partial
x^i}+\sum\limits_{i=1}^n b_0\cfrac{\partial u}{\partial t}+cu+f,  \\
u(0,x_1,\ldots\,x_n)&=\varphi_0(x_1,\ldots,x_n),  \\
u_t(0,x_1,\ldots\,x_n)&=\varphi_1(x_1,\ldots,x_n),  \\
\end{array}%
\right.\end{aligned}}\eqno(1)
$$


Перейдем к системе первого порядка. Обозначим $u_0=\cfrac{\partial
u}{\partial t},\,u_i=\cfrac{\partial u}{\partial
x^i},\,i=\overline{1,m},$ получим:

$${\begin{aligned}
\left\{%
\begin{array}{ll}
\cfrac{\partial u_0}{\partial t}&=\sum\limits_{i,j=1}^n
a_{ij}\cfrac{\partial^2 u}{\partial x^i
\partial x^j}+\sum\limits_{i=1}^n a_{0i}\cfrac{\partial^2 u}{\partial t
\partial x^i}+\sum\limits_{i=1}^n b_i\cfrac{\partial u}{\partial
x^i}+\sum\limits_{i=1}^n b_0\cfrac{\partial u}{\partial t}+cu+f,  \\
u(0,x_1,\ldots\,x_m)&=\varphi_0(x_1,\ldots,x_m),\\
u_0(0,x_1,\ldots\,x_m)&=\varphi_1(x_1,\ldots,x_m),\\
u_i(0,x_1,\ldots\,x_m)&=\cfrac{\partial
\varphi_0(x_1,\ldots,x_m)}{\partial x_i}.\\
\end{array}%
\right.\end{aligned}}\eqno(2)
$$

\textbf{Утверждение.} (1) $\Leftrightarrow$ (2).

$\square$

$(\Rightarrow)$ Очевидно.

$(\Leftarrow)$  Пусть $u,u_0,u_1,\ldots,u_m$ --- решение задачи
(2). Докажем, что $u$ является решением задачи (1). Имеем:
$\cfrac{\partial u_0}{\partial t}=\cfrac{\partial^2 u}{\partial
t^2},$ покажем, что $u_i=\cfrac{\partial u}{\partial x_i}.$ При
$t=0$ это так. При $t>0$: $$\frac{\partial
u_i(t,x_1,\ldots,x_n)}{\partial t}=\frac{\partial
u_0(t,x_1,\ldots,x_n)}{\partial x_i}=\frac{\partial}{\partial
t}\frac{\partial u(t,x_1,\ldots,x_n)}{\partial x_i}.$$ Фиксируем
$x_1,\ldots,x_n,$ т.е. $u_i(t,x_1,\ldots,x_n)=\cfrac{\partial
u(t,x_1,\ldots,x_n)}{\partial x_i}.\,\blacksquare$

\section{Аналитические функции}

Пусть $\Omega \subset \mathbb R^n, n\ge1.$

\textbf{Определение.} Пусть $f\colon \Omega \to \mathbb R^n$ ---
\textbf{вещественно аналитическая функция в точке
$x^0=(x_1^0,\ldots,x_n^0)$}, если $f$ разлагается в ряд:
$$f(x_1,\ldots,x_n)=\sum_{k_1,\ldots,k_n}^{\infty}c_{k_1,\ldots,k_n}(x_1-x_1^0)^{k_1}\cdot\ldots\cdot(x_n-x_n^0)^{k_n},$$
который абсолютно сходится в некоторой окрестности точки $x^0.$

\textbf{Определение.} $f\colon \Omega \to \mathbb R^n$ ---
\textbf{вещественно аналитическая функция в области $\Omega$},
если $f$ аналитична в каждой точке области $\Omega$.

$\{(x_1,\ldots,x_n)\colon
|x_1-x_1^0|<\varepsilon_1,\ldots,|x_n-x_n^0|<\varepsilon_n\}$

\textbf{Свойства аналитических функций:}
\begin{enumerate}
    \item $f,g$ --- аналитические в точке $x^0\in\Omega \Rightarrow
    f+g$ --- аналитична в точке $x^0\in\Omega$ (для доказательства надо сложить два ряда и перегруппировать в нем
    члены);
    \item $f,g$ --- аналитические в точке $x^0\in\Omega \Rightarrow
    f\cdot g$ --- аналитична в точке $x^0\in\Omega$ (для доказательства воспользоваться тем, при умножении сходящегося ряда на число он будет продолжать сходиться, и сумма сходящихся рядов снова дает сходящийся ряд);
    \item эту функцию можно разложить в ряд в окрестности точки
    $x^0\in\Omega$;
    \item $f,g$ --- аналитические в точке $x^0\in\Omega, g\ne0 \Rightarrow
    \cfrac fg$ --- аналитична в точке $x^0\in\Omega$;
    \item у каждой аналитической функции существует мажоранта.


\textbf{Определение.} Функция $\tilde f$, аналитичная в точке
$x^0\in\Omega$ называется \textbf{мажорантой функции $f$ в точке
$x^0\in\Omega$}, если $|f_{k_1\ldots k_n}|\le \tilde f_{k_1\ldots
k_n}$ для всех $k_1,\ldots,k_n=0,1,2,\ldots$, где
$$f(x_1,\ldots,x_n)=\sum_{k_1,\ldots,k_n=0}^{\infty}f_{k_1,\ldots,k_n}(x_1-x_1^0)^{k_1}\cdot\ldots\cdot(x_n-x_n^0)^{k_n},$$
$$\tilde f(x_1,\ldots,x_n)=\sum_{k_1,\ldots,k_n=0}^{\infty}\tilde
f_{k_1,\ldots,k_n}(x_1-x_1^0)^{k_1}\cdot\ldots\cdot(x_n-x_n^0)^{k_n}.$$

$\square$ (свойство 5)

Пусть степенной ряд для $f$ сходится абсолютно в $x_1,\ldots,x_n$,
где $|x_i-x_i^0|=a_i>0,\,i=\overline{1,n}.$ Тогда $|f_{k_1\ldots
k_n}a_1^{k_1}\cdot\ldots\cdot a_n^{k_n}|\le M$ для $\forall
k_1,\ldots,k_n=0,1,2,\ldots,\,f_{k_1\ldots k_n} \Rightarrow$

$|f_{k_1\ldots k_n}|\le\cfrac M{a_1^{k_1}\cdot\ldots\cdot
a_n^{k_n}}$ для $\forall k_1,\ldots,k_n=0,1,2,\ldots,$ $\tilde
f(x_1,\ldots,x_n)=\cfrac M{(1-\frac{x_1}{a_1})\cdot\ldots\cdot
(1-\frac{x_n}{a_n})},$
$$\frac1{1-\frac{x_i}{a_i}}=\sum_{k_i=0}^{\infty}\left(\frac{x_i}{a_i}\right)^{k_i}, |x_i-x_i^0|<a_i,$$
$$\tilde f(x_1,\ldots,x_n)=\sum_{k_1,\ldots,k_n=0}^{\infty}\frac M{a_1^{k_1}\cdot\ldots\cdot
a_n^{k_n}}(x_1-x_1^0)^{k_1}\cdot\ldots\cdot(x_n-x_n^0)^{k_n}.$$

Другие мажоранты: $U(x)=\cfrac M{1-\cfrac{\tilde x_1+\ldots+\tilde
x_n}a},$ где
$a=\min\limits_{x=(x_1,\ldots,x_n)}\{a_1,\ldots,a_n\}.$
$$U(x)=M\sum_{k_i=0}^{\infty}\left(\cfrac{\tilde x_1+\ldots+\tilde
x_n}a\right)^k=\sum_{k_i=0}^{\infty}\sum_{k_1+\ldots+k_n=k}\frac
M{a^k}\frac{k!}{k_1!\cdot\ldots\cdot k_n!}\tilde
x_1^{k_1}\cdot\ldots\cdot \tilde x_1^{k_1}.$$ Так как
$\cfrac{k!}{k_1!\cdot\ldots\cdot k_n!}\ge1,$ и $U(x)$ ---
мажоранта для $\tilde f\ \Rightarrow,$ то она является мажорантой
и для $f$ тоже. $\blacksquare$

\item $f$ --- аналитична в точке $x^0=(x_1^0,\ldots,x_n^0)$, то $f$ аналитична в некоторой окрестности точки
$x_0$.

$\square$ Пусть
$$f(x_1,\ldots,x_n)=\sum_{k_1,\ldots,k_n}^{\infty}f_{k_1\ldots k_n}(x_1-x_1^0)^{k_1}\cdot\ldots\cdot(x_n-x_n^0)^{k_n},\eqno (1)$$
$$y^0=(y_1^0,\ldots,y_n^0) \Rightarrow (x_i-x_i^0)^{k_i}=(x_i-y_i+y_i^0-x_i^0)^{k_i}=\sum_{k=0}^{k_i}c_k^{k_i}(y_i^0-x_i^0)^{k_i-k}(x_i-y_i^0)^k.\eqno (2)$$

Подставляем $(2)\to(1)$ и собираем подобные члены при
$(x_1-x_1^0)^{k_1}\cdot\ldots\cdot(x_n-x_n^0)^{k_n}.$

Почему перестановка членов корректна?

Рассмотрим мажоранту $\tilde f$ функции $f$ в точке $x^0$:
$$\tilde f(x_1,\ldots,x_n)=\sum_{k_1,\ldots,k_n=0}^{\infty}\tilde
f_{k_1,\ldots,k_n}(x_1-x_1^0)^{k_1}\cdot\ldots\cdot(x_n-x_n^0)^{k_n}.$$

Выберем $\tilde y_i^0, \tilde x_i^0$ такими,чтобы $\tilde y_i^0-
x_i^0=|y_i^0-x_i^0|,\, \tilde x_i^0-y_i^0\ge|x_i-y_i^0|,\,\tilde
x_i-x_i^0=|x_i-x_i^0|,$ $\tilde x_i\le\tilde y_i^0\tilde x_i^0.$

Ряд для $\tilde f(\tilde x_1,\ldots,\tilde x_n)$ мажорирует ряд
для $f(x_1,\ldots,x_n)$ и в нем можно переставить члены, раскрывая
скобки:
$$(\tilde x_i-\tilde y_i+\tilde y_i^0-x_i^0)^{k_i}=(\tilde x_i-x_i^0)^{k_i}=\sum_{k=0}^{k_i}c_{k_i}^k(\tilde y_i^0-x_i^0)^{k_i-k}(\tilde x_i-\tilde
y_i^0)^k. \blacksquare$$

\item $g$ --- аналитична в точке $x^0=(x_1^0,\ldots,x_n^0),$ $f$ --- аналитична в точке
$g(x^0),$ то $f(g(x))$ аналитична в точке $x^0$.

\textbf{Упражнение.} Доказать свойства 4 и 7.

\item \textbf{Упражнение.}
$$f(x_1,\ldots,x_n)=\sum_{k_1,\ldots,k_n}^{\infty}c_{k_1\ldots k_n}(x_1-x_1^0)^{k_1}\cdot\ldots\cdot(x_n-x_n^0)^{k_n},$$
--- аналитична в точке $x^0=(x_1^0,\ldots,x_n^0)$. Тогда $c_{k_1\ldots k_n}=\cfrac1{k_1!\cdot\ldots\cdot k_n!}\cfrac{\partial^{k_1+\ldots+k_n}f(x_1^0,\ldots,x_n^0)}{\partial x_1^{k_1}\ldots\partial
x_n^{k_n}}$.

\end{enumerate}

\section{Аналитическая поверхность. Задача Коши\\
для систем первого порядка}

$$\frac{\partial u_i}{\partial t}=\sum_{\substack{k=1,\ldots,m\\j=1,\ldots,N}}a_{ij}^k \frac{\partial u_j}{\partial x_k}+\sum_{j=1}^N b_{ij}(x)u_j+c(x),\,i=\overline{1,N},\eqno (1)$$
$$u_i(0,x_1,\ldots,x_n)=\varphi_i(x_1,\ldots,x_n),\, i=\overline{1,N}$$

\textbf{Теорема (Коши--Ковалевской).} Пусть
$a_{ij}^k,b_{ij},c,\varphi_i$ --- аналитические функции в точке
$0=(\underbrace {0,\ldots,0}_{m+1}).$ Тогда в некоторой
окрестности $U(0)$ существует единственное аналитическое решение
задачи (1).

\textbf{Замечание.} Без ограничения общности, можно считать ,что
$\varphi_i\equiv0,\,i=\overline{1,N}.$ В противном случае сделаем
замену:
$v_i(t,x_1,\ldots,x_n)=u_i(t,x_1,\ldots,x_n)-\varphi_i(t,x_1,\ldots,x_n).$

$\square$

(Единственность.) Пусть $u_i$ --- аналитическое решение системы
(1):
$$u_i(t,x_1,\ldots,x_n)=\sum_{k_1,\ldots,k_n}^{\infty}c_{k_1\ldots k_n}x_1^{k_1}\cdot\ldots\cdot x_m^{k_m},$$
Тогда $c_{i0\ldots0}=u_i(0,\ldots,0)=0,$ так как $\varphi_i=0.$
$$c_{i0k_1\ldots k_m}=\frac1{k_1!\cdot\ldots\cdot k_n!}\left.\frac{\partial u_i(0,x_1,\ldots,x_m)}{\partial x_1^{k_1}\ldots\partial
x_m^{k_n}}\right|_{x_1=\ldots=x_m=0}\equiv0.$$
\begin{multline}
\frac{\partial^{k_1+\ldots+k_n}u}{\partial t \partial x_1^{k_1}\ldots\partial
x_n^{k_n}}=\sum_{\substack{k=1,\ldots,m\\i=j=1,\ldots,N}}\frac{\partial}{\partial
x_1^{k_1}\ldots\partial x_n^{k_n}}\left(a_{ij}^k\frac{\partial
u_j}{\partial
x_k}\right)+\\+
\sum_{j=1}^N\frac{\partial^{k_1+\ldots+k_n}
(b_{ij}u_j)}{\partial x_1^{k_1}\ldots\partial
x_n^{k_n}}+\left.\frac{\partial^N c}{\partial
x_1^{k_1}\ldots\partial x_n^{k_n}}\right|_{t=x_1=\ldots=x_m=0}.
\end{multline}
Аналогично:
$$c_{i k_0k_1\ldots
k_m}=\left.\frac{\partial^{k_0+k_1+\ldots+k_n}u}{\partial t^{k_0}
\partial x_1^{k_1}\ldots\partial x_n^{k_n}}\right|_{t=x_1=\ldots=x_m=0}.$$
\begin{multline}
\sum_{\substack {k=1,\ldots,m\\j=1,\ldots,N}}\frac{\partial^{k_0-1+k_1+\ldots+k_m}}{\partial t^{k_0-1}\partial
x_1^{k_1}\ldots\partial x_n^{k_n}}\left(a_{ij}^k\frac{\partial
u_j}{\partial
x_k}\right)+\\+
\sum_{j=1}^N\frac{\partial^{k_0-1+k_1+\ldots+k_m}
(b_{ij}u_j)}{\partial t^{k_0-1}\partial x_1^{k_1}\ldots\partial
x_n^{k_n}}+\left.\frac{\partial^{k_0-1+k_1+\ldots+k_m} c}{\partial
t^{k_0-1}\partial x_1^{k_1}\ldots\partial
x_n^{k_n}}\right|_{t=x_1=\ldots=x_m=0}.
\end{multline}

(Существование.) Возьмет функцию $W(t,x_1,\ldots,x_m)=\cfrac
M{1-\cfrac{t+x_1+\ldots+x_m}{a_k}}$. $W$ является мажорантой для
всех коэффициентов $a_{ij}^k,b_{ij},c.$ Тогда $W(\cfrac
t2,x_1,\ldots,x_m)$ так же является мажорантой для
$a_{ij}^k,b_{ij},c$, если $\alpha \in (0,1).$ В самом деле, пусть:
$$W(t,x_1,\ldots,x_m)=\sum_{k_0,k_1,\ldots,k_m=0}^{\infty}W_{k_0 k_1\ldots k_m}t^{k_0}x^{k_1}\ldots x_m^{k_m}.$$
Тогда $$W_{\alpha}\Bigl(\cfrac t
{\alpha},x_1,\ldots,x_m\Bigr)=W\Bigl(\cfrac
t2,x_1,\ldots,x_m\Bigr)=\sum_{k_0,k_1,\ldots,k_m=0}^{\infty}W_{k_0
k_1\ldots k_m}\frac {t^{k_0}}{\alpha^{k_0}}x^{k_1}\ldots
x_m^{k_m}.$$ При этом, очевидно, $\cfrac {W_{k_0 k_1\ldots
k_m}}{\alpha^{k_0}}\ge W_{k_0 k_1\ldots k_m}.$

Напишем мажорирующую систему:
$$\frac{\partial u_i}{\partial t}=
\sum_{\substack{k=\overline {1,m}\\\,j=\overline {1,N}}}W
\frac{\partial u_j}{\partial x}+\sum_{i=1}^N W u+W.$$

Будем искать $u_i$ в виде:
$u_1(t,x_1,\ldots,x_m)=\ldots=u_N(t,x_1,\ldots,x_m)=u(\cfrac t
{\alpha}+x_1+\ldots+x_m)=u(z),$ где $z=(\cfrac t
{\alpha}+x_1+\ldots+x_m).$

$\cfrac1{\alpha}u'(z)=N m u'(z)\cdot A(z)+N A(z) \cdot u(z)+A(z),
\, A(z)=\cfrac M{1-\frac za},$

$(\cfrac1{\alpha}-N m A(z))u'(z)=N A(z) u(z)+A(z),$

$W(t,x_1,\ldots,x_m)=\cfrac M{1-\cfrac{t+x_1+\ldots+x_m}a},\,
W_{\alpha}(t,x_1,\ldots,x_m)=W(\cfrac t {\alpha},x_1,\ldots,x_m).$

$\cfrac {du}{Nu+1}=\cfrac {A(z)dz}{\cfrac1{\alpha}-N m A(z)}.$

Выбирая $\alpha \in (0,1)$ достаточно малым добьемся того, чтобы
$\cfrac1{\alpha}-N m A(z)\ge0$ в окрестности точки $x=0$. Тогда
$B(z)=\cfrac {A(z)}{\cfrac1{\alpha}-N m A(z)}$ раскладывается в
окрестности точки $x=0$ в ряд по степеням $z$ с неотрицательными
коэффициентами.

$$\frac1N\cdot \ln(Nu+1)=\int_0^zB(\zeta) d\zeta,$$

$$u(z)=\frac{e^{N\int_0^zB(\zeta) d\zeta-1}}N$$ --- имеет в
разложении по степеням $z$ в окрестности $z=0$ неотрицательные
коэффициенты, так как $$e^{N\int_0^zB(\zeta)
d\zeta-1}=1+\sum_{k=1}^{\infty}\frac1{k!}(N\int\limits_0^zB(\zeta)
d\zeta).$$ Ряд для $u$ мажорирует формальный ряд для решения
исходной системы и, следовательно, он сходится.$\blacksquare$


\section{Обобщенные функции}

\textbf{Пример.}

Рассмотрим поверхность с трением: брусок падает под углом
$\alpha_n$ и отражается от поверхности. Под каким наибольшим углом
$\alpha_0$ он отразится?

Скорость $\overline v=(v_x,v_y),$ импульс $\overline p=m\overline
v=(p_x,p_y)=(mm_x,mm_y),$

 $\tg\alpha_n=\cfrac {v_y}{v_x}=\cfrac
{p_y(\infty)}{p_x(\infty)},\,\tg\alpha_0=?$

$\cfrac {dS_y}{dt}=F_y \Rightarrow
p_y(t)=p_y(-\infty)+\int\limits_{-\infty}^t F_dt$ (считаем, что
при $t=0$ происходит соударение)

$p_y(\infty)=p_y(-\infty)+\int\limits_{-\infty}^{\infty}
F_dt=-p_y(\infty) \Rightarrow $ $$\int\limits_{-\infty}^{\infty}
F_dt=-2p_y(-\infty),\eqno (1)$$

$k$ --- коэффициент трения $\Rightarrow
p_x(t)=p_x(-\infty)-\int\limits_{-\infty}^t k(t)F_y(t)dt,
\Rightarrow$

$p_x(\infty)=p_x(-\infty)-\int\limits_{-\infty}^{\infty}
k(t)F_y(t)dt$

$k$ зависит от $t$, но меняется со временем очень медленно, т.е.
при $t\in(-\varepsilon,\varepsilon)$

$ k(t)=k(0)=\const   \Rightarrow
p_x(\infty)=p_x(-\infty)-k(0)\int\limits_{-\infty}^{\infty}
F_y(t)dt=p_x(-\infty)+2k(0)p_y(-\infty) \Rightarrow$

$\tg \alpha_o=\cfrac {p_y(\infty)}{p_x(\infty)}=\cfrac
{p_y(\infty)}{p_x(-\infty)+2k(0)p_y(-\infty)}=\cfrac {-\cfrac
{p_y(-\infty)}{p_x(-\infty)}}{1+2k(0)\cfrac
{p_y(-\infty)}{p_x(-\infty)}}=\cfrac {-\tg \alpha_n}{1+2k(0)\tg
\alpha_n}.$

Рассмотрим $\delta(t)$ --- \textbf{дельта-функция Дирака}:
$\delta(0)=\infty, \delta(x)=0, x\ne0,
\int\limits_{-\infty}^{\infty} \delta(t)dt=1.$

Если $F_y(t)=-2p_y(\infty)\delta(t)$, то получим (1).

\textbf{Определение.} \textbf{Носитель функции $\varphi\in
{C^\infty}(\mathbb R^n)$} --- замыкание множества

$\{x\in \mathbb R^n\colon\varphi(x)\ne0\}$.

Обозначение: $\supp \varphi.$

\textbf{Определение.} Множество функций $\varphi\in
{C^\infty}(\Omega)$ таких, что $\supp \varphi$ является компактом
называется \textbf{множеством (пространством) основных функций
$D(\Omega)$}, где $(\Omega)$ --- открытое подмножество $\mathbb
R^n,\,n\ge1.$

\textbf{Определение.} Последовательность функций $\varphi_k\in
D(\Omega),\,k=1,2,\ldots,$ \textbf{сходится} к функции $\varphi\in
D(\Omega),$ если:

1) существует компакт $H\subset\Omega\colon \supp \varphi_k\subset
H,\,k=1,2,\ldots,$

2) $\forall m
\|\,\varphi_k-\varphi\|_{C^m(\Omega)}\to0,\,k\to{\infty},$ где
$\|\,\psi\|_{C^m(\Omega)}=\sup\limits_{x\in\Omega}
\sum\limits_{|\,\alpha|\le m} |\,\partial^2\psi(x)|,$
$\alpha=(\alpha_1,\ldots,\alpha_n)$ --- мультииндекс,
$|\,\alpha|=\alpha_1+\ldots+\alpha_n, \,\partial^{\,\alpha}=\cfrac
{\partial^{\,|\alpha|}}{\partial x_1^{\,\alpha_1}\ldots \partial
x_1^{\,\alpha_n}}, \alpha_i$ --- целые неотрицательные числа.

\textbf{Определение.} Функционал $f\colon D(\Omega)\to\mathbb C$
называется \textbf{непрерывным}, если из того, что
$\varphi_k\to\varphi,\,k\to{\infty}$ в $D(\Omega)$ следует, что
$f(\varphi_k)\to f(\varphi),\,k\to{\infty}.$

\textbf{Определение.} \textbf{Пространство обобщенных функций} ---
множество линейных непрерывных функционалов $f\colon
D(\Omega)\to\mathbb C^n.$

$D(\Omega), D'(\Omega)$ --- линейные пространства над $\mathbb
C^n$.

$f_1,f_2\in D'(\Omega), \lambda_1,\lambda_2\in \mathbb C^n,
\varphi\in D(\Omega)\colon\,
(\lambda_1f_1+\lambda_2f_2)(\varphi)=\lambda_1f_1(\varphi)+\lambda_2f_2(\varphi).$

$\delta(t)$ --- тоже функционал!

\textbf{Определение.} Пусть $f\in D'(\Omega), \varphi\in
D(\Omega)$, тогда $(f,\varphi)=f(\varphi)$ --- результат действия
функционала на основную функцию $\varphi$.

$(\delta(x),\varphi(x))=\varphi(0)$ --- определение дельта-функции
Дирака.

Пусть $L_{1,loc}(\Omega)$ --- множество измеримых функций на
открытом множестве $\Omega\subset\mathbb R^n$, суммируемых на
всяком компакте $H\subset\Omega$. Вложим $L_{1,loc}(\Omega)$ в
$D(\Omega)$ следующим образом: для любого $f\in L_{1,loc}(\Omega)$
определим функционал $f\colon D(\Omega)\to\mathbb C^n$ формулой:
$$f\colon\varphi\to\int\limits_{\Omega}f\varphi dx, \, \varphi\in D(\Omega)$$

($f\varphi$ --- тоже измерима и $\,=0$ вне носителя функции
$\varphi$ (компакт)). На самом деле интеграл берется  не по
$\Omega$, а по компакту --- носителю $\varphi$.

\textbf{Упражнение*.} Доказать, что если
$\int\limits_{\Omega}f\varphi dx=0,\,\forall\varphi\in D(\Omega),$
то $f=0$ почти всюду в $\Omega$.

%lect6

\section{Действия над обобщенными функциями}

Пусть $f,g\in D'(\Omega),\,f,g\colon D(\Omega)\to\mathbb C^n$ ---
линейные непрерывные функционалы.

\begin{enumerate}
\item $(f+g)(\varphi)=f(\varphi)+g(\varphi)$ или
$(f+g,\varphi)=(f,\varphi)+(g,\varphi)$;
\item $(\lambda f)(\varphi)=\lambda f(\varphi)$ или $(\lambda
f,\varphi)=\lambda(f,\varphi)$;
\item $f\in D(\Omega),\,\psi\in C^{\infty}(\Omega) \Rightarrow
(f\psi,\varphi)=(f,\psi\varphi)$;
\item В частности, если $f,\psi\in C^{\infty}(\Omega)$, то
$(f\psi,\varphi)=\int\limits_{\Omega}(f\psi)\varphi\,
dx=\int\limits_{\Omega}f(\psi\varphi)\,dx=(f,\psi\varphi)$;
\item Дифференцирование обобщенных функций: если $f\in D'(\Omega),\,f\in C^{\infty}(\Omega),\,\varphi\in
D(\Omega)$, то

$\Bigl(\cfrac{\partial f}{\partial
x_i}\,,\varphi\Bigr)=\int\limits_{\Omega}\cfrac {\partial
f}{\partial x_i}\,\,\varphi\,dx=-\int\limits_{\Omega}f\cfrac
{\partial \varphi}{\partial x_i}\,dx=-\Bigl(f,\cfrac {\partial
\varphi}{\partial x_i}\Bigr)$

Считаем это определением: $(f\psi,\varphi)=-(f,\psi\varphi).$

\textbf{Пример.} $$\Theta(x)=\begin{cases} 1, x>0;\\ 0,
x<0.\end{cases}.$$ Очевидно, что $\Theta\in L_{1,loc}(\mathbb
R^n)$. Для $\forall \varphi \in D(\mathbb R^n)$, учитывая $\supp
\varphi\subset(-a,a)$, имеем:

$(\Theta'(x),\varphi(x))=-(\Theta(x),\varphi'(x))=-\int\limits_{-\infty}^{\infty}\Theta(x)\varphi'(x)\,dx=-\int\limits_{-a}^{a}\Theta(x)\varphi'(x)\,dx=$

$=-\int\limits_{-a}^0\underbrace{\Theta(x)}_{\equiv0}\varphi'(x)\,dx-\int\limits_0^a\underbrace{\Theta(x)}_{\equiv1}\varphi'(x)\,dx=-\int\limits_0^a\varphi'(x)\,dx=-\varphi(a)+\varphi(0)=\varphi(0).$

Таким образом, получаем:
$(\Theta'(x),\varphi(x))=\varphi(0)=(\delta(x),\varphi(x)).$
\item $f,g\in D'(\Omega) \Rightarrow \cfrac{\partial}{\partial
x_i}\,(f+g)=\cfrac{\partial f}{\partial x_i}+\cfrac{\partial
g}{\partial x_i}.$

$\square$

Для $\forall \varphi \in D$ имеем:

$\left(\cfrac{\partial}{\partial
x_i}\,(f+g),\varphi\right)=\mbox{по свойству
(5)}=-\left(f+g,\cfrac{\partial\varphi}{\partial
x_i}\right)=\mbox{по свойству (1)}=$

$=-\left(f,\cfrac{\partial\varphi}{\partial
x_i}\right)-\left(g,\cfrac{\partial\varphi}{\partial
x_i}\right)=\mbox{по свойству (5)}=\left(\cfrac{\partial
f}{\partial x_i}\,,\varphi\right)+\left(\cfrac{\partial
g}{\partial x_i}\,,\varphi\right)=\mbox{по свойству (1)}=$

$=\left(\cfrac{\partial f}{\partial x_i}+\cfrac{\partial
g}{\partial x_i}\,,\varphi\right). \blacksquare$

\item $f\in D'(\Omega), \psi\in C^{\infty}(\Omega)\Rightarrow \cfrac{\partial}{\partial
x_i}\,(f\psi)=\cfrac{\partial f}{\partial
x_i}\,\psi+f\cfrac{\partial\psi}{\partial x_i}.$

$\square$

Для $\forall \varphi \in D$ имеем:

$(\cfrac{\partial}{\partial x_i}\,(f\psi),\varphi)=\mbox{по
свойству (5)}=-(f\psi,\cfrac{\partial\varphi}{\partial
x_i})=\mbox{по свойству
(4)}=(-f,\psi\cfrac{\partial\varphi}{\partial x_i}).$

C другой стороны, $\left(\cfrac{\partial f}{\partial
x_i}\,\psi+f\cfrac{\partial\psi}{\partial
x_i},\varphi\right)=\mbox{по свойству (1)}=\left(\cfrac{\partial
f}{\partial x_i}\,\psi,\varphi\right)+f\left(\cfrac{\partial
\psi}{\partial x_i},\varphi\right)=$

$=\mbox{по свойству (4)}=\left(\cfrac{\partial f}{\partial
x_i}\,\psi,\varphi\right)+\left(f,\cfrac{\partial \psi}{\partial
x_i}\,\varphi\right)=\mbox{по свойствам (1), (5)}=$

$=-\left(f,-\cfrac{\partial}{\partial
x_i}\,(\psi\varphi)+\cfrac{\partial\psi}{\partial
x_i}\,\varphi\right)=\left(f,-\cfrac{\partial\psi}{\partial
x_i}\,\varphi-\psi\cfrac{\partial\varphi}{\partial
x_i}+\cfrac{\partial\psi}{\partial
x_i}\,\varphi\right)=-\left(f,\psi\cfrac{\partial\varphi}{\partial
x_i}\right).\blacksquare$

\end{enumerate}

\textbf{Упражнение.} Показать ,что при умножении обобщенной
функции на бесконечно гладкую, получается обобщенная функция, т.е.
линейный  непрерывный функционал.

\textbf{Упражнение.} Показать ,что при дифференцировании
обобщенной функции, получается обобщенная функция.

\section{Замена переменной у обобщенной функции}

$f\in C^{\infty}(\Omega),f\in D'(\Omega)$

$x=x(x')\colon \Omega'\to\Omega$ --- диффеоморфизм

$f(x)\colon \varphi(x)\to\int\limits_{\Omega}f(x(x'))\varphi (x)
dx,\,\varphi(x)\in D(\Omega)$

$$f(x(x'))\colon \psi(x')\to\int\limits_{\Omega'}f(x(x'))\psi (x')
dx',\eqno(1)$$

$dx'=\Bigl|\det \,\|\cfrac{\partial x'}{\partial x}\|\Bigr|\,dx,\,
dx'={dx^1}'\ldots{dx^n}',\,dx={dx^1}\ldots{dx^n},$

$$\Bigl\|\cfrac{\partial x'}{\partial x}\Bigr\|=\begin{vmatrix}
\,\,\,\,\cfrac{\partial {x^1}'}{\partial x^1}&\cdots& \cfrac{\partial {x^1}'}{\partial x^n}&\\
\,\,\,\,\vdots&\cdots&\vdots&\\
\,\,\,\,\cfrac{\partial {x^n}'}{\partial x^1}&\cdots&
\cfrac{\partial {x^n}'}{\partial x^n}&
\end{vmatrix}$$

$(1)=\int\limits_{\Omega}f(x)\underbrace{\psi (x'(x))\Bigl|\det
\,\|\cfrac{\partial x'}{\partial
x}\|\Bigr|}_{=\varphi(x)}\,dx=\left(f(x),\psi(x'(x))\Bigl|\det
\,\|\cfrac{\partial x'}{\partial x}\|\Bigr|\right).$

\textbf{Определение.}
$(f(x(x')),\psi(x'))=\left(f(x),\psi(x'(x))\Bigl|\det
\,\|\cfrac{\partial x'}{\partial
x}\|\Bigr|\right)=\left(f(x),\cfrac{\psi(x'(x))}{\Bigl|\det
\,\|\cfrac{\partial x}{\partial x'}\|\Bigr|}\right).$

\textbf{Пример.} $f(x)=\delta(x).$

$x=ax'+b \Rightarrow x'=\cfrac{x-b}a\Rightarrow
\cfrac{dx'}{dx}=\cfrac1a$ и
$\Bigl|\cfrac{dx'}{dx}\Bigr|=\cfrac1{|a|} \Rightarrow$

$$(\delta(ax'+b),\psi(x'))=\Bigl(\delta(x),\psi\Bigl(\cfrac{x-b}a\Bigr)\cdot\cfrac1{|a|}\,\Bigr)=\cfrac1{|a|}\,\psi\Bigl(-\cfrac
ba\Bigr)$$

Пусть $a=1, b=-x_0' \Rightarrow
(\delta(x'-x_0'),\psi(x'))=\psi(x_0')$

$x'_0=0 \Rightarrow
\int\limits_{-\infty}^{\infty}\psi(x)\delta(x)dx=\psi(0).$

\textbf{Пример.} $x=Ax'+b,\,x'\in \mathbb R^n, \,b\in\mathbb R^n,
$ матрица $A\colon \mathbb R^n\to\mathbb R^n, \,f\in D'(\mathbb
R^n) \Rightarrow $
$$(f(x(x')),\psi(x')))=(f(Ax'+b),\psi(x'))=\left(f(x),\cfrac{\psi(A^{-1}(x-b))}{|\,\det
A|}\right).$$

%lect7

Пусть $f\in C(\Omega), \Omega\subset \mathbb R^n, n\ge1, \Omega$
--- открытое, непустое, $\supp f=\{x\in \Omega\colon f(x)\ne0\}.$

Пусть теперь $f\in D'(\Omega)$, $(f,\varphi)=0$
$$f\colon \varphi\to\int\limits_{\Omega}f\varphi\,dx, \supp \varphi\subset \omega,\, \omega=\Omega \setminus \supp f$$

\textbf{Определение.} \textbf{Ограничением функции $f\in D'
(\Omega)$} на произвольном открытом подмножестве
$\omega\subset\Omega$ называется обобщенная функция
$f|_{\,\omega}\in D' (\omega)\colon
(f|_{\,\omega},\varphi)=(f,\varphi), \,\forall \varphi \in D
(\omega).$

\textbf{Определение.} \textbf{Носитель функции $f\in D' (\Omega)$}
--- множество $\supp f=\Omega \setminus \omega_{max},$ где
$\omega_{max}$ --- максимальное открытое подмножество
$\Omega\colon f|_{\,\omega}=0.$

\textbf{Утверждение 1.} Множество $\omega_{max}$ существует.

\textbf{Утверждение 2.} Пусть $\omega_1, \omega_2\subset
\Omega\colon f|_{\,\omega_1}=0, f|_{\,\omega_2}=0,$ где $f\in
D'(\Omega)$ --- некоторая обобщенная функция. Тогда
$f|_{\,\omega_1\cup\omega_2}=0.$

$\square$

Возьмем $\varphi\in D(\omega_1\cup\omega_2).$ Надо показать, что
$(f,\varphi)=0.$

$\supp \varphi\Subset \omega_1\cup\omega_2$ --- означает
одновременно,что $\supp \varphi$ является компактом и он (этот
компакт) принадлежит $\omega_1\cup\omega_2.$

Имеем: $\supp \varphi \setminus \omega_2 \Subset \omega_1.$ Таким
образом, существует открытое множество $v_1:$

$\supp \varphi \setminus \omega_2 \subset v_1,$ замыкание
$\overline {v_1}\Subset \omega_1.$

Очевидно, что $\supp \varphi \subset v_1\cup\omega_2.$ При этом
$\supp \varphi \setminus v_2\subset v_2.$ Тем самым, имеем $\supp
\varphi\subset v_1\cup v_2.$

Возьмем неотрицательную функцию $\psi_1\in D(\omega_1)\colon
\psi_1|_{\,v_1}\equiv1.$ Аналогично, возьмем неотрицательную
функцию $\psi_2\in D(\omega_2)\colon \psi_2|_{\,v_2}\equiv1.$

Пусть $\varphi_1=\begin{cases} \cfrac{\varphi(x)\psi_1(x)}{\psi_1(x)+\psi_2(x)}, x\in v_1\cup v_2;\\
0, x\in \mathbb R^n\setminus(v_1\cup v_2),\end{cases} \varphi_2=\begin{cases} \cfrac{\varphi(x)\psi_2(x)}{\psi_1(x)+\psi_2(x)}, x\in v_1\cup v_2;\\
0, x\in \mathbb R^n\setminus(v_1\cup v_2),\end{cases} \Rightarrow$

$\varphi_1, \varphi_2 \in C^{\infty}(\mathbb R^n)\ \Rightarrow$
\begin{enumerate}
\item Если $x\in \supp \,\varphi,$ то существует $U(x)\subset v_1\cup
v_2;$ если $x\notin \supp \,\varphi,$ то $\varphi(x)=0
\Rightarrow$

$\varphi_1(x)=\varphi_2(x)\equiv0.$
\item $\supp \,\varphi_1\Subset \omega_1, \supp \,\varphi_2\Subset
\omega_2,$ так как $\supp \,\psi_1\Subset \omega_1, \supp
\,\psi_2\Subset \omega_2.$
\item $\varphi_1(x)+\varphi_2(x)=\varphi(x),\,\forall x\in \Omega$
--- очевидно.
В свою очередь, если $x\notin v_1\cup v_2,$ то
$\varphi_1(x)=\varphi_2(x)=\varphi(x)\equiv0,$ так как $\supp
\,\varphi \subset v_1\cup v_2.$ Поэтому опять получим, что
$\varphi_1(x)+\varphi_2(x)=\varphi(x).$ Таким
образом,$(f,\varphi)=(f,\varphi_1+\varphi_2)=\underbrace{(f,\varphi_1)}_{=0,\,\mbox{т.к.}
 f|_{\,\omega_1}=0}+\underbrace{(f,\varphi_2)}_{=0,\,\mbox{т.к.}
 f|_{\,\omega_2}=0}=0. \,\blacksquare$
\end{enumerate}

$\square$ (Утверждение 1.)

Пусть $\omega_{max}=\cup%\limits
_{f|_{\,\omega}=0}\,\omega, \omega$ --- открытое подмножество
$\Omega.$ Пусть $\varphi\in D(\omega_{max}),$ тогда $\supp
\varphi$
--- компакт, принадлежащий $\omega_{max}.$ Из покрытия множества
$\varphi,$ образованного открытыми подмножествами $\omega \subset
\Omega\colon f|_{\,\omega}=0$ выделим конечное подпокрытие: $\supp
\,\varphi=\omega_1\cup\ldots\cup\omega_m,;$

$f|_{\,\omega_1}=0, \ldots, f|_{\,\omega_m}=0.$

По утверждению 2, $f|_{\,\omega_1\cup\omega_2}=0.$ Обозначим
$(\omega_1\cup\omega_2)\cup\ldots\cup\omega_m={\omega_1}'\cup\ldots\cup\omega_m,\,{\omega_1}'=\omega_1\cup\omega_2.
$

$f|_{\,{\omega_1}'\cup\omega_3}=0$ и т.д. Последовательно применяя
утверждение 2, будем в конце иметь
$$(f,\varphi)=f|_{\,\omega_1\cup\ldots\cup\omega_m}=0.\,\blacksquare$$


%lect8

\section{Свойства носителя}

\begin{enumerate}
\item $f,g\in D' (\Omega) \Rightarrow \supp (f+g)\subset \supp f\cup
\supp g.$

\textbf{Упражнение.} Доказать строго через определение.
\item $f\in D' (\Omega), \psi\in C^{\infty}(\Omega) \Rightarrow \supp f\psi\subset \supp f\cap
\supp \psi.$

\textbf{Упражнение.} Доказать строго через определение.
\end{enumerate}

\section{Обобщенные функции с компактным носителем}

Пусть $f\in D' (\Omega),\, \supp f\Subset\Omega,\, f\colon
D(\Omega)\to \mathbb C^n.$

\textbf{Утверждение 1.} $f$ можно продолжить на
$C^{\infty}(\mathbb R^n).$

$\square$

Пусть $\psi\in C^{\infty}(\mathbb R^n)$ и $(f,\psi)=(f,\eta\psi),$
где $\eta\in D(\Omega)\colon \eta\equiv1$ в окрестности $\supp f.$
Покажем, что $(f,\eta\,\psi)$ не зависит от выбора $\eta.$ В самом
деле, для двух функций $\eta_1,\eta_2\in D(\Omega)\colon
\eta_1\equiv1$ в окрестности $\supp f,$ $\eta_1,\eta_2\in
D(\Omega)\colon \eta_2\equiv1$ в окрестности $\supp f,$ получим:
$(f,\eta_1\psi)-(f,\eta_2\psi)=(f,(\eta_1-\eta_2)\,\psi)=0,$ так
как $\supp (\eta_1-\eta_2)\psi\Subset\Omega\setminus \supp f.
\,\blacksquare$

\textbf{Утверждение 2.} Пусть $g\in D' (\Omega), \psi\in
C^{\infty}(\Omega),$ причем $\psi \equiv1$ в окрестности $\supp
g.$ Тогда $g=g\psi.$

\textbf{Пример.}

$f(x)=\const  \ne0 \Rightarrow \supp f=\Omega.$

$f(x)\equiv0 \Rightarrow \supp f=\varnothing.$

$\square$

Пусть $\varphi\in D(\Omega),$ тогда
$(f\psi,\varphi)=(f,\psi\varphi)$ --- по определению умножения
обобщенной функции на бесконечно гладкую.

$(f\psi,\varphi)-(f,\varphi)=(f,(\psi-1)\varphi)=0,$ так как
$(\psi-1)\varphi\in D(\Omega\setminus \supp f). \,\blacksquare$

\textbf{Упражнение.} Покажите, что если в предыдущем утверждении
условие $\psi \equiv1$ в окрестности $\supp g$ заменить на условие
$\psi \equiv1$ на $\supp g,$ то утверждение НЕ верно.

\textbf{Теорема.} Пусть $f\in D' (\Omega),$ причем $\supp f$ ---
компакт, тогда существует
$$m,A\colon \forall \varphi\in D(\Omega)
\,\, |(f,\varphi)|\le A\|\varphi\|_{C^m(\Omega)},$$
где $m$\т порядок сингулярности.

$\square$ (от противного)

Т.е. $\forall m\, \forall A\, \exists \varphi\colon |(f,\varphi)|>
A\|\varphi\|_{C^m(\Omega)}.$

Возьмем $\eta\in D(\Omega),$ получим $$\forall \varphi\in
D(\Omega)\,\,\|\eta\,\varphi\|_{C^m(\Omega)}\le
B_m\|\varphi\|_{C^m(\Omega)},\eqno (1)$$ где $B_m=\const  >0,$
зависящая только от $m,\,\eta.$ В самом деле,
$$\|\eta\,\varphi\|_{C^m(\Omega)}=\sup_{x\in
\,\Omega}\,\sum_{|\alpha|\le
m}|\partial^{\,\alpha}(\eta(x)\,\varphi(x))|,$$ где
$\alpha=(\alpha_1,\ldots,\alpha_n)$ --- мультииндекс,
$|\alpha|=\alpha_1+\ldots+\alpha_n, \,\partial^{\,\alpha}=\cfrac
{\partial^{\,|\alpha|}}{\partial x_1^{\,\alpha_1}\ldots \partial
x_1^{\,\alpha_n}} \Rightarrow
\partial^{\,\alpha}(\eta\,\varphi)=\sum\limits_{\alpha'+\alpha''=\alpha}c_{\alpha'\alpha''}\partial^{\,\alpha'}\eta\,\partial^{\,\alpha''}\varphi.$

Например, $$\cfrac{\partial}{\partial
x_1}(\eta\,\varphi)=\cfrac{\partial \eta}{\partial
x_1}\,\varphi+\eta\,\cfrac{\partial \varphi}{\partial x_1},$$

$$\|\eta\,\varphi\|_{C^m(\Omega)}\le \tilde
B_m\|\eta\|_{C^m(\Omega)}\|\varphi\|_{C^m(\Omega)}.$$

Пусть $B_m=\tilde B_m\|\eta\|_{C^m(\Omega)},$ получим (1).
Согласно предположению, существует последовательность
$\varphi_m\in D(\Omega),\,m=1,2,\ldots:$

$$|(f,\varphi_m)|>A_m\|\varphi_m\|_{C^m(\Omega)}.\eqno (2)$$

Имеем: $\|\eta\,\varphi_m\|_{C^m(\Omega)}\le B_m
\|\varphi_m\|_{C^m(\Omega)},$ поэтому (2) влечет за собой
неравенство
$|(f,\varphi_m)|>\cfrac{A_m}{B_m}\,\|\eta\,\varphi_m\|_{C^m(\Omega)}.$
При этом, если $\eta \equiv1$ в окрестности $\supp f,$ то
$f\,\eta=f.$ Будем с самого начала считать ,что $\eta \equiv1$ в
окрестности $\supp f.$ Тем самым, получим:
$(f,\varphi_m)=(\eta\,f,\varphi_m)=(f,\eta\,\varphi_m),$ поэтому
$$|(f,\eta\,\varphi)|>\cfrac{A_m}{B_m}\,\|\eta\,\varphi_m\|_{C^m(\Omega)}=m \|\eta\,\varphi_m\|_{C^m(\Omega)}.\eqno (3)$$

Пусть
$\psi_m=\cfrac{\eta(x)\,\varphi_m(x)}{m\,\|\eta\,\varphi_m\|_{C^m(\Omega)}}.$
Разделив (3) на $m\,\|\eta\,\varphi_m\|_{C^m(\Omega)},$ получим
$$|(f,\varphi_m)|>1. \eqno(4)$$

Легко заметить,что $\psi_m\to0$ при $m\to \infty$ в пространстве
$D(\Omega).$ Действительно,

1) $\supp \psi_m\subset\eta\Subset\Omega,\,m=1,2,\ldots;$

2) для любого натурального $k\,\,
\|\psi_m\|_{C^k(\Omega)}\le\|\psi_m\|_{C^m(\Omega)},$ если $m\ge
k.$

Таким образом, $\lim\|\psi_m\|_{C^k(\Omega)}\le\lim\limits_{m\to
\infty}\|\psi_m\|_{C^m(\Omega)}.$

В то же время, $\|\psi_m\|_{C^m(\Omega)}=\cfrac
{\|\eta\,\varphi_m\|_{C^m(\Omega)}}{m
\|\eta\,\varphi_m\|_{C^m(\Omega)}}=\cfrac1m.$

Тем самым, $\lim\limits_{m\to \infty}\|\psi_m\|_{C^m(\Omega)}=0.$

Так как $f\colon D(\Omega)\to \mathbb C^n$ --- непрерывный
линейный функционал, то $\lim\limits_{m\to
\infty}|(f,\varphi_m)|=0,$ но, с другой стороны, по неравенству
(4), $\lim\limits_{m\to \infty}|(f,\varphi_m)|\ge1$. Получили
противоречие. $\,\blacksquare$

\textbf{Упражнение.} Пусть $f(x)=\delta'(x).$ чему будет равно
$m$?  ($f(x)=\delta(x) \rightarrow m=0$)

%lect9

\textbf{Теорема.} Пусть $f\in D' (\mathbb R^n), f'(x)\equiv0.$
Тогда $f(x)\equiv \const  .$

$\square$

Возьмем $\varphi \in D(\mathbb R^n).$ Пусть
$$P(\varphi)(x)=\varphi(x)-
\eta(x)\int\limits_{-\infty}^{\infty}\varphi(\xi) \,d\xi,$$ где
$\eta\in D(\mathbb R^n),$ причем
$\int\limits_{-\infty}^{\infty}\eta(x)\,dx=1.$ Тогда
$$\int\limits_{-\infty}^{\infty}P(\varphi)(x)\,dx=
\int\limits_{-\infty}^{\infty}\varphi(x)\,dx-
\underbrace{\int\limits_{-\infty}^{\infty}\eta(x)\,dx}_{=1}\int\limits_{-\infty}^{\infty}\varphi(x)\,dx=0.$$

Имеем $(f'(x),\underbrace{\int\limits_{-\infty}^x P(\varphi)(\xi)
\,d\xi}_{\in D(\Omega)})=0$.

Таким образом, $$0=\Bigl(f(x),\cfrac d{dx}\,\int\limits_{-\infty}^x
P(\varphi)(\xi)\,d\xi\Bigr)
=\Bigl(f(x),P(\varphi)(x)\Bigr)=\Bigl(f(x),\varphi(x)\Bigr)-\Bigl(f(x),\eta(x)\Bigr)\int\limits_{-\infty}^{\infty}\varphi(\xi)
\,d\xi,$$
$(f(x),\varphi(x))=(c,\varphi(x)),$ где $c=(f(x),\eta(x))$ ---
комплексное число, не зависящее от $\varphi(x).$ Другими словами,
$f(x)\equiv \const  . \,\blacksquare$

\textbf{Упражнение.}Пусть $\Omega$ --- область в $\mathbb
R^n,\,n\ge1,\,f\in D'(\Omega),$ причем $\cfrac{\partial
f}{\partial x_i}=0,\, i=\overline{1,n}.$ Тогда $f(x)\equiv \const
$ в $\Omega$.

\textbf{Теорема.} Пусть $f\in D'(\mathbb R).$ Тогда существует
$F\in D'(\mathbb R)\colon F'(x)=f(x).$

$\square\;$
Возьмем $\varphi(x)\in D(\mathbb R),$ как и в предыдущей теореме и
пусть
$$P(\varphi)(x)=\varphi(x)-\eta(x)\int\limits_{-\infty}^{\infty}\varphi(\xi)
\,d\xi,$$
где $\eta\in D(\mathbb R),$ причем
$\int\limits_{-\infty}^{\infty}\eta(x)\,dx=1.$ Тогда
$\int\limits_{-\infty}^{\infty}P(\varphi)(\xi)\,d\xi \in D(\mathbb
R).$ Тогда, если существует $F\in D'(\mathbb R)\colon F'(x)=f(x),$
то получим
\begin{multline}
\Bigl(f(x),\int\limits_{-\infty}^x
P(\varphi)(\xi)\,d\xi\Bigr) =\Bigl(F'(x),\int\limits_{-\infty}^x
P(\varphi)(\xi)\,d\xi\Bigr) =\\=
-\Bigl(f(x),\cfrac
d{dx}\,\int\limits_{-\infty}^x P(\varphi)
(\xi)\,d\xi\Bigr)=-\Bigl(F(x),P(\varphi)(x)\Bigr)=\\=
-\Bigl(F(x),\varphi(x)\Bigr)+\Bigl(f(x),\eta(x)\Bigr)\int\limits_{-\infty}^{\infty}\varphi(\xi)\,d\xi.
\end{multline}

Тем самым,
$$\Bigl(F(x),\varphi(x)\Bigr)=-\Bigl(f(x),\int\limits_{-\infty}^x
P(\varphi)(\xi)\,d\xi\Bigr)+\Bigl(c,\varphi(x)\Bigr)$$
 --- первообразная определена с точностью до константы:
$c=(f(x),\eta(x))$
--- комплексное число.

нам надо указать хотя бы одну функцию $F\in D'(\mathbb R)\colon
F'(x)=f(x).$

Пусть $(F(x),\varphi(x))=-\Bigl(f(x),\int\limits_{-\infty}^x
P(\varphi)(\xi)\,d\xi\Bigr).$ Проверим, что $F'(x)=f(x).$ В самом
деле, для любого $\varphi(x)\in D(\mathbb R)$
$$\Bigl(F'(x),\varphi(x)\Bigr)=-\Bigl(F(x),\varphi'(x)\Bigr)=\Bigl(f(x),\int\limits_{-\infty}^x
P(\varphi')(\xi)\,d\xi\Bigr).$$

$$P(\varphi')(\xi)=\varphi'(\xi)-\eta(\xi)\underbrace{\int\limits_{-\infty}^{\infty}\varphi'(t)\,dt}_{=0,\mbox{ т.к.} \varphi\in D(\mathbb R)}=\varphi(\xi),$$

$$(F'(x),\varphi(x))=\Bigl(f(x),\int\limits_{-\infty}^x
\varphi'(\xi)\,d\xi\Bigr)=(f(x),\varphi(x)).\,\blacksquare$$

\section{Прямое произведение обобщенных функций}

Пусть $f_1\colon \Omega_1\to \mathbb C,\,f_2\colon \Omega_2\to
\mathbb C, f_1\in C^{\infty}(\Omega_1),\,f_2\in
C^{\infty}(\Omega_2),\,\Omega_1\subset\mathbb
R^{n_1},\,\Omega_2\subset\mathbb R^{n_2}$ --- открытые
подмножества, $n_1,n_2\ge1.$

$f_1(x_1)f_2(x_2)\colon \Omega_1\times\Omega_2\to \mathbb C$ ---
\textbf{прямое произведение}, если
$x_1=(x_1^1,\ldots,x_1^{n_1})\in\Omega_1,\,x_2=(x_2^1,\ldots,x_2^{n_2})\in\Omega_2,
(x_1,x_2)\in \Omega_1\times\Omega_2.$

$(x_1,x_2)\mapsto f_1(x_1)f_2(x_2)$ --- функция $n_1+n_2$
вещественного аргумента.

Обозначения: $f_1(x_1)f_2(x_2),\,f_1(x_1)\cdot
f_2(x_2),\,f_1\otimes f_2(x_1,x_2).$

Будем понимать бесконечно гладкие функции
$f_1(x_1),\,f_2(x_2),\,f_1(x_1)f_2(x_2)$ как обобщенные.

\begin{multline}
f_1(x_1)f_2(x_2)\colon \varphi(x_1,x_2)\mapsto
\int\limits_{\Omega_1\times\Omega_2}f_1(x_1)f_2(x_2)\varphi(x_1,x_2)\,dx_1\,dx_2=\\=
\int\limits_{\Omega_1}f_1(x_1)\,dx_1
\underbrace{\int\limits_{\Omega_2}f_2(x_2)\varphi(x_1,x_2)\,dx_2}_{(f_2(x_2),\varphi(x_1,x_2))}
\end{multline}

\textbf{Лемма.} Пусть $f_2\in D'(\Omega_2).$ Тогда функция
$x_1\mapsto (f_2(x_2),\varphi(x_1,x_2))$ принадлежит пространству
$D(\Omega_1).$ Или $(f_2(x_2),\varphi(\cdot,x_2))\in D(\Omega_1).$

$\square$

Имеем $\varphi(x_1,x_2) \in D(\Omega_1\times\Omega_2),$ т.е.
$\varphi\in  C^{\infty}(\Omega_1\times\Omega_2),\,
\Omega_1\subset\mathbb R^{n_1},\,\Omega_2\subset\mathbb
R^{n_2},\,n_1,n_2\ge1.$

$K=\supp \varphi\Subset\Omega_1\times\Omega_2$ --- компакт. Пусть
$K_1$ --- проекция $K$ на $\Omega_1$, $K_2$ --- на $\Omega_2$.
Если $x_1\notin K_1,$ то $\forall
x_2\in\Omega_2\,\,\varphi(x_1,x_2)=0.$ Т.о.,
$(f_2(x_2),\varphi(x_1,x_2))=0,$ т.е. $\supp
(f_2(x_2),\varphi(\cdot,x_2))\Subset K_1.$ Осталось доказать, что
$(f_2(x_2),\varphi(\cdot,x_2))\in C^{\infty}(\Omega).$

А именно, докажем, что при $h_s\to0,\,s\to\infty$
$$\lim_{h_s\to0} \left(\frac1{h_s}\Bigl(f_2(x_2),\varphi(x_1^1,\ldots,x_1^i+h_s,\ldots,x_1^n,x_2)\Bigr)-
\Bigl(f_2(x_2),\varphi(x_1^1,\ldots,x_1^i,\ldots,x_1^n,x_2)\Bigr)\right)=$$
$$=\lim_{h_s\to0}
\left(f_2(x_2),\underbrace{\frac{\varphi(x_1^1,\ldots,x_1^i+h_s,\ldots,x_1^n,x_2)-\varphi(x_1^1,\ldots,x_1^i,\ldots,x_1^n,x_2)}{h_s}}_{\varphi_s(x_2)}\right)=$$
$$=\Bigl(f_2(x_2),\frac{\partial\varphi}{\partial
x_1}(x_1,x_2)\Bigr).$$

$\varphi_s(x_2)\to \cfrac{\partial \varphi(x_1,x_2)}{\partial
x_1^i}$ при $h_s\to0$ в пространстве $D(\Omega_2)$ при
фиксированном $x_1\in\Omega_1.$

Имеем:

1) существует компактное множество $H\Subset\Omega_2\colon
\varphi_s\subset H$ для всех достаточно больших индексов $s$. А
именно, в качестве $H$ достаточно взять $K_2$ --- проекцию $\supp
\varphi$ на $\Omega_2$;

2) $\Bigl\|\varphi_s(x_2)-\cfrac{\partial
\varphi(x_1,x_2)}{\partial x_1^i}\Bigr\|_{C^m(\Omega)}\to0$ при
$s\to\infty,\,\forall m.$

В самом деле:
\begin{multline}
\partial^{\,\alpha}_{x_2}\varphi(x_1^1,\ldots,x_1^i+h_s,\ldots,x_1^n,x_2)-
\partial^{\,\alpha}_{x_2}\varphi(x_1^1,\ldots,x_1^i,\ldots,x_1^n,x_2)=\\=
\int\limits_0^1\frac
d{dt}\,\underbrace{\partial^{\,\alpha}_{x_2}\varphi(x_1^1,\ldots,x_1^i+t
h_s,\ldots,x_1^n,x_2)}_{\equiv \psi(t)}\,dt,
\end{multline}
где
$x_2=(x_2^1,\ldots,x_2^{n_2}),\,\alpha=(\alpha_1,\ldots,\alpha_n)$
--- мультииндекс,
$$|\alpha|=\alpha_1+\ldots+\alpha_n,
\,\partial_{x_2}^{\,\alpha}=\cfrac
{\partial^{\,|\alpha|}}{\partial_{x_2} x_1^{\,\alpha_1}\ldots
\partial_{x_2} x_1^{\,\alpha_n}}.$$

Тогда $$\int\limits_0^1\frac
d{dt}\psi(t)\,dt=-\int\limits_0^1(1-t)'\frac
d{dt}\psi(t)\,dt=-\left.(1-t)\frac d{dt}\psi(t)\,dt\right|_0^1+
\int\limits_0^1(1-t)\frac {d^2}{dt^2}\psi(t)\,dt=$$
$$=\left.\frac d{dt}\psi(t)\right|_{t=0}+
\int\limits_0^1(1-t)\frac {d^2}{dt^2}\psi(t)\,dt,$$
$$\frac d{dt}\psi(t)=\partial_{x_1^i} \partial_{x_2}^{\,\alpha} \varphi (x_1^1,\ldots,x_1^i+t\,h_s,\ldots,x_1^n,x_2)\,h_s,$$
$$\frac {d^2}{dt^2}\psi(t)=\partial^2_{x_1^i} \partial_{x_2}^{\,\alpha} \varphi (x_1^1,\ldots,x_1^i+t\,h_s,\ldots,x_1^n,x_2)\,h_s^2,$$
$$\int\limits_0^1\frac d{dt}\psi(t)\,dt=\partial_{x_1^i} \partial_{x_2}^{\,\alpha} \varphi (x_1,x_2)\,h_s+
h_s^2\,\int\limits_0^1(1-t)\,\underbrace{\partial_{x_1^i}^2
\partial_{x_2}^{\,\alpha} \varphi (x_1^1,\ldots,x_1^i+t\,
h_s,\ldots,x_1^n,x_2)}_{\mbox{имеет компактный носитель}} \,dt,$$
$$\partial_{x_2}^{\,\alpha}\left(\varphi_s (x_2)-\frac{\partial\varphi(x_1,x_2)}{\partial x_1^i}\right)=
h_s\int\limits_0^1(1-t)\,\partial_{x_1^i}^2\partial_{x_2}^{\,\alpha}\varphi(x_1^1,\ldots,x_1^i+t\,h_s,\ldots,x_1^n,x_2)\,dt\to0,
\,s\to\infty$$ Сходимость по $x_2\in\Omega$ равномерная, так как
интеграл ограничен, a $h_s\to0$ при $s\to\infty.\,\blacksquare$

%lect10

\textbf{Определение.} Пусть $f_1\in D'(\Omega_1),\,f_2\in
D'(\Omega_2).$ \textbf{Прямым произведением $f_1$ и $f_2$}
называется функция $f_1(x_1)f_2(x_2)\in
D'(\Omega_1\times\Omega_2)\colon \forall\varphi \in
D(\Omega_1\times\Omega_2)$
$$\Bigl(f_1(x_1)f_2(x_2),\varphi(x_1,x_2)\Bigr)=\Bigl(f_1(x_1),(f_2(x_2),\varphi(x_1,x_2))\Bigr).$$

\textbf{Лемма.} Пусть $f_1\in D'(\Omega_1),\,f_2\in
D'(\Omega_2),\,\varphi \in D(\Omega_1\times\Omega_2).$ Тогда
$$\Bigl(f_1(x_1),(f_2(x_2),\varphi(x_1,x_2))\Bigr)=\Bigl(f_2(x_1),(f_1(x_2),\varphi(x_1,x_2))\Bigr).$$

$\square$

1) Если $\varphi(x_1,x_2)=\varphi_1(x_1)\,\varphi_2(x_2),$ где
$\varphi_1\in D(\Omega_1)\,\varphi_2\in D(\Omega_2),$ то
$$\Bigl(f_1(x_1),(f_2(x_2),\varphi(x_1,x_2))\Bigr)=
\Bigl(f_1(x_1),(f_2(x_2),\varphi_1(x_1)\,\varphi_2(x_2))\Bigr)=$$
$$=\left((f_1(x_1),\varphi_1(x_1)\Bigl(f_2(x_2),\varphi_2(x_2)\Bigr)\right)=
\Bigl(f_2(x_2),\varphi_2(x_2)\Bigr)\Bigl(f_1(x_1),\varphi_1(x_1)\Bigr).$$
Аналогично,
$$\Bigl(f_1(x_1),(f_2(x_2),\varphi(x_1,x_2))\Bigr)=
\Bigl(f_1(x_1),\varphi_1(x_1)\Bigr)\Bigl(f_2(x_2),\varphi_2(x_2)\Bigr).$$

2) В общем случае представим $\varphi(x_1,x_2)\in
D(\Omega_1\times\Omega_2)$ в виде ряда, сходящегося в
$D(\Omega_1\times\Omega_2)$:
$$\varphi(x_1,x_2)=\sum_{s=1}^{\infty}\varphi_s(x_1)\,\psi_s(x_2),$$
где $\varphi_s \in D(\Omega_1),\,\psi_s \in D(\Omega_2).$ Тогда
$$\Bigl(f_1(x_1),(f_2(x_2),\varphi(x_1,x_2))\Bigr)=
\sum_{s=1}^{\infty}\Bigl(f_1(x_1),(f_2(x_2),\varphi_s(x_1)\,\psi_s(x_2))\Bigr)=$$
$$=\sum_{s=1}^{\infty}\Bigl(f_1(x_1),\varphi_s(x_1)\Bigr)\Bigl(f_2(x_2),\psi_s(x_2)\Bigr)=
\Bigl(f_2(x_2),(f_1(x_1),\varphi(x_1,x_2))\Bigr).\,\blacksquare$$

%lect11

\textbf{Теорема.} Пусть $f_1\in D'(\Omega),\,\Omega\subset\mathbb
R^n$ --- открытое множество, $\supp f\Subset\Omega$ --- компакт.
Тогда существует $A>0,\,m\in \mathbb N\colon \forall \varphi\in
D(\Omega)\,\,\,|(f,\varphi)|\le A\|\varphi\|_{C^m(\Omega)}$.

\textbf{Следствие.} Пусть $f_1\in
D'(\Omega),\,\Omega\subset\mathbb R^n$ --- открытое множество,
$\supp f\Subset\Omega$ --- компакт. Тогда существует $A>0,\,m\in
\mathbb Z\colon \forall \varphi\in D(\Omega)\colon \supp
\varphi\subset K \,\,\,|(f,\varphi)|= A\|\varphi\|_{C^m(\Omega)}$.

$\square$

Возьмем $\eta\in D(\Omega)\colon \eta\equiv1$ в окрестности $K$.
Тогда $\forall \varphi\in D(\Omega)\colon \supp \varphi\subset K$
получим
$(f,\varphi)=(f,\underbrace{\eta\,\varphi}_{\equiv\varphi})=(\eta\,f,\varphi).$
При этом $\eta\,f$ имеет компактный носитель: $\supp
\eta\,f\subset \supp \eta\Subset\Omega.$

Т.о.
$|(f,\varphi)|=|(\eta\,f,\varphi)|=A\|\varphi\|_{C^m(\Omega)}$
(последнее равенство --- из предыдущей теоремы). $\blacksquare$

\textbf{Лемма.} Пусть $f_1\colon \Omega_1\to \mathbb C,\,f_2\colon
\Omega_2\to \mathbb C,\,\Omega_1\subset\mathbb
R^{n_1},\,\Omega_2\subset\mathbb R^{n_2}$ --- открытые
подмножества, $n_1,n_2\ge1.$ Тогда $\varphi(x_1,x_2)\mapsto
\Bigl(f_1(x_1),(f_2(x_2),\varphi(x_1,x_2))\Bigr)$ --- непрерывный
функционал на $D(\Omega_1\times\Omega_2).$

$\square$

Пусть $\varphi_i(x_1,x_2)\in
D(\Omega_1\times\Omega_2),\,i=1,2,\ldots;\, \varphi_i\to0,
i\to{\infty}.$ Докажем, что
$$\Bigl(f_1(x_1),(f_2(x_2),\varphi(x_1,x_2))\Bigr)\to0,\,i\to{\infty}.$$

В самом деле, существует компакт
$H\Subset\Omega_1\times\Omega_2\colon \supp \varphi_i\subset H.$
Тогда $\forall _1\in\Omega\,\forall i=1,2,\ldots.$

$\supp \varphi_i(x_1,\cdot)\subset H_2,$ где $H_2$ --- проекция
$H$ на $\Omega_2.$ Тогда
$$\forall s\in \mathbb Z\,\,\,\,
\|(f_2(x_2),\varphi_i(x_1,\cdot))\|_{C^s(\Omega_1)}\to0,\,i\to{\infty}.$$
Действительно, по следствию, $\exists A>0,\,m\in \mathbb Z\colon
\forall \alpha=(\alpha_1,\ldots,\alpha_n) \,\,\forall
x_1\in\Omega_1$
$$|\,\partial^{\,\alpha}_{x_1}\,(f_2(x_2),\varphi_i(x_1,x_2))|=
|\,(f_2(x_2),\partial^{\,\alpha}_{x_1}\varphi_i(x_1,x_2))|\le
A\,\|\partial^{\,\alpha}_{x_1}\varphi_i(x_1,\cdot))\|_{C^m(\Omega_2)}\le$$
$$A\,\|\varphi_i\|_{C^{m+|\alpha|}(\Omega_1\times\Omega_2)}\to0,\,i\to{\infty}.$$

Другими словами, мы доказали, что $\forall s\in \mathbb Z,\,s>0$
$$\|(f_2(x_2),\varphi_i(\cdot,x_2))\|_{C^s(\Omega_1)}\to0,\,i\to{\infty}.$$

Очевидно, что $\supp (f_2(x_2),\varphi_i(\cdot,x_2))\subset H_1,$
где $H_1\Subset\Omega_1$ --- проекция компакта $H$ на $\Omega_1.$

Т.о., $(f_2(x_2),\varphi_i(\cdot,x_2))\to0,\,i\to{\infty}$ в
пространстве $D(\Omega_1)$, поэтому
$$\Bigl(f_1(x_1),(f_2(x_2),\varphi_i(x_1,x_2))\Bigr)\to0,\,i\to{\infty}.$$

Аналогично может быть доказано ,что
$\Bigl(f_2(x_2),(f_1(x_1),\varphi_i(x_1,x_2))\Bigr)$ является
линейным не\-прерывным функционалом от $\varphi_i(x_1,x_2)\in
D(\Omega_1\times\Omega_2).\,\blacksquare$

\textbf{Лемма.} Пусть $\varphi_i(x_1,x_2)\in
D(\Omega_1\times\Omega_2),\,\Omega_1\subset\mathbb
R^{n_1},\,\Omega_2\subset\mathbb R^{n_2}$ --- открытые множества.
Тогда существует $\varphi_s\in D(\Omega_1), \psi_s\in
D(\Omega_2)\colon$
$$\varphi_i(x_1,x_2)=\sum_{s=1}^{\infty}\varphi_s(x_1)\,\psi_s(x_2),$$
где ряд сходится в $D(\Omega_1\times\Omega_2).$

$\square$

Без ограничения общности, будем предполагать,что $\Omega_1\in
\underbrace{(-\pi,\pi)\times\ldots\times(-\pi,\pi)}_{n_1},\,$

$\Omega_2\in\underbrace{(-\pi,\pi)\times\ldots\times(-\pi,\pi)}_{n_2},\,
\supp \varphi\Subset\Omega_1\times\Omega_2=
\underbrace{(-\pi,\pi)^{n_1+n_2}\times\ldots\times(-\pi,\pi)^{n_1+n_2}}_{n_1+n_2}.$

$\varphi_i(x_1,x_2)=\sum\limits_k C_k\,e^{ikx}=
\sum\limits_{k_1,k_2}C_{k_1,k_2}\,e^{ik_1x+ik_2x},\,x=(x_1,x_2),\,k=(k_1,k_2),$

$$c_k=\cfrac1{(2\pi)^{n_1+n_2}}\,\int\limits_{(-\pi,\pi)^{n_1+n_2}}\varphi(x)\,e^{ikx}\,dx,$$

$$\|\varphi\|_{L_2(I_{n_1+n_2})}^2=\sum_k |C_k|^2$$ --- равенство
Парсеваля, где $I_{n_1+n_2}=(-\pi,\pi)^{n_1+n_2}=
\underbrace{(-\pi,\pi)^{n_1+n_2}\times\ldots\times(-\pi,\pi)^{n_1+n_2}}_{n_1+n_2}.$

$$\underbrace{\partial^2\varphi(x)}_{\in \,L_2(I_{n_1+n_2})}=
\sum\limits_k a_k \,e^{ikx},\,
\|\partial^1\varphi(x)\|_{L_2(I_{n_1+n_2})}=(2\pi)^{n_1+n_2}
\sum\limits_k |a_k|^2 \,dx<\infty,$$

$$a_k=\frac1{(2\pi)^{n_1+n_2}}\,\int\limits_{I_{n_1+n_2}}\partial^2\varphi(x)\,e^{-ikx}\,dx=
\frac{(-1)^{|\alpha|}}{(2\pi)^{n_1+n_2}}\int\limits_{I_{n_1+n_2}}\varphi(x)\,\partial^2\,e^{-ikx}\,dx=$$
$$=\frac{(ik)^{\alpha}}{(2\pi)^{n_1+n_2}}\int\limits_{I_{n_1+n_2}}\varphi(x)\,e^{-ikx}\,dx=
(ik)^{\alpha}\,C_k,$$ где
$\alpha=(\alpha^1,\ldots,\alpha^{n_1+n_2}),\,x=(x^1,\ldots,x^{n_1+n_2}),\,k=(k^1,\ldots,k^{n_1+n_2}).$

$(i\,k)^{\alpha}=(i\,k^1)^{\alpha^1}\cdot\ldots\cdot(i\,k^{n_1+n_2})^{\alpha^{n_1+n_2}}
\Rightarrow |\,a_k|=|\,k^{\alpha}\,C_k|\le \const  ,$ не зависящей
от $k$, так как $\sum_k |\,a_k|^2<{\infty}.$

$$\varphi(x_1,x_2)=\sum\limits_{k_1,k_2}C_{k_1,k_2}\,e^{ik_1x_1}\,e^{ik_2x_2}.$$

Возьмем $\eta_1\in D(\Omega_1),\,\eta_2\in D(\Omega_2)\colon
\eta_1\equiv1$ в окрестности проекции $\supp \varphi$ на
$\Omega_1$; $\eta_2\equiv1$ в окрестности проекции $\supp \varphi$
на $\Omega_1$.

$$\varphi(x_1,x_2)=\eta_1(x_1)\,\eta_2(x_2)\,\varphi(x_1,x_2)=
\sum\limits_{k_1,k_2}C_{k_1,k_2}\eta_1(x_1)\,e^{ik_1x_1}\eta_2(x_2)\,e^{ik_2x_2}$$
 --- сходится абсолютно и равномерно со всеми своими производными в
$D(\Omega_1\times\Omega_2).\,\blacksquare$

%lect12

\textbf{Определение.} Последовательность обобщенных функций
$f_k\in D'(\Omega),\,k=1,2,\ldots,\,\Omega\subset\mathbb R^n$ ---
непустое открытое множество, \textbf{сходится (слабо) к обобщенной
функции $f\in D'(\Omega)$}, если $\forall\varphi\in D(\Omega)\,\,
\lim\limits_{k\to\infty}(f_k,\varphi)=(f,\varphi).$

\textbf{Теорема.} (без доказательства, Шилов, <<II спецкурс>>) Пусть
последовательность обобщенных функций $f_k\in
D'(\Omega),\,k=1,2,\ldots,$ слабо сходится к некоторому
функционалу
$$f\colon D(\Omega)\to\mathbb C^n,$$
т.е.
$$\forall\varphi\in D(\Omega)\,\,
\lim\limits_{k\to\infty}\underbrace{(f_k,\varphi)}_{f_k(\varphi)}=
\underbrace{(f,\varphi)}_{f(\varphi)}.$$
Тогда $f$ является
обобщенной функцией.

\section{Свертка обобщенных функций}

Пусть $f,g\in L_1(\mathbb R^n),\,n\ge1.$

$$f*g=\int\limits_{\mathbb
R^n}\,f\underbrace{(x-y)}_{\xi}\,g\underbrace{(y)}_{x-\xi}\,dy=
\int\limits_{\mathbb R^n}\,f(\xi)g(x-\xi)\,d\xi.$$

$f*g\in L_1(\mathbb R^n),$ так как $\int\limits_{\mathbb R^n}
|f*g(x)|\,dx= \int\limits_{\mathbb R^n}dx\,
\Bigl|\int\limits_{\mathbb R^n}f(x-y)\,g(y)\,dy\Bigr|\le
\int\limits_{\mathbb R^n}dx\, \int\limits_{\mathbb
R^n}\Bigl|f(x-y)\,g(y)\Bigr|\,dy=$

$=\int\limits_{\mathbb
R^n}dy\,|\,g(y)|\,\underbrace{\int\limits_{\mathbb
R^n}|\,f(x-y)|\,dx}_{\int\limits_{\mathbb R^n}|f(x)|\,dx}=
\int\limits_{\mathbb R^n}|\,g(y)|\,dx\int\limits_{\mathbb
R^n}|\,f(x)|\,dx<\infty.$

Т.о., интеграл сходится по теореме Фубини,
$$\Rightarrow f*g\in
L_1(\mathbb R^n),\,\|f*g\|_{L_1(\mathbb R^n)}\le\|f\|_{L_1(\mathbb
R^n)}\,\|g\|_{L_1(\mathbb R^n)}.$$

Как определить свертку двух обобщенных функций $f,g\in D'(\mathbb
R^n)$?

Будем понимать функции $f,g\in L_1(\mathbb R^n)$ как обобщенные.
Тогда $\forall\varphi\in D(\mathbb R^n).$

$(f,\varphi)=\int\limits_{\mathbb R^n}f\,\varphi\,dx,\,
(g,\varphi)=\int\limits_{\mathbb R^n}g\,\varphi\,dx,$

$$(f*g,\varphi)=\int\limits_{\mathbb R^n}f*g(x)\,\varphi(x)\,dx=
\int\limits_{\mathbb R^n}dx\,\varphi(x)\, \int\limits_{\mathbb
R^n}f(x-y)\,g(y)\,dy= \int\limits_{\mathbb
R^{2n}}\varphi(x)\,f(x-y)\,g(y)\,dx\,dy=$$
 $$=\int\limits_{\mathbb
R^{2n}}\varphi(\xi+y)\,f(\xi)\,g(y)\,d\xi\,dy,$$ где $\xi=x-y.$
Последний интеграл очень напоминает выражение
$\Bigl(f(\xi)g(y),\varphi(x+y)\Bigr)$ --- прямое произведение. Но
последнее выражение нельзя корректно определить для произвольных
$f,g\in D'(\mathbb R^n),$ так как функция $\varphi(x+y)\colon
\mathbb R^{2n}\to\mathbb C^n$ не имеет компактного носителя. Как
же быть?

$\supp \varphi\subset[-A,A], \,-A<\underbrace{y+\xi}_{=x}<A,$ то
есть $\supp \varphi$ в общем случае не является компактом (только
если $\supp \varphi=\varnothing$).

\textbf{Определение.} Последовательность функций $\eta_k\in
L_1(\mathbb R^{2n}),\,k=1,2,\ldots$ --- \textbf{исчерпание
единицы}, если:

1) для любого компакта $H\Subset\mathbb R^{2n}\,\,
\left.\eta_k\right|_H=1,$ начиная с некоторого $k$;

2) для любого мультииндекса $\alpha=(\alpha_1,\ldots,\alpha_{2n})$
существует $A_{\alpha}=\const  >0\colon$
$$\forall z\in\mathbb
R^{2n}\,\, \forall k=1,2,\ldots\,\,
|\partial^{\,\alpha}\eta_k(z)|\le A_{\alpha}.$$

Другими словами, $$\forall\, m\in \mathbb Z \,\,\,\exists A\colon
\forall k=1,2,\ldots\,\,\|\eta_k\|_{C^m(\mathbb R^{2n})}\le A$$

Есть хотя бы одно компактное исчерпание единицы?

\textbf{Пример.} Обозначим $B_r={z\in\mathbb R^{2n}\colon |z|<r}.$
Возьмем $\eta\in D(B_2)\colon \left.\eta\right|_{B_1}=1.$

Пусть $\eta_k(z)=\eta\Bigl(\,\cfrac zk\,\Bigr),\,k=1,2,\ldots,\,
\eta_k(z)\in D(B_{2k}),\, \left.\eta_k\right|_{B_k}=1.$

$$|\partial^{\,\alpha}\eta_k(z)|=\Bigl|\partial^{\,\alpha}\eta\Bigl(\,\cfrac
zk\,\Bigr)\Bigr|\le
\underbrace{k^{-|\alpha|}}_{\le1}\,\|\eta\|_{C^{\alpha}(B_2)},$$
где $|\alpha|=\alpha_1+\ldots+\alpha_{2n}$. Т.е. $\eta_k$ ---
компактное исчерпание единицы.

\textbf{Определение.} Пусть $f,g\in D'(\mathbb R^n).$
\textbf{Свертка $f*g$}: если для любого компактного исчерпания
единицы $\eta_k\in D(\mathbb R^{2n},\,k=1,\ldots,n)$ для любого
$\varphi\in\,D(\mathbb R^n)$ существует предел
$$\lim (f(x)*g(x),\varphi(x)):=\lim\limits_{k
\to\infty}(f(x)\,g(y),\eta_k(x,y)\varphi(x+y)).$$

\textbf{Утверждение.} Если для любого компактного исчерпания
единицы $\eta_k\in D(\mathbb R^{2n}),\,\,\,k=1,2,\ldots,\,$
указанный выше предел существует, то этот предел не зависит от
выбора компактного исчерпания.

$\square$

Пусть $\eta_k\in D(\mathbb R^{2n})$ и $\lambda_k\in D(\mathbb
R^{2n}),\,k=1,2,\ldots,$ --- два компактных исчерпания единицы.
построим компактное исчерпание единицы $\Theta_k$:
$$\Theta_{2k-1}=\eta_k,\,\Theta_{2k}=\lambda_k,\,k=1,2,\ldots$$
Получим последовательность:
$\eta_1,\,\lambda_1,\,\eta_2,\,\lambda_2,\,\eta_3,\,\lambda_3\ldots.$

Очевидно, что $\Theta_k, k=1,2,\ldots$ --- тоже компактное
исчерпание единицы. По условию, существует $\lim\limits_{k
\to\infty}(f(x)\,g(y),\Theta_k(x,y)\varphi(x+y)),\,\varphi\in
D(\mathbb R^n).$ Тогда для любой последовательности
$(f(x)\,g(y),\Theta_k(x,y)\varphi(x+y))$ при $k \to\infty$ предел
тоже существует и совпадает с пределом исходной
последовательности: $\lim\limits_{k
\to\infty}(f(x)\,g(y),\eta_k(x,y)\varphi(x+y)).\,\blacksquare$

\textbf{Упражнение.} Пусть $f,g\in L_1(\mathbb R^n).$ Тогда
свертка $f*g$, определенная в классическом смысле (как интеграл)
совпадает со сверткой, определенной в обобщенном смысле (как
свертка функций из $D'(\mathbb R^n)$).

\section{Свойства свертки обобщенных функций}

\begin{enumerate}
\item Пусть $f_1,f_2,g\in D'(\mathbb R^n)$ и при этом существует
свертка $f_1*g,\,f_2*g.$ Тогда существует
$(f_1+f_2)*g=f_1*g+f_2*g.$

$\square$

Пусть $\eta_s\in D(\mathbb R^{2n}),\,s=1,\ldots,n$ --- компактное
исчерпание единицы.

Для любого $\varphi\in\,D(\mathbb R^n)$
$$\lim\limits_{s\to\infty}\Bigl((f_1(x)+f_2(x))g(y),\eta_s(x,y)\varphi(x+y)\Bigr)=$$
$$=\lim\limits_{s\to\infty}\Bigl(f_1(x)g(y),\eta_s(x,y)\varphi(x+y)\Bigr)+
\lim\limits_{s\to\infty}\Bigl(f_2(x)g(y),\eta_s(x,y)\varphi(x+y)\Bigr).\,\blacksquare$$

\item Пусть $f,g\in D'(\mathbb R^n),\,\lambda\in\mathbb C^n$ и при этом существует
свертка $f*g.$ Тогда существует

$(\lambda f)*g=\lambda \,(f*g).$
\item Пусть $f,g\in D'(\mathbb R^n)$ и при этом существует
свертка $f*g.$ Тогда существует свертка $g*f,$ причем $f*g=g*f.$
\end{enumerate}

\textbf{Упражнение.} Доказать пункты 2 и 3.

\section{Дифференцирование сверток}

\textbf{Теорема.} Пусть $f,g\in D'(\mathbb R^n)$ и при этом
существует свертка $f*g.$ Тогда существуют $\cfrac{\partial
f}{\partial x_i}*g$ и $f*\cfrac{\partial g}{\partial x_i},$ причем
$\cfrac{\partial}{\partial x_i}\,(f*g)=\cfrac{\partial f}{\partial
x_i}*g+ f*\cfrac{\partial g}{\partial x_i}.$

$\square$

Пусть $\eta_s\in D(\mathbb R^{2n},\,s=1,\ldots,n)$ --- компактное
исчерпание единицы.

Для любого $\varphi\in\,D(\mathbb R^n)$

$$\Bigl(\cfrac{\partial}{\partial x_i}\,(f*g),\varphi\Bigr)=
-\Bigl(f*g,\cfrac{\partial \varphi}{\partial x_i}\Bigr)=
-\lim\limits_{s\to\infty}\Bigl(f(x)g(y),\eta_s(x,y)\,\cfrac{\partial
\varphi(x+y)}{\partial x_i} \Bigr),$$

\begin{multline}\Bigl(\cfrac{\partial f}{\partial x_i}\,g,\varphi\Bigr):=
\lim\limits_{s\to\infty}\Bigl( \cfrac{\partial f}{\partial
x_i}(x)\,g(y),\eta_s(x,y)\,\varphi(x+y) \Bigr)=\\=
\lim\limits_{s\to\infty}\Bigl( \cfrac{\partial f}{\partial
x_i}(x),\Bigl(g(y),\eta_s(x,y)\,\varphi(x+y)\Bigr) \Bigr)=
-\lim\limits_{s\to\infty}\Bigl(f(x),\Bigl(g(y),\cfrac{\partial}{\partial
x_i}\,(\eta_s(x,y)\,\varphi(x+y))\Bigr) \Bigr)=\\=
-\lim\limits_{s\to\infty}\Bigl(f(x)\,g(y),\cfrac{\partial}{\partial
x_i}\,(\eta_s(x,y)\,\varphi(x+y))\Bigr)=\\=
-\lim\limits_{s\to\infty}\Bigl(f(x)\,g(y),\eta_s(x,y)\,\cfrac{\partial\varphi}{\partial
x_i}\,(x+y)+ \cfrac{\partial\eta_s}{\partial
x_i}\,(x,y)\,\varphi(x+y)\Bigr).
\end{multline}

Рассмотрим последовательность функций
$\xi_s(x,y)=\cfrac{\partial\eta_s}{\partial x_i}\,(x,y)+
\eta_s(x,y),\,s=0,1,\ldots.$ Очевидно, что $\xi_s(x,y)$ образует
компактное исчерпание единицы. Тогда
$$\lim\limits_{s\to\infty}\Bigl(f(x)\,g(y),\eta_s(x,y)\,\varphi(x+y)\Bigr)=
\lim\limits_{s\to\infty}\Bigl(f(x)\,g(y),\xi_s(x,y)\,\varphi(x+y)\Bigr)=
(f*g,\varphi),$$ так как предел в определении свертки не зависит
от выбора компактного исчерпания.
\begin{multline}
\lim\limits_{s\to\infty}\Bigl(f(x)\,g(y),\eta_s(x,y)\,\varphi(x+y)\Bigr)=
\lim\limits_{s\to\infty}\Bigl(f(x)\,g(y),\eta_s(x,y)\,\varphi(x+y)\Bigr)+\\+
\lim\limits_{s\to\infty}\Bigl(f(x)\,g(y),\cfrac{\partial\eta_s(x,y)}{\partial
x_i}\,\varphi(x+y)\Bigr).\end{multline}

Первый и второй пределы существуют, а значит существует и третий
предел и, следовательно,
$\lim\limits_{s\to\infty}\Bigl(f(x)\,g(y),\cfrac{\partial\eta_s(x,y)}{\partial
x_i}\,\varphi(x+y)\Bigr)=0.$

Т.о., доказано, что свертка $\cfrac{\partial f}{\partial x_i}*g$
существует, причем $\cfrac{\partial f}{\partial
x_i}*g=\cfrac{\partial}{\partial x_i}\,(f*g).$

Аналогично доказывается, что существует свертка $f*\cfrac{\partial
g}{\partial x_i},$ причем $f*\cfrac{\partial g}{\partial
x_i}=\cfrac{\partial}{\partial x_i}\,(f*g)$ (или его можно
получить из коммутативности свертки).$\,\blacksquare$

\textbf{Утверждение.} Пусть $f,g\in D'(\mathbb R^n),$ причем
$\supp g\Subset\mathbb R^n$ --- компакт. Тогда свертка $f*g$
существует, причем для любого $\varphi\in\,D(\mathbb R^n).$

$\square$

Пусть $\eta_s(x,y)\in D(\mathbb R^{2n}),\,s=1,\ldots,n$ ---
компактное исчерпание единицы.

Для любого $\varphi\in D(\mathbb R^n)$

$$\lim\limits_{s\to\infty}\Bigl(f(x)\,g(y),\eta_s(x,y)\,\varphi(x+y)\Bigr)=
\lim\limits_{s\to\infty}\Bigl(f(x),\Bigl(g(y),\eta_s(x,y)\,\varphi(x+y)\Bigr)\Bigr)=$$

$$=\lim\limits_{s\to\infty}\Bigl(f(x),\Bigl(\lambda(y)\,g(y),\eta_s(x,y)\,\varphi(x+y)\Bigr)\Bigr)
=\lim\limits_{s\to\infty}\Bigl(f(x),\Bigl(g(y),\eta_s(x,y)\,\lambda(y)\,\varphi(x+y)\Bigr)\Bigr),$$
где $\lambda(y)\in\mathbb R^n,$ --- некоторая функция:
$\lambda\equiv1$ в окрестности компакта $\supp g$.

Очевидно, что $\lambda(y)\,\varphi(x+y)\in D(\mathbb R^n),$ в
частности, имеет компактный носитель. Т.о.,
$$\eta_s(x,y)\,\lambda(y)\,\varphi(x+y)=\lambda(y)\,\varphi(x+y),$$
начиная с некоторого номера $s$. Поэтому,
$$\lim\limits_{s\to\infty}\Bigl(f(x)\,g(y),\eta_s(x,y)\,\varphi(x+y)\Bigr)
=\Bigl(f(x),(g(y),\lambda(y)\,\varphi(x+y))\Bigr)=$$

$$=\Bigl(f(x),(\underbrace{\lambda(y)\,g(y)}_{=g(y)},\varphi(x+y))\Bigr)=
\Bigl(f(x),(g(y),\varphi(x+y))\Bigr).\,\blacksquare$$

\textbf{Упражнение.} Пусть $f,g\in D'(\mathbb R^n),$ причем $\supp
g\Subset\mathbb R^n$ --- компакт. Тогда для любого $\varphi\in
D(\mathbb R^n)\,\, (f*g,\varphi)=(g(y),(f(x),\varphi(x+y))).$


\section{Фундаментальное решение оператора $\mathcal
L=\sum\limits_{|\alpha|\le m}a_{\alpha}{\partial^{\, \alpha}}.$}

Рассмотрим оператор $\mathcal L=\sum\limits_{|\alpha|\le
m}a_{\alpha}{\partial^{\, \alpha}},\, a_{\alpha}\in \mathbb C,\,
|\,\alpha|=\alpha_1+\ldots+\alpha_n, \,\partial^{\,\alpha}=\cfrac
{\partial^{\,|\alpha|}}{\partial x_1^{\,\alpha_1}\ldots
\partial x_n^{\,\alpha_n}},$

$\mathcal L(f*g)=\mathcal L f*g=f*\mathcal L g.$

\textbf{Определение.} Функция $E(x)\in D'(\mathbb R^n)$
--- \textbf{фундаментальное решение оператора}
$\mathcal L=\sum\limits_{|\alpha|\le m}a_{\alpha}{\partial^{\,
\alpha}},\, a_{\alpha}\in \mathbb C,$ если $\mathcal
L\,E(x)=\delta(x).$

\textbf{Пример.} $\mathcal L=\cfrac d{dx},\,
E(x)=\Theta(x)=\begin{cases} 1, i=j;\\ 0, i \ne j.\end{cases}$
Действительно, $\mathcal L\,E(x)=\Theta'(x)=\delta(x).$

\textbf{Теорема.} (существования) Пусть $f\in D'(\mathbb R^n),\,
E(x)$ --- фундаментальное решение оператора $\mathcal
L=\sum\limits_{|\alpha|\le m}a_{\alpha}{\partial^{\, \alpha}},\,
a_{\alpha}\in \mathbb C,$ причем существует свертка $u=E*f.$ Тогда
$\mathcal L\,u=f$ и $u$ --- решение.

$\square$

По утверждению 1, $\mathcal L\,u=\mathcal
L\,E*f=\delta*f=f.\,\blacksquare$

\textbf{Утверждение 1.} Пусть $f\in D'(\mathbb R^n).$ Тогда
$\delta*f=f.$

$\square$

Свертка $\delta*f$ существует, так как $\supp \delta(x)=\{0\}$ ---
компактное множество. При этом, для любого $\varphi\in\,D(\mathbb
R^n)$
$(f*\delta,\varphi)=(f(x),(\delta(y),\varphi(x+y)))=(f(x),\varphi(x)).\,\blacksquare$

\textbf{Теорема.} (единственности) Пусть $u,f\in D'(\mathbb
R^n),\, E(x)$ --- фундаментальное решение оператора $\mathcal
L=\sum\limits_{|\alpha|\le m}a_{\alpha}{\partial^{\, \alpha}},\,
a_{\alpha}\in \mathbb C,$ причем $\mathcal L\,u=f$ и при этом
существует свертка $E*f=u.$

$\square$

Имеем $\mathcal L\,(E*u)=\mathcal L\,E*u=\delta*u=u.$ C другой
стороны, $\mathcal L\,(E*u)=E*\mathcal L\,u=E*f$. Тогда
$u=E*f.\,\blacksquare$

\section{Фундаментальное решение обыкновенного дифференциального оператора\\
с постоянными коэффициентами}

$$\mathcal L\,u=u^{(m)}+a_{m-1}u^{(m-1)}+\ldots+a_1u'+a_0u,\mbox{ где
} a_{m-1},\ldots,a_1,a_0\in \mathbb C.$$

\textbf{Теорема.} Пусть $E(x)=W(x)\,\Theta(x),$ где
$\Theta(x)=\begin{cases} 1, i=j;\\ 0, i \ne j,\end{cases}$ а
$W(x)$ --- решение задачи Коши $\mathcal L\,W=0,\,
W(0)=\ldots=W^{(m-2)}(0)=0,\,W^{(m-1)}(0)=1.$ Тогда $\mathcal
L\,E(x)=\delta(x).$

$\square$

 Пусть $\varphi\in\,D(\mathbb R).$ Тогда $(\mathcal
L\,E(x),\varphi(x))=(E(x),\mathcal L^*\,\varphi),$ где $$\mathcal
L^*\,\varphi=(-1)^m\varphi^{(m)}(x)+(-1)^{m-1}a_{m-1}\varphi^{(m-1)}(x)+\ldots+
a_1\varphi'(x)+a_0\varphi(x).$$

Имеем:
$$(E(x),\mathcal L^*\,\varphi)=
\int\limits_{-\infty}^{\infty}E(x)\,\mathcal L^*\varphi(x)\,dx=
\int\limits_{-\infty}^{\infty}\Theta(x)\,W(x)\,\mathcal
L^*\varphi(x)\,dx= \int\limits_0^{\infty}W(x)\,\mathcal
L^*\varphi(x)\,dx.\eqno (1)$$

$$\int\limits_0^{\infty}(-1)^k\varphi^{(k)}(x)W(x)\,dx=
\left.(-1)^k\varphi^{(k-1)}(x)W(x)\right|_0^{\infty}+
\int\limits_0^{\infty}(-1)^{k-1}\varphi^{(k-1)}(x)W'(x)\,dx=$$

$$=(-1)^{k-1}\varphi^{(k-1)}(0)W(0)+
\left.(-1)^{k-1}\varphi^{(k-2)}(x)W'(x)\right|_0^{\infty}+
\int\limits_0^{\infty}(-1)^{k-2}\varphi^{(k-2)}(x)W''(x)\,dx=\ldots=$$

$$=(-1)^{k-1}\varphi^{(k-1)}(0)W(0)+(-1)^{k-2}\varphi^{(k-2)}(0)W'(0)+\ldots+\varphi(0)W^{(k-1)}(0)+
\int\limits_0^{\infty}\varphi(x)W^{(k)}(x)\,dx$$

Продолжим равенство (1):

$$\sum_{k=1}^m \int\limits_0^{\infty}(-1)^k a_k\,\varphi^{(k)}(x)W(x)\,dx=
\sum_{k=1}^m\sum_{s=o}^{k-1}(-1)^{k-1-s}a_k\,\varphi^{(k-1-s)}(0)\,W^{(s)}s(0)+
\sum_{k=1}^m \int\limits_0^{\infty} W^{(k)}a_k \varphi(x)\,dx,$$

(при $k=0$ $\mathcal L\varphi(x)=a_0\varphi(x)$).

$$(E,\mathcal L\varphi(x))=\underbrace{
\sum_{k=1}^m\sum_{s=o}^{k-1}(-1)^{k-1-s}a_k\,\varphi^{(k-1-s)}(0)\,W^{(s)}s(0)}_{=\varphi(0)}+
\int\limits_0^{\infty}\underbrace{\mathcal L
W(x)a_k\varphi(x)}_{=0}.$$

Так как $W=0,\, W(0)=\ldots=W^{(m-2)}(0)=0,\,W^{(m-1)}(0)=1,$ то
$(\mathcal L E,\varphi)=\varphi(0),$ т.е. $\mathcal
L\,E(x)=\delta(x).\,\blacksquare$

%lect15

\section{Задача Коши для обыкновенных дифференциальных уравнений}

$w^{(m)}+a_{m-1}w^{(m-1)}+\ldots+a_0w=f(x),\,
w(0)=\ldots=w^{(m-1)}(0)=0,\,a_k\in \mathbb
C,\,k=\overline{0,m-1},$

$w_k\in\mathbb C,\,f\in C^{\infty}(\mathbb R)$ --- бесконечно
гладкая функция.

Это и есть задача Коши для обыкновенного дифференциального
уравнения.

Пусть $\tilde w(x)=\begin{cases} w(x), x\ge0;\\ 0,
x<0,\end{cases}$ $\tilde f(x)=\begin{cases} f(x), x\ge0;\\ 0,
x<0.\end{cases}$

$\mathcal L=\Bigl(\cfrac d{dx}\Bigr)^m+a_{m-1}\Bigl(\cfrac
d{dx}\Bigr)^{(m-1)}+\ldots+a_0\underbrace{\Bigl(\,\,\,\cdot\,\,\,\Bigr)}_{\mbox{сам
оператор}}$

$\tilde w(x)\in D'(\mathbb R),\,\varphi\in\,D(\mathbb R)\,\,
(\mathcal L\tilde w(x),\varphi(x))=(\tilde w(x),\mathcal
L^*\varphi(x)),$ где

$$\mathcal
L^*=(-1)^m\Bigl(\cfrac
d{dx}\Bigr)^{(m)}+(-1)^{m-1}a_{m-1}\Bigl(\cfrac
d{dx}\Bigr)^{(m-1)}+\ldots+a_0.$$

Тогда $$(\tilde w(x),\mathcal L^*\varphi(x))=
\int\limits_{-\infty}^{\infty}\tilde w(x)\,\mathcal
L^*\varphi(x)\,dx=\int\limits_0^{\infty} w(x)\,\mathcal
L^*\varphi(x)\,dx=\int\limits_0^{\infty}
w(x)\varphi^{\,(k)}(x)\,dx.$$

Обозначим $a_m\equiv1.$ Тогда имеем:
$$(-1)^k\int\limits_0^{\infty} w(x)\varphi^{\,(k)}(x)\,dx=
\left.(-1)^k w(x)\varphi^{\,(k-1)}(x)\right|_0^{\infty}+
(-1)^{k-1}\int\limits_0^{\infty} w'(x)\varphi^{\,(k-1)}(x)\,dx=$$

$$=(-1)^k w(0)\varphi^{\,(k-1)}(0)+\left.(-1)^{k-1} w(x)\varphi^{\,(k-2)}(x)\right|_0^{\infty}+
(-1)^{k-2}\int\limits_0^{\infty}
w''(x)\varphi^{\,(k-2)}(x)\,dx=\ldots=$$

$$=(-1)^{k-1}\underbrace{w(0)}_{w_0}\varphi^{\,(k-1)}(0)+(-1)^{k-2} \underbrace{w'(0)}_{w_1}\varphi^{\,(k-2)}(0)+\ldots+
\underbrace{w^{(k-1)}(0)}_{w_{k-1-s}}\varphi(0)+\int\limits_0^{\infty}
w^{(k)}(x)\varphi(x)\,dx=$$

$$=\sum_{s=0}^{k-1}(-1)^s w_{k-1-s}\varphi^{\,(k)}(0)+
\int\limits_0^{\infty} w^{(k)}(x)\varphi(x)\,dx.$$

Тогда
$$(\mathcal L\tilde w(x),\varphi(x))=
\sum_{k=1}^m a_k\sum_{s=0}^{k-1}(-1)^s
w_{k-1-s}\varphi^{\,(k)}(0)+
\int\limits_0^{\infty}\underbrace{\mathcal L
w^{(k)}(x)}_{f(x)}\varphi(x)\,dx.$$

Учитывая, что $(\delta(x),\varphi(x))=(-1)^s\varphi^{\,s}(0),$
получим:

$$\mathcal L\tilde w(x)=\tilde f(x)+\sum_{k=1}^m a_k\sum_{s=0}^{k-1}(-1)^s
w_{k-1-s}\delta^{(s)}(x).$$

$1\le s+1\le k\le m$ и переставим порядок суммирования:

$$\mathcal L\tilde w(x)=\tilde f+\sum_{s=0}^{m-1}\delta^{(s)}(x) \sum_{k=s+1}^m a_k w_{k-1-s}.\eqno (1)$$

Получилось такое уравнение, которому удовлетворяет функция $\tilde
w.$

\textbf{Пример.} $$\left\{
\begin{array}{ll}
    w''(x)+\omega^2\, w(x)=f(x), \omega=\const  , \\
    w(0)=w_0, \\
    w'(0)=w_1.  \\
\end{array}
\right.$$

$$\tilde w(x)=\begin{cases} w(x), x\ge0;\\ 0, x<0,\end{cases}
\tilde f(x)=\begin{cases} f(x), x\ge0;\\ 0, x<0.\end{cases}$$

Тогда $\tilde w''(x)+\omega^2\, \tilde w(x)=\tilde
f(x)+w_1\delta(x)+w_0\delta'(x)$.

Первая часть уравнения (1) является обобщенной функцией из
пространства $D'(\mathbb R)$ с носителем, принадлежащем множеству
$[0,\infty).$ Известно, что у оператора $\mathcal L$ существует
фундаментальное решение, с носителем, так же принадлежащим
$[0,\infty).$ возникает вопрос: существует ли свертка? Если да, то
решение существует и единственно. Обозначим через $A$ множество
обобщенных функций с носителем, принадлежащем множеству
$[0,\infty).$

\textbf{Теорема.} $A$ является коммутативной алгеброй над $\mathbb
C$ с единицей.

$\square$

Пусть $f,g\in A,$ т.е. $f,g\in D'(\mathbb R)$ и при этом $\supp
f\subset[0,\infty).$ Выберем компактное исчерпание единицы.
Возьмем $\Theta_{\varepsilon}\in C^{\infty}(\mathbb R)\colon
\left.\Theta_{\varepsilon}\right|_{[-\frac{\varepsilon}2,\infty)}=1,
\left.\Theta_{\varepsilon}\right|_{(-\infty,\varepsilon]}\equiv0.$
Пусть $\eta_k(x,y)$ --- компактное исчерпание единицы в $\mathbb
R^2,$ и пусть $\varphi\in\,D(\mathbb R).$ Тогда
$$\Bigl(f(x)\,g(y),\eta_k(x,y)\,\varphi(x+y)\Bigr)=
\Bigl(\Theta_{\varepsilon}(x)\,f(x)\,\Theta_{\varepsilon}(y)\,g(y),\eta_k(x,y)\,\varphi(x+y)\Bigr)=$$
$$=\Bigl(f(x)\,g(y),\Theta_{\varepsilon}(x)\,\Theta_{\varepsilon}(y)\,\eta_k(x,y)\,\varphi(x+y)\Bigr)$$

$\supp \varphi\subset[-A,A].$

Т.о., последовательность
$$\Bigl(f(x)\,g(y),\eta_k(x,y)\,\varphi(x+y)\Bigr)=
\Bigl(f(x)\,g(y),\Theta_{\varepsilon}(x)\,\Theta_{\varepsilon}(y)\,\eta_k(x,y)\,\varphi(x+y)\Bigr),\,k=1,2,\ldots,$$
стабилизируется с некоторого достаточно большого $k$.

Другими словами ,существует предел
$$\Bigl(f*g(x),\varphi(x)\Bigr)
=\lim\limits_{k\to\infty}\Bigl(f(x)\,g(y),\eta_k(x,y)\,\varphi(x+y)\Bigr)$$

Тем самым доказано существование свертки $f*g.$

Покажем, что $\supp (f*g)\subset[0,\infty).$

Пусть $\varphi \in D(\mathbb R\setminus [0,\infty)),$ причем
$\supp \varphi\subset[-A,A],\, A>0,\,a>0.$

Возьмем $\varepsilon>0$ таким, чтобы $$\{(x,y)\colon -a\le x+y\le
a\}\cap\{(x,y)\colon x\ge-\varepsilon,\,
y\ge-\varepsilon\}=\varnothing.$$ Тогда для любого $k$ будем
иметь:
$\Bigl(f(x)\,g(y),\underbrace{\Theta_{\varepsilon}(x)\,\Theta_{\varepsilon}(y)\,\eta_k(x,y)\,\varphi(x+y)}_{=0}\Bigr)=0.$
Другими словами, $(f*g(x),\varphi(x))=0.\,\blacksquare$

\section{Обобщенная задача Коши для ОДУ с постоянными коэффициентами}

$$\forall \tilde w\in A,\,\forall w_k\in \mathbb
C,\,k=\overline{0,m-1}.\mbox{ Найти решение }\tilde w\in A \mbox{
уравнения (1).}$$

\textbf{Теорема.} Обобщенная задача Коши для ОДУ с постоянными
коэффициентами имеет решение, и оно единственно.

$\square$

У оператора $\mathcal L=\sum\limits_{k=0}^n a_k \Bigl(\cfrac
d{dx}\Bigr)^k$ существует фундаментальное решение $E(x)\in A$ (по
теореме из предыдущей лекции). Тогда существует свертка $E(x)$ с
правой частью уравнения (1), которая будет решением уравнения (1).
С другой стороны, если $\tilde w\in A$ является решением уравнения
(1), то $w(x)$ есть свертка правой части (1) с $E(x)$, так как
существует свертка $w(x)$ с правой частью уравнения (1).
$\,\blacksquare$

%lect16

\textbf{Утверждение.} Пусть $\Omega\subset\mathbb R^n$ ---
ограниченная область с кусочно гладкой границей, $f\in
C^1(\Omega)\cap C(\overline{\Omega}).$ Тогда
$$\int\limits_{\Omega}\cfrac{\partial f}{\partial
x_i}\,dx=\int\limits_{\partial\Omega}\,f\,\cos(\overline{\nu},x_i)\,ds,$$
где $\overline{\nu}$ --- вектор внешней нормали к
$\partial\Omega,\,dx=dx_1\ldots dx_n$ --- элемент $n-$мерного
объема, $ds$ --- элемент $(n-1)-$мерного объема поверхности
$\partial\Omega$.

$\square$

Предположим сначала ,что $\Omega$ --- бесконечно гладкая область.
По формуле Стокса,
$$\int\limits_{\Omega}d\omega=\int\limits_{\partial\Omega}\omega$$
для любой внешней формы $\omega$ размерности $n-1$ на $\Omega$,
где ориентация такова, что орт $x_1$ непрерывен "во вне области
$\Omega$". Если нормаль направлена внутрь, то
$$\int\limits_{\Omega}d\omega=(-1)^{n-1}\int\limits_{\partial\Omega}\omega.$$

Возьмем $\omega=(-1)^{i-1}f\,dx_1\wedge\ldots\wedge dx_{i-1}\wedge
dx_{i+1}\wedge\ldots\wedge dx_n.\,\blacksquare$

\textbf{Пример.} $\Omega\subset\mathbb R^n$ --- ограниченная
область.
$$\int\limits_{\Omega}\cfrac{\partial f}{\partial
x_1}\,\,dx_1\,dx_2=\int\limits_{\partial\Omega}\,f\,\cos(\overline{\nu},x_1)\,ds,$$

Если $\omega=f\,dx_2,$ то $d\omega=\cfrac{\partial f}{\partial
x_1}\,\,dx_1\wedge dx_2.$

$$\int\limits_{\partial\Omega}\,f\,dx_2=
\int\limits_{\partial\Omega}\,f\,\cos(\overline{\nu},x_1)\,ds,$$

$$\int\limits_{\Omega}\cfrac{\partial f}{\partial
x_2}\,\,dx_1\,dx_2=
\int\limits_{\partial\Omega}\,f\,\cos(x_2,\overline{\nu})\,ds.$$


Если $\omega=f\,dx_1,$ то $d\omega=-\cfrac{\partial f}{\partial
x_2}\,\,dx_1\wedge dx_2.$

$$-\int\limits_{\Omega}\cfrac{\partial f}{\partial
x_1}\,\,dx_1\,dx_2= \int\limits_{\partial\Omega}\,f\,dx_1=
-\int\limits_{\partial\Omega}\,f\,\cos(\overline{\nu},x_2)\,ds.$$

$$-\underbrace{\int\limits_{\partial\Omega}\,f\,\cos(x_2,\overline{\nu})\,ds}_
{\int\limits_{\Omega}d\omega}=
-\underbrace{\int\limits_{\partial\Omega}\,f\,\cos(\overline{\nu},x_2)\,ds}_
{\int\limits_{\partial\Omega}\omega},$$ т.е.
$$\int\limits_{\Omega}d\omega=\int\limits_{\partial\Omega}\omega.$$

\textbf{Пример.} (Классическая формула Лейбница.)

$\Omega=(a,b),\,\partial\Omega=\{a,b\}.$
$$\int\limits_a^b\cfrac{\partial f}{\partial
x_1}\,\,dx=
\underbrace{f(a)\,\cos(\overline{\nu_a},x)}_{\cos\pi=-1}+
\underbrace{f(b)\,\cos(\overline{\nu_b},x)}_{\cos
0=1}=-f(a)+f(b).$$

\textbf{Утверждение.} (Формула интегрирования по частям для случая
многих вещественных переменных.) (В случае кусочно гладких
приближаем гладкими.)

Пусть $\Omega\subset\mathbb R^n$ --- ограниченная область с
кусочно гладкой границей, $g,h\in C^1(\Omega)\cap
C(\overline{\Omega}).$ Тогда
$$\int\limits_{\Omega}\cfrac{\partial g}{\partial
x_i}\,h\,dx=\int\limits_{\partial\Omega}\,g\,h\,\cos(\overline{\nu},x_i)\,ds-
\int\limits_{\Omega}g\cfrac{\partial h}{\partial x_i}\,dx,$$ где
$\overline{\nu}$ --- вектор внешней нормали к
$\partial\Omega,\,dx=dx_1\ldots dx_n$ --- элемент $n-$мерного
объема, $ds$ --- элемент $(n-1)-$мерного объема поверхности
$\partial\Omega$.

$\square$

Возьмем $f=g\,h$. По предыдущему утверждению,
$$\int\limits_{\Omega}\cfrac{\partial f}{\partial
x_i}\,\,dx=
\int\limits_{\partial\Omega}\,f\,\cos(\overline{\nu},x_i)\,ds.$$

Т.е.,
$$\int\limits_{\Omega}\cfrac{\partial g}{\partial
x_i}\,h\,dx+\int\limits_{\Omega}g\cfrac{\partial h}{\partial
x_i}\,dx=
\int\limits_{\partial\Omega}\,g\,h\,\cos(\overline{\nu},x_i)\,ds.\,\blacksquare$$

\section{Фундаментальное решение оператора Лапласа}

\textbf{Теорема.} Пусть $n\ge2$.
$$E_n(x)=\left\{%
\begin{array}{ll}
    \cfrac1{2\pi}\,\ln|\,x|,\,n=2,      \\
    -\cfrac1{(n-2)|S_1|}\,\cfrac1{|\,x|^{n-2}},\,n\ge3, \\
\end{array}%
\right.    $$ где $|S_1|$ --- $(n-1)$--мерный объем единичной
сферы. Тогда $\triangle E_n(x)=\delta(x).$ Другими словами,
$E_n(x)$ --- фундаментальное решение оператора Лапласа $\triangle
=\cfrac{\partial^2 }{\partial x_1^2}+\ldots+\cfrac{\partial^2
}{\partial x_n^2}.$

$\square$

Имеем: $$\Bigl(\triangle E_n,\varphi(x)\Bigr)=
\Bigl(E_n,\triangle\varphi(x)\Bigr)= \int\limits_{\mathbb
R^n}E_n\,\triangle\varphi(x)\,dx,\,\forall \,\varphi \in D(\mathbb
R^n).$$

Возьмем $R$ настолько большим, чтобы $\supp \varphi\subset B_R$
--- шар с радиусом $R$ и центром в нуле. Тогда
$$\int\limits_{\mathbb
R^n}E_n\,\triangle\varphi(x)\,dx=
\int\limits_{B_R}E_n\,\triangle\varphi(x)\,dx=
\lim_{\varepsilon\to+0}\int\limits_{B_R\setminus
B_{\varepsilon}}E_n\,\triangle\varphi(x)\,dx.$$

$$\int\limits_{B_R\setminus
B_{\varepsilon}}E_n\,\triangle\varphi(x)\,dx=
\int\limits_{B_R\setminus
B_{\varepsilon}}E_n\,\sum_{i=1}^n\cfrac{\partial^2\varphi(x)}{\partial
x_i^2}\,dx= \int\limits_{\partial (B_R\setminus
B_{\varepsilon})}E_n\,\underbrace{\sum_{i=1}^n\cfrac{\partial\varphi(x)}{\partial
x_i}\cos(\overline{\nu},x_i)}_{\overline{\triangledown}
p\,\overline{\nu}=\cfrac{\partial\varphi(x)}{\partial \nu}}\,ds-$$

$$-\int\limits_{B_R\setminus
B_{\varepsilon}}\underbrace{\sum_{i=1}^n\cfrac{\partial
E_n(x)}{\partial x_i}\,\cfrac{\partial\varphi(x)}{\partial
x_i}}_{=\overline{\triangledown}E_n\,\overline{\triangledown}\varphi}\,dx=
\int\limits_{\partial (B_R\setminus
B_{\varepsilon})}E_n\cfrac{\partial\varphi(x)}{\partial
\overline{\nu}}\,ds- \int\limits_{B_R\setminus
B_{\varepsilon}}\underbrace{\sum_{i=1}^n\cfrac{\partial
E_n(x)}{\partial
x_i}\,\cos(\overline{\nu},x_i)}_{\overline{\triangledown}
E_n\overline{\nu}=\cfrac{\partial\varphi(x)}{\partial
\nu}}\varphi(x)\,ds+$$

$$+\int\limits_{B_R\setminus
B_{\varepsilon}}\underbrace{\sum_{i=1}^n\cfrac{\partial^2
E_n(x)}{\partial x_n^2}}_{=\triangle E_n(x)}\varphi(x)\,dx=
\int\limits_{B_R\setminus B_{\varepsilon}}E_n(x)\, \cfrac{\partial
\varphi}{\partial \nu}\,ds- \int\limits_{B_R\setminus
B_{\varepsilon}}\cfrac{\partial E_n(x)}{\partial
\nu}\varphi(x)\,ds+ \int\limits_{B_R\setminus
B_{\varepsilon}}E_n(x)\,\varphi(x)\,dx.$$

Имеем: $\triangle E_n(x)=0$ для всех $x\in \mathbb
R^n\setminus\{0\}.$ В самом деле, переходя к полярным координатам,
получим:
$$\triangle=\cfrac{\partial^2}{\partial
r^2}+\cfrac{n-1}r\,\cfrac{\partial}{\partial r}+
\cfrac1{r^2}\triangle_{S_1},$$ где $\triangle_{S_1}$ --- оператор
Лапласа--Бельтрами на единичной $(n-1)$--мерной сфере (он содержит
производные только по угловым координатам, поэтому
$\triangle_{S_1}E_n\equiv0$).
\begin{multline}
\triangle E_n=\cfrac{\partial^2 E_n}{\partial
r^2}+\cfrac{n-1}r\,\cfrac{\partial E_n}{\partial r}+
\cfrac1{r^2}\underbrace{\triangle_{S_1}E_n}_{=0}=\\=
\left\{%
\begin{array}{ll}
    \ln(r)+\cfrac1r\,( \ln(r))'=-\cfrac1{r^2}+\cfrac1{r^2}=0,\,n=2, \\
    -\cfrac1{(n-2)|S_1|}\,\left( \Bigl(\cfrac1{r^{n-2}}\Bigr)''+\cfrac{n-1}r\,\Bigl(\cfrac1{r^{n-2}}\Bigr)'\right)=0, n\ge3.\\
\end{array}%
\right.
\end{multline}

%lect17

\begin{multline}
\int\limits_{B_R\setminus
B_{\varepsilon}}E_n(x)\,\triangle \varphi(x)\,dx=\\=
\int\limits_{B_R\setminus B_{\varepsilon}}E_n(x)\, \cfrac{\partial
\varphi}{\partial \nu}\,ds- \int\limits_{B_R\setminus
B_{\varepsilon}}\cfrac{\partial E_n(x)}{\partial
\nu}\,\varphi(x)\,ds+\int\limits_{B_R\setminus
B_{\varepsilon}}\underbrace{E_n(x)}_{=0}\,\varphi(x)\,dx=I_1-I_2+I_3.
\end{multline}

Рассмотрим $$I_1=\int\limits_{B_R\setminus
B_{\varepsilon}}E_n(x)\, \cfrac{\partial \varphi}{\partial
\nu}\,ds= \int\limits_{S_\varepsilon}E_n(x)\, \cfrac{\partial
\varphi}{\partial \nu}\,ds\to0,\,\varepsilon\to+0,$$

$$\Bigl|\int\limits_{S_\varepsilon}E_n(x)\, \cfrac{\partial
\varphi}{\partial \nu}\,ds\Bigr|\le \|\varphi\|_{c^1(\mathbb
R^n)}\supp E_n(x)\int\limits_{S_\varepsilon}ds=
\left\{%
\begin{array}{ll}
    \|\varphi\|_{c^1(\mathbb
R^n)}\varepsilon\,\ln\cfrac12,\,n=2  \\
    \|\varphi\|_{c^1(\mathbb
R^n)}\cfrac{\varepsilon^{n-1}}{n-2}, n\ge3. \\
\end{array}%
\right.    $$

Рассмотрим $$I_2=\int\limits_{B_R\setminus
B_{\varepsilon}}\cfrac{\partial E_n(x)}{\partial
\nu}\,\varphi(x)\,ds.$$

$$\left.\cfrac{\partial E_n(x)}{\partial
\nu}\,\varphi(x)\right|_{x\in S_{\varepsilon}}=
\left\{%
\begin{array}{ll}
    \left.-\cfrac1{2\pi} \cfrac {\partial}{\partial r}\ln r\right|_{r=\varepsilon},\,n=2  \\

    -\left.\cfrac1{(n-2)|S_1|}\,\cfrac {\partial}{\partial r}\,\cfrac1{r^{n-2}}\right|_{r=\varepsilon},
     n\ge3. \\
\end{array}%
\right.= -\cfrac1{|S_1|}\,\cfrac1{\varepsilon^{n-1}} \Rightarrow$$

$$I_2=\int\limits_{S_\varepsilon}\cfrac{\partial E_n(x)}{\partial
\nu}\,\varphi(x)\,ds=
\cfrac1{|S_1|}\varepsilon^{n-1}\int\limits_{S_\varepsilon}\varphi(x)\,ds,$$

$$\cfrac1{|S_1|\,\varepsilon^{n-1}}\int\limits_{S_\varepsilon}\varphi(x)\,ds=
\cfrac1{|S_1|\,\varepsilon^{n-1}}\int\limits_{S_\varepsilon}(\varphi(x)-\varphi(0))\,ds+
\cfrac1{|S_1|\,\varepsilon^{n-1}}\int\limits_{S_\varepsilon}\varphi(0)\,ds\to$$

$$\to\cfrac{\varphi(0)}{|S_1|\,\varepsilon^{n-1}}
\underbrace{\int\limits_{S_\varepsilon}ds}_{|S_1|\,\varepsilon^{n-1}}=\varphi(0),\,\varepsilon\to+0
$$

В самом деле,
$$\cfrac1{|S_1|\,\varepsilon^{n-1}}\int\limits_{S_\varepsilon}|\varphi(x)-\varphi(0)|\,ds\le
\cfrac{\sup\limits_{|x|=\varepsilon}\Bigl|\varphi(x)-\varphi(0)\Bigr|}{|S_1|\,\varepsilon^{n-1}}
\underbrace{\int\limits_{S_\varepsilon}ds}_{|S_1|\,\varepsilon^{n-1}}=
\sup\limits_{|x|=\varepsilon}|\varphi(x)-\varphi(0)|\to0,\,\varepsilon\to+0.$$

Т.о., $(\triangle E(x),\varphi(x))=$
$$=\lim_{\varepsilon\to+0}\int\limits_{B_R\setminus
B_{\varepsilon}}E_n\,\triangle\varphi(x)\,dx=
\underbrace{\lim_{\varepsilon\to+0}\int\limits_{\partial(B_R\setminus
B_{\varepsilon})}E_n(x)\, \cfrac{\partial \varphi}{\partial
\nu}\,ds}_{=0}-
\lim_{\varepsilon\to+0}\int\limits_{\partial(B_R\setminus
B_{\varepsilon})}\cfrac{\partial E_n(x)}{\partial
\nu}\varphi(x)\,ds=\varphi_{\varepsilon}(0).\,\blacksquare$$


\section{Фундаментальное решение одномерного волнового оператора}

\textbf{Теорема.} Пусть $E_1(x,t)=\cfrac1{2a}\,\Theta(a\,t-|x|),$
тогда $\square_a E_1(x,t)=\delta(x,t)$ в $\mathbb R^2,$ где
$$\square_a=\cfrac{\partial^2}{\partial
t^2}-a^2\cfrac{\partial^2}{\partial x^2},\,a=\const  >0.$$

$\square$

$\partial_t^2-a^2\partial_x^2=(\underbrace{\partial_t-a\,\partial_x}_{\partial_{\xi}})
(\underbrace{\partial_t+a\,\partial_x}_{\partial_{\xi}}),$

$$
\left\{%
\begin{array}{ll}
    \cfrac{\partial u}{\partial\xi}=\cfrac{\partial u}{\partial x}\,\cfrac{\partial x}{\partial\xi}+
    \cfrac{\partial u}{\partial t}\,\cfrac{\partial t}{\partial\xi},  \\
    \cfrac{\partial u}{\partial\eta}=\cfrac{\partial u}{\partial x}\,\cfrac{\partial x}{\partial\eta}+
    \cfrac{\partial u}{\partial t}\,\cfrac{\partial t}{\partial\eta}, \\
\end{array}%
\right.   \Rightarrow
\left\{%
\begin{array}{ll}
    \partial_\xi=\overbrace{\cfrac{\partial x}{\partial\xi}}^{=-a}\partial_x+
    \overbrace{\cfrac{\partial t}{\partial\xi}}^{=1}\partial_t,  \\
    \partial_\xi=\underbrace{\cfrac{\partial x}{\partial\eta}}_{=a}\partial_x+
    \underbrace{\cfrac{\partial t}{\partial\eta}}_{=1}\partial_t, \\
\end{array}%
\right. \Rightarrow
\left\{%
\begin{array}{ll}
    x=-a\,\xi +a\,\eta,  \\
    t=\xi + \eta. \\
\end{array}%
\right.
 $$

$(\partial_t-a\,\partial_x)(\partial_t+a\,\partial_x)E_1(x,t)=\delta(x,t).$

Применим формулу для замены переменных у обобщенных функций.

Пусть $f(x)\in C^{\infty}(\Omega),\, x=x(x')\colon
\Omega'\to\Omega$ --- диффеоморфизм.

$$\Bigl(f(x(x')),\varphi(x')\Bigr)=\int\limits_{\Omega}f(x(x'))\,\varphi(x')\,dx'=$$
$$=\int\limits_{\Omega}f(x)\,\varphi(x'(x))\,\Bigl|\det\Bigl\|
\cfrac{\partial x'}{\partial x}\Bigr\|\Bigr|\,dx=
\Bigl(f(x),\varphi(x'(x))\,\Bigl|\det\Bigl\| \cfrac{\partial
x'}{\partial x}\Bigr\|\Bigr|\Bigr)$$

$(x,t)$ --- старые координаты (без штриха), $(\xi,\eta)$ --- новые
координаты (штрихованные).

$$\Bigl(\delta(x(\xi,\eta),t(\xi,\eta)),\varphi(\xi,\eta) \Bigr)=
\Bigl(\delta(x,t),\varphi(\xi(x,t),\eta(x,t))\, \Bigl|\det\Bigl\|
\cfrac{\partial (\xi,\eta)}{\partial (x,t)}\Bigr\|\Bigr|\Bigr).$$

Т.к. $\eta=\cfrac{x+at}{2a},\,\xi=\cfrac{-x+at}{2a},$ то
$$\Bigl\|\cfrac{\partial (\xi,\eta)}{\partial (x,t)}\Bigr\|=
\begin{Vmatrix}
  \cfrac{\partial \xi}{\partial x} & \cfrac{\partial \xi}{\partial t} \\
  \cfrac{\partial \eta}{\partial x} & \cfrac{\partial \eta}{\partial t} \\
\end{Vmatrix}=
\begin{Vmatrix}
  -\cfrac1{2a} & \cfrac12 \\
   \cfrac1{2a} & \cfrac12  \\
\end{Vmatrix}=
-\cfrac1{2a} \Rightarrow
$$

$$\Bigl(\delta(x(\xi,\eta),t(\xi,\eta)),\varphi(\xi,\eta) \Bigr)=
\cfrac1{2a}\Bigl(\delta(x,t),\varphi(\xi(x,t),\eta(x,t))\Bigr)=
\cfrac1{2a}\,\varphi(\underbrace{\xi(0,0)}_{=0},\underbrace{\eta(0,0)}_{=0})=
\cfrac1{2a}\varphi(0)\Rightarrow$$

$$\delta(x(\xi,\eta),t(\xi,\eta))=\cfrac1{2a}\,\delta(\xi,\eta).$$

Т.о.
$(\partial_t-a\,\partial_x)(\partial_t+a\,\partial_x)E_1(x,t)=\delta(x,t)\Leftrightarrow$
$$\partial_{\xi}\,\partial_{\eta}E_1(x(\xi,\eta),t(\xi,\eta))=\cfrac1{2a}\,\delta(\xi,\eta).\eqno
(1)$$

Имеем:
$(\partial_t-a\,\partial_x)(\partial_t+a\,\partial_x)\Bigl(\Theta(\xi)\,\Theta(\eta)\Bigr)=
\Theta'(\xi)\,\Theta'(\eta)=\delta(\xi)\,\delta(\eta).$ Т.о.,
уравнение (1) выполнено, если
$E_1\Bigl(x(\xi,\eta),t(\xi,\eta)\Bigr)=\cfrac1{2a}\,\Theta(\xi)\,\Theta(\eta).$
Сделаем обратную замену переменных в последнем выражении:
$$E_1(x,t)=\cfrac1{2a}\,\Theta\Bigl(\cfrac{-x+at}{2a}\Bigr)\,\Theta\Bigl(\cfrac{x+at}{2a}\Bigr)=
\cfrac1{2a}\,\Theta\Bigl(at-|\,x|\Bigr).\,\blacksquare$$

Теперь рассмотрим случай, когда $n=3.$ Останется только $n=2,$ но
это выводится из случая $n=3.$

%lect19

\section{Фундаментальное решение трехмерного волнового оператора}

$\square_a=\cfrac{\partial^2}{\partial
t^2}-a^2\cfrac{\partial^2}{\partial x^2},\,a=\const  >0,\,
\triangle=\cfrac{\partial^2 u}{\partial x_1^2}+\cfrac{\partial^2
u}{\partial x_2^2}+\cfrac{\partial^2 u}{\partial x_3^2}.$

Определим обобщенную функцию $\delta_{S_r}(x)$ равенством:
$$(\delta_{S_r}(x),\varphi(x))=\int\varphi(x)\,ds.$$

\textbf{Теорема.} Пусть $E_3(x,t)=\cfrac{\Theta(t)}{4\pi
a^2t}\,\delta_{S_r}(x),$ тогда $\square_a E_3(x,t)=\delta(x,t)$ в
$\mathbb R^3.$

$\square$

$x=(x_1,x_2,x_3).$ По определению обобщенных функций, имеем:
$$\Bigl(\cfrac{\Theta(t)}{4\pi
a^2t}\,\delta_{S_{at}}(x),\varphi(x,t) \Bigr)= \cfrac1{4\pi
a^2}\int\limits_0^{\infty}\cfrac
{dt}t\int\limits_{S_{at}}\varphi(x,t)\,ds_x.$$

$$\Bigl(\square_a E_3(x,t),\varphi(x,t) \Bigr)=\Bigl(E_3(x,t),\square_a\varphi(x,t) \Bigr)=
\cfrac1{4\pi a^2}\int\limits_0^{\infty}\cfrac
{dt}t\int\limits_{S_{at}}\Bigl(\varphi_{tt}(x,t)-a^2\triangle\varphi(x,t)\Bigr)\,ds_x=$$

$$=\cfrac1{4\pi a^2}\int\limits_0^{\infty}\cfrac
{dt}t\int\limits_{S_{at}}\varphi_{tt}(x,t)\,ds-
\cfrac1{4\pi}\int\limits_0^{\infty}\cfrac
{dt}t\int\limits_{S_{at}}\triangle\varphi(x,t)\,ds_x.$$

Сделав замену,
$r=at,\,\varphi_{tt}=a^2\varphi_{rr},\,\cfrac{\partial
\varphi}{\partial t}=\cfrac{\partial \varphi}{\partial
r}\,\cfrac{\partial r}{\partial t}=a\,\cfrac{\partial
\varphi}{\partial
r},\,\cfrac{dt}t=\cfrac{dr}r,\,\psi(x,r)=\varphi\Bigl(x,\frac
ra\Bigr),$ продолжим цепочку равенств:

$$=\cfrac1{4\pi}\int\limits_0^{\infty}\cfrac
{dr}r\int\limits_{S_{r}}\psi_{rr}(x,r)\,ds-
\cfrac1{4\pi}\int\limits_0^{\infty}\cfrac
{dr}r\int\limits_{S_{r}}\triangle\psi(x,r)\,ds=$$

$$=\Bigl.\cfrac1{4\pi}\int\limits_{\mathbb R^3}\psi_{rr}(x,r)\Bigr|_{r=|x|}\cfrac{dx}{|\,x|}-
\Bigl.\cfrac1{4\pi}\int\limits_{\mathbb
R^3}\triangle_x\psi(x,r)\Bigr|_{r=|x|}\cfrac{dx}{|\,x|}.$$

Выберем $R>0$ таким, чтобы $\psi(0,r)\subset B_r$:
$$\Bigl.\cfrac1{4\pi}\int\limits_{B_R\setminus
B_{\varepsilon}}\triangle_x\psi(x,r)\Bigr|_{r=|x|}\cfrac{dx}{|\,x|}=$$

$$=\int\limits_{\partial(B_R\setminus B_{\varepsilon})}
\left.\cfrac{\partial \psi(x,r)}{\partial
\nu_x}\,\right|_{r=|x|}\cfrac{ds}{|\,x|}-
\Bigl.\cfrac1{4\pi}\int\limits_{B_R\setminus
B_{\varepsilon}}\triangledown_x\psi(x,r)\Bigr|_{r=|x|}
\triangledown|\,x|\cfrac{dx}{|\,x|}-
\Bigl.\cfrac1{4\pi}\int\limits_{B_R\setminus
B_{\varepsilon}}\triangledown_x\psi(x,r)\Bigr|_{r=|x|}
\triangledown\frac1{|\,x|}.\eqno (1)$$

Поясним:

$$\triangle_x\psi(x,r)\Bigr|_{r=|x|}
\frac1{|\,x|}=\left.\sum_{i=1}^{3}
\cfrac{\partial^2\psi(x,r)}{\partial
x_i^2}\right|_{r=|x|}\frac1{|\,x|}= \sum_{i=1}^{3}
\cfrac{\partial}{\partial x_i}
\left(\left.\cfrac{\partial\psi(x,r)}{\partial
x_i}\,\right|_{r=|x|}\frac1{|\,x|}\right)-$$

$$-\left.\sum_{i=1}^{3}
\cfrac{\partial^2\psi(x,r)}{\partial r\partial
x_i}\,\right|_{r=|x|}\cfrac{\partial|\,x|}{\partial
x_i}\,\,\frac1{|\,x|}- \left.\sum_{i=1}^{3}
\cfrac{\partial\psi(x,r)}{\partial
x_i}\,\right|_{r=|x|}\cfrac{\partial}{\partial
x_i}\,\,\frac1{|\,x|}$$

Теперь распишем второй интеграл из (1):

$$\Bigl.\int\limits_{B_R\setminus
B_{\varepsilon}}\triangledown_x\psi_r(x,r)\Bigr|_{r=|x|}
\triangledown|\,x|\cfrac{dx}{|\,x|}=
\int\limits_{\partial(B_R\setminus B_{\varepsilon})}
\Bigl.\psi_r(x,r)\Bigr|_{r=|x|}\cfrac{\partial |\,x|}{\partial
\nu}\,\cfrac1{|\,x|}\,ds-$$
$$-\Bigl.\int\limits_{B_R\setminus
B_{\varepsilon}}\psi_r(x,r)\Bigr|_{r=|x|}
\underbrace{\triangledown |\,x|\,\,\triangledown
|\,x|}_{=1}\frac{dx}{|\,x|}- \Bigl.\int\limits_{B_R\setminus
B_{\varepsilon}}\psi_r(x,r)\Bigr|_{r=|x|} \dv \cfrac{\triangledown
|\,x|}{|\,x|}\,dx,$$

так как $\triangledown |\,x|\,\,\triangledown |\,x|=(\triangledown
|\,x|)^2=\Bigl|\mbox{замена: }|\,x|=r\Bigr|=(\triangledown
r)^2=1^2=1.$

Распишем третий интеграл из (1):

$$\Bigl.\cfrac1{4\pi}\int\limits_{B_R\setminus
B_{\varepsilon}}\triangledown_x\psi(x,r)\Bigr|_{r=|x|}
\triangledown\frac1{|\,x|}=$$
$$\int\limits_{\partial(B_R\setminus
B_{\varepsilon})} \left.\psi(x,r)\right|_{r=|x|}\cfrac{\partial
}{\partial \nu}\,\cfrac{ds}{|\,x|}\,-
\Bigl.\int\limits_{B_R\setminus
B_{\varepsilon}}\psi_r(x,r)\Bigr|_{r=|x|}\triangledown|\,x|\,
\triangledown\frac1{|\,x|}\,dx- \int\limits_{\partial(B_R\setminus
B_{\varepsilon})}
\left.\psi(x,r)\right|_{r=|x|}\underbrace{\triangledown\frac1{|\,x|}}_{=0}\,dx,$$

где $\nu$ --- внешняя нормаль к границе $\partial(B_R\setminus
B_{\varepsilon}),\,\left.\psi\right|_{S_R}\equiv0.$

Т.о.,
$$\Bigl.\int\limits_{B_R\setminus
B_{\varepsilon}}\triangle_x\psi(x,r)\Bigr|_{r=|x|}\cfrac{dx}{|\,x|}=
\Bigl.\int\limits_{\partial(B_R\setminus B_{\varepsilon})}
\cfrac{\partial\psi(x,r)}{\partial
\nu_x}\,\Bigr|_{r=|x|}\,\frac{ds}{|\,x|}-
\int\limits_{\partial(B_R\setminus B_{\varepsilon})}
\left.\psi_r(x,r)\right|_{r=|x|}\cfrac{\partial |\,x|}{\partial
\nu}\,\cfrac{ds}{|\,x|}\,-$$
$$-\Bigl.\int\limits_{B_R\setminus
B_{\varepsilon}}\psi_{rr}(x,r)\Bigr|_{r=|x|}\cfrac{dx}{|\,x|}\,-
\int\limits_{\partial(B_R\setminus B_{\varepsilon})}
\Bigl.\psi(x,r)\Bigr|_{r=|x|}\cfrac{\partial }{\partial
\nu}\,\cfrac1{|\,x|}\,ds\,+$$
$$+\Bigl.\int\limits_{B_R\setminus
B_{\varepsilon}}\psi_r(x,r)\Bigr|_{r=|x|}\Bigl(\underbrace{\dv
\cfrac{\triangledown |\,x|}{|\,x|} +\triangledown|\,x|\,
\triangledown\frac1{|\,x|}}_{=0\mbox{ (доказательство
ниже)}}\Bigr)\,dx.$$

Рассмотрим каждый интеграл подробнее.

Пятый: так как
$|\,x|=\Bigl(\sum_{i=1}^3x_i^2\Bigr)^{\frac12},\,\cfrac{\partial|\,x|}{\partial
x_i}=\cfrac {x_i}{\sum x_i^2}=\cfrac {x_i}{|\,x|},$ то

$$\dv
\cfrac{\triangledown |\,x|}{|\,x|} +\triangledown|\,x|\,
\triangledown\frac1{|\,x|}=\sum_{i=1}^3\cfrac{\partial}{\partial
x_i}\left(\cfrac {x_i}{|\,x|^2}\right)+ \sum_{i=1}^3\cfrac
{x_i}{|\,x|}\,\cfrac{\partial}{\partial
x_i}\left(\cfrac1{|\,x|}\right)=$$

$$=\cfrac 3{|\,x|^2}-2\sum_{i=1}^3\cfrac{x_i^2}{|\,x|^4}-\sum_{i=1}^3\cfrac{x_i^2}{|\,x|^4}=
\cfrac 3{|\,x|^2}-\cfrac 2{|\,x|^2}-\cfrac 1{|\,x|^2}=0.$$

Первый:

$$\left|\,\Bigl.\int\limits_{\partial(B_R\setminus B_{\varepsilon})}
\cfrac{\partial\psi(x,r)}{\partial
\nu_x}\,\Bigr|_{r=|x|}\,\frac{ds}{|\,x|}\right|=
\left|\,\Bigl.\int\limits_{S_{\varepsilon}}
\cfrac{\partial\psi(x,r)}{\partial
\nu_x}\,\Bigr|_{r=|x|}\,\frac{ds}{|\,x|}\right|\le
\int\limits_{S_{\varepsilon}}\left|\cfrac{\partial\psi(x,r)}{\partial
\nu_x}\right|_{r=m}\,\frac{ds}{|\,x|}\le$$

$$\|\psi\|_{C^1(\mathbb R^4)}\cfrac1{\varepsilon}
\underbrace{\int\limits_{S_{\varepsilon}}ds}_{=4\pi\varepsilon^2}=
\|\psi\|_{C^1(\mathbb
R^4)}\cfrac1{\varepsilon}\cdot4\pi\varepsilon^2\to0,\,\varepsilon\to+0.$$

Второй:

$$\left|\,\int\limits_{\partial(B_R\setminus B_{\varepsilon})}
\left.\psi_r(x,r)\right|_{r=|x|}\cfrac{\partial |\,x|}{\partial
\nu}\,\cfrac{ds}{|\,x|}\right|=
\left|\int\limits_{S_{\varepsilon}}\left.\psi_r(x,r)\right|_{r=|x|}\cfrac{\partial
|\,x|}{\partial \nu}\,\cfrac{ds}{|\,x|}\right|\le$$

$$\le\int\limits_{S_{\varepsilon}}\Bigl.|\psi_r(x,r)|\Bigr|_{r=|x|}\left|\cfrac{\partial
|\,x|}{\partial \nu}\right|\,\cfrac{ds}{|\,x|}\le
\|\psi\|_{C^1(\mathbb R^4)}\cfrac1{\varepsilon}
\underbrace{\int\limits_{S_{\varepsilon}}ds}_{=4\pi\varepsilon^2}=
\|\psi\|_{C^1(\mathbb
R^4)}\cfrac1{\varepsilon}\cdot4\pi\varepsilon^2\to0,\,\varepsilon\to+0.$$

Четвертый:

$$\int\limits_{\partial(B_R\setminus B_{\varepsilon})}
\Bigl.\psi(x,r)\Bigr|_{r=|x|}\cfrac{\partial }{\partial
\nu}\,\cfrac1{|\,x|}\,ds=
\int\limits_{S_{\varepsilon}}\left.\psi(x,r)\right|_{r=|x|}\cfrac{\partial}{\partial
\nu}\,\cfrac1{|\,x|}\,ds,$$

так как $$\left.\cfrac{\partial}{\partial
\nu}\,\cfrac1{|\,x|}\right|_{x\in S_{\varepsilon}}=\left.-\cfrac
d{dr}\,\cfrac
1{dr}\right|_{r=\varepsilon}=\cfrac1{\varepsilon^2}.$$ Продолжим
равенства:

$$=\cfrac1{\varepsilon^2}\int\limits_{S_{\varepsilon}}(\psi(x,\varepsilon)-\psi(0))\,ds+
\cfrac{\psi(0)}{\varepsilon}\underbrace{\int\limits_{S_{\varepsilon}}ds}_{=4\pi\varepsilon^2}\to\psi(0),\,\varepsilon\to+0,$$

так как
$$\cfrac1{\varepsilon^2}\int\limits_{S_{\varepsilon}}|\psi(x,\varepsilon)-\psi(0)|\,ds\le
\sup|\psi(x,\varepsilon)-\psi(0)|\cfrac1{\varepsilon^2}\int\limits_{S_{\varepsilon}}ds=
4\pi \sup_{x\in
S_{\varepsilon}}|\psi(x,\varepsilon)-\psi(0)|\to0,\,\varepsilon\to+0.$$

Т.о.,
$$\Bigl.\int\limits_{\mathbb R^3}\triangle_x\psi(x,r)\Bigr|_{r=|x|}\cfrac{dx}{|\,x|}=
\lim_{\varepsilon\to+0}\Bigl.\int\limits_{B_R\setminus
B_{\varepsilon}}\triangle_x\psi(x,r)\Bigr|_{r=|x|}\cfrac{dx}{|\,x|}=
\Bigl.\int\limits_{\mathbb
R^3}\psi_{rr}(x,r)\Bigr|_{r=|x|}\cfrac{dx}{|\,x|}-4\pi\psi(0)\Rightarrow$$

$$\Bigl(\square_a E_3(x,t),\varphi(x,t) \Bigr)=\Bigl(E_3(x,t),\square_a\varphi(x,t) \Bigr)=
\cfrac1{4\pi a^2}\int\limits_0^{\infty}\cfrac
{dt}t\int\limits_{S_{at}}\Bigl(\varphi_{tt}(x,t)-a^2\triangle\varphi(x,t)\Bigr)\,ds_x=$$

$$=\cfrac1{4\pi}\Bigl.\int\limits_{\mathbb
R^3}\psi_{rr}(x,r)\Bigr|_{r=|x|}\cfrac{dx}{|\,x|}-
\cfrac1{4\pi}\Bigl.\int\limits_{\mathbb
R^3}\psi_{rr}(x,r)\Bigr|_{r=|x|}\cfrac{dx}{|\,x|}+
\cfrac1{4\pi}\cdot4\pi\psi(0)=\psi(0). \,\blacksquare$$

%lect20

\section{Фундаментальное решение двумерного волнового оператора}

$\square_a=\cfrac{\partial^2}{\partial
t^2}-a^2\cfrac{\partial^2}{\partial x^2},\,a=\const  >0,\,
\triangle=\cfrac{\partial^2 u}{\partial x_1^2}+\cfrac{\partial^2
u}{\partial x_2^2}.$

\textbf{Определение.} Обобщенная функция $f\in D'(\mathbb
R^{n-1})$ \textbf{допускает продолжения на бесконечно гладкие
функции} вида $\varphi(x)\,1(x_{n+1}),$ где $\varphi(x')\in
D(\mathbb R^{n}),\,x'=(x_1,\ldots,x_n),$ если для любого
компактного исчерпания единицы $\eta_k(x_{n+1}),\,k=1,2,\ldots,$ в
$\mathbb R$ существует
$$\lim_{h\to0}(f(x),\varphi(x')\eta_k(x_{n+1}))=
(F(x),\varphi(x')),\eqno (1)$$ где $F(x)$ --- продолжение функции
$f(x)$.

Напомним, что последовательность функций $\eta_k\in L_1(\mathbb
R),\,k=1,2,\ldots$ --- \textbf{компактное исчерпание единицы } в
$\mathbb R,$ если:

1) для любого компакта $H\Subset\mathbb R\,\,
\left.\eta_k\right|_H=1,$ начиная с некоторого $k_0$;

2) $\exists\,m\,\exists\, A=\const  \colon \forall
k=1,2,\ldots\,\, \|\eta_k\|_{C^m(\mathbb R)}\le A.$

\textbf{Упражнение.} Доказать, что предел в (1) не зависит от
выбора компактного исчерпания единицы $\eta_k$.

\textbf{Теорема.} Пусть
$$\mathcal
L=\sum\limits_{q=1}^m\Bigl(\cfrac{\partial}{\partial
x_{n+1}}\Bigr)^q L_q+L_0,$$
где $$L_q=\sum\limits_{|\alpha|'\le
m_q}a_{q,\alpha'}{\partial^{\, \alpha'}},\,q=\overline{0,m},\,
\alpha'=(\alpha'_1,\ldots,\alpha'_n),$$
$|\,\alpha'|=\alpha'_1+\ldots+\alpha'_n,
\,\partial^{\,\alpha'}=\cfrac {\partial^{\,|\alpha'|}}{\partial
x_1^{\,\alpha'_1}\ldots
\partial x_n^{\,\alpha'_n}}.$ Тогда, если $\mathcal
L u=f(x')\delta(x_{n+1}),\,f(x')\in D'(\mathbb R^n),$ и при этом
обобщенная функция $u(x)\in D'(\mathbb R^{n+1})$ допускает
продолжение на бесконечно гладкие функции вида
$\varphi(x)\,1(x_{n}),$ где $\varphi(x')\in D(\mathbb
R^{n}),\,x'=(x_1,\ldots,x_n),$ то продолжение $U(x')$ решения
$u(x)$ удовлетворяет уравнению $\mathcal L U(x')=f(x').$

$\square$

Пусть $\varphi(x')\in D(\mathbb R^{n+1}),\,\eta_k(x_{n+1})$ ---
компактное исчерпание единицы в $\mathbb R,$ тогда
$$\Bigl(\mathcal L
u(x'),\varphi(x')\eta_k(x_{n+1})\Bigr)=\Bigl(u(x),\mathcal
L^*(\varphi(x')\eta_k(x_{n+1}))\Bigr),$$ где
$$\mathcal L^*=\sum\limits_{q=1}^m (-1)^q \Bigl( \cfrac{\partial}{\partial
x_{n+1}}\Bigr)^q \mathcal L^*_q + \mathcal L^*_0,\,\,\,\mathcal
L^*_q=\sum\limits_{|\alpha|'\le
m_q}(-1)^{|\alpha|'}a_{q,\alpha'}{\partial^{\,
\alpha'}},\,q=\overline{0,m},$$ --- это оператор, формально
сопряженный к $\mathcal L.$

Т.о., $\Bigl(u(x'),\mathcal L^*(\varphi(x')\eta_k(x_{n+1}))\Bigr)=
\sum\limits_{q=1}^m (-1)^q \eta_k^{(q)}(x_{n+1})\mathcal L^*_q
\varphi(x')+\eta_k(x_{n+1})\mathcal L^*_0\varphi(x'),$ при этом
для $q=\overline{1,m}$ получим
$\Bigl(U(x),\eta_k^{(q)}(x_{n+1})\mathcal
L^*_q\varphi(x')\Bigr)\to0,\,k\to{\infty}.$

В самом деле, новое компактное исчерпание будет строиться так:
$\lambda_k(x_{n+1})=\eta_k(x_{n+1})+\eta_k^{(q)}(x_{n+1}).$
Поэтому
$\lim\limits_{k\to\infty}\Bigl(u(x),\lambda_k(x_{n+1})\underbrace{\mathcal
L^*_q(\varphi(x'))}_{\in D(\mathbb R^n)}\Bigr)=
\lim\limits_{k\to\infty}\Bigl(u(x),\eta_k(x_{n+1})\mathcal
L^*_q\varphi(x')\Bigr).$

Другими словами,
$$\lim_{k\to\infty}\Bigl(u(x),\eta_k(x_{n+1})\mathcal
L^*_q\varphi(x')\Bigr)+\underbrace{\lim_{k\to\infty}\Bigl(u(x),\eta_k^{(q)}(x_{n+1})\mathcal
L^*_q\varphi(x')\Bigr)}_{=0}=\lim_{k\to\infty}\Bigl(u(x),\eta_k(x_{n+1})\mathcal
L^*_q\varphi(x')\Bigr). $$

  Т.о. $$\lim_{k\to\infty}\Bigl(\mathcal
L u(x),\eta_k(x_{n+1})\varphi(x')\Bigr)=$$
$$= \sum\limits_{q=1}^m
(-1)^q (u(x),\eta_k^{(q)}(x_{n+1})\mathcal L^*_q \varphi(x'))+
(u(x),\eta_k(x_{n+1})\underbrace{\mathcal L^*_0\varphi(x'))}_{\in
D(\mathbb R^n)}\to (U(x'),\mathcal L^*_0\varphi(x')),$$ где
$U(x')$ --- продолжение непрерывной обобщенной функции $u(x)\in
D'(\mathbb R^{n+1})$ на бесконечно гладкие вида
$\psi(x)\,1(x_{n+1}),$ где $\psi(x')\in D(\mathbb R^{n}).$ При
этом, очевидно,
$$\Bigl(\mathcal
 L u(x),\eta_k(x_{n+1})\varphi(x')\Bigr)=
\Bigl(f(x')\delta(x_{n+1}),\varphi(x')\eta_k(x_{n+1})\Bigr)=
\Bigl(f(x'),(\delta(x_{n+1}),\varphi(x')\eta_k(x_{n+1}))\Bigr)=$$
$$=\Bigl(f(x'),\eta_k(0)\varphi(x'))\Bigr)\to\Bigl(f(x'),\varphi(x')\Bigr),\,
k\ge k_0\colon \eta_{k_0}(0)=1.$$

Т.о., $\Bigl(\mathcal L
U(x'),\varphi(x')\Bigr)=\Bigl(f(x'),\varphi(x')\Bigr),$ т.е.
$\mathcal L U(x')=f(x').\,\blacksquare$

\textbf{Теорема.} Пусть $E_2(x,t)=\cfrac{\Theta(at-|\,x|)}{2\pi
a\sqrt{a^2t^2-|\,x|^2}},$ где $x=(x_1,x_2)\in \mathbb R^2,$ тогда
$\square_a E_2(x,t)=\delta(x,t)$ в $\mathbb R^3,$ где
$\square_a=\cfrac{\partial^2}{\partial
t^2}-a^2\Bigl(\cfrac{\partial^2 }{\partial
x_1^2}+\cfrac{\partial^2 }{\partial x_2^2}\Bigr),\,a=\const  >0.$

$\square$

Пусть $E_3(x_1,x_2,x_3,t)=\cfrac{\Theta(t)}{4\pi
a^2t}\,\delta_{S_r}(x_1,x_2,x_3),$ тогда, по предыдущей лекции,
$$\left(\cfrac{\partial^2}{\partial t^2}-a^2\Bigl(\cfrac{\partial^2
}{\partial x_1^2}+\cfrac{\partial^2 }{\partial
x_2^2}+\cfrac{\partial^2 }{\partial
x_3^2}\Bigr)\right)E_3(x_1,x_2,x_3,t)=\delta(x_1,x_2,x_3,t)=\delta(x_1,x_2,t)\cdot\delta(x_3).$$

Покажем, что $E_3(x_1,x_2,x_3,t)$ допускает продолжения на
бесконечно гладкие функции вида $\varphi(x_1,x_2)\,1(x_3),$ где
$\varphi(x_1,x_2,t)\in D(\mathbb R^{3}).$ Пусть $\eta_k(x_3)$ ---
компактное исчерпание единицы в $\mathbb R.$ Тогда имеем:

$$\Bigl(E_3(x_1,x_2,x_3,t),\varphi(x_1,x_2)\eta_k(x_3)\Bigr)=
\cfrac1{4\pi a^2}\int\limits_0^{\infty}\cfrac
{dt}t\int\limits_{S_{at}}\varphi(x_1,x_2,t)\eta_k(x_3)\,ds\to$$
$$\to\cfrac1{4\pi a^2}\int\limits_0^{\infty}\cfrac
{dt}t\int\limits_{S_{at}}\varphi(x_1,x_2,t)\,ds.\mbox { (по
тоереме Лебега об ограниченной сходимости)}$$

Посчитаем
$\int\limits_{x_1^2+x_2^2+x_3^2=a^2t^2}\varphi(x_1,x_2)\,ds.$

Так как $dx_1\,dx_2=\cos(\overline{\nu},x_3)\,ds,$ где
$$\cos(\overline{\nu},x_3)=\cfrac{x_3}{at}=\cfrac{\sqrt{a^2t^2-x_1^2-x_2^2}}{at}
\mbox{ и } ds=\cfrac{at}{\sqrt{a^2t^2-x_1^2-x_2^2}}\,dx_1\,dx_2,$$
то
$$\int\limits_{x_1^2+x_2^2+x_3^2=a^2t^2}\varphi(x_1,x_2)\,ds=
2at\int\limits_{x_1^2+x_2^2+x_3^2=a^2t^2}\cfrac{\varphi(x_1,x_2)\,dx_1\,dx_2}
{\sqrt{a^2t^2-x_1^2-x_2^2}}.$$

Подставим в исходный интеграл:

$$\Bigl(E_3(x_1,x_2,x_3,t),\varphi(x_1,x_2)\eta_k(x_3)\Bigr)\to
\cfrac1{2\pi a} \int\limits_0^{\infty}\cfrac {dt}t
\int\limits_{x_1^2+x_2^2+x_3^2=a^2t^2}\cfrac{\varphi(x_1,x_2,t)\,dx_1\,dx_2}
{\sqrt{a^2t^2-x_1^2-x_2^2}}=$$

$$=\Bigl(\cfrac{\Theta(at-|\,x|)}{2\pi
a\sqrt{a^2t^2-|\,x|^2}},\varphi(x_1,x_2)\eta_k(x_3)\Bigr),$$ где
$\cfrac{\Theta(at-|\,x|)}{2\pi a\sqrt{a^2t^2-|\,x|^2}}$ ---
продолжение функции $E_3(x_1,x_2,x_3,t)$ на
$\varphi(x_1,x_2,t)1(x_3).$ Тогда, по пред\-ыдущей теореме,
$$\left(\cfrac{\partial^2}{\partial t^2}-a^2\Bigl(\cfrac{\partial^2
}{\partial x_1^2}+\cfrac{\partial^2 }{\partial
x_2^2}\Bigr)\right)E_2(x_1,x_2,t)=\delta(x_1,x_2,t).\blacksquare$$

%lect21

\section{Обобщенная задача Коши для волнового уравнения}

$${\begin{aligned}
\left\{%
\begin{array}{ll}
    u_{tt}-a^2\triangle u=f(x,t),\\
    u(x,0)=u_0(x), \\
    u_t(x,0)=u_t(x),  \\
\end{array}%
\right.\end{aligned}}\eqno(*)$$

$$(x,t) \in \mathbb R^n\times[0,\infty),\,
\triangle=\partial^2_{x_1}+\ldots+\partial^2_{x_n},\,a=\const  .$$

$u\in C_2(\mathbb R^n\times[0,\infty))\cap C^1(\mathbb
R^n\times[0,\infty)), f\in C(\mathbb R^n\times[0,\infty)).$
Обозначим
$$\tilde u(x,t)=\left\{%
\begin{array}{ll}
    u(x,t), t\ge0, \\
    0, t<0. \\
\end{array}%
\right.$$

Нужно написать для $\tilde u$ явную часть.

Возьмем $\varphi\in\,D(\mathbb R^{n+1}).$ Тогда
\begin{multline}
(\tilde u_{tt}-a^2\tilde u_{xx},\varphi)=(\tilde
u,\varphi_{tt}-a^2\triangle\varphi)= \int\limits_{\mathbb
R^n}\tilde
u(x,t)(\varphi_{tt}(x,t)-a^2\triangle\varphi(x,t))\,dx\,dt=\\=
\int\limits_0^{\infty}dt \int\limits_{\mathbb R^n}dx\,\tilde
u(x,t)(\varphi_{tt}(x,t)-a^2\triangle\varphi(x,t))=\\=
\int\limits_{\mathbb
R^n}dx\,\int\limits_0^{\infty}dt\,u(x,t)\varphi_{tt}(x,t)-
a^2\int\limits_0^{\infty}dt \int\limits_{\mathbb
R^n}dx\,u(x,t)\,\triangle\varphi(x,t).
\end{multline}

$x\in \mathbb R^n$ --- фиксирован.

$$\int\limits_0^{\infty}dt\,u(x,t)\varphi_{tt}(x,t)=
\Bigl.u(x,t)\varphi_{t}(x,t)\Bigr|_{t=0}^{\infty}-
\int\limits_0^{\infty}u_t(x,t)\varphi_{t}(x,t)\,dt=$$
$$=\Bigl.u(x,t)\varphi_{t}(x,t)\Bigr|_{t=0}^{\infty}-
\Bigl.u_t(x,t)\varphi_{t}(x,t)\Bigr|_{t=0}^{\infty}+
\int\limits_0^{\infty}u_{tt}(x,t)\varphi(x,t)\,dt=$$
$$=-u_0(x)\varphi_{t}(x,0)+u_1(x)\varphi_{t}(x,0)+
\int\limits_0^{\infty}u_{tt}(x,t)\varphi(x,t)\,dt.$$

$t\in [0,\infty)$ --- фиксирован. $\supp \varphi\subset B_r$  ---
шар достаточно большого радиуса с центром в точке 0.

\begin{multline}
\int\limits_{\mathbb R^n}u(x,t)\,\triangle\varphi(x,t)\,dx=
\int\limits_{B_r}u(x,t)\,\triangle\varphi(x,t)\,dx=\\=
\int\limits_{S_r}u(x,t)\cfrac{\partial \varphi(x,t)}{\partial
\nu}\,ds-=\int\limits_{B_r}\triangledown_x
u(x,t)\,\triangledown_x\varphi(x,t)\,dx=\\=
\int\limits_{S_r}u(x,t)\underbrace{\cfrac{\partial \varphi(x,t)}{\partial
\nu}}_{\equiv0},ds- \int\limits_{S_r}\cfrac{\partial
\varphi(x,t)}{\partial
\nu}\underbrace{\varphi(x,t)}_{\equiv0}\,ds+\int\limits_{B_r}\triangle
u(x,t)\,\varphi(x,t)\,dx.
\end{multline}

\begin{multline}
(\tilde u_{tt}-a^2\tilde u_{xx},\varphi)=
-\int\limits_{\mathbb
R^n}u_0(x)\varphi_t(x,0)\,dx+\int\limits_{\mathbb
R^n}u_1(x)\varphi_t(x,0)\,dx+\\+
\int\limits_{\mathbb R^n}dx
\int\limits_0^{\infty}u_{tt}(x,t)\varphi(x,t)\,dt-
a^2\int\limits_0^{\infty}dt \int\limits_{\mathbb R^n}dx\,\triangle
u(x,t)\,\varphi(x,t)=\\=
-\int\limits_{\mathbb
R^n}u_0(x)\varphi_t(x,0)\,dx+\int\limits_{\mathbb
R^n}u_1(x)\varphi_t(x,0)\,dx+\int\limits_{\mathbb R^n}dx
\int\limits_0^{\infty}(\underbrace{u_{tt}-a^2u_{xx}}_{=f(x,t)})\,\varphi(x,t)=\\=
(u_0\delta'(t),\varphi(x,t))+(u_1\delta(t),\varphi(x,t))+
(\tilde\rho(x,t),\varphi(x,t)).
\end{multline}

Пояснение: по определению прямого произведения:

$$(u_0\delta'(t),\varphi(x,t))=(u_0,\delta'(t)\,\varphi(x,t)),$$

$$(\delta'(t),\varphi(x,t))=-(\delta(t),\varphi_t(x,t))=-\varphi_t(x,0).$$
Т.о., $$(u_0\delta'(t),\varphi(x,t))= -(u_0,\varphi_t(x,t))=
-\int\limits_{\mathbb R^n}u_0(x)\varphi_t(x,0)\,dx.$$

Аналогично,
$$(u_t\delta(t),\varphi(x,t))=-\int\limits_{\mathbb R^n}u_0(x)\varphi(x,t)\,dx,$$

$$(\tilde f(x,t),\varphi(x,t))=\int\limits_0^{\infty}dt\int\limits_{\mathbb R^n}f(x,t)\varphi(x,t)\,dx.$$

Т.о., получим:

$$\tilde
u_{tt}-a^2\tilde u_{xx}=\tilde
f(x,t)+u_0\,\delta'(t)+u_1\,\delta(t).\,\blacksquare\eqno (1)$$

\textbf{Замечание.} Носитель правой части (1) принадлежит
множеству $\mathbb R^n\times[0,\infty).$

\textbf{Лемма.} Пусть $F,G\in D'(\mathbb R^{n+1}),$ причем $\supp
G\subset K=\{(x,t)\in D\mathbb R^{n+1}\colon |\,x|\le at\}$ ---
замкнутый конус. Тогда существует $F*G$, причем $\supp F*G\subset
\mathbb R^n\times[0,\infty),$ более того:
$$\forall\varphi\in\,D(\mathbb R^{n+1})\,\,\,
(F*G,\varphi)=\Bigl(F(x,y)G(y,\tau),\lambda(t)\lambda(\tau)\lambda(a^2t^2-y^2)\varphi(x+y,t+\tau))\Bigr),$$
где $\lambda\in C^{\infty}(\mathbb R)\colon
\left.\lambda\right|_{(-\infty,-2\varepsilon)}\equiv0,\,
\left.\lambda\right|_{[-\varepsilon,\infty)}\equiv1,\,\varepsilon>0.$

$\square$

Видно ,что все справа - бесконечно гладкая функция с компактным
носителем. Пусть $\varphi\in\,D(\mathbb R^{n+1}).$ Возьмем
компактное исчерпание единицы $\eta_s(x,y,t,\tau)$ в $\mathbb
R^{2n+1}\colon$
$$\lim_{s\to\infty}(F(x,y,)G(y),\eta_s(x,y,t,\tau)\varphi(x+y,t+\tau))=$$
$$=\lim_{s\to\infty}(\lambda(t)F(x,y,),\underbrace{\lambda(\tau)\lambda(a^2t^2-y^2)G(y,t)}_{=G(y,t)}\eta_s(x,y,t,\tau)\varphi(x_y,t+\tau))=$$
$$=\lim_{s\to\infty}(F(x,y)G(y,\tau),\lambda(t)\lambda(\tau)\lambda(a^2t^2-|\,y|^2)\varphi(x+y,t+\tau)),\eqno(2)$$
где
$\lambda(t)\lambda(\tau)\lambda(a^2t^2-|\,y|^2)\varphi(x+y,t+\tau)$
--- бесконечно гладкая с компактным носителем, т.к. у $\varphi$
компактный носитель. $\lambda(t)\varphi(x+y,t+\tau)\ne0$ при
$t>-2\varepsilon,$ $\lambda(\tau)\varphi(x+y,t+\tau)\ne0$ при
$\tau>-2\varepsilon,$ $\lambda(a^2t^2-|\,y|^2)\ne0$ при $y>a\tau.$
Если $t>A, t+\tau\in(-\alpha,\alpha)\Rightarrow \tau$ ---
маленькое. Если $\tau$ большое, то $\lambda(t)=0,$ то
$-2\varepsilon<t\le A, \,-2\varepsilon<\tau\le A.$

$|\,y|^2<C.$ Если $y$ принадлежит компакту, $x+y$ принадлежит
компакту $\Rightarrow x$ принадлежит компакту, $|\,x|<k.$

$(\eta_s(x,y,t,\tau)=f)$ на $\supp
(\lambda(t)\lambda(\tau)\lambda(a^2t^2-|\,y|^2)\varphi(x+y,t+\tau)).$


Т.о., при достаточно больших $s$
$$(2)=(F(x,y)G(y,\tau),\lambda(t)\lambda(\tau)\lambda(a^2t^2-|\,y|^2)\varphi(x+y,t+\tau)).$$

Т.о., доказано, что предел (свертка) существует и формула верна.
$\blacksquare$

\textbf{Определение.} \textbf{Обобщенная задача Коши для волнового
уравнения:} пусть $F\in D'(\mathbb R^{n+1}), \supp F\subset\mathbb
R^n\times[0,\infty).$ Найти функцию $u\in D'(\mathbb
R^{n+1})\colon u\in\mathbb R^n\times[0,\infty)\colon$
$$u_{tt}-a^2\triangle u=F.\eqno (*)$$

\textbf{Теорема.} Обобщенная задача Коши $(*)$ имеет единственное
решение.

$\square$

(Существование.) Т.к. $\supp F\subset\mathbb R^n\times[0,\infty)$
и $\supp E\subset K_n,$ где $E$ --- фундаментальное решение, то
существует свертка $F*G$ --- решение $(*)$.

(Единственность). Т.к. существует $u*E$.$\,\blacksquare$

%lect22

$$\left\{%
\begin{array}{ll}
    \square_a u=f(x,t),x\in \mathbb R^n,t>0,\\
    u(x,0)=u_0(x),\\
    u_t(x,0)=u_1(x). \\
\end{array}%
\right.$$

$\square_a=\cfrac{\partial^2}{\partial
t^2}-a^2\cfrac{\partial^2}{\partial x^2},\,a=\const  >0,\,
\triangle=\cfrac{\partial^2 u}{\partial
x_1^2}+\ldots+\cfrac{\partial^2 u}{\partial x_n^2}.$

$u\in C^1(\mathbb R^n\times[0,\infty))\cap C^2(\mathbb
R^n\times(0,\infty)),\,u_0,u_1\in C(\mathbb R^n),\, f\in C(\mathbb
R^n\times[0,\infty)).$

Обозначим
$$\tilde u(x,t)=\left\{%
\begin{array}{ll}
    u(x,t), t\ge0, \\
    0, t<0, \\
\end{array}%
\right.
\tilde f(x,t)=\left\{%
\begin{array}{ll}
    f(x,t), t\ge0, \\
    0, t<0. \\
\end{array}%
\right.
$$

$$\square_a \tilde u(x,t)=f(x,t)+u_0(x)\delta'(t)+u_1(x)\delta(t)=F(x,t),$$
где $\supp F\subset \mathbb R^n\times[0,\infty).$

Обобщенная задача Коши:

$\forall F\in D'(\mathbb R^{n+1})\colon \supp F\subset \mathbb
R^n\times[0,\infty),$ найти $u\in D'(\mathbb R^{n+1})\colon
\square_a u(x,t)=F(x,t),$

 $\supp u\subset \mathbb
R^n\times[0,\infty).$

\textbf{Теорема.} Обобщенное решение задачи Коши существует и
единственно.

\section{Классическое решение задачи Коши\\для волнового уравнения}

$$\left\{%
\begin{array}{ll}
    \square_a u=f(x,t),x\in \mathbb R^n,t>0,\\
    u(x,0)=u_0(x),\\
    u_t(x,0)=u_1(x). \\
\end{array}%
\right.$$

Будем предполагать ,что на правую часть $f$ и начальные значения
$u_0,u_1$ следующие условия:

1) в случае $n=2,3$: $f\in C^1(\mathbb R^n\times[0,\infty)),\,
u_0\in C^3(\mathbb R^n),\,u_0\in C^2(\mathbb R^n);$

2) в случае $n=1$: $f\in C^1(\mathbb R^n\times[0,\infty)),\,
u_0\in C^2(\mathbb R^n),\,u_0\in C^1(\mathbb R^n);$

\textbf{Определение.} \textbf{Найти классическое решение задачи
Коши $(*)$} --- значит найти функцию $u\in C^1(\mathbb
R^n\times[0,\infty))\cup\in C^2(\mathbb R^n\times(0,\infty)),$
удовлетворяющую соотношениям $(*)$ в классическом (не обобщенном)
смысле.

\textbf{Теорема.} При сформулированных выше условиях на
$f,u_0,u_1$ классическое решение задачи Коши $(*)$ существует и
единственно. Более того, оно равно:

1) \textbf{формула Кирхгофа} $n=3$:
$$u(x,t)=
\cfrac1{4\pi
a^2}\int\limits_{B_{at}^x}\cfrac{f\Bigl(\xi,t-\Bigr|\frac{x-\xi}a\Bigl|\Bigr)}{|\,x-\xi|}\,d\xi+
\cfrac1{4\pi a^2} \int\limits_{S_{at}^x}u_1(\xi)\,
ds_{\varepsilon}+ \cfrac1{4\pi a^2}\, \cfrac{\partial}{\partial
t}\Bigl(\cfrac1t\int\limits_{S_{at}^x}u_0(\xi)\,
ds_{\varepsilon}\Bigr);$$

2) \textbf{формула Пуассона} $n=2$:
$$u(x,t)=\cfrac1{2\pi a}\int\limits_0^t d\tau\int\limits_{B_{a(t-\tau)}^x}
\cfrac{f(\xi,\tau)}{\sqrt{a^2(t-\tau)^2-|\,x-\xi|^2}}\,d\xi+$$
$$+\cfrac1{2\pi a}\int\limits_{B_{at}^x}
\cfrac{u_1(\xi)}{\sqrt{a^2(t-\tau)^2-|\,x-\xi|^2}}\,d\xi+
\cfrac1{2\pi a}\,\cfrac{\partial}{\partial
t}\int\limits_{B_{at}^x}
\cfrac{u_0(\xi)}{\sqrt{a^2(t-\tau)^2-|\,x-\xi|^2}}\,d\xi;$$

3) \textbf{формула Даламбера} $n=1$:

$$u(x,t)=\cfrac1{2a}\int\limits_0^t d\tau
\int\limits_{x-a(t-\tau)}^{x+a(t-\tau)} f(\xi,\tau)\,d\xi+
\cfrac1{2a}\int\limits_{x-at}^{x+at} u_1(\xi)\,d\xi+
\cfrac12\Bigl(u_0(x+at)+u_0(x-at)\Bigr).$$

$\square$

$$\square_a\tilde u=\tilde f(x,t)+u_0\,\delta'(t)+u_1\,\delta(t).\eqno
(**)$$

Свертка правой части $(**)$ с фундаментальным решением оператора
$\square_a$ существует и она равна $\tilde u(x,t),$ т.к. свертка
$\tilde u(x,t)$ с фундаментальным решением оператора $\square_a$
тоже существует. Надо показать, что свертка фундаментального
решения с правой частью $(**)$ определяется приведенными выше
формулами Кирхгофа, Пуассона и Даламбера (в зависимости от $n$).
Тем самым, будет доказана единственность решения. Чтобы доказать
существование, надо показать, что свертка фундаментального решения
оператора $\square_a$ имеет класс гладкости $C^1(\mathbb
R^n\times[0,\infty))\cap C^2(\mathbb R^n\times(0,\infty)).$

Фундаментальные решения оператора $\square_a$:

$$E_1(x,t)=\cfrac1{2a}\,\Theta(a\,t-|x|),\,n=1,$$
$$E_2(x,t)=\cfrac{\Theta(at-|\,x|)}{2\pi a\sqrt{a^2t^2-|\,x|^2}},\,n=2,$$
$$E_3(x,t)=\cfrac{\Theta(t)}{4\pi
a^2t}\,\delta_{S_at}(x),\,n=3.$$

В случае $n=1,2 \, E_n(x,t)$ --- локально суммируемые функции,
поэтому для свертки $E_n(x,t)$ с правой частью $(**)$ в этом
случае считается элементарно. Надо только учесть ,что
$$E_n(x,t)*(u_0(x)\delta'(t))= \cfrac{\partial}{\partial
t}\Bigl(E_n(x,t)*u_0(x)\delta(t)\Bigr).$$

Нетривиальным является лишь случай $n=3.$ Найдем свертку
$E_3(x,t)*\tilde f(x,t).$ По доказанному на предыдущей лекции,
$\forall \varphi\in D(\mathbb R^{n+1})$:
$$(E_3*\tilde f,\varphi)=
\Bigr(E_3(\xi,\tau),\lambda(t)\lambda(\tau)\lambda(a^2\tau^2-|\,\xi|^2)\varphi(x+\xi,t+\tau))\Bigr),$$
где $\lambda$ --- бесконечно гладкая функция.

Т.к. $$\Bigr(E_3(\xi,\tau),\psi(\xi,\tau)\Bigr)=
\left(\cfrac{\Theta(\tau)}{4\pi
a^2\tau}\,\delta_{S_a\tau}(\xi),\psi(\xi,\tau)\right)=
\cfrac1{4\pi
a^2}\int\limits_0^{\infty}\cfrac{d\tau}{\tau}\int\limits_{S_{a\tau}}\psi(\xi,\tau)\,
ds_{\varepsilon}=$$
$$\Bigl|\mbox{замена }a\tau=\rho\Bigr|=
\cfrac1{4\pi a^2} \int\limits_0^{\infty}\cfrac{d\rho}{\rho}
\int\limits_{S_{a\rho}}\psi\Bigl(\xi,\cfrac{\rho}a\Bigr)\,
ds_{\varepsilon}= \cfrac1{4\pi a^2} \int\limits_{\mathbb R^3}
\cfrac{\psi\Bigl(\xi,\frac{\rho}a\Bigr)}{|\,\xi|}\,d\xi,$$ то

$$(E_3*\tilde f,\varphi)=
\Bigr(E_3(\xi,\tau),\lambda(t)\lambda(\tau)\lambda(a^2\tau^2-|\,\xi|^2)\varphi(x+\xi,t+\tau))\Bigr)=$$
$$=\cfrac1{4\pi a^2} \int\limits_{\mathbb R^3}
\cfrac{\lambda(t)\lambda(\overbrace{\frac{|\,\xi|}a}^{=0})\lambda(\overbrace{a^2(\frac{|\,\xi|}a)^2-|\,\xi|^2}^{=0})
\varphi(x+\xi,t+\frac{|\,\xi|}a)}{|\,\xi|}\,d\xi=
\cfrac{\lambda(t)}{4\pi a^2} \int\limits_{\mathbb R^3}
\cfrac{\varphi(x+\xi,t+\frac{|\,\xi|}a)}{|\,\xi|}\,d\xi.$$

\begin{multline}(E_3*f,\varphi)=
\cfrac1{4\pi a^2}\int\limits_0^{\infty}dt\int\limits_{\mathbb
R^3}dx\,f(x,t)\overbrace{\lambda(t)}^{=1} \int\limits_{\mathbb
R^3} \cfrac{\varphi(x+\xi,t+\frac{|\,\xi|}a)}{|\,\xi|}\,d\xi=\\=
\cfrac1{4\pi a^2}\int\limits_0^{\infty}dt\int\limits_{\mathbb
R^3}dx\,f(x,t)\int\limits_{\mathbb R^3}
\cfrac{\varphi(y,t+\frac{|\,y-x|}a)}{|\,y-x|}\,dy= \cfrac1{4\pi
a^2}\int\limits_{\mathbb R^3}dx\int\limits_{\mathbb
R^3}dy\int\limits_0^{\infty}dt\,f(x,t)\cfrac{\varphi(y,t+\frac{|\,y-x|}a)}{|\,y-x|}=\\=
\cfrac1{4\pi a^2}\int\limits_{\mathbb R^3}dx
\int\limits_{\frac{|\,y-x|}a}^{\infty}dt\,f\Bigl(x,\Theta-\frac{|\,y-x|}a\Bigr)\cfrac{\varphi(y,\Theta)}{|\,y-x|}=\\=
\cfrac1{4\pi a^2} \int\limits_0^{\infty}d\Theta
\int\limits_{\mathbb R^3}dy\,\varphi(y,\Theta)
\int\limits_{|\,x-y|\le\, a\Theta}dx
\cfrac{f\Bigl(x,\Theta-\frac{|\,y-x|}a\Bigr)}{|\,y-x|}=\\=
\Bigl| \Theta\ge \frac{|\,y-x|}a \Leftrightarrow |\,y-x|\le\,a\Theta \Bigr|=
\left(\cfrac1{4\pi a^2}\int\limits_{B_{a\Theta}^x}f\Bigl(x,\Theta-\frac{|\,y-x|}a\Bigr)\,dx,\varphi(y,\Theta)\right),
\end{multline}
где
$$\cfrac1{4\pi a^2}\int\limits_{B_{a\Theta}^x}f\Bigl(x,\Theta-\frac{|\,y-x|}a\Bigr)\,dx$$
является первым слагаемым в формуле Кирхгофа.

%lect23

Второе слагаемое в формуле Кирхгофа:
$$\Bigl(u_1(x)\delta(t)*E_3(x,t),\varphi(x,t)\Bigr)=
\Bigl(u_1(x)\delta(t)*E_3(\xi,\tau),
\lambda(t)\lambda(\tau)\lambda(a^2\tau^2-|\,\xi|^2)\varphi(x+\xi,t+\tau)\Bigr)=$$
$$=\Bigl(u_1(x)\delta(t),
(E_3(\xi,\tau),\lambda(t)\lambda(\tau)\lambda(a^2\tau^2-|\,\xi|^2)\varphi(x+\xi,t+\tau))\Bigr).$$

Так как $$\Bigl(E_3(\xi,\tau),\psi(\xi,\tau)\Bigr)= \cfrac1{4\pi
a^2}\int\limits_{\mathbb R^3}
\cfrac{\psi(\xi,\frac{|\,\xi|}a)}{|\,\xi|}\,d\xi,$$ получим:
$$\Bigl(E_3(\xi,\tau),\lambda(t)\lambda(\tau)\lambda(a^2\tau^2-|\,\xi|^2)\varphi(x+\xi,t+\tau)\Bigr)=$$
$$=\cfrac1{4\pi a^2}\int\limits_{\mathbb R^3}
\cfrac{\lambda(t)\overbrace{\lambda(\frac{|\,\xi|}a)}^{=1}\lambda(a^2(\frac{|\,\xi|}a^2-|\,\xi|)^2)
\varphi(x+\xi,t+\frac{|\,\xi|}a)}{|\,\xi|}\,d\xi=
\cfrac{\lambda(t)} {4\pi a^2}\int\limits_{\mathbb
R^3}\cfrac{\varphi(x+\xi,t+\frac{|\,\xi|}a)}{|\,\xi|}\,d\xi,$$

$$\Bigl(u_1(x)\delta(t)*E_3(x,t),\varphi(x,t)\Bigr)=
\cfrac1{4\pi a^2}\Bigl(u_1(x)\delta(t),\int\limits_{\mathbb
R^3}\cfrac{\varphi(x+\xi,t+\frac{|\,\xi|}a)}{|\,\xi|}\,d\xi\Bigr)=$$
$$=\cfrac1{4\pi a^2}\Bigl(u_1(x),\Bigl(\delta(t),\int\limits_{\mathbb
R^3}\cfrac{\varphi(x+\xi,t+\frac{|\,\xi|}a)}{|\,\xi|}\,d\xi\Bigr)\Bigr)=
\cfrac1{4\pi a^2}\Bigl(\underbrace{u_1(x)}_{\in \,C^2(\mathbb
R^3)},\int\limits_{\mathbb
R^3}\cfrac{\varphi(x+\xi,\frac{|\,\xi|}a)}{|\,\xi|}\,d\xi\,dx\Bigr)=$$
$$=\cfrac1{4\pi a^2}\int\limits_{\mathbb
R^3}dy\int\limits_{\mathbb R^3}d\xi\,u_1(y-\xi)
\cfrac{\varphi(y,\frac{|\,\xi|}a)}{|\,\xi|}.$$

$y$ --- фиксирован. Сделаем замену $\cfrac ra=\rho$. Тогда
внутренний интеграл:

$$\int\limits_{\mathbb R^3}d\xi\,u_1(y-\xi)
\cfrac{\varphi(y,\frac{|\,\xi|}a)}{|\,\xi|}=
\int\limits_0^{\infty}dr \int\limits_{\xi\in S_r}dS_r u_1(y-\xi)
\cfrac{\varphi(y,\frac{r}a)}{r}=$$
$$=\int\limits_0^{\infty}\cfrac{d\rho}{\rho} \int\limits_{\xi\in S_a\rho}dS_r u_1(y-\xi)
\varphi(y,\rho).$$

Т.о.,
$$\Bigl(u_1(x)\delta(t)*E_3(x,t),\varphi(x,t)\Bigr)=
\cfrac1{4\pi a^2}\int\limits_{\mathbb
R^3}dy\int\limits_0^{\infty}\cfrac{d\rho}{\rho}\,\varphi(y,\rho)
\int\limits_{\xi\in S_a\rho}dS_r
u_1(\underbrace{y-\xi}_{=\Theta})=$$
$$=\cfrac1{4\pi a^2}\int\limits_{\mathbb
R^3}dy\int\limits_0^{\infty}\cfrac{d\rho}{\rho}\,\varphi(y,\rho)
\int\limits_{\Theta\in S^y_a\rho}dS_{\Theta}\,u_1(\Theta)=
\Bigl(\varphi(y,\rho),\cfrac1{4\pi
a^2}\int\limits_{S^y_a\rho}u_1(\Theta)\,d\Theta\Bigr).$$

Третье слагаемое получается аналогично второму, но с учетом того,
что
$$\Bigl(u_0(x)\delta'(t)\Bigr)*E_3(x,t)=
\cfrac{\partial}{\partial
t}\Bigl(u_0(x)\delta(t)*E_3(x,t)\Bigr).$$

\textbf{Упражнение.} Получить самостоятельно формулы Пуассона и
Даламбера.

\subsection{Гладкость правых частей формулы Кирхгофа}

Первое слагаемое:

$$\cfrac1{4\pi a^2}
\int\limits_{B_{at}^x}\cfrac{f(\xi,t-\Bigr|\frac{x-\xi}a\Bigl|}{|\,x-\xi|}\,d\xi=
\cfrac1{4\pi a^2}
\int\limits_{B_{at}}\cfrac{f(x-y,t-\frac{|\,y|}a)}{|\,y|}\,dy,\eqno(1)$$

где $B_{at}$ --- шар с центром в 0,
$|\,y|=at,\,y=tw,\,|\,w|=a,\,dy=t^3dw.$ Тогда продолжим равенства:

$$(1)=\int\limits_{B_{a}}\cfrac{f(x-at,t-\frac{t|\,w|}a)}{t|\,w|}\,t^3\,dw=
t^2\int\limits_{B_{a}}\cfrac{f(x-at,t-\frac{t|\,w|}a)}{|\,w|}\,t^2\,dw$$

--- принадлежит классу $C^2(\mathbb R^3\times[0,\infty))$ по
переменным $(x,t)$, т.к. $f\in C^2(\mathbb R^2\times[0,\infty)).$

Второе слагаемое:

$$\cfrac1{4\pi a^2} \int\limits_{S_{at}^x}u_1(\xi)\,
dS_{\varepsilon}=\Bigl|\mbox{замена }\xi=y+x\Bigr|= \cfrac1{4\pi
a^2} \int\limits_{S_{at}}u_1(y+x)\, dS_y,\eqno(2)$$

Если $y=t\Theta,\, |\,y|=at,\, |\,w|=a,\, dS_y=t^2dS_w,$ то
$$(2)=\cfrac{t^2}{4\pi
a^2} \int\limits_{S_{a}}u_1(tw+x)\, dS_w$$
--- принадлежит классу $C^2(\mathbb R^3\times[0,\infty))$ по
переменным $(x,t)$.

Аналогично, так же и третье слагаемое является класса $C^2(\mathbb
R^3\times[0,\infty))$ по переменным $(x,t)$.

\textbf{Упражнение.} Проделать аналогичные выкладки для формул
Пуассона и Даламбера.

\textbf{Утверждение.} Единственность решения классической задачи
Коши для волнового уравнения вытекает из теории единственности
решения обобщенной задачи Коши.

Остается ответить на вопрос, почему для функции $$u(x,t)=(\tilde
f(x,t)+u_0\,\delta'(t)+u_1\,\delta(t))*E_n(x,t),\,n=1,2,3$$ из
того, что $u\in C^2(\mathbb R^3\times(0,\infty))\cap C(\mathbb
R^3\times[0,\infty))$ следует из того, что $u(x,t)$ удовлетворяет
в классическом смысле начальным условиям Коши, т.е.
$u(x,0)=u_0(x),\,u_t(x,0)=u_1(x).$

$$\tilde u(x,t)=\begin{cases} u(x,t), t\ge0;\\ 0,
t<0,\end{cases}\tilde f(x,t)=\begin{cases} f(x,t), t\ge0;\\ 0,
t<0.\end{cases}$$

Получим: $$\square_a\tilde u(x,t)= \tilde
f(x,t)+u(x,0)\,\delta'(t)+u_t(x,0)\,\delta(t),$$ с другой стороны:
$$\square_a\tilde u(x,t)=\tilde
f(x,t)+u_0(x)\,\delta'(t)+u_1(x)\,\delta(t).$$

Значит, $u(x,0)=u_0(x)$ и $u_t(x,0)=u_1(x),$ т.к.
$u(x,0),u_0(x),u_t(x,0),u_1(x)$ --- непрерывные функции.

\textbf{Упражнение.} Доказать строго последнее утверждение.

\section{Задача Коши для уравнения теплопроводности}

$${\begin{aligned}
\left\{%
\begin{array}{ll}
    u_{t}=a^2\triangle u+f(x,t),\,x\in\mathbb R^n\\
    u(x,0)=u_0(x). \\
\end{array}%
\right.
\end{aligned}}\eqno(*)$$

\textbf{Определение.} \textbf{Классическое решение задачи $(*)$}
--- решение $u\in C(\mathbb R^n\times[0,\infty)),$
$\partial_x^{\,\alpha}\in C(\mathbb R^n\times[0,\infty)),\,
|\,\alpha|\le2,\,\alpha=(\alpha_1,\ldots,\alpha_n),
|\alpha|=\alpha_1+\ldots+\alpha_n, \,\partial^{\,\alpha}=\cfrac
{\partial^{\,|\alpha|}}{\partial x_1^{\,\alpha_1}\ldots \partial
x_1^{\,\alpha_n}}$

\textbf{Определение.} Через \textbf{M} обозначим множество
измеримых функций на
$$\mathbb R^n\times[0,\infty)\colon |\,u(x,t)|\le C \mbox{ для почти всех }
(x,t)\in \mathbb R^n\times[0,T].$$

Пусть $u$ --- классическое решение задачи Коши $(*)$,
$$\tilde u(x,t)=\begin{cases} u(x,t), t\ge0;\\ 0,
t<0,\end{cases}\tilde f(x,t)=\begin{cases} f(x,t), t\ge0;\\ 0,
t<0.\end{cases}$$

Тогда $\tilde u_{t}-a^2\triangle \tilde u=\tilde
f(x,t)+u_0(x)\,\delta(t).$

\textbf{Упражнение.} Доказать последнее равенство строго
(аналогично задаче Коши для волнового оператора).

\textbf{Определение.} \textbf{Обобщенное решение задачи Коши для
уравнения теплопроводности:} пусть $F\in D'(\mathbb R^{n+1}),
\supp F\subset \mathbb R^n\times[0,\infty).$ Требуется найти $u\in
D'(\mathbb R^{n+1}),\supp u \subset \mathbb
R^n\times[0,\infty)\colon$
$$u_{t}+a^2\triangle u=F(x,t).\eqno(**)$$

\textbf{Теорема.} Пусть $f\in M,u_1\in L_{\infty}(\mathbb
R^n),F=\tilde f(x,t)+u_0(x)\,\delta(t).$ Тогда обобщенная задача
Коши $(**)$ имеет единственное решение из класса $M$:
$$u(x,t)=
\Theta(t)\int\limits_0^t \int\limits_{\mathbb R^n}
\cfrac{f(\xi,\tau)\,e^{-\frac{|\,x-\xi|^2}{4a^2(t-\tau)}}}
{(2a\sqrt{\pi(t-\tau)})^n}\,d\xi\,dt+
\cfrac{\Theta(t)}{(2a\sqrt{\pi t})^n} \int\limits_{\mathbb
R^n}u_0(\xi)\,e^{-\frac{|\,x-\xi|^2}{4a^2\tau}}.$$

$\square$

Возьмем фундаментальное решение оператора теплопроводности:
$$E(x,t)=\cfrac{\Theta(t)}{(2a\sqrt{\pi t})^n}\,e^{-\frac{|\,x-\xi|^2}{4a^2t}}.$$
Докажем, что свертка $F*E$ существует:

$$\tilde f*E(X,t)=
\int\limits_{\mathbb R^{n+1}}\tilde
f(\xi,\tau)\,E(x-\tau)\,d\xi\,d\tau= \int\limits_{\mathbb
R^{n+1}}\tilde f(\xi,\tau)\cfrac{\Theta(t-\tau)}{(2a\sqrt{\pi
(t-\tau)})^n}\,e^{-\frac{|\,x-\xi|^2}{4a^2(t-\tau)}}\,d\xi\,d\tau=$$
$$=\Theta(t)\int\limits_0^t \int\limits_{\mathbb R^n}
\cfrac{f(\xi,\tau)\,e^{-\frac{|\,x-\xi|^2}{4a^2(t-\tau)}}}
{(2a\sqrt{\pi(t-\tau)})^n}\,d\xi\,dt.$$

При этом, $\forall x\in \mathbb R^n, t\in (0,T), T>0:$
$$|u(x,t)|\le
\int\limits_0^t \int\limits_{\mathbb R^n}
\cfrac{|f(\xi,\tau)|\,e^{-\frac{|\,x-\xi|^2}{4a^2(t-\tau)}}}
{(2a\sqrt{\pi(t-\tau)})^n}\,d\xi\,dt\le \|f\|_{L_{\infty}(\mathbb
R^n\times[0,\infty))} \int\limits_0^t d\tau \int\limits_{\mathbb
R^n}\cfrac{e^{-\frac{|\,x-\xi|^2}{4a^2(t-\tau)}}}
{(2a\sqrt{\pi(t-\tau)})^n}\,d\xi=$$
$$\Bigl|\mbox{замена }y=\cfrac{|\,x-\xi|}{2a\sqrt{t-\tau}}\Bigr|=
\|f\|_{L_{\infty}(\mathbb R^n\times[0,\infty))}\int\limits_0^t
d\tau \underbrace{\int\limits_{\mathbb R^n}\cfrac{e^{-|\,y|^2}}
{\pi^{\frac n2}}\,dy}_{=1}= T\cdot\|f\|_{L_{\infty}(\mathbb
R^n\times[0,\infty))}.$$

$$\Bigl(u_0(x)\delta(t)\Bigr)*E(x,t)=
\Bigl(u_0(x)\delta(t)E(x,t),\eta_k(x,t,\xi,\tau)\underbrace{\varphi(x+\xi,t+\tau)}_{\in
D(\mathbb R^{n+1})}\Bigr)=$$
$$=\Bigl(u_0(x)\delta(t),(E(\xi,\tau),\eta_k(x,t,\xi,\tau)\varphi(x+\xi,t+\tau))\Bigr)=$$
$$\Bigl(u_0(x)\delta(t),\int\limits_{\mathbb
R^{n+1}}E(\xi,\tau)\eta_k(x,t,\xi,\tau)\varphi(x+\xi,t+\tau)\,d\xi
\,d\tau\Bigr)=$$
$$=\Bigl(u_0(x),(\delta(t),\int\limits_{\mathbb
R^{n+1}}E(\xi,\tau)\eta_k(x,t,\xi,\tau)\varphi(x+\xi,t+\tau)\,d\xi\,d\tau)\Bigr)=$$
$$\Bigl(u_0(x),\int\limits_{\mathbb
R^{n+1}}E(\xi,\tau)\eta_k(x,0,\xi,\tau)\varphi(x+\xi,\tau)d\xi
d\tau\Bigr)=$$
$$=\int\limits_{\mathbb R^{n}}dx\,u_0(x)\int\limits_{\mathbb
R^{n+1}}d\xi\,d\tau
E(\xi,\tau)\eta_k(x,0,\xi,\tau)\varphi(x+\xi,\tau)=
\Bigl|\mbox{замена }y=x+\xi \Bigr|=$$
$$=\int\limits_{\mathbb R^{n}}dx\,u_0(x)\int\limits_{\mathbb
R^{n+1}}dy\,d\tau
E(y-x,\tau)\eta_k(x,0,y-x,\tau)\varphi(y,\tau)=$$
$$=\Bigl(\int\limits_{\mathbb R^{n}}E(y-x,\tau)u_0(x)\,dx,\varphi(y,\tau)\Bigr)=
\Bigl(\int\limits_{\mathbb R^{n}} \cfrac{\Theta(t)}{(2a\sqrt{\pi
t})^n}\,e^{-\frac{|\,y-x|^2}{4a^2t}}\,dx,\varphi(y,\tau)\Bigr).$$

Оценка для второго слагаемого в формуле Пуассона:

$$\left|\cfrac{\Theta(t)}{(2a\sqrt{\pi
t})^n}\,e^{-\frac{|\,x-\xi|^2}{4a^2t}}\int\limits_{\mathbb
R^{n}}e^{-\frac{|\,x-\xi|^2}{4a^2t}}\,d\xi\right|\le
\cfrac{\Theta(t)}{(2a\sqrt{\pi t})^n}\,\|u_0\|_{L_{\infty}(\mathbb
R^n\times[0,\infty))} \int\limits_{\mathbb
R^{n}}e^{-\frac{|\,x-\xi|^2}{4a^2t}}\,d\xi=$$
$$=\Bigl|y=\cfrac{x-\xi}{2a\sqrt t} \Bigr|=
\|u_0\|_{L_{\infty}(\mathbb R^n\times[0,\infty))}
\underbrace{\cfrac1{\pi^{\frac n2}} \int\limits_{\mathbb R^{n}}
e^{-|\,y|^2}\,dy}_{=1}.$$

Т.к. у любой функции $u\in M\colon \supp u\subset
R^n\times[0,\infty),$ существует свертка с $E(x,t)$, то решение
задачи Коши будет единственным в классе функций $u\in M\colon
\supp u\subset R^n\times[0,\infty).$

%lect25

$u\in C^{2,1}(\mathbb R^n\times(0,\infty))\cap C(\mathbb
R^n\times[0,\infty))$ --- класс гладкости $C^2$ по $x$ и $C^1$ по
$t$.

\textbf{Теорема.} (существование и единственность классического
решения.) Пусть
$$u_0\in C^{2}(\mathbb R^n)\cap L_{\infty}(\mathbb
R^n),\,f\in C^{2}(\mathbb R^n\times(0,\infty))\cap M,\,\partial_x
f,\,\partial_x^2f\in M.$$
Тогда задача Коши для уравнения
теплопроводности:
$$\left\{%
\begin{array}{ll}
    u_{t}=a^2\triangle u+f(x,t),\,x\in\mathbb R^n\\
    u(x,0)=u_0(x), \\
\end{array}%
\right.$$ имеет единственное решение из класса $C^{2,1}(\mathbb
R^n\times(0,\infty))\cap C(\mathbb R^n\times[0,\infty))$.

$\square$

Из теоремы единственности обобщенного решения задачи Коши для
уравнения теплопроводности получим:
$$u(x,t)=
\Theta(t)\int\limits_0^t \int\limits_{\mathbb R^n}
\cfrac{f(\xi,\tau)\,e^{-\frac{|\,x-\xi|^2}{4a^2(t-\tau)}}}
{(2a\sqrt{\pi(t-\tau)})^n}\,d\xi\,dt+
\cfrac{\Theta(t)}{(2a\sqrt{\pi t})^n} \int\limits_{\mathbb
R^n}u_0(\xi)\,e^{-\frac{|\,x-\xi|^2}{4a^2\tau}}=I_1+I_2$$
--- обобщенное решение из класса $M$. Надо показать, что $u\in C^{2,1}(\mathbb R^n\times(0,\infty))\cap C(\mathbb
R^n\times[0,\infty))$. Сделаем замену
$\zeta=\cfrac{|\,x-\xi|}{2a\sqrt{t-\tau}},$ тогда
$$I_1=\cfrac1{\pi^{\frac n2}}\int\limits_0^t d\tau \int\limits_{\mathbb R^n}
f(x-2a\sqrt{t-\tau}\zeta,\tau)\,e^{-|\,\zeta|^2}d\zeta.$$

Аналогично, заменяя $\zeta=\cfrac{|\,x-\xi|}{2a\sqrt{t}},$
получим: $I_2=\cfrac1{\pi^{\frac n2}} \int\limits_{\mathbb R^n}
u_0(x-2a\sqrt{t}\zeta,\tau)d\zeta.\,\blacksquare$

\section{Пространство С.\,Л.\,Соболева}

$\Omega\subset\mathbb R^{n}$ --- открытое подмножество,
$\Omega\ne0.$

\textbf{Определение.} $W_p^m(\Omega)=\{u\in D'(\Omega)\colon u\in
L_p(\Omega),\,\partial^{\,\alpha} u \in L_p (\Omega),|\,\alpha|=m,
p>1,m\in\mathbb N_0\}$
--- \textbf{пространство Соболева.}

\textbf{Определение.} \textbf{Норма в пространстве Соболева}:
$$\|u\|_{W_p^m(\Omega)}=\|u\|_{L_p(\Omega)}+\sum_{|\,\alpha|=m}
\|\partial^{\,\alpha}u\|_{L_p(\Omega)}.$$

Более подробно:
$$\|u\|_{W_p^m(\Omega)}=\left(\int\limits_{\Omega}|\,u|^p dx\right)^{\frac1p}
+\sum_{|\,\alpha|=m}\left(\int\limits_{\Omega}|\,\partial^{\,\alpha}u|^p
dx\right)^{\frac1p}.$$

Другие, эквивалентные нормы на $W_p^m(\Omega)$:
$$\|u\|_{W_p^m(\Omega)}=\left(\int\limits_{\Omega}|\,u|^p dx
+\sum_{|\,\alpha|=m}\int\limits_{\Omega}|\,\partial^{\,\alpha}u|^p
dx\right)^{\frac1p}.$$

\textbf{Замечание.} Нормы $\|\cdot\|_1$ и $\|\cdot\|_2$ на
линейном пространстве $V$ \textbf{эквивалентны}, если

$\exists \,0<c_1<c_1\colon \forall v\in V\,\,\,
c_1\|v\|_1\le\|v\|_2\le c_2\|v\|_1.$

\textbf{Теорема.} Пространство $W_p^m(\Omega)$ полно.

$\square$

Пусть $u_k\in W_p^m(\Omega)$ --- фундаментальная
последовательность $(k=1,2,\ldots)$. Докажем, что существует $u\in
W_p^m(\Omega)\colon \|u-u_k\|_{W_p^m(\Omega)}\to0,\,k\to{\infty}.$

Тогда $u_k\in W_p^m(\Omega)$ является фундаментальной в
$L_p(\Omega)$, т.к. $\forall v\in W_p^m(\Omega)\,\,
\|v\|_{L_p(\Omega)}\le \|v\|_{W_p^m(\Omega)}.$

Аналогично, $\forall \alpha\colon |\,\alpha|=m$ последовательность
$\partial^{\,\alpha}u_k,\,k=1,2,\ldots,$ также является
фундаментальной в $L_p(\Omega)$. В самом деле, $\forall v\in
W_p^m(\Omega)\,\, \|\,\partial^{\,\alpha}v_k\|_{L_p(\Omega)}\le
\|v\|_{W_p^m(\Omega)},\,\, |\,\alpha|=m.$ В силу полноты
пространства $L_p(\Omega)$, существуют функции $u\in
L_p(\Omega),\,u_{\alpha}\in L_p(\Omega) |\,\alpha|=m\colon$

$ \|u-u_k\|_{L_p(\Omega)}\to0,\,k\to{\infty}.$

Покажем, что $\partial^{\,\alpha}u=u_{\alpha}.$ Другими словами,
надо показать, что
$$(\partial^{\,\alpha}u,\varphi)=(-1)^{\alpha}\int\limits_{\Omega}u\,\partial^{\,\alpha}\varphi\,dx=
\int\limits_{\Omega}u_{\alpha}\varphi\,dx= (u_{\alpha},\varphi).$$

Имеем:
$$(-1)^{|\,\alpha|}\int\limits_{\Omega}u_k\,\partial^{\,\alpha}\varphi\,dx=
\int\limits_{\Omega}\partial^{\,\alpha}u_k\,\varphi\,dx,$$ при
этом
$$\int\limits_{\Omega}u_k\,\partial^{\,\alpha}\varphi\,dx\to
\int\limits_{\Omega}u\,\partial^{\,\alpha}\varphi\,dx\to0,\,k\to{\infty},\eqno(*)$$
$$\int\limits_{\Omega}\partial^{\,\alpha}u_k\,\varphi\,dx\to
\int\limits_{\Omega}u_{\alpha}\varphi\,dx\to0,\,k\to{\infty},\eqno(**)$$

Докажем $(*)$. По неравенству Гельдера, получим:
$$\left|\int\limits_{\Omega}u_k\,\partial^{\,\alpha}\varphi\,dx-
\int\limits_{\Omega}u\,\partial^{\,\alpha}\varphi\,dx\right|\le
\left|\int\limits_{\Omega}|\,u_k-u|\,|\,\partial^{\,\alpha}\varphi|\,dx\right|\le$$
$$\le \underbrace{\left(\int\limits_{\Omega}|\,u_k-u|^p
dx\right)^{\frac1p}}_{\to0,\,\,\,k\to{\infty}}\cdot
\underbrace{\left(\int\limits_{\Omega}|\,\partial^{\,\alpha}\varphi|^{\frac
p{p-1}} dx\right)^{\frac {p-1}p}}_{\mbox{не зависит от
k}}\to0,\,\,k\to{\infty}$$ Аналогично доказывается $(**)$.

Переходя к пределу, получим:
$$(-1)^{\alpha}\int\limits_{\Omega}u\,\partial^{\,\alpha}\varphi\,dx=
\int\limits_{\Omega}u_{\alpha}\varphi\,dx,$$ т.е.
$\partial^{\,\alpha}u=u_{\alpha}.$ Т.о., $u_k\to u,\,k\to{\infty}$
в пространстве $W_p^m(\Omega).\,\blacksquare$

\textbf{Теорема.} Пространство $W_p^m(\Omega)$ сепарабельно.

\textbf{Определение.} Пространство называется
\textbf{сепарабельным}, если существует счетное всюду плотное
множество.

$\square$

Рассмотрим вложение $\pi\colon u\mapsto
(u_1,\ldots,\partial^{\,\alpha}u_1,\ldots), |\,\alpha|=m,$
пространства $W_p^m(\Omega)$ в пространство
$V=\underbrace{L_p(\Omega)\times\ldots\times L_p(\Omega)}_{N},$
где в $V=L_p(\Omega)\times\ldots\times L_p(\Omega)$ норма
определяется равенством
$$\|\,v\|_V=\|\,v_1\|_{L_p(\Omega)}+\ldots|+\|\,v_N\|_{L_p(\Omega)},\,
v=(v_1,\ldots\,v_N)\in V.$$

Отображение $\pi\colon W_p^m(\Omega)\to V$ является изометрией,
т.\,е.
$$\forall u\in W_p^m(\Omega)\,\,\|\,\pi(u)\|_V=\|u\|_{W_p^m(\Omega)}.$$

Пространство $V=\underbrace{L_p(\Omega)\times\ldots\times
L_p(\Omega)}_{N}$ сепарабельно, т.к. $L_p(\Omega)$ ---
сепарабельно. А замкнутое подпространство сепарабельного
пространства тоже сепарабельно. Что равносильно сепарабельности
$W_p^m(\Omega).\blacksquare$

%lect26

Рассмотрим первую краевую задачу Дирихле с однородным краевым
условием:
$${\begin{aligned}
\left\{%
\begin{array}{ll}
    \triangle u=f(x),\\
    \Bigl.u\Bigr|_{\partial \Omega}=0. \\
\end{array}%
\right.
\end{aligned}}\eqno(*)$$

$\Omega\subset \mathbb R^n$ --- ограниченная область,
$f_0,f_1,\ldots,f_n\in L_2(\Omega).$

\textbf{Определение.} \textbf{Обобщенное (по Соболеву)
решение задачи} $(*)$ $u\in \overset{\circ}{W_2^1}(\Omega)\colon$
$$-\int\limits_{\Omega}\underbrace{\sum\limits_{i=1}^n
\cfrac{\partial u}{\partial x_i} \, \cfrac{\partial
\varphi}{\partial x_i}}_{=\triangledown
u\,\triangledown\varphi}\,dx=
\int\limits_{\Omega}f_0(x)\,\varphi\,dx-
\int\limits_{\Omega}\sum\limits_{i=1}^n f_i(x) \, \cfrac{\partial
\varphi}{\partial x_i}\,dx,\,\,\,\forall \varphi \in D(\Omega).$$

\textbf{Пример.} Пусть $u\in C^{\infty}(\overline{\Omega}),$
$\Omega\subset \mathbb R^n$ --- ограниченная область с бесконечно
гладкой границей, $f\in C^{\infty}(\overline{\Omega})$ и при этом:
$${\begin{aligned}
\left\{%
\begin{array}{ll}
    \triangle u=f_0(x)+\sum\limits_{i=1}^n \cfrac{\partial f_i(x)}{\partial x_i},\\
    \Bigl.u\Bigr|_{\partial \Omega}=0. \\
\end{array}%
\right.
\end{aligned}}\eqno(**)$$

Тогда $\forall \varphi \in D(\Omega)$ будем иметь
$$\int\limits_{\Omega}\triangle u\,\varphi\,dx=
\int\limits_{\Omega}f(x)\,\varphi\,dx,$$
$$-\int\limits_{\Omega}\triangledown
u\,\triangledown\varphi\,dx=
\int\limits_{\Omega}f(x)\,\varphi\,dx,$$ т.е. и является
обобщенным (в смысле Соболева) решением $(**)$.

\textbf{Теорема.} Обобщенное решение задачи Дирихле $(*)$
существует и единственно.

$\square$

\textbf{Лемма.} \textbf{(Неравенство Фридрихса).} Пусть
$\Omega\subset \mathbb R^n$ --- ограниченная область. Тогда
$$\forall u\in \overset{\circ}{W_2^1}(\Omega)\,\exists
C>0\colon\,\,\,\int\limits_{\Omega}|\,u|^2dx\le C
\int\limits_{\Omega}|\,\triangledown u|^2 dx.$$

$\square$

Без ограничения общности, можно считать, что $u\in D(\Omega)$.
Т.к. $\Omega$ --- ограниченная область, то существуют $a<b$
 такие, что $\Omega$ лежит в полосе $a<x_n<b$ (т.е. $\forall
 x=(x_1,\ldots,x_n)\in\Omega$ справедливо неравенство $a<x_n<b$).

$x'=(x_1,\ldots,x_{n-1}).$ По теореме Ньютона-Лейбница,
$$u^2(x',x_n)=u^2(x',a)+\int\limits_a^{x_n}\cfrac{\partial}{\partial t}u^2(x',t)\,dt.$$

При этом мы подразумеваем, что функция продолжается нулем на
множество $\mathbb R^n\setminus\Omega.$ Т.о.,
$u(x',a)=0\Rightarrow$

$$u^2(x',x_n)=\int\limits_a^{x_n}\cfrac{\partial}{\partial t}u^2(x',t)\,dt=
2\int\limits_a^{x_n}u(x',t)\,\cfrac{\partial u(x',t)}{\partial
t}\,dt,$$

$$u^2(x',x_n)\le \int\limits_a^{b}|u(x',t)|\,\cdot\,\left|\cfrac{\partial u(x',t)}{\partial
t}\right|\,dt,\,\forall (x',x_n)\in\Omega,$$

$$u^2(x',x_n)\le
\varepsilon
\int\limits_{\Omega'}\int\limits_a^{b}|\,u(x',t)|^2\,dt+
\cfrac1{\varepsilon} \int\limits_{\Omega'}\left|\cfrac{\partial
u(x',t)}{\partial t}\right|^2\,dt,\,\varepsilon>0.$$

Т.к. $\alpha\beta\le \cfrac12\,(\alpha^2+\beta^2)\le
\alpha^2+\beta^2,$ то, учитывая, что
$\alpha=\varepsilon^{\frac12}|\,u(x',t)|,\,
\beta=\varepsilon^{-\frac12}\left|\cfrac{\partial
u(x',t)}{\partial t}\right|$, получим:

$$\int\limits_{\Omega'}u^2(x',x_n)\,dx'\le
\varepsilon
\int\limits_{\Omega'}\int\limits_a^{b}\,u^2(x',t)\,dx'\,dt+
\cfrac1{\varepsilon}
\int\limits_{\Omega'}\int\limits_a^{b}\left|\cfrac{\partial
u(x',t)}{\partial t}\right|^2\,dx'\,dt,$$ где $\Omega'$ ---
проекция области $\Omega$ на гиперповерхность
$x_n=0,\,dx'=dx_1\ldots dx_{n-1}.$

Интегрируя последнее неравенство по $x_n\in(a,b),$ получим:
$$\int\limits_a^{b}\int\limits_{\Omega'}u^2(x',x_n)\,dx'\,dx_n\le
\varepsilon\,(b-a)\int\limits_{\Omega'}\int\limits_a^{b}u^2(x',t)\,dx'\,dt+
\cfrac1{\varepsilon}\,(b-a)
\int\limits_{\Omega'}\int\limits_a^{b}\left|\cfrac{\partial
u(x',t)}{\partial t}\right|^2\,dx'\,dt.$$

Другими словами,
$$\int\limits_{\Omega}u^2\,dx\le
\varepsilon\,(b-a)\int\limits_{\Omega}u^2\,dx+
\cfrac1{\varepsilon}\,(b-a) \int\limits_{\Omega}|\,\triangledown
u|^2\,dx.$$

Возьмем $\varepsilon=\cfrac1{2(b-a)},$ получим:
$$\int\limits_{\Omega}u^2\,dx\le
\cfrac12\int\limits_{\Omega}u^2\,dx+2(b-a)^2\int\limits_{\Omega}|\,\triangledown
u|^2\,dx \Leftrightarrow \int\limits_{\Omega}u^2\,dx\le
4(b-a)^2\int\limits_{\Omega}|\,\triangledown
u|^2\,dx.\,\blacksquare$$

$\square$(теоремы)

$\Omega$ --- ограниченная область, поэтому билинейная форма
$$[u,v]=\int\limits_{\Omega}\triangledown u\,\triangledown
v\,dx,\,u,v\in \overset{\circ}{W_2^1}(\Omega) \eqno(***)$$
является скалярным произведением в
$\overset{\circ}{W_2^1}(\Omega),$ т.к., по неравенству Фридрихса,
$$\|u\|_{W_p^m(\Omega)}^2\le C\|\triangledown
u\|^2_{L_2(\Omega)}=C[u,v].$$

Более того, скалярное произведение $(***)$ порождает норму,
эквивалентную Соболевской норме в
$\overset{\circ}{W_2^1}(\Omega)$. Рассмотрим в
$\overset{\circ}{W_2^1}(\Omega)$ следующие линейные функционалы:
$$l_0=-\int\limits_{\Omega}f\,u\,dx,\,
l_i=\int\limits_{\Omega}f_i\,\cfrac{\partial u}{\partial
x_i}\,dx.$$ Эти функционалы, очевидно, будут непрерывны на
$\overset{\circ}{W_2^1}(\Omega).$ В самом деле,
$$|\,l_0(u)|\le \int\limits_{\Omega}|\,f|\,|\,u|\,dx\le
\Bigl(\int\limits_{\Omega}f^2\,dx\Bigr)^{\frac12}
\Bigl(\int\limits_{\Omega}u^2\,dx\Bigr)^{\frac12}\le \const \,
\|u\|_{W_p^m(\Omega)}^2,$$
$$|\,l_i|\le
\int\limits_{\Omega}|\,f_i|\,\Bigl|\cfrac{\partial u}{\partial
x_i}\,\Bigr|\,dx\le
\Bigl(\int\limits_{\Omega}f_i^2\,dx\Bigr)^{\frac12}\,
\Bigl(\int\limits_{\Omega}|\,\triangledown u|^2dx\Bigr)^{\frac12}
\le \const \, \|u\|_{W_p^m(\Omega)}^2.$$

По теореме Рисса, существует единственное $u\in
\overset{\circ}{W_2^1}(\Omega)\colon $
 $$\forall \, v\in
\overset{\circ}{W_2^1}(\Omega)\,\,\,\,\, [u,v]=l_0(v)+\sum_{i=1}^n
l_i(v).$$

Другими словами,
$$\int\limits_{\Omega}\triangledown u\,\triangledown
v\,dx=-\int\limits_{\Omega}f_0\, v\,dx+
\sum_{i=1}^n\int\limits_{\Omega}f_i\,\cfrac{\partial v}{\partial
x_i}\,dx.\,\blacksquare$$

Неоднородная задача Дирихле в области $\Omega\subset \mathbb R^n$:
$${\begin{aligned}
\left\{%
\begin{array}{ll}
    \triangle u=f_0(x)+\sum\limits_{i=1}^n \cfrac{\partial f_i(x)}{\partial x_i},\\
    \Bigl.u\Bigr|_{\partial \Omega}=u_0(x),\,u_0\in\overset{\circ}{W_2^1}(\Omega). \\
\end{array}%
\right.
\end{aligned}}\eqno(****)$$

\textbf{Теорема.} $\Bigl.u\Bigr|_{\partial \Omega}=u_0(x)$
\textbf{в обобщенном смысле}, если
$u-u_0\in\overset{\circ}{W_2^1}(\Omega)$.

\textbf{Теорема.} Обобщенная неоднородная задача Дирихле $(****)$
имеет единственное решение $u\in\overset{\circ}{W_2^1}(\Omega)$.

$\square$

Обозначим $v=u-u_0.$ Тогда $\forall \varphi\in D(\Omega)$ получим:
$$\int\limits_{\Omega}\triangledown v\,\triangledown
\varphi\,dx= \int\limits_{\Omega}\triangledown
u\,\triangledown\varphi\,dx- \int\limits_{\Omega}\triangledown
u_0\,\triangledown \varphi\,dx.$$

По определению обобщенного решения $(****)$:
$$-\int\limits_{\Omega}\triangledown u\,\triangledown
v\,dx=\int\limits_{\Omega}f_0\, v\,dx-
\sum_{i=1}^n\int\limits_{\Omega}f_i\,\cfrac{\partial v}{\partial
x_i}\,dx,$$

$$-\int\limits_{\Omega}\triangledown v\,\triangledown
\varphi\,dx=\int\limits_{\Omega}f_0\, v\,dx-
\sum_{i=1}^n\int\limits_{\Omega}f_i\,\cfrac{\partial v}{\partial
x_i}\,dx+\sum_{i=1}^n\int\limits_{\Omega}f_i\,\cfrac{\partial
u_0}{\partial x_i}\,\cfrac{\partial \varphi}{\partial x_i}\,dx$$

Т.о., $v$ является решением однородной обобщенной задачи Дирихле:
$$\left\{%
\begin{array}{ll}
    \triangle u=f_0(x)+\sum\limits_{i=1}^n \cfrac{\partial g_i(x)}{\partial x_i},\\
    \Bigl.u\Bigr|_{\partial \Omega}=0, \\
\end{array}%
\right.$$ где $g_i=f_i-\cfrac{\partial u_0}{\partial x_i}\in
L_2(\Omega),\,i=\overline{1,n}.$ По предположению, эта задача
имеет единственное решение. $\blacksquare$

%lect27

\section{Гармонические функции}

\textbf{Определение.} Функция $f\in D'(\Omega),$ где $\Omega \in
\mathbb R^n$ --- область, называется \textbf{гармонической} в
области $\Omega$, если $\triangle u=0$ в $\Omega.$

\textbf{Теорема.} Пусть $u\in D'(\Omega)$\т гармоническая
функция в области $\Omega$, тогда $u\in C^{\infty}(\Omega)$.

$\square$

Пусть $x_0\in \Omega.$ Возьмем $\varepsilon>0$ настолько малым,
чтобы $B_{2\varepsilon}^{x_0}\subset \in \Omega$, где
$B_{2\varepsilon}^{x_0}$ --- шар радиуса $\varepsilon$ с центром в
точке $x_0$. Пусть также $\eta\in D(B_{2\varepsilon}^{x_0}),$
причем $\Bigl.\eta\Bigr|_{B_{2\varepsilon}^{x_0}}\equiv1.$
$\triangle u=\sum\limits_{i=1}^n\cfrac{\partial^2}{\partial
x_i^2}.$

Имеем:

$$\triangle(\eta\,u)=\triangle\eta\,u+2\triangle\eta\,\triangle
u+\eta\,\underbrace{\triangle u}_{=0},\,\, \eta\,u\in D'(\mathbb
R^n),\mbox{ т.к. }\supp \eta\,u\Subset B_{2\varepsilon}^{x_0},$$
$$\triangle(\eta\,u)=\triangle\eta\,u+2\triangle\eta\,\triangle
u \Rightarrow \mbox{ по теореме единственности, }
\eta\,u=E*(\triangle\eta\,u+2\triangle\eta\,\triangle u),$$
$$\mbox{где }
E(x)=\left\{%
\begin{array}{ll}
    \cfrac1{2\pi}\,\ln|\,x|,\,n=2,      \\
    -\cfrac1{(n-2)|S_1|}\,\cfrac1{|\,x|^{n-2}},\,n\ge3, \\
\end{array}%
\right.$$ Обозначим
$f(x)=\triangle\eta\,u+2\triangle\eta\,\triangle u\in
D'(B_{2\varepsilon}^{x_0}),\, \supp f\Subset
B_{2\varepsilon}^{x_0}.$ Тогда $\forall \varphi\in D'(\mathbb
R^n)$
$$\Bigl(E*f(x),\varphi(x)\Bigr)=
\Bigl(f(x),(E(y),\varphi(x+y))\Bigr)=
\Bigl(f(x),\int\limits_{\mathbb
R^n}E(y)\,\varphi(x+y)\,dy\Bigr),$$

$$\int\limits_{\mathbb
R^n}E(y)\,\varphi(\underbrace{x+y}_{=\xi})\,dy=
\int\limits_{\mathbb R^n}E(\xi-x)\,\varphi(\xi)\,d\xi,$$

$$\Bigl(E*f(x),\varphi(x)\Bigr)=
\Bigl(f(x),\int\limits_{\mathbb
R^n}E(\xi-x)\,\varphi(\xi)\,d\xi\Bigr)=
\Bigl(f(x)\,\lambda(x),\int\limits_{\mathbb
R^n}E(\xi-x)\,\varphi(\xi)\,d\xi\Bigr),$$ где $\lambda\in
D(B_{2\varepsilon}^{x_0}\setminus
\overline{B_{\varepsilon/2}^{x_0}}),\, \lambda\equiv1$ на $\supp
f$.

$$\Bigl(E*f(x),\varphi(x)\Bigr)=
\Bigl(f(x),\int\limits_{\mathbb
R^n}\underbrace{\lambda(x)\,E(\xi-x)\,\varphi(\xi)}_{\in\,
C^{\infty}(\overline{B_{2\varepsilon}^{x_0}}\times
\overline{B_{\varepsilon/2}^{x_0}})}\,d\xi\Bigr)=
\int\limits_{\mathbb
R^n}\Bigl(f(x),\lambda(x)\,E(\xi-x)\,\underbrace{\Theta(\xi)\,\varphi(\xi)}_{\equiv
\varphi(\xi)}\Bigr)\,d\xi,$$ где $\Theta\in
D(B_{\varepsilon/2}^{x_0})\colon
\Bigl.\Theta\Bigr|_{B_{\varepsilon/2}^{x_0}}\equiv1.$ Т.о.,

$$\Bigl(E*f(x),\varphi(x)\Bigr)=
\int\limits_{\mathbb
R^n}\underbrace{\Bigl(f(x),\lambda(x)\,E(\xi-x)\,\Theta(\xi))\Bigr)}_{\in\,
C^{\infty}(B_{2\varepsilon}^{x_0}\times
B_{\varepsilon/2}^{x_0})}\,\varphi(\xi)\,d\xi,$$

$$k(x,\xi)=\lambda(x)\,E(\xi-x)\,\Theta(\xi)\in D(B_{2\varepsilon}^{x_0}\times
B_{\varepsilon/2}^{x_0}),$$
$$\Bigl(f(x),k(x,\cdot)\Bigr) \in
D(B_{\varepsilon/2}^{x_0}).$$

$$\forall \varphi\in D'(B_{\varepsilon/4}^{x_0})\,\,\,\,
\Bigl((\eta\,u),\varphi\Bigr)= \Bigl(E*f,\varphi\Bigr)=
\int\limits_{\mathbb R^n}
\underbrace{\Bigl(f(x),k(x,\xi)\Bigr)}_{\mbox{беск. гладкая по
$\xi$}}\,\varphi(\xi)\,d\xi,$$

$u=\eta\,u$ --- бесконечно гладкая функция в шаре
$B_{\varepsilon/4}^{x_0}$. Осталось доказать лемму.

\textbf{Лемма.} Пусть $L\in D(G\times H),$ где $G,H$ --- области в
$\mathbb R^n,$ $f\in D'(G).$ Тогда
$$\Bigl(f(x),\int\limits_H L(x,\xi)\,d\xi \Bigr)=
\int\limits_H\Bigl(f(x), L(x,\xi)\Bigr)\,d\xi.$$

$\square$ Без ограничения общности, можно считать, что $H$ --- это куб в
$\mathbb R^n$. D простом случае возьмем вместо $H$ достаточно
большой куб, содержащий это множество. $\xi_i\in\Delta_i$ ---
элементарная ячейка.
$$\sum_{i=1}^N L(x,\xi_i)\,|\,\Delta_i| \to \int\limits_H
L(x,\xi)\,d\xi,$$ когда диаметр разбиения стремится к $0$. В виду
линейности обобщенной функции $f(x)$, получим:
$$\underbrace{\Bigl(f(x),\int\limits_H \sum_{i=1}^N L(x,\xi_i)\,|\,\Delta_i|\,d\xi \Bigr)}_
{\to \Bigl(f(x),\int\limits_H L(x,\xi)\,d\xi \Bigr)\mbox { в
$D(G)$}}= \underbrace{\sum_{i=1}^N \Bigl(f(x),\int\limits_H
L(x,\xi_i)\,|\,\Delta_i|\,d\xi \Bigr)}_{\to \int\limits_H
\Bigl(f(x), L(x,\xi)\Bigr)\,d\xi}$$ Второе верно, т.к.
$(f(x),L(x,\cdot))\in D(H)$. Первое надо доказать. Имеем:

1) $\supp \Bigl(\,\sum\limits_{i=1}^N
L(\cdot,\xi_i)\Bigr)\,|\,\Delta_i|$
--- проекция $\supp L(\cdot,\cdot)$ на $G.$ Т.к. $\supp
L(\cdot,\cdot)$ --- компакт, то и проекция $\supp L(\cdot,\cdot)$
на $G$ --- тоже компакт.

2) Для любого мультииндекса $\alpha$
$$\Bigl\|\, \partial_x^{\,\alpha} \sum_{i=1}^N L(x,\xi_i)\,|\,\Delta_i| -
\int\limits_H \partial_x^{\,\alpha} L(x,\xi)\,d\xi
\Bigr\|_{C(F)}\to0.$$
$$\sum_{i=1}^N \partial_x^{\,\alpha}L(x,\xi_i)\,|\,\Delta_i|-
\int\limits_H \partial_x^{\,\alpha} L(x,\xi)\,d\xi= \sum_{i=1}^N
\partial_x^{\,\alpha}L(x,\xi_i)\,|\,\Delta_i|-
\sum_{i=1}^N \int\limits_{\Delta_i} \partial_x^{\,\alpha}
L(x,\xi)\,d\xi,\eqno(1)$$ По теореме о среднем, $\forall
x\,\forall i\,\exists y_i\in\Delta_i\colon \int\limits_{\Delta_i}
\partial_x^{\,\alpha} L(x,\xi)\,d\xi=
\partial_x^{\,\alpha}L(x,y_i)\,|\,\Delta_i|,$ продолжим:

$$(1)=\sum_{i=1}^N \Bigl(\partial_x^{\,\alpha}L(x,\xi_i)-
\partial_x^{\,\alpha}L(x,y_i) \Bigr)\,|\,\Delta_i|.$$
Т.к. $\partial_x^{\,\alpha}L(x,\xi)$ равномерно непрерывен на
$\supp L(\cdot,\cdot)\Subset G\times H,$ то $$\sum_{i=1}^N
\Bigl(\partial_x^{\,\alpha}L(x,\xi_i)-
\partial_x^{\,\alpha}L(x,y_i) \Bigr)\,|\,\Delta_i|\le
\varepsilon_N\,\sum_{i=1}^N |\,\Delta_i|=\varepsilon_N\, \mes H,\,
\varepsilon_N\to0,$$ когда диаметр разбиения стремится к нулю:
$|\,\xi_i-y_i|\le \diam T$ --- диаметр разбиения, т.е.
$\xi_i\in\Delta_i,\,y_i\in\Delta_i.\,\blacksquare$

Теорема доказана. $\blacksquare$

\section{Теорема о среднем для гармонических функций}

\textbf{Теорема.} Пусть $u$ --- гармоническая функция в области
$\Omega\subset\mathbb R^n,\, B_r^{x_0}\subset\Omega,$ тогда
$$u(x_0)=\cfrac1{|\,S_r|}\,\int\limits_{S_r^{x_0}}u\,ds,$$ где
$|\,S_r|$ --- площадь $(n-1)$--мерной сферы радиуса $r$.

$\square$

Как было показано в предыдущей лекции, $u\in
C^{\infty}(B_r^{x_0}).$ Имеем:
$$0=\int\limits_{B_r^{x_0}\setminus B_{\varepsilon}^{x_0}} E(x-x_0)\,\triangle
u\,dx,$$
$$\mbox{где }
E(x)=\left\{%
\begin{array}{ll}
    \cfrac1{2\pi}\,\ln|\,x|,\,n=2,      \\
    -\cfrac1{(n-2)|S_1|}\,\cfrac1{|\,x|^{n-2}},\,n\ge3, \\
\end{array}%
\right.$$ --- фундаментальное решение оператора Лапласа.

Интегрируя  по частям, получим:
$$0=\int\limits_{B_r^{x_0}\setminus B_{\varepsilon}^{x_0}} E(x-x_0)\,\triangle
u\,dx= \int\limits_{\partial(B_r^{x_0}\setminus
B_{\varepsilon}^{x_0})} E(x-x_0)\,\cfrac{\partial
u}{\partial\nu}\,ds- \int\limits_{B_r^{x_0}\setminus
B_{\varepsilon}^{x_0}} \triangledown_x E(x-x_0)\, \triangledown
u\,dx=$$

$$=\int\limits_{\partial(B_r^{x_0}\setminus
B_{\varepsilon}^{x_0})} E(x-x_0)\,\cfrac{\partial
u}{\partial\nu}\,ds- \int\limits_{\partial(B_r^{x_0}\setminus
B_{\varepsilon}^{x_0})} \cfrac{\partial
E(x-x_0)}{\partial\nu_x}\,u\,ds- \int\limits_{B_r^{x_0}\setminus
B_{\varepsilon}^{x_0}}\triangle_x E(x-x_0)\, u\,dx,$$ где $\nu$
--- внешняя нормаль к границе области $\partial(B_r^{x_0}\setminus
B_{\varepsilon}^{x_0}).$

Т.о., для любого $\varepsilon\in(0,r)$ будем иметь:

$$\int\limits_{S_r^{x_0}} E(x-x_0)\,\cfrac{\partial
u}{\partial\nu}\,ds+ \int\limits_{S_{\varepsilon}^{x_0}}
E(x-x_0)\,\cfrac{\partial u}{\partial\nu}\,ds=
\int\limits_{S_r^{x_0}} \cfrac{\partial
E(x-x_0)}{\partial\nu_x}\,u\,ds+
\int\limits_{S_{\varepsilon}^{x_0}} \cfrac{\partial
E(x-x_0)}{\partial\nu_x}\,u\,ds,$$

$$\int\limits_{S_r^{x_0}} |\,E(x-x_0)|\,\Bigl|\cfrac{\partial
u}{\partial\nu}\,\Bigr|\,ds\le
\|\,u\|_{C(\overline{B_r^{x_0}})\,|\,S_1|\,\varepsilon^{n-1}}
\left\{%
\begin{array}{ll}
    \cfrac1{2\pi}\,\ln|\,\varepsilon|,\,n=2,      \\
    -\cfrac1{(n-2)|S_1|}\,\cfrac1{|\,\varepsilon|^{n-2}},\,n\ge3, \\
\end{array}%
\right.$$ где $|\,S_1|\,\varepsilon^{n-1}$ --- площадь
$(n-1)$--мерной сферы $S_{\varepsilon}^{x_0}.$

$$\int\limits_{S_r^{x_0}} E(x-x_0)\,\cfrac{\partial
u}{\partial\nu}\,ds+ \Bigl.E(x-x_0) \Bigr|_{x\in S_{r}^{x_0}}
\int\limits_{S_r^{x_0}}\cfrac{\partial u}{\partial\nu}\,ds,$$ по
формуле Грина,
$$\int\limits_{S_r^{x_0}}\cfrac{\partial u}{\partial\nu}\,ds=
\int\limits_{S_r^{x_0}}\triangle u\,dx=0.$$

Тем самым,
$$\int\limits_{S_{\varepsilon}^{x_0}} \cfrac{\partial
E(x-x_0)}{\partial\nu_x}\,u\,ds=-\cfrac{\partial
E}{\partial\nu_x}\,(x-x_0)=-\cfrac d{dr} \left.
\left\{%
\begin{array}{ll}
    \cfrac1{2\pi}\,\ln r,\,n=2,      \\
    -\cfrac1{(n-2)|S_1|}\,\cfrac1{ r^{n-2}},\,n\ge3, \\
\end{array}%
\right.\right|_{r=\varepsilon}=
-\cfrac1{|S_1|}\,\cfrac1{|\,\varepsilon|^{n-1}},$$

$$\cfrac1{|S_1|}\,\cfrac1{|\,\varepsilon|^{n-1}}
\int\limits_{S_{\varepsilon}^{x_0}} u\,ds=
\cfrac1{|S_1|}\,\cfrac1{|\,\varepsilon|^{n-1}}
\int\limits_{S_{\varepsilon}^{x_0}} u(x_0)\,ds+
\cfrac1{|S_1|}\,\cfrac1{|\,\varepsilon|^{n-1}}
\int\limits_{S_{\varepsilon}^{x_0}} (u(x)-u(x_0))\,ds,$$

$$\cfrac1{|S_1|}\,\cfrac1{|\,\varepsilon|^{n-1}}
\int\limits_{S_{\varepsilon}^{x_0}} |\,u(x)-u(x_0)|\,ds\le
\sup_{x\in
S_{\varepsilon}^{x_0}}|\,u(x)-u(x_0)|\to0,\,\varepsilon\to0,$$
т.к. $u$ --- непрерывна.

Т.о.,

$$u(x_0)=\int\limits_{S_r^{x_0}} \cfrac{\partial
E(x-x_0)}{\partial\nu_x}\,u\,ds= \cfrac1{|S_1|}\,\cfrac1{r^{n-1}}
\int\limits_{S_{r}^{x_0}} u\,ds=\cfrac1{|S_r|}
\int\limits_{S_{r}^{x_0}} u\,ds.\,\blacksquare$$

\textbf{Теорема.} Пусть $\Omega$ --- область в $\mathbb R^n,\,
\overline{B_r^{x_0}},\,u$ --- гармоническая функция в $\Omega,$
тогда
$$u=\cfrac1{|\,B_r^{x_0}|}\int\limits_{B_r^{x_0}}
u(x)\,dx.$$

$\square$

$\forall \rho \in (0,r)$ имеем по предыдущей теореме:
$u(x_0)=\cfrac1{|S_{\rho}|} \int\limits_{S_{\rho}^{x_0}} u\,ds,$
или  $u(x_0)\,|S_{\rho}^*|=\int\limits_{S_{\rho}^{x_0}} u\,ds.$
Проинтегрируем последнее равенство от $\rho$ до $r$:

$$u(x_0)\int\limits_0^r d\rho\,|S_{\rho}|=
\int\limits_0^r d\rho \int\limits_{S_{\rho}^{x_0}} u\,ds\,\,
\Rightarrow |\,B_r^{x_0}|\,u(x_0)= \int\limits_{B_r^{x_0}}
u\,dx.\,\blacksquare$$

\section{Принцип максимума для гармонических функций}

\textbf{Теорема.} Пусть $\Omega \subset \mathbb R^n$ --- область,
$u(x)$ --- гармоническая функция в $\Omega$ и
$u(x_0)=\sup\limits_{\Omega}u$ для некоторого $x_0\in\Omega.$
Тогда $u(x)\equiv \const$ в $\Omega$.

$\square$

Обозначим через $E$ множество точек $x\in\Omega\colon
u(x)=\sup\limits_{\Omega}u$. Т.к. $x_0\in E$, то
$E\ne\varnothing.$

Т.к. $u\in C(\Omega),$ то $E$ --- замкнуто в топологии,
индуцированной на $\Omega\subset \mathbb R^n.$ Для достаточного
малого $\varepsilon>0\colon B_{\varepsilon}^{x}\subset \mathbb
R^n,$ получим: $u(y)=\sup\limits_{y}u,\,\forall y\in
B_{\varepsilon}^{x}.$ Почему так?

Пусть $u(y)\ne \sup u$ для некоторого $\in B_{\varepsilon}^{x}.$
Тогда $u(y)<\sup\limits_{\Omega}u.$ Т.о., из непрерывности $u$
следует, что существует множество $\omega \subset
B_{\varepsilon}^{x}\colon \Bigl.u\Bigr|_{\omega}<
\sup\limits_{\Omega}u< \lambda,\, \mes \omega>0$ для некоторого
$\lambda>0.$ Поэтому,

$$\cfrac1{|\,B_{\varepsilon}^{x_0}|} \int\limits_{B_{\varepsilon}^{x_0}}
u(x)\,dx= \cfrac1{|\,B_{\varepsilon}^{x_0}|} \int\limits_{\omega}
u(x)\,dx+ \cfrac1{|\,B_{\varepsilon}^{x_0}|}
\int\limits_{B_{\varepsilon}^{x_0}\setminus\omega} u(x)\,dx\le$$
$$\le \cfrac1{|\,B_{\varepsilon}^{x_0}|} \Bigl(\sup\limits_{\Omega}u-\lambda\Bigr)\,\mes \omega+
\cfrac1{|\,B_{\varepsilon}^{x_0}|}
\Bigl(|\,B_{\varepsilon}^{x_0}|-\mes
\omega\Bigr)\,\sup\limits_{\Omega}u\le$$
$$\le \sup\limits_{\Omega}u - \lambda\,\cfrac{\mes \omega}{|\,B_{\varepsilon}^{x_0}|}<
\sup\limits_{\Omega}u=u(x_0)\mbox{ --- противоречие.}$$

Т.о., $B_{\varepsilon}^{x_0}\subset E\Rightarrow E$ --- открытое и
одновременно замкнутое множество в топологии, индуцированной на
$\Omega\subset \mathbb R^n.$ Т.к. $\Omega$ --- связное множество и
$E\ne\varnothing$, то $E=\Omega.\,\blacksquare$

\textbf{Следствие.} Пусть $u\in C(\overline{\Omega}), u$ ---
гармоническая функция в $\Omega$, где $\Omega\subset \mathbb R^n$
--- область. Тогда $\max\limits_{\overline{\Omega}}u=\max\limits_{\partial \Omega}u.$

\textbf{Теорема.} (\textbf{Неравенство Харнака}) Пусть $u$ ---
гармоническая функция в $B_r$. Тогда $\forall r\in (0,R) \,\,
\sup\limits_{B_R}u\le\gamma \sup\limits_{B_r}u,$ где
$\gamma=\const>0$ и зависит только от $n,\frac rR.$

\textbf{Лемма.} Пусть $u$ --- гармоническая функция в $B_1$. Тогда
$\sup\limits_{B_{1/4}}u\le\gamma \inf\limits_{B_{1/4}}u,$ где
$\gamma=\const>0$ и зависит только от $n.$

$\square$

Пусть $x\in B_{1/4},\, u(x)=\inf\limits_{B_{1/4}}u.$ По теореме о
среднем, $u(x)=\cfrac1{|\,B_{3/4|}}\int\limits_{B_{3/4|}^x}
u(\xi)\,d\xi.$ Пусть $y\in B_{1/4|}\colon
u(y)=\sup\limits_{B_{1/4}}u.$ По теореме о среднем,
$u(y)=\cfrac1{|\,B_{1/4|}}\int\limits_{B_{1/4|}^y} u(\xi)\,d\xi\le
\cfrac1{|\,B_{1/4|}}\int\limits_{B_{1/4|}^x} u(\xi)\,d\xi,$ т.к.
$B_{1/4|}^y\subset B_{3/4|}^x$. Т.о., $u(y)\le
\cfrac{|\,B_{3/4|}}{|\,B_{1/4|}}\,u(x).\,\blacksquare$

\textbf{Следствие.} Пусть $u\ge0$ --- гармоническая функция в
$\overline{B_{\varepsilon}^{x_0}},$ тогда
$\sup\limits_{B_{\varepsilon/4}^{x_0}}u\le\gamma
\inf\limits_{\overline{B_{\varepsilon/4}^{x_0}}}u,$ где
$\gamma=\const>0$ и зависит только от $n.$

$\square$

Заменой переменных переводим $B_{\varepsilon}^{x_0}$ в $B_1$ и
применяем теорему. $\blacksquare$

$\square$ (неравенство Харнака)

Найдется $x\in \overline{B_{r}^{x_0}}\colon
u(x)=\sup\limits_{B_{r}^{x_0}}u$ и $y\in
\overline{B_{r}^{x_0}}\colon u(y)=\inf\limits_{B_{r}^{x_0}}u.$
Соединим $x$ и $y$ прямой в~ $L$. Возьмем $\varepsilon>0\colon
\forall z\in L\,\, B_{\varepsilon}^{z}\subset B_{r}^{x_0}.$
Возьмем конечную последовательность точек $x=z_1,\ldots,z_N=y.$
Шары $\Bigl\{ B_{\varepsilon/4}^{z_i} \Bigr\}_{i=1}^N$ покрывают
$L$. Для $i=1,\ldots,N$ имеем:
$\sup\limits_{B_{\varepsilon/4}^{z_i}}u\le\gamma
\inf\limits_{B_{\varepsilon/4}^{z_i}}u.$ Т.о.,
$u(x)\le\gamma^N\,u(y).\,\blacksquare$

%lect29

\section{Теорема Лиувилля}

Пусть $\triangle u=0$ в $\mathbb R^n$ и при этом $u(x)\ge0$ в
$\mathbb R^n$. Тогда $u\equiv\const$ в $\mathbb R^n$.

$\square$

Возьмем некоторое $x_0\in\mathbb R^n.$ Для некоторого
вещественного $r>0$ имеем: $\sup\limits_{B_r^{x_0}}u\le\gamma
u(x_0),$ где $\gamma=\const$ и зависит от $n$ --- из неравенства
Харнака для шаров $B_r^{x_0}\subset B_{2r}^{x_0}$. Устремляя $r$ в
бесконечность, получим: $\sup\limits_{\mathbb R^n}u\le\gamma\inf
u(x_0)=0 \Rightarrow u\equiv0$ в $\mathbb R^n.\,\blacksquare$


\section{Принцип максимума для решения параболических неравенств}

Пусть $\omega\in \mathbb R^{n+1}$ --- ограниченная область.

\textbf{Определение.} $(x_0,t_0)\in \partial\omega$
\textbf{принадлежит верхней крышке $\gamma$ множества $\omega$},
если найдется вещественные числа $r>0,t_2>t_1\colon$

1) $t_0\in(t_1,t_2);$

2) $B_r^{x_0}\times (t_0,t_2)\subset \mathbb R^{n+1}\setminus
\omega;$

3) $B_r^{x_0}\times (t_1,t_0)\subset \omega,$ где
$B_r^{x_0}=\Bigl\{ x\in \mathbb R^n\colon |\,x-x_0|<r \Bigr\}.$

\textbf{Определение.} Множество
$\Gamma=\partial\omega\setminus\gamma$, где $\gamma$ --- верхняя
крышка области $\omega$, называется \textbf{параболической
(собственной) границей.}

\textbf{Пример.} $\omega=B_R\times(0,T).$

\textbf{Теорема (принцип максимума).} Пусть $\omega\in \mathbb
R^{n+1}$ --- область, $\gamma$ --- верхняя крышка области, a
$\Gamma=\partial\omega\setminus\gamma$ --- параболическая граница
области $\omega$. Обозначим $\Omega=\omega\cup\gamma.$ И пусть
$u\in C^1(\Omega)\cap C(\overline{\omega})\colon
\left\{%
\begin{array}{ll}
    \triangle u-u_t\ge0 \mbox { в } \Omega,      \\
    \Bigl. u \Bigr|_{\partial\Gamma}\le0. \\
\end{array}%
\right.$ Тогда $u(x)\le0,\,\forall x\in \overline{\omega}.$

$\square$ (от противного)

Пусть $\max\limits_{\overline{\omega}}u>0.$

Т.к. $u\in C(\overline{\omega})$, то существует $(x_0,t_0)\in
\overline{\omega}\colon
u(x_0,t_0)=\max\limits_{\overline{\omega}}u.$ $\Bigl. u
\Bigr|_{\Gamma}\le0\Rightarrow (x_0,t_0)\in {\Omega}.$ Рассмотрим
$v(x,t)=u(x,t)-\varepsilon (t-t_0).$ Очевидно, что $\triangle
v-v_t\ge0 \mbox { в } \Omega.$ C другой стороны, выбирая
$\varepsilon>0$ достаточно малым, получим: $v(x_0,t_0)=u(x_0,t_0)$
и
$$\max\limits_{\Gamma}v=\max\limits_{\Gamma}u+\varepsilon\diam\omega<
u(x_0,t_0)<v(x_0,t_0).$$ Т.о., найдется точка $(x_1,t_1)\in \Omega
\colon v(x_1,t_1)=\max\limits_{\omega}v.$ Если $v_t(x_1,t_1)<0,$
то $v(x_1,t_1)<v(x_1,t_2)$ для некоторого $t_1<t_2\colon
(x_1,t_2)\in {\Omega}.$ Т.о., $v$ не может достигать максимума в
точке $(x_2,t_1).$ Пусть $v_t(x_1,t_1)>0,$ тогда $\triangle
v(x_1,t_1)>v_t(x_1,t_1)\ge0.$ Т.к. $(x_1,t_1)\in \Omega$ и в этой
точке достигается максимум $v$, то $\triangledown_x v(x_1,t_1)=0$
и $$\triangle v(x_1,t_1)=\sum_{i=1}^n \cfrac{\partial^2
v(x_1,t_1)}{\partial {x^i}^2}>0 \Rightarrow \cfrac{\partial^2
v(x_1,t_1)}{\partial {x^i}^2}>0 \mbox{для некоторого $i$}.$$
Фиксируя все переменные, кроме $x^i,$ получим:
$$v(x_1^1,\ldots,x_1^i+h,\ldots,x_i^n,t_1)=
v(x_1^1,\ldots,x_1^i,\ldots,x_i^n,t_1)+
\underbrace{\cfrac{\partial v(x_1,t_1)}{\partial {x^i}}}_{=0}\,h+
\underbrace{\cfrac{\partial^2 v(x_1,t_1)}{\partial
{x^i}^2}}_{=0}\,h^2+o(h^2)>0.$$ Они равны нулю, т.к.
$\triangledown_x v(x,t)=0$ для достаточно малых $h$.

Т.о., $v(x_1^1,\ldots,x_1^i+h,\ldots,x_i^n,t_1)>
v(x_1^1,\ldots,x_1^i,\ldots,x_i^n,t_1)$ для некоторого $h$,
которое может быть настолько малым, что
$v(x_1^1,\ldots,x_1^i+h,\ldots,x_i^n,t_1)\in\Omega.$ Это
противоречит тому, что в
$(x_1,t_1)=(x_1^1,\ldots,x_1^i,\ldots,x_i^n,t_1)$ достигается
максимум функции $v$ на множестве
$\overline{\omega}.\,\blacksquare$

\newpage

\section*{Экзаменационные вопросы}

\begin{enumerate}

\item
Классификация линейных уравнений второго порядка по
И.Г.~Петровскому.

\item
Характеристические и не характеристические поверхности. Теорема
Коши\ч Ковалевской (на примере линейных уравнений второго
порядка).

\item
Пространства ${\mathcal D} (\Omega)$ и ${\mathcal D}' (\Omega)$
основных и обобщенных функций. Операции сложения обобщенных
функций и умножения обобщенной функции на бесконечно гладкую.
Определение производной от обобщенной функции.

\item
Слабый предел обобщенных функций. Замкнутость пространства
обобщенных функций относительно слабого предела (без
доказательства).

\item
Первообразная от обобщенной функции. Докажите, что у всякой
обобщенной функции $f \in {\mathcal D}' ({\mathbb R})$ существует
первообразная, определенная с точностью до константы.


\item
Замена переменной у обобщенной функции.

\item
Носитель обобщенной функции. Свойства носителя обобщенных функций:
$$
    \supp (f + g)
    \subset
    \supp f \cup \sup g,
$$
$$
    \supp (f \psi)
    \subset
    \supp f \cap \supp \psi,
$$
$$
    \supp \frac{\partial f }{\partial x_i}
    \subset
    \supp f,
$$
где $f,g \in {\mathcal D}' (\Omega)$, $\psi \in C^\infty
(\Omega)$, $\Omega$ --- открытое подмножество ${\mathbb R}^n$.

\item
Обобщенные функции с компактным носителем. Докажите, что всякая
обобщенная функция $f \in {\mathcal D}' (\Omega)$ с компактным
носителем $\supp f \Subset \Omega$, где $\Omega$ --- открытое
подмножество ${\mathbb R}^n$, является непрерывным функционалом на
пространстве $C^m (\Omega)$ для некоторого целого числа $m \ge 0$.

\item
Прямое произведение обобщенных функций. Докажите эквивалентность
двух определений прямого произведения:
$$
    (f (x), (g (y), \varphi (x, y)))
    =
    (g (y), (f (x), \varphi (x, y)))
$$
для всех $f \in {\mathcal D}' (X)$, $g \in {\mathcal D}' (Y)$ и
$\varphi \in {\mathcal D} (X \times Y)$, где $X \subset {\mathbb
R}^n$ и $Y \subset {\mathbb R}^m$
--- открытые множества.

\item
Свертка обобщенных функций. Коммутативность свертки. Существование
свертки в случае, когда одна из двух обобщенных функций имеет
компактный носитель.

\item
Дифференцирование свертки обобщённых функций.


\item
Фундаментальное решение линейного дифференциального оператора с
постоянными коэффициентами.

\item
Теоремы существования и единственности решений уравнения
$$
    {\mathcal L}
    u
    =
    f (x)
$$
в пространстве обобщенных функций ${\mathcal D}' ({\mathbb R}^n)$,
где $
    {\mathcal L}
    =
    \sum_{
        |\alpha| \le m
    }
    c_\alpha
    \partial^\alpha
$
--- линейный дифференциальный оператор с постоянными коэффициентами.

\item
Фундаментальное решение линейного обыкновенного дифференциального
оператора с постоянными коэффициентами.

\item
Обобщенная задача Коши для линейного обыкновенного
дифференциального уравнения с постоянными коэффициентами.
Существование и единственность решения.

\item
Формулы Грина: многомерный аналог формулы Ньютона-Лейбница
$$
    \int_\Omega
    \frac{\partial h}{\partial x_i}
    \,
    dx
    =
    \int_{
        \partial \Omega
    }
    h
    \cos (\nu, x_i)
    dS
$$
и формула интегрирования по частям
$$
    \int_\Omega
    \frac{\partial f}{\partial x_i}
    g
    \,
    dx
    =
    \int_{
        \partial \Omega
    }
    fg
    \cos (\nu, x_i)
    dS
    -
    \int_\Omega
    \frac{\partial g}{\partial x_i}
    f
    \,
    dx,
$$
где $h,f,g \in C^1 (\Omega) \cap C (\overline \Omega)$, $\Omega$
--- ограниченное открытое подмножество ${\mathbb R}^n$ с кусочно
гладкой границей, а $\nu$ --- вектор внешней нормали к $\partial
\Omega$.

\item
Фундаментальное решение оператора Лапласа
$$
    {\mathcal E}_n (x)
    =
    \left\{
        \begin{array}{ll}
            \frac{1}{2 \pi}
            \ln |x|,
            &
            n = 2,
            \\
            - \frac{
                1
            }{
                (n - 2)
                |S_1|
            }
            \frac{
                1
            }{
                |x|^{n - 2}
            },
            &
            n \ge 3.
        \end{array}
    \right.
$$

\item
Фундаментальное решение волнового оператора $
    \square_a = \partial_t^2 - a^2 \partial_x^2,
$ $a > 0$, случай одномерного основного пространства:
$$
    {\mathcal E}_1 (x, t)
    =
    \frac{1}{2 a}
    \theta (a t - |x|),
    \quad
    (x, t)
    \in
    {\mathbb R}^2.
$$

\item
Фундаментальное решение волнового оператора $
    \square_a = \partial_t^2 - a^2 \triangle,
$ $a > 0$, случай трехмерного основного пространства:
$$
    {\mathcal E}_3 (x, t)
    =
    \frac{
        \theta (t)
    }{
        4
        \pi
        a^2
        t
    }
    \delta_{
        S_{at}
    }
     (x),
    \quad
    (x, t)
    \in
    {\mathbb R}^4,
$$
где
$$
    (\delta_{S_r}, \varphi)
    =
    \int_{
        S_r
    }
    \varphi
    \,
    dS,
    \quad
    \varphi \in {\mathcal D} ({\mathbb R}^3).
$$

\item
Фундаментальное решение волнового оператора $
    \square_a = \partial_t^2 - a^2 \triangle,
$ $a > 0$, случай двумерного основного пространства:
$$
    {\mathcal E}_2 (x, t)
    =
    \frac{
        \theta (at - |x|)
    }{
        2
        \pi
        a
        \sqrt{
            a^2 t^2 - |x|^2
        }
    },
    \quad
    (x, t)
    \in
    {\mathbb R}^3.
$$
Теорема о продолжении обобщенных функций из ${\mathcal D}'
({\mathbb R}^{n+1})$ на функции вида $\varphi (x) 1 (x_n)$, где
$\varphi \in {\mathcal D} ({\mathbb R}^n)$.

\item
Фундаментальное решение оператора теплопроводности $
    \partial_t - a^2 \triangle,
$ $a > 0$,
$$
    {\mathcal E}_n (x, t)
    =
    \frac{
        \theta (t)
    }{
        (
            2
            a
            \sqrt{
                \pi t
            }
        )^n
    }
    e^{
        - \frac{
            |x|^2
        }{
            4 a^2 t
        }
    },
    \quad
    (x, t) \in {\mathbb R}^{n+1}.
$$

\item
Обобщенная задача Коши для волнового уравнения. Теорема
существования и единственности решения (случай размерности
основного пространства $n = 1,2,3$).

{\it Подсказка.} Считать известным, что в случае размерности
основного пространства $n = 1,2,3$ фундаментальными решениями
волнового оператора $
    \partial_t^2 - a^2 \triangle,
$ $a > 0$, являются следующие функции:
$$
    {\mathcal E}_1 (x, t)
    =
    \frac{1}{2 a}
    \theta (a t - |x|),
    \quad
    (x, t)
    \in
    {\mathbb R}^2,
$$
$$
    {\mathcal E}_2 (x, t)
    =
    \frac{
        \theta (at - |x|)
    }{
        2
        \pi
        a
        \sqrt{
            a^2 t^2 - |x|^2
        }
    },
    \quad
    (x, t)
    \in
    {\mathbb R}^3,
$$
и
$$
    {\mathcal E}_3 (x, t)
    =
    \frac{
        \theta (t)
    }{
        4
        \pi
        a^2
        t
    }
    \delta_{
        S_{at}
    }
     (x),
    \quad
    (x, t)
    \in
    {\mathbb R}^4,
$$
где
$$
    (\delta_{S_r}, \varphi)
    =
    \int_{
        S_r
    }
    \varphi
    \,
    dS,
    \quad
    \varphi \in {\mathcal D} ({\mathbb R}^3).
$$

\item
Классическая задача Коши для одномерного волнового уравнения
$$
    \left\{
        \begin{array}{l}
            u_{tt} = a^2 u_{xx} + f (x, t),
            \quad
            x \in {\mathbb R},
            \:
            t > 0,
            \\
            u (x, 0) = u_0 (x),
            \\
            u_t (x, 0) = u_1 (x),
        \end{array}
    \right.
$$
где $a > 0$, $f \in C^1 ({\mathbb R} \times [0, \infty))$, $u_0
\in C^2 ({\mathbb R})$, $u_1 \in C^1 ({\mathbb R})$.
%$u \in C^2 ({\mathbb R} \times (0, \infty) \cap C^1 ({\mathbb R} \times [0, \infty))$.
Существование и единственность решения. Формула Даламбера
\begin{align*}
    u (x, t)
    &
    =
    \frac{1}{2 a}
    \int_0^t
    \int_{
        x - a (t - \tau)
    }^{
        x + a (t - \tau)
    }
    f (\xi, \tau)
    \,
    d\xi
    d\tau
    +
    \frac{1}{2 a}
    \int_{
        x - a t
    }^{
        x + a t
    }
    u_1 (\xi)
    \,
    d\xi
    \\
    &
    +
    \frac{1}{2}
    (
        u_0 (x + at)
        +
        u_0 (x - at)
    ).
    \nonumber
\end{align*}
{\it Подсказка.} Считать известным, что фундаментальным решением
одномерного волнового оператора $
    \partial_t^2 - a^2 \partial_x^2,
$ $a > 0$, является следующая функция:
$$
    {\mathcal E}_1 (x, t)
    =
    \frac{1}{2 a}
    \theta (a t - |x|),
    \quad
    (x, t)
    \in
    {\mathbb R}^2.
$$

\item
Классическая задача Коши для трехмерного волнового уравнения
$$
    \left\{
        \begin{array}{l}
            u_{tt} = a^2 \triangle u + f (x, t),
            \quad
            x \in {\mathbb R}^3,
            \:
            t > 0,
            \\
            u (x, 0) = u_0 (x),
            \\
            u_t (x, 0) = u_1 (x),
        \end{array}
    \right.
$$
где $a > 0$, $f \in C^2 ({\mathbb R}^3 \times [0, \infty))$, $u_0
\in C^3 ({\mathbb R}^3)$, $u_1 \in C^2 ({\mathbb R}^3)$.
Существование и единственность решения. Формула Кирхгофа
\begin{align*}
    u (x, t)
    =
    {}
    &
    \frac{
        1
    }{
        4
        \pi
        a^2
    }
    \int_{
        B_{at}^x
    }
    \frac{
        f
        \left(
            \xi,
            t
            - \frac{
            |x - \xi|
            }{
                a
            }
        \right)
    }{
        |x - \xi|
    }
    \,
    d\xi
    +
    \frac{
        1
    }{
        4
        \pi
        a^2
        t
    }
    \int_{
        S_{at}^x
    }
    u_1 (\xi)
    \,
    dS
    \\
    &
    +
        \frac{
            1
        }{
            4
            \pi
            a^2
        }
    \frac{\partial}{\partial t}
    \left(
        \frac{
            1
        }{
            t
        }
        \int_{
            S_{at}^x
        }
        u_0 (\xi)
        \,
        dS
    \right).
\end{align*}
{\it Подсказка.} Считать известным, что фундаментальным решением
трехмерного волнового оператора $
    \partial_t^2 - a^2 \triangle,
$ $a > 0$, является следующая обобщенная функция:
$$
    {\mathcal E}_3 (x, t)
    =
    \frac{
        \theta (t)
    }{
        4
        \pi
        a^2
        t
    }
    \delta_{
        S_{at}
    }
     (x),
    \quad
    (x, t)
    \in
    {\mathbb R}^4,
$$
где
$$
    (\delta_{S_r}, \varphi)
    =
    \int_{
        S_r
    }
    \varphi
    \,
    dS,
    \quad
    \varphi \in {\mathcal D} ({\mathbb R}^3).
$$


\item
Классическая задача Коши для двумерного волнового уравнения
$$
    \left\{
        \begin{array}{l}
            u_{tt} = a^2 \triangle u + f (x, t),
            \quad
            x \in {\mathbb R}^2,
            \:
            t > 0,
            \\
            u (x, 0) = u_0 (x),
            \\
            u_t (x, 0) = u_1 (x),
        \end{array}
    \right.
$$
где $a > 0$, $f \in C^2 ({\mathbb R}^2 \times [0, \infty))$, $u_0
\in C^3 ({\mathbb R}^2)$, $u_1 \in C^2 ({\mathbb R}^2)$.
Существование и единственность решения. Формула Пуассона
\begin{align*}
    u (x, t)
    =
    {}
    &
    \frac{
        1
    }{
        2
        \pi
        a
    }
    \int_0^t
    \int_{
        B_{
            a (t - \tau)
        }^x
    }
    \frac{
        f (\xi, \tau)
        \,
        d\xi
        d\tau
    }{
        \sqrt{
            a^2 (t - \tau)^2
            -
            |x - \xi|^2
        }
    }
    +
    \frac{
        1
    }{
        2
        \pi
        a
    }
    \int_{
        B_{
            a t
        }^x
    }
    \frac{
        u_1 (\xi)
        \,
        d\xi
    }{
        \sqrt{
            a^2 t^2
            -
            |x - \xi|^2
        }
    }
    \\
    &
    +
    \frac{
        1
    }{
        2
        \pi
        a
    }
    \frac{\partial}{\partial t}
    \left(
        \int_{
            B_{
                a t
            }^x
        }
        \frac{
            u_0 (\xi)
            \,
            d\xi
        }{
            \sqrt{
                a^2 t^2
                -
                |x - \xi|^2
            }
        }
    \right).
\end{align*}
{\it Подсказка.} Считать известным, что фундаментальным решением
двумерного волнового оператора $
    \partial_t^2 - a^2 \triangle,
$ $a > 0$, является следующая функция:
$$
    {\mathcal E}_2 (x, t)
    =
    \frac{
        \theta (at - |x|)
    }{
        2
        \pi
        a
        \sqrt{
            a^2 t^2 - |x|^2
        }
    },
    \quad
    (x, t)
    \in
    {\mathbb R}^3.
$$

\item
Теорема существования и единственности решения обобщенной задачи
Коши для уравнения теплопроводности
$$
    \left\{
        \begin{array}{l}
            u_t = a^2 \triangle u + f (x, t),
            \quad
            x \in {\mathbb R}^n,
            \:
            t > 0,
            \\
            u (x, 0) = u_0 (x),
        \end{array}
    \right.
$$
где $f \in {\mathcal M}$, $u_0 \in L_\infty ({\mathbb R}^n)$.
Формула Пуассона
\begin{align*}
    u (x, t)
    =
    {}
    &
    \int_0^t
    \int_{
        {\mathbb R}^n
    }
    \frac{
        f (\xi, \tau)
    }{
        (
            2
            a
            \sqrt {
                \pi
                (t - \tau)
            }
        )^n
    }
    e^{
        - \frac{
            |x - \xi|^2
        }{
            4
            a^2
            (t - \tau)
        }
    }
    \,
    d\xi
    d\tau
    \\
    &
    +
    \frac{
        1
    }{
        (
            2
            a
            \sqrt {
                \pi
                t
            }
        )^n
    }
    \int_{
        {\mathbb R}^n
    }
    u_0 (\xi)
    e^{
        - \frac{
            |x - \xi|^2
        }{
            4
            a^2
            t
        }
    }
    \,
    d\xi.
\end{align*}

{\it Подсказка.} Считать известным, что фундаментальным решением
оператора теплопроводности $
    \partial_t - a^2 \triangle,
$ $a > 0$, является следующая функция:
$$
    {\mathcal E}_n (x, t)
    =
    \frac{
        \theta (t)
    }{
        (
            2
            a
            \sqrt{
                \pi t
            }
        )^n
    }
    e^{
        - \frac{
            |x|^2
        }{
            4 a^2 t
        }
    },
    \quad
    (x, t) \in {\mathbb R}^{n+1}.
$$

\item
Существование и единственности решения классической задачи Коши
для уравнения теплопроводности
$$
    \left\{
        \begin{array}{l}
            u_t = a^2 \triangle u + f (x, t),
            \quad
            x \in {\mathbb R}^n,
            \:
            t > 0,
            \\
            u (x, 0) = u_0 (x).
        \end{array}
    \right.
$$
Докажите, что если все производные по пространственным переменным,
начиная с нулевого и до второго порядка включительно, от функции
$f$ принадлежат классу ${\mathcal M} \cap C ({\mathbb R}^n \times
[0, \infty))$, а от функции $u_0$ --- классу $L_\infty ({\mathbb
R}^n) \cap C ({\mathbb R}^n)$, то формула Пуассона
\begin{align*}
    u (x, t)
    =
    {}
    &
    \int_0^t
    \int_{
        {\mathbb R}^n
    }
    \frac{
        f (\xi, \tau)
    }{
        (
            2
            a
            \sqrt {
                \pi
                (t - \tau)
            }
        )^n
    }
    e^{
        - \frac{
            |x - \xi|^2
        }{
            4
            a^2
            (t - \tau)
        }
    }
    \,
    d\xi
    d\tau
    \\
    &
    +
    \frac{
        1
    }{
        (
            2
            a
            \sqrt {
                \pi
                t
            }
        )^n
    }
    \int_{
        {\mathbb R}^n
    }
    u_0 (\xi)
    e^{
        - \frac{
            |x - \xi|^2
        }{
            4
            a^2
            t
        }
    }
    \,
    d\xi
\end{align*}
гарантирует, что $u \in C ({\mathbb R}^n \times [0, \infty))$,
$u_t \in C ({\mathbb R}^n \times (0, \infty))$, и при этом $u_{x_i
x_j} \in C ({\mathbb R}^n \times (0, \infty))$ для всех $i, j = 1,
\ldots, n$.

\item
Пространства C.Л.~Соболева $
    W_p^m
    (
        \Omega
    )
$ и $
    \stackrel{\rm o}{W}\!{}_p^m
    (
        \Omega
    ).
$ Полнота и сепарабельность пространств C.Л.~Соболева.

\item
Неравенство Фридрихса.

\item
Обобщенное в смысле С.Л.~Соболева решение первой краевой задачи
$$
    \left\{
        \begin{aligned}
            &
            \triangle u
            =
            f_0 (x)
            +
            \sum_{i = 1}^n
            \frac{\partial f_i}{\partial x_i}
            (x)
            \quad
            \mbox{в } \Omega,
            \\
            &
            \left.
                u
            \right|_{
                \partial \Omega
            }
            =
            u_0,
        \end{aligned}
    \right.
$$
где $\Omega$ --- ограниченная область в ${\mathbb R}^n$, $n \ge
2$, $
    u_0
    \in
    W_2^1
    (
        \Omega
    )
$ и $f_i \in L_2 (\Omega)$, $i = 0,1,\ldots,n$.

Существование и единственность обобщенного в смысле С.Л.~Соболева
решения первой краевой задачи.

\item
Гармонические функции. Докажите, что всякая гармоническая функция
из пространства ${\mathcal D}' (\Omega)$ принадлежит классу
$C^\infty (\Omega)$.

{\it Подсказка.} Воспользуйтесь сверткой с фундаментальным
решением оператора Лапласа
$$
    {\mathcal E}_n (x)
    =
    \left\{
        \begin{array}{ll}
            \frac{1}{2 \pi}
            \ln |x|,
            &
            n = 2,
            \\
            - \frac{
                1
            }{
                (n - 2)
                |S_1|
            }
            \frac{
                1
            }{
                |x|^{n - 2}
            },
            &
            n \ge 3.
        \end{array}
    \right.
$$

\item
Теоремы о среднем для гармонических функций.

\item
Принцип максимума для гармонических функций.

\item
Неравенство Харнака для гармонических функций.

\item
Теорема Лиувилля для гармонических функций.

\item
Принцип максимума для решений параболических неравенств.

\end{enumerate}

\end{document}
