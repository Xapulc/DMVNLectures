\documentclass[a4paper,draft]{article}
\usepackage[simple,utf]{dmvn}

\title{Программа экзамена по теории чисел}
\author{Лектор~--- А.\,И.\,Галочкин}
\date{VII семестр, 2006 г.}

\begin{document}
\maketitle

\begin{nums}{0}
\item Простейшие свойства делимости. Представление наибольшего общего
делителя $d$ чисел $a$ и  $b$ в форме  $d=au+bv$. Теорема о
существовании и единственности разложения чисел на простые
сомножители. Бесконечность множества простых чисел.

\item Лемма о равенстве верхних и нижних пределов функций
$\frac{\theta (x)}{x}$, $\frac{\psi (x)}{x}$ и $\frac{\pi (x) \ln x}{x}$. Связь между
асимптотическим поведением функции Чебышева $\psi (x)$ и функции
$$ \omega (x)=  \intl{1}{x} \frac {\psi (t)}{t} \, dt.$$

\item Оценки Чебышева функции  $\pi (x)$. Оценки $n$-го простого
числа. Расходимость ряда $ {\sums p \frac 1p}$.

\item Аналитичность дзета-функции Римана в области $\sigma >1$.
Разложение в ряд Дирихле ее логарифмической производной.
Представление дзета-функции в виде бесконечного произведения.

\item Преобразования Абеля в интегральной форме. Аналитическое
продолжение $\zeta$-функции в область~${\sigma >0}$.

\item Отсутствие нулей дзета-функции в области $\sigma \ge 1$.

\item Оценки функций $\zeta (s)$, $\zeta '(s)$  и $\frac{\zeta'(s)}{\zeta (s)}$ в области $\sigma \ge 1$, $|t| \ge 3$.

\item Вывод формулы, выражающей функцию

$$ \omega (x)= \intl 1x \frac {\psi (t)}{t} \, dt$$

через дзета-функцию.

\item Сдвиг контура интегрирования и выделение главного члена в
интегральной формуле для функции    $\omega (x)$.

\item Доказательство асимптотического закона распределения простых
чисел. Асимптотическая формула   $n$-го простого числа.

\item Простейшие свойства сравнений. Группа  $(\Z/m\Z)^*$. Теорема
Эйлера. Малая теорема Ферма. Элементарные доказательства
бесконечности множества простых чисел в прогрессиях вида  $4n+1$ и
$4n+3$.

\item Простейшие свойства мультипликативных функций. Явная формула
для значений функции Эйлера,  мультипликативность этой функции.

\item Простейшие свойства групповых характеров. Построение
характеров. Вычисление сумм  $\sums{a \in G} \chi (a)$ и
$\sums{\chi} \chi (a)$ для характеров $\chi$
 группы $G$. Определение и свойства числовых характеров.

\item Аналитичность функции Дирихле $L(s, \chi)$   в области
$\sigma >1$. Разложение в ряд Дирихле ее логарифмической
производной. Отсутствие нулей $L$-функции  в области
$\sigma >1$. Представление $L$-функции в виде бесконечного произведения.
Аналитическое продолжение функции  $L(s, \chi_0)$ в область
$\sigma >0$.

\item Теорема о почленном дифференцировании ряда Дирихле. Область
аналитичности функции  $L(s, \chi)$  при $\chi \ne \chi_0$.

\item Теорема об области сходимости ряда Дирихле с неотрицательными
коэффициентами.

\item Неравенство $L(1, \chi ) \ne 0$   для действительного
характера $\chi$.

\item Неравенство $L(1, \chi ) \ne 0$   при $\chi^2 \ne \chi_0$.

\item Доказательство теоремы Дирихле о бесконечности множества
простых чисел в арифметической прогрессии.

\item Свойства минимального многочлена алгебраического числа. Целые
алгебраические числа. Лемма Гаусса и ее следствия, относящиеся к
целым алгебраическим числам.

\item Формулировка основной теоремы о симметрических многочленах.
Теорема о симметрическом многочлене от нескольких систем
сопряженных алгебраических чисел. Поле алгебраических чисел и
кольцо целых алгебраических чисел. Алгебраическая замкнутость поля
алгебраических чисел.

\item Алгебраическое числовое поле конечной степени. Каноническая
форма представления его элементов. Теорема о примитивном элементе.

\item Две теоремы о приближении действительных чисел рациональными
дробями.

\item Теорема Лиувилля о приближении алгебраических чисел.
Построение трансцендентных чисел при помощи теоремы Лиувилля.

\item Теорема Бореля о характере приближений <<почти всех>> действительных чисел.

\item Иррациональность и трансцендентность числа   $e$.

\item Иррациональность  числа   $\pi$.

\item Обобщение теоремы Лиувилля на многочлены от нескольких
алгебраических чисел.

\item Лемма Зигеля об оценках решений систем линейных уравнений с
целыми коэффициентами.

\item Формулировка теоремы Линдемана. Ее следствия. Построение
вспомогательной функции для доказательства теоремы Линдемана, оценки ее порядка нуля.

\item Оценки вспомогательной функции и завершение доказательства
теоремы Линдемана. Ее связь с проблемой квадратуры круга.

\item Седьмая проблема Гильберта. Формулировка теоремы
Гельфонда~-- Шнейдера. Ее следствия. Построение
вспомогательной функции для доказательства теоремы
Гельфонда~-- Шнейдера, оценки ее порядка нуля.

\item Оценки вспомогательной функции и завершение доказательства
теоремы Гельфонда~-- Шнейдера.
\end{nums}

\medskip
\dmvntrail
\end{document}
