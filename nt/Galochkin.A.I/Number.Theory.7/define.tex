
\newcommand{\newlecture}[2]{\begin{flushright} \textsc{Лекция №#1. #2} \end{flushright}}
\newcommand{\intlim}{\int\limits}
\newcommand{\sumlim}{\sum\limits}
\newcommand{\limlim}{\lim\limits}
\newcommand{\prodlim}{\prod\limits}
\newcommand{\llra}{\Longleftrightarrow}
\newcommand{\lRa}{\Longrightarrow}
\newcommand{\eq}{\equiv}
\newenvironment{mlc}{\begin{equation}\begin{gathered}}{\end{gathered}\end{equation}}
\newenvironment{mlc*}{\begin{equation*}\begin{gathered}}{\end{gathered}\end{equation*}}
\newcommand{\notion}{\emph}
\DeclareMathOperator{\li}{li}

\makeatletter

\renewcommand{\@listI}{%
\leftmargin=40pt
\rightmargin=0pt
\labelsep=5pt
\labelwidth=20pt
\itemindent=0pt
\listparindent=0pt
\topsep=2pt plus 1pt minus 1pt
\partopsep=1pt plus 1pt
\parsep=1pt plus 1pt
\itemsep=\parsep}

\renewcommand{\@listii}{%
\leftmargin=25pt
\rightmargin=0pt
\labelsep=5pt
\labelwidth=20pt
\itemindent=0pt
\listparindent=0pt
\topsep=0pt plus 1pt
\partopsep=0pt plus 1pt
\parsep=0pt
\itemsep=\parsep}

\makeatother

% hyperref options
%\hypersetup{linkcolor = blue}    % Цвет текста ссылок на мишени внутри документа; по умолчанию --- red.
%\hypersetup{filecolor = cyan}    % Цвет текста ссылок на локальные PDF файлы; по умолчанию --- cyan.
%\hypersetup{citecolor = green}   % Цвет библиографических ссылок, которые печатает команда \cite; по умолчанию --- green.
%\hypersetup{urlcolor  = magenta} % Цвет текста URL-ссылок; по умолчанию --- magenta.
%\hypersetup{unicode   = true}
\newenvironment*{authornote}
{\setmainfont{URW Chancery L}
  \begin{flushleft}
  \tiny
  \hangafter 0
  \hangindent=0.5\textwidth
}
{\end{flushleft}}
