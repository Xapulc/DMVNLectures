\documentclass{article}
\usepackage[simple]{dmvn}

\title{Последние три лекции по теории чисел}
\author{Лектор Александр Иванович Галочкин}
\date{Last compiled \today~г.}

\begin{document}
\maketitle

\section{Теорема Линдемана}

ПРИМЕЧАНИЕ: в тех случаях, когда написано <<легко доказать>>, <<легко следует>>,
<<легко получить>> и т. п., читайте: <<Докажите это самостоятельно>>.

ОБОЗНАЧЕНИЯ: $\A$\т поле алгебраических чисел, $\Z_\A$ кольцо целых алгебраических чисел.
Длиной многочлена $P$ называется сумма модулей его коэффициентов $L(P)$.

\subsection{Обобщение теоремы Лиувилля на многочлены от нескольких алгебраических чисел}

\begin{theorem}
Пусть $\al_1, \dots, \al_s$ алгебраические числа степеней соответственно $m_1, \dots, m_s$.
Тогда существует такая положительная постоянная $C=C(\al_1, \dots, \al_s)$, что для любого многочлена
$P(x_1, \dots, x_s) \in \Z[x_1, \dots, x_s]$ либо
$P(\al_1, \dots, \al_s)=0$, либо выполняется неравенство
\eqn{\label{One}|P(\al_1, \dots, \al_s)| \ge L^{1-(m_1 \cdots m_s)} C^{-d},}
где $d$ и $L$ соответственно степень и длина многочлена $P(x_1, \dots, x_s)$.
\end{theorem}
\begin{proof}
Доказательство разобьем на отдельные пункты.

\pt{1} Существует такое натуральное число $a$, что все числа $a \al_1, \dots, a \al_s$ целые алгебраические.

Утверждение было доказано ранее. $a$ равно произведению старших коэффициентов
канонических многочленов чисел $\al_1, \dots, \al_s$.

\pt{2} Число $\be = a^d P(\al_1, \dots, \al_s) \in \Z_{\A}$.

Действительно, если $k_1, \dots, k_s$ неотрицательные целые числа и
$k_1+ \cdots k_s \le d$, то
$$
a^d \al_1^{k_1} \cdots \al_s^{k_s} = (a \al_1)^{k_1} \cdots
(a \al_s)^{k_s}a^{d-k_1- \cdots k_s} \in \Z_{\A},
$$
откуда легко следует утверждение.
\vskip 5mm

\pt{3} Пусть $\al_{i1}, \dots, \al_{im_i}$ числа, сопряженные алгебраическому числу  $\al_i$, $i= \ol{1,s}$.
Тогда все числа
$$|a^dP(\al_{1r_1}, \dots, \al_{sr_s})| \le C_1^d L,$$
где $1 \le r_i \le m_i$, а положительная постоянная  $C_1$ не зависит от многочлена $P$.

Утверждение легко доказывается с
$$C_1= a \max_{i,j} (1, |\al_{ij}|).$$

\pt{4}
$$
A(x)= \prod_{r_1=1}^{m_1} \cdots \prod_{r_s=1}^{m_s} (x-a^dP(\al_{1r_1}, \dots, \al_{sr_s})) \in \Q[x]
$$

Утверждение легко следует из того, что многочлен
$$A(x) = A(x \,| \, \ol {\al}_1, \dots, \ol {\al}_s)$$
симметрический относительно $s$  систем переменных
$\ol{\al}_i= (\al_{i1}, \dots, \al_{im_i})$ и из ранее доказанной
леммы о симметрических многочленах от нескольких систем сопряженных
алгебраических чисел.

\textbf{Доказательство теоремы.} Пусть
$$
B(x)= x^n+b_{n-1}x^{n-1}+ \cdots +b_0 = (x- \be_1) \cdots (x- \be_n)
$$
-- минимальный многочлен числа $\be = \be_1$. Поскольку
$\be \in \Z_{\A}$ (см. пункт 2)), то $B(x) \in \Z [x]$ и
$|b_0| \ge 1$, если только $P(\al_1, \dots, \al_s) \ne 0$.

Многочлены $A(x)$  и $B(x)$  имеют рациональные коэффициенты и
общий корень $\be$,  а к $B(x)$ минимальный многочлен числа
$\be$,    то $B(x) \,| \, A(x)$  и все корни $B(x)$  суть корни $A(x)$.
А тогда по утверждению 3)
$$
1 \le |b_0| = |\be| \cdot |\be_2 \cdots \be_s| \le
a^d|P(\al_1, \dots, \al_s)| ( C_1^d L)^{n-1},
$$

А так как $n \le m_1 \cdots m_s$, то из этого неравенства следует утверждение
теоремы с $C=a C_1^{(m_1 \cdots m_s)-1}$.
\end{proof}

\subsection{Лемма Зигеля об оценках решений систем уравнений с целыми коэффициентами}

\begin{lemma}
Пусть $a_{ij} \in \Z$, $|a_{ij}| < A$  и
$$
L_i(\ol x)= \sum_{j=1}^q a_{ij}x_j, \qquad  i= \ol {1,p}; \quad p<q.
$$
Тогда система уравнений
$$
L_i(\ol x)=0, \qquad i= \ol {1,p}
$$
имеет решение $(x_1^{(0)}, \dots, x_q^{(0)})$, $x_j^{(0)} \in \Z$,
такое, что
$$
0< \max_j |x_j^{(0)}| \le 1 + (qA)^{\dfrac p{q-p}}.
$$
\end{lemma}
%\pagebreak

\begin{proof}
Пусть $X$  натуральное число, которое будет выбрано
в дальнейшем, и каждая из величин $x_j$
пусть независимо друг от друга принимает
значения $0, \pm 1, \dots, \pm X.$ Всего получим $(2X+1)^q$ наборов
$\ol x=(x_1, \dots, x_q)$. Каждому из этих наборов соответствует набор
$\ol L(\ol x) = (L_1(\ol x ), \dots, L_p ( \ol x ))$, причем
$|L_i(\ol x )| \le qAX$  и, следовательно, всего может быть не более
$(2qAX+1)^p$ различных наборов   $\ol L ( \ol x )$.   Если
\begin{equation}[2]
(2X+1)^q > (2qAX+1)^p,
\end{equation}
то по принципу Дирихле можно найти два  набора  $\ol x$: $\ol x^{(1)}$
и $\ol x^{(2)}$, которым соответствует
один и тот же набор значений $\ol L (\ol x)$, то есть
$$
\ol L (\ol x^{(2)}) - \ol L (\ol x^{(1)}) = \ol L (\ol x^{(2)}- \ol x^{(1)})
= \ol 0,
$$
а, значит, $\ol x^{(0)} = \ol x^{(2)}- \ol x^{(1)}  $
-- решение системы, причем $|\ol x^{(0)}| \le 2X$.

Неравенство \thetag 2   выполняется, если
$$
(2X+1)^q > ((qA)(2X+1))^p,
$$
то есть при
$$
2X > (qA)^{\dfrac p{q-p}} -1,
$$
а значит можно найти такое решение $\ol x^{(0)}$, что
$$
2X \le (qA)^{\dfrac p{q-p}} +1,
$$
откуда следует утверждение леммы.
\end{proof}

\subsection{Теорема Линдемана и её следствия. Построение вспомогательной функции, оценки её порядка нуля.}

\begin{theorem}[Линдеман] Если  $\al$   алгебраическое
число, отличное от нуля, то число $e^{\al}$ трансцендентно.
\end{theorem}

\begin{imp}
Число $e$ трансцендентно.
\end{imp}

\begin{imp}
Число $\pi$ трансцендентно.
\end{imp}
\begin{proof}
Легко следует из равенства $e^{\pi i}=-1$.
\end{proof}

\begin{imp}
Если $\al$ алгебраическое число, отличное от $0$    и
$1$, то число $\ln {\al}$ трансцендентно.
\end{imp}
\begin{proof}
Легко следует из равенства $e^{\ln \al}= \al$.
\end{proof}

\begin{imp}
Если $\al \ne 0$ алгебраическое число, то числа
$\sin \al$, $\cos \al$, $\tg \al$ трансцендентны.
\end{imp}
\begin{proof}
Эти утверждения легко следуют из равенств
$$
\sin \al = \frac {e^{i \al}- e^{-i \al}}{2i}, \qquad
\cos \al = \frac {e^{i \al}+ e^{-i \al}}{2}.
$$
\hfill\end{proof}

В дальнейшем пусть $n$ натуральное число, которое будет выбрано достаточно
большим, $\ga_1, \ga_2 \dots, $
не зависящие от $n$ положительные постоянные.

\begin{lemma}
Существует такая функция
\begin{equation}[3]
f(z)=\sum_{k=0}^{n-1} \sum_{l=0}^{n-1} a_{kl}z^k e^{lz}
\end{equation}
с коэффициентами $a_{kl} \in \Z$, что
%\comment
\begin{gather}
0 < \max_{k,l} |a_{kl}| < n^{\ga_1n},
\tag 4 \\
f^{(t)}(0)=0, \quad t= \ol {0,[n^{3/2}]-1},
\tag 5
\end{gather}
%\endcomment
где $[\cdot]$  целая часть числа.
\end{lemma}
\begin{proof}
Из формулы Лейбница следует, что
\begin{equation}[6]
f^{(t)}(z)= \sum_{k,l=0}^{n-1}a_{kl} \sum_{s=0}^{\min (t,k)}C_t^s
k(k-1) \cdots (k-s+1) z^{k-s} l^{t-s} e^{lz}.
\end{equation}
Поэтому
$$
f^{(t)}(0)=\sum_{k,l=0,\, k\le t}^{n-1}C_t^k(k!)l^{t-k}a_{kl},
$$
и для завершения доказательства нам осталось оценить
решение системы из $p=[n^{3/2}]$ уравнений \thetag {5} относительно
$q=n^2$   неизвестных  $a_{kl}$. Их коэффициенты
$$
|C_t^k(k!)l^{t-k}|<2^{n^{3/2}}n^nn^{n^{3/2}}<n^{(3n^{3/2})}=A
$$
По лемме Зигеля существует ненулевое решение этой системы в целых числах
$a_{kl}$, удовлетворяющих неравенству
$$
|a_{kl}|<1+(qA)^{\dfrac p{q-p}}<n^{\ga_1n}.
$$
\hfill\end{proof}

Обозначим через $\ord_{z=a}f(z)$  порядок нуля функции $f(z)$  в точке
$z=a$.

\begin{lemma}
$[n^{3/2}] \le \ord_{z=0}f(z) \le n^2$.
\end{lemma}
\begin{proof}
Оценка снизу следует из \thetag 5. Докажем правое
неравенство.  Все функции $z^ke^{lz}, \quad k,l= \ol{0,n-1}$, являются решениями
дифференциального уравнения
$$
D^n(D-1)^n \cdots (D-n+1)^ny=0, \quad D=\frac d{dz},
$$
с постоянными коэффициентами порядка $n^2$, следовательно, функция  $f(z)$
тоже является решением этого уравнения и, если $f^{(t)}(0)=0, \quad
t=\ol{0,n^2-1}$, то по теореме о единственности решения дифференциального
уравнения  $f(z) \equiv 0$, что невозможно, поскольку $f(x) \to \infty$ при
$x\to  +\infty, \quad x\in\R$.
\end{proof}

\subsection{Оценка вспомогательной функции и завершение доказательства
теоремы Линдемана. Ее связь с проблемой квадратуры круга}

Пусть $X$  не зависящее от $n$
натуральное число, которое будет выбрано в дальнейшем,
\eqn{\label{Seven}
T=\min_{x=\ol{0,X}}\ord_{z=x \al}f(z)}

\begin{lemma}
Справедливы неравенства
$$
|f^{(T)}(x \al)| < n^{-\ga_2n^{3/2}-1/3(X-6)T}, \qquad x=\ol{0,X}.
$$
\end{lemma}
\begin{proof}
Из \eqref{Seven} и леммы 2 следует, что функция
$$
g(z)=f(z)z^{-[n^{3/2}]}(z-\al)^{-T} \cdots (z-X\al)^{-T}
$$
имеет лишь устранимые особые точки, поэтому для нее справедлив принцип
максимума модуля. Возьмем $r=X|\al |+1<\sqrt n$. Тогда
$$
\max_{|z| \le r}|g(z)| \le \max_{|u|=2\sqrt n}|g(u)|.
$$
Поэтому   при достаточно большом $n$

\begin{gather}
M_r=\max_{|z| \le r}|f(z)| \le  \\
\le\max_{|u|=\sqrt n}|f(u)| \cdot \
max_{|z| \le r,\,  |u|=\sqrt n}
\biggl|\biggl(\frac zu \biggr)^{[n^{3/2}]}
\biggl(\frac{z-\al}{u-\al}\biggr)^T   \cdots
\biggl(\frac{z-X\al}{u-X\al}\biggr)^T \biggr| \le \\
\le n^2n^{\ga_1n}(\sqrt n)^n e^{n^{3/2}}\cdot
 n^{-0,4n^{3/2}-0,4XT}<n^{-1/3n^{3/2}-1/3XT}
\tag 8
\end{gather}

Далее,
$$
f^{(T)}(x\al)= \frac{T!}{2\pi i}\oint_{|z-x\al|=1}
\frac {f(z)\,dz}{(z-x\al)^{T+1}},
$$
поэтому по лемме 2
$$
f^{(T)}(x\al) \le (T!)M_r \le T^TM_r \le n^{2T}M_r
$$
и из \thetag 8 следует утверждение леммы.
\end{proof}

\subsection{Доказательство теоремы Линдемана}

\begin{proof}
Из \thetag 7 следует, что существует
такой индекс $x_0$, что $f^{(T)}(x_0\al) \ne 0$, причем эта производная является
многочленом $P(\al,e^{\al})$ с целыми коэффициентами.

Допустим, что при некотором ненулевом
$\al$     оба числа $\al$  и $e^{\al}$  алгебраические степеней
соответственно $m_1$ и  $m_2$. Тогда
к многочлену $P(\al, e^{\al}) $ можно применить обобщенную теорему
Лиувилля. С помощью равенства \thetag 6 оценим его длину и степень:
$$
L(P) \le n^2n^{\ga_1n}(n!)x_0^n\sum_{s=0}^TC_T^s(n-1)^{T-s} \le
n^{\ga_3n+T}, \qquad \deg P \le n+Xn.
$$
Из обобщенной теоремы Лиувилля получаем, что
$$
|f^{(T)}(x_0\al)|=|P(\al, e^{\al})|\ge (L(P))^{1-m_1m_2}C^{-\deg P} >
n^{-\ga_4n-m_1m_2T}.
$$

С другой стороны, по лемме 3 при $X=3m_1m_2+6$ выполняется неравенство
$$
|f^{(T)}(x_0\al)| <n^{-\ga_2n^{3/2}-m_1m_2T}.
$$

Последние две оценки при достаточно большом $n$ противоречивы. Теорема
Линдемана  доказана.
\end{proof}

Об отрицательном решении проблемы квадратуры круга можно прочитать во
втором издании книги А.\,И.\,Га\-лочкина, Ю.\,В.\,Нестеренко, А.\,Б.\,Шидловского
<<Введение в теорию чисел>>.

\section{Теорема Гельфонда Шнейдера}

\subsection{Седьмая проблема Гильберта. Формулировка ТГШ и её следствия.
Построение вспомогательной функции для доказательства ТГШ, оценки ее порядка нуля}

В 1900 году Д.\,Гильберт в своем докладе на Втором международном конгрессе
математиков назвал 23 проблемы <<исследование которых может стимулировать
дальнейшее развитие науки>>. Под номером семь фигурировала проблема
трансцендентности алгебраических степеней алгебраических чисел.

Частичное решение этой проблемы было найдено А.\,О.\,Гельфондом в 1929 году и
Р.\,О.\,Кузьминым в 1930 году. Полностью ее решили независимо в 1934 году
А.\,О.\,Гельфонд и Т.\,Шнейдер.

\begin{theorem}[Гельфонда Шнейдера]
Пусть $a$  алгебраическое число,
отличное от $0$  и $1$, а $\be$   алгебраическое число,
не являющееся рациональным.  Тогда число $a^{\be}=e^{\be\ln a}$
трансцендентно.
\end{theorem}

\begin{note}
Под $\ln a$ понимается значение, взятое на любой ветви комплексного логарифма.
\end{note}

\begin{imp}Число $e^{\pi}$ трансцендентно.
\end{imp}
\begin{proof} Утверждение легко следует из равенства $(e^{\pi})^i=-1$. \end{proof}

\begin{imp}
Если $a$  и $b$  алгебраические числа,
отличные  от $0$  и $1$,  то число $\log_ab$ $=(\ln a)/(\ln b)$
либо рационально, либо трансцендентно.
\end{imp}
\begin{proof}
Утверждение следует из основного логарифмического тождества.
\end{proof}

\begin{lemma}\label{LemmaFour}
Пусть $\be\in\Z_{\A}$ и
\eqn{\label{Nine}\be^m=b_{m-1}\be^{m-1}+ \dots +b_1\be + b_0, \qquad b_j\in \Z, \quad
|b_j| \le B.}


Тогда для любой натуральной степени   числа  $\be$ справедливы утверждения:
$$
\be^t=b_{t,m-1}\be^{m-1}+\dots + b_{t,1}\be +b_{t,0}, \qquad
b_{t,j}\in\Z, \quad |b_{t,j}| \le (B+1)^t.
$$
Кроме того, если $k$  и $l$   неотрицательные целые
числа, не превосходящие $n$,  то
$$
(k+l\be)^t=B_{t,k,l,m-1}\be^{m-1}+\dots +B_{t,k,l,1}\be +B_{t,k,l,0};
\quad B_{t,k,l,j}\in \Z, \quad |B_{t,k,l,j}|\le (B+2)^tn^t.
$$
\end{lemma}

\begin{proof}
\textbf{Доказательство первого утверждения}
проводится по индукции. При $t\le m$ утверждение следует
из \eqref{Nine}. Пусть оно верно при $t$. Тогда в силу \eqref{Nine}
$$
\be^{t+1}=b_{t,m-1}(b_{m-1}\be^{m-1}+ \dots +b_0)+
b_{t,m-2}\be^{m-1}+ \dots +b_{t,0}\be,
$$
и из предположения индукции легко следует справедливость утверждения  при $t+1$.

Докажем \textbf{второе} утверждение
$$
(k+l\be)^t=\sum_{s=0}^tC_t^sk^{t-s}l^s\sum_{j=0}^{m-1}b_{sj}\be^j,
$$
откуда следует, что коэффициенты при $\be^j$ не превосходят
$$
\sum_{s=0}^tC_t^sk^{t-s}l^s(B+1)^s=(k+l(B+1))^t \le (B+2)^tn^t.
$$
Лемма доказана.
\end{proof}

\begin{lemma}
Пусть $\be$  целое алгебраическое число степени $m$.
Тогда существует такая функция
$$
f(z)=\sum_{k=0}^{n-1} \sum_{l=0}^{n-1} a_{kl} e^{(k+l\be)z}
$$
с коэффициентами $a_{kl} \in \Z$, что
%\comment
\begin{gather}
0 < \max_{k,l} |a_{kl}| < n^{\ga_5n}, \\
f^{(t)}(0)=0, \quad t= \ol {0,[n^{3/2}]-1},
\end{gather}
%\endcomment
\end{lemma}
\begin{proof}
Мы имеем:
\eqn{\label{Ten}f^{(t)}(z)=\sum_{k,l=0}^{n-1}a_{kl}(k+l\be)^te^{(k+l\be)z},}
поэтому по лемме \ref{LemmaFour}
$$
f^{(t)}(0)=\sum_{k,l=0}^{n-1}a_{kl}(k+l\be)^t=
\sum_{k,l=0}^{n-1}\sum_{s=0}^{m-1}B_{t,k,l,s}\be^sa_{kl}
$$
Приравняем к нулю коэффициенты при степенях $\be^s$. Получим систему
$$
\sum_{k,l=0}^{n-1}\sum_{s=0}^{m-1}B_{t,k,l,s}a_{kl}=0, \qquad
t=\ol{0,[n^{3/2}]-1}, \quad s=\ol{0,m-1},
$$
состоящую из $p=m[n^{3/2}]$   уравнений относительно $q=n^2$
неизвестных $a_{kl}$.  По лемме 4
$$
|B_{t,k,l,s}|<(B+2)^tn^t<n^{2n^{3/2}}=A \qquad (t<n^{3/2})
$$
и для завершения доказательства осталось применить лемму Зигеля.
\end{proof}

Пусть $X$  не зависящее от $n$ натуральное число,
которое будет выбрано в дальнейшем,
\eqn{\label{Eleven}
T=\min_{x=\ol{0,X}}\ord_{z=x \ln a}f(z)}
\begin{lemma}
$[n^{3/2}] \le \ord_{z=0}f(z) \le n^2$.
\end{lemma}
\begin{proof}
Оценка снизу следует из леммы 6. Докажем правое
неравенство. Допустим противное. Тогда
$$
f^{(t)}(0)=\sum_{k,l=0}^{n-1}a_{kl}(k+l\be)^t= 0, \quad t=\ol {0, \,n^2-1}.
$$

Получили систему из $n^2$ линейных уравнений с $n^2$  неизвестными $a_{kl}$.
Определитель системы есть определитель Вандермонда. Он отличен от нуля,
так как, ввиду иррациональности числа $\be$, все числа $k+l\be$
различны между собой. Следовательно, система может иметь лишь нулевое решение,
что противоречит лемме 5.
\end{proof}

\subsection{Оценки вспомогательной функции и завершение доказательства ТГШ}


\begin{lemma}
Справедливы неравенства
$$
|f^{(T)}(x \ln a)| < n^{-\ga_6n^{3/2}-1/3(X-6)T}, \qquad x=\ol{0,X}.
$$
\end{lemma}

\begin{proof}
Доказательство этой леммы весьма сходно с доказательством леммы 3. На этот
раз надо применить принцип максимума модуля к функции

$$
g(z)=f(z)z^{-[n^{3/2}]}(z-\ln a)^{-T} \cdots (z-X\ln a)^{-T}
$$
и положить $r=X|\ln a|+1<\sqrt n$.

Аккуратно это доказательство проведите самостоятельно (на лекции его
предполагается изложить более подробно).
\end{proof}

\begin{proof} \textbf{Доказательство теоремы Гельфонда Шнейдера.} Без ограничения общности
можно считать, что число $\be$ целое алгебраическое в противном случае
умножим его на такое натуральное число $b$,  чтобы
$b\be \in \Z_{\A}$,  докажем, что число $a^{b\be}$ трансцендентно,
и уже отсюда легко установим трансцендентность числа $a^{\be}$.

Из \eqref{Eleven} следует, что существует
такой индекс $x_0$, что $f^{(T)}(x_0\ln a) \ne 0$, причем эта производная является
многочленом $P(\be, a, a^{\be})$ с целыми коэффициентами.

Допустим, что при выполненных условиях теоремы все три числа
$\be$, $a$   и $a^{\be}$  алгебраические степеней
соответственно $m$, $m_1$ и  $m_2$. Тогда
к многочлену $P(\be, a, a^{\be}) $ можно применить обобщенную теорему
Лиувилля. С помощью равенства \thetag {10} оценим его длину и степень:
$$
L(P) \le n^2n^{\ga_5n}(2n)^T, \quad \deg P \le T+2nX.
$$
Из обобщенной теоремы Лиувилля получаем, что
$$
|f^{(T)}(x_0\ln a)|=|P(\be, a, a^{\be})|> (L(P))^{1-mm_1m_2}C^{-\deg P} >
n^{-\ga_7n-mm_1m_2T}.
$$

С другой стороны, по лемме 7 при $X=3mm_1m_2+6$ выполняется неравенство
$$
|f^{(T)}(x_0\ln a)| <n^{-\ga_6n^{3/2}-mm_1m_2T}.
$$

Последние две оценки при достаточно большом $n$ противоречивы. Теорема
доказана.
\end{proof}

\end{document}
