% -*- latex -*-
Пусть $\alpha \in \R$. Будем изучать, как $\alpha$ приближается рациональными дробями. Пусть задана $f(q) \to 0$, хотим решить неравенство
$$
  \left| \alpha - \frac{p}{q}\right| < f(q),
$$
для каждого $q$, где $p \in \Z$, $q \in \N$.

\begin{theorem}[Дирихле]
  Пусть $\alpha \in \R$. Тогда для любого $x \in \N$ найдется дробь $\frac{p}{q}$ такая, что верно
  \begin{enumerate}
    \item
      $\left| \alpha - \frac {p}{q} \right| < \frac{1}{xq}$
    \item
      $1 \leqslant q \leqslant x$
  \end{enumerate}
\end{theorem}
\begin{proof}
  Рассмотрим дробные части $\{t \alpha\}$, где $t \in \{0, \ldots, x\}$. Разобьем $[0, 1)$ на $x$ равных частей. Всего дробных частей $x + 1$, значит, по крайней мере две попадут в один из маленьких полуинтервалов.
  $$
    |\{\alpha t_2\} - \{\alpha t_1\}| < 1/x
  $$
  $$
    |\alpha (t_2 - t_1) - ([\alpha t_2] - [\alpha t_1])| < 1/x
  $$
  Обозначим $t_2 - t_1 =: q$, $[\alpha t_2] - [\alpha t_1] =: p$. А дальше заметим, что все хорошо.
\end{proof}

\begin{theorem}
  Если $\alpha \in \R \setminus \Q$, то существует бесконечное число различных дробей $\frac{p}{q}$ таких, что верно неравенство:
  $$
    \left| \alpha - \frac{p}{q}\right| < \frac{1}{q^2}.
  $$
\end{theorem}
\begin{proof}
  Пользуемся теоремой Дирихле. Там была оценка $\left| \alpha - \frac {p}{q} \right| < \frac{1}{xq} \leqslant \frac{1}{q^2}$. Осталось показать, что различных дробей бесконечно много. Докажем от противного. $\left| \alpha - \frac {p}{q} \right| < \frac{1}{xq} \leqslant \frac{1}{x}$, если дробей конечно много, то, так как $\alpha \notin \Q$, то получается, что его нельзя приблизить рациональными лучше, чем с некоторой точностью.
\end{proof}

Возникает вопрос: насколько это неравенство точное?

\begin{theorem}[Без доказательства]
  Если $\alpha \in \R \setminus \Q$, то существует бесконечное число различных дробей $\frac{p}{q}$ таких, что верно неравенство:
  $$
    \left| \alpha - \frac{p}{q}\right| < \frac{1}{q^2 \sqrt{5}}.
  $$
\end{theorem}

\begin{stm}
  Для цепной дроби из единиц оценка из последней теоремы точна.
\end{stm}

Теперь мы хотим какую-нибудь оценку снизу. Но, оказывается, что верна такая теорема:
\begin{theorem}
  Для любой $f(q) > 0$ существует $\alpha \in \R$ такое, что неравенство
  $$
    0 < \left| \alpha - \frac{p}{q} \right| < f(q)
  $$
  имеет бесконечно много решений $(p, q)$.
\end{theorem}
\begin{proof}
  Ищем $\alpha$ в виде $\displaystyle\sum_{k=0}^{\infty} 10^{-n_k}$, где $n_0 \leqslant n_1 \leqslant \ldots$~— 
  натуральные числа. Положим $n_0 = 1$. Пусть мы построили $n_0, \ldots, n_m$ такие, что
  $|\alpha - p_k / q_k| < f(q_k)$, где $q_k = 10^{n_k}$, для $k \leqslant m$.

  Возьмем $n_{m+1}$ настолько большим, чтобы $2 / 10^{n_{m+1}}$ было меньше $f(10^{n_m})$. Тогда
  $$
    0 < \alpha - \frac{p_m}{q_m} \leqslant \sum_{k=m+1}^{\infty} 10^{-n_k} \leqslant \frac{2}{10^{n_{m+1}}} < f(10^{n_m}).
  $$
\end{proof}

Легко видеть, что таких $\alpha$ можно построить несчетно много. Но, оказывается, что при достаточно быстро убывающих $f$ таких $\alpha$ будет меры ноль.

\begin{stm}
  $\left| \sqrt{2} - \frac{p}{q} \right| > \frac{1}{4q^2}$.
\end{stm}

\begin{theorem}[Лиувилля]
  Пусть $\alpha \in \R \cap \A$ степени $n \geqslant 2$ (иррациональное). Тогда существует $C = C(\alpha) > 0$ такая, что для любой дроби $p/q$ выполняется
  $$
    \left| \alpha - \frac{p}{q} \right| > \frac{C}{q^n}.
  $$
\end{theorem}
\begin{proof}
  Рассмотрим канонический многочлен для $\alpha$: $A(x) = a_n x^n + \ldots + a_0 \in \Z[x]$. Рассмотрим $A(p/q)$. 
  Это не ноль, так как, в противном случае, $A$ был бы приводимым. $q^n A(p/q) \in \Z$, $|q^n A(p/q)| \geqslant 1$. 
  Можно считать, что $|\alpha - p / q| \leqslant 1$ (иначе можно просто взять  $C < 1$).

  $$
    1 / q^n \leqslant |A(p / q)| = |A(\alpha) - A(p / q)| = |A'(\theta)| \cdot |\alpha - p / q|,
  $$
  где $\theta$ лежит между $\alpha$ и $p / q$. Отсюда следует, что $\theta$ лежит между $[\alpha - 1, \alpha + 1]$. На этом отрезке производная ограничена. Поделив на нее, получим требуемое.
\end{proof}

Теперь построим явно трансцендентное число.

\begin{df}
  Число $\alpha \in \R$ называется \emph{числом Лиувилля}, если для любого $m \in \N$ неравенство
  $$
    0 < \left| \alpha - \frac{p}{q} \right| < \frac{1}{q^m}
  $$
  имеет бесконечно много решений.
\end{df}

\begin{theorem}
  Числа Лиувилля трансцендентны.
\end{theorem}
\begin{proof}
  Очевидно следует из теоремы Лиувилля.
\end{proof}

\begin{theorem}
  Числа Лиувилля существуют.
\end{theorem}
\begin{proof}
  Мы это уже доказали на самом деле, когда строили число, которое приближается очень хорошо рациональными. Можно просто взять сверхполиномиальную $f$ и voil\`a.

  Но можно построить конкретный пример: $\alpha = \sum\limits_{n\ge0} 10^{-n!}$.
\end{proof}

\begin{stm}[теорема Рота]
  В условиях теоремы Лиувилля для любого $\ep > 0$ найдется $C = C(\alpha, \ep) > 0$ такое, что
  $$
    \left| \alpha - \frac{p}{q} \right| > \frac{C}{q^{2+\ep}}.
  $$
\end{stm}

\begin{theorem}[Бореля]
  Пусть $m > 2$~— вещественное число. Тогда мера $\alpha \in \R$ таких, что неравенство
  $$
    \left| \alpha - \frac{p}{q} \right| < \frac{1}{q^m}
  $$
  имеет бесконечно много решений, равна нулю.
\end{theorem}

\begin{theorem}[Хинчина]
  Пусть $f(q)$ положительная, монотонно стремящаяся к нулю функция.
  Если $\displaystyle\sum_{q} f(q) < \infty$, то для почти всех $\alpha$ неравенство $|\alpha - p/q| < f(q) / q$ имеет конечное число решений.
  Если $\displaystyle\sum_{q} f(q) = \infty$, то для почти всех $\alpha$ неравенство $|\alpha - p/q| < f(q) / q$ имеет бесконечное число решений.
\end{theorem}
