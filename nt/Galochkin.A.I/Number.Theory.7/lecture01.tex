\section{Асимптотический закон распределения простых }
\subsection{Делимость целых чисел}
\subsubsection{Определение делимости и основные свойства}

Работаем над $\Z$.
\begin{df}
  Пусть $b\neq0$. Говорят, что \textit{$b$ делит~$a$}, если существует~$c$ такое, что $a = bc$.
\end{df}
\begin{denote}
  $b\divs a$
\end{denote}

\textbf{Некоторые свойства:}
\begin{points}{0}
  \item $c \divs b,\;b\divs a \Rightarrow c\divs a$,
  \item $b \divs a\Rightarrow b\divs ac$,
  \item $b \divs a_1,\ldots,b \divs a_n \Rightarrow b \divs a_1\pm\ldots\pm a_n$,
  \item $b \divs a_1,\ldots,b \divs a_{n-1},b \ndivs a_n \Rightarrow b \ndivs a_1\pm\ldots\pm a_n$,
  \item $b \divs a,\;c \divs d \Rightarrow bc \divs ad$.
\end{points}

\subsubsection{Теорема о делении с остатком}

Если $b\ndivs a$, то можно говорить о делении с остатком.
\begin{theorem}[о делении с остатком]
  Если $a,b\;(b>0)$~--- целые числа, то существуют и единственные целые 
	$q,r\;(0\leqslant r<b)$ такие, что $a=bq+r$.
\end{theorem}
\begin{proof}
  $\exists$: найдется наибольшее целое $q$ такое, что $bq\le a$. Тогда
  $$
    bq\le a<b\,(q+1),
  $$
  откуда
  $$
    0\le a-bq<b.
  $$

  Обозначив $r:=a-bq$, получим требуемое разложение.

  $!:$ пусть существует второе разложение $a=bq_1+r_1,\,0\le r_1<b.$ Вычтем разложения друг из друга:
  \begin{equation}
    \label{1}
    0=b\,(q-q_1)+r-r_1,
  \end{equation}
  причем
  $$
    |r-r_1|<b.
  $$

  Но из (\ref{1}) следует, что $b\divs r-r_1,$ значит, $r-r_1=0$. Получаем $r=r_1,$ откуда $q=q_1$.
\end{proof}

\subsubsection{НОД и НОК. Теорема о представлении НОД через исходные числа}

\begin{df}
  Общим делителем двух или нескольких чисел называется число, которое делит каждое из них. 
	\notion{Наибольшим общим делителем}~--- наибольший из их общих делителей.
\end{df}
\begin{denote}
  $(a,b)$.
\end{denote}

\begin{df}
  Общим кратным двух или нескольких чисел называется число, которое делят все данные. 
	\notion{Наименьшим общим кратным}~--- наименьшее натуральное из их общих кратных.
\end{df}
\begin{denote}
  $[a,b]$.
\end{denote}

\begin{stm}
  Для любых целых $a$, $b$ существуют целые $u$, $v$ такие, что $au + bv = (a, b)$.
	\label{au+bv=d}
\end{stm}
\begin{proof}
  \textit{Первый способ.} Поднимаемся снизу вверх по алгоритму Евклида~и~получаем требуемое.

  \textit{Второй способ.} Рассмотрим
  $$
    M := \{ax+by>0,\;x,y\in \Z\}.
  $$

  Очевидно, $M$ непусто, $M\subset\N$. Значит, существует $\min\limits_{m\in M}{m}=:d,d=au+bv$ для некоторых $u,v$.
  Докажем, что $d=(a,b)$. Действительно,
  \begin{points}{0}
  	\item $d\divs a$: иначе 
			$$
				a=u_1d+v_1,
			$$
			но тогда 
			$$
				v_1=a-u_1d=a-u_1(au+bv)=ax_1+by_1\in M,
			$$
	 		но $v_1<d$~--- противоречие с минимальностью $d$ в~$M$,

  	\item $d\divs b$: аналогично,

  	\item $d_1\divs a,\;d_1\divs b\Rightarrow d_1\divs au+bv=d\Rightarrow d_1\leqslant d$~--- 
			значит, $d$~--- наибольший из общих делителей.
	\end{points}

  Таким образом, $d=(a,b)$ по определению.
\end{proof}

\begin{imp}
  Если $c\divs ab$ и $(c,a)=1$, то $c\divs b$.
\end{imp}
\begin{proof}
  Воспользуемся утверждением $\ref{au+bv=d}$ для $a$ и~$c$:
  $$
    au+cv=1,
  $$
  домножим обе части на~$b$, получим
  $$
    abu + bcv = b.
  $$
  Левая часть делится на~$c$, значит, и правая делится.
\end{proof}

\begin{imp}
  Если $b\divs a,\;c\divs a$ и $(b,c)=1$, то $bc\divs a$.
\end{imp}
\begin{proof}
  Аналогично предыдущему.
\end{proof}

\subsection{Некоторые элементарные теоремы о простых числах}
\subsubsection{Определение простых чисел и основные свойства делимости на них}       

\begin{df}
  Натуральное число $p$ называется \notion{простым}, если имеет 
	ровно два различных натуральных делителя: $1$ и $p$.
\end{df}

\begin{imp}
  \label{prime divs product}
  \begin{points}{0}
    \item Если $p\divs ab$, $p$~--- простое, то $p\divs a$ или $p\divs b$.
    \item Если $p\divs a_1\sd a_n$, $p$~--- простое, то $\exi i\colon p\divs a_i$.
    \item Если $p\divs p_1\sd p_n$, $p,p_i$~--- простые, то $\exi i\colon p=p_i$.
  \end{points}
\end{imp}
\begin{proof}
  \begin{points}{0}
    \item В силу простоты $p$ выполнено либо $(a,p)=p$, либо $(a,p)=1$. В первом случае все доказано. Во втором существуют $u$,~$v$, такие что
    $$
      au + pv = 1,
    $$
    откуда
    $$
      abu + pbv = b,
    $$
    и левая часть делится на $p$, следовательно, $p\divs b$.

    \item Индукция по $n$.

    \item Из \pt{2} $p\divs p_i$, откуда в силу простоты обоих $p = p_i$.
  \end{points}
\end{proof}

\subsubsection{Основная теорема арифметики}

\begin{theorem}[основная теорема арифметики]
  Всякое натуральное число, большее 1, можно представить в виде произведения простых, и это произведение будет единственным
  с точностью до перестановок множителей.
\end{theorem}
\begin{proof}
  $\exists:$ рассмотрим множество всех возможных разложений на множители, большие~1, числа $n \in \mathbb{N}$. Оно непусто (есть элемент~$n$).
  Выберем самое длинное из них~--- такое, очевидно, есть. Оно и будет нужным, потому как если хотя бы один из множителей не прост, то разложим его и получим еще более длинное разложение исходного числа $n$.
  \smallskip

  $!:$\,\textit{Первый способ.} Пусть
  $$
    a=p_1\sd p_n=q_1\sd q_m.
  $$
  
  В силу следствия \ref{prime divs product}
  $$
    \exi k\colon p_1=q_k.
  $$

  Без ограничения общности можем считать, что $k=1$. Сократим на равные множители. Действуя далее таким же образом, получим
  $$
    1=q_{n+1}\sd q_m,
  $$
  откуда $n=m,$ и наборы совпадают.
  \smallskip

  \textit{Второй способ.} Рассмотрим
  $$
    M:=\{a\in\mathbb{N}:a \text{ имеет хотя бы два различных разложения}\}.
  $$

  Предположим, $M$ непусто. Значит, в $M$ существует наименьший элемент $a=p_1\sd p_n=q_1\sd q_m$, и в силу минимальности
  $a\;\fa i,j\;p_i\neq q_j$. Рассмотрим
  $$
    a_1:=(p_1-q_1)\cdot p_2\sd p_n=q_1\cdot(q_2\sd q_m-p_2\sd p_n)<a.
  $$

  Разложим скобки на простые множители:
  \begin{gather*}
    p_1 - q_1 = u_1\sd u_s,\\
    q_2\sd q_m - p_2\sd p_n = v_1\sd v_r.
  \end{gather*}

  Для $a_1$ получаем
  $$
    a_1=u_1\sd u_s\cdot p_2\sd p_n=q_1\cdot v_1\sd v_r
  $$
  ~--- разложения различны, так как $q_1 \neq p_k$ и $q_1\neq u_j$, потому что в противном случае $q_1\divs p_1=u_1\ldots u_s+q_1$. Но $a_1 < a$. Противоречие с минимальностью~$a$.
\end{proof}
