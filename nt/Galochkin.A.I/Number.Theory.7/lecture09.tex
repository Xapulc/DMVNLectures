Продолжим изучение рядов Дирихле.

\begin{theorem}
  \label{series extending}
  Пусть $ f(s) = \sum\limits_{n=1}^\infty \frac{a_n}{n^s}$ в области $\sigma > \sigma_2$, $f$ аналитическая при $\sigma > \sigma_1$, где $\sigma_1 < \sigma_2 \in \R$, причем все $a_n$ неотрицательны. Тогда $f$ задается тем же рядом при $\sigma > \sigma_1$.
\end{theorem}
\begin{proof}
  Рассмотрим $\sigma_1 < \sigma < \sigma_2 < \sigma_0 \in \R$. Рассмотрим ряд Тейлора с центром в $\sigma_0$, его радиус сходимости не меньше $\sigma_0 - \sigma_1$:
  $$
    f(\sigma) = \sum_{k=0}^{\infty} \frac{f^{(k)}(\sigma_0)}{k!} (\sigma - \sigma_0)^k.
  $$
  В точке $\sigma_0$ $f$ задается рядом Дирихле, значит:
  $$
    f^{(k)}(\sigma_0) = \sum_{n=1}^{\infty} \frac{a_n (-\ln n)^k}{n^{\sigma_0}}.
  $$
  Подставляем эту формулу в предыдущую и получаем:
  $$
    f(\sigma) = \sum_{k=0}^{\infty} \sum_{n=1}^{\infty} \frac{(\sigma_0 - \sigma)^k}{k!} \cdot \frac{a_n (\ln n)^k}{n^{\sigma_0}}.
  $$
  Так как $a_n \geq 0$, то ряд сходится абсолютно, и можно менять порядок суммирования:
  $$
    f(\sigma) = \sum_{n=1}^{\infty} \frac{a_n}{n^{\sigma}} \sum_{k=0}^{\infty} \frac{(\sigma_0 - \sigma)^k (\ln n)^k}{k! n^{\sigma_0-\sigma}} = \sum_{n=1}^{\infty} \frac{a_n}{n^{\sigma}}.
  $$
  Таким образом, $f$ задается рядом в точке $\sigma$. А по теореме с предыдущей лекции(~\ref{42} ) мы знаем, что $f$ задается таким же рядом в области $\{\Re z > \sigma\}$.
\end{proof}

Вернемся к $L$-функции Дирихле. Сформулируем самую трудную часть доказательства теоремы Дирихле.
\begin{theorem}
  Для любого характера $\chi \ne \chi_0$ выполняется $L(1, \chi) \ne 0$.
\end{theorem}
\begin{proof}
  Рассмотрим два случая.
  Первый случай, когда $\chi$~— действительный характер ($\chi(n) \in \{-1, 0, 1\}$), второй случай~— когда это неверно (тогда $\chi^2 \ne \chi_0$).
  
  \begin{lemma}
    Рассмотрим \textbf{действительный} $\chi \ne \chi_0$, $f(s) = \zeta(s) L(s, \chi)$.
    \begin{enumerate}
      \item При $\sigma > 1$ $f(s)$ раскладывается в ряд Дирихле.
      \item Все $a_n$ в этом ряду неотрицательны.
      \item $a_{n^2} \geq 1$.
      \item Ряд расходится в $s = 1/2$.
    \end{enumerate}   
  \end{lemma}
  \begin{proof}
    $$
      f(s) = \sum_{k=1}^{\infty} \frac{1}{k^s} \cdot \sum_{d=1}^{\infty} \frac{\chi(d)}{d^s} = \sum_{n=1}^{\infty} \frac{a_n}{n^s},
    $$
    где
    $$
      a_n = \sum_{d \divs n} \chi(d) = \prod_{j=1}^r (1 + \chi(p_j) + \ldots + \chi(p_j)^{k_j}),
    $$
    где $n = p_1^{k_1} \ldots p_r^{k_r}$.
    Посмотрим, чему равна каждая скобка произведения. Будем пользоваться тем, что $\chi$~— вещественный.
    
    Если характер в точке $p_j$ равен $1$, то это будет $k_j + 1$, если $\chi(p_j) = 0$, то будет $1$, иначе~— $1$ или $0$ в зависимости от четности $k_j$.
    
    Из этого рассуждения нетрудно вывести первые три пункта.
    
    Теперь докажем четвертый пункт о расходимости ряда. Возьмем подряд $$
      \sum_{n=1}^{\infty} \frac{a_{n^2}}{n^{2 \cdot 1/2}} \geq \sum_{n=1}^{\infty} \frac 1n = +\infty.
    $$
  \end{proof}
  Теперь докажем теорему для случая вещественного неглавного характера. Допустим, что $L(1, \chi) = 0$.
  $L(s, \chi) = (s - 1) g(s)$, где $g$~— аналитическая в $\sigma > 0$.
  Мы знаем, что $\zeta(s) = \frac{1}{s-1} + h(s)$, $f$~— аналитическая в $\sigma > 0$. Если эти выражения формально перемножить, то получится, что
  $$
    f(s) = \zeta(s) L(s, \chi) = g(s) + (s - 1) h(s) g(s)
  $$
  является аналитической в $\sigma > 0$. Но по лемме она раскладывается в ряд Дирихле при $\sigma > 1$, но тогда по теореме \ref{series extending} она раскладывается в ряд Дирихле при $\sigma > 0$, но соответствующий ряд по лемме расходится в $1/2$. Противоречие.
  
  Второй случай: $\chi^2 \ne \chi_0$ (заметим, что $\chi^2$ тоже является характером).
  
  \begin{lemma}
    Пусть $\sigma > 1$. Тогда выполняется такое неравенство:
    $$
      A := |L^3(\sigma, \chi_0)L^4(\sigma, \chi)L(\sigma, \chi^2)| \geq 1.
    $$
  \end{lemma}
  \begin{proof}
    Будем пользоваться ранее доказанным неравенством
    $$
      |(1-r)^3(1-r e^{i \varphi})^4 (1 - r e^{2i \varphi})| \leq 1
    $$
    при $0 < r < 1$, $\varphi \in \R$.
    
    При $\sigma > 1$
    $$
      L(s, \chi) = \prod_p \left( 1 - \frac{\chi(p)}{p^s}\right)^{-1}.
    $$
    
    Таким образом,
    $$
    A = \prod_{p \ndivs n} |(1 - \frac{1}{p^{\sigma}})^3(1 - \frac{\chi(p)}{p^{\sigma}})^4 (1 - \frac{\chi^2(p)}{p^{\sigma}})|^{-1}
    $$
    Теперь по неравенству получаем сразу $A \geq 1$, положив $r := 1 / p^{\sigma}$.
  \end{proof}
  
  Теперь докажем теорему для второго случая. Предположим, что $L(1, \chi) = 0$. $0 < L(\sigma, \chi_0) < \zeta(\sigma) < \frac{\sigma}{\sigma - 1}$. Если $1 < \sigma \leq 2$, то $0 < L(\sigma, \chi_0) < \frac{2}{\sigma - 1}$.
  
  $L(s, \chi^2)$ аналитическая при $\sigma > 0$ ($\chi^2 \ne \chi_0$), значит $|L(\sigma, \chi^2)|$ ограничен при $1 < \sigma \leq 2$.
  
  $L'(1, \chi) = \lim_{\sigma \to 1 + 0} \frac{L(\sigma, \chi)}{\sigma - 1}$. Значит, $|L(1, \chi) / (\sigma - 1)|$ ограничен при $1 < \sigma \leq 2$.
  
  Итак, получаем, что $1 \leq A \leq C(\sigma - 1)$ для некоторой константы $C$. Противоречие (если устремить $\sigma \to 1 + 0$).
\end{proof}

\begin{theorem}[Дирихле]
  Пусть $m, l \in \N$, $(m, l) = 1$, $m > 1$. Тогда существует бесконечно много простых вида $mx + l$, где $x \in \N$.
\end{theorem}
\begin{proof}
  В доказательстве будем рассматривать характеры по модулю $m$.
  Сравнение $lu \equiv 1 \pmod m$ разрешимо. Рассмотрим функцию
  $$
    g(s) = \sum_{\chi} \chi(u) \cdot \left( -\frac{L'(s, \chi)}{L(s, \chi)}\right) = \chi_0(u) \cdot \left( -\frac{L'(s, \chi_0)}{L(s, \chi_0)}\right) + \sum_{\chi \ne \chi_0} \chi(u) \cdot \left( -\frac{L'(s, \chi)}{L(s, \chi)}\right)
  $$
  $\chi_0(u) = 1$, при $\sigma > 0$ $L(s, \chi_0) = \frac{a_m}{s - 1} + f_m(s)$, где $f_m$~— аналитическая. Получаем:
  $$
    -\frac{L'(s, \chi_0)}{L(s, \chi_0)} = \frac{1}{s - 1} + f(s),
  $$
  где $f$~— аналитическая в окрестности $s = 1$.
  
  Остальные члены суммы являются аналитическими в окрестности $s = 1$, так как $L(s, \chi) \ne 0$ при $\chi \ne \chi_0$.
  
  Итого:
  $$
    g(s) = \frac{1}{s - 1} + h(s),
  $$
  где $h$~— аналитическая в окрестности $s = 1$.
  
  Получаем, что $g(s)$ не ограничена при $s \in (1, 2]$.
  
  Пусть простых вида $mx + l$ конечное число. Сведем это к противоречию, выведя отсюда, что $g$ ограничена при $s \in (1, 2]$.
  
  $$
    g(s) = \sum_{\chi} \chi(u) \sum_{n = 1}^{\infty} \frac{\Lambda(n) \chi(n)}{n^s} = \sum_{n = 1}^{\infty}\frac{\Lambda(n)}{n^s} \sum_{\chi} \chi(nu) = \varphi(m) \sum_{n = 1}^{\infty}\frac{\Lambda(n)}{n^s} I\{n \equiv l \pmod m\}.
  $$
  Разобьем сумму на две: $n$~— простое или нет.
  $$
    g(s) = \varphi(m) \sum_{p \equiv l \pmod m} \frac{\ln p}{p^s} + R_m(s)
  $$
  Оценим $R_m(s)$.
  $$
    0 < R_m(s) = \varphi(m) \sum_p \sum_{p^k \equiv l \pmod m, k \geq 2} \frac{\ln p}{p^{ks}} = O(1).
  $$
  Итак, если в прогрессии конечное множество простых, то $g(s)$ ограничена при $s \in (1, 2]$. Противоречие!
\end{proof}

\begin{stm}
  $\sum_{p \equiv l \pmod m} \frac{\ln p}{p} = +\infty$.
\end{stm}

\begin{theorem}
  $\pi(m, l, x) \sim \frac{x}{\varphi(m) \cdot \ln x}$ ($\pi(m, l, x)$~— количество простых в прогрессии, которые не превосходят $x$).
\end{theorem}
