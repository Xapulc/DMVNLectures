\begin{theorem}[Бореля]
  Пусть $\delta > 0$. Рассмотрим множество
  $$
    M = \left\{\alpha \in \R : \mbox{$\left| \alpha - \frac{p}{q} \right| < \frac{1}{q^{2+\delta}}$ имеет бесконечное число решений}\right\}.
  $$
  Тогда мера $M$ равна нулю.
\end{theorem}
\begin{proof}
  Докажем, что мера $M \cap [0, 1]$ равна нулю (очевидно, из этого будет следовать общая теорема~— дроби можно подвинуть). Обозначим $M := M \cap [0, 1]$.
  
  Рассмотрим натуральное $Q$. Рассмотрим
  $$
    M_Q := \left\{\alpha \in \R : \mbox{$\left| \alpha - \frac{p}{q} \right| < \frac{1}{q^{2+\delta}}$ имеет хотя бы одно решение с $q \geqslant Q$}\right\}.
  $$
  Очевидно, что $M \subset M_Q$ для любого $Q$. Если $\alpha \in M_Q$, то $\alpha \in \left( \frac{p}{q} - \frac{1}{q^{2 + \delta}}, \frac{p}{q} + \frac{1}{q^{2 + \delta}}\right) =: I(p, q)$. Видно, что $M_Q \subset \bigcup_{q=Q}^{\infty} \bigcup_{p=0}^{q} I(p, q)$.
  
  Посчитаем меру $M_Q$.
  
  $$
    \mu(M_Q) \leqslant \sum_{q=Q}^{\infty} \sum_{p=0}^{q} \frac{2}{q^{2+\delta}} \leqslant \sum_{q=Q}^{\infty}\frac{4}{q^{1+\delta}},
  $$
  
  что стремится к нулю при $Q \to \infty$. Таким образом, $\mu(M) \leqslant \lim_{Q \to \infty} \mu(M_Q) = 0$.
\end{proof}

Теперь займемся рассмотрением конкретных чисел. Видно, что, чтобы доказывать трансцендентность чисел, очень редко получится пользоваться теоремой Лиувилля.

\begin{theorem}
  Число $e$ иррационально.
\end{theorem}
\begin{proof}
  $$
  e = \sum_{k=0}^{\infty} \frac{1}{k!}
  $$
  Предположим, что $e = m / n$. Тогда $n! e \in \N$.
  $$
    n! e = n! \sum_{k=0}^{n} \frac{1}{k!} + R_n.
  $$
  $$
    0 < R_n = \frac{1}{n+1} + \frac{1}{(n+1)(n+2)} + \ldots < \frac{1}{n+1} + \frac{1}{(n+1)^2} + \ldots = \frac{1}{n} \leqslant 1
  $$
  Противоречие.
\end{proof}

\begin{theorem}
  Число $e$ трансцендентно.
\end{theorem}
\begin{proof}
  Пусть $f(x) \in \Q[x]$. Рассмотрим
  $$
    M = \int_0^{+\infty} f(x) e^{-x} dx.
  $$
  Напомним, что
  $$
    \Gamma(n) = \int_0^{+\infty} x^{n-1}e^{-x} dx = (n - 1)!.
  $$
  Возьмем некоторое натуральное $k$.
  $$
    M = \int_k^{+\infty} f(x) e^{-x} dx + \int_0^{k} f(x) e^{-x} dx =: I_k + \int_0^{k} f(x) e^{-x} dx.
  $$
  $$
    I_k = e^{-k} \int_0^{\infty} f(y + k) e^{-y} dy
  $$
  Если обозначить
  $$
    M_k := \int_0^{\infty} f(x + k) e^{-x} dx,
  $$
  то видно, что верна формула
  $$
    M_0e^{k} - M_k = \ep_k,
  $$
  где
  $$
    \ep_k = e^{k} \int_0^{k} f(x) e^{-x} dx.
  $$
    Допустим, что $e$ алгебраическое. Тогда выполняется соотношение
  $$
    a_n e^{m} + \ldots + a_0 = 0,
  $$
  где $a_k \in \Z$, $a_m, a_0 \ne 0$.
  
  $$
    M_0 \sum_{k=0}^{m} a_k e^k - \sum_{k=0}^{m} a_k M_k = \sum_{k=0}^{m} a_k \ep_k
  $$
  
  Первое слагаемое в левой части по предположению равно нулю.
  
  $$
    \sum_{k=0}^{m} a_k M_k = -\sum_{k=0}^{m} a_k \ep_k
  $$
  
  Теперь выберем $f$ (которое входит под интегралы).
  
  $$
    f(x) = \frac{1}{(n-1)!} x^{n-1}((x-1)\ldots(x-m))^n,
  $$
  где $n$~—a параметр.
  
  Идея дальнейшего доказательства: покажем, что $\sum_{k=0}^{m} a_k M_k$ целое ненулевое число, $-\sum_{k=0}^{m} a_k \ep_k$ по модулю меньше единицы.
  
  $$
    f(x) = \frac{1}{(n-1)!}((-1)^{mn}(m!)^n x^{n-1} + B_n x^n + \ldots + B_N x^N)
  $$
  
  Получаем, что
  $$
    M_0 = \int_0^{\infty} f(x) e^{-x} dx = (-1)^{mn}(m!)^n + A_0 n.
  $$
  
  $$
    f(x + k) (n - 1)! \mbox{ делится на $x^n$}
  $$
  
  Получаем, что
  $$
    M_k\mbox{ делится на $n$} (=A_k n).
  $$
  
  Отсюда видно, что
  $$
    \sum_{k=0}^{m} a_k M_k
  $$
  целое число.

  $\ep_k \to 0$ при $n \to \infty$, $\sum_{k=0}^{m} a_k \ep_k \to 0$, значит, выбрав $n$ достаточно большое, можно сделать эту сумму очень маленькой. Выберем также $n$ взаимно простым с $a_0 m!$. Тогда $\sum_{k=0}^{m} a_k M_k \ne 0$.
\end{proof}

Теперь давайте докажем иррациональность $\pi$.

\begin{theorem}
  Число $\pi$ иррационально.
\end{theorem}
\begin{proof}
  Рассмотрим
  $$
    I_n = \int_0^{\pi} f(x) \sin x dx,
  $$
  где $f \in \Q[x]$.
  Обозначим $F(x) = f(x) - f''(x) + f^{(4)}(x) - \ldots$. Тогда первообразная $f(x) \sin x$ равна $F'(x) \sin x - F(x) \cos x$. Отсюда $I = F(0) + F(\pi)$. Если $f(x) = f(\pi - x)$, то $F(0) = F(\pi)$ и $I = 2F(0)$.
  
  Пусть $\pi = a / b$, $a, b \in \N$. Выберем $f_n(x) = \frac{1}{n!} x^n (\pi - x)^n b^n = \frac{1}{n!} g_n(x)$, где $g_n \in \Z[x]$.
  
  Рассмотрим $I_n = \frac{1}{n!} \int_0^{\pi} x^n (\pi - x)^n b^n \sin x dx$. Докажем некоторые свойства:
  \begin{enumerate}
    \item $I_n > 0$,
    \item $I_n \to 0$ при $n \to \infty$,
    \item $I_n \in \Z$.
  \end{enumerate}
  Первые два свойства очевидны, докажем третье.
  $I_n = 2 \sum_{k=0}^{\infty} (-1)^k f^{(2k)}(0) = 2 \sum_{2k \geqslant n} (-1)^k f^{(2k)}(0)$. Видно, что $g_n^{(l)}(x) / n! \in \Z[x]$ при $l \geqslant n$.
  Значит $I_n \in \Z$. Но видно, что три доказанные свойства несовместны. Противоречие.
\end{proof}

Осталось доказать трансцендентность $\pi$. Будем доказывать, что для $\alpha \in \A$ $e^{\alpha}$ является трансцендентным.
