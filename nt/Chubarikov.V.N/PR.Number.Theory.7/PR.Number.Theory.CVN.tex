\documentclass{article}
\usepackage[simple,utf]{dmvn}

\begin{document}

\begin{problem}
Решить уравнение
$$\hs{\log_2[x]} = \hs{\log_2 x}.$$
\end{problem}

\begin{problem}

\begin{petit}
Очень странная задача, возможно формулировка неточна
\end{petit}

Доказать, что если $n = \sums{k} \ep_k p^k$ в $p$-ричной системе,
где $0 \le \ep_k p - 1$,
$$
    p^{\nu_p(n!)} || n!,
$$
и
$$
    S_n = \sums{k} \ep_k,
$$
тогда
$$
    \nu_p(n!) = \frac{n - S_n}{p - 1}.
$$
\end{problem}

\begin{problem}
Доказать, что
$$
    \sums{n \le x} \frac1{\ph(n)} = O\br{\ln(x)},
$$
где $\ph(x)$~--- функция Эйлера.
\end{problem}

\begin{solution}
Оцениваем сумму $\sums{n\le x}\frac1{\ph(n)}$. Везде $n,m\in\N, x\in\R, x\ge2, p$ простое.
$$
\ph(n)=n\prod_{p|n}\left(1-\frac1p\right)
$$
$$
\frac1{\ph(n)}=\frac1n\prod_{p|n}\left(1-\frac1p\right)^{-1}=\frac1n\prod_{p|n}
\left(1+\frac1p+\frac1{p^2}+\dots\right)=\frac1n\sum_m\frac1m,
$$
где сумма берётся по всем $m$, у которых в разложении на простые встречаются только
сомножители из разложения $n$. Теперь навешиваем сумму по $n\le x$ и переставляем суммирование
по $n$ и по $m$ (сумма конечного числа абсолютно сходящихся рядов):
\begin{equation}
\sums{n\le x}\frac1{\ph(n)}=\sum_
{m=p_1^{\alpha_1}\dots p_s^{\alpha_s}}\frac1m\sum_
{\substack{n\le x\\p_1\dots p_s|n}}\frac1n
\end{equation}
$$
\sum_{\substack{n\le x\\k|n}}\frac1n=\sum_{n\le x/k}\frac1{kn}
$$
Поскольку $\sums{n\le x}\frac1n<\ln x+\ga+\frac1{2x}<\ln x+1$, то
$$
\sum_{\substack{n\le x\\k|n}}\frac1n<\frac1k\left(\ln\frac xk+1\right)\le
\frac{\ln x+1}k
$$
При фиксированном наборе $\{p_1,\dots,p_s\}$ можно посчитать сумму по $\alpha_i$ (от 1 до
$\infty$) от $\frac1m=p_1^{-\alpha_1}\dots p_s^{-\alpha_s}$, получится
$$
\frac1{(p_1-1)\dots(p_s-1)}
$$
Следовательно,
$$
\sums{n\le x}\frac1{\ph(n)}<\sum_{p_1\dots p_s\le x}\frac1{(p_1-1)\dots(p_s-1)}
\frac{\ln x+1}{p_1\dots p_s}
$$
Для $p\ge3$ выполнено неравенство $p-1>\sqrt{p}$. Для $p=2$ выполнено
неравенство $p-1>\sqrt{p}/2$.
$$
\sums{n\le x}\frac1{\ph(n)}<\sum_{p_1\dots p_s\le x}\frac2{\sqrt{p_1}\dots\sqrt{p_s}}
\frac{\ln x+1}{p_1\dots p_s}<
(\ln x+1)\sum_{k=1}^\infty \frac2{k^{3/2}}=O(\ln x).
$$
Утверждение доказано.
\end{solution}

\begin{note} Если использовать равенство $\sums{n\le x}\frac1n=\ln x+\ga+O(\frac1x)$ и
аккуратно всё расписать, то с использованием тех же идей нетрудно получить асимптотическую
формулу
$$
\sums{n\le x}\frac1{\ph(n)} = C_1\ln x+C_2+O\left(\frac1{x^{1-\varepsilon}}\right),
$$
где $C_1$ и $C_2$~--- абсолютные постоянные, а $\ep>0$ сколь угодно мало.
$$
C_1=\sum_{\substack{s\ge0\\p_1<\dots<p_s}}\frac1{p_1\dots p_s(p_1-1)\dots(p_s-1)}=
\prod_p\left(1+\frac1{p(p-1)}\right)=1.943596\dots
$$
$$
C_2=C_1\ga-\sum_{\substack{s\ge0\\p_1<\dots<p_s}}\frac{\ln(p_1\dots p_s)}{p_1\dots p_s
(p_1-1)\dots(p_s-1)}=-0.060574\dots
$$
\end{note}
\end{document}
