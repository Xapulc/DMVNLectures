\documentclass[a4paper]{article}
\usepackage[utf]{dmvn}
\usepackage{amscd}
%\usepackage[dvips]{graphicx}

\DeclareMathOperator{\Span}{span}

\newcommand{\dn}{\,d\nu}
\newcommand{\nsb}{n\rightarrow \infty} %N Stremitsya k Beskone4nosti
\newcommand{\fo}{f(\omega)}
\newcommand{\dd} {\stackrel{\mathrm{def}}{=}}
\newcommand{\io} {\int \limits_{\Omega}}
\newcommand{\lin} {\lim \limits_{\nsb} }
\newcommand{\asn} {a\rightarrow 0}
\newcommand{\ii} {\io \fo \, d\nu}
\newcommand{\iin} {\io f_n(\om) \, d\nu}
\newcommand{\fn}{ f_n(\om) }
\newcommand{\sus}{\sum \limits_{n=1}^{\infty} }
\newcommand{\fd} { f(x)\, dx}
\newcommand{\iz} { \int \limits_0^1 }
\newcommand{\Th} {\As}
\newcommand{\jkl}{ l_\infty}
\newcommand{\ff} {\varphi}
\newcommand{\owff}{\overline{\widehat{\varphi}}}
\newcommand{\woff}{\widehat{\overline{\varphi}}}
\newcommand{\coff}{\check{\widehat{\varphi}}}
\newcommand{\cw}{\check{\widehat{\delta(x+a)}}}
\newcommand{\mro} {\R^1}
\newcommand{\ir} {\int \limits_{-\infty}^{\infty}}
\newcommand{\inr} {\int \limits_{\mathbb R}}
\newcommand{\il} {\int\limits_}
\newcommand{\Llra} {\Longleftrightarrow}
\newcommand{\lo}{l_p}
\newcommand{\dw}{\delta^{\prime}}
\newcommand{\w}{\widehat}
\newcommand{\wphi}{\widehat{\varphi}}
\newcommand{\fw}{\widehat{f}}
\newcommand{\gn}{g_{*}}
\newcommand{\gv}{g^{*}}
\newcommand{\RR}{\R^1}
\newcommand{\Rr}{\R}
\newcommand{\uu}{ \upharpoonright}
\newcommand{\pr}{\mathop{\mathrm{pr}}\nolimits}
\newcommand{\rr}{\pr_{\mbox{$_{H_1}$}} x}
\newcommand{\pp}{\pr_{\mbox{$_{H_1}$}} a}
\newcommand{\ap}{\alpha_{\psi}}
\newcommand{\asd}{\alpha\in \mathbb {\mathfrak a}}
\newcommand{\qwe}{\sup\limits_{\|x\|\, \leqslant 1} \|Ax\|}
\newcommand{\sxo}{\sup\limits_{\|x\|\, \leqslant 1}}
\newcommand{\sxpo}{\sup\limits_{\|x\| = 1}}
\newcommand{\spec}{\mathop{\mathrm{spec}}\nolimits}
\newcommand{\ess}{\mathop{\mathrm{ess sup}}}

\pagestyle{plain}
\begin{document}
\begin{titlepage}
\topskip=6cm
\begin{center}
\Large \textbf{ЛЕКЦИИ ПО ФУНКЦИОНАЛЬНОМУ АНАЛИЗУ\\
ОЛЕГА ГЕОРГИЕВИЧА СМОЛЯНОВА\\}\vspace{1cm}
\large \textbf{ДЛЯ СТУДЕНТОВ 3 КУРСА 6 СЕМЕСТРА} \vspace{3cm}
\end{center}
\large \raggedleft \slshape \makebox[4.7cm][s]{Записано}\\
\makebox[4.7cm][s]{Яхонтовой Валерией,}\\
\makebox[4.7cm][s]{Грицук Светланой,}\\ \makebox[4.7cm][s]{Голяковой Алевтиной}\\
\makebox[4.7cm][s]{ и Приходько Игорем} \vspace{7cm} \normalfont
\begin{center}
\large 2005 год
\end{center}
\end{titlepage}
\makeatletter
\renewcommand{\@oddhead}{\leftmark\hfil}
\renewcommand{\@evenhead}{\leftmark\hfil}
\renewcommand{\@oddfoot}{\hfil\thepage\hfil}
\renewcommand{\textit}{\textsl}
\makeatother
\renewcommand{\sectionmark}[1]{\markboth{\sffamily \thesection\hspace{1em}#1}{}}

\section{Лекция}
%
Пусть $\Om=\R^1$ и
$$
\Ps_0=\{[a,b),a\in \R^1, b\in \R^1,
a\leqslant b\}\cup\{\varnothing\}, \mbox{ так что $\Ps_0$
- полукольцо}.
$$
(здесь и далее $(a,a] = (a,a)= [a,a)= \varnothing$).

Пусть еще
$$
\Ps_1=\{[a,b), [a,b], (a,b], (a,b), \{a\}, a\leqslant b
\},
$$
так что $\Ps_1$ - также полукольцо, причем
$\Ps_1 \supset \Ps_0.$

Мера $\nu_{L}$ на $\Ps_1$ определяется так:
$$
\nu_L ([a, b)) = \nu_L ([a, b]) = \nu_L ((a, b)) = \nu_L ((a, b])
= b - a.
$$

Сужение $\nu_L$ на $\Ps_0$ обозначается тем же символом.
Непосредственно проверяется, что это определение корректно, то
есть функция $\nu_L$ конечно аддитивна на $\Ps_1$ (а тем
самым и на $\Ps_0$).

\begin{theorem}
$\nu_L$ является счетно-аддитивной мерой на полукольце
$\Ps_0$.
\end{theorem}
\begin{proof}
Нужно доказать, что, $$\mbox{если }\ [a, b)\  \subset \
\bigcup\limits_{i=1}^\infty [a_i, b_i),\ \mbox{ то } \ \nu
\bigl([a,b) \bigr) \leqslant\sum_{i=1}^\infty \,\bigl( [a_i,b_i)
\bigr).
$$

$$\epsfbox{picture.1}$$
$$\epsfbox{picture.2}$$

Зададимся $\varepsilon>0$, тогда для фиксированного $\varepsilon$,
выполнено следующее условие:
$$
\exists c \quad  a<c<b:\quad \varepsilon + \nu[a,c] > \nu[a,b)
$$
$$
\forall \ i \in N \quad \exists \ d_i: d_i < a_i < b_i \quad
\mbox{ и }\quad \nu \bigl((d_i, b_i)\bigr) - \nu \bigl([a_i,
b_i)\bigr) < \frac{\varepsilon}{2^i}.
$$

Мы получили, что
$$
[a,c] \subset [a,b) \subset \bigcup_{i=1}^{\infty} \,(d_i, b_i)
$$
То есть компакт $[a, c]$ покрыт открытыми множествами, поэтому
существует такое $n\in N,$ что $$[a, c]\  \subset \
\bigcup_{i=1}^{n} (d_i, b_i).$$

Так как $\nu_L$ конечно аддитивна на $\Ps_1,$ отсюда
следует, что
$$
\nu ([a, c]) \leqslant \sum\limits_{i=1}^n \nu \bigl( (d_i, b_i)
\bigr).
$$
Поэтому
$$
\nu ([a, b)) - \varepsilon \ < \ \sum\limits_{i=1}^n \Bigl( \nu
\bigl([a_i, b_i)\bigr) + \frac{\varepsilon}{2^i}\Bigr)\ < \
\sum\limits_{i=1}^\infty \Bigl( \nu \bigl([a_i, b_i)\bigr) +
\frac{\varepsilon}{2^i}\Bigr) \ = \ \sum\limits_{i=1}^\infty \nu
\bigl([a_i, b_i)\bigr) + \varepsilon.
$$

Так как $\varepsilon>0$ произвольно, это означает, что
$$
\nu  \bigl([a, b)\bigr)\ \leqslant\ \sum\limits_{i=1}^\infty
\Bigl( \nu \bigl([a_i, b_i)\bigr).
$$
Теорема доказана.

\end{proof}

\begin{df}
Мерой Лебега называется (единственное) счетно аддитивное
продолжение $\nu_L$ на $\sigma$-алгебру $\As_L\ $
$\nu_L$-измеримых подмножеств. Это продолжение будет обозначаться
тем же символом $\nu_L$.
\end{df}
\bigskip

\subsection{Измеримые функции на $\R^1$.}

\begin{df} Функция $f: \R^1 \rightarrow \R^1$
является измеримой, если она измерима, как отображение:
$$f:
\bigl(\R^1,\As_L) \rightarrow
(\R^1,B(\R^1) \bigr),
$$
то есть $\forall a \in \R^1$ множество $\{x: f(x) < a \}
\in \As_L,$ то есть измеримо по Лебегу.
\end{df}
\begin{problem}
Показать, что всякая непрерывная функция измерима.
\end{problem}


Если функция $f$ - непрерывна, то множество $\{x : \ f(x)<a \}$ -
открыто. Значит,
$$
\{ x : \ f(x)<a\} \in B(\R^1).
$$
Так как
$B(\R) \subset \As_L$, то $\{x :f(x)<a \} \in \As_L.$\\

\begin{problem}
Доказать, что любая  функция $f$ на $\R^1,$ интегрируемая
по Риману, измерима.
\end{problem}

Далее для каждого множества $A$ символ $\gamma_A$ обозначает его
индикатор.

Пусть $\Om$  - некоторое множество. Напомним, что функция на $\Om$
называется простой, если множество ее значений конечно. Отсюда
следует, что множество всех простых функций образует алгебру (то
есть (конечная) линейная комбинация простых функций и произведение
простых функций - снова простая функция).

\begin{prop}
Функция является простой тогда и только тогда, когда она является
(конечной) линейной комбинацией индикаторов каких-то множеств.
\end{prop}

\begin{proof}

Так как индикатор (всякого) множества - это простая функция, то и
их конечная линейная комбинация - также простая функция.

С другой стороны, если $f$ - простая функция и $\{a_1, a_2,
\ldots, a_n\}$ - множество ее значений (так что $a_i \not = a_j,
\mbox{ если } i \not = j$), то
$$
f(x)\ = \ \sum\limits_{j=1}^n a_j \ \gamma_{\{z\in \Om\colon f(z)\
= \ a_i\}} (x).
$$

\end{proof}
\begin{note}
Если $i \not = j,$ то
$$
\{z\in \Om\colon \ f(z) = a_j\} \cap \{z\in \Om\colon\ f(z) =
a_j\}\ = \ \varnothing;
$$
таким образом, из доказательства предложения 1 вытекает, что
всякая простая функция является линейной комбинацией индикаторов
попарно непересекающихся множеств.
\end{note}

\begin{prop}
Произвольная функция $f: \Om\rightarrow \R^1$ является
поточеченым пределом последовательности $f_n$ простых функций, то
есть
$$
\forall f\colon \, \Om \rightarrow \R^1  \quad \exists
\mbox{ последовательность } \{f_n\} \mbox{ такая, что } \forall n
\ \ f_n \mbox{ - простая функция,}
$$
причем $\forall \om \in \Om \quad f_n(\om) \rightarrow \fo$. Если
$f$ - измерима, то все $f_n$ могут быть выбраны измеримыми.
\end{prop}

$$\epsfbox{picture.3}$$

\begin{proof}
Для каждого натурального $n$  рассмотрим разбиение отрезка оси
ординат от 0 до $n$ на полуинтервалы длины $\frac{1}{2^n}$.

Таким образом, число отрезков разбиения равно $2 n \cdot 2^n$.
Определим функцию $\ff_k$ равенством:
$$
\ff_k (\om) = k \cdot 2^{-n}  \cdot   \gamma_{\{w:\  k \cdot \
2^{-n} \ \leqslant \  f(\om)\  <\  (k+1) \cdot \  2^{-n}\}}.
$$
Положим
$$
f_n(\om)=\sum_{k=-n \cdot \  2^n}^{n \cdot \ 2^n} \ff_k(\om) +
(n+1) \cdot \  \gamma_{\{w:\  n \leqslant \ f(\om)\}} \  - \ n
\cdot \ 2^{-n} \cdot  \gamma_{\{w:\ f(\om)< \ n\}}.
$$

В качестве упражнения предлагается проверить, что
$$
\forall \om \in \Om \quad f_n(\om)\rightarrow \fo.
$$
\end{proof}

\begin{note}
Если $\forall \om \in \Om \ \ \fo \geqslant 0, $ то построенная
выше последовательность $f_n$ сходится к $\fo$ не убывая.
Доказательство этого замечания проведите самостоятельно.
\end{note}

Пусть $(\Om, \As)$ - измеримое пространство.

\begin{note}
Напомним, что индикатор множества $A\subset \Om$ измерим в
точности тогда, когда множество $A$ измеримо (то есть когда $A\in
\As$).

Действительно, если $A\subset \Om$ и функция $\gamma_A$ измерима,
то множество $A$ измеримо, так как
$$
A\ = \ f^{-1} \bigl( \{ 1\}\bigr).
$$

С другой стороны, если множество $A$ измеримо, то функция
$\gamma_A$ измерима, так как:
$$
\forall c\in \R^1, \quad \{\om \in \Om\colon \ f(w) < c\}\
= \ \begin{cases}
 \varnothing, & \text{если $c \leqslant 0$};\\
 \Om\backslash A, & \text{если $0 < c \leqslant 1$};\\
 \Om, & \text{если $c > 1$},\\
\end{cases} \quad\quad
$$
причем множества $A$ и $\Om\backslash A$ измеримы или нет
одновременно.
\end{note}

\begin{prop}
Простая функция $f$ измерима в точности тогда, когда она является
линейной комбинацией индикаторов попарно непересекающихся
измеримых множеств.
\end{prop}
\begin{proof}
Действительно, из последнего замечания следует, что индикаторы
измеримых множеств, а следовательно, и их линейные комбинации
измеримы.

Обратное утверждение вытекает из первого замечания к предложению
2, так как множества
$$
\{z\in \Om\colon \ f(z) = a_i\}\ = \ f^{-1} (\{a_i\}),
$$
о которых говорится в этом замечании, измеримы.

\end{proof}

\begin{note}
Следующее утверждение НЕверно:

если $\Om = \bigsqcup \limits_{n=1}^m \Om_n$ и функция $f,$
определяемая равенством:
$$
\fo = \sum \limits_{j=1}^m \lambda_j \cdot \gamma_{\Om_j}(\om)
\mbox{ - измерима,}
$$
то все $f_k(\om) \mbox{ - измеримы}.$
\end{note}

\begin{proof}
Приведем такой контрпример:

$$\epsfbox{picture.4}$$
$$
\mbox{ Пусть }\  \Om = \Om_1 \sqcup \Om_2,\  \mbox{ тогда } \fo
\equiv 1 = \gamma_{\Om_1}(\om) + \gamma_{\Om_2}(\om),
$$

но функции $\gamma_{\Om_1}$ и $\gamma_{\Om_2}$ не будут измеримы,
если одно из множеств $\Om_1, \, \Om_2$ (конечно, тогда и другое)
неизмеримо.
\end{proof}

\subsection {Интеграл Лебега.}

Всюду далее $(\Om, \As, \nu)$ - пространство с мерой.

\begin{df}
Пусть $f$ - простая неотрицательная измеримая функция, так что
$$
\fo=\sum_{i=1}^n  \lambda_i \cdot \gamma_{\Om_i} (\om), \quad
\bigsqcup \limits_{j=1}^n \Om_j = \Om, \quad \lambda_j \geqslant
0, \quad \Om_j \in \As.
$$

Интеграл Лебега от функции $f$ по множеству $\Om,$ обозначаемый
одним из символов $\ \io f(\om) \nu (d\om), \ \io f(\om) \, d\nu,\
$ определяется так:
$$
\int \limits_{\Om} \fo \nu( \, d\om)= \int \limits_{\Om} \fo
\,d{\nu} \dd \sum \limits_{i=1}^{n} \lambda_i \cdot \nu \{ \om:
\fo= \lambda_i\} = \sum_{i=1}^n \lambda_i \cdot \nu(\Om_i).
$$
\end{df}
Проверим корректность определения, то есть независимость значения
интеграла от выбора семейства (измеримых) множеств $\Om_j,$ для
которого справедливо равенство:
$$
f(\om) = \sum\limits_{i=1}^{n} \la_i \gamma_{\Om_i} (\om).
$$


Пусть
$$
\fo \stackrel{(1)}= \sum \limits_{j=1}^n {a_j \gamma_{A_j} (\om)}
\qquad \mbox{ и } \qquad \fo \stackrel{(2)}= \sum\limits_{k=1}^m
{b_k \gamma_{B_k} (\om)};
$$
нам нужно показать, что
$$
\sum\limits_{i=1}^{n} a_j \, \nu (A_j)\ = \ \sum\limits_{k=1}^{m}
b_k \, \nu (B_k).
$$
Положим $C_{jk}=A_j\bigcap B_k$, тогда
$$
\fo \stackrel{(3)}= \sum\limits_{j,k}{c_{jk} \cdot \gamma_{C_{jk}}
(\om)}, \ \ \mbox{ где } c_{jk} = b_k = a_j,
$$
так как, если $\om \in \ A_j \cap B_k,$ то одновременно $f(\om) =
b_k = a_j.$

Поэтому, с одной стороны,
$$
\sum\limits_{j, k} c_{jk} \, \nu (A_j \cap B_k) \ = \
\sum\limits_{i=1}^n \sum\limits_{k=1}^{m} c_{jk} \, \nu (A_j \cap
B_k) \ = \ \sum\limits_{i=1}^n \sum\limits_{k=1}^{m} a_j \, \nu
(A_j \cap B_k)\ = \ \sum\limits_{j=1}^n a_j \, \nu (A_j);
$$

С другой стороны,
$$
\sum\limits_{j, k} c_{jk} \, \nu (A_j \cap B_k) \ = \
\sum\limits_{k=1}^{m} \sum\limits_{i=1}^n c_{jk} \, \nu (A_j \cap
B_k) \ = \ \sum\limits_{k=1}^{m} \sum\limits_{i=1}^n b_k \, \nu
(A_j \cap B_k)\ = \ \sum\limits_{k=1}^m b_k (B_k),
$$
так что $\sum\limits_{i=1}^{n} a_j \, \nu (A_j)\ = \
\sum\limits_{k=1}^{m} b_k \, \nu (B_k).$

\begin{df}
Пусть $f$ - неотрицательная измеримая функция и $f_n$ -
неубывающая последовательность неотрицательных простых функций,
причем $\forall \om \quad \fo=\lim \limits_{n\rightarrow \infty}
f_n(\om)$ (в этом случае мы пишем $f_n (\om) \nearrow f(\om)$).
Тогда
$$\int \limits_{\Om} \fo \nu( \, d\om) = \int
\limits_{\Om} \fo \, d\nu \dd \lim \limits_{n\rightarrow \infty}
\int \limits_{\Om} f_n(\om) \nu ( \, d\om).$$
\end{df}
\begin{df} Если
$$
\io f(\om) \, \nu (d\om)\ \Bigl(= \lim \limits_{\nsb} \int
\limits_{\Om} f_n(\om) \nu( \, d\om)\Bigr) < \infty,
$$
 то функция $f_n$ называется
интегрируемой  (суммируемой) по Лебегу на множестве $\Om$.
\end{df}

\begin{df} Любую функцию $\fo: \Om \rightarrow \R$ можно представить в виде разности
двух неотрицательных функций:
$$
\forall \om \quad f(\om) =f^+(\om) - f^-(\om), \mbox{ где }$$
$$f^+(\om) = \begin{cases}
  \fo, &   \text{если $\fo > 0$} \\
  0,   &   \text{если $\fo \leqslant 0 $}
\end{cases}
\quad \mbox{ и } \quad f^{-}(\om) = \begin{cases}
  0,   &   \text{если $\fo \geqslant 0$} \\
 -\fo, &   \text{если $\fo < 0$}
\end{cases}
$$
\end{df}
\begin{df} Говорим, что $f$ - интегрируема, если интегрируемы  $f^+$
и $f^-$ и по определению  $$ \io \fo \dn \dd \io f^+ (\om) \, d\nu
- \io f^- (\om) \, d\nu.$$
\end{df}
Если интегрируема только одна из функций $f^+,\  f^-,\ $ то $f$ -
неинтегрируема, но $\io f(\om) \dn $ определен и равен $+\infty
(-\infty)$ (такие функции называются квазиинтегрируемыми). Если
обе функции $f^+, f^-$ неинтегрируемы, то $f$ - неинтегрируема.
\begin{problem}
Показать, что если $\fo$ - измерима, то $f^+(\om)$ и $f^-(\om)$ -
измеримы.
\end{problem}
\begin{problem} Если функция интегрируема по Риману на отрезке [0,1] то она
интегрируема по Лебегу и интегралы совпадают.
\end{problem}
\bigskip

Следующее предложение показывает, что определение 4 корректно (то
есть, что значение интеграла не зависит от выбора
последовательности $(f_n)$).

\begin{prop}
Если $(f_n)$ и $(g_n)$ - последовательности неотрицательных
простых функций, причем $f_n \nearrow f$ и $ g_n \nearrow f,$ то
$$
\lim \limits_{\nsb} \io  f_n \, d\nu = \lim \limits_{\nsb}
\io g_n \, d\nu
$$
(доказательство будет приведено в следующей лекции).
\end{prop}
\begin{df} Если $f$ - неотрицательная простая функция и $\Om_1 \subset \Om$, $\Om \in
\As$. Тогда
$$\int \limits_{\Om_1} \fo \, d\nu \dd \io \gamma_{\Om_1} (\om) \fo \, d\nu.$$- интеграл Лебега по измеримому
подмножеству.
\end{df}
\subsection*{Свойства интеграла Лебега от простых неотрицательных
функций.}
\begin{enumerate}
\item $\io \bigl( f_1(\om) + f_2(\om) \bigr) \, d\nu =  \io
f_1(\om) \, d\nu + \io f_2(\om) d\nu $.\\
\item $ \io c \fo\, d\nu = c \io \fo \, d\nu ,  \  c
\geqslant 0$.\\
\item Если $f(u)=0$ почти всюду, то $\io \fo\, d\nu =0$ .\\
\item Если $\Om =\bigsqcup \limits_{i=1}^n \Om_i$ и множества
$\Om_i$ - измеримы,  то $\io \fo \,d\nu = \sum \limits_{i=1}^n \
\int \limits_{\Om_i} \fo
\, d\nu$.\\
\item Если $f_1(\om)\geqslant f_2(\om) \geqslant 0 \quad \forall
\om $, то $\io f_1(\om) \, d\nu \geqslant
\io f_2(\om) \, d\nu. $ \\
\begin{proof}(5) Рассмотрим представление
$$ \underbrace{\bigl( f_1(\om) - f_2(\om) \bigr)}_{ \geqslant \  0}
+ \underbrace{f_2(\om)}_{ \geqslant \  0} = f_1(\om).$$

Тогда можно записать так:
$$
\underbrace{\io \bigl(f_1(\om) - f_2(\om) \bigr)}_{\geqslant \  0
\ \text{(из определения)}}  d\nu + \underbrace{\io f_2(\om)}_{
\geqslant \  0}  d\nu = \io f_1(\om) d\nu \  \Rightarrow \  \io
f_2(\om) \,d\nu \leqslant \io f_1(\om) \, d\nu.$$
\end{proof}
\end{enumerate}
%

\section {Лекция}
Определим интеграл Лебега для произвольных неотрицательных
измеримых функций.

\begin{df}
Пусть $\fo \geqslant 0 \quad \forall \om$, \  $f$ - измеримая
функция на $(\Om,\As)$, тогда $\exists$
последовательность $g_n(\om)$  простых функций:
$$g_n(\om)
\nearrow \fo \mbox{(существование докажем позже).}$$

По определению интегралом Лебега называется следующий интеграл
$$\io \fo \, d\nu \dd \lim \limits_{\nsb} \io g_n(\om) \,d\nu.$$
$f$ - называется интегрируемой, если предел конечен.
\end{df}
Докажем, что это определение не зависит от выбора
последовательности $g_n(\om)$.\\
\begin{proof}
Пусть имеются две последовательности $$g_n(\om):\forall \om \quad
g_n(\om) \nearrow \fo,$$   $$\varphi_n(\om):\forall \om \quad
\varphi_n(\om) \nearrow \fo.$$ Достаточно проверить, что
$$\forall k  \ \ \lim \limits_{\nsb} \io g_n \dn \geqslant \io
\varphi_k(\om) \dn.$$ Тогда отсюда будет следовать, что $$\lim
\limits_{\nsb} \io g_n \dn \geqslant  \lim \limits_{k \rightarrow
\infty} \io \varphi_k(\om) \dn.$$ Так как ситуация симметрична, то
будет верно и обратное неравенство $$\lim \limits_{\nsb} \io g_n
\dn \leqslant \lim \limits_{k \rightarrow \infty} \io
\varphi_k(\om) \dn.$$
Отсюда будет следовать корректность определения. \\
Итак, нам известно, что \begin{equation*} \left.
 \begin{aligned}
  \forall \om, \quad g_n(\om) & \nearrow \fo \\
  \forall \om, \quad \varphi_n(\om) & \nearrow \fo \\
 \end{aligned}
\right \} \text{$\Rightarrow \forall k, \ \forall \om \quad  \lim
\limits_{\nsb} g_n(\om) \geqslant \varphi_k(\om), \ \
\varphi_k(\om) = \sum \limits_{j=1}^{n_k} \la_j \cdot
\gamma_{A_j}.$}
\end{equation*}

Пусть
$$
A_n^k\ =\ \{ \om: \  g_n(\om) + \varepsilon > \varphi_k(\om) \};
$$
$$
C_n^k\ =\ \{ \om: \ g_n(\om) + \varepsilon \leqslant
\varphi_k(\om)\};
$$
тогда $\forall \, k$
$$
\lim \limits_{\nsb} (g_n(\om) + \varepsilon) > \varphi_k(\om).
$$
При этом $\forall \, k \quad C_1^k \supset C_2^k \supset \ldots $
и $\bigcap \limits_{n=1}^{\infty} C_n^k = \varnothing.$

Действительно, если $\ \om \in C_{n+1}^k,\ $ то $ \ g_{n+1} (\om)
+ \varepsilon \leqslant \ff_k (\om),\ $ а так как $\ g_n (\om)
\leqslant g_{n+1} (\om),\ $ то тем более
$$
g_{n} (\om) + \varepsilon \leqslant \ff_k (\om), \  \mbox{ так что
} \ \om \in C_n^k.
$$

Далее $\ \forall \, \om \in \Om, \quad \exists \, n_0\in
\mathbb{N},\ $ такое что
$$
\forall \, n > n_0 \qquad g_n (\om) + \varepsilon > \ff_k (\om);
$$
но это значит, что для таких $n \quad \om \not\in C_n^k;\ $ тем
более $\om \not\in \bigcap \limits_{n=1}^{\infty} C_n^k.$ Так как
это верно для каждого $\ \om \in \Om,\ $ то получается, что
$$
\bigcap \limits_{n=1}^{\infty} C_n^k = \varnothing.
$$

Вспомним критерий счетной аддитивности. Если имеет место
последовательность измеримых множеств
$$
C_1 \supset C_2 \supset \ldots \  \mbox{ и }\  \bigcap
\limits_{j=1}^{\infty} C_j = \varnothing,\  \mbox{ то }\  \nu
(C_j) \longrightarrow 0,
$$

так что $\ \forall \, k \quad \nu C^k_n
\stackrel{\nsb}{\longrightarrow} 0$

Докажем теперь, что $\ \forall \, \varepsilon > 0, \ \forall k$
$$
\lim \limits_{\nsb} \io (g_n(\om) + \varepsilon) \dn \geqslant \io
 \varphi_k(\om) \, \dn.
$$

Заметим, что $\ \forall k, \, n$
\begin{eqnarray}
 & \io \bigl(g_n(\om)+ \varepsilon \bigr) \dn + \int \limits_{C_n^k}
 \varphi_k(\om) \dn = \int\limits_{A_n^k} \bigl(g_n(\om)+ \varepsilon
 \bigr) \dn + \int\limits_{C_n^k} \bigl(g_n(\om)+ \varepsilon \bigr) \dn
 + \int\limits_{C_n^k} \varphi_k(\om) \dn \geqslant \nonumber & \\
 &  \geqslant \int \limits_{A_n^k} \varphi_k(\om) \dn + \int \limits_{C_n^k}
 \varphi_k(\om) \dn + \int \limits_{C_n^k}\bigl(g_n(\om)+ \varepsilon
 \bigr) \dn = \io \varphi_k(\om) \dn + \int \limits_{C_n^k} \bigl(g_n(\om)+ \varepsilon
 \bigr) \dn.
\end{eqnarray}

Проверим, что $\forall \, k \quad \int\limits_{C_n^k} \ff_k(\om)
\dn \rightarrow 0$ при $\nsb.$

Пусть $\ M_k = \max\limits_{\om} \ff_k(\om);\ $ тогда
$$
\forall \, k,\ \forall \om \quad \ff_k(\om) \leqslant M_k\
\Rightarrow\  \int \limits_{C_n^k} \ff_k(\om) \dn \leqslant \int
\limits_{C_n^k} M_k \dn \leqslant M_k \nu(C_n^k) \longrightarrow
0;
$$
так как
$$
\forall \, k, \, n \qquad \int\limits_{C_n^k} \ff_k(\om) \dn
\geqslant \int \limits_{C_n^k} \bigl(g_n(\om)+\varepsilon \bigr)
\dn, \  \mbox{ то } \ \forall \, k \quad \int\limits_{C_n^k} (g_n
(\om) + \varepsilon) \, \dn \stackrel{\nsb}{\longrightarrow} 0.
$$


Переходя к пределу (по $n$) в неравенстве (1) получим:
$$
\forall \, k \qquad  \lim \limits_{\nsb} \ \io \bigl( g_n(\om)
+\varepsilon \bigr) \dn = \lim \limits_{\nsb} \io g_n(\om) \dn +
\varepsilon \nu(\Om) \geqslant \io \ff_k(\om) \dn.
$$

Так как это верно для каждого $\varepsilon > 0,\ $ то
$$
\forall \, k \qquad \lim\limits_{\nsb} \io g_n (\om) \, \dn
\geqslant \ff_k (\om) \, \dn.
$$
\end{proof}

\section {Лекция}

Свойства интеграла Лебега для произвольных $f: \Om \rightarrow
\R$ такие же, как и для простых неотрицательных функций:\\
\begin{enumerate}
\item Если $\Om= A_1 \bigsqcup A_2 $, то  $\ii = \int
\limits_{A_1} \fo \, d\nu + \int \limits_{A_2} \fo \, d\nu$. \item
$\io c \fo \, d\nu = c \ii$ - верно $\forall c \in \R.$
\item $\io (f_1(\om) + f_2(\om) ) \, d\nu = \io f_1(\om) \, d\nu +
\io f_2(\om) \, d\nu.$
\end{enumerate}
\begin{proof}
Докажем свойство (3) интеграла Лебега. Разобьем $\Om$ на 6
множеств $$\Om=\bigsqcup \limits_{j=1}^{6} A_j, \mbox{ где }$$
$\begin{aligned}
  A_1 & = \{ \om \in \Om: \ f_1(\om) \geqslant 0, f_2(\om) \geqslant 0 \}\\
  A_2 & = \{ \om \in \Om: \ f_1(\om) \geqslant 0, f_2(\om) < 0, f_1 (\om) + f_2(\om) \geqslant 0 \}\\
  A_3 & = \{ \om \in \Om: \ f_1(\om) \geqslant 0, f_2(\om) < 0, f_1 (\om) + f_2(\om) < 0 \}\\
  A_4 & = \{ \om \in \Om: \ f_1(\om) < 0, f_2(\om) \geqslant 0, f_1 (\om) + f_2(\om) \geqslant 0 \}\\
  A_5 & = \{ \om \in \Om: \ f_1(\om) < 0, f_2(\om) \geqslant 0, f_1 (\om) + f_2(\om) < 0 \}\\
  A_6 & = \{ \om \in \Om: \ f_1(\om) < 0, f_2(\om) < 0 \}\\
 \end{aligned}$\\

Из $(1)$ вытекает равенство $\io g \dn = \sum \limits_{j=1}^6 \
\int \limits_{A_j} g \dn.$ Следовательно, для доказательства (3)
достаточно доказать такое же равенство для
каждого $A_j$.\\


Для $A_1, A_6$ - очевидно. Рассмотрим теперь множество $A_2$.
Нужно доказать, что
$$
\int \limits_{A_2} ( f_1+f_2)  \dn = \int \limits_{A_2} f_1 \dn +
\int \limits_{A_2} f_2 \dn.
$$
Запишем тождество $\quad f_1(\om) = \underbrace{\bigl(f_1(\om) +
f_2(\om) \bigr)}_{\geqslant 0} + \underbrace{\bigl( - f_2(\om)
\bigr)}_{ >0} \  \Rightarrow$
$$
\int \limits_{A_2} f_1 \dn = \int \limits_{A_2} \bigl(f_1+f_2
\bigr) \dn + \int \limits_{A_2} (- f_2) \dn.
$$
Из свойства (2) следует, что
$$
\int \limits_{A_2} (-f_2) \, d\nu = - \int \limits_{A_2} f_2 \,
d\nu,$$ значит,
$$\int \limits_{A_2} f_1 \dn + \int
\limits_{A_2} f_2 \dn = \int \limits_{A_2} ( f_1+ f_2 ) \dn.$$

Для множеств $A_3, A_4, A_5$ доказательство аналогично.

\end{proof}

Пусть  $\bar{L}_1 (\Om,\As, \nu)$ - (линейное)
пространство всех измеримых и интегрируемых по мере $\nu$ функций.
Из определения следует, что $f \in \bar{L}_1\  \Leftrightarrow\
\mid f \mid \in \bar{L_1}.$

Обозначим через $\bar{L}_0 (\Om, \As, \nu)$ множество
всех измеримых функций. Говорят, что измеримые функции $f,g$
совпадают почти всюду, если множество
$$
A=\{ \om: \fo \neq g(\om) \} \mbox{ - имеет меру нуль.}$$

Проверим, что  $$\nu A = 0 \ \Rightarrow\  \int \limits_A f \dn
=0.$$

Для простых неотрицательных функции это следует прямо из
определения; для неотрицательных измеримых получается с помощью
перехода; для произвольных измеримых - из разложения  $f=f^+ -
f^-.$
\begin{stm} Пусть
$f=g \ \  \nu$-почти всюду, тогда $\ii= \io g(\om) \dn.$
\end{stm}
\begin{proof}
Пусть $A\ =\ \{\om \colon \ f (\om)  = g (\om)\}.$ Тогда
$$\ii = \int \limits_{\Om\backslash A} f \dn + \int \limits_A f \dn
= \int \limits_{\Om\backslash A} f \dn = \int
\limits_{\Om\backslash A} g \dn =  \int \limits_{\Om\backslash A}
g \dn + \int \limits_A g \dn = \int \limits_{\Om} g \dn.$$
\end{proof}

Рассмотрим подпространство $E_0 = \{ f \in \bar{L}_0 (\Om,
\As, \nu); f=0 \ \nu$-почти всюду\} в $\bar{L}_0 (\Om,
\As, \nu).$
\begin{df}
Две функции называются эквивалентными, если они совпадают почти
всюду.
\end{df}
Удобно рассматривать фактор-пространства
$$
L_0(\Om, \As, \nu) = \bar{L}_0(\Om, \As, \nu) /
E_0.$$ $$L_1(\Om, \As, \nu) = \bar{L}_1(\Om,
\As, \nu) / E_0.
$$
Элемент этого фактор-пространства - это класс эквивалентности.
Будем обозначать элементы пространства $\bar{L}_0$ так же, как и
элементы пространства $L_0$, то есть писать $\ f_1(\cdot) \in
L_0,\ $ имея ввиду, что $f_1(\cdot)$ - это представитель класса
эквивалентности.

\subsection{Неравенство Чебышева.}

Пусть $f \in L_1(\Om, \As, \nu)$. Тогда
$$
\forall \, c >0\quad \nu \{ \om \in \Om: \ |\fo| \geqslant c\} \
\leqslant\  \frac{1}{c} \io |\fo| \dn.
$$
\begin{proof}
Введем два множества
$$
C_1 = \{\om: |\fo| \geqslant c \}
$$
$$
C_2 = \{\om: |\fo| < c \}
$$
Тогда
\begin{eqnarray*}
&\io |\fo| \dn = \int  \limits_{C_1} |\fo| \dn  + \int
\limits_{C_2} |\fo| \dn \geqslant \int  \limits_{C_1} |\fo| \dn \geqslant
 \int \limits_{C_1} c \dn = c \int \limits_{C_1} \dn = c \ \nu \{
 \om: |\fo| \geqslant c \}
\end{eqnarray*}
\end{proof}
\begin{stm}
Если $ \ f \in L_1(\Om, \As, \nu)\ $, то $\ |\fo| <
\infty \quad \nu$ почти всюду.
\end{stm}
\begin{proof}
$$
\{ \om: |\fo| = \infty \}\  = \ \bigcap \limits_{n=1}^{\infty}
\{\om : |\fo| > n \}.
$$
Оценим меру каждого из этих множеств $$ \nu \{\om:|\fo| > n \}\
\leqslant\  \frac{1}{n} \io |\fo| \dn \longrightarrow 0 \  (n
\rightarrow \infty ),$$ так как $\io |\fo| \, d\nu < \infty.$

Ясно, что
$$\{ \om: |\fo| \geqslant n \} \supset \{ \om: |\fo|
\geqslant n+1 \}.$$ Далее пользуемся тем, что для
последовательности вложенных множеств мера их пересечения равна
пределу мер этих множеств.

\end{proof}
\begin{theorem}[Беппо-Леви]
Пусть $f_n(\om)$ -  последовательность измеримых неотрицательных
функций, такая что $f_n(\om)\nearrow \fo$ в каждой точке. Тогда
$$
\lim \limits_{\nsb} \io f_n(\om) \dn = \ii
$$
(в частности, если $f$ - интегрируема, то предел конечен). (Если в
формулировке добавить слова почти всюду, то теорема все еще будет
верна).
\end{theorem}
\begin{proof}
Пусть $g_{ij}$ -  последовательности простых функций, такие что
\[ \begin{array}{cc}
 & 0 \leqslant g_{11} \leqslant g_{12} \ldots \rightarrow f_1 \\
 & 0 \leqslant g_{21} \leqslant g_{22} \ldots \rightarrow f_2 \\
 & \ldots \\
 & 0 \leqslant g_{k1} \leqslant g_{k2} \ldots \rightarrow f_k \\
 &\\
 & f_1 \leqslant f_2 \leqslant \ldots \leqslant f_k \rightarrow f(\om)\\
\end{array} \]
Введем новые функции $g_n(\om)=\max \limits_{1\leqslant k
\leqslant n} g_{kn}(\om)$.

Заметим, что $\ \forall \, \om$
$$
g_1(\om) \leqslant g_2(\om) \leqslant \ldots \ \leqslant g_n(\om)
\leqslant f_n(\om)\rightarrow \fo.$$

Следовательно, в силу монотонности $\ \exists \ \lim
\limits_{\nsb} g_n(\om) = g(\om)$ и  $g(\om) \leqslant f(\om).$
Предположим, что доказано, что $\ \forall \, \om\colon \
g(\om)=\fo$. Тогда по определению интеграла получим: $$ \io g_n
\dn \rightarrow \ii$$ Так как
$$\forall n \quad g_n(\om) \leqslant f_n(\om) \leqslant \fo, $$ то
$$\forall n
\ \io g_n(\om) \dn \leqslant \io f_n(\om) \dn \leqslant \ii.$$
Переходя к пределу, получим $$\ii \leqslant \lim\io f_n(\om) \dn
\leqslant \ii.$$

Таким образом, $ \io f_n \dn \rightarrow \ii$.\\
Тем самым заключение теоремы доказано. Осталось проверить, что
$g(\om) = \fo$.

Нам известно, что
$$
\forall k \leqslant n \quad g_{kn}(\om) \leqslant g_n(\om)
\stackrel{\nsb}{\rightarrow}g(\om) \leqslant \fo. $$

Значит,
$$
\forall k \leqslant n \ g_{kn} (\om) \leqslant g(\om) \leqslant
\fo.$$

Но
$$
\forall \, k \quad g_{kn}(\om)\stackrel{k \rightarrow
\infty}{\rightarrow} f_k(\om) \leqslant g(\om).$$


Значит,
$$
\lim \limits_{\nsb} f_k(\om)= \fo \leqslant g(\om) \leqslant \fo \
\ \Longrightarrow \ \  \fo=g(\om).$$
\end{proof}
\begin{note}
В теореме можно условие $0 \leqslant f_n$ заменить на условие $F
\leqslant f_1$, где $F$ - интегрируемая функция. Чтобы свести это
к предыдущему случаю, достаточно рассмотреть последовательность
$$
0 \leqslant f_1(\om) - F(\om) \leqslant f_2(\om) - F(\om)
\leqslant \ldots
$$
\end{note}
\begin{theorem}[Фату-Лебега]
Пусть $(f_n)$ - последовательность измеримых функций. Тогда, если
$$
\forall \, n, \ \forall \, \om \qquad  f_n(\om) \geqslant F(\om),
\text{ где } \ F \in L_1(\Om, \As, \nu).$$
$$\io \liminf \limits_{\nsb} f_n(\om) \dn \leqslant \liminf \limits_{\nsb} \io f_n \dn;$$  если
$$\forall n \ \forall \, \om \quad f_n(\om) \leqslant F(\om), \mbox{ то }$$
$$ \io \overline{\lim
\limits_{\nsb}} f_n(\om) \dn \geqslant \overline{\lim
\limits_{\nsb}} \io f_n(\om) \dn.$$
\end{theorem}
\begin{proof}

Напомним, что
$$
\varliminf \limits_{\nsb} f_n(\om)\ \equiv \  \liminf
\limits_{\nsb} f_n(\om) \dd \lim \limits_{\nsb} \{ \inf \limits_{k
\geqslant n} f_k(\om) \}. $$

Введем  последовательность функций
$$g_n(\om)= \{ \inf \limits_{k
\geqslant n} f_k(\om) \}
 \Rightarrow \forall k \geqslant n \ g_n(\om) \leq g_k(\om).$$

 Мы знаем, что $$F(\om) \leq g_1(\om) \leq g_2(\om) \leq \ldots
 \rightarrow \varliminf \limits_{\nsb} f_n(\om).$$

 Следовательно по теореме
 Беппо-Леви $\quad \lin \io g_n \dn = \io \varliminf \limits_{\nsb} f_n(\om)
 \dn.$

 $$\forall k \geqslant n \ \io g_n(\om) \dn \leqslant \io f_k(\om) \dn
 \Rightarrow \forall n \ \io g_n(\om) \dn
 \leqslant \varliminf \limits_{k \rightarrow \infty} \io f_k(\om) \dn \Rightarrow$$
 $$\Rightarrow \lin \io g_n(\om) \dn \leqslant \varliminf \limits_{k \rightarrow \infty}  \io f_k(\om) \dn.$$
$$\io \underset{\nsb}{\underline{\lim}} f_n(\om) \dn \leqslant
\underset{\nsb}{\underline{\lim}} \io f_n(\om) \dn.$$

Таким образом, доказана первая часть теоремы. Докажем теперь
вторую часть.
$$
-F(\om) \leqslant - \fo,
$$
$$
\varliminf \limits_{\nsb} \bigl( - f_n(\om) \bigr) = -
\overline{\lin} f_n(\om),
$$
$$
- \io \bigl( \overline{\lin} f_n(\om) \bigr) \dn\ = \ \io
\varliminf \limits_{n \rightarrow \infty} \bigl(-f_n(\om)\bigr)
\dn \ \leqslant \ \varliminf \limits_{k \rightarrow \infty} \bigl(
- \io f_n(\om) \dn \bigr)\ = \ - \overline{\lin} \io f_n(\om) \dn,
$$
$$
\mbox{ так что }\  \io \overline{ \lin} f_n(\om) \dn \geqslant
\overline{\lin} \io f_n(\om) \dn.
$$
\end{proof}
\begin{note}
В теореме Фату-Лебега можно предполагать, что все ее условия
выполнены лишь почти всюду.
\end{note}

\section{Лекция}

\begin{theorem} [Лебега о мажорируемой сходимости или О предельном переходе под знаком
интеграла] Пусть $F \in L_1(\Om, \As, \nu)$ и $(f_n)$ -
последовательность измеримых функций, такая что $f_n(\om)
\rightarrow f(\om) \ \nu$-почти всюду. Предположим, что
$$\forall n \ | f_n(\om) | \leqslant F(\om) \quad \nu \mbox{ почти всюду.} $$

Тогда $f \in L_1(\Om, \As, \nu)$ и
$$\ii = \lin \io f_n(\om) \dn.$$
\end{theorem}
\begin{proof}
В условии теоремы $-F(\om) \leq f_n(\om) \leq F(\om) \quad \nu
\mbox{ почти всюду.} \  \Rightarrow\ $ можно применить теорему
Фату-Лебега.
  $$\overline{\lin} \io f_n(\om) \dn \leqslant \io \overline{\lin}
  f_n(\om) \dn = \io \varliminf \limits_{\nsb} f_n(\om)
  \dn \leqslant \varliminf \limits_{\nsb} \io f_n(\om) \dn $$
Следовательно,
$$\varliminf \limits_{n \rightarrow \infty}
\iin \geqslant \overline{\lin} \iin, \ \mbox{ так что }$$
$$\varliminf \limits_{\nsb} \iin = \overline{\lin}
\iin\  \Rightarrow\  \exists \ \lin \iin \ \mbox{ и }$$
$$\lin \iin = \io \lin f_n(\om) \dn.$$
\end{proof}
\begin{theorem}[Фату] Пусть $(f_n)$ - последовательность
неотрицательных измеримых функций, причем
$$
0 \leqslant \fn \rightarrow \fo \  \nu \mbox{-почти всюду} \quad
\mbox{ и } \quad  \exists C>0 :\  \forall \, n \  \io \fn
\leqslant C.
$$
Тогда
$$
f \in L_1(\Om, \As, \nu) \mbox{ и } \ii \leqslant C.
$$
\end{theorem}
\begin{proof}
Последовательность $f_n(\cdot)$ удовлетворяет условиям теоремы
Фату-Лебега
$$
\io \varliminf \limits_{\nsb} \fn \dn  \leqslant \varliminf
\limits_{\nsb} \iin \leqslant C \ \ \text{и так как } f_n(\om)
\rightarrow \fo,
$$
то $\ \ii \leqslant C.$

\end{proof}
В условиях теоремы Фату заключение теоремы Лебега не обязано
выполняться.

\begin{note}[1]
Приведем пример.
$$\epsfbox{picture.5}$$
$$\epsfbox{picture.6}$$

Пусть $\Om= [0,1], \ \As$ - сигма-алгебра измеримых по
Лебегу множеств,$\ \nu$ - мера Лебега. Рассмотрим функции,
изображенные на первом рисунке следующей страницы.

Очевидно, что $f_n(\om) \stackrel{\nsb}{\rightarrow} 0 \quad
\forall \om$.
$$\forall \,n \quad \int\limits_0^1 \fn \dn \leqslant \frac{1}{2}.$$

В этом случае $\lin \iin = \frac{1}{2} \neq \io \lin \fn \dn = 0.$
\end{note}
\begin{note}[2]
Если в теореме Фату исключить условие $\fn \geqslant 0$, то
теорема станет неверной.

Рассмотрим последовательность функции, изображенную на втором
рисунке. Считаем, что
$$
\forall \, n \quad f_n (\om)\ =
\begin{cases}
1, & \text{если $x \in [\frac{1}{n},1]\cup \{0\}$;}\\
-(n-1), & \text{если $x\in (0, \frac{1}{n})$.}
\end{cases}
$$
Тогда $\fn \stackrel{\nsb} \longrightarrow 1$
$$\forall n \ \iin =0 \leq \frac{1}{2}=C \ \text{и} \ \io \lin \fn
\dn =1 \nleqslant \frac{1}{2}.$$
\end{note}

\begin{df}
Пусть $(\Om, \As, \nu)$ - пространство с мерой, $\ C \in
\As , \ \nu C > 0$. $\As_C - \sigma$-алгебра
подмножеств $C$, состоящая из всех множеств из $\As$,
которые содержатся в $C$. Тогда пространство $(C,\As_C,
\nu_C)$ называется измеримым подпространством пространства $(\Om,
\As, \nu).$
\end{df}
\begin{prop}
Пусть $f \in L_1(\Om, \As, \nu),$ тогда
$$
\forall A \in \As \quad f\in L_1 (A, \As_A,
\nu_A).
$$
Следовательно, определена функция $\nu_f\colon \ \As \ni
C \mapsto \int \limits_C \fo \dn.$ Эта функция является
счетно-аддитивной.
\end{prop}
\begin{proof}
Пусть $C=\bigsqcup \limits_{n=1}^{\infty} C_n.$ Нужно доказать,
что $$\int \limits_C \fo \dn = \sum \limits_{n=1}^{\infty} \int
\limits_{C_n} \fo \dn$$
$$\left(\int
\limits_C \fo \dn = \io \fo \gamma_c(\om) \dn\right).$$

Так как $C=\bigsqcup \limits_{n=1}^{\infty} C_n, \ $ то
$\gamma_c(\om)=\sum \limits_{n=1}^{\infty} \gamma_{C_n}(\om)\
\Rightarrow$ $$\fo \gamma_c(\om)= \sum \limits_{n=1}^{\infty} \fo
\gamma_{C_n}(\om).$$ Положим $g_k(\om)= \sum \limits_{n=1}^{k} \fo
\gamma_{C_n}(\om).$

Из того, что $f \in L_1 \text{ следует, что }$
$$
\fo\gamma_C(\om) \in L_1 \  \Rightarrow \  |\fo \gamma_c(\om)| \in
L_1.$$
Тогда
$$
\forall k \ |g_k(\om)| = | \sum \limits_{n=1}^{k} \fo
\gamma_{C_n}(\om)| \leqslant |\fo|,$$

то есть выполнено одно из условий теоремы Лебега.

Кроме того,
$$\fo \gamma_C(\om) = \sum \limits_{n=1}^{\infty} \fo \gamma_{C_n}(\om)
= \lim \limits_{\nsb} g_n(\om).$$ Поэтому
$$\int \limits_C \fo \dn = \io
\fo \gamma_C(\om) \dn = \lim \limits_{k \rightarrow \infty} \io
g_k(\om) \dn = \lim \limits_{k \rightarrow \infty} \sum
\limits_{n=1}^{k} \io \fo \gamma_{C_n} \dn =$$$$=\lim \limits_{k
\rightarrow \infty} \sum \limits_{n=1}^{k} \int \limits_{C_n} \fo
\dn = \sum \limits_{n=1}^{\infty} \int \limits_{C_n} \fo \dn$$ по
теореме Лебега.

\end{proof}
\begin{imp}[из теоремы Беппо-Леви]
Пусть $\ \forall n \quad g_n(\om) \geqslant 0$ - почти всюду и
\linebreak
 $\ g(\om) = \sum \limits_{n=1}^{\infty} g_n(\om).$ Тогда
$$\io g(\om)
\nu (d\om)= \sus \io g_n(\om) \nu(d\om).$$ Если еще $\sus \io
g_n(\om) \nu(d\om) <\infty$, то $\io g(\om) \nu(d\om) < \infty,$
то есть $g(\cdot) \in L_1$ и, следовательно, $$g(\om)< \infty \
\nu\mbox{-почти всюду.}$$
\end{imp}
\begin{problem}
Доказать следствие.
\end{problem}
Согласно предложению 1 всякой функции $ f \in \bar{L_1} (\Om,
\As, \nu)$ можно сопоставить функцию множества $\nu_f$ на
$ \As:$
$$\nu_f(A)= \int \limits_A f \dn.$$
\begin{df} Пусть $\nu, \mu$ - неотрицательные меры на измеримом пространстве $(\Om,
\As)$. Говорят, что мера $\nu$ абсолютно непрерывна $(\nu
\ll \mu)$ относительно меры $\mu$, если $\forall C \in
\As$ из того, что $\mu C = 0 $ следует, что $\nu C =0 .$
\end{df}
\begin{ex}
$\nu_f \ll \nu\ (f\geqslant 0).$
\end{ex}
\begin{theorem}
$\nu \ll \mu \Leftrightarrow\ \forall\,$ последовательность $\{C_n
\in \As\colon\ n\in \mathbb{N}\}$, из того, что $\mu C
\rightarrow 0$ следует, что $\nu C \rightarrow 0.$

(Другими словами, $\forall \varepsilon >0 \ \exists \delta> 0 \
\forall C \in \As$ если $\mu C < \delta,$ то $\nu C <
\varepsilon$.)
\end{theorem}
\begin{proof}
$\Leftarrow$ Если $\mu C =0 $, то $ \nu C=0$ (это очевидно).\\
$\Rightarrow$ Будем доказывать от противного. Нам надо привести к
противоречию следующее утверждение:
$$\exists \, \varepsilon >0: \forall \delta >0 \ \exists C_{\delta} \in
\As : \mu C_{\delta} <\delta \ \text{, но} \ \nu
C_{\delta} > \varepsilon.$$ Будем брать в качестве $\delta$
элементы $\delta_n$ сходящейся последовательности
$$
\sum \limits_{n=1}^{\infty} \delta_n < \infty, \quad \forall n \
\delta_n >0 \  \Longrightarrow \ \forall \, n> 0 , \ \exists
C_{\delta_n} \equiv C^n\colon \ \mu\,C^{n} < \delta_n, \ \  \nu
C^n > \varepsilon.
$$

Положим $A_k= \bigcup \limits_{n=k}^{\infty} C^n.$ Ясно, что эти
множества вложены друг в друга:
$$
A_1 \supset A_2\supset \ldots;$$

Введем также следующее обозначение $A= \bigcap
\limits_{k=1}^{\infty} A_k$.

Теперь оценим меру множества $A_k$:
$$
\mu A_k \leqslant \sum \limits_{n=k}^{\infty} \delta_n \rightarrow
0 \ ( k \rightarrow \infty).$$

Следовательно,
$$(\forall k \quad A \subset A_k) \  \Longrightarrow \ \mu A \leqslant \mu
A_k \stackrel{k \rightarrow \infty} \rightarrow 0 \
\Longrightarrow \ \mu A = \mu \bigcap \limits_{k=1}^{\infty}A_k =
0.$$

Отсюда следует, что $\nu A =0$. Из счетной аддитивности меры $\nu$
вытекает, что $$\nu A_k \rightarrow \nu \bigcap
\limits_{k=1}^{\infty} A_k = 0.$$

С другой стороны, $A_k= \bigcup \limits_{n=k}^{\infty} C^n, $
значит $A_k \supset C^n$, если $n \geqslant k$.

Вспомним, что $\ \nu C^n > \varepsilon\  \Rightarrow\  \nu A_k
\geqslant \nu C^n > \varepsilon \ \Rightarrow$ мы получили
противоречие.

\end{proof}

Пусть снова $(\Om, \As, \mu)$ - пространство с мерой и $f
\in L_1(\Om, \As, \mu);\ $ положим
$$
(f\mu)(A)=\mu_f(A) = \int \limits_A \fo\, d\mu.$$

Говорят, что эта мера $f_\mu \equiv \mu_f$ - это произведение меры
$\mu$ и $f$.\ При этом
$$\mu_f \ll \mu \ (f \mu \ll \mu),$$
так как, в силу одного из свойств интеграла,\  $\mu A = 0
\Rightarrow \mu_f A =0.$


\textbf{Свойство абсолютной непрерывности интеграла Лебега.}

$$\forall \varepsilon > 0 \ \exists \delta > 0 \ \forall A \in
\As, \ \mbox{ если }\  \mu A < \delta, \ \mbox{ то }\
\left| \int \limits_A \fo \, d\mu \right| < \varepsilon.$$

Для того, чтобы получить это свойство, нужно просто заметить, что
$\mu_{|f|} \ll \mu$\ и воспользоваться неравенством:
$$| \int \limits_A \fo \, d\mu| \leq
\io |\fo| \,d\mu < \varepsilon.$$

\section{Лекция}
Пусть $S$ - кольцо подмножеств множества $\Om.$
\begin{df}
Мера $\nu$ на кольце $S$ подмножеств множества $\Om$ называется
$\sigma$-конечной, если
$$
\Om = \bigcup \limits_{j=1}^{\infty} \Om_j, \quad \Om_j \in S,
\quad \nu \Om_j <\infty \ \mbox{ для каждого } j.
$$
\end{df}
Введем вместо множества $\Om_n$ множество $\Om'_n = \bigcup
\limits_{j=1}^{n} \Om_j$, тогда
$$\Om'_1 \subset \Om'_2 \subset \Om'_3 \subset
\ldots$$

В дальнейшем будем использовать множества $\Om'_n$, но штрих
писать не будем для удобства записи.

В этом случае для любого $j$ рассмотрим $(\Om_j, S_j, \nu_j)$, где
$$S_j=\{A \in S | A \subset \Om_j\}$$
и $\nu_j$ - сужение меры $\nu$ на совокупность множеств $S_j;\ $
при этом $S_j$ - это алгебра множеств.

Пусть теперь, для всякого $j,\ \overline{\nu}_j$ - продолжение
$\nu_j$ на $\sigma$-алгебру $\overline{\sigma (S_j)}\
\nu_j$-измеримых подмножеств множества $\Om_j.$

Объявим, по определению, что $$A \in \overline{\sigma(S)} \
\Longleftrightarrow \ \forall j \quad A \cap \Om_j \in
\overline{\sigma(S_j)} \quad \bigl( \Om_j \in S_j \subset
\sigma(S_j) \subset \overline{\sigma (S_j)} \bigr).$$
$$\overline{\nu}A = \lim \overline{\nu}_j ( A \bigcap \Om_j) \quad
-$$ - этот предел всегда существует, так как $\overline{\nu}$ не
убывает.
\begin{problem} Доказать, что
$\overline{\nu}$ - счетно-аддитивная мера, принимающая, быть
может, бесконечные значения.
\end{problem}
Пусть $\Om=\R$, $S$ - полукольцо конечных полуинтервалов
$[\alpha,\beta)$ и $\nu \bigl( [\alpha, \beta) \bigr) = \beta -
\alpha$. \\
В качестве $\Om_n$ возьмем отрезок $[-n, n]$. Тогда
получим $(\R, \overline{\sigma (S)}, \overline{\nu}).$

\begin{df}
$\overline{\sigma(S)} - \sigma$-алгебра подмножеств прямой,
измеримых по Лебегу, а $\overline{\nu}$ - мера Лебега. Аналогично
определяется и мера Лебега на $\R^n.$
\end{df}
Все предыдущие теоремы справедливы (иногда после небольших
изменений) также и для меры, которая принимает бесконечные
значения. Только в определении интеграла надо рассматривать
простые неотрицательные функции, которые являются конечными
линейными комбинациями индикаторов множеств, имеющих конечную
меру.
\begin{theorem}
Если вещественная функция $f: [0,1] \rightarrow\R$
интегрируема по Риману в собственном смысле, то она интегрируема
по Лебегу и интегралы совпадают.
\end{theorem}
\begin{proof}
Если $f$ - интегрируема по Риману, то для любого разбиения отрезка
$[0,1]$ суммы Дарбу сходятся к интегралу Римана. Рассмотрим
разбиение отрезка на $2^n$ одинаковых частей.

$$\epsfbox{picture.7}$$

Запишем суммы Дарбу. $\forall k \quad \overline{a}_k = \sup
\limits_{x \in [\frac{k-1}{2^n}, \frac{k}{2^n})} f(x) \
\Rightarrow \ \sum \limits_{k=1}^{2^n} \overline{a}_k
\frac{1}{2^n} = \overline{S}_n(f)$ - верхняя сумма Дарбу.

$\forall k \quad \underline{a}_k = \inf \limits_{x \in
[\frac{k-1}{2^n} , \frac{k}{2^n})} f(x) \  \Rightarrow \  \sum
\limits_{k=1}^{2^n}\underline{a}_k \frac{1}{2^n}
=\underline{S}_n(f)$ - нижняя сумма Дарбу.

Обе эти суммы имеют пределы, которые равны: $$(R) \int \limits_0^1
f(x) \, dx = \lin \overline{S}_n(f) = \lin \underline{S}_n(f).$$

Введем некоторые вспомогательные функции, для которых интеграл
Лебега будет совпадать с суммами Дарбу.
$$\overline{f}_n(x) = \sum
\limits_{k=1}^{2^n} \overline{a}_k \gamma_{[\frac{k-1}{2^n},
\frac{k}{2^n})}(x), \quad \underline{f_n}(x) =\sum
\limits_{k=1}^{2^n} \underline{a_k} \gamma_{[\frac{k-1}{2^n} ,
\frac{k}{2^n})}(x)$$ По определению $\forall n, x$ справедливо
равенство $\underline{f_n}(x) \leqslant f(x) \leqslant
\overline{f}_n(x).$

Более того, $$\inf \limits_{x \in [0,1]} f(x) \leqslant
\underline{f_1}(x) \leqslant \underline{f_2}(x) \leqslant \ldots
\leqslant f(x) \leqslant \ldots \leqslant \overline{f}_2(x)
\leqslant \overline{f}_1(x) \leqslant \sup \limits_{x \in [0,1]}
f(x).$$

Так как функция интегрируема по Риману, то $|\inf f(x)|, |\sup
f(x)| < \infty$ и, в силу монотонности соответствующих
последовательностей,
$$\exists \lin \underline{f_n}(x) = \underline{f}(x) \leqslant
f(x) \leqslant \overline{f}(x) =\lin \overline{f}_n(x)$$

Заметим, что
$$\forall n \quad (L) \int \limits_{[0,1]}
\overline{f}_n(x) \, dx = \int \limits_{[0,1]} \sum
\limits_{k=1}^{2^n}\overline{a}_k \gamma_{[\frac{k-1}{2^n} ,
\frac{k}{2^n})}(x) \, dx =  \sum \limits_{k=1}^{2^n}\overline{a}_k
\int \limits_{[0,1]} \gamma_{[\frac{k-1}{2^n} , \frac{k}{2^n})}(x)
\, dx = \sum \limits_{k=1}^{2^n}\overline{a}_k \frac{1}{2^n}
=\overline{S}_n(f)
$$ Аналогично проверяется, что $\forall n \quad (L)\int \limits_{[0,1]}
\underline{f_n}(x) \, dx = \underline{S}_n(f)$.

Следовательно,

$(L)\int \limits_{[0,1]} \overline{f}_n(x) \, dx=
\overline{S}_n(f) \rightarrow (R) \int \limits_0^1 f(x) \,
dx$\\$(L)\int \limits_{[0,1]} \underline{f_n}(x) \, dx =
\underline{S}_n(f) \rightarrow (R) \int \limits_0^1 f(x) \, dx$

К последовательностям $\underline{f_n} , \overline{f}_n$ применима
теорема Лебега о предельном переходе под знаком интеграла. Значит,
$$(L)\int \limits_{[0,1]} \overline{f}_n(x) \,dx \rightarrow \int \limits_{[0,1]} \overline{f}(x) \,
dx = (R) \int \limits_0^1 f(x) \, dx$$
$$(L)\int \limits_{[0,1]} \underline{f_n}(x) \,dx \rightarrow \int \limits_{[0,1]} \underline{f}(x) \,
dx = (R) \int \limits_0^1 f(x) \, dx$$

Получаем, что $$(L) \int \limits_{[0,1]} \bigl( \overline{f}(x) -
\underline{f}(x) \bigr) \, dx = 0.$$

Но
$$\forall x \quad \underline{f}(x) \leqslant f(x)
\leqslant \overline{f} (x) \quad (*)$$

Значит, $\overline{f}(x)= \underline{f}(x)$ почти всюду и,
следовательно,
$$\underline{f}(x) = f(x) =\overline{f}(x)\mbox{ почти всюду.}$$

А, значит, $f(x)$ - измерима (так как $\overline{f}(x)$ и
$\underline{f}(x)$ - измеримы) - см. предложение 6 ниже, - и
интеграл Лебега от нее совпадает с интегралом от
$\overline{f}(x)$. Следовательно, получаем, что
$$(L)\int \limits_{[0,1]} f(x) \, dx = \int \limits_{[0,1]}
\overline{f}(x) \, dx = (R) \int \limits_0^1 f(x) \, dx.$$
\end{proof}
\begin{problem} Вывести из неравенства Чебышева, что если интеграл
Лебега от неотрицательной функции равен 0, то сама функция равна
нулю почти всюду.
\end{problem}

Будем говорить, что функции $f$ и $g$ на пространстве с мерой (не
предполагаемые измеримыми) совпадают почти всюду, если множество
$\{\om \colon \ f(\om)\not = g (\om)\}$ содержится в некотором
множестве меры нуль.

\begin{df}
Рассмотрим пространство с мерой $(\Om, \As, \nu)$. Мера
$\nu$ - называется полной, если  $\forall\, A \ \in \As
:\  \nu A =0,\ \forall\, B\, \subset A \ \Rightarrow \ B \in
\As, \ \nu B =0$.
\end{df}
\begin{prop}
Пусть $f$ - измеримая функция на пространстве с полной мерой,
тогда если $g(x)=f(x) \ \nu-$почти всюду, то $g(x)$ - измерима.
\end{prop}

\begin{proof}
$\forall c \in\R$ рассмотрим множество $\{g(x)< c\}$ и
сравним это множество с множеством $\{f(x)< c\} \in \As$. Из
определения совпадения функций $f$ и $g$ $\nu$-почти всюду, можно
вывести, что
$$\{ g(x)<c\}\setminus \{ f(x)<c \} \subset D, \nu D =0,$$ $$\{
f(x)<c\}\setminus \{ g(x)<c \} \subset D_1, \nu D_1 =0,$$
$$\{ g(x)
< c \} \ = \ \bigl( \{f(x) <c \}\setminus \bigl(\{f(x)
<c\}\setminus \{g(x)<c\} \bigr)\bigr)\bigcup \bigl( \{ g(x) < c\}
\setminus \{f(x) <c \} \bigr).$$

Итак, мы доказали, что $g(x)$ измерима, если мера $\nu$ полна.

\end{proof}
\begin{ex}
Мера, которая была построена в доказательстве теоремы Каратеодори,
является полной. Поэтому полна и мера Лебега на $\mro$ (и на
$\R^n$).
\end{ex}

\section{Лекция}

\begin{df}
Пусть $f\colon \ [0,1] \rightarrow \mro,\ $ причем
$$\ \forall \,
n\in \mathbb{N},\ \mbox{ функция $f$ интегрируема по Риману на
отрезке } \left[\frac{1}{n}, 1\right].$$

Функция $f$ - интегрируема в несобственном смысле но Риману на
отрезке $[0, 1]$ (иногда в этом случае говорят о полуинтервале
$(0, 1]$),
$$\mbox{ если } \ \exists  \lin (R) \int \limits_{\frac{1}{n}}^1 \fd;$$
этот предел называется несобственным интегралом Римана;\  мы
обозначим его так:
$$(R) \int \limits_0^1 \fd; $$
этот интеграл сходится абсолютно, если $$\exists \  \lin \int
\limits_{\frac{1}{n}}^1 |f(x)| \, dx = (R) \int \limits_0^1 |f(x)|
\, dx$$
\end{df}
\begin{theorem}
Пусть $f$ - вещественная функция на $[0, 1],\ $ непрерывная на
$(0,1].$ Функция $f$ интегрируема по Лебегу тогда и только тогда,
когда она интегрируема по Риману, быть может, в несобственном
смысле, и интеграл сходится абсолютно.
\end{theorem}
\begin{proof}
Далее мы не различаем отрезок $[0, 1]$ и полуинтервал $(0,1].$
Рассмотрим вспомогательную функцию $g_n(x)$

$$\epsfbox{picture.8}$$

$$g_n(x) = \begin{cases}
 f(x), & \text{если $x \geqslant \frac{1}{n}$}\\
     0,  & \text{если $x < \frac{1}{n}$ } \end{cases}$$

Заметим, что $\lin g_n(x) =f(x), \ x>0$. На основании предыдущей
теоремы, можем записать
$$
(R) \int \limits_{\frac{1}{n}}^1 \fd = (R) \int \limits_0^1
g_n(x) \, dx = (L) \int \limits_0^1 g_n(x) \, dx
$$
$$
(R) \int \limits_{\frac{1}{n}}^1 |f(x)| \, dx = (R) \int \limits_0^1
|g_n(x)| \, dx = (L) \int \limits_0^1 |g_n(x)| \, dx.
$$
$$
(R) \int \limits_{\frac{1}{n}}^1
|f(x)| \, dx \rightarrow (R) \int \limits_0^1 |f(x)| \, dx \
\text{(по определению)}
$$
$$
(L) \int \limits_0^1 |g_n(x)| \, dx \rightarrow (L) \int \limits
|f(x)| \, dx \ \text{(по теореме Беппо-Леви, так как }
|g_n(x)|\nearrow |f(x)|)
$$
Так как $(R) \iz |f(x)| \, dx < \infty,$ получаем, что
$$
(L) \iz |f(x)| \, dx < \infty\  \Rightarrow \ |f(\cdot)| \in
\overline{L}_1 (0,1),
$$

а, значит, сама функция $f$ - интегрируема по Лебегу ($f(\cdot)
\in \overline{L}_1(0,1)$).

Осталось доказать, что $$(L) \iz \fd = (R) \iz \fd.$$

По определению $\forall n$
$$ \begin{aligned}
   |g_n(x)| & \leqslant |f(x)| \\
 g_n(x) & \rightarrow f(x).
\end{aligned}$$

Значит, мы находимся в условиях теоремы Лебега о предельном
переходе под знаком интеграла.
$$
\begin{aligned}
(R) \iz g_n(x) \, dx & = (L) \iz g_n(x) \, dx \rightarrow (L) \iz
\fd \ \text{(по теореме Лебега)} \\
(R) \iz g_n(x) \, dx & = (R) \int \limits_{\frac{1}{n}}^1 \fd
\stackrel{def}{\rightarrow} (R) \iz \fd \in \R.
\end{aligned}
$$
Мы доказали, что интегралы совпадают, если интеграл Римана
существует.

Если проделать те же выкладки в обратную сторону, можно получить и
обратное утверждение (оставляем в качестве упражнения).
\end{proof}

\begin{ex}
\begin{enumerate}
\item $f(x)= \frac{1}{x^{\alpha}}, \ 0<\alpha<1$ интегрируема по
Лебегу. \item $f(x) = \frac{1}{x}$ не интегрируема по Риману
$\Rightarrow$ не интегрируема по Лебегу. \item Пусть $f(x) = a_1,\
$ если $x\in \left( \frac12, \, 1 \right],\ $ далее $f(x) = a_2, \
$ если $x\in \left( \frac14, \, \frac12 \right], \ \ldots \ f(x) =
a_n,\ $ если $x\in \left( \frac {1}{2^n}, \, \frac{1}{2^{n-1}}
\right].$ Тогда (собственно) $f$ интегрируема по Риману, а по
Лебегу неинтегрируема.

$$\epsfbox{picture.9}$$

Выбираем $a_i$ так, чтобы $$a_1 \cdot \frac{1}{2}=1, \  a_2 \cdot
\frac{1}{4}=\frac{1}{2},\  \ldots ,\  a_n \cdot \frac{1}{2^n}=
\frac{1}{n}.$$

 $$(R) \iz \fd = \sus \frac{
(-1)^{n+1}}{n},$$ так что $f$ интегрируема по Риману (в
несобственном смысле).

Но ряд из модулей расходится $\Rightarrow $  по Лебегу $f$
неинтегрируема.
\end{enumerate}
\end{ex}
\begin{df}
$\overline{L}_p (\Om, \As, \nu)$ - множество всех
измеримых функций $f$ на $\Om$, таких что $$\io |f(x)|^p \nu
(d\om) < \infty.$$
\end{df}
Докажем, что это пространство линейно и определим на нем норму. Из
определения следует, что
$$f(\cdot) \in \overline{L}_p \
\Rightarrow \  \alpha f(\cdot) \in \overline{L}_p \quad \forall
\alpha \in \R.$$

Докажем, что сумма функций из $\overline{L}_p$ тоже принадлежит
$\overline{L}_p$. Воспользуемся неравенством
$$\forall a,b \geqslant 0  \quad (a+b)^p
\leqslant C_p (a^p + b^p),$$

предварительно доказав его.

\begin{proof}
Положим
$$
\psi (t) = \frac{(1 + t)^p}{1 + t^p}, \ t \geqslant 0
$$

Ясно, что $\psi (t)$ является непрерывной функцией, причем $\psi
(0) = 1, \ \psi(t) \rightarrow 1 \ (t \rightarrow \infty).$
Следовательно, $\psi (t) \leqslant C_p.$

Теперь сделаем замену:
$$t = \frac{a}{b},\mbox{ где } a \geqslant 0 \mbox{ и
} b>0,$$ тогда
$$
(1 + t)^p \leqslant C_p(1 + t^p) \  \Longleftrightarrow \ (a +
b)^p \leqslant C_p (a^p + b^p)
$$
\end{proof}

Предположим, что $f(\cdot), g(\cdot) \in \overline{L}_p$, то есть
$|f(\cdot)|^p, |g(\cdot)|^p \in \overline{L}_1 $ и положим
$$a=|f(\om)|, b=|g(\om)|.$$

Тогда имеет место следующая цепочка неравенств:
$$|f(\om) + g(\om)|^p \leqslant \bigl( |f(\om)| + |g(\om)| \bigr)^
p \leqslant C_p \bigl( |f(\om)|^p + |g(\om)|^p \bigr) \in
\overline{L}_1.$$

Для интегралов эти неравенства сохраняются, значит, $$f(\cdot)+
g(\cdot) \in \overline{L}_p.$$

Далее мы воспользуемся неравенством
$$(1)\qquad ab \leqslant \frac{a^p}{p} +
\frac{b^q}{q}, \quad  \mbox {где } 1 < p, \ q < \infty, \quad
\frac{1}{p}+\frac{1}{q}=1, \quad \forall a,b \geqslant 0.$$

Доказательство этого неравенства остается в качестве упражнения.

Определим функционал на пространстве $\overline{L}_p$ таким
образом:
$$(2)\qquad ||f||_p= \bigl( \io |\fo|^p \,
\nu \,(d\om)\bigr)^{\frac{1}{p}}.$$

Докажем, что это полунорма. Имея $f(\cdot) \in \overline{L}_p,\
g\in \overline{L}_q,$ положим
$$a=\frac{|\fo|}{||f||_p}, \  b=\frac{|g(\om|}{||g||_q}.$$

и подставим их в неравенство (1). Тогда получим:
$$\frac{|\fo||g(\om)|}{||f||_p
||g||_q} \  \leqslant \  \bigl(\frac{|\fo|}{||f||_p} \bigr)^p
\frac{1}{p} \ +\ \bigl(\frac{|g(\om)|}{||g||_q} \bigr)^q
\frac{1}{q}.$$

Интегрируем это неравенство
$$\io \frac{|\fo||g(\om)|}{||f||_p
||g||_q} \nu(d\om) \  \leqslant \  \io \bigl( \frac{1}{p\,
||f||_p^p} |\fo|^p + \frac{1}{q\, ||g||_q^q}|g(\om)|^q \bigr)\,
\nu\,(d\om)\  \stackrel{\mbox{в силу (2)}} = $$
$$=\ \frac{1}{p} + \frac{1}{q}=1 $$
В итоге получим неравенство:
$$\io \frac{|\fo||g(\om)|}{||f||_p ||g||_q} \nu(d\om) \leqslant 1,$$
которое можно переписать так:
$$\io |\fo||g(\om)|
\nu(d\om) \leqslant ||f||_p \,||g||_q \  - $$
 - Это неравенство
носит название неравенства Гельдера.

Неравенство Коши-Буняковского является частным случаем неравенства
Гельдера (при $a = b = 2$).

Также отсюда следует, что $\fo g(\om)$ - интегрируема. Надо
проверить, что $||f||_p$  - полунорма.
\begin{enumerate}
\item $||f||_p \geqslant 0.$ \item $||\alpha f||_p = |\alpha|
\cdot ||f||_p.$ \item $||f+g||_p \leqslant ||f||_p + ||g||_p$
(неравенство Минковского).
\end{enumerate}
При доказательстве неравенства Гельдера мы пользовались тем, что
$||f||_p, ||g||_q > 0$.

Теперь предположим, что $$||f||_p = 0.$$ Тогда это будет
эквивалентно следующему условию:

$$\io |\fo|^p \nu(d\om) = 0.$$
А, значит, $$|\fo|^p= 0 \text{ $\nu-$почти всюду} \ \Rightarrow\
\fo = 0 \text{ $\nu-$почти всюду}\  \Rightarrow\ \fo g(\om) = 0
\text{ $\nu-$почти всюду},$$ следовательно, неравенство Гельдера
справедливо и при $||f||_p \geqslant 0, ||g||_p \geqslant 0.$
Теперь докажем неравенство Минковского.

\begin{proof}
Справедливы соотношения $$\|f+g\|^p_p \ = \ \io |\fo + g(\om)|^p
\nu(d\om) = \io |\fo +g(\om)|\, |\fo + g(\om)|^{p-1}
\nu(d\om)\leqslant $$

$$ \leqslant
\io |\fo|\, |\fo + g(\om)|^{p-1}\, \nu\,(d\om)+ \io |g(\om)|\,
|\fo + g(\om)|^{p-1}\, \nu\,(d\om) \ =
$$
$$\ = \ \left\{ \mbox{ пользуемся тем, что } (p-1) = \frac{p}{q} \right\}\ = \ $$
$$
= \ \io |\fo| \, |\fo + g(\om)|^{\frac{p}{q}} \, \nu \, (d\om) \
+\ \io |g(\om)| \, |\fo + g(\om)|^{\frac{p}{q}} \, \nu \, (d\om) \
\leqslant
$$
$$\leqslant \  \|f\|_p \,\,
\|(f +g)^{\frac{p}{q}}\|_q \ + \ \|g\|_p \,\, \|(f
+g)^{\frac{p}{q}}\|_q
$$
Но
$$
\|(f+g)^{\frac{p}{q}}\|_q  = \bigl( \io |\fo+ g(\om)|^p \nu(d\om)
\bigr)^{\frac{1}{q}}  = \|f+g\|_p^{\frac{p}{q}},
$$
$$
\mbox{ так как }\ \|f+g\|_p^p   = \io |\fo + g(\om)|^p \nu(d\om),
$$
поэтому
$$
\|f+g\|_p^p \  \leqslant \  \|f\|_p \,\, \|f+q\|_p^{\frac{p}{q}} \
+\  \|g\|_p \,\, \|f + g\|_p^{\frac{p}{q}}\  \Leftrightarrow \
\|f+g\|^{p - \frac{p}{q}}_p \  \leqslant \  \|f\|_p \ +\  \|g\|_p
\  \Leftrightarrow
$$
$$\Leftrightarrow\ ||f+g|| \leqslant ||f||_p + ||g||_p, \mbox{ так как } p-\frac{p}{q} = 1.$$
\end{proof}

Определим фактор-пространство
$$\overline{L}_p (\Om,\As,
\nu) / \{ f \in \overline{L}_p , \fo =0 \text{ $\nu-$почти всюду}
\} = L_p (\Om, \As, \nu ).$$

Получаем пространство классов эквивалентности функций $L_p$,
модули которых интегрируемы в р-ой степени. Для $f \in L_p$
определим норму
$$||f||_p= \bigl( \io |\fo|^p \,\nu\, (d\om)
\bigr)^{\frac{1}{p}}.$$

Норма не зависит от выбора представителя класса эквивалентности.


\begin{theorem}
Пространство $L_p$ полно.
\end{theorem}
\begin{proof}
Пусть $(f_n) \subset L_p$ - фундаментальная последовательность в
$L_p$.

Пусть $\{\varepsilon_n\}$ - числовая последовательность  с таким
свойством:
$$\sus \varepsilon_n < \infty, \varepsilon_i >0 \quad \forall \, i$$
Тогда
$$
\exists n_1: \forall n, k > n_1 \ ||f_n - f_k||_p <
\varepsilon_1
$$
$$
\exists n_2: \forall n, k > n_2 \ ||f_n - f_k||_p < \varepsilon_2
$$
и так далее. Тогда ряд функций
$$f_1 + (f_{n_1} - f_1) + (f_{n_2} - f_{n_1}) + \ldots$$
(абсолютно) сходится почти всюду, так как ряд $|f_1| + |f_{n_1} -
f_1| + |f_{n_2} - f_1| + \ldots $ - сходится почти всюду.

Сходимость
$$
\io |f_{n_{j+1}} - f_{n_j}|\, \nu\,(d\om) \ \leqslant\
\|f_{n_{j+1}} - f_{n_j}\|_p\,\, \|1\|_q
$$
$$
\|1\|_q = \bigl( \io \nu(d\om) \bigr)^{\frac{1}{q}} = \bigl
(\nu(\Om)\bigr)^{\frac{1}{q}} = c >0 \ \Rightarrow \
$$
$$
\Rightarrow \ \io |f_{n_{j+1}}(\om) - f_{n_j}(\om)| \nu (d\om)
\leqslant \|f_{n_{j+1}} - f_{n_j}\|_p \ c\ < \ \varepsilon_p \cdot
c.
$$
Поэтому
$$\io |f_1(\om)| \dn + \io
|f_{n_1}(\om) - f_1(\om)| \dn + \ldots < \infty .$$ По теореме
Беппо-Леви:
$$|f_1(\om)| + |f_{n_1}(\om) - f_1(\om)| + \sum\limits_{j=1}^{\infty} |f_{n_{j+1}} - f_{n_j}|
 < \infty \text{\ \ почти всюду}$$
Следовательно, ряд $\ f_1(\om) + (f_{n_1}(\om) - f_1(\om)) +
\ldots\ $ сходится почти всюду.

\section{Лекция}
Частичными суммами этого ряда являются:
$$
s_1 = f_1 (\om), s_2 = f_{n_1} (\om), \ldots, s_{k+1} = f_{n_k}
(\om).
$$
Но сходимость ряда означает сходимость последовательности его
частных сумм. Итак, мы получаем, что последовательность
$f_{n_j}(\om) \rightarrow f (\om) \quad \nu-$почти всюду.
 Докажем, что $f(\cdot) \in L_p$ и что $f_{n_j}(\om) \rightarrow \fo $ в
 метрике $L_p$.

Так как $f_n(\om)$ - фундаментальна в $L_p$, то фундаментальна и
ее подпоследовательность $(f_{n_j})$ $\ \Rightarrow $

$$\forall \, \varepsilon>0 \quad
\exists m \in \mathbb{N}:\  \forall\, k,r > m \ \Rightarrow\
\|f_{n_k} - f_{n_r}\|_{L_p} < \varepsilon$$ Это значит, что
$$\io |f_{n_k}(\om) - f_{n_r}(\om)|^p \dn  <\varepsilon^p \ \mbox{ для таких } k,\, r$$

Зафиксируем $k$ и устремим $r$ к бесконечности. Тогда получим, что
последовательность
$$|f_{n_k}(\om) - f_{n_r}(\om)| \  \stackrel{r \rightarrow \infty}{\longrightarrow} \
|f_{n_k}(\om) - \fo|.$$

Таким образом, мы можем применить теорему Фату. То есть получаем,
что
$$
|f_{n_k}(\om) - \fo| \in L_p\ , \ \io |f_{n_k}(\om) -\fo|^p
\dn < \varepsilon^p  \ \mbox{ для всех } k,\, r  \quad (*)
$$

Отсюда следует, что\\
\\
1. $f_{n_k}(\om) - \fo \in L_p \ $, а так как $L_p$ - линейное
пространство и $f_{n_k} \in L_p$ по условию,
то $f \in L_p$.\\
\\
2. Неравенство $(*)$ верно при $\forall \, k>n_0$ $$\io
|f_{n_k}(\om) - \fo|^p < \varepsilon^p \ \Longleftrightarrow\
||f_{n_k} -f||_p < \varepsilon \quad \forall k
>n_0.$$

Это и означает, что $f_{n_k} \rightarrow f$ в норме $L_p$.

Используем теперь следующее предположение.

\begin{prop}
Пусть $E$ - метрическое пространство и $\{x_n\} \in E$ -
фундаментальная последовательность. Пусть также $x_{n_k} \
\rightarrow\  x \in E.\ $ Тогда $x_n \rightarrow x$ (доказать это
в качестве упражнения).
\end{prop}
В нашем случае роль $E$ играет $L_p,\ $ а роль $x_n$ играет $f_n.\
$ Следовательно, $f_n \rightarrow f$ в $L_p$. Таким образом, мы
доказали, что $L_p -$ полно.

\end{proof}

\subsection{Формула замены переменной в интеграле Лебега.}

\begin{df}
Пусть $F$ - отображение $\ (\Om_1, \As_1)\ $ в $\ (\Om_2,
\As_2), \ $ причем
$$
F - \mbox{ измеримо, то есть } \forall A \in \As_2 \quad
F^{-1} (A) \in \As_1.
$$

Пусть $\nu_1$ - мера на $\As_1$. Определим (прямой) образ
меры $\nu_1 : F_* \nu_1 $ следующим образом:
$$
(F_*\nu_1)(A) \equiv (\nu_1 F^{-1}) = \nu_1 (F^{-1} (A)).
$$
\end{df}

Так как $F^{-1} (A) \in \As_1$ (то есть измеримо), то
число $\nu_1 \, (F^{-1} (A))$ - определено, так что определение
корректно.

\begin{theorem}

Пусть $F$ - определенное выше измеримое отображение:
$$
F: (\Om_1, \As_1)\ \rightarrow\  (\Om_2, \As_2)
$$
 и  $\ g \colon\  \Om_2 \rightarrow \mathbb R$ -
измеримая функция. Тогда верна формула замены переменных:
$$
\int \limits_{\Om_2} g(\om_2) (F_* \nu) (d\om_2) = \int
\limits_{\Om_1} g(F(\om_1)) \nu(d\om_1),
$$
причем интеграл слева существует $\ \Longleftrightarrow\ $
существует интеграл справа.

\end{theorem}
\begin{proof}
1. Проверим для $g= \gamma_{A_2}$, где $A_2 \in \As_2$

$$\int \limits_{\Om_2} \gamma_{A_2} (\om_2) (F_* \nu) (d\om_2) =
(F_*\nu) (A_2) = \nu (F^{-1} A_2)$$

$$ \int \limits_{\Om_1}
\gamma_{A_2} (F(\om_1)) \nu(d\om_1) = \int \limits_{\Om_1}
\gamma_{F^{-1} A_2} (\om_1) \nu (d\om_1) = \nu (F^{-1} A_2).$$

Справедливость первого равенства в последней строке вытекает из
следующей цепочки эквивалентностей:
$$ \gamma_{F^{-1} A_2}
(\om_1) = 1 \ \Longleftrightarrow\  \om_1 \in F^{-1}(A_2) \
\Longleftrightarrow\  F(\om_1) \in A_2 \ \Longleftrightarrow\
\gamma_{A_2} (F(\om_1)) = 1 $$

Таким образом, мы проверили, что формула замены переменных верна
для индикаторов.

2. Отсюда сразу же следует, что она выполняется для $g$ -
неотрицательной, простой.

3. Поэтому она верна и для неотрицательных измеримых функций
(доказывается переходом к пределу, так как последовательность
простых функций стремится к измеримой).

4. Произвольную функцию представим как разность двух
неотрицательных.

\end{proof}

\begin{ex}
Пусть функция $\psi: [c,d] \rightarrow [a, b]$ непрерывно
дифференцируема и обладает такой же обратной. Тогда по формуле
замены переменных, известной из математического анализа, получим
(если $g$ - непрерывная функция):
$$(*) \qquad \int \limits_a^b g(x) \, dx = \int \limits_{\psi^{-1}a =
c}^{\psi^{-1}b =d} g (\psi(z)) \psi^{\prime} (z) \, dz$$

Сравним это равенство с нашей формулой.

Пусть $\Om_1 = [c,d], \ \Om_2 = [a,b], F = \psi$ и $c \leqslant
\alpha \leqslant \beta \leqslant d$. Определим меру на
$\sigma$-алгебре борелевских подмножеств отрезка $[c, d]$ так:

$$\nu[\alpha, \beta) = \int \limits_{\alpha}^{\beta} \psi'(z) \, dz
= \psi(\beta) - \psi(\alpha).$$

Интеграл по этой мере определяется так $$\int
\limits_{\psi^{-1}a}^{\psi^{-1}b} g (\psi(z)) \psi'(z) \, dz =\int
\limits_{\psi^{-1}a}^{\psi^{-1}b} g(\psi(z)) \nu(dz).$$

Найдем образ меры:
$$
(\psi_{*} \nu) [\alpha_1, \beta_1) = \nu
\psi^{-1} ([\alpha_1, \beta_1)) = \nu [\psi^{-1} \alpha_1,
\psi^{-1} \beta_1) = \int \limits_{\psi^{-1} \alpha_1}^{\psi^{-1}
\beta_1} \psi'(z) \, dz = \beta_1 - \alpha_1 = \int
\limits_{\alpha_1}^{\beta_1} \, dx,
$$
то есть $\psi_{*} \nu$ - это мера Лебега на $[a, b]$ и
$$
\il{a}^b g(\om)\, (\psi_{*} \nu) \, (d\om)\ = \ \il{a}^b g(x)\,
dx.
$$

То есть формула из математического анализа является частным
случаем нашей формулы из теоремы 10.
\end{ex}

\begin{prop}
Пусть $g$ - измеримая вещественная функция на $(\Om,
\As),$ \linebreak  $f \in L_1(\Om,\As, \nu)$.

Тогда $$g \in L_1(\Om,\As, f \nu)\ \Longleftrightarrow \
gf \in L_1(\Om,\As, \nu).$$

При этом верно равенство: $$\io g(\om) (f\nu)(d\om) = \io g(\om)
\fo \nu(d\om).$$
\end{prop}

\begin{proof}
Проверим для индикаторов $A \in \As$:
$$\io \gamma_{A}
(\om) (f \nu) (d\om) = (f\nu)(A) = \int \limits_A f \nu(d\om) =
\io \gamma_A (\om) \fo
\nu (d\om).$$

Дальнейшие рассуждения провести самим аналогично предыдущему доказательству.\\
\end{proof}

Определим произведение измеримых пространств и пространств с
мерой.
\begin{df}
Произведением измеримых пространств $(\Om_1, \As_1)$ и
$(\Om_2, \As_2)$ называется измеримое пространство,
обозначаемое символом:
$$
(\Om_1, \As_1) \times (\Om_2, \As_2) \ \mbox{
или }\ (\Om_1 \times \Om_2, \As_1 \otimes
\As_2), \mbox{ где }
$$
$\As_1  \otimes  \As_2$ - $\sigma$-алгебра,
порожденная полукольцом $\Ps$ прямоугольников;\
прямоугольником называется всякое множество вида $A_1 \times A_2,\
A_j \in \As_j.$
\end{df}
\begin{problem}
 Пусть $A_i\in \As_i$. Доказать, что $\Ps$ - полукольцо.
\end{problem}

Далее предполагается, что все рассматриваемые меры счетно
аддитивны.
\begin{df}
Произведение пространств с мерой:
$$ (\Om_1, \As_1,
\nu_1) \times (\Om_2, \As_2, \nu_2) = (\Om_1 \times
\Om_2, \As_1 \otimes \As_2, \nu_1 \otimes
\nu_2).$$

При этом (тензорное) произведение мер $\nu_1 \otimes \nu_2$
определяется так: на полукольце $\Ps$ прямоугольников оно
определяется равенством
$$(\nu_1 \otimes \nu_2) (A_1
\times A_2) = \nu_1(A_1)  \cdot  \nu_2(A_2).$$

Затем проверяется, что эта мера счетно аддитивна, после чего она
продолжается на порожденную полукольцом $\sigma$-алгебру.
\end{df}

Замечание о полных мерах.

Напомним, что мера $\nu$ на измеримом пространстве $(\Om,
\As)\ $  полна, если $\forall A \in \As$, из
того, что $\nu A =0$, следует, что
$$\forall B \subset A ,\  B \in \As\  \mbox{ и }\  \nu B
=0.$$


Если имеется неполная мера, то ее можно пополнить. Пополнение
выглядит так:
$$\overline{\As} = \{ A \in \Om; \ \exists B
\in \As, \ \exists C \in \As : A\vartriangle B
\subset C , \nu C = 0 \};\ \mbox{ если } A\in
\overline{\As},$$

 то полагаем, что $\overline{\nu} A = \nu B$.

Только что определенное произведение не обязано быть полной мерой,
но ее можно пополнить; \ тогда получится пополненное произведение
мер.

\section{Лекция}

\begin{theorem}[Фубини]
Пусть $f$ - числовая функция на $\Om_1  \times \Om_2.$

\textbf{1.} Пусть $f \in \overline{L}_0(\Om_1 \times \Om_2,
\As_1  \otimes \As_2, \nu_1\otimes\nu_2)$.

Тогда $$\forall \om _1 \in \Om_1, \  \Om_2 \ni \om_2 \mapsto
f(\om_1, \om_2) - \nu_2 \mbox{ - измерима и }$$
$$\forall \om_2 \in
\Om_2, \ \Om_1 \ni \om_1 \mapsto f(\om_1, \om_2) - \nu_1 \mbox{ -
измерима.}$$

\textbf{2.} Пусть $f \in \overline{L}_1(\Om_1  \times \Om_2,
\As_1 \times \As_2, \nu_1\otimes\nu_2)$.

Тогда для $\nu_2$-почти всех $\om_2 \in \Om_2$ функция
$$\om_1
\mapsto f(\om_1, \om_2) - \nu_1 \mbox{-интегрируема,}$$ причем
функция
$$\om_2 \mapsto \int \limits_{\Om_1} f(\om_1, \om_2)
\nu_1(d\om_1)$$ (определенная $\nu_2$ - почти всюду) будет
измерима и $\nu_2$-интегрируема и
$$
\int \limits_{\Om_1} \bigl( \int \limits_{\Om_2} f(\om_1, \om_2)
\, \nu_2\,(d\om_2)\bigr)\, \nu_1\,(d\om_1) \ \stackrel{(1)}=
$$
$$= \ \int \limits_{\Om_2} \bigl(
\int \limits_{\Om_1} f(\om_1, \om_2)\, \nu_1\,(d\om_1) \bigr) \,
\nu_2\,(d\om_2) = \int \limits_{\Om_1  \times \Om_2} f(\om_1,
\om_2) \, \nu_1 \otimes \nu_2 \, (d\om_1, d\om_2)$$ Если функция
$f$ неотрицательна, то предполагать интегрируемость необязательно
и равенство (1) выполнено автоматически.
\end{theorem}

Формулировка теоремы Фубини для функций на пополненном
произведении мер отличается только в пункте 1; \ в случае
пополненного произведения функция $\om_2 \mapsto f(\om_1, \om_2)$
определена и $\ \nu_2$-измерима лишь для $\ \nu_1$-почти всех
$\om_2,$ а функция $\om_1 \mapsto f(\om_1, \om_2)$ определена и $\
\nu_1$-измерима лишь для $\ \nu_2$-почти всех $\om_1.$


\begin{ex}
Пусть $\Om=\mathbb{N}$, $\As_N$ - $\sigma$-алгебра всех
подмножеств множества $\mathbb{N}$ и\\ $\nu(A)$ - число элементов
$A, \ A\subset \mathbb N$. Надо показать, что для $f:
\Om\rightarrow\R$ выполнено:

$$f \in L_1(\Om, \As,
\nu) \ \Longleftrightarrow \ \sum \limits_{n=1}^{\infty} |f(n)| <
\infty$$

$$\mbox{ и }\io \fo \nu(d\om) = \sum \limits_{n=1}^{\infty} f(n).$$

Нас интересует, как будет выглядеть в этом случае теорема Фубини.
Пусть
$$f: \Om_1 \times \Om_2 \rightarrow\R \qquad (\Om_1 = \Om_2 = \mathbb N)$$

$$
f \in L_1(\Om_1 \times \Om_2, \As_1 \times
\As_2, \nu_1 \otimes \nu_2) \ \Longleftrightarrow\  \sum
\limits_{n,k=1}^{\infty}|f(n,k)| <\infty,$$

$$\int \limits_{\Om_1
\times \Om_2} \fo (\nu_1 \otimes \nu_2) (d\om) = \sum
\limits_{n,k=1}^{\infty} f(n,k)$$

$$\int \limits_{\Om_2} \bigl(
\int \limits_{\Om_1} f(\om_1, \om_2) \, \nu_1 \, (d\om_1) \bigr)
\nu_2(d\om_2) = \sum \limits_{n=1}^{\infty} \sum
\limits_{k=1}^{\infty} f(n,k).$$

Значит, теорема Фубини в этом случае утверждает, что если ряд
сходится абсолютно, то суммы повторного и двойного рядов
совпадают.
\end{ex}
\begin{ex} Пусть
$ \bigl( f(n,k) \bigr) = \left(%
\begin{array}{ccccc}
   1 & -1 &  &  & 0 \\
  -1 &\ \ 1 &  &  &  \\
   &  & 1 & -1 & \vdots \\
   &  & -1 & \ \ 1 &  \\
  0 &  &  \ldots&  & \ddots \\
\end{array}%
\right)$ - элементы бесконечной матрицы - это значения,
принимаемые функцией $f$ на соответствующих парах $(n,k)$.

Тогда
$$\sum \limits_{n,k} |f(n,k)| = \infty \ \mbox{ и }\ $$
$$\int \limits_{\Om_2}\bigl( \int \limits_{\Om_1} f(n,k) \nu_1(dk)
\bigr) \nu_2 (dn) = 0 = \int \limits_{\Om_1}\bigl( \int
\limits_{\Om_2} f(n,k) \nu_2(dn) \bigr) \nu_1 (dk),$$ так что
повторные интегралы существуют и равны, хотя двойной интеграл не
существует.
\end{ex}
\begin{ex} Пусть
$ \bigl( f(n,k) \bigr) = \left(%
\begin{array}{ccccc}
   1 & -1 &  &  & 0 \\
   & \ \ 1 & -1 &  &  \\
   &  & \ \ 1 & -1 & \vdots \\
  0 &  &  \ldots&  & \ddots \\
\end{array}%
\right)$ - бесконечная матрица.

Тогда
$$\sum \limits_{n,k} |f(n,k)| = \infty , \quad \int
\limits_{\Om_2}\bigl( \int \limits_{\Om_1} f(n,k) \nu_1(dk) \bigr)
\nu_2 (dn) = 0,$$

$$\mbox{ но } \int \limits_{\Om_1}\bigl( \int
\limits_{\Om_2} f(n,k) \nu_2(dn) \bigr) \nu_1 (dk) = 1.$$

Здесь двойные интегралы существуют, но не равны.
\end{ex}
Стандартное обозначение $$l_p = L_p (\Om, \As, \nu) \
(\Om = \mathbb{N}).$$

$l_p$ - множество всех последовательностей натуральных чисел.

$$\sum
\limits_{n=1}^{\infty} |x_n|^p < \infty, \mbox{ при этом норма
равна } ||(x_n)||_{l_p} = \bigl(\sum \limits_{n=1}^{\infty}
|x_n|^p \bigr)^{\frac{1}{p}}.$$

\begin{proof}\textbf{теоремы Фубини}\ (в доказательстве предполагается, что меры
конечны).

1. Пусть $P$ - полукольцо всех прямоугольников

$$
A_1 \times A_2,\  \mbox{ тогда } \ (\nu_1 \otimes \nu_2 ) (A_1
\times A_2) = \nu_1 A_1 \cdot \nu_2 A_2.
$$

Докажем, что мера $\nu_1 \otimes \nu_2$ счетно аддитивна на $P$.
Пусть
$$A_1 \times A_2 = \bigsqcup \limits_{n} (A_1^n
\times A_2^n) \ \ (*).$$

Мы должны доказать, что
$$\nu_1 A_1 \cdot \nu_2 A_2 \  =\  (\nu_1 \otimes \nu_2)(A_1 \times A_2) \
=\ \sum \limits_{n=1}^{\infty} (\nu_1 \otimes \nu_2) (A_1^n \times
A_2^n) \  = \  \sum \limits_{n=1}^{\infty} \nu_1 A_1^n \cdot \nu_2
A_2^n.$$

Введем функцию $f(\om_1) = \nu_2 A_2 \cdot \gamma_{A_1} (\om_1)$ и
для каждого $n$ также введем функцию $$f_n(\om_1) = \nu_2 \,
A_2^n\,  \gamma_{A_1^n} (\om_1).$$

Утверждается, что из (*) следует, что $\forall \om_1 \quad
f(\om_1) = \sum \limits_{n=1}^\infty f_n(\om_1)$. Это утверждение
является верным, потому что мера $\nu_2$ - счетно аддитивна.

$$\epsfbox{picture.10}$$

$\forall \, \om_1 \in A_1 \ f(\om_1)\  =\  \nu_2 \, A_2 \  = \
\sum\limits_{n=1}^{\infty} \nu_2 \, A_{2}^{n} \ =\  \sum
\limits_{n=1}^{\infty} f_n(\om_1).$

Так как все функции неотрицательны, то по т. Беппо - Леви
справедливо равенство

$$\nu_2 A_2 \cdot \nu_1 A_1 = \int \limits_{\Om_1} f(\om_1) \nu_1
(d\om_1) = \sum \limits_{n=1}^{\infty} \int \limits_{\Om_1}
f_n(\om_1) \nu_1 (d\om_1) = \sum \limits_{n=1}^{\infty} \nu_2
A_2^n \cdot \nu_1 A_1^n  = \sum \limits_{n=1}^{\infty} (\nu_1
\otimes \nu_2) (A_1^n \times A_2^n)$$

Таким образом, мы проверили счетную аддитивность меры $\nu_1
\otimes \nu_2$ на $P$.

2. Докажем теорему Фубини сначала для индикаторов измеримых
множеств.

$$\epsfbox{picture.11}$$

Для индикаторов измеримых множеств, являющихся прямоугольниками,
она верна:

$$\int \limits_{\Om_1 \times \Om_2} \gamma_{A_1 \times A_2}(\om)
(\nu_1 \otimes \nu_2) (d\om) = \nu_1 A_1 \cdot \nu_2 A_2\ =$$
$$ =\  \int \limits_{\Om_1} \bigl( \int \limits_{\Om_2}
f(\om_1, \om_2) \nu_2 (d\om_2) \bigr) \nu_1 (d\om_1)\ = $$ $=\
\int \limits_{\Om_1} \nu_2 A_2 \ \nu_1 (d\om_1) = \nu_2 A_2\cdot
\nu_1 A_1 \Rightarrow$ для индикаторов доказано.

3. Значит, она верна для конечных линейных комбинаций индикаторов
прямоугольников. В частности, для индикаторов множеств,
принадлежащих кольцу $S(P),\ $ порожденному полукольцом $P.$

Пусть $\Phi$ - совокупность всех измеримых множеств таких, что для
их индикаторов теорема Фубини верна:
$$\Phi = \{ A \in
\As_1 \times \As_2, \text{для $\gamma_A$ теорема
Фубини верна.} \}.$$
$$\ A_1 \subset A_2 \subset A_3 \subset \ldots, \ A_j \in \Phi \
\Rightarrow $$ $$\Rightarrow\  \bigcup\limits_{j} A_j \in \Phi
\text{ (этот факт вытекает из теоремы Беппо-Леви).}$$

Известно, что для $\gamma_{A_j}$ - теорема верна, так как $A_j$
возрастает, то $$\gamma_{A_j} (\om_1, \om_2) \rightarrow
\gamma_{\cup A_j} (\om_1, \om_2),$$ следовательно, по теореме
Беппо-Леви:
$$\int \limits_{\Om_1 \times \Om_2} \gamma_{A_j} (\om_1, \om_2)
(\nu_1 \otimes \nu_2) (d\om) \rightarrow \int \limits_{\Om_1
\times \Om_2} \gamma_{\cup A_j} (\om_1, \om_2) (\nu_1 \otimes
\nu_2) (d\om),$$ покажем, что для таких объединений теорема Фубини
верна.

Далее, если $A, B \in \Phi,\ B \subset A,\ $ то
$\gamma_{A\backslash B} = \gamma_{A} - \gamma_B$ и так как теорема
Фубини верна для $\gamma_A$ и для $\gamma_B,\ $ то она верна и для
их разности. Итак, совокупность $\wp$ обладает следующими
свойствами:
\begin{enumerate}
\item $\Om \in \Phi;$

\item $A_1, \, A_2 \in \Phi\ \Rightarrow\  A_1 \backslash A_2 \
\in \Phi;$

\item Если $\ \forall \, n \quad A_n \in \wp, \ A_1 \subset
A_2\subset A_3\subset \ldots,\ $ то
$$
\bigcup\limits_{n} A_n \in \Phi;
$$

\item $S(P) \subset \Phi.$
\end{enumerate}

Пусть $\Phi_0$ - минимальная совокупность множеств со свойствами
(1)-(4), так что $\Phi_0\subset \Phi.$

Докажем, что, если $A_1,\,  A_2\ \in \wp,\ $ то $\ A_1 \cap A_2 \
\in \Phi_0.$

Пусть
$$\Phi_1\ = \ \{A\in \Phi\colon \ \forall\, B\in S(P),\ A\cap
B \ \in \Phi_0\}.\ $$ Тогда
$$
(A) \qquad S(P) \subset \Phi_1 \ (\mbox{ так как } \forall \, A\in
S(P),\ \forall \, B\in S(P), \ A\cap B\  \in S(P) \subset \Phi_0;
$$
$$
(B) \qquad \mbox{ если  $A_j \in \Phi_1$ и $A_1 \subset A_2
\subset A_3 \subset \ldots,$ то } \ \cup A_j \ \in
\Phi_1.\qquad\qquad\qquad\quad
$$

Действительно, раз $A_j \in \Phi_1,\ $ то
$$
\forall \, B\in S(P), \ \ A_j \cap B \ \in \Phi_0\quad
\Longrightarrow \bigcup\limits_{j}\, A_j\cap B \ \in \Phi_0.
$$
$$
(C) \qquad \mbox{ если  $A_1, \, A_2 \in \Phi_1$ и $A_1 \supset
A_2,\ $ то } \ A_1\backslash A_2\ \in
\Phi_1.\qquad\qquad\qquad\qquad\quad
$$

Действительно, раз $A_j \in \Phi_1,\ $ то
$$
\forall \, B\in S(P), \ \ A_j \cap B \ \in \Phi_0\quad (A_1 \cap
B) \supset (A_2\cap B)
$$
и, значит, в силу свойства (2) системы $\Phi_0,$
$$
(A_1 \cap B) \backslash (A_2\cap B)\ = \ (A_1 \backslash A_2) \cap
B \ \in \Phi_0,
$$
а это значит, что $A_1\backslash A_2\ \in \Phi_1.$ То, что $\Om
\in \Phi_1,\ $ очевидно.

Таким образом, система $\Phi_1$ обладает теми же свойствами
(1)-(4) и содержится в $\Phi_0;\ $ так как $\Phi_0$ - минимальная
система с такими свойствами, то $\Phi_1 = \Phi_0.$

Пусть
$$
\Phi_2\ = \ \{A\in \Phi_0\colon \ \forall \, B \in \Phi_0, \ A
\cap B \ \in \Phi_0\}.
$$

Тогда система $\Phi_2$ снова обладает свойствами (1)-(4).
Действительно,
$$
S(P) \subset \Phi_2, \ \mbox{ так как } \ \forall \, B\in S(P)
\quad \forall \, A \in \Phi_1, \ A\cap B \ \in \Phi_0,\ \mbox{ а
}\  \Phi_1 = \Phi_0.
$$
Свойства (2), (3) доказываются аналогично.

Таким образом, система $\Phi_2$ обладает свойствами (1)-(4) и
содержится в минимальной системе $(\Phi_0)$ с такими свойствами.
Значит, $\Phi_2 = \Phi_0.$\\

Таким образом, система $\Phi_0$ обладает свойствами (1)-(4) и
сверх того следующим свойством:
$$
\forall \, A,\, B \in \Phi_0,\ A\cap B \in \Phi_0.
$$
Но это значит, что $\Phi_0$ - это $\sigma$-алгебра, содержащая
$S(P)$ и, значит, содержащая минимальную $\sigma$-алгебру,
содержащую $S(P)$ (в действительности она с ней совпадает, так как
минимальная $\sigma$-алгебра, содержащая $S(P),$ также обладает
свойствами (1)-(4)).\\

Итак, доказано, что для индикатора каждого множества из
$\As_1 \otimes \As_2$ (то есть для каждого
измеримого множества) теорема Фубини верна. Тогда она верна и для
конечных линейных комбинаций таких индикаторов, то есть для
простых функций, а значит, и для их монотонных пределов, то есть
для произвольных неотрицательных измеримых функций;\ так как
произвольная измеримая функция - это разность неотрицательных, то
теорема Фубини доказана полностью.

\end{proof}

\section {Лекция}

\begin{theorem}[Радона-Никодима]
Пусть $(\Om, \As)$ - измеримое пространство, $\ \nu, \mu$
- меры на сигма-алгебре $\As$ множеств подмножества $\Om$
(неотрицательные конечные).

Пусть также $\mu$ - абсолютна непрерывна относительно $\nu$ $(\mu
<< \nu)$, то есть из того, что $\nu A =0$ следует, что $\mu A =0.$

Тогда $$\exists f \in L_1(\Om, \As, \nu): \mu = f \nu,$$
то есть $$\forall A \in \As \ \mu(A) = \int \limits_A \fo
\nu (d\om) ).$$
\end{theorem}
\begin{proof}
Рассмотрим новую меру $$\eta= \mu + \nu, \mbox{ то есть } \forall
\, A \in \As \quad \eta(A) = \mu(A) + \nu(A).$$

Рассмотрим новое пространство $L_2(\Om, \As, \eta)$ и
функционал $F$ на нем: $$F(\varphi) = \io \varphi(\om)
\mu(d\om).$$

Докажем, что всякая функция $\ff\in L_2 (\Om, \As, \eta)$
$\ \mu$-интегрируема.

Действительно, если $\varphi \in L_2(\Om, \As, \eta),$ то
$\varphi \in L_1(\Om, \As, \eta):$
$$
\bigl(\io |\varphi| \, d\eta \bigr)^2 \leqslant \io |\varphi|^2 \,
d\nu \cdot \io 1 \, d\eta = \io |\varphi|^2 \, d\eta \cdot
\eta(\Om);
$$
поэтому, если $\ff\in L_2 (\Om, \As, \eta),$ то
$$
\infty\ > \ \io |\varphi| \, d\eta = \io |\varphi| \, d\nu + \io
|\varphi| \, d\mu \  \Longrightarrow\  \varphi \in L_1 (\Om,
\As, \mu).
$$
При этом
$$
|F(\varphi)| = |\io \varphi \, d\mu| \leqslant \io |\varphi| \,
d\mu \leqslant \io |\varphi|\, d\eta \leqslant ||\varphi||_{\eta}
\cdot (\eta(\Om))^{\frac{1}{2}}$$
Следовательно, функционал $F$ - непрерывен.\\
\\
\\
В силу теоремы Рисса об общем виде линейного непрерывного
функционала на гильбертовом пространстве можно утверждать, что
$$\exists \,  g \in L_1(\Om, \As, \eta) \colon \, F(\varphi) =
(g,\varphi) = \io g(\om) \varphi(\om) \eta (d\om).$$

Рассмотрим множество $$A_0 = \{\om \in \Om: g(\om) \geqslant 1
\}.$$

Докажем, что это множество имеет нулевую меру $\eta$.
$$F(\varphi) = \io
\varphi(\om) \mu (d\om) = \io g(\om) \varphi(\om) \eta(d\om) = \io
g(\om) \varphi(\om) \mu (d\om) + \io g(\om) \varphi(\om)
\nu(d\om).$$

Пусть $\varphi = \gamma_{A_0}. \ $ Тогда:

$$F(\gamma_{A_0}) = \io \gamma_{A_0}(\om) \, d\mu = \mu(A_0) = \int
\limits_{A_0} g(\om) \mu (d\om) + \int \limits_{A_0} g(\om)
\nu(d\om) \ \geqslant \mu(A_0) + \nu(A_0) \ \Longrightarrow$$
$$\Longrightarrow \ \nu(A_0) \leqslant 0 \ \Rightarrow \ \nu(A_0)
=0\  \Rightarrow \  \mu (A_0) =0.$$

В итоге,
$$F(\varphi) = \int \limits_{\Om \setminus A_0} \varphi
\, d\mu = \int \limits_{\Om \setminus A_0} g(\om) \varphi(\om)
\eta( d\om) = \int \limits_{\Om \setminus A_0} g \varphi \, d\mu +
\int \limits_{\Om \setminus A_0} g \varphi \, d\nu \
\Rightarrow$$

$$\int \limits_{\Om \setminus A_0} (1 - g) \varphi \, d\mu = \int
\limits_{\Om \setminus A_0} g \varphi \, d\nu, \quad  \forall\,
\varphi \geqslant 0, \ \varphi \in L_2.$$

Используя предельный переход, получим, что это неравенство верно
$\forall \, \varphi \geqslant 0.$

Пусть
$$\forall\, A \in \As \quad \varphi_A (\om) =
\frac{\gamma_A(\om)}{1 - g(\om)}.$$

Тогда
$$\mu(A) = \io \gamma_A(\om) \, d\mu = \io \frac{g(\om)}{1-
g(\om)} \gamma_A (\om) \, d\nu = \int \limits_A \frac{g(\om)}{1-
g(\om)} \, d\nu.$$

В частности, если $A = \Om$ и $f(\om) \ = \ \frac{g(\om)}{1 -
g(\om)},\ $ то
$$\mu(\Om) = \io \frac{g}{1-g} \,
d\nu = \io \fo \dn;$$  следовательно, $f$ - интегрируема по мере
$\nu$. Итак, мы получили, что $$\forall \, A \in \As,
\quad \mu(A) = \int \limits_A \fo \dn.$$
\end{proof}

\subsection{Обобщенные функции. Теория обобщенных функций.}

Пусть $D(\R)$ - пространство финитных, бесконечно
дифференцируемых функций на $\R$.

$$\epsfbox{picture.12}$$

\begin{df}
Говорят, что последовательность $\varphi_n \in D$ - сходится к
нулю, если выполняются следующие условия:\\
$\begin{aligned} (1) & \forall k \quad
|\varphi_n^{(k)} (t)| \rightarrow 0 \text{ равномерно по } t; \\
(2) & \exists \, a>0:  \ \forall n , \ \forall t: \ |t|>a\
\Rightarrow\ \varphi_n(t) = 0.
\end{aligned}$
\end{df}
\begin{problem}Показать, что на пространстве $D$ нельзя ввести метрику, сходимость
относительно которой совпадает с только что определенной.
\end{problem}
\begin{df}Говорят, что последовательность $\ff_n \in D$ сходится к
$\ff \in D$:
$$\varphi_n \stackrel{\nsb}\rightarrow \varphi \in D(\R)\
\Longleftrightarrow \  \varphi_n - \varphi \stackrel{n \rightarrow
\infty}\rightarrow 0.$$
\end{df}

Пусть $D = \bigcup \limits_{n=1}^{\infty} D_{[-n,n]}, \  \mbox{
где } D_{[-a,a]} = \{ \varphi \in D; \  \forall t \notin [-a,a];\
\varphi(t) =0 \},$ \\
тогда:

$$D_{[-a,a]} \ni \varphi_n \rightarrow 0 \ \Longleftrightarrow\
\forall k \ |\varphi_n^{(k)}(t)|\rightarrow 0 \ \mbox{ (при
$\nsb$, равномерно по $t$).} $$\\

Мы хотим ввести топологию на $D_{[-n,n]}$. Она задается с помощью
семейства норм: $$k=0,1,... \ \ p_k(\varphi) = ||\varphi||_k =\max
\limits_{t \in [-n,n]} |\varphi^{(k)}(t)|.$$
\begin{problem}
Показать, что сходимость последовательности функций в этой
топологии совпадает с той, которую мы только что определили.
\end{problem}



Топология в $D_{[-n,n]}$ - это та топология, которая задается
семейством норм $\{p_k \}$.\\
Сходимость $\varphi_n \rightarrow 0$ в $D_{[-n,n]}$ эквивалентна
тому, что $\forall\, k \ \ p_k(\varphi_n) \rightarrow 0 $. Это
совпадает с нашим определением.

Определим топологию в $D$.
\begin{df}
Выпуклое подмножество $V \in D$ называется открытым тогда и только
тогда, когда:
$$\forall n \ V\cap D_{[-n,n]} \mbox{ открыто в }
D_{[-n,n]}.$$
\end{df}

Произвольное множество в $D$ называется открытым, если его можно
представить в виде счетного объединения выпуклых открытых
множеств.

\begin{problem}
Показать, что сходимость в $D$ относительно только  что
определенной топологии - это та, которую мы определили вначале.
\end{problem}

\begin{ex}

$$\epsfbox{picture.13}$$

\[f(t) = \begin{cases}
 \ 0 & \text{если $t \leqslant 0 $},\\
 \ e^{-\frac{1}{t^2}}  & \text{иначе}.\\
   \end{cases} \]

Рассмотрим $\psi(t) = f(-t), \  \varphi_{ab}(t)
   = \psi(t-b) f(t-a), \ a < b.$

Значит, $\varphi_{ab}(t) \in D,\ $ так что пространство $D$
нетривиально (состоит не только из нуля). Пусть $g(t) = \ff_{a-
\varepsilon,\  a}(t) - \ff_{b, \ b +\varepsilon} (t), \ a < b$  и
$$
F_{a,b}^{\varepsilon}(t) = c \int \limits_{-\infty}^t g(x) \, dx,
\quad c=\frac{1}{\int \limits_{- \infty}^{\infty} \varphi_{a -
\varepsilon,\  a} (x) \, dx}.
$$

%$$\epsfbox{pictures.eps}$$
\end{ex}

\newpage
Обобщенная функция $F$ над $D$ - это линейный непрерывный
функционал на
$D$, $F \in D'$.\\
\\
Пусть, как обычно, $L_1(\R^1)$ - пространство всех
(классов эквивалентности) интегрируемых по Лебегу функций на
прямой:
$$
L_1^{loc}(\R^1) - \mbox{ пространство всех локально
интегрируемых по Лебегу функций,}
$$
то есть $f \in L_1^{loc}(\R^1) \ \Longleftrightarrow\
\forall a>0 \ f \gamma_{[-a,a]} \in L_1(\R^1)$.\\
\\
Примером типичной функции из $L_1^{loc}(\RR)$ является любой
многочлен.

Теперь построим вложение $L_1^{loc}(\RR)\rightarrow D'$. Для любой
функции $g\in L_1^{loc}(\RR) $ зададим соответствие
$$g \mapsto F_g \in D'\text{, где } F_g(\varphi)= \int
\limits_{-\infty}^{\infty} g(t) \varphi(t) \, dt.$$

$$F_{\lambda_1 g_1 + \lambda_2 g_2} = \lambda_1 \cdot F_{g_1} +
\lambda_2 \cdot F_{g_2}\  - $$ - то
есть отображение линейно.\\
Нужно доказать, что $F_g = 0$ в $D'$, то есть для любой функции
$\varphi \in D\ $ $ F_g(\varphi) = 0 $ тогда и только тогда, когда
$g(t) = 0$ почти всюду.
\begin{ex}
Пусть $\varphi \in D \mapsto \varphi'(0)$. Этот функционал не
совпадает ни с одним функционалом $F_g$.
\end{ex}
Каждой мере $\nu$ можно сопоставить $F_{\nu} = \int \limits_{\RR}
\varphi \, d\nu$ (но функционал $\ff\mapsto \ff^{\prime} (0)$ не
будет образом никакой меры и тем более никакой функции из
$L_1^{loc}(\RR)$).

\section{Лекция}

Ранее было введено пространство $D$: $$D= \bigcup
\limits_{n=1}^{\infty} D_{[-n,n]}.$$
\begin{df}
Топология в $D$ задается семейством $\Ps$. $\Ps$
состоит из всех полунорм $p$, то есть $$p \in \Ps
\Leftrightarrow \forall n \ p \upharpoonright D_{[-n,n]} - \mbox{
непрерывная полунорма на } D_{[-n,n]}.$$
\end{df}
\begin{problem}
Доказать, что это определение топологии эквивалентно предыдущему.
\end{problem}
Напомним, что для любой $\psi \in L_1^{loc} (\RR)$ имеется
отображение: $$\psi \mapsto F_{\psi} \in D',\  \mbox{ где }
F_{\psi}(g) = \int \limits_{-\infty}^{\infty} \psi(x) g(x) \,
dx.$$

Надо доказать, что если $\forall g \ \ F_{\psi}(g) = 0,$ то
$\psi(x) = 0$ почти всюду.

Заметим, что $\forall [a,b] \subset \Rr$
$$
\gamma_{[a,b]}(x) = \lin g_n(x) \quad \forall n \ \  g_n \in D.
$$

%$$\epsfbox{picture.17}$$

Для $\forall\, n$ функция $g_n$ определяется следующим образом.

\[g_n(x) = \begin{cases}
 1, & \text{если } x \in [a,b]\\
 0, & \text{если } x \notin [a- \frac{1}{2n}, b+ \frac{1}{2n}]\\
 & \text{иначе, промежуточное значения}
\end{cases} \]

Например, можно положить $g_n (x) = F_{a,\  b}^{1/2 n} (x).$

Для любого $n \in \mathbb{N}$ выполняется следующее:
$$F_{\psi}(g_n) = 0 = \int \limits_{\Rr} g_n(x) \psi(x) \, dx
\ \longrightarrow\  \int \limits_{\Rr} \gamma_{[a,b]}(x) \psi(x)
\, dx = \int \limits_{[a, b]} \psi(x) \, dx.$$

По определению измеримого по Лебегу множества $A \subset \Rr,$
$$
\forall \varepsilon> 0 \quad \exists\, P_{\varepsilon} - \mbox{
конечное объединение отрезков: }\  \nu (A\triangle
P_{\varepsilon}) < \varepsilon.$$

Отсюда и из предыдущего следует, что:
$$
\forall \varepsilon >0 \quad \int \limits_{P_{\varepsilon}}
\psi(x) \, dx = 0 \ \Longrightarrow\  \int \limits_{A} \psi(x) \,
dx =0;
$$

так как это верно для каждого $A,$ в частности, для $\ A_{+} =
\{x\colon \ \psi (x) \geqslant 0\}$  и для $\ A_{-} = \{x \colon\
\psi(x) \leqslant 0\}, \ \mbox{ то }$
$$
\psi(x) = 0 \ \mbox{ почти всюду.}$$

Таким образом, мы доказали, что вложение $L_1^{loc} (\RR)$ в
$D^{*}$ инъективно. Это вложение дает нам право называть $D^{*}$ -
пространством обобщенных функций.
$$
L_1^{loc} (\RR) \subset D^*.
$$

Аналогично определяется и вложение в $D^{*}$ пространства
$\mathfrak{M}(\RR) \in D^*$ мер на $\RR:$
$$\mathfrak{M}(\RR) \ni \nu \mapsto F_{\nu} \in D^*; \quad
F_{\nu}(g)  = \int \limits_{\RR} g(x) \, dx.$$

Аналогично можно доказать, что, если для любой функции $g \in D \
F_{\nu}(g) = 0,$
$$
\text{ то $\forall A $ - барелевское
множество } A\subset \Rr, \ \nu(A) = 0.
$$

Подробнее. Вначале надо доказать, что $$\forall [a,b] \subset \Rr
\ \ \nu([a,b]) = 0.$$ А потом перейти к множеству А (оно может
быть аппроксимировано отрезками с любой степенью точности).

Введем теперь операции над обобщенными функциями:
\begin{enumerate}
\item Умножение на число. Пусть $\Ec(\mro) =
\Ec$ - векторное пространство бесконечно дифференцируемых
функций на $\mro.$ Пусть $\ff \in \Ec, F \in D'$, тогда
$\forall \, g\in D, \ \ff g \in D$ и произведение $\ff F$
определяется так:
$$
(\varphi\cdot F, g) = (F, \varphi \cdot g),\quad  [F(g) \equiv
(F,g)], \quad \forall\, g\in D.
$$


Проверим, что это определение согласуется с обычным:\\
\\Рассмотрим $\psi \in L_1^{loc} (\Rr)$ и $\varphi \in
\Ec.$ Тогда $\varphi \psi \in L_1^{loc}(\Rr).$

Проверяем:
$$
F_{\varphi \psi}(g)  = (F_{\ff \psi}, g) = \int \limits_{\Rr}
\varphi(x) \psi(x) g(x) \, dx = \int \limits_{\Rr} \psi(x) \cdot
(\varphi \cdot g) \, dx = (F_{\psi}, \varphi \cdot g) = (\ff \cdot
F_{\psi}, g).$$

\item Дифференцирование. Если $F \in D', \ \varphi \in D$, то по
определению считаем, что
$$
(F', \varphi) = - (F, \varphi'),
\varphi' \in D.
$$

Проверим, что это определение согласуется с обычным.

Пусть $$g \in C^1(\Rr) \subset L_1^{loc}(\Rr), \ \mbox{ тогда }$$
$$\bigl( (F_g)',
\varphi \bigr) = - (F_g, \varphi') = - \int \limits_{\Rr} g(x)
\varphi'(x) \, dx = \int \limits_{\Rr} g'(x) \varphi(x) \, dx =
(F_{g'}, \ff),$$ то есть $(F_y)^{\prime} = F_{y^{\prime}}.$

\end{enumerate}
Можно говорить вместо ``обобщенная функция сопоставленная
обычной`` - ``рассмотрим обычную функцию как обобщенную`` \  и
записывать
$$(F_g, \varphi) = (g, \varphi).$$
$$\Ec \subset C_1 (\Rr) \subset L_1^{loc} (\Rr) \subset
D'$$

Если ввести топологию на $D'$, то эти пространства будут плотны в
$D'$. Определим $\delta$-функцию Дирака следующим образом:
$$
(\delta, \varphi) = \varphi(0) = \int \limits_{\Rr} \varphi(x)
\nu_{\delta} (dx) = \varphi(0) \cdot \nu_{\delta}(\{0\})
=\varphi(0),
$$
причем $\nu_{\delta} (A) = 1,$ если $A \ni 0,$ и $\nu_{\delta} (A)
= 0,$ если $A \not\ni 0.$

\begin{problem}
Найти функцию $\Theta$, такую что $\Theta' = \delta$.
\end{problem}

Введем $S$ - пространство всех бесконечно дифференцируемых быстро
убывающих функций, причем  $D \subset S \subset \Ec.$ То
есть:
$$
\varphi \in S \ \Longrightarrow\  \varphi \in \Ec \mbox{ и }\  \forall\, n, k \in \mathbb{Z}_+ \quad \sup (1+x^2)^n
|\varphi^{(k)} (x)| < \infty.$$

В этом пространстве можно ввести семейство полунорм:
$$\Ps \ni p \  \Longleftrightarrow\  \exists n,k: \ p_{nk} = \max
\limits_{x \in \Rr} (1+ x^2)^n |\varphi^{(k)} (x)|, \ \ \varphi
\in S .$$

Заметим, что метрика:
$$\rho (\varphi, \psi) = \sum \limits_{k,n}
\frac{1}{2^{kn}} \frac{ p_{nk}(\varphi - \psi) } {1+
p_{nk}(\varphi - \psi) }\ \  $$ - определяет такую же топологию в
$S$, что и семейство этих норм. Отметим еще раз, что $D \subset
S.$

\begin{problem}
Доказать, что $D$ - плотно в $S$.
\end{problem}

\begin{problem} Напомним, что
$D = \bigcup \limits_{n=1}^{\infty} D_{[-n,n]}$. При этом:
$$\forall n \quad  D_{[-n,n]} \subset S \subset \Ec.$$
\end{problem}

Топология в $D_{[-n,n]}$, порожденная топологией пространства $S$
совпадает с той, которую ввели в $D_{[-n,n]}$ раньше, но для $D$
это не так.

\begin{note}
 Пусть $F$ - линейный функционал на $D$. $F$ - непрерывен тогда и
 только тогда, когда
$\forall\, n $  сужение $F$ на $D_{[-n,n]}$ - непрерывно.

Докажем $\Leftarrow$ ($\Rightarrow$ очевидно). Пусть
$\Ps_D$ - семейство полунорм в $D$, которое задает
топологию на $D$. Пусть полунорма определяется так:
$$
p_F(\varphi) = |F(\varphi)|.
$$
Сужение $p_F$ на каждое $D_{[-n,n]}$ непрерывно. Следовательно, по
определению $\Ps_D, \ p_F \in  \Ps_D.$

Функционал $F$ - непрерывен на $D$, если
$$
\exists\, p \in \Ps_D \ \ \exists\, c > 0 :\quad
|F(\varphi)| \leqslant c p(\varphi).
$$ (Это достаточное условие непрерывности
функционала.) В качестве $p$ можно взять $p_F (\varphi)$.
\end{note}

Введем пространство $(D^*, \sigma(D^*, D)), \mbox{ где }
\sigma(D^*, D)$ - слабая топология.

Сходимость $F_n \rightarrow F$ в слабой топологии равносильна
тому, что
$$(F_n, \varphi) \rightarrow (F, \varphi) \quad \forall \, \ff \in D.$$

То есть:
$$\Bigl[D^*
\ni F_n \rightarrow F \in D^* \mbox{ в слабой топологии }
\sigma(D^*, D)\Bigr] \ \Longleftrightarrow$$
$$\ \Longleftrightarrow\
\Bigl[\mbox{ (по определению слабой топологии) } \ \forall\,
\varphi \in D \quad (F_n, \varphi) \rightarrow (F,
\varphi)\Bigr].$$

\begin{theorem}
Пусть $F$ - линейный функционал на $D$, причем
$$
(F_n, \varphi) \rightarrow (F, \varphi) \quad \forall\, \varphi
\in D\ \mbox{ и }\  \forall\, n \ F_n \in D^*.$$

Тогда функционал $F$ будет непрерывен.
\end{theorem}

\begin{proof}
В силу теоремы Банаха-Штейнхауза, примененной к $D_{[-n,n]}, \ $
$\forall k$ сужение $F$ на $D_{[-n,n]}$ - непрерывно,
следовательно, $F \in D^*$.
\end{proof}

Про это свойство говорят, что $D^*$ - слабо секвенциально полно.


\subsection{Преобразование Фурье.}

Преобразование Фурье функции $f \in L_1 (\Rr),$ определяется так:
$$f \mapsto \w{f}, \ \w{f}(z) = \int \limits_{\Rr} e^{-izx} f(x)
\, dx$$

и обратное преобразование Фурье
$$f(x) = \frac{1}{2\pi} \int \limits_{\Rr} e^{izx} \w{f} (z) \, dz.$$

\section{Лекция}

\textbf{Свойство 1.} Пусть $$f \in C^1 (\Rr) \bigcap L_1(\Rr)\
\mbox{ и }\ f' \in L_1 (\Rr).$$ Тогда преобразование Фурье
производной функции $f:$
$$
\fw'(z) = iz \fw(z).
$$


\begin{proof}
$$\fw'(z) =\lin \int \limits_{-n}^{n} e^{-izx} f'(x) \, dx = \lin
f(x) e^{-izx}\mid_{-n}^{n} + \int \limits_{-n}^n iz f(x) e^{-ixz}
\, dx.$$

Так как $f(x) = f(0) + \int \limits_0^x f'(t) \, dt,\ \ $ то:
$$
f(x) \stackrel{x\rightarrow \infty}{\rightarrow} C_1 \ \mbox{ и }\
f(x) \stackrel{x\rightarrow -\infty}{\rightarrow} C_2,
$$

так как $f(x) \in C^1(\Rr),\ \mbox{ то } C_i=0.$ Поэтому
$$
\fw'(z) = iz \int \limits_{-\infty}^{\infty} f(x) e^{-izx} \,
dx.$$
\end{proof}
\textbf{Свойство 2.} Если $f \in L_1$ и $[x \mapsto x f(x)] \in
L_1$, то $\fw$ - дифференцируема и
$$
\widehat{(-i x {f}(x))} (z) =
(\fw)' (z).
$$

\begin{proof}
Так как, каковы бы ни были $\alpha, \, \beta\in \mro,$
$$
|e^{i \beta} - e^{i \alpha}| \ \leqslant\  |\beta - \alpha|
$$
(это неравенство выражает тот факт, что длины дуги между точками
$\alpha$ и $\beta$ на окружности больше расстояния между ними, то
есть длины хорды; \ его можно получить с помощью теоремы Лагранжа
о конечном приращении) и, следовательно,
$$
\left| \frac{e^{- ix(z + \Delta z)} - e^{- ixz}}{\Delta z} \right|
\  \leqslant \ |x|,
$$
причем функция $x \mapsto x f(x)$ по условию интегрируема, то,
применяя теорему Лебега о мажорированной сходимости, получим

$$
(\fw)'(z)\  =\  \lim \limits_{\triangle z \rightarrow 0} \ir
\frac{e^{-ix(z + \triangle z)} - e^{-ixz}}{\triangle z} f(x) \,
dx\ =
$$

$$\ = \ \ir (-ix) f(x) e^{-ixz}
\, dx = \widehat{(-ix f(x))} (z).$$
\end{proof}
\textbf{Свойство 3.} Если
$$
f_n \in L_1 \mbox{ и } f_n \rightarrow f \mbox{ в }L_1, \ \mbox{
то есть } \|f_n - f \|_{L_1} \rightarrow 0, \  \mbox{ то }$$
$$
\fw_n (z) \rightarrow \fw (z) \mbox{ равномерно по } z \in \RR.
$$

\begin{proof}
$\forall z \in \Rr$ имеем:
$$
|\fw_n(z) - \fw(z)|\  =\  \left| \ir \bigl( f_n(x) - f(x) \bigr)
e^{-ixz} \, dx\right|\  \leqslant \  \ir |f_n(x) - f(x)| \, dx\
=\  ||f_n - f ||_{L_1} \ \stackrel{\nsb}\rightarrow\ 0.
$$
\end{proof}
\textbf{Свойство 4.} Пусть $f \in L_1,\ $ тогда $\fw (z)
\rightarrow 0,\ $ если $ |z| \rightarrow \infty$.\\
\\
\begin{proof}
Докажем это сначала для случая, когда $f$ - индикатор отрезка.
$$
\w{\gamma_{[a,b]}}(z)\  =\  \ir \gamma_{[a,b]} (x) e^{-izx} \, dx\  =\  \int
\limits_a^b e^{-ixz} \, dx\  = \ - \frac{1}{iz} (e^{-izb} -
e^{-iza})\  \longrightarrow 0, \ \mbox{ при  } |z| \rightarrow \infty.
$$

Значит, это верно для функций, являющихся конечными линейными
комбинациями индикаторов отрезков. Так как множество таких функций
плотно в $L_1 (\mro)$, то для любой функции $f\in L_1$ существует
последовательность $f_n$ конечных линейных комбинаций индикаторов
отрезков, такая что
$$
f_n \rightarrow f \mbox{ в } L_1.
$$

Следовательно,
$$
|\fw_n(z) - \fw(z)| \rightarrow 0, \ \mbox{ при } \nsb,\ \mbox{
равномерно по } z \in \Rr.
$$

Так как $\forall n \quad \fw_n \stackrel{|z| \rightarrow 0}
\rightarrow 0,\  $ получим, что $\fw(z) \rightarrow 0.$

\end{proof}

Отметим, что преобразование Фурье функции из $L_1$ не обязательно
принадлежит $L_1$ (таково, например, преобразование Фурье индикатора отрезка).\\
\textbf{Свойство 5.} Пусть $f \in L_1$ и $a \in \RR, \  a \not =
0.$ Тогда:
$$
\widehat{f\left(\frac{x}{a}\right)}(z) = \ir
f\left(\frac{x}{a}\right) e^{-ixz} dx = \ir f(v) e^{-iavz} a \, dv
= a \w{f}(az).
$$
(мы воспользовались заменой $\frac{x}{a} = v$).\\
\textbf{Свойство 6.}
$$
\w{f(x+a)}(z) \ =\  \ir f(x+a) e^{-ixz} \, dx \ = \  \ir f(v)
e^{-iz(v-a)} \, dv \ =\  e^{iza} \fw(z).
$$
(мы воспользовались заменой $x+a=v$).\\

\begin{prop}
Пусть $f \in L_1, \varphi \in L_1 $, тогда
$$
\ir \fw(z) \varphi(z) \, dz = \ir f(z) \widehat{\varphi}(z) \, dz\
 \  \mbox{ \textbf{(равенство Парсеваля).}}
$$
\end{prop}

\begin{proof}
$$\ir \fw (z) \varphi(z) \, dz\  =\  \ir \left( \ir f(x) e^{-ixz} \, dx \right)
\varphi(z) \, dz \ =
$$
$$
=\  \ir \left( \ir \varphi (z) e^{-ixz} \, dz \right) f(x) \, dx \
=\  \ir f(x) \widehat{\varphi(z)} \, dx.
$$

Равенства выполняются в силу теоремы Фубини, так как $(x,z)\mapsto
f(x)\varphi(z)$ интегрируема на плоскости (снова по т. Фубини) и
$$
\int \limits_{\R^2}| f(x)\varphi(z)| \, dx \, dz = \ir
|f(x)| \, dx \ir |\varphi(z)| \, dz.$$ Так интегралы в правой
части конечны, то, произведение $f \cdot \ff$ интегрируемо как
функция двух аргументов.

\end{proof}

\begin{ex}
Найдем преобразование Фурье функции
$$
f(x) = \frac{1}{\sqrt{2 \pi}} e^{-\frac{x^2}{2}}.
$$
\end{ex}

Заметим, что  $f'(x) + f(x) x = 0.$ Отсюда следует, что $(\fw)'(z)
+ z \fw(z) = 0,$ так как
$$
(\fw)'(z)\  =\  \w{ ( -ixf(x))} (z) \  =\ \w{i f'} (z) \ =\  i
\cdot i z \fw (z)\  =\  - z \fw(z).
$$

Общее решение уравнения $\w{f}^\prime (z) + z \fw(z) = 0$ имеет
вид $$\fw(z) = C e^{-\frac{z^2}{2}};$$  постоянная $C$ вычисляется
так:

$$
\fw (z)\  =\  \frac{1}{\sqrt{2 \pi}} \ir e^{-\frac{x^2}{2}} e^{-i
x z} \, dx\  \Rightarrow
$$
$$
\Rightarrow\  \fw(0)\  =\  C
=\frac{1}{\sqrt{2 \pi}} \ir e^{-\frac{x^2}{2}} \, dx \  =\
\frac{1}{\sqrt{2 \pi}} \sqrt{2 \pi}\  =\  1 \Rightarrow$$ $$\fw
(z) \  =\ e^{-\frac{z^2}{2}}.
$$

\subsection{Свойства преобразования Фурье в $S$.}

Пространство $S$ преобразуется преобразованием Фурье в $S$ и более
того, $$\Phi : S \rightarrow S \mbox{ непрерывно (это будет
доказано ниже).}$$

Предположим, что $\varphi \in S$ и мы знаем, что $\wphi \in S$ и
по $\wphi$ хотим
восстановить $\varphi$.\\
Пусть $\psi, \widehat{\psi} \in S;$ тогда
$$
\ir \varphi\left(\frac{z}{a} \right) \widehat{\psi} (z) \, dz =
\ir a\  \varphi(x)\, \widehat{\psi} (ax) \, dx  = \ir \varphi(x)\,
\widehat{\psi} \left(\frac{x}{a}\right) \, dx = \ir \wphi(x)\,
\psi\left(\frac{x}{a}\right) \, dx
$$
(мы воспользовались заменой $z = ax,$ затем свойством 5 и
равенством Парсеваля).

Переходя к пределу при $a \rightarrow \infty,$ получим
$$\varphi(0) \ir \w{\psi(z)} \, dz = \psi(0) \cdot \ir
\wphi(z) \, dz.$$

Пусть $\varphi(x) = \frac{1}{\sqrt{2\pi}} e^{-\frac{x^2}{2}} \
\Rightarrow\  \wphi (z) = e^{-\frac{z^2}{2}}.$ Тогда
$$
\frac{1}{\sqrt{2 \pi}} \ir \widehat{\psi}(z) \, dz = \psi(0)
\sqrt{2\pi} \ \Longrightarrow\  \psi(0) = \frac{1}{2\pi} \ir
\widehat{\psi} (z) \, dz.
$$

Далее нужно воспользоваться формулой (свойство 6):
$$
\w{f(x+a)}(z) = e^{-iza} \fw(z).$$

Пусть $\psi_1(x) = \psi (x+a); \quad$ тогда $\ \psi_1(0) =
\frac{1}{2\pi} \ir \w{\psi}_1(z) \, dz$,
$$
\mbox{а так как, по свойству 6,}\quad \w{\psi}_1 (z) = \w{\psi
(x+a)} (z) = e^{iza} \, \w{\psi} (z), \ \mbox{ то }
$$
$$
\psi(a) = \frac{1}{2\pi} \ir e^{iza} \widehat{\psi}(z) \, dz.
$$

Таким образом, мы получили
$$
\psi(x) = \frac{1}{2\pi} \ir e^{ixz} \widehat{\psi}(z) \, dz;
$$
поэтому, определив обратное преобразование Фурье функции $f$
равенством
$$\check{f}(x) = \frac{1}{2\pi} \ir e^{ixz} f(z) \, dz,
$$
получим $\ \psi = \check{\w{\psi}}.$

Из определения обратного преобразования Фурье следует, что его
свойства аналогичны свойствам (прямого) преобразования Фурье. В
частности, обратное преобразование Фурье тоже переводит $S$ в $S$.

Докажем, что
$$
\forall \, g \in S \quad \exists \, \varphi \in S: \quad g =
\wphi.
$$

Применим к $g$ обратное преобразование, получим функцию $\ff$ из
$S$, к ней применим прямое преобразование Фурье, получим функцию
$g:\ \varphi = \check{g}$ и $g = \w{\ff} = \w{\check{g}}$.
$$
\mbox{Иначе говоря, } \ \forall g \in S \quad \Phi \Phi^{-1} g = g
\quad (\Phi\mbox{ - преобразование Фурье}).
$$

Аналогично можно доказать и равенство
$$
\forall\, \ff \in S \quad  \Phi^{-1} \Phi \,\varphi = \varphi.
$$
Следовательно, $\Phi$ - взаимно однозначное отображение $S$ на
$S$.

\begin{df}
Преобразование Фурье обобщенной функции $F \in S^*$ определяется
следующим образом:
$$(\widehat{F}, \varphi) = (F, \wphi).$$
\end{df}

Пусть $g \in L_1$. Тогда утверждаем, что $\widehat{F}_g =
F_{\widehat{g}}$ (согласованность определений преобразования Фурье в $S^{*}$  и в $L_1$).\\
Проверим это:
$$
(\w{F}_g, \varphi)\  =\  (F_g, \wphi)\  =\  (g, \wphi)\  =\  \ir
g(x) \wphi(x) \, dx \ =\  \ir \widehat{g}(x) \varphi(x) \, dx =
(\w{g}, \varphi) = (F_{\widehat{g}}, \varphi).
$$

Проверим, что так определенный линейный функционал $\w{F}$ на $S$
непрерывен, то есть действительно является элементом пространства
$S^{*}.$ Так как функционал $\w{F}$ линеен, достаточно доказать
его непрерывность в нуле.

Итак, пусть $\ff_n \rightarrow 0$ в $S.$ Тогда
$$
(\w{F}, \ff_n)\ = \ (F, \w{\ff}_n) \ \longrightarrow\ 0,
$$
так как, в силу непрерывности преобразования Фурье в $S$ (она
будет доказана позже), $\w{\ff}_n \rightarrow 0$ в $S.$


\section{Лекция}

Преобразование Фурье в $S^{*}$ будем обозначать тем же символом
$\Phi;$ таким образом,
$$\Phi : S^* \rightarrow S^*.$$

\subsection{Свойства преобразования Фурье в $S^*$.}

\begin{enumerate}
\item Наделим $S^*$ слабой топологией $(S^*, \sigma(S^*, S))$.
Тогда сходимость в слабой топологии определяется так:
$$
F_n \rightarrow F \ \Longleftrightarrow\  \forall\, \varphi \in
S\quad (F_n, \varphi) \rightarrow (F, \varphi).$$ Покажем, что
$\Phi$ непрерывно в $(S^{*}, \sigma(S^{*}, S)).$ Таким образом, мы
должны проверить, что
$$
\mbox{ если }\ F_n \rightarrow F,\ \mbox{ то } \w{F}_n \rightarrow
\w{F} \ \Longleftrightarrow\  \forall \,\psi \in S \quad (\w{F}_n,
\psi) \rightarrow (\widehat{F}, \psi).
$$

Вот эта проверка: $(\widehat{F}_n, \psi) = (F_n, \widehat{\psi})
\rightarrow (F, \widehat{\psi}) = (\widehat{F}, \psi)$.\\
\\
\emph{Докажем аналогии свойств 1, 2 для преобразования Фурье в $S^{*}.$}\\

\item Было показано, что $\forall \, \ff\in S, \quad  \w{\ff'}(z)
= iz \wphi (z)$. Пусть $f \in S^*, \forall \varphi \in S$:
$$
(\w{f'}, \ff) = (f^{\prime}, \w{\ff}) = - (f,
\w{\varphi}^{\prime}) = - (f, - \w{ix \ff}) = - (\fw, -ix\ff) =
(ix \fw, \ff)\ \Rightarrow
$$
$\Rightarrow\  \w{f'} = ix \fw.$\\

\item Было показано, что $\forall \, \ff\in S, \quad
(\w{\varphi})^{\prime}(z) = \Bigl(\w{-ix \varphi(x)}\Bigr)(z)$.
Пусть $F \in S^*, \forall \varphi \in S:$
$$
(\w{F^{\prime}}, \ff) = - (\w{F}, \ff^{\prime}) = - (F, (\wphi)')
= - (F, ix \wphi (x)) = - (ixF, \wphi) = - ( \w{ixF}, \ff) \
\Rightarrow
$$
$\Rightarrow\  (\w{F})^{\prime} \ =\ - \w{ixF}. $
\end{enumerate}

Аналоги остальных свойств предлагается доказать в качестве
упражнения.

\begin{stm}
Образ любой функции из $S$ при преобразовании Фурье лежит в $S$.
Другими словами, $\Phi(S) \subset S$ (или $\varphi \in S\
\Rightarrow\  \wphi \in S$).
\end{stm}
\begin{proof}
Нужно проверить, что
$$\forall n,k \quad |(1+x^2)^n \wphi^{(k)}(x)|
\rightarrow 0, \mbox{ при } |x| \rightarrow \infty.
$$
Сделаем это:
$$
\left|(1+x^2)^n \cdot\wphi^{(k)}(x)\right| = \left| (1+ x^2)^n
\cdot \w{(-ix)^k \varphi}\right| = \left| \frac{(1+
x^2)^n}{x^{2n+2}} \cdot (ix)^{2n+2} \cdot \widehat{(-ix)^k
\varphi}\right|\  =
$$
$$
= \ \left|\frac{(1+x^2)^n}{x^{2n+2}} \w{\bigl( (-ix)^k
\varphi\bigr)^{2n+2}}\right| \rightarrow 0 \text{ при } |x|
\rightarrow \infty,
$$
так как
$$
\frac{(1+x^2)^n}{x^{2n+2}} \rightarrow 0 \text{ при } |x|
\rightarrow \infty,
$$
а функция под знаком преобразования Фурье справа содержится в $S$
и тем более в $L_1$ и, значит, ее преобразование Фурье ограничено.

\end{proof}

\begin{theorem}
Преобразование Фурье $\Phi: S \rightarrow S$ - непрерывно.
\end{theorem}

\begin{proof}
$$
\mbox{Пусть }\ \varphi_n \rightarrow 0 \ \mbox{ в } S\
\Longleftrightarrow \ \forall \,k,m \ \  x^k \varphi_n^{(m)}(x)
\rightarrow 0 \ \mbox{ равномерно по $x$ при $\nsb$.}
$$
Нам надо доказать, что $\w{\ff}_n \rightarrow 0$ в $S,$ то есть,
что $\forall k,m \quad z^k \wphi ^{(m)}_n (x) \rightarrow 0$
равномерно по $x$ при $\nsb,$ то есть, что:
$$
(-i)^k \w{((-ix)^m \, \varphi (x))^{(k)}} \
\stackrel{\nsb}{\rightarrow} \ 0
$$
равномерно по $x$. Последовательность функций под знаком
преобразований Фурье сходится к нулю в $S$. Мы покажем, что она
сходится к нулю в $L_1$; тогда последовательность ее
преобразований Фурье будет сходиться к нулю равномерно, что и
требуется.
$$
\mbox{Пусть }\ \forall \, m,k \quad g_{n}^{mk} = ((-ix)^m \,
\varphi_n (x))^k
$$
Докажем, что
$$ \ff_n \rightarrow 0 \mbox{ в } S \ \Longrightarrow \ \forall \, m,\,k \quad g_{n}^{mk}
\rightarrow 0 \mbox{ в } L_1.$$

Если $\ff_n \rightarrow 0 \mbox{ в } S,$ то
$$\ff_n (x) \rightarrow 0 \ \mbox{ и }\ (1 + x^2) \, \ff_n (x)
\stackrel{\nsb}{\rightarrow} 0 \mbox{ равномерно по } x \ (\mbox{и
даже в } S).$$

Кроме того,
$$\|\ff_n\|_{L_1} \  =\  \int\limits_{\mathbb R} |\ff_n (x)| \, dx \ = \
\int\limits_{\mathbb R} \left| \frac{(1 + x^2) \, \ff_n (x)}{(1 +
x^2)} \right| dx \ =$$
$$
= \ \il{- \infty}^{-A} \frac{|(1 + x^2) \, \ff_n (x)|}{1 + x^2} dx
+ \il{A}^{\infty} \frac{|(1 + x^2) \, \ff_n (x)|}{1 + x^2} dx +
\il{-A}^{A} |\ff_n (x)|\, dx
$$
Оцениваем получившиеся интегралы:
$$
\il{- \infty}^{-A} \frac{|(1 + x^2) \, \ff_n (x)|}{1 + x^2} dx
\leqslant \max |(1 + x^2) \, \ff_n (x)| \cdot \il{-\infty}^{-A}
\frac{1}{1 + x^2} dx;
$$
$$
\il{A}^{\infty} \frac{|(1 + x^2) \, \ff_n (x)|}{1 + x^2} dx
\leqslant \max |(1 + x^2) \, \ff_n (x)| \cdot \il{A}^{\infty}
\frac{1}{1 + x^2} dx;
$$
$$
\forall \, \varepsilon \quad \exists \, A>0\colon \il{A}^{\infty}
\frac{1}{1 + x^2} dx < \varepsilon \ \mbox{ и }\ \il{-\infty}^{-A}
\frac{1}{1 + x^2} dx < \varepsilon.
$$
Теперь выберем $n_0$ такое, что:
$$
\quad \forall \, n > n_0 \quad \il{-A}^{A} |\ff_n (x)| \, dx <
\varepsilon;
$$
последнее возможно, так как $\ff_n (x) \rightarrow 0$ равномерно
по $x.$ Следовательно, мы доказали, что $\|\ff_n\|_{\, L_1}
\rightarrow 0,$ а, следовательно, их преобразования Фурье
стремятся к нулю равномерно.
\end{proof}
\bigskip

Определим преобразование Фурье в $L_2 (\Rr)$. Докажем сначала, что
если $\ff, \, \psi \in S,$ то
$$
2\pi (\varphi,\psi)_{L_2} = (\wphi, \widehat{\psi})_{L_2} \qquad
(*).
$$
Равенство означает, что преобразование Фурье сохраняет норму $L_2$
(с точностью до умножения на $2 \pi$), а так как $S$ - плотно в
$L_2$, то тогда можно продолжить $\Phi$ на $L_2$.

Если $g \in L_2$ , то  (так как $S$ плотно в $L_2$):
$$
\exists \,\varphi_n \in S: \quad \varphi_n \rightarrow g \mbox{ в
} L_2.
$$

Значит, последовательность $\varphi_n$ - фундаментальна в $L_2,$
то есть
$$
\|\varphi_n - \varphi_{n+k}\|_{L_2} \rightarrow 0 \ \mbox{ при }
\nsb \ \mbox{ равномерно по } k.
$$

Так как нормы сохраняются (с точностью до умножения на $2 \pi$),
то
$$
\|\wphi_n - \wphi_{n+k}\|_{L_2} \rightarrow 0.
$$
Из условия полноты $L_2$ получим, что $\exists$ функция $\w{g}$ в
$L_2,$ такая что \  $\wphi \rightarrow \w{g}$ в $L_2$.

Эту функцию $\w{g}$ мы называем преобразованием Фурье функции $g
\in L_2$.

\begin{proof} (*)
Из формул прямого и обратного преобразования Фурье:
$$
\wphi = \int \limits_{\R} e^{-ixz} \varphi(x) \, dx\
\mbox{ и } \ \check{\ff} = \frac{1}{(2\pi)} \int
\limits_{\R} e^{ixz} \w{\ff} (z) \, dz.
$$
следует, что $\overline{\w{\overline{\varphi}}} = (2\pi)
\check{\varphi}.$ Итак,
$$
(\varphi, \psi)_{L_2} = \ir \varphi(x)
\overline{\psi}(x) \, dx = (\varphi, \overline{\psi}) =
(\widehat{\check{\varphi}}, \overline{\psi}) = (\check{\varphi},
\widehat{\overline{\psi}}) = 2 \pi (\check{\varphi},
\overline{\check{\psi}} )= 2\pi (\check{\varphi},
\check{\psi})_{L_2}.
$$

Мы доказали, что
$$
\mbox{ если } \varphi, \psi \in S,\ \mbox{ то } (\varphi,
\psi)_{L_2} = 2\pi (\check{\varphi}, \check{\psi})_{L_2}.
$$

Пусть $\varphi  = \wphi_1, \psi= \widehat{\psi}_1,$ тогда
$$(\wphi_1, \widehat{\psi}_1)_{L_2} = 2\pi (\varphi_1,
\psi_1)_{L_2}.$$
\end{proof}

\section{Лекция}

\begin{problem}
Пусть $g \in L_2 (\mro);$ тогда
$$
\forall \, \ff \in S \quad g \cdot \ff \in L_1 (\mro)
$$
и, более того, функционал $F_g (\ff) = \il{\mro} g \cdot \ff \,
dx\ $ непрерывен на $S,$ то есть является элементом пространства
$S^{*}.$ Поэтому определено его преобразование Фурье $\w{F_g}$,
про которое мы говорим, что это преобразование Фурье функции $g$ в
смысле теории обобщенных функций над $S.$  С другой стороны, мы
только что определили преобразование Фурье для функции из $L_2$
непосредственно. Покажем, что они совпадают , то есть что
$$
\w{F}_g = F_{\w{g}}.
$$
\end{problem}

\begin{proof}
Заметим, что в силу (непосредственного) определения преобразования
Фурье функции $g\in L_2 \quad \forall \, \ff \in S (\subset L_2):$
$$
(\w{g}, \ \w{\ff})_{L_2}\ = \ 2 \pi (g, \ \ff)_{L_2}.
$$
Поэтому  $\forall \, \varphi \in S$
$$
(\w{F_g}, \, \ff)\ = \ (F_g, \w{\ff})\ = \ \il{\mro} g(t) \,
\w{\ff} (t) \, dt\ = \ (g, \overline{\w{\ff}})_{L_2}\ =
$$
$$
\ = \ \frac{1}{2 \pi} (\w{g}, \w{\overline{\w{\ff}}})_{L_2}\ = \
(\w{g}, \overline{\ff})_{L_2}\ =\  (F_{\w{g}}, \ff).
$$

Следовательно $\w{F}_g = F_{\w{g}}.$ Что и требовалось доказать.
\end{proof}
\begin{note}
Известно, что $\forall \, t > 0$
$$
\frac{1}{\sqrt{2 \pi t}} \inr e^{- \frac{x^2}{2t}} e^{-ixz} \, dx
= e^{\frac{- tz^2}{2}}.
$$
Это значит, что
$$
\w{\frac{1}{\sqrt{2 \pi t}} e^{- \frac{x^2}{2t}}} (z) \ = \ e^{\frac{- tz^2}{2}}.
$$
Пусть теперь для каждого $\gamma = \rho\, e^{i \alpha} \colon\
\rho > 0, \ \alpha \in \left[-\frac{\pi}{2}, \
\frac{\pi}{2}\right],$
$$
g_{\gamma} (x) \ = \ \frac{1}{2 \pi \gamma} \  e^{-\frac{x^2}{2
\gamma}} \quad \mbox{ и } \quad \psi_{\gamma} (z) \ = \
e^{-\frac{\gamma z^2}{2}}.
$$
Тогда функции $\gamma \mapsto F_{g_\gamma} \ \in S^{*}\ $ и $\
\gamma \mapsto F_{\psi_\gamma}\  \in S^{*}$ аналитичны на области
$$
\{\gamma\ = \ \rho \, e^{i \alpha} \colon\ \rho > 0, \ \alpha \in
\left(-\frac{\pi}{2}, \ \frac{\pi}{2}\right)\}
$$
и непрерывны на множестве
$$ \{\gamma\ = \ \rho \, e^{i \alpha}
\colon\ \rho > 0, \ \alpha \in \left[-\frac{\pi}{2}, \
\frac{\pi}{2}\right]\} $$

Это значит, что $\forall \, \ff \in S,$ на этих областях
аналитичны (соответственно, непрерывны) функции
$$
\gamma \mapsto (F_{\psi_\gamma}, \ff), \quad \gamma \mapsto
(F_{g_\gamma}, \ff).
$$
При этом для $\gamma \in (0, \infty)\quad \w{F_{g_\gamma}} =
F_{\psi_\gamma}.$ Поэтому, в силу теоремы единственности для
аналитических функций, это равенство справедливо для всех
допустимых $\gamma.$

Для $\gamma \ = \ e^{i \frac{\pi}{2}}$ оно означает, что
$$
\w{\frac{e^{- \frac{i \pi}{4}} e^{\frac{i x^2}{2}}}{\sqrt{2 \pi}}
}  (z) \ = \ e^{- \frac{i z^2}{2}}.
$$

Таким образом, с помощью теории обобщенных функций мы сосчитали
преобразование Фурье неинтегрируемой функции на прямой.
\end{note}
\begin{note}
Пусть $F\in S^{*}$ и $g\in \Ec(\mro).$ Для некоторых
функций можно выбрать естественное продолжение функционала $F,$
первоначально определенного на $S,$ на более широкую область,
содержащую $g$.

%\begin{figure}
%\centering
%\includegraphics{picture.17}
%\end{figure}

Пусть $\psi \in D,$

\textbf{1.} \ $\psi (t) = 1,$ если $t \in [-1, 1].$

\textbf{2.} \ $\psi (t) = 0,$ если $t \notin [-1 - \varepsilon, 1
+ \varepsilon].$

Пусть еще для $a > 0 \quad \psi_a (t) = \psi (a\, t).\quad$ Ясно,
что $\forall \, t \quad \psi_a (t) \stackrel{a\rightarrow 0}
\rightarrow 1$ (кроме того, $\psi_{\alpha} \rightarrow 1$ и в
$S^{*}$).

Более того, $\forall \, \ff \in S,$
$$
\mbox{если}\ \  (\psi_a \cdot \ff) (t)\  =\ \psi_a (t) \cdot \ff
(t),
$$
$$
\mbox{ то }\quad \psi_a \cdot \ff \  \stackrel{a\rightarrow 0}
\longrightarrow \ \ff \ \mbox{ в }\  S \qquad (*)
$$
$$
\mbox{ и }\quad \ \psi_a (t) \cdot \ff (t) \ \longrightarrow \ \ff
(t) \quad \forall \, t
$$
\end{note}

\begin{problem}
Доказать, что выполняется $(*).$
\end{problem}
\begin{proof}
Из $(*)$ получаем, что
$$\psi_a \cdot F \  \stackrel{a\rightarrow
0} \longrightarrow \ F  \ \mbox{ в }\  S^{*} \qquad (**)$$

Чтобы проверить $(**),$ нужно посмотреть, выполняется ли:

$$\forall \, \ff \in
S \quad (\psi_a \cdot F, \ \ff) \  \stackrel{a\rightarrow 0}
\longrightarrow \ (F, \ \ff).$$


Проверим это:
$$
(\psi_a \cdot F, \ \ff) \  = \ (F, \ \psi_a \cdot \ff) \
\stackrel{a\rightarrow 0} \longrightarrow \ (F, \ \ff)
$$
Этим мы доказали, что $(*) \ \Rightarrow\ (**).$
\end{proof}

Теперь пусть $g\in \Ec, g\notin S$ и $F\in S^{*}$ обладают
следующим свойством:
$$
\exists \, \lim\limits_{\asn} \ (F, \psi_a \cdot g).
$$

Тогда по определению полагаем, что:
$$
(F, g) \  \stackrel{def} = \ \lim \, (F, \psi_a \cdot g).
$$

\textsl{Типичный \textbf{пример} такого расширения области
определения функционала  (обобщенной функции) $F$.}\\
\\
Пусть $F_1 = \delta, \  F_2 = \dw,$ тогда
$$
\forall \, g\in \Ec \quad (\delta, g) \ = \ g(0) \ \mbox{
и } \ (\dw, g) \ = \ -g^{\prime} (0).
$$
Это и есть продолжение.\\
\\
Пусть $\ff \in S(\R^2)$ и $F\in S^{*} (\mro).$ Ясно, что
$$
\forall \, \ff\in \mro \qquad \left[z \ \mapsto \  \ff (x, z)
\right] \in S
$$
Более того, мы можем проинтегрировать (сейчас - в обычном смысле)
и получим, что
$$
\left[z \ \mapsto \ \il{\mro} \ff (x, z) \, dz\right] \in S
$$
Мы хотим посчитать: \ $(F, \ \il{\mro} \ff (x, z) \, dz).$ Для
удобства записи введем традиционное обозначение действия:

при $\Phi \in S^{*}$
$$
(\Phi, \psi) \ = \ \inr \Phi (t) \psi (t) \, dt \quad-
$$
- то есть мы вводим символ\label{simv}, который по определению
равен левой части.

Итак,
$$
(F, \ \il{\mro} \ff \,(\cdot, z) \, dz) \ = \ \inr F(x) \inr \ff
(x, z) \, dz \ =
$$
$$
\ \stackrel{(1)}{=} \inr \ \left( \ \inr F(x) \ff (x, z)
\,dx\right) \,dz \quad \boxed{=}
$$
Далее без использования символа $\int$ напишем то же самое:
$$
\boxed{=} \quad \inr (F, \ff \,(\cdot, z))\, dz
$$

Мы хотим доказать (1). Сделаем это:

Этот интеграл $\inr \ff \, (\cdot, z) \, dz$ мы будем понимать,
как интеграл Римана, так что\\
$\inr \ff \, (x, z)\, dz \ -$ \ это предел последовательности
римановых сумм вида $\ \sum\limits_{j} \,\ff (x, z_j)\  \Delta
z_j,\ $ каждая из которых будет зависеть от $ \ x,\ $ и как
функция $x$ будет снова элементом пространства $S$. Полученная
последовательность будет сходиться к $\ \inr \ff \, (\cdot, z)\,
dz\ $ в пространстве $S.$

Следовательно, применение $F$ к этому интегралу можно понимать как
предел результатов применений $F$ к римановым суммам:
$$
\begin{CD}
(F, \, \sum\limits_{j} \ff \, (\cdot, \ z_j)\  \Delta z_j)\ @>>> \
(F, \ \int \ff \, (\cdot, \, z)\, dz )\\
@| @|\\
\sum\limits_{j} \left( (F, \, \ff\,(\cdot, \ z_j)) \ \Delta
z_j\right) @>>> \inr (F, \, \ff \,(\cdot, z)) \, dz
\end{CD}
$$
Учитывая, что $F\in S^{*}, \ \ff \in S,$ получаем:
$$
(\w{F}, \ff) = (F, \w{\ff}) = (F, \ \inr \ff (x) \cdot e^{-ixz} \,
dx)\ =
$$
$$
= \ \lim\limits_{\asn} \ (\psi_a \cdot F, \ \inr \ff(x) \cdot
e^{-ixz} \, dx)\ = \ \lim\limits_{\asn} \ (F, \ \inr \psi_a
(\cdot)\ \ff(x) e^{-ix \cdot}\, dx) \ =
$$
$$
= \ \lim\limits_{\asn} \ \inr \ff(x) \bigl(F, \ \psi_a (\cdot) \,
e^{-ix\cdot}\bigr)\, dx \quad \boxed{=}
$$
Предположим, что $\quad \exists \, \lim \limits_{\asn} \ (F, \
\psi_a (\cdot) \ e^{-ix\cdot}) = (F, \ e^{-ix\cdot}).\quad$ Тогда

$$
\boxed{=} \quad \inr \ff(x) \ \lim\limits_{\asn} \ (F, \psi_a
(\cdot) \ e^{-ix\cdot})\,dx \ = \ \left((F, \, e^{-ix\cdot}), \
\ff \right)
$$
Следовательно, мы получаем, что: $\w{F}(z) = (F, \, e^{-ixz}).$

Если $g\in L_1,$ то элемент $F_g \in S^{*}$ можно применить к
функции $x \mapsto e^{- ixz}$ (не являющейся элементом $S$):
$$
\w{F}_g \, (z)\  = \ (F_g, \, e^{-ixz})\ = \ \inr g(x) \, e^{-ixz}
\, dx.
$$

Таким образом, это определение преобразования Фурье действительно
согласуется с обычным определением. Примеры:

$$
\w{\delta} (z) = (\delta, \, e^{-ixz}) = 1,
$$
$$
\w{\dw} (z) = (\dw, \, e^{-ixz}) = iz,
$$
$$
\w{F} (z)\  = \ (F, \, e^{-ixz})\ = \ \inr F(x)\, e^{-ixz} \, dz \
- \ \mbox{ символ, который мы ввели ранее на стр. \pageref{simv}}
$$
Аналогично,
$$
\check{F} (z) = \frac{1}{2\pi} \, (F, \, e^{ixz}).
$$

\section{Лекция}

\subsection{Прямые и обратные образы обобщенных функций.}

\begin{df}
Пусть $E_1 = \R^n, \ E_2 = \R^k$ и $g$ -
отображение $E_1$ в $E_2,\ \ff$ - функция на $E_2$ и $\nu$ - мера
на $\sigma$-алгебре борелевских подмножеств $E_1.$

Тогда прямой образ $\gn \nu$ меры $\nu$ (если $g$ измеримо) и
обратный образ $\gv \ff$ функции $\ff$ определяются так:

$$
\gv \ff\ - \ \mbox{ это функция на $E_2,$ определяемая
равенством:}
$$
$$
(\gv \ff) (x)\ = \ \ff (g (x)),
$$

а $\gn \nu$ - это мера на $E_2$ (мы ее определили в разделе,
посвященном теории меры).
\bigskip

Пусть $F_1 \in S^{*} (E_1)\ $ и $\ F_2 \in S^{*} (E_2).$ Прямой
образ $\gn F$ обобщенной функции $F$ при отображении $g$
определяется так:
$$\gn F\in S^{*} (E_2),$$
$$
(\gn F,\, \ff) \ = \ (F, \, \gv \ff), \qquad \gv\ff\in
S(E_1),\quad\forall \,\ff \in S(E_2).
$$
Если $\ff\in S(E_2),$ то необязательно, что $\gv\ff \in S(E_1),$
то есть прямой образ определен лишь для тех $F,$ которые можно
продолжить на обратные образы функций $\gv \ff,\\  \ff\in S
(E_1)$.
\bigskip

Обратный образ обобщенной функции $F\in S^{*}(E_2)$ при
отображении $g$ определяется так:
$$
(\gv F,\ \ff)\ = \ (F, \ \gn\ff) \qquad \forall \, \ff\in S (E_1)
$$
(при определении прямого образа $\gn \ff$ функции $\quad \ff\in
S(E_1)$; она рассматривается как элемент пространства $S^{*}
(E_1)$)

Если $\forall\, \ff\in S \quad \gn \ff \in S,$  то обратный образ
определен для всех обобщенных функций.

Обозначение: $(\gv F)(x)\ = \ F(g(x)).$
\end{df}

\begin{note}
Если $\nu_\ff$ - это мера с плотностью $\ff$ и $\psi$ - плотность
меры $\gn \nu_\ff,$ тогда $$\gn \ff\ = \ \psi.$$
\end{note}

\begin{problem}
Доказать, что если $F$ - обычная функция, а $g$ - взаимно
однозначное отображение, то
$$
(\gn F)(x)\ = \ F (g^{-1}(x)).
$$
\end{problem}

\begin{ex}
Пусть $g: \mathbb R \rightarrow \mathbb R,$
$$g(x) = ax + b \mbox{
и }F = \delta.$$  Тогда, что такое обратный образ $\delta$-функции
$\gv \delta$ ?

Найдем $\gv \delta \ (= \delta (ax + b))$ следующим образом:
$$
(\gv \delta, \ \ff) \equiv \inr \delta \underbrace{(ax + b)}_{z}
\, \ff (x)\, dx \ = \ \inr \delta(z) \, \ff
\left(\frac{z-b}{a}\right) \frac{dz}{a} \ =
$$
$$
\ \frac{1}{a}\, \ff (- \frac{b}{a})\ = \ (\delta (ax + b) , \
\ff)\ =\ \il{\mro} \delta (ax + b) \, \ff (x)\, dx;
$$
В частности, $ \int \delta (x+ b) \, \ff (x) \, dx\ = \ \ff (-b).$
\end{ex}

\begin{problem}
Пусть есть аффинное отображение $g\colon \R^2 \rightarrow
\mro$
$$
(x_1, x_2) \mapsto ax_1 + bx_2 + c
$$
Найти
$$
(\gv \delta)(x)\ = \ \delta (ax_1 + bx_2 + c).
$$

\end{problem}

\begin{problem}
Доказать, что следующее выражение верно и найти $C$:
$$\w{e^{ixz}}
= C\cdot e^{-ixz} \ (\mbox{ экспоненты считаются обобщенными
функциями двух аргументов}).$$
\end{problem}
\begin{proof}
Найдем преобразование Фурье функции $e^{i(x,z)}\colon$
$$
\w{e^{i (x, z)}} (x_1, z_1)\ = \ \il{\Rr} \il{\Rr} e^{i (x, z)} \,
e^{-i (x, x_1)}\, e^{i (z, z_1)}\, dx \, dz\ = \
$$
$$
\ = \ \il{\Rr} e^{- iz z_1} \left( \il{\Rr} e^{i x z} \, e^{-i x
x_1} \, dx \right) \, dz\ = \ \il{\Rr} 2 \pi \cdot \delta (z -
z_1) \cdot e^{- i z z_1} \, dz\ = \ 2 \pi \cdot e^{-i x_1 z_1}.
$$
\end{proof}

\subsection{Свертка обобщенных функций.}
\begin{df}
Пусть $F_1, \ F_2 \in S^{*} (\mro).$ Тогда тензорное произведение
обобщенных функций $F_1 \otimes F_2 \in S^{*} (\R^2)$
определяется следующим образом:
$$
\left( F_1 \otimes F_2, \ \ff_1 (x) \cdot \ff_2 (z)\right)\ = \
(F_1, \ff_1)\cdot (F_2, \ff_2)
$$
По линейности и непрерывности его можно продолжить на все $S.$
Тензорное произведение можно определить и другим образом,
например:
$$
(F_1 \otimes F_2, \ \ff(x_1, \, x_2))\ = \ (F_1, \ (F_2, \,
\ff(x_1, \cdot))).
$$
\end{df}

%\begin{note}
%Можно взять два множества: $\{a_1,  a_2\}$ и $\{b_1, b_2, b_3\}$ с
%такими свойствами:
% \textbf{1.}$\quad\{a_1,  a_2\} \oplus \{b_1, b_2, b_3\}\ = \
%\{a_1, a_2, b_1, b_2, b_3\};$
% \textbf{2.}$\quad$ Декартово произведение: $\{a_1,  a_2\}\times
%\{b_1, b_2, b_3\};$
%\textbf{3.}$\quad$ Размерность пространства: $(\ff_1, \, \ff_2)
%\mapsto \ff_1 \cdot \ff_2$ ($\ff_1$ зависит от $a_1,  a_2;$
%$\ff_2$ - от $b_1, b_2, b_3$).
%
%\end{note}

\begin{ex}
Пусть, например:\\
$$F_1 \ = \ g_1 (\cdot) \ \in\ L_1;$$
$$F_2 \ = \ g_2 (\cdot) \ \in\ L_1.$$

Тогда тензорным произведением  этих функций будет их простое
произведение:
$$
(g_1 \otimes g_2) (x_1, x_2)\ = \ g_1 (x_1)\cdot g_2(x_2)
$$
\end{ex}

\begin{df}
Определим свертку двух обобщенных функций. Пусть
$$
\Phi\colon (x_1, \, x_2) \mapsto x_1 + x_2, \quad \R^2
\rightarrow \mro.
$$
Тогда свертка обобщенных функций $F_1, F_2$ определяется так:
$$
(F_1 * F_2) \ = \ \Phi_{*} (F_1 \otimes F_2)
$$
(конечно, это значит, что свертка обобщенных функций $F_1$ и $F_2$
определена не всегда).
\end{df}

Пусть $f_1$ и $f_2\in L_1 (\R^1),$ тогда
$$
(f_1 * f_2, \, \ff)\ = \ (\gn(f_1 \cdot f_2), \, \ff)\ = \ (f_1
\cdot f_2, \, \gv\ff)\ = \ (f_1 \cdot f_2, \, \ff(x_1 + x_2))\ =
$$
$$
\ = \ \il{\R^2} f_1 (x_1) \cdot f_2(x_2)\, \ff
\overbrace{(x_1 + x_2)}^{V}\, dx_1 \, dx_2\ =
$$
$$
\ = \ \il{\R^2} f_1(x_1)\, f_2(V-x_1)\, \ff(V)\,dV\,dx_1\
=
$$
$$
\ = \ \il{\mro} \left(\il{\mro} f_1(x_1)\, f_2(V-x_1)\,dx_1
\right)\ff(V)\,dV \ = \ \left( \il{\mro} f_1(x_1)\,
f_2(V-x_1)\,dx_1, \ \ff\right)
$$
Следовательно,
$$
(f_1 * f_2)(V) \ = \ \il{\mro} f_1(x)\, f_2(V-x)\,dx_1
$$
(отметим еще раз, что $\ \left[(x_1, x_2) \mapsto \ff(x_1 +
x_2)\right]\notin S(\R^2),\ $ даже если $\ff \in S
(\mro)$).

Чтобы было все корректно, необходимо проверить, что $f_1 (x_1)
\cdot f_2 (x_2) \in L_1 (\R^2).$

Пользуясь теоремой Фубини, получим:
$$
\int \il{\R^2} |f_1 (x_1)\cdot f_2 (x_2)|\, dx_1\, dx_2 \
=
$$
$$
= \ \il{\mro} |f_1 (x_1)| \left( \il{\mro} |f_2 (x_2)|\, dx_2
\right) \, dx_1 < \infty
$$
Следовательно, мы доказали, что $f_1 (x_1) \cdot f_2 (x_2) \in L_1
(\R^2).$

Теперь проверим, что $$f_1 (x_1) \cdot f_2 (V - x_1) \in L_1
(\R^2),$$  пользуясь теоремой Фубини:
$$ \int
\il{\R^2} |f_1 (x_1) \cdot f_2 (V - x_1)|\, dx_1\, dV\ =
$$
$$
\ = \ \il{\mro} |f_1 (x_1)| \left(\il{\mro} |f_2 (V - x_1)| \, dV
\right)\, dx_1 < \infty.
$$
Заметим, что $\ \forall \, x_1$ функция $\left[ V \mapsto |f_2 (V
- x_1)|\right] \in L_1,$ так как $f_2(x_1) \in L_1
(\R^1),$ следовательно, интеграл $\il{\mro} |f_2 (V -
x_1)| \, dV$ конечен.
\bigskip

Поэтому интеграл $\int f_1 (x_1) \cdot f_2 (V - x_1) dx_1$
определен для почти всех $V$ и функция $$f_1 * f_2 \in L_1.$$

То, что мы только что проверили, можно записать еще в такой форме:
$$
F_{f_1 * f_2}\ = \ F_{f_1} * F_{f_2}.
$$

Сделаем замену переменных: пусть $V - x_1 = z,$  тогда
$$
f_1 * f_2\ = \il{\mro} f_1 (x_1) \cdot f_2 (V - x_1) \, dx_1\ = \
\il{\mro} f_1(V - z)\cdot f_2(z)\, dz
$$
Следовательно, мы можем сделать вывод, что свертка является
коммутативной операцией.

Определим преобразование Фурье свертки сначала для функций из
пространства $L_1.$
$$
\w{f_1 * f_2} (z)\ = \ \il{\mro} \left( \il{\mro} f_1 (x_1) \cdot
f_2 (V - x_1)\, dx_1 \right) \,e^{-iVz} \, dV \ \boxed{=}
$$

Опять здесь применима теорема Фубини. Пусть $$V - x_1 = x_2 \
\mbox{ и }\ V = x_1 + x_2,$$  тогда

$$
\boxed{=} \ \il{\mro} \il{\mro} f_1 (x_1) \cdot f_2 (x_2) \,
e^{-iz \,(x_1 + x_2)} \, dx_1 \, dx_2\ = \
$$
$$
\ = \ \il{\mro} f_1 (x_1) \, e^{-iz x_1} \, dx_1 \cdot \il{\mro}
f_2 (x_2) \, e^{-iz x_2} \, dx_2 \ = \ \w{f_1} (z) \cdot \w{f_2}
(z)
$$
Пусть $F \in S^{*} (\mro)\quad$ и
$$
(\w{F},\, \ff)\ = \ ((F, \, e^{-ixz}), \, \ff)\ = \ \il{\mro}
\left( \il{\mro} F(x) \, e^{-ixz} \, dx \right) \ff (z)\, dz.
$$
Тогда:
$$
(\w{F_1 * F_2}, \, e^{-iz \cdot})\ =
$$
$$
= \ \il{\mro} \il{\mro} F_1 (x_1) \cdot F_2 (x_2) \, e^{-iz \,(x_1
+ x_2)} \, dx_1 \, dx_2 \ \stackrel{\mbox{по ``обобщенной
т.Фубини``}}{=} \ \w{F_1}\cdot \w{F_2}
$$
Свертка обобщенных функций определена не всегда.\\
\\
$$
\w{\delta} (z)\ = \ (\delta,\, e^{-ixz})\ = \ 1.
$$

Если $\ff\in S,$ то
$$
\w{\ff(x+a)}(z)\ = \ e^{iaz} \, \w{\ff} (z),
$$
тогда
$$
\w{\delta(x+a)}(z)\ = \ e^{iaz} \, \w{\delta}(z) = e^{iaz} \quad
\mbox{ и } \quad \cw \ = \ \check{e^{iaz}}\ = \ \delta (x+a)
$$

$$\epsfbox{picture.19}$$

$$\psi_a (x)\ = \ \psi (ax)$$
$\psi(a)  \stackrel{\asn}{\rightarrow} 1 \mbox{ в }  S^{*}.$

Мы можем написать для обратного преобразования Фурье равенство
$$
\check{\psi} (a) (x)\ = \ \frac{1}{2 \pi} \il{\mro} \psi_a (z) \,
e^{ixz} \, dz
$$
$$
\check{1}(x)\ = \ \lim\limits_{\asn} \frac{1}{2 \pi} \il{\mro}
\psi_a (z) \, e^{ixz} \, dz\ = \ \frac{1}{2 \pi} \il{\mro} e^{ixz}
\, dz
$$
$$
\check{1}(x)\ = \ \frac{1}{2 \pi} \inr e^{ixz} \, dx\ = \ \lin
\frac{1}{2 \pi} \int\limits^{n}_{-n} e^{ixz} \, dx
$$
$$
\check{e^{iaz}}(x)\ = \ \delta (x+a)\ = \ \frac{1}{2 \pi}
\il{\mro} e^{iz(x+a)} \, dx
$$
\begin{stm}
Операция дифференцирования свертки выглядит так:
$$
(F_1 * F_2)^{\prime} \ = \ (F_1^{\prime} * F_2)\ = \ (F_1 *
F_2^{\prime})
$$
\end{stm}

\begin{proof}
$$\w{(F_1 * F_2)^{\prime}} (z)\ = \ (iz) \w{F_1} (z) \cdot \w{F_2}(z)\ = \ $$
$$
\ = \ \bigl(iz \cdot \w{F_1}(z)\bigr) \cdot \w{F_2} (z)\ = \
\w{F_1^{\prime}}(z) \cdot \w{F_2}(z)\ = \ \w{F_1^{\prime}
* F_2}
$$
\end{proof}

Пусть $D$ - дифференциальный оператор  в пространстве обобщенный
функций:
$$
(D\ff)(x)\ = \ \sum\limits_{j=0}^{n} a_j \, \ff^{(j)}(x)
$$
Фундаментальная функция $F$ дифференциального оператора
определяется равенством: $$DF = \delta.$$

\begin{theorem} Пусть $F$ - фундаментальная функция
дифференциального оператора $D,$ тогда решением уравнения $D \Phi
= G \in S^{*}$ является:
$$\Phi \ = \ F*G.$$
\end{theorem}

\begin{proof}
$$D\Phi \ = \ D(F*G)\ = \ (DF * G) \ = \ \delta * G.$$
Докажем, что $\delta * G = G:$
$$
\w{\delta * G}\ = \ \w{\delta} \cdot \w{G}\ = \ 1 \cdot \w{G}\  =
\ \w{G}\ \Rightarrow \ \delta * G \ = \ G
$$
\end{proof}

\begin{df}
Рассмотрим задачу Коши:
$$
\left\{
\begin{aligned}
\frac{\partial F}{\partial t} \ &= \ D(F(t))\\
F (t) &\stackrel{t \rightarrow 0}{\rightarrow} G \in S^{*}\\
\end{aligned}
\right.
$$
Где $F\colon (0, a) \rightarrow S^{*}, \ \forall\, t \  F(t)\in
S^{*}, \ G\in S^{*}.$

Тогда функция $F$ называется решением задачи Коши. Фундаментальным
решением задачи Коши называется такое $F,$ что
$$
F(t) \ \stackrel{t\rightarrow 0}\longrightarrow \delta.
$$
\end{df}

\begin{theorem}
Рассмотрим  также задачу Коши:
$$
\left\{
\begin{aligned}
\frac{\partial \Phi}{\partial t} \ &= \ D(\Phi(t))\\
\Phi (t) &\stackrel{t \rightarrow 0}{\rightarrow} G \in S^{*}\\
\end{aligned}
\right.
$$
Тогда ее решением является: $\Phi(t)\ = \ (F(t)*G).$
\end{theorem}

\begin{proof}
$$
\frac{\partial \Phi}{\partial t} \ = \ \frac{\partial (F(\cdot)
* G)}{\partial t} (t) \ = \ \left( \frac{\partial F}{\partial t} * G
\right)\ = \ (DF(t) * G)\ = \ D(F(t) * G)\ = D(\Phi(t)).
$$

Так как $F(t) \stackrel{t \rightarrow 0}{\rightarrow} \delta,$ то:
$$
\Phi(t) \stackrel{t \rightarrow 0}{\rightarrow} (\delta * G )\ = \
G.
$$
\end{proof}

\section{Лекция}

\begin{note}
Было $S\supset D$ и $S^{*} \supset D^{*}.$ Преобразование Фурье не
переводит $D$ в себя. Пусть $\w{D}$ - множество преобразований
Фурье и $\check{D}$ - множество обратных преобразований Фурье
функций из $D$ с топологиями, заимствованными из $D.$
$$
\mbox{ Тогда }\ \check{D}\  =\  \w{D}\ \equiv \ Z\subset S,
$$
но топология в $Z$ - не та, которая порождается топологией
пространства $S$ (а сильнее).

Преобразование Фурье (обратное преобразование Фурье) элемента
$F\in D^{*}$ - это функционал на $Z,$ определяемый так:

$$\forall \, \ff \in Z, \ F\in D^{*}, \ff\in D: \quad
(\w{F}, \ \ff)\ = \ (F, \ \w{\ff}); \quad (F, \  \check{\ff}).
$$


Предположим, что:
$$
f(x)\ = \ e^{bx}, \quad S^{*} \not\ni F_{f} \in D^{*};
$$
ранее было показано, что $\w{\delta (z+a)} (x) \ = \ e^{-
iax}(z).$

Аналогично,
$$
\w{\delta (z + ib)} (x) \ = \ e^{bx};
$$
$$
(\delta (z - ib), \ \psi)\ = \ \psi(ib), \quad \forall \, \psi\in
Z = \w{D}
$$
\end{note}

\subsection{Теория линейных операторов в гильбертовом пространстве.}

В этом разделе рассматриваются гильбертовы пространства над полем
комплексных чисел. Скалярное произведение $(\cdot, \cdot)$ в
комплексном гильбертовом пространстве $H$ обладает свойствами:
\begin{enumerate}
\item $\forall \, x\in H \quad (x, x) \geqslant 0, \qquad (x, x) =
0 \ \Leftrightarrow\  x = 0; $

\item $\forall \, x_1,\, x_2,\, z\in H, \quad \forall \, \alpha,
\beta \in \mathbb{C}\qquad (\alpha x_1 + \beta x_2, \ z)\ = \
\alpha (x_1, \ z) + \beta (x_2, \ z);$

\item $\forall \, x_1, \, x_2 \in H \qquad (x_1, x_2) \ = \
\overline{(x_1, x_2)}.$

\end{enumerate}

Из (2) и (3) следует, что $\forall \, x_1, \, x_2, \, z \in H,
\quad \forall \, \alpha, \, \beta \in \mathbb{C}:$
$$
(z, \ \alpha x_1 + \beta x_2)\ = \ \overline{\alpha} \cdot (z,\
x_1) + \overline{\beta} \cdot (z, \ x_2).
$$

Действительно,
$$
(z, \ \alpha x_1 + \beta x_2 )\ = \ \overline{(\alpha x_1 + \beta
x_2, \ z)}\ = \ \overline{\alpha} \cdot \overline{(x_1, \ z)} +
\overline{\beta} \cdot \overline{(x_2, \ z)}\ = \
\overline{\alpha} \cdot (z,\ x_1) + \overline{\beta} \cdot (z, \
x_2).
$$

Примером комплексного гильбертова пространства является
комплексное пространство $L_2 (\mro),$  состоящее из всех (классов
эквивалентности) - комплекснозначных функций  $f,$ для которых:
$$
\il{\mro} |f|^2 \, dx < \infty,
$$
со скалярным произведением $(\cdot, \cdot),\ $ определяемым так:
$$
(f_1, \ f_2)\ = \ \il{\mro} f_1 (x) \, f_2 (x) \, dx.
$$

Корректность определения проверяется аналогично тому, как это было
сделано для вещественного случая. Аналогично определяется для
каждого пространства с мерой $(\Om, \As, \nu)$
комплексное пространство  $L_2 (\Om, \As, \nu)$ и, в
частности, комплексное $l_2.$


Напомним определение сопряженного оператора:
$$
(A^{*}x, \ z)\ = \ (x, \ Az) \qquad \forall\, x, z\in H
$$

\begin{df}
Пусть $H$ - гильбертово пространство. Оператор $A\in L(H)$
называется нормальным, если выполняется следующее условие:
$$
A^{*} \cdot A\ = \ A \cdot A^{*}
$$


\end{df}

\begin{df}
Оператор $V$ называется унитарным, если:
$$
V^{*} \cdot V \ = \ V \cdot V^{*} = I \qquad \forall \, x \quad I
\cdot x = x
$$
\end{df}

\begin{df}
Оператор $A$ называется самосопряженным, если:
$$
\quad A^{*}\ = \ A.$$
\end{df}
\emph{Унитарные и самосопряженные операторы являются нормальными.}

\begin{problem}
Привести пример не единичного оператора, являющегося одновременно
самосопряженным и унитарным.
\end{problem}

\subsection{Спектр операторов в гильбертовом пространстве.}

Пусть $A\in L(H).$ Говорят, что $\la\in \mathbb{C}$ не принадлежит
спектру $A$ $\Bigl(\la\notin \spec A\Bigr),$ если оператор
$A_{\la} \ = \ A - \la I$ обладает обратным, который всюду
определен, то есть тогда и только тогда, когда
$$\exists \, A_{\la}^{-1} (\in L(H)).$$

\begin{ex}
Пусть $H = L_2(\mro)$ и
$$
(Af)(x)\ = \ g_{A} (x) \cdot f(x),
$$
где $g_A$ является непрерывной комплекснозначной функцией. Для
того, чтобы оператор был линейным и непрерывным необходимо и
достаточно, чтобы функция $g$ была ограниченной.
\end{ex}

\begin{problem}
Показать, что
$$
\|A\|\ = \ \sup\limits_{x} |g_A (x)|.
$$
\end{problem}

\begin{problem}
Доказать, что:
$$
(A^{*} f)(x)\ = \ \overline{g}_{A} (x) \cdot f(x).
$$
\end{problem}

Следовательно, $A^{*} \cdot A\ = \ A \cdot A^{*},$ а, значит,
каждый такой оператор нормален.

\begin{problem}
Оператор $A$ является унитарным $\Longleftrightarrow \ |g_A (x)|\
= \ 1 \quad \forall \, x.$
\end{problem}

\begin{problem}
Оператор $A$ является самосопряженным $\Longleftrightarrow \ g_A
(x)\in \mro \quad \forall \, x.$
\end{problem}

Аналогично определяется в $L_2 (\Om, \Bs, \nu)$ оператор
умножения на (комплексную) существенно ограниченную функцию.

\begin{problem}
Доказать, что:
$$
\|A\|\ = \ess\limits_{\om \in \Om} |f(\om)| \ = \ (\mbox{то есть
сущеcтвенная верхняя грань})\ =
$$
$$
\ = \inf_{\substack{N\in \Bs\\ \nu N = 0}} \sup
\{|f(\om)|: \ \om\in \Om \backslash N\}.
$$
\end{problem}

\begin{theorem}
Если $A$ - нормальный линейный непрерывный оператор  в
гильбертовом пространстве $H,$ то существуют пространство $(\Om,
\As, \nu)$ со счетно-аддитивной $\sigma$-конечной мерой
$\nu$ и такой изоморфизм из $H$ в $L_2(\Om, \As, \nu),$
что при нем оператор $A$ переходит в
$$
(A f)(x) \ = \ g_A (x) \cdot f(x),
$$
то есть в оператор умножения на какую-то комплекснозначную функцию
$g_A (x).$
\end{theorem}

\subsection{Классификация точек спектра.}

Пусть $\la\in \spec A$:\\

\textbf{1.} $\quad$ Если $\Ker A_{\la} \not = \{0\},$ то говорят,
что элемент $\la$ принадлежит точечному спектру; таким образом,
$\la$ принадлежит точечному спектру тогда и только тогда, когда
$$\exists \, x \in H, \quad x \not = 0: \quad
A_{\la} x \ = \ 0 \quad (x\in \Ker A_{\la}), \ \mbox{ так что }
$$
все векторы из $A_{\la}$ являются собственными векторами.\\

\textbf{2.} $\quad$ Если $\Ker A_{\la} \ = \ \{0\}$ и
$\overline{\Im A_{\la}}\ = \ H,$ то говорят, что $\la$
принадлежит непрерывному спектру (черта обозначает замыкание).\\

\textbf{3.} $\quad$ Если $\Ker A_{\la} \ = \ \{0\}$ и
$\overline{\Im A_{\la}}\ \not = \ H,$ то $\la$ принадлежит
остаточному спектру.\\

\begin{prop}
Нормальный оператор не имеет остаточного спектра.
\end{prop}

\begin{proof}
Будем доказывать от противного, то есть предположим, что $\la \in$
остаточному спектру $A$. Тогда по определению:
$$
\overline{\Im (A - \la I)} \not = H;
$$
$$
\overline{\Im (A - \la I)}^{\bot} \ = \ (\Im (A - \la I))^{\bot}\
= \ \Ker A^{*}_{\la}\ = \ \Ker (A^{*}\ - \ \overline{\la} I), \
\mbox{ так как }
$$
$$
\qquad (\la I)^{*} \ = \ \overline{\la} I \ \  \mbox{(здесь черта
- это знак комплексного сопряжения)};
$$
следовательно, $\Ker A^{*}_{\la} \not = \{0\}.$ Это значит, что
$$
\exists \, x\in H, \quad x\not = 0\colon \quad (A^{*}\ - \
\overline{\la} I)x \ = \ 0.
$$
Исходя из этого, получаем, что:
$$
0 \ = \ (A - \la I) (A^{*}\ - \ \overline{\la} I)x \ = \ (A^{*}\ -
\ \overline{\la} I)(A - \la I)x
$$

$$
\left.
\begin{aligned}
\Im (A - \la I)\  \ni \ (A - \la I)x &\in \Ker (A^{*}\ - \
\overline{\la} I)\\
\Im (A - \la I)  \subset \overline{\Im  (A - \la I)} &= (\Ker
(A^{*}\ - \ \overline{\la} I))^{\bot}\\
\end{aligned}
\right\} \quad \Longrightarrow
$$

$$\Rightarrow \quad (A - \la I)x = 0 \ \Rightarrow \quad x \in \Ker
A_{\la} \ \Rightarrow\quad \Ker A_{\la} \not = \{0\}.$$

Следовательно, мы получили противоречие.
\end{proof}

\begin{theorem}
Пусть $A$ - нормальный линейный непрерывный оператор. Тогда:
$$
\bigl[\la \in \spec A\bigr] \ \Longleftrightarrow \ \bigl[ \exists
\, \mbox{ последовательность } \{x_n\} \subset H\colon \quad x_n
\nrightarrow 0, \quad A_{\la} x_n \rightarrow 0\bigr].
$$
\end{theorem}

\begin{proof}\\
$\Leftarrow\quad$ Пусть существует последовательность $\{x_n\}
\subset H\colon \quad x_n \nrightarrow 0, \quad A_{\la} x_n
\rightarrow 0.$ Тогда возможны два варианта:

\textbf{1.} $\quad \Ker A_{\la} \not = \{0\} \ (\mbox{значит, все
доказано});$

\textbf{2.} $\quad \Ker A_{\la} \  = \ \{0\}.$

Предположим, что $z_n = A_{\la} x_n \quad \forall \, n,$ тогда
$z_n \rightarrow 0,$ но
$$
A_{\la}^{-1} z_n\ = \ A_{\la}^{-1} A_{\la} x_n \ = \ x_n
\nrightarrow 0
$$
Следовательно, $A_{\la}^{-1}$ не непрерывен, а это значит, что
$\la \ \in \spec A$ (в этой части доказательства нормальность не
используется).

$\Rightarrow \quad$ Пусть $\la \in \spec A.$ Тогда мы снова
сталкиваемся с двумя случаями:

\textbf{(1.)} $\quad \Ker A_{\la} \not = \{0\} \ \Longrightarrow\
\exists\, x \in \Ker A_{\la},\  x \not = 0.$

Пусть $x_n = x \quad \forall \, n:$
$$
\forall \, n \qquad A_n x_n = A_{\la} x = 0 \ \Longrightarrow \
\mbox{ нужная последовательность существует.}
$$

\textbf{(2.)} $\quad \Ker A_{\la} = \{0\}.$ Тогда $\la \in$
непрерывному спектру, так как $A$ нормален. Значит,
$$
A_{\la}^{-1}\colon \Im A_{\la} \rightarrow H, \ \mbox{ причем }
\overline{\Im A_{\la}}\ = \ H.
$$

$$
\left.
\begin{aligned}
\forall x \in \Im A_{\la}\quad &A_{\la} A_{\la}^{-1} x \ = \ x\\
\forall z \in H \quad &A_{\la}^{-1} A_{\la} z \ = \ z
\end{aligned}
\right\} \ \Longrightarrow \ A_{\la}^{-1} \mbox{ не непрерывен на
непрерывном спектре,}
$$
так как, если бы он был непрерывным, то его можно было бы
продолжить по непрерывности до оператора $B,$ определенного на
замыкании $\Im A_\la,$ то есть на всем $H;$ этот оператор $B$
обладал бы свойствами:
$$
\forall \, x\in H \quad A_\la B x \ = \ x, \quad B A_\la x = x,
$$
что противоречит тому, что $\la \in \spec A.$

Таким образом,
$$
\exists \, z_n \rightarrow 0 \qquad \forall\, n \ \  z_n\in \Im
A_{\la};
$$
$$
x_n\ = \ A_{\la}^{-1} z_n \nrightarrow 0;
$$
$$
A_{\la} x_n = z_n \rightarrow 0.
$$
\end{proof}

\begin{imp}
Пусть $A$ - самосопряженный оператор, тогда $\spec A\subset \mro.$
\end{imp}

\begin{proof}
Пусть $\quad\la = \alpha + i \beta;\ \mbox{ где }\beta \not = 0.$
Докажем, что тогда $\la\notin \spec A.$
$$
A_{\la}\ = \ A  - \la I \quad \mbox{ и } \quad A_{\la}\  =\
A_{\alpha} - i\beta I
$$
На основании этого, получаем:
$$
(A_{\la} x, \, A_{\la} x)\ = \ ((\ A  - \la I)x, \, (\ A  - \la
I)x)\ =
$$
$$
= \ ((A_{\alpha} - i\beta I)x, \, (A_{\alpha} - i\beta I)x)\ = \
\|A_{\alpha} x\|^2 + \beta^2 \|x\|^2 \geqslant \beta^2 \|x\|^2
$$

$\la$ принадлежит такому спектру в точности тогда, когда:
$$
x_n \nrightarrow 0 \ \mbox{ на } \ A_{\la} x_n \rightarrow 0.
$$
Поэтому
$$
\|A_{\la} x_n\|^2 \geqslant \beta^2 \|x\|^2\ \Longrightarrow\
\beta = 0
$$
\end{proof}

\begin{problem}
Показать, что спектр унитарного оператора находится на единичной
окружности комплексной плоскости.
\end{problem}

\section{Лекция}

\begin{theorem}
Если $A = A^{*}$, тогда
$$
\|A\|\ = \ \sxo |(Ax, x)| = \sxpo |(Ax, x)|.
$$
\end{theorem}

\begin{proof}
\textbf{1.} $\quad \forall\, x \qquad |(Ax, x)| \leqslant \|Ax\|
\cdot \|x\| \leqslant \|A\|\cdot  \|x\|^2$

$$\|x\| = 1 \ \Longrightarrow \ |(Ax, x)| \leqslant \|A\|\
\Longrightarrow\ C = \sxpo |(Ax, x)| \leqslant \|A\|.
$$

\textbf{2.} $\quad$ Докажем, что $\|A\| \leqslant \sxpo |(Ax,
x)|.$

Введем следующее обозначение: $Q_{A} (x) = (Ax, x),$ тогда
$$
(1)\qquad\qquad\quad\  Q_{A} (x+z) - Q_{A} (x-z)\ = \ (Ax, z) +
(Az, x) + (Ax, z) + (Az, x)\ =
$$
$$
= \ 2(Ax, z) + 2(Az, x)\ = \ 2 \Bigl((Ax, z) + (Az, x)\Bigr)\ = \
$$
$$
= \ 2 \Bigl((Ax, z) + \overline{(Ax, z)}\Bigr) = 4 \Re (Ax, z).
$$
$$
(2)\qquad |Q_{A} (x+z) - Q_{A} (x-z)| \ = \ |4 \Re (Ax, z)| \
\leqslant \ |Q_{A} (x+z)| + |Q_{A} (x-z)|\ \leqslant
$$
$$
\leqslant\ C(\|x+z\|^2 + \|x-z\|^2)\ = \ C(2 \|x\|^2 + 2 \|z\|^2).
$$

Предпоследнее неравенство вытекает из того, что
$$
|Q_A (x)| \ = \ |(Ax, x)| \  \leqslant \  C \|x\|^2;
$$
в свою очередь, неравенство $|(Ax, x)| \leqslant C \|x\|^2$
доказывается так.

Если $\|x\| = 1,$ то $|(Ax, x)|\leqslant C;$ поэтому, если
$$
x = \frac{x_1}{\|x_1\|}, \ \mbox{ то } \ \frac{|(Ax_1,
x_1)|}{\|x_1\|^2} \leqslant C \quad \Rightarrow \ |(Ax_1, x_1)|
\leqslant C \|x\|^2
$$
Из (1) и (2) следует, что
$$
|4 \Re (Ax, z)| \ \leqslant \ 2C(\|x\|^2 + \|z\|^2).
$$
$$
\mbox{Пусть }\ z = \frac{\|x\| \cdot Ax}{\|Ax\|}; \quad \mbox{
тогда }\ \  4 \|x\| \cdot \|Ax\| \leqslant 4\cdot C \|x\|^2\quad
\Longrightarrow
$$
$$
\Longrightarrow\quad \|Ax\| \leqslant C \|x\| \quad \forall\,
x\quad \Longrightarrow \quad \|A\|\leqslant C.
$$
\end{proof}

\begin{df}
Оператор $A\in L(H)$ - компактен, если он переводит всякое
ограниченное множество в относительно компактное.
\end{df}

\subsection{Свойства компактного оператора.}

\textbf{1.} $\quad$ Множество всех компактных операторов в
гильбертовом пространстве образует идеал в алгебре всех
операторов, то есть:

$\quad$ \textbf{a.} Это множество - линейное подпространство;

$\quad$ \textbf{б.} $\forall \, B \in L(H), \  \forall K\in L_0
(K)\ \Longrightarrow\ K B \in L_0 (H), B K\in L_0 (K).$\\
\\
\textbf{2.} $\quad$ Этот идеал является замкнутым.

\begin{ex}
Пусть задано $L_2 (\Om, \Bs, \nu),$ где $\Om =
\mathbb{N},\ $ а $\nu$ приписывает множеству значение, равное
числу элементов в нем.

Тогда $(K, f) (n)\ = \ \alpha(n) \cdot f(n).$
\end{ex}

\begin{problem}
Этот оператор $K$ является компактным в точности тогда, когда \\
$\alpha(n) \rightarrow 0$  и самосопряженным, если $\forall \, n
\quad \alpha(n) \in \R.$
\end{problem}

\begin{theorem}[Гильберта-Шмидта]

Пусть $K$ - компактный самосопряженный оператор в сепарабельном
гильбертовом пространстве $H.$ Тогда в $H$ существует
ортонормированный базис $\{e_{n}\}$  с таким свойством:

$\quad$ все $e_{n}$ - собственные векторы этого оператора, и
оператор $K$ действует следующим образом:
$$
\forall\, x\in H \qquad Kx = \ \sus \la_n (x, e_n)\, e_n,
$$
где $\la_n$ - собственное значение для собственного вектора
$e_{n}$; при этом $\la_n \rightarrow 0.$
\end{theorem}

\begin{proof}
Покажем сначала, что то, что $\la_n \rightarrow 0,$ вытекает из
всего остального.

Пусть $\la_n \nrightarrow 0 \ \Longrightarrow$
$$
\exists\, \la_{n_1}, \la_{n_2}, \ldots : \quad |\la_{n_j}| >
\varepsilon > 0;
$$
тогда $\forall\, j \quad K e_{n_j} = \la_{n_j} e_{n_j}$
$$
\|\la_{n_j} e_{n_j} - \la_{n_k} e_{n_k}\| \geqslant \varepsilon
\|e_n\| = \varepsilon.
$$
Следовательно, мы можем сделать вывод, что из $\{e_{n_j}\}$ нельзя
выбрать фундаментальной последовательности, а это значит, что мы
получили противоречие с компактностью оператора.


Докажем остальные утверждения. \\

\textbf{1.} $\quad$ \texttt{Применение} \textbf{леммы
Куратовского-Цорна.}

Пусть $S$ - множество всевозможных ортонормированных семейств
собственных векторов $K$ (каждое такое семейство не более чем
счетно, так как $H$ сепарабельно). Это множество является частично
упорядоченным по включению.

При этом каждое его линейно упорядоченное подмножество имеет
мажоранту. Действительно, пусть $A \subset S$ и $A$ линейно
упорядочено, то есть
$$
\mbox{если  }s_1,\, s_2 \in A, \ \mbox{ то } s_1 \subset s_2 \
\mbox{ или }\ s_2 \subset s_1 \mbox{ (включения не обязательно
строгие).}
$$

Тогда множество $\bigcup\limits_{s\in A} s \ \in S$ - это
мажоранта множества $A.$ Поэтому в силу теоремы Куратовского-Цорна
в $S$ существуют максимальные элементы. Пусть $s_{\max} \ \in S$ -
один из них.\\

\textbf{2.} $\quad$ Мы утверждаем, что $s_{max}$ - тот
ортонормированный базис, существование которого утверждается в
теореме. Пусть это не так, то есть
$$
s_m = \{e_n \in H\} \ \mbox{ не образует базиса;}
$$
тогда, если $\Span s_{m}$ - наименьшее замкнутое линейное
пространство, содержащее $s_m$, то $\Span s_{m} \not = H.$
$$
\Longrightarrow \quad (\Span s_m)^{\bot} \ = \ (\overline{\Span
s_{m}})^{\bot}\ = \ H_1 \not = \{0\}.
$$

Покажем, что $H_1$ - инвариантное подпространство нашего
оператора, то есть
$$
\forall\, x\in H_1, \ Kx\in H_1.
$$

Это - частный случай следующего общего факта:
$$
\mbox{ если }\ H = H_0 \oplus H_1 \quad \mbox{ и }
$$
$H_0$ инвариантно относительно действия самосопряженного оператора
$A\in L (H)$, то $H_1$ инвариантно. Этот факт доказывается так:
$$
[x \in H_0 \ \Leftrightarrow\ \forall \, z\in H_1, \quad (z, x) =
0] \quad \Longrightarrow \quad \forall \, z \in H_1 \quad (Ax, z)
= (x, Az) = 0,
$$
так как $Az \in H_1 \ \Rightarrow \ Ax\in H_0.$

Остается положить $H_0 = \overline{\Span s_m}.$\\

\textbf{3.} $\quad$ Осталось доказать, что каждый компактный
оператор имеет хотя бы один собственный вектор. Это вытекает из
следующих двух утверждений:

$\quad$ \textbf{a.} Если $\sup|(Ax, x)|$ достигается на векторе,
то этот вектор является собственным для оператора. \label{predloj}

$\quad$ \textbf{b.} если оператор компактен и самосопряжен, то
этот $\sup|(Ax, x)|$ достигается обязательно.

Предположим, что это уже доказано. Так как сужение $A$ на $H_1$ -
это компактный самосопряженный оператор в $H_1,$ то, если $H_1
\not = \{0\},$ то $A$ обладает (ненулевым) собственным вектором в
$H_1,$  а противоречие с максимальностью $s_m.$

Таким образом, $s_m = \{e_n\}$ - это ортонормированный базис в $H$
и
$$
\forall \, n \quad \exists \, \la_n \in \mro, \quad K e_n = \la_n
e_n.
$$
Поэтому $\forall\, x \in H$
$$
x \ = \ \sus (x, \ e_n)\  e_n \quad \mbox{ и}
$$
$$
Kx = \sus \la_n \ (x, \ e_n)\ e_n.
$$
\end{proof}

\section{Лекция}

В доказательстве того, что, если $A$ компактен и самосопряжен, то
$\sxo |(Ax, x)|$ достигается на некотором векторе, используется
следующая теорема, имеющая и самостоятельное значение.

\begin{theorem}
Всякий замкнутый шар $S$ в сепарабельном гильбертовом пространстве
секвенциально компактен в слабой топологии, то есть
$$
\forall \, (x_n)\subset S\subset H \ \exists\, \mbox{
подпоследовательность } (x_{n_k}) \ \mbox{ и } \ \exists\, z\in
S\colon \ x_{n_k} \stackrel{\mbox{слабо}}\longrightarrow z \mbox{
при } k\rightarrow\infty.
$$
\end{theorem}
\begin{proof}
Достаточно доказать теорему для замкнутого единичного шара с
центром в нуле. Итак, пусть $S$ - такой шар. Пусть $(a_n)$ -
счетное всюду плотное подмножество гильбертова пространства
($(a_n) \subset H, \ \overline{(a_n)} = H$).

Так как $\ \forall \, j$
$$
|(a_j, x_n)| \leqslant \|a_j\| \, \|x_n\| \leqslant \|a_j\|,
$$
то каждая из последовательностей $\ \{(a_1, x_n)\}\ $ скалярных
произведений ограничено. Следовательно, из последовательности
скалярных произведений $\ \{(a_j, x_n)\}$ можно извлечь сходящуюся
подпоследовательность, то есть\\

$\quad\exists \,$ возрастающая последовательность  $\, \{n_1
(k)\}$ натуральных чисел, для которой последовательность скалярных
произведений $\{(a_1, x_{n_1 (k)})\}$ сходится.\\

Применяя к последовательности $\{(a_2, x_{n_1 (k)})\}$ аналогичное
рассуждение, получим, что существует (возрастающая)
последовательность $\{n_2 (k)\}$ натуральных чисел, являющаяся
подпоследовательностью последовательности $\{n_1 (k)\},$ такая что
последовательность скалярных произведений $\{(a_2, x_{n_2 (k)})\}$
сходится.


Продолжая это рассуждение, получим, что существует
последовательность \linebreak
$\Bigl\{ \{ n_j (k)\colon \ k\in
\mathbb{N}\}\colon \ j= 1, 2, \ldots \Bigr\}$ (возрастающих)
последовательностей натуральных чисел, каждая из которых является
подпоследовательностью предыдущей, причем все числовые
последовательности $\{ (a_j, x_{n_j (k)})\colon \ k = 1, 2,
\ldots\}$ сходятся. Тогда диагональная последовательность индексов
$\{n_k (k)\colon \ k=1,2,\ldots\}$ обладает следующим свойством:\\

$\forall \, j \ $ последовательность $\{ (a_j, x_{n_j (k)})\colon
\ k \in \mathbb{N}\}$ сходится (так как при $k > j$ она является
подпоследовательностью сходящейся последовательности $\{(a_j,
x_{n_j (k)})\}$).\\

Положим, для каждого $k, \quad m(k) = n_k (k);\ $ тогда, в силу
предыдущего, $\forall \, i\ \quad  (a_i, x_{m(j)})$ - сходится.

Докажем, что $ \forall a\in H$ последовательность $\{ (a,
x_{m(j)}) \}$ также сходится. Для этого достаточно, чтобы
последовательность $(a, x_{m(j)})$ была фундаментальной, то есть,
чтобы
$$
\forall \, \varepsilon \quad  \exists \, j_0\colon\ \forall\,
j>j_0, \ \forall \, k\in \mathbb{N}\qquad \left| (a, x_{m(j)})\ -
\ (a, x_{m(j+k)})\right| \, < \, \varepsilon
$$

Проверим, что такое $j_0$ существует. Так как множество $\{a_j\}$
всюду плотно в $H,\ $ то
$$
\exists \, a_r\colon \ \|a_r - a\| < \frac{\varepsilon}{3}.
$$
Последовательность $\{(a_r, x_{m(j)})\colon \ j\in \mathbb{N}\}$
сходится, следовательно, она фундаментальна, то есть
$$
\exists \, j_0\colon \ \forall \, j>j_0, \ \forall k\in \mathbb{N}
\qquad \left| (a_r, x_{m(j)})\ - \ (a_r, x_{m(j+k)})\right|\ < \
\frac{\varepsilon}{3}.
$$
Поэтому для таких $j$ и $k$
$$
\left| (a, x_{m(j)})\ - \ (a, x_{m(j+k)})\right| \ \leqslant \ |
(a, x_{m(j)})\ - \ (a_r, x_{m(j)}) \ +
$$
$$
+ \ (a_r, x_{m(j)})\ - \ (a, x_{m(j+k)}) \ + \ (a_r, x_{m(j+k)})\
- \ (a_r, x_{m(j+k)})| \ \leqslant
$$
$$
\leqslant\  \left| (a, x_{m(j)})\ - \ (a_r, x_{m(j)})\right| \ + \
\left| (a_r, x_{m(j + k)})\ - \ (a, x_{m(j+k)})\right| \ +
$$
$$
+ \ \left| (a_r, x_{m(j+k)})\ - \ (a_r, x_{m(j)})\right| \
\leqslant \ \varepsilon,
$$
так как
$$
\left| (a_r, x_{m(j)}) - (a_r, x_{m(j+k)}) \right| <
\frac{\varepsilon}{3} \quad \mbox{в силу выбора $j$ и $k,$}
$$
$$
\left| (a, x_{m(j)})\ - \ (a_r, x_{m(j)})\right| \ \leqslant \
\|a_r - a\| \, \underbrace{\|x_{m(j)}\|}_{1} \ <
\frac{\varepsilon}{3} \quad \mbox{ и }
$$
$$
\mbox{ аналогично, } \ \left| (a, x_{m(j + k)})\ - \ (a_r,
x_{m(j+k)})\right| < \frac{\varepsilon}{3} \quad \mbox{в силу
выбора $a_r.$}
$$

Так как $\ \forall \, a\in H$ последовательность  $\{(a,
x_{m(j)})\}$ сходится, то по теореме Рисса
$$
\forall f\in H^{*} \quad \{(f, x_{m(j)})\} \mbox{ сходится, }
\Rightarrow \ \mbox{ так как гильбертово пространство рефлексивно}
$$
(по той же теореме Рисса), то
$$
\exists \, z\in H \quad \forall \, f\in H^{*} \qquad (f, x_{m(j)})
\rightarrow (f, z).
$$

Это и означает, что $x_{m(j)}
\stackrel{\mbox{слабо}}{\longrightarrow} z;\ $  при этом,
поскольку замкнутый шар замкнут и в слабой топологии, $\ z\in S;\
$ теорема доказана.

\end{proof}

\begin{prop}
Пусть $A$ - компактный оператор. Тогда он переводит слабо
сходящуюся последовательность в последовательность, сходящуюся по
норме.
\end{prop}
\begin{proof}
Проведем доказательство в два шага.\\
\\
\textbf{1.} $\quad$ Если $A$ - произвольный линейный непрерывный
оператор и $\ (x_n)$ - слабо сходящаяся последовательность, $\
(x_n) \stackrel{\mbox{слабо}}{\longrightarrow} x_0, \ $ то $\ Ax_n
\stackrel{\mbox{слабо}}{\longrightarrow} Ax_0.$  Надо проверить,
что
$$
\forall \, z\in H \quad (Ax_n, z) \rightarrow (Ax_0, z).
$$
Это делается так:
$$
(Ax_n, z) = (x_n, A^{*}z)\  \rightarrow\  (x_0, A^{*}z) = (Ax_0,
z)
$$
\textbf{2.} $\quad$ Проверим, что,
$$
\mbox{если }\  A  \mbox{ - компактен и $(x_n)$ - слабо сходящаяся
к $x_0 \in H$ последовательность,}
$$
то $Ax_n \rightarrow Ax_0 \mbox{ по норме}.$
\bigskip

Предположим, что это не так. Тогда $\ \exists \, \varepsilon > 0\
$ и бесконечная подпоследовательность $(x_{n_k}),$ такая что
$$
\forall \, k \qquad \|Ax_0 - Ax_{n_k}\| > \varepsilon \qquad\qquad
(*)
$$

Так как $A$ компактен, а слабо сходящаяся последовательность
$(x_n)$ ограничена по норме (в силу теоремы Банаха-Штейнхауза), то
из последовательности $(Ax_{n_k})$ можно выбрать
подпоследовательность $(Ax_{n_{k(j)}}),$ сходящуюся по норме к
некоторому элементу $z\in H; \ $
$$\mbox{ тем более, }\  Ax_{n_{k(j)}}
\stackrel{\mbox{слабо}}{\longrightarrow} z.\
$$
Но $Ax_{n_{k(j)}} \stackrel{\mbox{слабо}}{\longrightarrow} Ax_0,\
$ так как $Ax_n \stackrel{\mbox{слабо}}{\longrightarrow} Ax_0;\ $
поэтому $z = Ax_0,\ $ то есть $Ax_{n_{k(j)}} \longrightarrow Ax_0$
по норме, что противоречит $(*).$

\end{proof}
\begin{stm}
Если $A$ - самосопряженный компактный оператор, то функция $\ x
\mapsto |(Ax, x)|\ $ достигает верхней грани на замкнутом
единичном шаре.
\end{stm}
\begin{proof}
Выберем подпоследовательность $(x_n),$ такую что $\ \forall \, n
\quad \|x\| = 1\ $ и
$$
\Bigl| (Ax_n, x_n)\Bigr| \longrightarrow \|A\| = \sup \Bigl| (Ax,
x)\Bigr|, \  \mbox{ причем } \|x_n\| = 1 \ \ \forall n
$$

Выберем слабо сходящуюся подпоследовательность последовательности
$(x_n)$ и обозначим ее элементы также $x_n.$ Так как
$$
x_n \stackrel{\mbox{слабо}}{\longrightarrow} x_0, \ \mbox{ то, по
предыдущему, }\  Ax_n \stackrel{\mbox{сильно}}{\longrightarrow}
Ax_0.
$$
Мы хотим доказать, что $(Ax_n, x_n) \longrightarrow (Ax_0, x_0) =
\|A\|.$ Итак,
$$
\Bigl| (Ax_n, x_n) - (Ax_0, x_0)\Bigr| \ \leqslant \ \Bigl|
\underbrace{(Ax_n, x_n) - (Ax_0, x_n)}_{\substack{\downarrow\\
0\\ (\mbox{ так как } \| Ax_n - Ax_0\| \rightarrow 0)}} \Bigr| +
\Bigl|
\underbrace{(Ax_n, x_0) - (Ax_0, x_0)}_{\substack{\downarrow\\
0\\ (\mbox{ так как } Ax_n \stackrel{\mbox{слабо}}{\rightarrow}
Ax_0)}}\Bigr| \ \longrightarrow 0
$$
\end{proof}
Таким образом, доказано, что
$$
\exists \, x_0\colon \ \ |(Ax_0, x_0)| = \sxo |(Ax, x)| = \|A\|,
$$
а, следовательно, в силу предложения \textbf{a} (на странице
\pageref{predloj}, конец предыдущей лекции)\linebreak $x_0$ -
(нормированный) собственный вектор оператора $A,$ то есть
$$
\exists \, \la\in \R^1\colon \ \  Ax_0 = \la x_0.
$$
При этом
$$
\bigl| (Ax_0, x_0)\bigr| \ = \ \bigl| (\la x_0, x_0)\bigr| = |\la|
= \|A\|,
$$
то есть $\la = \pm \|A\|.$

Итак, для полного доказательства теоремы Гильберта-Шмидта, нам
осталось доказать предложение \textbf{a} из предыдущей лекции.

\begin{df}
Отображение $f\colon \ B_1 \rightarrow B_2,$ где $B_1$ и $B_2$ -
банаховы пространства, называется дифференцируемым в точке $x,$
если $\exists \, f^{\prime} (x) \in L(B_1, B_2):$
$$
\forall h \quad f(x+h) - f(x)\ = \ f^{\prime} (x) h + r_x (h),\
\mbox{ где }\  \frac{r_x (th)}{t} \rightarrow 0 \qquad (*)
$$
Причем:

Если $(*) \rightarrow 0 \ \ \forall \, h,$ то это
дифференцируемость по Гато.

Если $(*) \rightarrow 0$ равномерно по $h$ из каждого
ограниченного множества, то это дифференцируемость по Фреше.

Если $(*) \rightarrow 0$ равномерно по $h$ из каждого компактного
множества, то это дифференцируемость по Адамару.\\

Отметим, что дифференцируемость по Фреше влечет дифференцируемость
по Адамару, а дифференцируемость по Адамару влечет
дифференцируемость по Гато.
\end{df}

\begin{stm}
Пусть имеется цепочка отображений:
$$
B_1 \stackrel{f}{\longrightarrow} B_2
\stackrel{g}{\longrightarrow} B_3,
$$
дифференцируемых всюду по Фреше (по Адамару); тогда их композиция
$\ g \circ f\ $ дифференцируема по Фреше (по Адамару), и
$$
(g \circ f)^{\prime} (x) \ = \ g^{\prime} (f(x)) \circ f^{\prime}
(x)
$$
(для дифференцируемости по Гато это не так).
\end{stm}
\begin{ex}
Пусть $H \rightarrow \mro, \ \  f\colon \ x \mapsto (Ax, x),$
тогда $f$ - дифференцируема по Фреше.
\end{ex}
\begin{problem}
Доказать, что при $x \in H, \ h\in H$ справедливо равенство:
$$
f^{\prime} (x) h\ = \ 2 \Re (Ah, x)
$$
\end{problem}

Докажем теперь, что, если существует вектор $x_0,\ $ для которого
$$
(Ax_0, x_0) = \sup (Ax, x), \ \mbox{ то } \ \exists \, \la: Ax_0 =
\la x_0.
$$
(напомним, что здесь предполагается лишь, что $A$ -
самосопряженный оператор, но не предполагается, что он компактен).

Возьмем элемент $z_0\colon \ z_0 \bot Ax_0\ $ и рассмотрим кривую
в гильбертовом пространстве:
$$
\psi (t)\ = \ x_0 \cos t + z_0 \sin t
$$
$$
f\circ \psi \colon \ t \mapsto (A \psi (t), \psi (t))
$$
$$
\mro \stackrel{\psi}{\longrightarrow} H \stackrel{x \mapsto (Ax,
x)}{\longrightarrow} \mro
$$
Причем эта композиция достигает своего максимума при $t = 0,$
значит, ее производная в нуле равна 0: \ $(f \circ \psi)^\prime
(0) = 0.$

Но
$$
(f \circ \psi)^{\prime} (0)\ = \ f^{\prime} \underbrace{(\psi
(0))}_{\substack{\|\\ x_0}} \circ \ \underbrace{\psi^{\prime}
(0)}_{\substack{\|\\ z_0}} \ = \
$$
$$
= \ f^{\prime} (x_0)\  z_0 = 2 \Re (Az_0, x_0)\ = \ 2 \Re (Ax_0,
z_0) = 0
$$

Так как, если $\ z_0 \bot Ax_0,\ $ то и $i z_0 \bot Ax_0, \ $ и $\
2 \Re (Ax_0, i z_0) = 0;\ $
$$\mbox{ но } \quad \Re(Ax_0, i z_0) = \Re (-i \,
(Ax_0, z_0))\ = \ \Im (Ax_0, z_0), \ $$
$$
\mbox{ так что } \Im (Ax_0, z_0) = 0;\ \mbox{ поэтому } (Ax_0,
z_0) = 0.
$$

Отсюда следует, что элемент
$$
Ax_0\in (x_0^{\bot})^{\bot} \ = \ \{\la x_0 \colon \ \la \in
\mathbb{C}\}^{\bot\bot}\ = \ \{\la x_0\colon \ \la\in
\mathbb{C}\}.
$$
Это значит, что $\ \ \exists \, \la_0\colon \ Ax_0 = \la_0 x_0$

\subsection{Теоремы Фредгольма.}

Пусть $K$ - компактный оператор в гильбертовом пространстве $H, \
\la \not = 0,\ \la \in \mathbb{C},$
$$
K_\la = K - \la I.
$$
$I$ - тождественное  отображение $H$ в $H.$

В теоремах Фредгольма рассматриваются уравнения
$$
\begin{aligned}
K_\la \, \ff &= f,\\
K_\la \, \ff &= 0
\end{aligned}
$$
и сопряженные уравнения
$$
\begin{aligned}
K_\la^{*} \psi &= g,\\
K_\la^{*} \psi &= 0.
\end{aligned}
$$

При этом уравнения $\ K_\la \, \ff = 0,\ K_\la^{*} \psi = 0\ $
называются однородными, а уравнения $\ K_\la \, \ff = f, \
K_\la^{*} \psi = g\ $ - неоднородными ($K_\la^{*} = K^{*} -
\overline{\la} I$).

Далее мы будем использовать следующую теорему Шаудера.

\begin{theorem}[Теорема Шаудера]
Операторы $K$ и $K^{*}$ компактны или нет одновременно.
\end{theorem}

\emph{Доказательство ее можно прочитать в $\S$ 6 главы 4 книги
Колмогорова-Фомина.}

\begin{theorem}[1]
$$
\Im K_\la = (\Ker K^{*}_\la )^{\bot},
$$
то есть неоднородное уравнение $K_\la \, \ff = f$ разрешимо тогда
и только тогда, когда $$f \in (\Ker K^{*}_\la )^{\bot}.$$
\end{theorem}

\begin{proof}
Ранее для произвольного линейного непрерывного оператора $A$ в $H$
было доказано равенство
$$
\overline{\Im A}\ = \ (\Ker A^{*})^{\bot};
$$
поэтому чтобы доказать, что $\Im K_\la = (\Ker K^{*}_\la)^{\bot},$
достаточно проверить, что подпространство $\Im K_\la$ замкнуто, то
есть надо доказать, что,
$$
\mbox{ если } z_n  \in \Im K_\la\ \mbox{ и } \ z_n \rightarrow z
\mbox{ в } H,\ \mbox{ то } z\in \Im K_\la.
$$

Так как $\ \forall \, n \quad z_n \in \Im K_\la,\ $ то $\ z_n =
K_\la  x_n\ $ для некоторого $x_n \in H.$

Если $\ \sup\limits_{n} \|x_n\| < \infty,\ $ то из компактности
$K$ следует, что для некоторой подпоследовательности $(x_{n_k})$
последовательности $(x_n)$ последовательность $(Kx_{n_k})$
сходится;\\
тогда сходится и сама последовательность $(x_{n_k}),\ $ так как
$$
x_{n_k}\ = \ \frac{1}{\la} \, (Kx_{n_k} - K_\la x_{n_k}) \ = \
\frac{1}{\la} \, (Kx_{n_k} - z_{n_k}),
$$
а последовательность $(z_{n_k})$ сходится как
подпоследовательность сходящейся последовательности $(z_n).$ Если
$\ x = \lim x_{n_k},\ $ то
$$
K_\la x\ =\ \lim K_\la x_{n_k}\ = \  \lim z_{n_k}\ = \ z,
$$
так что $z\in \Im K_\la,\ $ что и требовалось.\\

Покажем теперь, что всегда можно добиться, чтобы предположение
$\sup \|x\| < \infty$ было выполнено.

Итак, пусть $\ \forall n \quad z = K_\la x_n.\ $ Положим
$\overline{x_n} = x_n - \pr_{\Ker K_\la} x_n, \ $ где символ
$\pr_{\Ker K_\la} x_n$ обозначает проекцию вектора $x_n$ на
подпространство $\Ker K_\la$ (так что $x_n \in (\Ker
K_\la)^{\bot}$);\ тогда
$$
z_n = K_\la \, \overline{x_n}, \ \mbox{ причем } \sup\limits_{n}
\|\overline{x_n}\| < \infty.
$$
Действительно, если это не так, то существует
подпоследовательность $\overline{x}_{n_r},\ $ такая что
$\|\overline{x}_{n_r}\| \rightarrow \infty.\ $ Тогда
$$
\frac{z_n}{\|\overline{x}_{n_r}\|} \longrightarrow 0, \ \mbox{ так
как }\ z_n = K_\la \overline{x_n},
$$
это означает, что
$$
\frac{K_\la \overline{x}_{n_r}}{\|\overline{x}_{n_r}\|}\ = \ K \,
\frac{\overline{x}_{n_r}}{\|\overline{x}_{n_r}\|} -  \la\,
\frac{\overline{x}_{n_r}}{\|\overline{x}_{n_r}\|}\ \longrightarrow
\ 0.
$$

Положим, для каждого $r,$ $$y_r =
\frac{\overline{x}_{n_r}}{\|\overline{x}_{n_r}\|}.\ $$ Тогда $K
y_r - \la y_r \ \rightarrow 0,\ $ причем $\ \forall \, r \quad
\|y_r\| = 1.$ Поэтому, в силу компактности $K,\ $ некоторая
подпоследовательность $(K y_{r(j)})$ последовательности $(Ky_r)$
сходится; \ из сходимости последовательности $(K y_r - \la y_r)\ $
следует, что сходится и последовательность $(y_{r(j)}).$\\

Пусть $a = \lim y_{r(j)}.$ Так как $\ \forall \, j \quad
\|y_{r(j)}\| = 1,\ $ то $\ \|a\| = 1;\ $ кроме того,
$$\mbox{ так как } \ K
y_r - \la y_r \ \rightarrow 0,\ \mbox{ то } \ Ka - \la a = K_\la =
0,\
$$
то есть $a\in \Ker K_\la.$

Но этого не может быть, так как $\ \forall r\quad y_r \in (\Ker
K_\la)^{\bot},\ $ а множество $(\Ker K_\la)^{\bot}$ замкнуто. Тем
самым доказано, что $\sup\limits_{n} \|\overline{x_n}\| < \infty.$

\end{proof}

\begin{theorem}[2]\textbf{``Альтернатива Фредгольма.``}

Всякий ненулевой элемент спектра компактного оператора является
его собственным значением

(название ``альтернатива Фредгольма`` объясняется тем, что теорему
(2) можно переформулировать так:

При $\ \la \not = 0\quad$ либо $\ \Ker K_\la \not = 0,\ $ либо $\
\la \not\in \spec K$).
\end{theorem}

\begin{proof}
Пусть, для каждого $n \in \mathbb{N}, \ \ H^{n} = \Im K_\la^n.\ $
Тогда, в силу теоремы (1), все подпространства $H^n$ замкнуты,
причем
$$
H^1 \supset H^2 \supset H^3 \ldots;
$$
конечно, $\ \forall \, n \quad K_\la (H^n) = H^{n+1}.$

При этом из компактности $K$ вытекает, что
$$
\exists \, j, \ \mbox{ такое что }\ H^j = H^{j+1}= H^{j+2} =
\ldots
$$
Действительно, пусть все $H^n$ различны и пусть, для каждого $n,\
$
$$
x_n \in H^n, \quad x_n \bot H^{n+1} \ \mbox{ и }\ \|x_n\| = 1.
$$

Но тогда:
$$
K x_n - K x_{n+r}\ = \ (K - \la I + \la I) x_n - (K - \la I + \la
I) x_{n+r}\ = \
$$
$$
(K_\la + \la I) x_n - (K_\la + \la I) x_{n+r}\ = \ K_\la x_n -
K_\la x_{n+r} + \la (x_n - x_{n+r});
$$
при этом
$$
K_\la x_n \in H^{n+1},\quad K_\la x_{n+r} \in H^{n+r+1} \subset
H^{n+1}, \quad \la x_{n+r}\in H^{n+1},
$$
но $\la x_n \in H^n, \ x_n \bot H^{n+1}.\ $ Поэтому
$$
\|K x_n - K x_{n+r}\| \geqslant \|\la x_n\| = \|\la\| \ \mbox{ для
всех } n\in \mathbb{N}, \ r\in \mathbb{N};
$$
но это означает, что из последовательности $(K x_n)$ нельзя
выбрать сходящейся подпоследовательности, вопреки компактности
оператора $K.$\\

Вернемся к доказательству теоремы (2).

Заметим, что
$$
\Ker K_\la = \{0\}\ \Rightarrow\ \Im K_\la = H.
$$
Пусть $\ \Ker K_\la = \{0\}\ $ и $\ H^1 = K_\la H \not = H;\ $
тогда, в силу взаимной однозначности $K_\la,$
$$
H^2 = K_\la H^1 = K_\la^2 H \ \not = \ K_\la H = H^1,
$$
$$
H^3 = K_\la^2 H^1 \ \not =\  K_\la^2 H = H^2
$$
и так далее, то есть цепочка состоит из попарно различных
подпространств, что противоречит доказанному выше. Поэтому $\Im
K_\la = H.$


Так как, по теореме Шаудера, оператор $K^{*}$ также компактен, то
$$
\Ker K_\la^{*}\ = \  \{0\}\ \Rightarrow\ \Im K_\la^{*} = H.
$$

Докажем теперь, что $\ \Im K_\la = H\ \Rightarrow\ \Ker K_\la =
\{0\}.$ Если $\Im K_\la = H, \ $ то $\ \Ker K_\la^{*}\ = \  \{0\}\
$ и, следовательно, $\Im K_\la^{*} = H,\ $ откуда следует, что
$\Ker K_\la = \{0\}.$\\

Таким образом, $\Im K_\la = H\ \Longleftrightarrow\ \Ker K_\la =
\{0\};\ $ но это и означает, что, если $\la \in \spec K,\ $ то
$\la$ - собственное значение ($\la \not = 0$).

\end{proof}

\begin{theorem}[3]
$$
\dim \Ker K_\la \ = \ \dim \Ker K_\la^{*} \ < \infty.
$$
\end{theorem}

\begin{proof}
Если $\dim \Ker K_\la  = \infty, \ $ то в $K_\la$ содержится
бесконечное множество попарно ортогональных нормированных векторов
$\{e_n\}.$ При этом $\ \forall n$
$$
(K - \la I) \, e_n = 0,
$$
так что при $n \not = m$
$$
\|K e_n - K e_m\| = |\la| \|e_n - e_m\| = |\la| \sqrt{2},
$$
так что из последовательности $\{K e_n\}$ нельзя выбрать
сходящейся подпоследовательности.\\

Таким образом, доказано, что $\dim \Ker K_\la < \infty;\ $ так
как, по теореме Шаудера, $\ K^{*} \in L_0 (H), \ $ то
$$
\dim \Ker K_\la^{*} \ ( = \dim \Ker (K^{*})_{\overline{\la}})\ <\
\infty.
$$

Осталось доказать, что $\ \dim \Ker K_\la \ = \ \dim \Ker
K_\la^{*}.$ Пусть
$$
\alpha\ = \ \dim \Ker K_\la \ <\ \dim \Ker K_\la^{*} = \beta
$$
и пусть $e_1, \ldots, e_\alpha$ - ортонормированный базис в $\Ker
K_\la\ $ и $\ e_1^{*}, \ldots, e^{*}_\beta$ - ортонормированный
базис в $\Ker K^{*}_\la.$ Положим
$$
\overline{K}x = Kx + \sum\limits_{j=1}^{\alpha} (x, e_j) e_j^{*},
$$
оператор $\overline{K}$ компактен как сумма двух компактных
операторов:\ \ оператора $K$ и оператора $x \mapsto
\sum\limits_{j=1}^{\alpha} (x, e_j) e_j^{*}\ $ (последний
компактен, так как непрерывен и обладает конечномерным образом).

Покажем, что $\ \Ker \overline{K}_\la \ = \ \{0\}.\ $ Пусть $x\in
\overline{K}_\la;\ $ это значит, что
$$
K_\la x + \sum\limits_{j=1}^{\alpha} (x, e_j) e_j^{*} = 0.
$$
Так как $\ e_j^{*}\in \Ker K_\la^{*},\ $ то $\ e_j^{*}\in (\Im
K_\la)^{\bot},\ $ так что
$$
K_\la x\ \bot\ \sum\limits_{j=1}^{\alpha} (x, e_j) e_j^{*}.
$$

Так как сумма этих двух ортогональных векторов равна нулю, то оба
они также равны нулю:
$$
K_\la x = 0, \quad \sum\limits_{j=1}^{\alpha} (x, e_j) e_j^{*} =
0.
$$

Следовательно, $\ \forall\, j = 1,2, \ldots, \alpha, \quad (x,
e_j) \equiv 0,\ $ так как векторы $e_1, \ldots, e_\alpha$ образуют
базис в $\Ker K_\la,\ $ то это значит, что
$$
x \in (\Ker K_\la)^{\bot};
$$
но так как $K_\la x = 0, \ $ то $ x \in \Ker K_\la;\ $ это
возможно только, если $x = 0.$

Таким образом, мы доказали, что $\Ker \overline{K}_\la = \{0\}.$
Тогда по теореме (2) $\ \exists \, z:$
$$
\overline{K}_\la z\ = \ e_{\alpha + 1}, \ \mbox{ то есть }
$$
$$
(*)\qquad K_\la z + \sum\limits_{j=1}^{\alpha} (z, e_j) e_j^{*}\ =
\ e^{*}_{\alpha + 1}.\quad
$$

Но $\ \forall\, j = 1,2, \ldots, \alpha, \quad e_j^{*} \,\bot\,
e_{\alpha+1}^{*}\ $ и $\ K_\la z\, \bot \, e_{\alpha+1}^{*}, \ $
так как
$$
(\Im K_\la)\ = \ (\Ker K^{*}_\la)^{\bot}, \ \mbox{ а } \
e_j^{*}\in \Ker K_\la^{*}.
$$

Таким образом, векторы в левой и правой частях равенства $(*)$
ортогональны друг другу, что возможно лишь, если они равны нулю,
тогда как $\|e^{*}_{\alpha +1}\| = 1.$\\

Таким образом, предположение, что $\beta > \alpha$ привело к
противоречию, так что $\alpha \geqslant \beta.$ Аналогично
доказывается и что $\alpha \leqslant \beta.$
\end{proof}

\begin{imp}
Всякий ненулевой элемент спектра компактного оператора является
его собственным значением конечной кратности.

\end{imp}

\end{document}
