\documentclass[a4paper]{article}
\usepackage[utf]{dmvn}

\def\fsp{\frac{1}{\sqrt{2\pi}}}

\def\z{,\;}

\newcommand{\bsub}[2]{\bigl|_{#1}^{#2}}
\newcommand{\Bsub}[2]{\Bigl|_{#1}^{#2}}
\newcommand{\Bbsub}[2]{\Biggl|_{#1}^{#2}}

\newcommand{\dx}{\,dx}
\newcommand{\dy}{\,dy}
\newcommand{\dt}{\,dt}
\newcommand{\dla}{\,d\la}
\newcommand{\dxi}{\,d\xi}

\begin{document}
\dmvntitle{Курс лекций по}{функциональному анализу}{Лектор -- Анатолий Михайлович Стёпин}
{III курс, 6 семестр, поток математиков}{Москва, 2010 г.} \pagebreak

\pagestyle{plain}
\tableofcontents
\pagebreak

\section*{Введение}

\subsection*{Предисловие}

Убедительная просьба ко всем читателям: в~случае обнаружения ошибок
немедленно сообщайте авторам на \dmvnmail{} или загляните на \dmvnwebsite{} и посмотрите, где можно
достать в настоящее время самих авторов. Все пожелания и предложения по поводу оформления
и содержания документа будут обязательно приняты к сведению.

\subsection*{Release Notes}

Под <<принципом равномерной ограниченности для рядов Фурье>> в программе экзамена
понимается существование непрерывных функций, для которых ряд Фурье расходится в~точке.
Это можно прочесть в~книге~\hbox{\cite[гл.VIII,~\S1, п.\,1]{kf}}.

Если Вы хотите узнать всё про свёртки, а также уметь отвечать на вопрос про оператор свёртки в~$L_2$,
читайте по этому поводу~\cite[гл.\,III,~\S9]{bogachev}.

В~последней версии добавлено доказательство теоремы Пэли -- Винера.

\subsection*{Слова благодарности}

Спасибо всем, кто замечал ошибки и~присылал свои комментарии, а~именно Алексею Басалаеву, Григорию Мерзону,
Нине Прудовой, Михаилу Берштейну, Дмитрию Рыжову, Николаю Рудому, Владиславу Короткову, Владимиру Филатову, Ивану
Вегнеру и Наталье Побыванец.


\subsection*{Принятые в тексте соглашения и используемые сокращения}

\begin{points}{-3}
\item Следуя \cite{rokhlin}, топологические понятия обозначаются сокращениями соответствующих английских слов.
      Так, $\Int A$ -- множество внутренних точек множества~$A$, $\Cl A$ -- замыкание множества $A$.
\item Под термином <<гладкий индикатор отрезка>> мы понимаем гладкую функцию, которая равна единице на этом отрезке,
и~нулю вне некоторой окрестности этого отрезка.
\end{points}

\begin{thebibliography}{ГШ}
\setlength\itemsep{-2pt}
\bibitem[КФ]{kf}
    А.\,Н.\,Колмогоров, С.\,В.\,Фомин. \emph{Элементы теории функций и функционального анализа.} -- М.: Наука, 1981.
\bibitem[РС]{rs}
    М.\,Рид, Б.\,Саймон. \emph{Методы современной математической физики.} -- М.: Мир, 1977.
\bibitem[Ш]{sh}
    Г.\,Е.\,Шилов. \emph{Математический анализ. Второй специальный курс.} -- М.: Физматгиз, 1965.
\bibitem[КГ]{kg}
    А.\,А.\,Кириллов, А.\,Д.\,Гвишиани. \emph{Теоремы и задачи функционального анализа.} -- М.: Наука, 1988.
\bibitem[ХР]{hr}
    Г.\,Г.\,Харди, В.\,В.\,Рогозинский. \emph{Ряды Фурье.} -- М.: Физматгиз, 1959.
\bibitem[Вл]{vl}
    В.\,С.\,Владимиров. \emph{Обобщённые функции в математической физике.} М.: Наука, 1976.
\bibitem[ГШ]{gs}
    И.\,М.\,Гельфанд, Г.\,Е.\,Шилов. \emph{Обобщённые функции.} -- М.: Физматгиз, 1959.
\bibitem[Р]{ru}
    У.\,Рудин. \emph{Основы функционального анализа.} -- М.: Мир, 19??.
\bibitem[Х]{h}
    А.\,Я.\,Хелемский. \emph{Лекции по функциональному анализу.} -- М.: МЦНМО, 2004.
\bibitem[РФ]{rokhlin}
    В.\,А.\,Рохлин, Д.\,Б.\,Фукс. \emph{Начальный курс топологии.} -- М.: Наука, 1977.
\bibitem[Б]{bogachev}
    В.\,И.\,Богачёв. \emph{Основы теории меры. Том 1.} -- Ижевск: РХД, 2003.
\end{thebibliography}

\medskip\dmvntrail

\pagebreak
\pagestyle{headings}

\makeatletter
  \renewcommand{\headheight}{11mm}
  \renewcommand{\headsep}{2mm}
  \renewcommand{\sectionmark}[1]{}
  \renewcommand{\subsectionmark}[1]{}
  \renewcommand{\subsubsectionmark}[1]{\markright{\thesubsubsection. #1}}
  \renewcommand{\@oddhead}{\vbox{\hbox to \textwidth{\scriptsize\thepage\hfil\rightmark\strut}\hrule}}
  \renewcommand{\@oddfoot}{\hfil\thepage\hfil}
\makeatother

\section{Ряды и преобразование Фурье}

\subsection{Ряды Фурье}

При изучении рядов Фурье мы будем предполагать, что функции у нас $2\pi$ -- периодические.
Поэтому их изучение сводится к рассмотрению интервала $(-\pi,\pi)$.

\begin{df}
Пусть $f \in L_1(-\pi,\pi)$. \emph{Рядом Фурье} функции $f$ называется ряд
\eqn{\sums{\Z} c_n e^{inx}, \quad c_n := \frac{1}{2\pi}\intl{-\pi}{\pi} f(x) e^{-i n x}\dx,}
числа $c_n$ называются \emph{коэффициентами Фурье} функции $f$.
\end{df}

Далее мы везде считаем, что функция интегрируема по Лебегу на интервале $(-\pi,\pi)$. В~противном
случае нет гарантии, что коэффициенты Фурье существуют.

\subsubsection{Сходимость ряда Фурье в точке}

\begin{df}
Говорят, что функция $f$ удовлетворяет \emph{условию Дини} в точке $x$, если
функция
\eqn{\ph(h) := \frac{f(x+h)-f(x)}{h}}
суммируема в некоторой $\de$ -- окрестности нуля.
\end{df}


\begin{lemma}[Римана -- Лебега]
Пусть функция $f$ суммируема на отрезке $[a,b]$. Тогда
\eqn{\intl{a}{b}f(t)e^{i s t}\dt \ra 0, \quad s \ra \bes.}
\end{lemma}
\begin{proof}
Пусть сначала $f=\Ibb_{[a,b]}$. Тогда
\eqn{\intl{a}{b}e^{i s t}\dt= \frac{1}{is}\ub{\hr{e^{isb}- e^{isa}}}_{\text{ограничена}} \ra 0, \quad s \ra \bes.}

Далее, по линейности утверждение леммы верно для конечных линейных комбинаций индикаторов.
В общем случае приблизим произвольную функцию $f$ ступенчатой функцией $f_\ep$ по норме $L_1$
с точностью $\ep$.
Тогда
\eqn{\bbm{\intl{a}{b}f(t)e^{ist}dt} \le \intl{a}{b}\hm{f-f_\ep}\cdot\hm{e^{ist}}\dt + \bbm{\intl{a}{b}f_\ep(t) e^{ist}\dt}.}
Первое слагаемое не превосходит $\ep$, а второе стремится к нулю по уже доказанному.
\end{proof}

\begin{note}
Далее  в разделе о рядах Фурье мы не будем писать пределы интегрирования,
если мы интегрируем \textbf{по периоду} $(-\pi,\pi)$.
\end{note}

\begin{theorem}[Признак Дини]
Если функция непрерывна в точке $x$ и удовлетворяет условию Дини в точке $x$, то её ряд Фурье
сходится в этой точке к $f(x)$.
\end{theorem}
\begin{proof}
Пусть $S_n(x)$ -- частичная сумма ряда Фурье. Имеем
\eqn{\label{eqn:partSum}S_n(x) = \suml{-n}{n} c_k e^{i k x} = \suml{-n}{n}\frac{1}{2\pi}\int f(\xi)e^{- i k\xi}\dxi \cdot e^{i k x} =
\frac{1}{2\pi}\int f(\xi)\suml{-n}{n}e^{i k (x-\xi)}\dxi.}

По формуле геометрической  прогрессии имеем
\eqn{e^{-i n t} \spl e^{i n t} = \frac{e^{i t(-n)} - e^{i t(n+1)}}{1 - e^{i t}} =
\frac{\exp \hs{-i \hr{n+\frac12}t } - \exp\hs{i \hr{n+\frac12}t }}{\exp\hr{-\frac{it}{2}} - \exp\hr{\frac{i t}{2}}}=
\frac{\sin\hr{n+\frac12}t}{\sin\frac{t}{2}} =: D_n(t).}
Заметим, что \emph{ядро Дирихле} $D_n(t)$ является чётной функцией.
Продолжая формулу~\eqref{eqn:partSum} с использованием этого равенства, получаем
\eqn{S_n(x) = \frac{1}{2\pi}\int f(\xi)D_n(x-\xi)\dxi.}
В силу периодичности функции можно сделать замену $t = \xi - x$, не изменяя пределов интегрирования:
\eqn{S_n(x) = \frac{1}{2\pi}\int f(x+t)D_n(t)\dt.}
Пользуясь исходным представлением для ядра Дирихле, получаем, что
\eqn{\frac{1}{2\pi}\int D_n(t)\dt = 1.}
Следовательно, имеем
\mln{f(x) - S_n(x) = f(x) \cdot \frac{1}{2\pi}\int D_n(t)\dt - \frac{1}{2\pi}\int f(x+t)D_n(t)\dt =\\=
\frac{1}{2\pi}\int \frac{f(x) - f(x+t)}{t} \cdot t \cdot D_n(t)\dt =
\frac{1}{2\pi}\ints{|t| < \de} + \frac{1}{2\pi}\ints{\de<|t|<\pi}.}
Второе слагаемое стремится к нулю по лемме Римана -- Лебега, так как знаменатель ядра Дирихле
отделён от нуля. Что касается первого слагаемого, то оно в силу абсолютной непрерывности интеграла
Лебега и суммируемости функции $\frac{f(x) - f(x+t)}{t}$ за счёт выбора $\de$ может быть
сделано сколь угодно малым (множитель $t$ в числителе глушит $\sin\frac t2$ в знаменателе).
\end{proof}

\subsubsection{Вычисление интеграла Дирихле}

Докажем, что
\eqn{\intl{0}{\bes}\frac{\sin x}{x} \dx = \frac\pi2.}
Легко видеть, что в силу леммы Римана -- Лебега
\eqn{I_n := \intl{0}{\pi} \hr{\frac{2}{t} - \frac{1}{\sin\frac t2}} \sin\hr{n+\frac12} t\dt \ra 0, \quad n \ra \bes.}
С другой стороны
\eqn{I_n = 2 \intl{0}{\pi} \frac{\sin\hr{n+\frac12} t}{t}\dt - \intl{0}{\pi} D_n(t) =
2 \intl{0}{\pi} \frac{\sin\hr{n+\frac12} t}{t}\dt - \pi.}
Делая замену $x = \hr{n+\frac12}t$, получаем
\eqn{\label{eqn:intDirichlet}I_n = 2 \intl{0}{\hr{n+\frac12}\pi} \frac{\sin x}{x}\dx - \pi.}
Но так как
\eqn{\intl{0}{\hr{n+\frac12}\pi} \frac{\sin x}{x}\dx \ra \intl{0}{\bes}\frac{\sin x}{x} \dx, \quad n \ra \bes,}
переходя к пределу в формуле~\eqref{eqn:intDirichlet}, получаем требуемое.

\subsubsection{Достаточное условие равномерной сходимости ряда Фурье}

\begin{stm}
Система функций $\hc{e^{inx}}$ полна в пространстве $\Cb[-\pi,\pi]$.
\end{stm}
\begin{proof}
Покажем, что если для непрерывной функции $f$ выполнено условие $f \bot e^{inx}$ при всех $n \in \Z$,
то $f \equiv 0$. Допустим противное. Без ограничения общности $f(0) =: C > 0$. Построим тригонометрический
многочлен $T_m$, для которого $\int T_m(x)\dx  = 1$ и $|T_m(x)| < \ep$ вне $\de$ -- окрестности нуля
(здесь $\de$ и $\ep$~--- произвольные наперёд заданные положительные числа). Рассмотрим многочлен
\eqn{T_m(x) := \frac{(1 + \cos x)^m}{\int (1+ \cos x)^m\dx}.}
Возьмём произвольную точку $x_0 \neq 0$ и обозначим $g(x) := 1 + \cos x$.
Тогда числитель у $T_m(x_0)$ равен $g^m(x_0)$, а интеграл в знаменателе, как хорошо видно из картинки
$$\epsfbox{pictures.1}$$
можно оценить снизу числом
\eqn{\hr{\frac{|x_0|}{2} + \frac{|x_0|}{2}}\cdot g^m\hr{\frac{x_0}{2}},}
выражающим площадь заштрихованной части под графиком функции $g^m(x)$.
Так как $g\hr{\frac{x_0}{2}} > g(x_0)$, основание показательной функции в знаменателе больше,
чем в числителе. Следовательно, знаменатель задавит числитель с ростом~$m$ (коэффициент
при знаменателе не зависит от $m$ и потому не повредит). Далее, ясно, что если
$|x| > |x_0|$, то оценка только улучшится. Таким образом, $T_m(x) \rra 0$ вне всякой
окрестности нуля при $m \ra \bes$.

В силу непрерывности, $f(x) > \frac{C}{2}$ в некоторой $\de$ -- окрестности нуля. Поэтому
\eqn{\int f(x) T_m(x)\dx \ge (1-\al)\cdot \frac{C}{2} - \al\cdot \hn{f}_\Cb, \quad \al := 2\ep(\pi-\de).}
Осталось взять $m$ столь большим, чтобы последнее слагаемое стало маленьким
за счёт $\ep$, а второе -- близким к~$\frac C2$ за счёт того же $\ep$.
Но это означает, что скалярное произведение $(f, T_m) > 0$. Противоречие.
\end{proof}

Для обоснования корректности следующей теоремы нам потребуется сделать два замечания.
Во-первых, если функция абсолютно непрерывна, то у неё почти всюду существует производная
и для таких функций работает интегрирование по частям.

Во-вторых, произведение абсолютно непрерывных функций снова является абсолютно непрерывной
функцией (доказательство очень похоже на доказательство того, что произведение непрерывных функций
непрерывно).

\begin{theorem}
Пусть функция $f$ абсолютно непрерывна и $f' \in L_2(-\pi,\pi)$. Тогда ряд Фурье сходится
к функции $f$ равномерно.
\end{theorem}
\begin{proof}
Коэффициенты Фурье для производной будем обозначать~$c_n'$. Оценим коэффициенты Фурье функции~$f$,
интегрируя по частям:
\eqn{c_n = \frac{1}{2\pi}\int f(x) e^{-inx}\dx = -\frac{1}{2\pi}\cdot
f(\pi)\hr{\frac{e^{i n \pi}}{i n} - \frac{e^{-i n \pi} }{i n}} + \frac{1}{i n}\cdot
\frac{1}{2\pi}\int f'(x) e^{-inx}\dx = 0 + \frac{c_n'}{i n}.}
Применяя неравенство $ab \le \frac{1}{2}(a^2 + b^2)$, получаем:
\eqn{|c_n| = \hm{c_n' \cdot \frac{1}{i n} } \le \frac12\hr{|c_n'|^2 + \frac1{n^2}}.}
Так как $f'\in L_2$, ряд $\sum |c_n'|^2$ сходится в силу неравенства Бесселя. В силу приведённой
оценки, сходится и ряд $\sum |c_n|$, а поскольку ряд $\sum c_n e^{inx}$ мажорируется
рядом $\sum|c_n|$, он сходится равномерно по признаку Вейерштрасса. Значит, он сходится
к некоторой непрерывной функции $\ph$.
В силу этой равномерной сходимости, функция~$\ph$ имеет те же коэффициенты
Фурье, что и функция~$f$. Значит, непрерывная функция $f - \ph$ имеет нулевые коэффициенты
Фурье и по предыдущему утверждению тождественно равна нулю.
\end{proof}

\subsubsection{Теорема Фейера}

\begin{stm}
Пусть $f \in L_1[a,b]$, а $\hc{\ph_k}$ -- равномерно ограниченная константой $M$
ортонормированная система на отрезке $[a,b]$. Тогда коэффициенты Фурье функции $f$
по этой системе стремятся к нулю.
\end{stm}
\begin{proof}
Пусть $f_\ep \in L_2[a,b]$ и $\hn{f - f_\ep}_{L_1} < \ep$. Тогда
\eqn{\int f \ol \ph_k = \int (f - f_\ep) \ol\ph_k + \int f_\ep \ol\ph_k.}
Первое слагаемое не превосходит $M\ep$, а второе стремится к нулю при $k \ra \bes$, потому что $f_\ep \in L_2$.
\end{proof}

Пусть $S_n(x)$~--- частные суммы ряда Фурье для функции $f$. Рассмотрим \emph{средние Фейера}:
\eqn{P_n(x) := \frac{S_0(x)\spl S_n}{n+1} = \frac{1}{2\pi(n+1)}\int f(x+t)\sumkzn D_k(t)\dt.}
Домножим числитель и знаменатель ядра Дирихле на $\sin\frac t2$. Тогда
\eqn{P_n(x) = \frac{1}{2\pi(n+1)}\int f(x+t) \sumkzn \frac{\sin\hr{k+\frac12}t\cdot \sin \frac t2}{\sin^2\frac t2}\dt.}
Свернём сумму под интегралом и получим \emph{ядро Фейера}: имеем
\eqn{\sin\hr{k+\frac12}t\cdot \sin \frac t2 = \frac12\br{\cos k t - \cos(k+1)t},}
поэтому в сумме числителей почти всё сократится, и останется $\frac{1 - \cos(n+1)t}{2} = \sin^2\frac{n+1}{2}t$.
Итак,
\eqn{F_n(t) := \frac{1}{n+1}\sumkzn \frac{\sin\hr{k+\frac12}t\cdot \sin \frac t2}{\sin^2\frac t2} =
\frac{1}{n+1}\cdot \frac{\sin^2\frac{n+1}{2}t}{\sin^2\frac t2}}
и
\eqn{P_n(x) = \frac{1}{2\pi}\int f(x+t) F_n(t)\dt.}

Отметим несколько необходимых нам свойств ядра Фейера.
\begin{items}{0}
\item $F_n(t) \ge 0$.
\item $\int F_n(t)\dt = 2\pi$, так как ядро Фейера -- это усреднённая сумма $(n+1)$ -- го ядра Дирихле.
\item При всяком фиксированном $\de > 0$ имеем
\eqn{\ints{|t| > \de}F_n(t)\dt \ra 0, \quad n \ra \bes.}
В самом деле, при $|t|> \de$ знаменатель ядра подпирается снизу константой $\frac\de\pi$, поэтому
\eqn{\frac{1}{n+1}\cdot \frac{\sin^2\frac{n+1}{2}t}{\sin^2\frac t2} \le
\frac{1}{n+1} \cdot \hr{\frac{\pi}{\de}}^2 \ra 0, \quad n \ra\bes.}
\end{items}

\begin{theorem}[Фейера]
Если функция $f$ непрерывна, то средние Фейера сходятся к ней равномерно.
\end{theorem}
\begin{proof}
В силу свойств ядра Фейера имеем
\eqn{f(x) - P_n(x) = \frac{1}{2\pi}\int \br{f(x) - f(x+t)}F_n(t)\dt =
\frac{1}{2\pi} \ints{|t| < \de} + \frac{1}{2\pi} \ints{|t| > \de} =: \frac{1}{2\pi}(I_1 + I_2).}
Поскольку функция $f$ непрерывна, она ограничена: $|f| \le M$. Кроме того, она равномерно
непрерывна, то есть $|f(x) - f(x+t)| < \ep$, как только $|t| < \de$. Тогда
\eqn{|I_1| \le \ep \ints{|t|<\de} F_n(t)\dt \le \ep \int F_n(t)\dt = 2\pi\ep,
\qquad |I_2| \le 2M \ints{|t|\ge \de} F_n(t)\dt.}
Сначала выберем маленькое $\ep$, для него найдётся какое-то $\de$, но по одному из свойств ядра Фейера,
последний интеграл стремится к нулю при любом фиксированном $\de$. Следовательно, $P_n \rra f$.
\end{proof}


\subsection{Преобразование Фурье}

\subsubsection{Определение преобразования Фурье. Формула обращения}

В этом разделе мы не будем писать пределы интегрирования,
если мы интегрируем по всей прямой~$\R$. Под пространством $L_1$ понимается пространство $L_1(\R)$.

\begin{df}
\emph{Преобразованием Фурье} функции $f \in L_1$ называется функция
\eqn{\wh f(\la) := \fsp \int f(x)e^{-i\la x}\dx.}
\end{df}

\begin{theorem}[Формула обращения]
Если в точке $x$ функция $f$ удовлетворяет условию Дини и непрерывна в этой точке, то значение
функции в точке $x$ можно восстановить по формуле
\eqn{f(x) = \vp \fsp \int \wh f(\la) e^{i\la x}\dla.}
\end{theorem}
\begin{proof}
Рассмотрим аналог частных сумм ряда Фурье:
\eqn{f_n(x) = \fsp \intl{-n}{n} \bbs{\fsp \int f(\xi) e^{-i\la \xi}\dxi} e^{i\la x}\dla.}
В силу интегрируемости $f$ на всей прямой имеем
\eqn{\bbm{\intl{-n}{n} \int f(\xi) e^{-i \la(\xi-x)}\dxi\dla} \le 2n\cdot \hn{f}_{L_1},}
а потому применима теорема Фубини, и можно поменять порядок интегрирования. Следовательно, имеем
\mln{f_n(x) = \fsp \int f(\xi)\bbs{\fsp \intl{-n}{n} e^{-i\la (\xi-x)}\dla}\dxi =
\fsp \int f(\xi)\bbs{\fsp \cdot \frac{e^{in(x-\xi)} - e^{-in(x-\xi)}}{i(x-\xi)}}\dxi=\\=
\frac{1}{\pi}\int f(\xi) \frac{\sin n(x-\xi)}{x-\xi}\dxi =
\frac{1}{\pi}\int f(x+t) \frac{\sin n t}{t}\dt.}

Как мы уже знаем,
\eqn{\int \frac{\sin nt}{t}\dt = \pi.}
Поэтому
\eqn{f_n(x) - f(x) = \frac1\pi\int \frac{f(x+t)-f(x)}{t}\sin nt\dt =
\frac1\pi\ints{|t| < A} + \frac1\pi\ints{|t|>A} =: \frac1\pi(I_1 + I_2).}
С первым интегралом всё ясно: для него применима лемма Римана -- Лебега,
поэтому он стремится к нулю с ростом $n$.
Второй интеграл разбивается в сумму двух интегралов:
\eqn{I_2 = \ints{|t|> A} \frac{f(x+t)}{t}\sin nt\dt + \ints{|t|>A} \frac{f(x)}{t}\sin nt\dt.}
Каждый из них может быть сделан маленьким за счёт выбора достаточно большого $A$.
\end{proof}

\begin{petit}
\textbf{Контрольный вопрос:} о каких интегралах в этом доказательстве идёт речь -- о
Римановских или о Лебеговских?
\end{petit}

\subsubsection{Свойства преобразования Фурье}

Непосредственно из определения следует, что
\eqn{|\wh f(\la)| \le \fsp \hn{f}_{L_1} \text{ при всех } \la \in \R.}
Отсюда следует, что если $f_n \ra f$ в $L_1$, то $\wh f_n \rra \wh f$ на $\R$.

\begin{stm}
Преобразование Фурье функции $f \in L_1$ является непрерывной функцией.
\end{stm}
\begin{proof}
В случае, когда $f = \Ibb_{[a,b]}$, доказательство тривиально:
\eqn{\label{eqn:ftInd}\wh f(\la) = \intl{a}{b}e^{-i\la x}\dx = -\frac{e^{-ib\la} - e^{-ia\la}}{i\la},}
и особенность в нуле, очевидно, устраняется. По линейности наше утверждение верно и для
ступенчатых функций. В общем случае приблизим нашу функцию ступенчатыми функциями~$f_n$.
Поскольку равномерный предел непрерывных функций непрерывен, получаем, что функция $\wh f$ тоже
непрерывна.
\end{proof}

\begin{imp}
Преобразование Фурье $\wh f(\la)$ убывает к нулю при $\la \ra \bes$.
\end{imp}
\begin{proof}
Для индикаторов это следует из формулы~\eqref{eqn:ftInd}. По линейности это верно и для
ступенчатых функций. Общий случай -- следствие равномерной сходимости.
\end{proof}

\begin{theorem}
Если $\wh f \equiv 0$, то и $f = 0$ в~$L_1$.
\end{theorem}
\begin{proof}
\eqn{0 \equiv \wh f(\la) = \fsp \int f(x) e^{- i\la x}\dx  = \fsp e^{-i\la t} \int f(x+t)e^{-i\la x}\dx.}
Сократим на ненулевой множитель перед интегралом и проинтегрируем наш тождественный нуль от~$0$ до~$T$:
\eqn{0\equiv \intl{0}{T} \int f(x+t)e^{-i\la x}\dx\dt= \int \bbs{\intl{0}{T}f(x+t)\dt} e^{-i\la x}\dx.}
Функция $F(x) := \intl{0}{T}f(x+t)\dt$ абсолютно непрерывна по $x$, значит, почти всюду дифференцируема,
а потому удовлетворяет условию Дини почти всюду. Значит, её преобразование Фурье сходится к ней почти
всюду.\footnote{Мы доказывали это для рядов Фурье. Для преобразования Фурье это делается аналогично,
нужно только заменить сумму $S_n$ на интеграл от $e^{-i\la x}$.}
Значит, $F(x) = \intl{x}{x+T}f(\xi)\dxi \equiv 0$, так как по предыдущей формуле
\eqn{\int F(x) e^{-i\la x}\dx \equiv 0,}
а это и есть преобразование Фурье для $F(x)$. Но поскольку $T$ произвольно, в силу абсолютной
непрерывности интеграла Лебега отсюда следует, что $f=0$ в $L_1$.
\end{proof}


\subsubsection{Связь между гладкостью и убыванием для функции и её образа Фурье}

\begin{lemma}
Пусть $f\in L_1 \cap \AC$ и $f' \in L_1$. Тогда
\eqn{\wh{f'}(\la) = i\la \wh f(\la).}
\end{lemma}
\begin{proof}
Имеем
\eqn{\int f'(x)e^{-i\la x}\dx = f(x)e^{-i\la x}\biggl|_{-\bes}^{+\bes} + i\la \int f(x)e^{-i\la x}\dx.}
Покажем, что внеинтегральные члены равны нулю. Действительно, имеем $f(x) = f(0) + \intl{0}{x}f'(t)\dt$,
но поскольку $f'\in L_1$, существует предел интеграла от $f'$ при $x\ra \bes$. Но этот предел может быть
только нулём, так как иначе разойдётся интеграл от $f$. Таким образом, $\wh{f'}(\la) = i\la \wh f(\la)$.
\end{proof}
\begin{imp}
Если $f,f',f''\in L_1 \cap \AC$, то $\wh{f''}(\la) = -\la^2\wh f(\la)$ и $\wh f$ суммируема.
\end{imp}
\begin{proof}
Первое утверждение очевидно. Далее,
имеем $\wh f(\la) = -\frac{\wh{f''}(\la)}{\la^2}$, а так как $\wh{f''}(\la)\ra 0$ при $\la \ra \bes$,
то $\wh f(\la) = o\hr{\frac1{\la^2}}$.
\end{proof}

\begin{imp}
Если $f\sco f^{(k)} \in L_1 \cap \AC$, то
\eqn{\wh{f^{(k)}}(\la) = (i\la)^k\wh f(\la), \qquad \wh f(\la) = o\hr{\frac1{\la^k}}.}
\end{imp}
\begin{proof}
Применяем индукцию, и всё получается.
\end{proof}

Покажем, что эти свойства обратимы:

\begin{lemma}
Если $f \in L_1$ и $xf(x) \in L_1$, то $(\wh f)' = -i \wh{x f(x)}$.
\end{lemma}
\begin{proof}
Будем искать производную преобразования Фурье функции $f$:
\eqn{\frac{\wh f(\la+h) - \wh f(\la)}{h} = \fsp \int f(x) \frac{e^{-i(\la + h)x} - e^{-i\la x}}{h}\dx.}
Поскольку дробь под интегралом ограничена, применима теорема Лебега о предельном переходе. Следовательно,
\eqn{\liml{h\ra 0} \frac{\wh f(\la+h) - \wh f(\la)}{h} = \fsp \int f(x)(-i x) e^{-i \la x}\dx = -i \wh{x f(x)},}
что и требовалось доказать.
\end{proof}

\begin{imp}
Если $x^m f(x) \in L_1$ при $m=0\sco k$, то $(\wh f)^{(k)} = (-i)^k \wh{x^k f(x)}$.
\end{imp}
\begin{proof}
Снова применяем индукцию, и снова всё получается.
\end{proof}

\begin{theorem}
Если для некоторого $\de > 0$ функция $f(x) e^{\de|x|}$ интегрируема, то функция $\wh f$ аналитична
в полосе $|\Im \ze| < \de$.
\end{theorem}
\begin{proof}
Рассмотрим преобразование Фурье с комплексным параметром $\ze = \la + i \mu$:
\eqn{\wh f(\ze) = \fsp \int f(x) e^{-i\ze x}\dx.}
Видно, что при $|\mu| < \de$ интеграл существует. Поступая так же, как и при доказательстве предыдущей леммы,
имеем
\eqn{\frac{\wh f(\ze+h) - \wh f(\ze)}{h} = \fsp \int f(x) \frac{e^{-i(\ze + h)x} - e^{-i \ze x}}{h}\dx \ra
\fsp \int f(x)(-i x) e^{-i \ze x}\dx, \quad h \ra 0.}
Тем самым показано, что функция дифференцируема в указанной полосе. Но, как мы знаем из комплексного анализа,
этого достаточно для аналитичности. Впрочем, и так очевидно, что полученная производная сама удовлетворяет условиям
теоремы, откуда следует бесконечная дифференцируемость $\wh f$.
\end{proof}

Введём обозначения для операторов дифференцирования и умножения на независимую переменную. Положим $\Dc f(x) := i \frac{df}{dx}$,
и $\Mc f(x) := xf(x)$. Оператор преобразования Фурье обозначим через $\Fc$. По доказанному выше,
справедливы следующие коммутационные соотношения:
\eqn{\Fc\Dc = -\Mc\Fc, \qquad \Dc \Fc = \Fc\Mc.}


\subsubsection{Равенство Парсеваля для преобразования Фурье. Теорема Планшереля}


В дискретном случае имеет место равенство $\hn{f}^2_{L_2} = \sum |c_n|^2$. Имеется и непрерывный аналог этого равенства.

\begin{df}
\emph{Носителем непрерывной функции} называется множество $\supp f := \Cl \hc{x\cln f(x)\neq 0}$.
\end{df}

\begin{df}
Говорят, что функция $f\cln\R^n\ra\R^m$ \emph{финитна}, если найдётся шар $B \subs \R^n$ такой,
что $f \equiv 0$ вне $B$.
\end{df}

\begin{lemma}[Равенство Парсеваля]
Преобразование Фурье сохраняет скалярное произведение в $L_2$.
\end{lemma}
\begin{proof}
Пусть для простоты функции $f,g\in \Cb^2(\R)$ и финитные. Для таких функций работает формула обращения:
\mln{(f,g) = \int f(x)\ol g(x)\dx = \int f(x)\ol{\hs{\fsp \int \wh g(\la)e^{i\la x}\dla}}\dx =
\iint \fsp f(x)\ol{\wh g(\la)}e^{-i\la x}\dx\dla =\\=
\int\hs{\fsp \int f(x)e^{-i\la x}\dx}\ol{\wh g(\la)}\dla  = \int \wh f (\la)\ol{\wh g(\la)}\dla = (\wh f, \wh g),}
что и требовалось доказать.
\end{proof}

\begin{theorem}[Планшереля]
Пусть $f\in L_2(\R)$. Тогда функции
\eqn{\fsp \intl{-n}{n}f(x)e^{-i\la x}\dx}
при $n\ra\bes$ сходятся по норме $L_2(\R)$ к некоторой функции $U f$, где $U\cln L_2\ra L_2$ -- унитарный оператор.
Если при этом $f\in L_1(\R)$, то $Uf = \wh f$.
\end{theorem}
\begin{proof}
Проверим равенство Парсеваля для финитных функций. Пусть сначала функция $f$ равна нулю вне некоторого отрезка $[A,B]$. Рассмотрим последовательность $\Cb^2$ -- гладких
финитных функций $\hc{\ph_k}$ такую, что $\ph_k \ra f$ в~$L_2$. Для функции $f$ определено обычное
преобразование Фурье, так как она лежит в $L_1(\R)$. Заметим, что $\ph_k \ra f$ в $L_1$, а потому $\wh \ph_k \rra \wh f$
на $\R$. Кроме того, в силу уже доказанного для хороших функций равенства Парсеваля, последовательность $\hc{\wh \ph_k}$
фундаментальна в $L_2(\R)$, а потому имеет там некоторый предел. Но в силу  равномерной сходимости $\wh \ph_k$ это
может быть только функция $\wh f$. Таким образом, $\nn{\wh f} = \hn{f}$.

В общем случае рассмотрим функции $f_n := f \cdot \Ibb_{[-n,n]}$. Для них справедливо предыдущее рассуждение, значит,
$\nn{\wh f_n} = \hn{f_n}$. В силу фундаментальности $f_n$ в $L_2$ и уже доказанного для таких функций равенства
Парсеваля, последовательность $\wh f_n$ фундаментальна в $L_2$, а потому сходится к некоторой функции $U f$.
В силу непрерывности $L_2$ -- нормы имеем
\eqn{\hn{U f} = \liml{n\ra\bes}\nn{\wh f_n} = \liml{n\ra\bes}\hn{f_n} = \hn{f}.}
Таким образом, оператор $U$ сохраняет норму (а значит, и скалярное произведение).
Тем самым получено изометричное отображение $U\cln L_2\ra L_2$.

Докажем теперь вторую часть теоремы. Пусть теперь $f \in L_1 \cap L_2$. Тогда
$f_n \ra f$ в $L_1$ и $f_n\ra f$ в $L_2$. Значит, $\wh f_n \rra \wh f$, а
в силу равенства Парсеваля $\nc{\wh f_n}$ фундаментальна в $L_2$ и потому сходится к некоторой функции $g$.
Отсюда $\wh f = g$ почти всюду.
\end{proof}

\subsubsection{Система функций Чебышёва -- Эрмита}

Изучим подробнее оператор $U$, полученный в теореме Планшереля.
Ниже будет показано, что оператор $U^4 = \id$ в $L_2(\R)$, поэтому
в силу леммы об отображении спектра, $\Sig(U)$ содержится среди корней четвёртой степени из единицы.

Рассмотрим систему функций
\eqn{p_n(x) := x^n e^{-\frac{x^2}{2}}.}
Очевидно, что $p_n \in L_2(\R)$. Их линейная независимость очевидна. Применяя к этой системе процесс
ортогонализации, получаем ортонормированную систему, которая называется системой Чебышёва -- Эрмита.

Можно проверить, что функции
\eqn{\exp\hr{-\frac{x^2}{2}}, \qquad (1-2x^2)\exp\hr{-\frac{x^2}{2}},
\qquad (-3x+2x^3)\exp\hr{-\frac{x^2}{2}}, \qquad x\exp\hr{-\frac{x^2}{2}}}
являются собственными для оператора $U$ с собственными значениями со значением $1$, $-1$, $i$ и $-i$ соответственно.
Таким образом, спектр оператора $U$ является точечным и состоит в точности из чисел $i^k$, $k = 0,1,2,3$.

\begin{theorem}
Пусть $f \in L_2(\R)$ и не равна нулю почти всюду на~$\R$.
Кроме того, пусть $f(x) e^{\de|x|} \in L_2(\R)$ для некоторого $\de > 0$. Тогда система
функций $\hc{g_n(x) := x^n f(x)}$ полна в $L_2(\R)$.
\end{theorem}
\begin{proof}
Пусть нашлась функция $h$ такая, что $(h,g_n)=0$ при всех $n$. Покажем, что $h=0$. Имеем
\eqn{\int x^n f(x)\ol{h(x)}\dx = 0.}
В частности, этот интеграл существует при $n=0$, поэтому $f(x)\ol {h(x)}\in L_1$. Значит, к этой функции можно
применить преобразование Фурье:
\eqn{g(\la) := \fsp \int f(x)\ol{h(x)}e^{-i\la x}\dx.}
Так как $f(x)e^{\de|x|}\in L_2(\R)$, то функция $g(\la)$ будет аналитической в полосе ширины $2\de$.
Заметим, что все её производные
в нуле с точностью до коэффициента совпадают со скалярными произведениями $(h,g_n)$, которые равны нулю:
\eqn{g^{(n)}(0) = (-i)^n \cdot \fsp \int x^n f(x)\ol{h(x)}\dx.} Значит, $g\equiv 0$, а потому
$f(x)h(x)=0$ почти всюду. Но так как $f(x)\neq 0$ почти всюду, то $h(x)=0$ почти всюду.
\end{proof}

\subsubsection{Свёртка функций и её преобразование Фурье}

\begin{df}
Пусть $f,g\in L_1(\R)$. \emph{Свёрткой} функций $f$ и $g$ называется функция
\eqn{(f*g)(x) := \int f(x-y)g(y)\dy.}
\end{df}

Из теоремы Фубини следует, что свёртка существует, так как существует кратный интеграл, который линейной
заменой сводится к интегралу $\iint f(x)g(y)\dx\dy$, в существовании которого сомнений не возникает.
\begin{stm}
Имеет место равенство $\wh{f * g} = \sqrt{2\pi}\cdot\wh f \cdot \wh g$.
\end{stm}
\begin{proof}
Имеем
\mln{
\fsp \int \bbs{\int f(x-y)g(y)\dy}e^{-i\la x}\dx =
\fsp \int \bbs{\int f(x-y)e^{-i\la(x-y)}e^{-i\la y}\dx}g(y)\dy=\\=
\fsp \int \bbs{\int f(\xi)e^{-i\la\xi}\dxi}e^{-i\la y}g(y)\dy =
\wh f(\la)\cdot \int e^{-i\la y}g(y)\dy= \sqrt{2\pi}\cdot \wh f(\la)\cdot \wh g(\la),}
что и требовалось доказать.\footnote{Кстати, можно было бы избавиться от множителя $\sqrt{2\pi}$
в этой формуле, если коэффициент $\fsp$  добавить в определение свёртки.}
\end{proof}

\subsubsection{Решение уравнения теплопроводности}

Пусть функция $u(x,t)$ удовлетворяет уравнению теплопроводности
\eqn{u_t = u_{xx}, \quad x \in \R, \quad t \ge 0,}
причём известно начальное распределение тепла $u(x,0) = \ph(x)$.

Наложим на функцию $u(x,t)$ дополнительные условия:
\begin{enumerate}
\item Пусть $u$, $u_x$, $u_{xx}$ интегрируемы по всей оси~$x$ для любого
фиксированного $t\ge 0$.
\item Существует интегрируемая функция $f$, для которой $|u_t(x,t)|\le f(x)$.
\end{enumerate}

Возьмём преобразование Фурье от производной $u_t$ по переменной $x$:
\eqn{\wh{u_t}(\la)=\ints{\R} u_t e^{-i\la x}\dx =
\pf{}{t}\ints{\R} u(x,t) e^{-i\la x}\dx = \pf{}{t} v(\la,t), \quad \text{ где }
\quad v(\la,t):=\ints{\R}u(x,t)e^{-i\la x}\dx.}
Аналогично,
\eqn{\wh{u_{xx}}(\la) = -\la^2\wh u (\la) = -\la^2 v(\la,t).}

Таким образом, получаем обыкновенное дифференциальное уравнение ($\la$ считаем параметром):
\eqn{\frac{d}{dt} v = -\la^2 v.}
Решая его, получаем
$v(\la,t)=C(\la) e^{-\la^{2}t}$. Вычислим константу, исходя из начальных условий.
\eqn{C(\la) = v(\la,0)= \ints{\R}u(x,0)e^{-i\la x}\dx = \wh \ph(\la),}
то
\eqn{v(\la,t)=\wh\ph(\la) e^{-\la^{2}t}.}
Применяя формулу обращения к функции $v$, получаем
\eqn{u(x,t)=\frac{1}{2\sqrt{\pi t}}\ints{\R}e^{-\frac{\xi^2}{4t}}\ph(x-\xi)d\xi.}
Здесь мы пользуемся тем, что обратное преобразование Фурье переводит произведение
преобразований Фурье в свёртку исходных функций, и тем, что обратное преобразование Фурье
функции $e^{-\la^2 t}$ легко считается.
Полученное выражение называется интегралом Пуассона.


\subsubsection{Оператор Фурье в пространстве Шварца}

\begin{df}
\emph{Пространство Шварца} $\Sc$ состоит из всех бесконечно дифференцируемых функций, для которых выполнено
\eqn{\hn{f}_{p,q} := \maxl{\R} \bm{x^p f^{(q)}} < \bes \quad \text{ при всех } p, q \in \Z_+.}
Введённая система полунорм $\hn{\cdot}_{p,q}$ превращает $\Sc$ в метрическое пространство.
\end{df}

\begin{lemma}
Пространство $\Sc$ плотно в $L_2(\R)$ по норме $L_2$.
\end{lemma}
\begin{proof}
Легко видеть, что функции Чебышёва -- Эрмита, введённые нами ранее, являются функциями класса $\Sc$.
Осталось вспомнить, что они образуют полную ортонормированную систему.
\end{proof}

\begin{stm}
Оператор $\Fc\cln \Sc \ra \Sc$ корректно определён.
\end{stm}
\begin{proof}
Пусть $f \in \Sc$. Покажем, что $\wh f \in \Sc$. Выкладки проведём с точностью до множителей:
\eqn{\la^p(\wh f)^{(q)}(\la) = \Mc^p\Dc^q\Fc f = \Mc^p\Fc\Mc^q f = \Fc\Dc^p\Mc^q f = \Fc\hs{\br{x^q f(x)}^{(p)}}.}
Расписывая $p$ -- ю производную по правилу Лейбница, получим линейную комбинацию произведений
производных функции $f$ и степеней~$x$. Поскольку $f \in \Sc$, функции вида $x^m f^{(k)}$ заведомо
лежат в $L_1(\R)$. А мы знаем, что преобразование Фурье функции из $L_1(\R)$ -- это ограниченная функция.
Значит, каждое слагаемое в нашей сумме равномерно ограничено, а потому и вся сумма равномерна ограничена.
Итак, мы показали, что $\nn{\wh f}_{p,q} < \bes$ при всех $p, q$, но это и означает, что $\wh f \in\Sc$.
\end{proof}

Заметим, что для функций класса $\Sc$ работает формула обращения. Пусть $g \in \Sc$.
Через $\Oc$ обозначим оператор отражения: $\Oc f(x) := f(-x)$. Рассмотрим
\eqn{f(\la) := \wh g(\la) = \fsp \int g(x) e^{-i\la x}\dx.}
Запишем для $g$ формулу обращения:
\eqn{g(x) = \fsp \int f(\la)e^{i\la x}\dla = \fsp \int f(-\la)e^{-i\la x}\dla.}
Таким образом, получаем, что
\eqn{g = \Fc \Oc\Fc g.}
Поскольку $\Oc$ и $\Fc$ коммутируют, получаем, что $\Fc^2 = \Oc$ и $\Fc^4 = \id$.

Теперь заметим, что эти свойства можно распространить на всё пространство $L_2(\R)$,
поскольку $\Sc$ в нём плотно.

\subsubsection{Теорема Пэли -- Винера}

\begin{petit}
Данный раздел добавлен в лекции в мае 2010 года по просьбе лектора. Доказательство теоремы
прислано Натальей Побыванец (\texttt{n.pobyvanets@gmail.com}). Нами были исправлены \emph{замеченные}
опечатки, но это не гарантирует, что оставшийся текст не содержит ошибок и неточностей.
\end{petit}

\begin{theorem}[Пэли -- Винера]
Функция $f \in \mbox{L}_2(-a,a)$ тогда и только тогда, когда
$\wh{f}$ -- целая функция и $|\wh{f}(\la)| \leqslant C_f \cdot e^{a|\emph{Im}\la|}$.
\end{theorem}

\begin{proof}
\textbf{Необходимость.}
Пусть $f\in\mbox{L}_2 (-a,a)$. Тогда $\wh{f}(\la)=\frac{1}{\sqrt{2\pi}}\intl{-a}{a} f(x) e^{-ix\la}\: dx$. Верна оценка
$$|\wh{f}(\la)|\leqslant \frac{1}{\sqrt{2\pi}}\intl{-a}{a} |f(x)| e^{x|\mathrm{Im}\la|}\: dx \leqslant C_f e^{a|\mathrm{Im}\la|}.$$
Функция $\wh{f}(\la)=\frac{1}{\sqrt{2\pi}}\intl{-a}{a} f(x) e^{-ix\la}\dx$ непрерывна в~каждой точке комплексной плоскости
и~имеет непрерывную производную. Следовательно, это целая функция.

\textbf{Достаточность.}
Пусть $g\in\mbox{L}_2(\R )$ и $|g(\la)|\leqslant C e^{a|\mathrm{Im}\la|}$. Докажем, что $g = \wh{f}$, где $f\in\R $ и обращается
в~ноль почти всюду вне $[-a,a]$. Наложим дополнительное условие на~$g$: $g(\la)\leqslant\frac{Ce^{a|\mathrm{Im}\la|}}{|1+\la^2|}$.
Воспользуемся формулой обращения. Тогда $f(x) = \frac{1}{\sqrt{2\pi}}\ints{\R }g(\la)e^{i\la x}\: d\la$.
По лемме Коши контур интегрирования можно сдвинуть в комплексную плоскость:
$$f(x)=\frac{1}{\sqrt{2\pi}}\ints{\R} g(\la + ib)e^{i(\la +ib) x}\, d\la.$$
Тогда для $\la\in\R $:
$$
|f(x)|\leqslant\frac{1}{\sqrt{2\pi}}\ints{\R}\frac{Ce^{ab}}{|1+\la^2|}|e^{(i\la - b)x}|\,d\la \leqslant
\frac{1}{\sqrt{2\pi}}\ints{\R}\frac{Ce^{ab}}{|1+(\la +ib)^2|}e^{-bx}\: d\la \rightarrow 0, \quad  b\rightarrow\infty.
$$
Так как значение интеграла на~самом деле не зависит от~$b$, получаем, что $f(x)=0$ при $x>a$. Аналогично, устремляя $b$ к $-\infty$,
получаем, что $f(x)=0$ при $x<-a$.

Остается избавиться от условия убывания $g(\la)$ на бесконечности.
Пусть $g\in\mbox{L}_2(\R)$, тогда $\exi f\in\mbox{L}_2(\R )$, такая что $\wh{f}=g$. Возьмём
$\ph_{\ep}\colon \R \rightarrow\mathbb{R_+}$, $\ph_{\ep}\in C^{\infty}$, $\supp \ph_{\ep}\in [-\ep, \ep]$
и~$\int\ph_{\ep}=1$. Из уже доказанной части теоремы следует, что $|\wh{\ph_{\ep}}(\la)|\leqslant C_{\ep}e^{\ep|\mathrm{Im}\la|}$.
Вместо функции $g$ рассмотрим функцию $g\cdot\wh{\ph_{\ep}}$. Тогда $f\ast\ph_{\ep}$ -- обратное преобразование Фурье
функции $g\cdot\wh{\ph_{\ep}}$. Заметим, что $g$ -- целая и
$|g\cdot\wh{\ph_{\ep}}(\la)|\leqslant\frac{De^{(a+\ep)|\mathrm{Im}\la|}}{|1+\la^2|}$, $\ph_{\ep}$ -- гладкие,
$|\wh{\ph_{\ep}}|$ -- убывает (в силу гладкости). Из уже доказанной части теоремы следует, что $f=0$ почти всюду вне $[-a,a]$.
Таким образом, $f\ast\ph_{\ep} = 0$ вне $[-a-\ep, a+\ep]$. Докажем, что $f=0$ почти всюду вне $[-a,a]$ для $\fa g$.
Нам потребуется следующие определения:
\begin{df}
Пусть $M$ -- измеримое по Лебегу множество. Точка $c$ называется точкой плотности для $M$, если
$\frac{\la(M\cap [c-\ep, c+\ep])}{2\ep}\rightarrow 1$ при $\ep\rightarrow 0$, где $\la$ -- мера Лебега.
\end{df}
\begin{df}
Точка $c$ -- точка Лебега для функции $f$, если $\exi a\in\R $, что
$$\frac{1}{2\ep}\intl{c-\ep}{c+\ep}|f(x)-a|\; dx \rightarrow 0.$$
\end{df}
\begin{ex}
Любая точка непрерывности -- точка Лебега.
\end{ex}
\begin{problem}
Обратное неверно, приведите контрпример.
\end{problem}
\begin{note}
Заметим, что точка Лебега для индикатора множества является точкой плотности для этого множества.
\end{note}
Доказательство того, что $f=0$ почти всюду вне $[-a,a]$ для $\forall g$ поведем от противного.
Пусть $f \ne 0$ почти всюду левее $-a$. Тогда $\exi c < -a$, такое что $c$ -- точка Лебега для $f$.
Тогда верно, что
$$\frac{1}{2\de}\ints{[c-\de,c+\de]}|f(x)-b|\dx < \frac{1}{100}.$$
По определению свертки $f\ast\ph_{\ep} = \ints{|s|<\ep}f(x-s)\ph_{\ep}(s)\; ds$.
Тогда в любой $s$-окрестности (при $s<\ep$) любой точки $\overline{x}$ можно построить функцию
$f(x-s)\ph_{\ep}(s)$ и приблизить ее функцией $b\cdot\ph_{\ep}(s)$. Тогда (по определению $\ph_{\ep}$)
получаем $f\ast\ph_{\ep}\ne 0$ почти всюду вне $[-a-\ep, a+\ep]$. Получаем противоречие.
Аналогично доказывается в случае предположения, что $f \ne 0$ почти всюду правее~$a$.
Следовательно, $f=0$ почти всюду вне $[-a-\ep, a+\ep]$.
\end{proof}

\section{Обобщённые функции}

\subsection{Обобщённые функции на пространстве $\Dc$}

\subsubsection{Пространство $\Dc$ основных функций}

\begin{df}
Пространство $\Dc$ состоит из бесконечно дифференцируемых функций с компактным носителем.
Оно ещё иногда обозначается $\Cb^\bes_0$.
\end{df}

\begin{df}
Пусть $ \hc{\ph_n} \subs \Dc$. Говорят, что $\ph_n \ra \ph \in \Dc$, если найдётся отрезок $[A,B]$ такой,
что $\supp \ph_n \bw\subs [A,B]$ при всех $n$ и $\ph_n^{(m)} \rra \ph^{(m)}$ при всех $m$.
\end{df}

\begin{df}
\emph{Обобщённой функцией} на пространстве $\Dc$ называется линейный непрерывный функционал $F\cln \Dc \ra \Cbb$.
Непрерывность означает, что если $\ph_n \ra \ph$ в $\Dc$, то и $F(\ph_n) \ra F(\ph)$. Функции $\ph\in\Dc$
называются \emph{основными} (а иногда ещё \emph{тестовыми} или \emph{пробными}).
Пространство обобщённых функций на $\Dc$ обозначается через $\Dc'$.
\end{df}

Как мы потом узнаем, сходимость в $\Dc$ можно задать топологией. Вообще, топологизуемость сходимости -- очень
нетривиальный факт. Следующее утверждение показывает, что это не всегда можно сделать.

\begin{stm}
Сходимость почти всюду нельзя задать топологией.
\end{stm}
\begin{proof}
Рассмотрим пример Рисса последовательности функций, сходящихся по~мере к~нулю,
но не сходящихся почти всюду. Допустим, что есть топология, задающая нашу сходимость.
Так как нет сходимости почти всюду, то, в~частности, нет сходимости почти всюду к~нулю (то есть
сходимости к~нулю в~нашей топологии). Значит, найдётся окрестность нуля $U$,  вне которой находится
бесконечно много элементов последовательности. Эти элементы образуют подпоследовательность
исходной последовательности, поэтому тоже сходятся по~мере к~нулю. По теореме Рисса из~них можно выбрать
из них подпоследовательность, сходящуюся почти всюду. С~одной стороны, она обязана сходиться почти всюду
именно к~нулю. С~другой стороны, вся она лежит вне некоторой окрестности $U$ нуля. Противоречие.
\end{proof}

\subsubsection{Примеры обобщённых функций}

\begin{ex}
Пусть $f$ -- локально -- суммируемая на прямой функция. Тогда функционалы вида
\eqn{F_f(\ph) := \ints{\R} f(x)\ph(x)\dx}
являются обобщёнными функциями. Проверим непрерывность: пусть $\ph_n \ra 0$ в~$\Dc$ и~$\supp \ph_n \subs [A,B]$,
тогда
\eqn{F(\ph_n) = \ints{\R} f(x)\ph_n(x)\dx \le \maxl{\R}\hm{\ph_n(x)}\cdot\intl{A}{B}\hm{f}\dx \ra 0, \quad n \ra \bes.}
\end{ex}

\begin{df}
Обобщённые функции, значения которых на основных функциях задаются как интеграл
от произведения основной функции и локально -- суммируемой функции,
называются \emph{регулярными}. Все остальные обобщённые функции называются \emph{сингулярными}.
\end{df}

Часто значение обобщённой функции $F$ на основной функции $\ph$ записывают так: $\ha{F,\ph}$.

Пока мы не знаем, существуют ли сингулярные функции. Сейчас узнаем\ldots

\begin{ex}
\emph{Дельта -- функцией Дирака} называется функция, действующая по правилу
\eqn{\ha{\de,\ph} := \ph(0).}
Линейность и непрерывность такого функционала очевидна.
\end{ex}

\begin{stm}
$\de$ -- функция сингулярна.
\end{stm}
\begin{proof}
Допустим, что существует такая локально -- суммируемая функция $f$, что
\eqn{\ha{\de,\ph} = \ph(0) = \int f(x)\ph(x)\dx \text{ для всех } \ph \in\Dc.}
Рассмотрим функцию
\eqn{\ph_a(x) := \case{\exp\frac{a^2}{x^2-a^2},& |x| < a;\\0,& |x|\ge a.}\qquad\lower1.5pc\hbox{\epsfbox{pictures.2}}}
Имеем $\ph_a(0) = \frac1e$ при всех $a$. Легко видеть, что $\ph_a \ra 0$ почти всюду при $a \ra 0$,
поэтому интеграл тоже обязан стремиться к нулю. С другой стороны, он должен быть равен ненулевой константе.
Противоречие.
\end{proof}

\begin{ex}
Производная $\de$ -- функции действует так:
\eqn{\ha{\de',\ph} := -\ph'(0).}
Линейность и непрерывность такого функционала тоже очевидна.
\end{ex}

\begin{note}
Пока <<производная>> -- это только название. Чуть позже мы увидим, что обобщённые функции
можно дифференцировать сколько угодно раз, и окажется, что производная от $\de$ -- функции
действует именно таким образом.
\end{note}

Всякой локально суммируемой функции мы поставили в соответствие некоторую обобщённую функцию.
Покажем, что это соответствие инъективно.

\begin{stm}
Если для локально -- суммируемой функции $f$ имеет место равенство
\eqn{\int f(x)\ph(x)\dx = 0 \text{ при всех } \ph \in \Dc,}
то $f = 0$ почти всюду.
\end{stm}
\begin{proof}
В нашем распоряжении есть основные функции, являющиеся <<почти индикаторами>> отрезков,
то есть функции $\ph_\ep$, равные $1$ на заданном отрезке $[a,b]$ и нулю вне $\ep$ -- окрестности этого отрезка.
Пусть $\ph_\ep$ -- последовательность  <<почти  индикаторов>> отрезка~$[a,b]$.
Имеем
\eqn{\int f(x)\ph_\ep(x)\dx = 0,}
поэтому по теореме Лебега о предельном переходе при $\ep \ra 0$ получаем, что
\eqn{\intl{a}{b}f(x)\dx = 0.}
Итак, интеграл по всякому отрезку от функции $f$ равен нулю. Значит, $f = 0$ почти всюду.
\end{proof}

\begin{ex}
Действие обобщённой функции $\Pc\frac1x$ задаётся так:
\eqn{\ha{\Pc\frac1x,\ph} = \vp\int\frac{\ph(x)}{x}\dx.}
\end{ex}

Покажем, что функция $\Pc\frac1x$ сингулярна. В самом деле,
если бы существовала регулярная функция $F_f$, задающая то же действие,
что и $\Pc\frac1x$, то, в частности, оно совпадало бы на основных функциях,
для которых $\supp\ph \subs \R\wo \hc{0}$. Но для таких функций
регуляризация уже есть -- это функция $\frac1x$ (так как носитель $\ph$ не содержит
нуля, он отделён от него некоторой окрестностью, поэтому значение $\vp$ -- интеграла
совпадает с обычным интегралом). Значит, если регуляризация есть, то она
должна почти всюду совпадать с $\frac1x$. Но эта функция, очевидно, не является
локально -- суммируемой.

\subsubsection{Действия над обобщёнными функциями. Дифференцирование}

Начнём с наводящих соображений. Пусть $F_f$ -- регулярная обобщённая функция,
причём $f$ дифференцируема. Тогда
\eqn{\ha{F_f,\ph} = \int f(x)\ph(x)\dx.}
Спросим себя, чему должно быть равно $F_{f'}$?
Интегрируя по частям и вспоминая, что основные функции финитны, получаем
\eqn{\ha{F_{f'},\ph} = \int f'\ph\dx = -\int f\ph'\dx = -\ha{F_f,\ph'}.}
Итак, в этом случае обобщённые функции можно дифференцировать, перекидывая производные
на основную функцию. Это соображение и положено в основу следующего определения.

\begin{df}
Пусть задана обобщённая функция $F$. Её \emph{производной} называется функционал
\eqn{\ha{F',\ph} := -\ha{F,\ph'}.}
\end{df}

Линейность такого функционала очевидна, осталось доказать непрерывность.
В самом деле, если $\ph_n \ra 0$ в $\Dc$, то и $\ph_n' \ra 0$ в $\Dc$. Следовательно,
\eqn{\ha{F',\ph_n} = -\ha{F, \ph'_n} \ra 0, \quad n \ra \bes.}

Напомним, что $*$ -- слабой сходимостью обобщённых функций называется следующее:
говорят, что $F_n \ra F$, если для всех $\ph \in \Dc$ имеем $F_n(\ph) \ra F(\ph)$.

\begin{stm}
Оператор дифференцирования на обобщённых функциях непрерывен в смысле $*$ -- слабой сходимости.
\end{stm}
\begin{proof}
Пусть $F_n \convws F$. Покажем, что $F_n'\convws F'$. Действительно,
\eqn{\ha{F'_n,\ph} = -\ha{F_n,\ph'}\ra -\ha{F,\ph'} = \ha{F',\ph},}
что и требовалось доказать.
\end{proof}

Из определения дифференцирования следует, что обобщённые функции можно дифференцировать
сколько угодно раз, так как $\ph\in \Cb^\bes$.

Скажем пару слов о \emph{замене переменной} в обобщённых функциях. Определим, например, что такое $\de(x-a)$.
Ответ ясен: это функционал, который основной функции $\ph$ ставит в соответствие её значение в точке $a$.

Пусть $x(\xi)$ -- гладкая монотонная замена переменной. Рассмотрим регулярную функцию $F_f$. Тогда
\eqn{\ha{F_f(x),\ph(x)} = \int f\br{x(\xi)}\ph\br{x(\xi)}\frac{dx}{d\xi}\dxi.}
Тогда, в силу того что определена обратная функция $\xi(x)$, положим
\eqn{\ha{F\br{x(\xi)},\ph(\xi)} := \ha{F(x), \ph\br{\xi(x)}\frac{d\xi}{dx}}.}

А ещё обобщённые функции можно \emph{умножать на гладкие функции}.
Пусть $\psi \in \Cb^\bes$, а $F \in \Dc'$. Тогда
\eqn{\ha{\psi F, \ph} := \ha{F, \psi\ph}.}

\subsubsection{Формула суммирования Пуассона}

\begin{df}
\emph{Функцией Хевисайда} называется функция
\eqn{\ta(x) := \case{1, & x\ge0,\\0,&x < 0.}}
\end{df}

\begin{lemma}
$\ta'(x) = \de(x)$.
\end{lemma}
\begin{proof}
В самом деле, $\ha{\ta',\ph} = -\ha{\ta,\ph'} = -\intl{0}{\bes}\ph'(x)\dx = \ph(0) = \ha{\de,\ph}$.
\end{proof}

\begin{lemma}
Частичные суммы ряда $\suml{n=1}{\bes}\frac{\sin nx}{n}$ ограничены в совокупности.
\end{lemma}
\begin{proof}
В силу периодичности и нечётности синуса достаточно доказать утверждение для $x\in[0,\pi]$. При $x = 0$ доказывать нечего.
Разобьём сумму на два слагаемых:
пусть $m := \min\hc{N, \hs{\frac\pi x}}$, тогда
\eqn{\suml{n=1}{N}\frac{\sin n x}{n} = \suml{n=1}{m} + \suml{n=m+1}{N} =: S_1 + S_2.}
Так как $|\sin x| \le |x|$, то
\eqn{|S_1| \le \suml{n=1}{m}\frac{n x}{n} \le \hs{\frac\pi x}\cdot x \le \pi.}
Для оценки второй суммы применим преобразование Абеля:
\eqn{\suml{k=1}{n} a_k b_k  = a_n B_n + \suml{k=1}{n-1}B_k(a_k-a_{k+1}), \quad \text{ где }\quad  B_k := \suml{i=1}{k}b_i.}
Положим $a_k := \frac1k$, а $b_k := \sin kx$. Тогда
\eqn{S_2 = \suml{n = m+1}{N}\frac{\sin n x}{n} = \frac1N\suml{k=m+1}{N}\sin k x +
\suml{n=m+1}{N-1}\hs{\suml{k=m+1}{n} \sin k x} \hr{\frac1n-\frac1{n+1}}.}
Первое слагаемое можно грубо оценить по модулю числом~$1$, поскольку в сумме меньше, чем~$N$, слагаемых,
не превосходящих по модулю~$1$. Далее, домножим и поделим второе слагаемое на $\sin\frac x2$.
После сворачивания суммы синусов числитель оценивается по модулю константой~$1$. Тогда
\mln{|S_2| \le 1 + \suml{n=m+1}{N-1}\frac{1}{\sin\frac x2}\hr{\frac1n-\frac1{n+1}} =
1 + \frac{1}{\sin\frac x2}\suml{n=m+1}{N-1}\hr{\frac1n-\frac1{n+1}}=\\=
1 + \frac{1}{\sin\frac x2}\hr{\frac{1}{m+1} - \frac{1}{N}} \le
1 + \frac{1}{\sin\frac x2(m+1)}.}
Так как $\sin \frac x2 \ge \frac x\pi$, а $m \approx \frac{\pi}{x}$, знаменатели равномерно ограничены некоторой константой, что и требовалось.
\end{proof}

\begin{theorem}[Формула Пуассона]
\eqn{\sums{\Z} e^{ikx} = 2\pi \sums{\Z} \de(x-2\pi k).}
\end{theorem}
\begin{proof}
Сразу скажем, как всё это следует понимать. Подразумевается, что выражения равны как обобщённые функции, то есть
они совпадают при действии на тестовых функциях. Тогда слева получится сумма коэффициентов Фурье функции $\ph$,
а справа -- сумма её значений в точках $2\pi k$ (ввиду финитности функции $\ph$ сумма конечна).

Покажем, что ряд Фурье $\suml{n=1}{\bes}\frac{\sin nx}{n}$ сходится к функции $\frac{\pi-x}{2}$, продолженной
с отрезка $[0,2\pi]$ периодически на всю ось. Рассмотрим функцию
\eqn{\ln (1-z) = -z - \frac{z^2}{2} - \frac{z^3}{3} - \ldots}
и подставим $z = e^{i\ph}$. Тогда
\eqn{\Im\ln(1-z) = \arg\hr{1-e^{i\ph}} = -\frac{\pi -\ph}{2} = -\frac{\sin\ph}{1} - \frac{\sin2\ph}{2} - \frac{\sin 3\ph}{3}-\ldots}
Значит, $\suml{n=1}{\bes}\frac{\sin n\ph}{n} = \frac{\pi-\ph}{2}$.

Теперь рассмотрим наш ряд как сумму регулярных обобщённых функций и продифференцируем его:
\eqn{\suml{n=1}{\bes} \cos n x =  -\frac12 + \pi \sums{\Z}\de(x-2\pi n),}
так как в точках $2\pi n$ скачок функции равен $\pi$. Далее, сумму косинусов представим в виде экспонент:
\eqn{\suml{n=1}{\bes}\cos n x  + \frac12 = \suml{n=1}{\bes} \frac{e^{inx}+ e^{-inx}}{2} + \frac12 = \frac12\sums{\Z}e^{inx}.}
Домножая полученное равенство на $2$, получаем искомую формулу. Законность всех этих преобразований обеспечивается
предыдущими леммами.
\end{proof}

\subsubsection{Дифференциальные уравнения в классе обобщённых функций}

\begin{lemma}
Функция $\ph \in \Dc$ является производной функции из $\Dc$ тогда и только тогда, когда $\int \ph = 0$.
\end{lemma}
\begin{proof}
Справа налево это очевидно: $\int \psi' =\psi\Bsub{-\bes}{+\bes} = 0$. Наоборот: положим
$\psi(x) := \intl{-\bes}{x}\ph(t)\dt$. Полученная функция будет финитной, так как при $x > \max \hc{t\cln t\in \supp \ph}$
интеграл обнулится.
\end{proof}
\begin{imp}
Любая функция $\ph\in D$ представима в виде $\ph = \ph_0 + \la \ph_1$, где $\ph_1$ -- некоторая фиксированная
функция, для которой $\int \ph_1 = 1$, а $\int \ph_0 = 0$.
\end{imp}
\begin{proof}
Действительно, пусть $\ph_1$ -- такая функция, что $\int \ph_1 =1$. Положим
$\la := \int\ph$ и $\ph_0 := \ph - \la \ph_1$.
\end{proof}

\begin{theorem}
Дифференциальное уравнение $F' = 0$, где $F\in \Dc'$, имеет лишь постоянные решения $F_C$.
\end{theorem}
\begin{proof}
Имеем $\ha{F',\ph} = -\ha{F,\ph'}=0$. По предыдущей лемме функции производные функций из $\Dc$ есть в точности такие
функции $\ph$, что $\int\ph =0$. Представим любую функцию из $\Dc$ в виде $\ph = \ph_0 + \la \ph_1$ и положим
$C := \ha{F,\ph_1}$. Тогда, если $F$ -- решение, то
\eqn{\ha{F,\ph} = \ha{F, \ph_0 + \la \ph_1} = \ub{\ha{F,\ph_0}}_{0}+\la \ha{F,\ph_1} = C\la = C\int\ph = \ha{F_C,\ph}.}
Таким образом, уравнение имеет только <<классическое>> решение $F= \const$.
\end{proof}

\begin{note}
Решений в классе обобщённых функций может быть как больше, так и меньше.
Например, уравнение $xy'=0$ имеет два решения: $y(x)\equiv 0$ и $y(x)=\ta(x)$. А вот уравнение $-\frac12x^3 y'=y$
имеет классическое решение $y = e^{\frac{1}{x^2}}$, не допускающее регуляризации (это мы докажем позже), поэтому
в классе $\Dc'$ имеется лишь тривиальное решение $y\equiv0$.
\end{note}


\begin{lemma}
Обобщённые решения линейных однородных систем с гладкими коэффициентами суть классические.
\end{lemma}
\begin{proof}
Рассмотрим систему
\eqn{y' = A(x)y.}
Пусть $\Ph(x)$ -- фундаментальная матрица этой системы. Тогда $\Ph' = A\Ph$.
Сделаем замену $y = \Ph z$ и подставим в исходную систему. Получим
$\Ph' z + \Ph z' = Ay = A\Ph z$, но $\Ph'z = A\Ph z$, поэтому получим уравнение $\Ph z' = 0$.
А про его решения мы всё знаем: они только классические. Значит, исходная система имеет только
классические решения.
\end{proof}


\begin{theorem}
Для любой обобщённой функции $G$ уравнение $F' = G$ имеет решение.
\end{theorem}
\begin{proof}
Если $F$ -- решение, то
\eqn{\ha{F, \ph'} = - \ha{F', \ph} = -\ha{G,\ph}.}
Таким образом, мы уже знаем, как $F$ должно действовать на производные финитных функций.
Представим функцию $\ph$ в виде
\eqn{\ph = \wt\ph + \ph_0\int\ph, \text{ где } \int\ph_0 = 1.}
Доопределим $F$: положим $\ha{F,\ph_0} := 0$.

Покажем, что так определённый линейный функционал $F$
непрерывен. Пусть $\ph_n \ra 0$. Из разложения
\eqn{\ph_n = \wt \ph_n + \ph_0\int\ph_n}
видно, что $\wt\ph_n \ra 0$. Тогда
\eqn{\ha{F,\ph_n} = \ha{F, \wt\ph_n} + \int\ph_n\cdot \ub{\ha{F,\ph_0}}_0 \ra 0,}
потому что мы знаем, что действие $F$ на $\wt\ph_n$ -- это (с точностью до знака)
действие $G$ на первообразную $\psi_n(x) := \intl{-\bes}{x}\wt\ph_n(t)\dt$. Но ясно, что
$\psi_n \ra 0$, поэтому и $\ha{G,\psi_n} \ra 0$.
\end{proof}

Допустим, что мы умеем решать однородные системы вида $y' = Ay$. Научимся решать системы вида
\eqn{y'-Ay = \vec F.}
Пусть $\Ph$ -- фундаментальная матрица однородной системы. Тогда $\Ph' = A\Ph$.
Снова делаем замену $y = \Ph z$ и подставим в систему:
\eqn{\Ph'z + \Ph z' - A\Ph z = \vec F.}
Но $\Ph'z = A\Ph z$, поэтому  уравнение примет вид
\eqn{\Ph z' = \vec F.}
Отсюда получаем систему $z' = \Ph^{-1} \vec F$, которую мы уже умеем решать.

\subsection{Структура обобщённых функций}

\subsubsection{Регуляризация обобщённых функций}

Пусть у функции $f$ в нуле имеется неинтегрируемая особенность, а во всех остальных точках
всё хорошо. Тогда, конечно, $\int f \ph$ может и не существовать. Однако, если $0 \notin \supp \ph$,
интеграл $\int f\ph$ существует. Поэтому если $f$ растёт не более чем полиномиально (как $x^{-d}$)
при $x \ra 0$, то функцию $f$ как обобщённую можно \emph{регуляризовать}.
Пусть $\Ibb$ -- гладкий индикатор отрезка $[-1,1]$.
Представим всякую финитную функцию в виде суммы функции, имеющей ноль не менее, чем $d$ -- го порядка
в нуле (подпространство таких функций обозначим через $D_d$),
и некоторой линейной комбинации фиксированных функций из $\Dc$. Легко видеть, что разложение
\eqn{\ph(x) = \BS{\ph(x) - \suml{k=0}{d-1} \frac{\ph^{(k)}(0)}{k!}x^k \cdot \Ibb(x)} +
\suml{k=0}{d-1} \frac{\ph^{(k)}(0)}{k!}x^k \cdot \Ibb(x)}
является искомым.
На функциях $\ph_d \in D_d$ определим функционал так:
\eqn{\ha{F_f,\ph_d} := \int f(x)\ph_d(x)\dx.}
А на функциях $\psi_k := x^k \cdot \Ibb(x)$ доопределим его так:
\eqn{\ha{F_f, \psi_k} := (-1)^k\ba{\de^{(k)},\psi_k} = \ba{\de,\psi_k^{(k)}} = k!}
Таким образом, на произвольной функции $\ph \in \Dc$ функционал $F_f$ будет действовать так:
\eqn{\ha{F_f,\ph} = \int f(x)\BS{\ph(x) - \suml{k=0}{d-1} \frac{\ph^{(k)}(0)}{k!}x^k \cdot \Ibb(x)}\dx +
\suml{k=0}{d-1}\ph^{(k)}(0).}

\begin{df}
\emph{Горбом} в точке $a$ называется гладкая неотрицательная функция $\ph$ с носителем $[-1,1]$, такая, что
$\ph(x) = 1$ при $x \in \hs{a-\frac14,a+\frac14}$ и $\int \ph = 1$.
\end{df}

\begin{theorem}
Если $f$ растёт быстрее любой обратной степени $x$ при $x \ra 0$, то задача регуляризации неразрешима,
то есть не существует такой обобщённой функции $F$, что на функциях $\ph \in \Dc$, носитель которых не
содержит нуля, она действует как
\eqn{\ha{F, \ph} = \ha{F_f,\ph} = \int f\ph\dx.}
\end{theorem}
\begin{proof}
Рассмотрим горб $\psi(x)$ в точке $0$. Теперь сожмём его в $n$ раз и сдвинем в точку $\frac2n$,
то есть рассмотрим функцию \eqn{\psi_n(x) := \psi\hr{n\hr{x- \frac2n}}.}
А теперь рассмотрим функции
\eqn{\ph_n(x) := \ep_n \cdot \psi_n(x),}
где $\ep_n := \frac{2n}{f\hr{\frac3n}} \ra 0$, потому что $f$ стремится к нулю быстрее любой степени~$x$.
Заметим, что $\supp \psi_n = \hs{\frac1n,\frac3n}$. Тогда
\eqn{0 \ar \ha{F,\ph_n} = \ep_n\int f(x) \psi\hr{n\hr{x- \frac2n}}\dx \ge \ep_n\frac1{2n}f\hr{\frac3n} = 1.}
Противоречие.
\end{proof}

\subsubsection{Разбиение единицы}

\begin{theorem}[О разбиении единицы]
Пусть $\Om \subs\R$ -- открытое множество, а $U_\al$ -- его открытое покрытие.
Тогда существует набор $\hc{\psi_i} \subs \Dc$ таких, что носитель каждой функции $\psi_i$
лежит в некотором~$U_\al$ (для разных $i$ индекс $\al$ может быть различным), и таких,
что
\begin{points}{-2}
\item $0 \le \psi_i \le 1$;
\item $\sum\psi_i(x) \equiv 1$ на $\Om$;
\item Для всякого компакта $K \subs \Om$ найдутся $i_1\sco i_m$ и окрестность $U \sups K$ такие, что
$\psi_{i_1} \spl \psi_{i_m} = 1$ на $U$.
\end{points}
\end{theorem}
\begin{proof}
Будет написано позже.
\end{proof}

\begin{df}
Такой набор функций $\hc{\psi_i}$ называется \emph{разбиением единицы}, подчинённым покрытию $U_\al$.
\end{df}


\subsubsection{Вторая конструкция разбиения единицы}


\begin{df}
Говорят, что покрытие $\hc{V_\be}$ \emph{вписано} в покрытие $\hc{U_\al}$, если всякое $V_\be$ целиком
содержится в некотором $U_\al$.
\end{df}

\begin{df}
Покрытие $U_\al$ называется \emph{локально конечным}, если для любой точки $x \in \bigcup U_\al$ над ней
висит лишь конечное число элементов покрытия.
\end{df}

Простой пример $U_n := \hr{-1 + \frac1n,1 - \frac1n}$ показывает, что не из всякого покрытия
можно выделить локально конечное. Однако во всякое покрытие можно вписать локально конечное.
Доказательство этого факта мы пока предоставляем читателю.

Мы будем писать $U \Subset V$, если $\Cl U \subs V$.

\begin{stm}
Пусть $\hc{U_i}$ -- локально конечное покрытие $\R$. Тогда найдётся разбиение единицы,
подчинённое покрытию~$\hc{U_i}$.
\end{stm}
\begin{proof}
Без ограничения общности можно считать $U_i$ ограниченными.
Впишем в $\hc{U_i}$ покрытие $\hc{V_i}$ такое, что $V_i \Subset U_i$: рассмотрим дополнение $F_1$
к множеству $U_2\cup U_3\cup\dots$. Имеем $F_1 \subs U_1$. Теперь найдём $V_1 \sups F_1$ такое,
что $V_1 \Subset U_1$. Заметим, что $V_1 \cup U_2 \cup U_3 \cup\dots$ покрывают $\R$.

Аналогично, на $k$ -- м шаге рассмотрим дополнение $F_k$ к множеству
\eqn{V_1\cup V_2 \cup \ldots \cup V_{k-1} \cup U_{k+1} \cup U_{k+2} \cup\dots.}
и аналогично получим множество $V_{k+1} \Subset U_{k+1}$. И так далее.

\begin{petit}
Продолжение следует...
\end{petit}
\end{proof}



\subsubsection{Носитель обобщённой функции}

\begin{df}
Говорят, что обобщённая функция $F$ \emph{равна нулю} на интервале $I$, если $\ha{F,\ph} = 0$ для всех $\ph$,
у которых $\supp \ph \subs I$.
\end{df}

Легко видеть, что в этом определении интервал можно заменить произвольным
открытым множеством.

\begin{df}
\emph{Носителем} $\supp F$ обобщённой функции $F$ называет дополнение к наибольшему открытому множеству~$U \subs \R$,
на котором $F = 0$.
\end{df}

Возникает вопрос: почему такое определение корректно, или, другими словами, почему носитель вообще существует?
Ответ на него даёт следующее предложение.

\begin{prop}
Пусть $F = 0$ на каждом из открытых множеств $U_\al$. Тогда она равна нулю на их объединении $U := \bigcup U_\al$.
\end{prop}
\begin{proof}
Рассмотрим разбиение единицы~$\hc{\psi_i}$, подчинённое покрытию $\hc{U_\al}$.
Пусть $\supp \ph \subs U$. Тогда, поскольку $\sum \psi_i = 1$ в $U$, можно написать,
что $\ph = \sum\ph\psi_i$. Поскольку $\supp\ph$ -- компакт, в силу локальной конечности
разбиения единицы найдутся такие $i_1\sco i_m$, что $\psi_{i_1}\spl \psi_{i_m} = 1$ в некоторой
окрестности $\supp \ph$, то есть в этой окрестности сумма $\sum\ph\psi_i$ на самом деле
конечна. Поэтому
\eqn{\ha{F,\ph} = \Ba{F, \suml{k=1}{m}\ph\psi_{i_k}} = \suml{k=1}{m}\ha{F,\ph\psi_{i_k}} = 0,}
поскольку носитель каждой из функций $\ph\psi_{i_k}$ лежит в некотором множестве $U_\al$ (а~на~каждом~$U_\al$
функция $F$ равна нулю по условию).
\end{proof}

\begin{df}
Точку $x$ назовём \emph{существенной} для $F$, если для любой окрестности $U(x)$ существует $\ph$
с носителем в $U$, для которой $\ha{F,\ph} \neq 0$.
\end{df}

\begin{problem}
Множество всех существенных точек совпадает с носителем функции $F$.
\end{problem}

\subsection{Другие виды основных и обобщённых функций: пространства $\Sc$ и $\Ec$}

\subsubsection{Пространство $\Ec$. Вложение $\Dc$ в $\Ec$}

Введём в пространстве $\Cbb^\bes(\R)$ систему полунорм: пусть $K$ -- компакт, тогда
положим
\eqn{\hn{\ph}_{K,m} := \maxl{K} \bm{\ph^{(m)}(x)} = \bn{\ph^{(m)}}_{\Cb(K)}.}
Пространство $\Cb^\bes$ с такой системой полунорм обозначим через $\Ec$. Скажем, что
$\ph_n \ra 0$ в $\Ec$, если $\hn{\ph_n}_{K,m} \ra0$ для всех компактов $K$ и для всех $m$.

\begin{note}
Естественно, что можно ограничиться только какой -- нибудь счётной последовательностью компактов,
исчерпывающих $\R$. Например, можно рассматривать не все компакты, а только отрезки $[-n,n]$,
тогда множество полунорм будет счётно. В дальнейшем мы придумаем ещё более экономную систему
полунорм.
\end{note}

\begin{df}
Пространство $\Ec'$ -- это пространство непрерывных линейных функционалов на пространстве основных функций~$\Ec$.
\end{df}


\begin{stm}
Пространство $\Dc$ непрерывно вкладывается в $\Ec$ и плотно в $\Ec$ по метрике $\Ec$.
\end{stm}
\begin{proof}
Покажем непрерывность вложения: если $\ph_n \ra 0$ в $\Dc$, то последовательность
их образов в $\Ec$ (то есть этих же самых функций, но рассматриваемых в другом пространстве)
тоже сходится к нулю в $\Ec$. В самом деле, пусть $\supp\ph_n \subs [-a,a]$ при всех $n$.
По определению сходимости в $\Dc$, на отрезке $[-a,a]$ имеется равномерная сходимость
к нулю всех производных функций $\ph_n$. Значит, все полунормы в пространстве $\Ec$ тем более
пойдут к нулю.

Докажем, что $\Dc$ плотно в $\Ec$. Возьмём произвольную функцию из $\Ec$ и домножим её
на гладкий индикатор~$\psi_n$ отрезка $[-n,n]$. Тогда при всяком фиксированном компакте $K$
и фиксированном $m$ имеем
\eqn{\bn{\psi_n\ph-\ph}_{K,m} = 0,}
как только $K \subs [-(n+1),n+1]$.
\end{proof}


\begin{stm}
Имеет место вложение $\Ec' \inj \Dc'$ и $\Ec'$ плотно в $\Dc'$.
\end{stm}
\begin{proof}
Пусть $F \in \Ec'$. Покажем, что её можно рассматривать и как обобщённую функцию
на пространстве $\Dc$. Проблемы могут быть только с непрерывностью, потому что
с областью определения всё заведомо хорошо: $\Dc \subs \Ec$.

Пусть $\ph_n \ra 0$ в $\Dc$. Тогда по только что доказанному утверждению,
имеем $\ph_n\ra 0$ в $\Ec$. Но $F$ -- обобщённая функция на $\Ec$, поэтому она
непрерывна, а значит, $\ha{F,\ph_n} \ra 0$. Но это и надо доказать.

Далее, пусть $F \in \Dc'$. Найдём последовательность обобщённых функций $\hc{F_n}$ из $\Ec'$
такую, что $F_n \ra F$ в $\Dc'$. Рассмотрим функции $F_n := \psi_n F$,
где $\psi_n$ -- гладкий индикатор отрезка $[-n,n]$. Покажем, что это обобщённые функции из $\Ec'$.
Действительно, возьмём $\hc{\ph_k} \subs \Cb^\bes$ такую, что $\ph_k \ra 0$ в $\Ec$.
Легко видеть, что $\psi_n\ph_k \ra 0$ в $\Dc$ при $k \ra \bes$, а, так как $F \in \Dc'$, получаем, что
\eqn{\ha{F_n,\ph_k} = \ha{\psi_n F,\ph_k} = \ha{F,\psi_n\ph_k} \ra 0, \quad k \ra \bes.}

Далее, покажем, что $F_n \ra F$ в $\Dc'$ при $n \ra \bes$.
Действительно, рассмотрим $\ph \in \Dc$. Тогда
\eqn{\ha{F_n,\ph} = \ha{\psi_n F,\ph} = \ha{F,\psi_n\ph} = \ha{F, \ph},}
как только $\supp \ph \subs [-n,n]$. Таким образом, сходимость есть.
\end{proof}


\subsubsection{Ещё раз о системе полунорм в $\Ec$}

Время идёт, и настала пора сэкономить на полунормах. Итак, начинаем перестройку\dots

\begin{lemma}[О перестройке системы полунорм в $\Ec$]
Сходимость в пространстве $\Ec$ эквивалентна сходимости по системе полунорм
\eqn{P_m(\ph) := \maxl{k \le m} \maxl{x \in [-m,m]} \bm{\ph^{(k)}(x)}.}
\end{lemma}
\begin{proof}
Для доказательства в одну сторону достаточно загнать компакт $K$ в отрезок $[-m,m]$
и потребовать, чтобы порядок производной в определении старой полунормы не превосходил $m$,
после чего требуемая оценка очевидна.

Обратно, пусть имеется сходимость по старой системе. Докажем, что будет сходимость и по новой.
Для этого возьмём компакт побольше и запихнём в него отрезок $[-m,m]$. Осталось потребовать,
чтобы $\hn{\ph_n}_{K,k} \le \ep$ при всех $k \le m$.
\end{proof}

\begin{stm}
Пусть $X$ -- счётно -- нормированное пространство с системой полунорм $\hn{\cdot}_k$. Тогда $X$ метризуемо.
\end{stm}
\begin{proof}
Введём в $X$ метрику:
\eqn{\rho(x,y) := \suml{m=1}{\bes} \frac{1}{2^m}\cdot \frac{\hn{x-y}_m}{1 + \hn{x-y}_m}.}
То, что это метрика, легко проверяется (неочевидно только неравенство треугольника,
да и то легко доказывается геометрически с использованием свойств выпуклости функции $\frac{x}{1+x}$).

Докажем эквивалентность сходимости. Если есть сходимость по такой метрике,
то, очевидно, каждое слагаемое должно стремиться к нулю. Но тогда и полунормы устремятся
к нулю в силу того, что $\frac{x}{1+x} \ra 0$ тогда и только тогда, когда $x \ra 0$.

Обратно, пусть все полунормы сходятся к нулю. Пусть нужно сделать
расстояние меньше, чем $2\ep$. Поскольку $\frac{x}{1+x} < 1$,
ряд мажорируется прогрессией, поэтому его хвост можно сделать меньше $\ep$, начиная с некоторого $n$.
Осталось дождаться, пока первые $n$ полунорм станут в сумме меньше, чем $\ep$, тогда сумма всего
ряда не превзойдёт $2\ep$.
\end{proof}

\begin{stm}
Пусть $F \in \Ec'$. Тогда найдутся $m$ и $C> 0$ такие, что
\eqn{|F(\ph)| \le C \cdot P_m(\ph).}
\end{stm}
\begin{proof}
В силу непрерывности функционала $F$ получаем, что если $\rho(0,\ph) < \de$, то и $|F(\ph)| < 1$.
В силу сходимости ряда, его остаток можно сделать малым при всех $\ph$: выберем $m_0$ таким,
чтобы сумма $m_0$ -- хвоста была меньше~$\frac{\de}{2}$.
Выберем теперь $\de_1$ так, чтобы
\eqn{\suml{m=0}{m_0} \frac{1}{2^m}\cdot \frac{P_m(\ph)}{1+ P_m(\ph)} < \frac{\de}{2},}
когда $P_{m_0}(\ph) < \de_1$. Заметим, что полунормы $P_m$ монотонно возрастают,
поэтому достаточно взять $\de_1 < \frac{\de}{4}$.

Итак, расстояние от $\ph$ до нуля не превосходит $\frac{\de}{2} + \frac\de2 = \de$,
когда $P_{m_0}(\ph) < \frac{\de}{4}$, поэтому для таких $\ph$ имеем $F(\ph) < 1$.
Пусть $\ph$ -- произвольная функция. Если $P_{m_0}(\ph) \neq 0$, то всё доказано:
имеем
\eqn{P_{m_0}\hr{\frac{\de_1}{P_{m_0}(\ph)}\ph} = \de_1,}
поэтому для функции $\frac{\de_1}{P_{m_0}(\ph)}\ph$ имеем
\eqn{\hm{F\hr{\frac{\de_1}{P_{m_0}(\ph)}\ph}} < 1,}
откуда в силу линейности функционала $F$ получаем требуемую оценку
\eqn{\hm{F(\ph)} < \frac1{\de_1} \cdot P_{m_0}(\ph).}

Однако может получиться мелкая неприятность: если $P_{m_0}(\ph) = 0$, то на ноль делить нехорошо.
Но это не испортит нам жизнь: покажем, что в этом случае $F(\ph) = 0$.
Допустим, что это не так. Тогда, если $P_{m_0}(\ph) = 0$, то и $P_{m_0}(k\ph) = 0$.
Но мы знаем, что когда $P_{m_0}(\ph) < \frac{\de}{4}$, то и $|F(\ph)| < 1$.
Таким образом, при всех $k$ имеем $\hm{F(k\ph)} = k\hm{F(\ph)} < 1$. Но этого не может быть.
\end{proof}

\subsection{Структура обобщённых функций на $\Dc$}

\subsubsection{Обобщённые функции с компактным носителем}

Мы уже знаем, что $\Ec' \subs \Dc'$. Теперь мы узнаем, какую именно часть они там составляют.

\begin{theorem}
Обобщённые функции из $\Dc'$ с компактным носителем -- это в точности функции из пространства $\Ec'$.
\end{theorem}
\begin{proof}
Пусть $F \in \Ec'$. Покажем, что её носитель компактен. По предыдущему утверждению,
найдётся $m$ такое, что $|F(\ph)| \le C\cdot P_m(\ph)$. Пусть $\supp\ph \subs \br{\R \wo [-m,m]}$,
тогда $P_m(\ph)=0$ и потому $F(\ph) = 0$. Таким образом, $\supp F \subs [-m,m]$.

Обратно, пусть $F \in \Dc'$ и множество $K := \supp F$ компактно.
Покажем, что $F$ можно продолжить до непрерывного функционала на $\Ec$.
Возьмём ограниченную окрестность $U$ компакта $K$ и рассмотрим гладкую финитную функцию $\psi$,
равную $1$ в этой окрестности. Положим
\eqn{\wt F(\ph) := F(\psi\ph).}
Проверим, что $\wt F \in \Ec'$ и что $\wt F$ продолжает $F$ с $\Dc$ на $\Ec$.
Пусть $\ph_n \ra \ph$ в $\Ec$. Тогда $\psi\ph_n \ra \psi\ph$ в $\Ec$ по формуле Лейбница.
Но так как $\psi\ph_n \in \Dc$, то уже из непрерывности $F$ на $\Dc$ следует сходимость
$\wt F (\ph_n) \ra \wt F(\ph)$.

Покажем, что $\wt F = F$ на $\Dc$. Пусть $\ph \in \Dc$. Тогда
$\wt F(\ph) - F(\ph) = F(\psi\ph) - F(\ph) = F\br{(\psi-1)\ph} = 0$, потому что
$\supp (1-\psi) \cap \supp F = \es$.

Тот факт, что от выбора окрестности $U$ ничего не зависит, докажите самостоятельно. Это очевидно.
\end{proof}

\begin{problem}
Сходимость в $\Dc$ не метризуема.
% Саше Пановой посвящается
\end{problem}
\begin{solution}
Вначале докажем лемму, которая полезна сама по себе.
\begin{lemma}
Пусть $(M,\rho)$ -- метрическое пространство, $f_{ij}\in M$. Пусть
\eqn{
\begin{aligned}
f_{11}\z f_{12}\z f_{13}\z&\ldots\ra f_1,\\
f_{21}\z f_{22}\z f_{23}\z&\ldots\ra f_2,\\
& \ldots  \\
f_{n1}\z f_{n2}\z f_{n3}\z&\ldots\ra f_n,\\
& \ldots
\end{aligned}}
Пусть последовательность $\hc{f_n} \subs M$ тоже сходится к некоторой функции~$f\in M$. Тогда
можно выбрать из каждой строки таблицы по функции $f_{ni_n}$ так, что $f_{n\,i_n} \ra f$.
\end{lemma}
\begin{proof}
В силу сходимости в первой строке, существует номер $i_1$ такой, что $\rho(f_{1i_1}, f_1) < \frac12$.
Далее, в силу сходимости во второй строке, существует номер $i_2$ такой, что $\rho(f_{2i_2}, f_2) < \frac14$.
Продолжая аналогично, получим, что существует номер $i_n$ такой, что $\rho(f_{ni_n}, f_n) < \frac1{2^n}$.
Таким образом, $\rho(f_{ni_n}, f_n) \ra 0$, но по условию $\rho(f_n, f) \ra 0$. Отсюда следует,
что пределы последовательностей $f_{ni_n}$ и $f_n$ совпадают (по неравенству треугольника).
\end{proof}
Предположим, что в $\Dc$ можно ввести такую метрику $\rho$, что сходимость в $\Dc$ равносильна
сходимости по этой метрике. Покажем, что в пространстве $\Dc$ утверждение леммы неверно.
Пусть $\Ga_n$ -- стандартный горб высоты~$1$, сосредоточенный на отрезке $[-n,n]$.
В качестве таблицы функций рассмотрим $\hc{\Ga_{nm}}$, где ${\Ga_{nm} := \frac1m\Ga_n}$.
Очевидно, что при всяком фиксированном $n$ имеем $\Ga_{nm} \ra 0$ в $\Dc$ при $m \ra \bes$.
В то же время, любая выборка по одной функции из строки не может сходиться в $\Dc$, потому что
носители <<расползаются>> вширь (это противоречит определению сходимости в $\Dc$).
\end{solution}

\subsubsection{Пространство $L_1^*$}

\begin{theorem}
Пусть $M$ -- измеримое подмножество в $\R$. Тогда $L_1^*(M) = L_\bes(M)$.
\end{theorem}
\begin{proof}
Пусть сначала $M = I = (a,b)$.
Рассмотрим линейный непрерывный функционал $\Ph\cln L_1(M) \ra \Cbb$. Рассмотрим функцию
\eqn{F(x) := \Ph(\Ibb_{(a,x)}).}
Она абсолютно непрерывна, потому что
\eqn{|F(x) - F(y)| \le \hn{\Ph}\cdot|x-y|.}
Значит, для неё справедлива формула Ньютона -- Лейбница, и она представляется интегралом
\eqn{F(x) = \intl{a}{x} h(t)\dt,}
где $h = F'$ -- некоторая локально суммируемая функция.
Покажем, что она существенно ограничена. В самом деле, в силу липшицевости,
производная функции $F$ равномерно ограничена константой $\hn{\Ph}$.

Осталось показать, что для произвольной функции $g \in L_1$ имеем
\eqn{\Ph(g) = \intl{a}{b} h(t)g(t)\dt.}
Для индикаторов это уже проверено. По линейности это верно и для ступенчатых функций.
Покажем, что это верно и для произвольных функций. Приблизим функцию $g \in L_1$
ступенчатыми функциями $g_n$. Тогда $\Ph(g_n) \ra \Ph(g)$ в силу непрерывности функционала.
С другой стороны, имеем
\eqn{\Ph(g_n) = \intl{a}{b} h(t)g_n(t)\dt \ra \intl{a}{b} h(t)g(t)\dt,}
потому что
\eqn{\Bm{\intl{a}{b} h(t)g_n(t)\dt - \intl{a}{b} h(t)g(t)\dt} \le \hn{\Ph}\cdot \hn{g_n-g}_{L_1} \ra 0.}
Но последовательность $\Ph(g_n)$ не может иметь двух пределов, значит,
\eqn{\Ph(g) = \intl{a}{b} h(t)g(t)\dt,}
что и требовалось доказать.

Для произвольных множеств $M \subs \R$ сделаем так: положим его в интервал $(a,b)$,
определим функционал на $(a,b)$ и ограничим его на множество $M$.
\end{proof}

\begin{petit}
Последнюю часть этой теоремы стоило бы написать поподробнее. Но
это будет сделано позже.
\end{petit}


\subsubsection{Локальное устройство обобщённых функций из $\Dc'$}

Пусть $K$ -- компакт. Символом $\Dc_K$ мы будем обозначать пространство
функций, у которых $\supp \ph \subs K$.

\begin{theorem}
Всякая функция $F \in \Dc'$ локально есть производная от непрерывной функции.
\end{theorem}
\begin{proof}
Утверждение теоремы означает, что для любого компакта $K$ существует непрерывная
функция $f$ такая, что на всех функциях из $\Dc_K$ имеем $\ha{F,\ph} = \ba{f^{(m)},\ph}$.

Без ограничения общности, $K \subs (0,1)$. Будем рассматривать функции, у которых $\supp \psi \subs (0,1)$.
Там, где это не оговорено противное, интегралы берутся по отрезку $(0,1)$.

\eqn{\max |\psi(x)| \le \int\bm{\psi'(t)}\dt \le \int \max \bm{\psi'(t)}\dt = \max |\psi'(x)|,}
и аналогично,
\eqn{\max \bm{\psi^{(s-1)}(x)} \le \int\bm{\psi^{(s)}(t)}\dt \le \max \bm{\psi^{(s)}(x)}.}
Отсюда следует, что
\eqn{\hn{\psi}_m  = \maxl{s\le m}\bm{\psi^{(s)}(x)} \le \int \bm{\psi^{(m+1)}(t)}\dt.}

Рассмотрим гладкий индикатор $\Ibb_K$ нашего компакта, такой, что $\supp \Ibb_K\subs (0,1)$.
Тогда функция $G:=\Ibb_K\cdot F$ имеет компактный носитель, и для неё верна оценка $|G(\ph)|\le C\hn{\ph}_M$
для некоторого $M$. В силу полученного выше неравенства имеем
\eqn{|G(\ph)| \le C \cdot \int \bm{\psi^{(M+1)}(t)}\dt.}
Заметим, что финитная функция восстанавливается по своей производной однозначно. Значит, она восстанавливается
и по $(n+1)$ -- й производной. Рассмотрим функционал
\eqn{\wt G\cln\ph^{(M+1)}\mapsto G(\ph), \text{ где } \ph\in \Dc_K.}
Он ограничен относительно нормы в $L_1$. Продолжим его до линейного непрерывного функционала на всём $L_1$.
По доказанной ранее теореме он имеет вид
\eqn{G(\ph) = \int h(t)\ph^{(M+1)}(t)\dt = -\int f(t)\ph^{(M+2)}(t)\dt, \text{ где } f(x) := \intl{-\bes}{x}h(t)\dt.}
Равенство обосновано финитностью $\ph$ и формулой интегрирования по частям.
Но это ровно то, что нужно, поскольку на нашем компакте $K$ функция $G$ совпадает с $F$.
\end{proof}

Для доказательства общего случая теоремы о структуре обобщённых функций с компактным носителем
надо лишь воспользоваться разбиением единицы.

\subsection{Преобразование Фурье обобщённых функций}

\subsubsection{Преобразование Фурье в $\Sc$}

Мы кое -- что уже знаем про оператор Фурье $\Fc\cln\Sc\ra\Sc$. Покажем, что он непрерывен.
Пусть $\ph\in\Sc$. Имеем
\mln{\maxl{\la}\hm{\la^k\wh\ph^{(m)}(\la)} = \maxl{\la}\hm{\Fc\hr{\hs{x^m\ph(x)}^{(k)}}(\la)}\le
\int \hm{\hs{x^m\ph(x)}^{(k)}}\dx \le \sum c_{\al\be} \int\hm{x^\al\ph^{(\be)}(x)}\dx =\\ =
\sum c_{\al\be}\int \frac{1+x^2}{1+x^2}\cdot \hm{x^\al\ph^{(\be)}(x)}\dx = \sum c_{\al\be} \Br{\int \frac{\hm{x^\al\ph^{(\be)}(x)}}{1+x^2}\dx
+ \int\frac{\hm{x^{\al+2}\ph^{(\be)}(x)}}{1+x^2}\dx}.}
А числители оцениваются соответствующими полунормами в $\Sc$, поэтому интегралы сходятся.
Значит,
\eqn{\hn{\wh\ph}_{k,m}\le \sum C_{\al\be} \hn{\ph}_{\al,\be}.}

\begin{theorem}
Оператор $G\cln\Sc\ra\Sc$, коммутирующий с $\Mc$ (умножением на $x$), есть оператор умножения на некоторую функцию $\al(x) \in \Cb^\bes$.
А если он еще и коммутирует с $\Dc$ (дифференцированием), то $\al(x)\equiv\const$.
\end{theorem}
\begin{proof}
Нам потребуется одна несложная лемма.
\begin{lemma}
Если $\ph(a)=0$, то $(G\ph(a))=0$.
\end{lemma}
\begin{proof}
Если у функции есть нуль в точке $a$, то она имеет вид $\ph(x)=(x-a)\psi(x)$, где $\psi\in\Cb^\bes$.
Заметим, что $\psi\in \Sc$, так как $\psi(x)= \frac{\ph(x)}{x-a}$, а функция $\ph$ убывает быстрее любой обратной степени.
Имеем $\ph = \Mc\psi-a\psi$ и $G\ph = \Mc G\psi - aG\psi$. Но тогда
$(G\ph)(a) = a(G\psi)(a)-a(G\psi)(a) = 0$.
\end{proof}
Докажем утверждение теоремы.
Рассмотрим <<горб>> -- гладкую функцию  $\Ga_a(x)$, которая равна~$1$ на отрезке $[a-1,a+1]$, и нулю вне интервала $(a-2,a+2)$.
Положим $\al(a) := (G\Ga_a)(a)$ -- это будет некоторая функция. Рассмотрим функцию
\eqn{\psi(x) := \ph(x)-\ph(a)\cdot \Ga_a(x).}
У неё будет нуль в точке $a$, значит, по лемме имеем
\eqn{(G\psi)(a)= G\br{\ph(x)-\ph(a)\cdot \Ga_a(x)}(a) = (G\ph)(a)-\ph(a)\cdot (G\Ga_a)(a)= 0 \quad \Ra \quad (G\ph)(a)=\ph(a)\al(a).}
Но это верно для любого $a$, поэтому $(G\ph)(x)=\ph(x)\al(x)$.
Покажем теперь, что $\al(x)\in \Cb^\bes$. При любом $a\in \R$ имеем
\eqn{(G\Ga_a)(x)=\al(x)\Ga_a(x).}
Левая часть лежит в $\Cb^\bes$ по определению оператора $G$, а правая часть совпадает с $\al(x)$ на интервале $(a-1,a+1)$.
В силу произвольности $a$ это верно на всей прямой, значит, $\al(x)\in\Cb^\bes$.

\medskip

Докажем второе утверждение. Имеем $\Ga_a'(x)=0$ на интервале $(a-1,a+1)$. Поэтому на этом интервале имеем
\eqn{0 = (G\Dc\Ga_a)(x) = \br{\al(x)\Ga_a(x)}' = \al'(x).}
Снова в силу того, что $a$ произвольно, имеем $\al'(x)\equiv0$.
\end{proof}

\begin{imp}
$\Fc^{-1}\Fc = \id$.
\end{imp}
\begin{proof}
Этот оператор коммутирует с $\Mc$ и с $\Dc$. Значит, это скалярный оператор $\la I$.
У него есть собственная функция $e^{-\frac{x^2}{2}}$ с собственным значением $1$. Значит, $\la =1$.
\end{proof}

\begin{petit}
Тут на самом деле мухлёж: функция при двукратном применении оператора Фурье переворачивается.
Просто функция $e^{-\frac{x^2}{2}}$ чётна, поэтому она переходит в себя. Впрочем, это не так важно.
\end{petit}

\begin{stm}
Пусть $h(x)$ -- такая функция, что $h(x)e^{a|x|}\in L_1$, и $\int x^n h(x)\dx=0$ для
всех $n \ge0$. Тогда $h\eqae 0$.
\end{stm}
\begin{proof}
Это очевидно: преобразование Фурье $\wh h$ от такой функции аналитически
продолжается в полосу $|\Im z| \bw< a$, а эти интегралы есть в точности производные
от преобразования Фурье в нуле. Значит, по теореме единственности $\wh h =0$, а тогда $h\eqae 0$.
\end{proof}

\begin{stm}
В пространстве $\Sc$ существует функция $\ph$, для которой $\int x^n\ph(x)\dx=0$ для всех $n$.
\end{stm}
\begin{proof}
Пользуясь тем, что $\Fc$ (а значит, и $\Fc^{-1}$) -- изоморфизм $\Sc$, такую функцию мы легко получим,
рассмотрев обратное преобразование Фурье от функции $\psi\in\Sc$, <<плоской>> в нуле,  --е $\ph := \Fc^{-1}\psi$,
а $\psi \equiv 1$ в $U(0)$.
\end{proof}
\begin{imp}
В пространстве $\Sc$ бывают такие функции $\ph$, для которых $\ph(x)e^{a|x|}\notin L_1$ ни при каком $a$.
\end{imp}

\subsubsection{Преобразование Фурье в $\Sc'$ и в $\Dc'$}

Преобразование Фурье обобщённых функций определяется по аналогии другим операциям с обобщёнными функциями.

\begin{df}
Пусть $f\in \Sc'$, а $\ph\in \Sc$. Положим $\ba{\wh f, \ph} := \ha{f, \wh \ph}$.
\end{df}

\begin{stm}
Оператор преобразования Фурье $*$-слабо непрерывен на $\Sc$.
\end{stm}
\begin{proof}
Пусть $f_n\ra f$ на каждой тестовой функции $\ph\in\Sc$. Тогда
\eqn{\ha{\Fc f_n,\ph} = \ha{f_n,\Fc \ph} \ra \ha{f,\Fc \ph} = \ha{\Fc f, \ph},}
что и означает $*$-слабую непрерывность.
\end{proof}

А вот в пространстве $\Dc'$ преобразование Фурье ведёт себя намного хуже. Прежде всего заметим такой прикольный факт:
$\Fc(D) \cap D = \es$. В самом деле, образ Фурье функции из $\Dc$ -- это голоморфная функция, поэтому
она не может быть нулём вне некоторого компакта, если она не нулевая.

\begin{stm}
Не существует $*$-слабо непрерывного продолжения оператора Фурье до отображения $\Fc\cln \Dc'\ra\Dc'$.
\end{stm}
\begin{proof}
Пусть существует оператор $\Ac\cln \Dc'\ra\Dc'$ с такими свойствами. Пусть $f\in \Dc'$, а $\psi_1\in \Dc$.
Тогда $(\Ac f)(\psi_1)$ как функция от $f$ есть слабо непрерывный функционал на $\Dc'$. Но любой слабо непрерывный
функционал на $\Dc'$ есть значение $f$ на некоторой другой основной функции, то есть
$(\Ac f)(\psi_1) = f(\psi_2)$. Теперь возьмём в качестве~$f$ функцию $\ph\in\Dc$. Тогда
\eqn{\int  \psi_2\ph\dx = \int \ol{\ol\psi}_1\Fc\ph\dx = \int \Fc\ol\psi_1\ph\dx.}
Значит, $\psi_2=\Fc\ol\psi_1$, чего не бывает, поскольку $\Dc\cap \Fc(\Dc)=\es$.
\end{proof}

\end{document}


%% Local Variables:
%% eval: (setq compile-command (concat "latex  -halt-on-error -file-line-error " (buffer-name)))
%% End:
