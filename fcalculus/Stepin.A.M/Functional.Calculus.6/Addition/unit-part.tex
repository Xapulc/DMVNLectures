\documentclass[12pt]{article}
\usepackage[utf8]{inputenc}
\usepackage[russian]{babel}
\usepackage{amssymb,amsfonts,amsmath}
\usepackage{graphicx}
\usepackage{enumerate}
\newcommand{\supp}{supp\,}

\begin{document}
{\bf Теорема.}
\textit{Пусть имеется открытое множество $\Omega\subset\mathbb{R}^n$ и его открытое покрытие
$\Omega\subset\bigcup\limits_{\alpha} U_{\alpha}$. Тогда найдутся функции $\psi_i$ ($i=1,2,\dots$), такие что
\begin{enumerate}
\item $\forall\,i\;\exists\,\alpha\; \supp\psi_i\subset U_{\alpha};$
\item $0\le\psi_i\le1\;\forall\,i;$
\item $\sum\limits_{i=1}^{\infty} \psi_i=1$ на $\Omega;$
\item для любого компакта $K\subset\Omega$ найдётся окрестность $U$ этого компакта ($K\subset U\subset\Omega$)
и число $m\in\mathbb{N}$, для которых $\psi_1+\dots+\psi_m\equiv1$ на множестве $U$.
\end{enumerate}
}

{\bf Доказательство.} Рассмотрим все замкнутые шары с рациональными радиусами и с центрами в рациональных точках,
содержащиеся в $\Omega\cap U_{\alpha}$ для некоторого $\alpha$. Таких шаров не более чем счётное множество. Кроме
того, их количество не может быть конечным. В самом деле, если $x\in\Omega$, то $x$ лежит в некотором
$U_{\alpha}$, и для некоторого $\varepsilon>0$ $\varepsilon$-окрестность точки $x$ целиком лежит в открытом
множестве $\Omega\cap U_{\alpha}$. Возьмём рациональную точку $y\in U_{\varepsilon/2}(x)$ (такая всегда есть).
Тогда $\forall\,r<\varepsilon/2\;\overline{U_r(y)}\subset U_{\varepsilon}(x)\subset\Omega\cap U_{\alpha}$, так как
$r+\rho(x,y)<\varepsilon/2+\varepsilon/2=\varepsilon$. Но рациональных чисел, меньших $\varepsilon/2$, бесконечно
много, и все замкнутые шары с такими радиусами с центрами в рациональной точке $y$ лежат в $\Omega\cap
U_{\alpha}$. Значит, таких шаров бесконечно много, то есть они образуют счётное множество, и мы можем их
обозначить $B_i$, $i=1,2,\dots$.

Теперь обозначим через $V_i$ открытый шар, концентрический $B_i$, но с вдвое меньшим радиусом. Утверждается, что
$\bigcup\limits_{i=1}^{\infty} V_i=\Omega$. Действительно, $V_i\subset B_i\subset\Omega$, поэтому включение
"$\subset$" очевидно. Докажем обратное включение. Возьмём точку $x\in\Omega$. Тогда $x$ лежит в некотором
$U_{\alpha}$. Так как $\Omega\cap U_{\alpha}$ открыто, то $\exists\,r>0\;U_r(x)\subset\Omega\cap U_{\alpha}$.
Рассмотрим такую рациональную точку $y$, что $\rho(x,y)<r/3$, и рациональное число $q\in(\rho(x,y);\,r/3)$. Тогда
шар $U_q(y)$ содержит точку $x$, поскольку $\rho(x,y)<q$. Кроме того, $\rho(x,y)+2q<r/3+2r/3=r$, откуда
$\overline{U_{2q}(y)}\subset U_r(x)\subset\Omega\cap U_{\alpha}$. Значит, $\exists\,i\;\overline{U_{2q}(y)}=B_i$,
так как $2q\in\mathbb{Q}$, а $y$ - рациональная точка. Поэтому $V_i=U_q(y)\ni x$, то есть
$x\in\bigcup\limits_{i=1}^{\infty} V_i$. Последнее верно для любой точки $x\in\Omega$. Мы доказали включение
"$\supset$". Итак, $\bigcup\limits_{i=1}^{\infty} V_i=\Omega$.

Теперь для каждого $i$ введём функцию $\varphi_i$ на множестве $\mathbb{R}^n$ такую, что $0\le\varphi_i\le1$, вне
$B_i\,$ $\,\varphi_i\equiv0$, а на $\overline{V_i}\,$ $\,\varphi_i\equiv1$. Положим по определению
$$\psi_n:=\bigl(\prod\limits_{i=1}^{n-1} (1-\varphi_i)\bigr)\varphi_n=$$
$$=\prod\limits_{i=1}^{n-1} (1-\varphi_i)-\prod\limits_{i=1}^n (1-\varphi_i).$$

Тогда
$$
  \sum\limits_{k=1}^n \psi_k=
  \sum\limits_{k=1}^n \bigl(\prod\limits_{i=1}^{k-1} (1-\varphi_i)-\prod\limits_{i=1}^k (1-\varphi_i)\bigr)=
$$
$$
  =\prod\limits_{i=1}^0 (1-\varphi_i)-\prod\limits_{i=1}^n (1-\varphi_i)=1-\prod\limits_{i=1}^n (1-\varphi_i).
$$

Поскольку все функции $\varphi_i$ заключены между $0$ и $1$, то там же заключены и $1-\varphi_i$, а значит, и все
функции $\psi_n=\bigl(\prod\limits_{i=1}^{n-1} (1-\varphi_i)\bigr)\varphi_n$. Тем самым 2-е условие для набора
функций $\psi_i$ выполнено.

Для всякого $n$ вне шара $B_n$ равна нулю функция $\varphi_n$ а значит, и функция
$\psi_n=\bigl(\prod\limits_{i=1}^{n-1} (1-\varphi_i)\bigr)\varphi_n$ ~--- тоже. Поскольку шар $B_n$ замкнут, то он
содержит носитель функции $\psi_n$, а так как этот шар содержится в некотором $U_{\alpha}$, то $\supp\psi_n\subset
B_n\subset U_{\alpha}$, что обеспечивает нам выполнение 1-го условия.

Если $x\in\Omega$, то $\exists\,k\;x\in V_k$, и при $n\ge k$
$$
  \sum\limits_{i=1}^n \psi_i(x)=1-\prod\limits_{i=1}^n (1-\varphi_i(x))=1,
$$
так как в произведение входит множитель, соответствующий $i=k$, а именно, $1-\varphi_k(x)$, который равен нулю,
ибо $x\in V_k$ и $\varphi_k(x)=1$. Следовательно, всё произведение равно нулю, а $\sum\limits_{i=1}^n
\psi_i(x)=1$. Если рассмотреть числовой ряд $\sum\limits_{i=1}^{\infty} \psi_i(x)$, то все его частичные суммы,
начиная с $k$-й, равны $1$. Таким образом, 3-е условие также имеет место ~--- функциональный ряд
$\sum\limits_{i=1}^{\infty} \psi_i$ поточечно сходится к тождественной единице на множестве $\Omega$.

Рассмотрим теперь компакт $K\subset\Omega=\bigcup\limits_{i=1}^{\infty} V_i$. Здесь построено открытое покрытие
этого компакта, значит, из него можно выбрать конечное подпокрытие, причём можно считать, что оно состоит из
нескольких первых множеств набора $V_i$: $K\subset U$, где $U:=\bigcup\limits_{i=1}^m V_i$. Тогда если $x\in U$,
то $x\in V_k$ для некоторого $k\in\{1,\dots,m\}$, $\varphi_k(x)=1$, $1-\varphi_k(x)=0$, $$\sum\limits_{i=1}^m
\psi_i(x)=1-\prod\limits_{i=1}^m (1-\varphi_i(x))=1.$$ То есть $\sum\limits_{i=1}^m \psi_i\equiv1$ на всём $U$,
поэтому функции $\psi_i$ удовлетворяют и 4-му условию. То, что $U\subset\Omega$, следует из того, что $\Omega$
есть объединение
\textit{всех} множеств $V_i$, $i\in\mathbb{N}$, а $U$ ~--- объединение конечного набора $V_i$ (не
по всем $i$, а только по $i\le m$).

Итак, построенные нами функции $\psi_i$ ~--- как раз искомые. $\square$

\end{document}
