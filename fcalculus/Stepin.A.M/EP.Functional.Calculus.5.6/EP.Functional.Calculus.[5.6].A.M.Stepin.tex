\documentclass[a4paper]{article}
\usepackage[utf,simple]{dmvn}

\title{Программа экзамена по функциональному анализу}
\author{Лектор А.\,М.\,Стёпин}
\date{V--VI семестр, 2004--2005 г.}

\begin{document}
\maketitle

\section*{V семестр}
\begin{nums}{-3}
\item Полные метрические пространства. Теорема о вложенных шарах и теорема Бэра. \cite[II, 3]{kf}
\item Принцип сжимающих отображений и его применения. \cite[II, 4]{kf}
\item Пополнение метрических пространств, связь полноты и замкнутости. \cite[II, 3]{kf}
\item Компактные метрические пространства, критерий компактности (Хаусдорфа). \cite[II, 7]{kf}
\item Критерий компактности подмножества в $\Cb[a,b]$. \cite[II, 7]{kf}
\item Нормированные пространства. Теорема Банаха Хана. \cite[IV, 1]{ls}
\item Отделимость выпуклых множеств. \cite[IV, \S~1, п.~3]{kf}
\item Общий вид линейного ограниченного функционала на $\Cb[a,b]$. \cite{ls}
\item Линейные непрерывные функционалы на гильбертовом пространстве.
\item Изоморфизм сепарабельных бесконечномерных гильбертовых пространств. \cite{kf}
\item Норма линейного оператора. Формулы: $\hn A = \hn{A^*}$, $\hn{A^*A} = \hn{A}^2$.
\item Резольвента, её аналитические свойства, спектральный радиус. \cite[IV, 6]{kf}
\item Спектр оператора. Непустота спектра ограниченных операторов. \cite[V, 1]{kg}
\item Теорема об отображении спектра для полиномов. \cite[V, 1]{kg}, \cite[I том]{rs}
\item Связь нормы и спектрального радиуса. \cite[V, 1]{kf}
\item Спектральная теорема для самосопряжённого ограниченного оператора c циклическим вектором. \cite{rs}
\item Диагонализуемость компактных самосопряжённых операторов (существование полной ОНС из собственных векторов).
\item Принцип равномерной ограниченности (теорема Банаха Штейнгауза). \cite[III, 4]{ls}
\item Теорема Банаха об обратном операторе. \cite[III, 5]{ls}
\item Слабая компактность шара в пространстве, сопряжённом к сепарабельному нормированному. \cite[IV, 3]{kf}
\item Компактные операторы, их основные свойства (компактность суммы двух компактных
операторов; компактность произведения компактного и ограниченного операторов
в любом порядке). \cite[IV, 6]{kf}
\item Теорема о сохранении непрерывного спектра при компактном возмущении. \cite{gl}
\item Компактность интегральных операторов с ядрами Гильберта Шмидта. \cite{sh}
\item Сохранение компактности при сопряжении. \cite[IV, 6]{kf}
\item Эквивалентность норм в конечномерных пространствах. \cite[II, 2]{ls}
\item Лемма Рисса о <<почти перпендикуляре>>.
\item Собственные значения компактных операторов. \cite[IV, 6]{kf}
\item 2 я теорема Фредгольма. \cite{ls}
\item 3 я теорема Фредгольма.
\item Теория Фредгольма для операторов в гильбертовом пространстве.
\end{nums}

\pagebreak

\section*{VI семестр}

\begin{nums}{-3}
\item Убывание к нулю коэффициентов Фурье интегрируемой на отрезке функции относительно
ограниченной ортонормированной системы.
\item Условие Дини сходимости ряда Фурье. \cite[8.1]{kf}
\item Применение принципа равномерной ограниченности к рядам Фурье. \cite[8.1]{kf}
\item Свойство единственности для рядов Фурье интегрируемых функций. \cite[7.1]{sh}
\item Условие равномерной сходимости рядов Фурье. \cite[8.1]{kf}
\item Ограниченная сходимость ряда $\sum\frac{\sin nx}{n}$. \cite[3.7]{hr}
\item Преобразование Фурье интегрируемых функций; основные свойства. \cite[8.4]{kf}
\item Формула обращения. \cite[8.4]{kf}
\item Свойство единственности для преобразования Фурье. \cite[8.4]{kf}
\item Связь гладкости и убывания на бесконечности $f$ и $\wh f$. \cite[8.4]{kf}.
\item Полнота системы функций Чебышёва Эрмита. \cite[8.4]{kf}
\item Спектр оператора преобразования Фурье в $L_2(\R)$.
\item Свёртка и преобразование Фурье, оператор свёртки в $L_2(\R)$.
\item Применение преобразования Фурье к решению уравнения теплопроводности. \cite[8.4]{kf}
\item Равенство Парсеваля и преобразование Фурье в $L_2(\R)$. Теорема Планшереля. \cite[8.5]{kf}
\item $^*$ Сходимость последовательностей и топология в $\Cb_0^\bes$.
\item Пространство $\Dc$, регулярные и сингулярные обобщённые функции. \cite{vl}, \cite{gs}, \cite{ru}
\item Достаточность запаса основных функций; сингулярность $\Pc\frac{1}{x}$. \cite{lec}, \cite{vl}
\item Действие над обобщёнными функциями, существование первообразной. \cite[1.2]{gs}
\item Формула суммирования Пуассона. \cite[1.2]{gs}
\item Локальное совпадение обобщённых функций; носитель обобщённой функции. \cite[6]{ru}
\item Пространство $\Ec$, инъекция $\Ec'\to \Dc'$ и плотность образа. \cite{lec}
\item Метризуемость сходимости в $\Ec$ и неметризуемость сходимости в $\Dc$. \cite{lec}
\item $\Ec'$ пространство обобщённых функций с компактным носителем. \cite{lec}
\item Общий вид линейного непрерывного функционала на $L_1$. \cite{lec}
\item Теорема о структуре обобщённых функций с компактным носителем. \cite[6]{ru}
\item Преобразование Фурье на классе $\Sc$. \cite[7.3]{h}, \cite[IV.2.3]{kg}
\item Обобщённые функции умеренного роста и их преобразование Фурье.
\item Разбиение единицы.
\end{nums}

\begin{note}
Звезданутый вопрос отличатеся тем, что он является необязательным, и на экзамене можно просить его заменить.
\end{note}

\begin{thebibliography}{ГШ}
\setlength\itemsep{-2pt}
\bibitem[КФ]{kf}
    А.\,Н.\,Колмогоров, С.\,В.\,Фомин. \emph{Элементы теории функций и функционального анализа.} М.: Наука, 1981.
\bibitem[ЛС]{ls}
    Л.\,А.\,Люстерник, В.\,И.\,Соболев. \emph{Элементы функционального анализа.} М.: Наука, 1965.
\bibitem[РС]{rs}
    М.\,Рид, Б.\,Саймон. \emph{Методы современной математической физики.} М.: Мир, 1977.
\bibitem[Г]{gl}
    Н.\,И.\,Глазман. \emph{Прямые методы качественного спектрального анализа сингулярных
    дифференциальных операторов.} М.: Физматгиз, 1963.
\bibitem[Ш]{sh}
    Г.\,Е.\,Шилов. \emph{Математический анализ. Второй специальный курс.} М.: Физматгиз, 1965.
\bibitem[КГ]{kg}
    А.\,А.\,Кириллов, А.\,Д.\,Гвишиани. \emph{Теоремы и задачи функционального анализа.} М.: Наука, 1988.
\bibitem[ХР]{hr}
    Г.\,Г.\,Харди, В.\,В.\,Рогозинский. \emph{Ряды Фурье.} М.: Физматгиз, 1959.
\bibitem[Вл]{vl}
    В.\,С.\,Владимиров. \emph{Обобщённые функции в математической физике.} М.: Наука, 1976.
\bibitem[ГШ]{gs}
    И.\,М.\,Гельфанд, Г.\,Е.\,Шилов. \emph{Обобщённые функции.} М.: Физматгиз, 1959.
\bibitem[Р]{ru}
    У.\,Рудин. \emph{Основы функционального анализа.} М.: Мир, 19??.
\bibitem[Х]{h}
    А.\,Я.\,Хелемский. \emph{Лекции по функциональному анализу.} М.: МЦНМО, 2004.
\bibitem[Л]{lec}
    А.\,М.\,Стёпин. \emph{Лекции.} \dmvnwebsite, 2004--2005.
\end{thebibliography}

\medskip\dmvntrail
\end{document}
